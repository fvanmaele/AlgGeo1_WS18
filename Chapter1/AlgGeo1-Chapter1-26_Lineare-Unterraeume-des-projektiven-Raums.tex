
\section{Lineare Unterräume von $\mathbb{P}^{n}$}
\label{sec:lineare-unterraeume-von-pn}

Sei $\varphi:k^{n+1}\rightarrow k^{n+1}$ ein \emph{injektiver} Homomorphismus
von $k$-Vektorräumen. $\varphi$ induziert eine injektive Abbildung:
\[
  \imath:\mathbb{P}^{n}(k)\rightarrow\mathbb{P}^{n}(k)
\]

der ein Morphismus von Prävarietäten ist nach Satz 56. Das Bild von
$\imath$ ist eine abgeschlossene Untervarietät. Ist $A=(a_{ij})\in M_{l\times(n+1)}$
mit $\text{im}(\varphi)=\ker(k^{n+1}\xrightarrow{A}k)$ und
\[
  f_{i}:=\sum_{j=0}^{n}a_{ij}X_{j}\in k[X_{0},\ldots,X_{n}],
\]

so identifiziert $\imath$ $\mathbb{P}^{n}(k)$ mit $V_{+}(f_{1},\ldots,f_{l})$.
(Die Abbildung $\imath:\mathbb{P}^{n}(k)\rightarrow V_{+}(f_{1},\ldots,f_{l})$
ist ein Isomorphismus von Prävarietäten, mit Umkehrabbildung $\varphi^{-1}:\varphi(k^{n+1})\rightarrow k^{n+1}$
induziert.)
\begin{example*}
  $\mathbb{P}^{m}=V_{+}(X_{m+1},\ldots,X_{n})\subset\mathbb{P}^{n}$.
  Solche Unterräume heißen \textbf{lineare Unterräume} (der Dimension
  $m$).

  $m=0$: Punkte

  $m=1$: Geraden

  $m=2$: Ebenen

  $m=n-1$: Hyperebenen in $\mathbb{P}^{n}(k)$.
  \begin{itemize}
  \item Zu zwei Punkten $p\neq q\in\mathbb{P}^{n}(k)$ existiert genau eine
    gerade $\overline{pq}$ in $\mathbb{P}^{n}(k)$, die $p$ und $q$
    enthält, da zu zwei verschiedenen Ursprungsgeraden im $k^{n+1}$ genau
    eine Ebene (in $k^{n+1})$ existiert, die beide Geraden enthält.
  \end{itemize}
\end{example*}
\begin{itemize}
\item Je zwei verschiedene Geraden in $\mathbb{P}^{2}(k)$ schneiden sich
  in genau einem Punkt, da Geraden in $\mathbb{P}^{2}$ Ebenen in $k^{3}$
  entsprechen, und zwei Ebenen sich dort genau in einer Geraden, d.h.
  einem Punkt des $\mathbb{P}^{2}$, schneiden. Dimensionsformel (lineare
  Algebra):
  \[
    \dim E_{1}\cap E_{2}=-\underbrace{\dim E_{1}+E_{2}}_{3}+\underbrace{\dim E_{1}}_{2}-\underbrace{\dim E_{2}}_{2}=1
  \]
  \emph{Später}: Verallgemeinerung: Satz von Bézout für allgemeine Unterprävarietäten
  $V_{+}(f)$.
\end{itemize}

