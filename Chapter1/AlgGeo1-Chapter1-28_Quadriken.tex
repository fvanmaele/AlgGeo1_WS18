\section{Quadriken}
\label{sec:quadriken}

Sei in diesem Abschnitt char$(k)\neq2$.
\begin{defn}[orig. 57]
  \label{def:quadrik}
  Eine abgeschlossene Unterprävarietät $Q\subseteq\mathbb{P}^{n}(k)$
  von der Form $V_{+}(q)$, $0 \neq q\in k[X_{0},\ldots,X_{n}]_{2}$
  heißt \textbf{Quadrik}.
  \[
    Q=V_{+}(q)
  \]

  Zur quadratischen Form $q$ gehört eine assoziierte Bilinearform $\beta$ auf
  $k^{n+1}$ (vgl. lineare Algebra), 
  \[
    \beta(v,w):=\frac{1}{2}(q(v+w)-q(v)-q(w)),\quad v,w\in k^{n+1}
  \]

Es gibt eine Basis von $k^{n+1}$, sodass die Strukturmatrix $B$
von $\beta$ die Gestalt
\[
  B=\begin{pmatrix}
    \begin{array}{ccc}
      1\\
      & \ddots\\
      &  & 1
    \end{array} & 0\\
    0 &
    \begin{array}{ccc}
      0\\
      & \ddots\\
      &  & 0
    \end{array}
  \end{pmatrix}
\]

hat, d.h. Koordinatenwechsel zur Basiswechselmatrix liefert einen
Isomorphismus
\[
  Q\xrightarrow{\sim}V_{+}(X_{0}^{2}+\cdots+X_{r-1}^{2}),\quad r=\text{rk }B
\]
\end{defn}
\begin{lem}[orig. 58]
\label{lem:irreduzibilitaet-quadriken}
	\begin{enumerate}
	
		\item $X_{0}^{2} + \ldots + X_{r-1}^{2}$ ist irreduzibel $\iff$ $r > 2$
		\item $V_{+}(X_{0}^{2} + \ldots + X_{r-1}^{2})$ ist irreduzibel $\iff$ $r \neq 2$
	\end{enumerate}
\end{lem}
\begin{proof}
	\begin{itemize}
		\item $r=0,1: X_{0}^2 = X_{0} \cdot X_{0} \Rightarrow V_{+}(X_{0}^2) = V_{+}(X_{0})$ irreduzibel
		\item $r=2: X_{0}^{2} + X_{1}^{2} = (X_{0} + i\cdot X_{1})\cdot(X_{0} - i \cdot X_{1})$ für $i = \sqrt{-1}$ 
		\item $r>2: $ Angenommen $\sum_{i}{a_{i} X_{i}} \cdot \sum_{j}{b_{j}X_{j}} = X_{0}^{2} + \ldots X_{r-1}^{2}$.\\
		Ausmultiplizieren $+$ Koeffizientenvergleich $\Rightarrow$ Widerspruch.
	\end{itemize}
\end{proof}

\begin{prop}[orig. 59]
  \label{prop:quadrik-in-normalform}
  Ist $r\neq s$, so sind $V_{+}(T_{0}^{2}+\cdots+T_{r-1}^{2})$ und
  $V_{+}(T_{0}^{2}+\cdots+T_{s-1}^{2})$ nicht isomorph.
\end{prop}
\begin{proof}
  Später: Es gibt keinen Koordinatenwechsel von $\mathbb{P}^{n}(k)$,
  der die beiden Mengen miteinander identifiziert, damit auch kein Automorphismus von
  $\mathbb{P}^{n}(k)$.
\end{proof}

\begin{defn}
  \label{def:dim-und-rang-einer-quadrik}
  Eine Quadrik $Q\subseteq\mathbb{P}^{n}(k)$ mit
  $Q\cong V_{+}(T_{0}^{2}+\cdots+T_{r-1}^{2})$, $r\geq1$, hat \textbf{Dimension $n-1$} und den \textbf{Rang $r$}. (nach Satz
  eindeutig!)
\end{defn}

\begin{cor}[orig. 61]
\label{cor:klassifikation-von-quadriken}
  Zwei Quadriken $Q_{1}$ und $Q_{2}$ sind genau dann isomorph als
  Prävarietäten, wenn sie dieselbe Dimension und denselben Rang haben.
\end{cor}
\begin{proof}
  \mbox{}
  \begin{itemize}
  \item[,,$\Leftarrow$``]
    $Q_{1}\cong V_{+}(T_{0}^{2}+\cdots+T_{n-1}^{2})\cong Q_{2}$ in
    dem selben $\mathbb{P}^{n}$.
  \item[,,$\Rightarrow$``] Für $Q\subseteq\mathbb{P}^{n}(k)$ berechne
    $K(Q)$. Ohne Einschränkung
    $Q=V_{+}(X_{0}^{2}+\cdots+X_{n-1}^{2})$.
    \begin{enumerate}
    \item $r=1$: $V_{+}(X_{0}^{2})=V_{+}(X_{0})=\mathbb{P}^{n-1}(k)$:
      $K(Q)=k(T_{1},\ldots,T_{n-1})$.
    \item $r=2$: reduzibel: Zerlegung in zwei Hyperebenen
      $Z\cong\mathbb{P}^{n-1}$

      $\Rightarrow K(Z)\cong k(T_{1},\ldots,T_{n-1})$.
    \item $r>2$:
      $U=V(1+T_{1}^{2}+\cdots+T_{n-1}^{2})\subseteq\mathbb{A}^{n}(k)$
      ist nichtleere offene affine Teilmenge von $Q$.

      $\Rightarrow K(Q)=K(U)=\text{Quot}(\Gamma(U))=\text{Quot}(k[T_{1},\ldots,T_{n}]/(1+T_{1}^{2}+\cdots+T_{n-1}^{2})$

      $\Rightarrow\text{trgrad}_{k}\ K(Q)=n-1$. 
    \end{enumerate}
  \end{itemize}
\end{proof}
\begin{example}
  \label{bsp:quadrik-joe-harris}
  $Q$ Quadrik in $\mathbb{P}^{n}$ (vgl. Joe Harris, Seite 34).
  \begin{enumerate}
  \item In $\mathbb{P}^{1}(k)$. 
    \begin{itemize}
    \item \emph{Rang 2:} 2 Punkte, reduzibel. 
    \item \emph{Rang 1:} 1 Punkt (Doppelpunkt). 
    \end{itemize}
  \item In $\mathbb{P}^{2}(k)$.
    \begin{itemize}
    \item \emph{Rang 3:} Glatter Kegel
      $\cong\mathbb{P}^{1}(k)$. $X_{0}^{2}+X_{1}^{2}-X_{2}^{2}=0$
    \item \emph{Rang 2:} 2 verschiedene Geraden, reduzibel. 
    \item \emph{Rang 1:} (Doppel)gerade.
    \end{itemize}
  \item In $\mathbb{P}^{3}(k)$.
    \begin{itemize}
    \item Rang 1: Doppelebene (2-dimensionaler linearer Unterraum)
    \item Rang 2: (insert image)
    \item Rang 3: (insert image)
    \item Rang 4: (insert image)
    \end{itemize}
  \end{enumerate}
\end{example}
Die Quadrik $Q\subseteq\mathbb{P}^{n}(k)$ heißt \textbf{glatt}, falls
$r=n+1$, d.h. falls die Matrix $B$ zu $q$ maximalen Rang hat. Für
$\text{rk}(Q)>3$, $\dim(Q)=d$, ist
$Q\cong\overline{\widetilde{Q},\Lambda}$ Kegel über einer
\textbf{glatten} Quadrik $\widetilde{Q}$, da Dimension $r-2$
bzgl. einer $(d-r+2)$-dimensionalen Unterraums 1.
\begin{itemize}
\item $r=1,2$ ausgeartet.
\item $r=1$. $Q=V_{+}(X_{0}^{2})=V_{+}(X_{0})$ Hyperebenen in
  $\mathbb{P}^{n}(k)$.  Der Unterschied zwischen $V_{+}(X_{0}^{2})$
  und $V_{+}(X_{0})$ ist für eine projektive Varietät $Q$ nicht
  sichtbar, jedoch in der Theorie der Schemata unterscheidbar!
\item $r=2$. $Q=V_{+}(X_{0}^{2}+X_{1}^{2})$ reduzibel, d.h. keine
  Prävarietät in unserem Sinne! Auch hier werden uns Schemata später helfen.
\end{itemize}
\medskip{}

$Q=V_{+}(X_{0}^{2}+X_{1}^{2}+\cdots+X_{n-1}^{2})\subseteq\mathbb{P}^{d+1}$,
$r\leq d+2$

$\tilde{Q}=V_{+}(X_{0}^{2}+\cdots+X_{n-1}^{2})\subseteq\mathbb{P}^{r-1}$
glatt.

$A=\mathbb{P}^{d+1-v}=V_{+}(X_{0},\ldots,X_{n-1})\subseteq\mathbb{P}^{d+1}$

$Q=\overline{\widetilde{Q},\Lambda}$
