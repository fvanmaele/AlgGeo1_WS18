
\section{Funktorielle Eigenschaften von $\Gamma(X)$}
\label{sec:koordinatenring-funktiorialitaet}
\begin{prop}
  \label{prop:koordinatenringfunktor}
  Für einen Morphismus $X\xrightarrow{f}Y$ affiner algebraischer Mengen
  definiert 
  \begin{align*}
    \Gamma(f):\quad\Gamma(Y) & \rightarrow\Gamma(X)\\
    g & \mapsto g\circ f
  \end{align*}
  ein Homomorphismus von $k$-Algebren. Der so definierte \emph{kontravariante}
  Funktor
  \[
    \Gamma:\{\text{affine algebraische Mengen}\}\rightarrow\{\text{reduzierte endl. erz. }k\text{-Algebren}\}
  \]
  liefert eine Kategorienäquivalenz, welche durch Einschränkung eine Äquivalenz
  \[
    \Gamma:\{\text{irred. aff. alg. Mengem\}}\rightarrow\{\text{integre endl. erz. }k\text{-Algebren\}}
  \]
  induziert.
\end{prop}
\begin{proof}
  Sei $Y\xrightarrow{g}\mathbb{A}^{1}(k)\in\Gamma(Y)$ ein Morphismus. Es
  folgt:
  \[
    g\circ f:X\xrightarrow{f}Y\xrightarrow{g}\mathbb{A}^{1}(k)
  \] 
  ist Morphismus,  d.h. $g \circ f\in\Gamma(X)$. $\Gamma(f):\Gamma(Y)\rightarrow\Gamma(X)$
  ist ein $k$-Algebren-Homomorphismus mit $\Gamma(\text{id}_{X})=\text{id}_{\Gamma(X)}$. Da ferner gilt, dass $\Gamma(f_{1}\circ f_{2})=\Gamma(f_{2})\circ\Gamma(f_{1})$ ist $\Gamma$ ein kontravarianter Funktor.
  \begin{claim*}
    $\Gamma$ ist volltreu, d.h.
    \begin{align*}
      \Gamma:\hom(X,Y) & \rightarrow\hom_{k\text{-Alg}}(\Gamma(Y),\Gamma(X))\\
      f & \mapsto\Gamma(f)
    \end{align*}
    ist \emph{bijektiv} für alle affinen algebraischen Mengen $X,Y$.
  \end{claim*}
  \begin{proof}
    Wir konstruieren eine Umkehrabbildung wie folgt: Zu $\varphi:\Gamma(Y)\rightarrow\Gamma(X)$
    für $X\subseteq\mathbb{A}^{m}(k)$, $Y\subseteq\mathbb{A}^{n}(k)$ existiert ein Lift $\tilde\varphi$, s.d.
    \[
      \xymatrix{k[T_{1}',\ldots,T_{n}']\ar[r]^{\tilde{\varphi}}\ar@{->>}[d] & k[T_{1},\ldots,T_{m}]\ar@{->>}[d]\\
        \Gamma(Y)\ar[r]^{\varphi} & \Gamma(X)
      }
    \]
    kommutiert; $\tilde{\varphi}(T_{i}'):= f_i$ mit $f_i \in \pi^{-1}(\varphi(T_{i}')) \subseteq k[T_1,...,T_n]$, wobei $\pi : k[\underline{T}] \to \Gamma(X)$ die kanonische Projektion bezeichne. 
    Definiere:
    \begin{align*}
      f:X & \rightarrow Y\\
      x=(x_{1},\ldots,x_{n}) & \mapsto(\tilde{\varphi}(T_{1}')(x_{1},\ldots,x_{n}),\ldots,\tilde{\varphi}(T_{n}')(x_{1},\ldots,x_{n}))
    \end{align*}
  \end{proof}
  \begin{claim*}
    $\Gamma$ ist essentiell surjektiv, d.h. zu jeder reduzierten endlich
    erzeugten $k$-Algebra $A$ existiert eine affine algebraische Menge
    $X$ mit $A\cong\Gamma(X)$.
  \end{claim*}
  \begin{proof}
    Da nach Voraussetzung $A\cong k[T]/\mathfrak{a}$ für ein Radikalideal
    $\mathfrak{a}$, können wir etwa $X:=V(\mathfrak{a})\subseteq\mathbb{A}^{n}(k)$
    setzen. Der Rest folgt aus Satz \ref{prop:eigenschaften-koordinatenring}.
  \end{proof}
\end{proof}
\begin{prop}
  \label{prop:funktiorialitaet-specm}
  Sei $f:X\rightarrow Y$ ein Morphismus affiner algebraischer Mengen und $\Gamma(f):\Gamma(Y)\rightarrow\Gamma(X)$
  der zugehörige Homomorphismus der Koordinatenringe. Dann gilt $\forall x\in X$:
  $\Gamma(f)^{-1}(\mathfrak{m}_{x})=\mathfrak{m}_{f(x)}$.
\end{prop}
\begin{proof}
  \[
    \Gamma(f)^{-1}(\mathfrak{m}_{x})=\{g\in\Gamma(Y)\mid g\circ f \in \mathfrak{m}_{x}\}=\{g\in\Gamma(Y)\mid g(f(x)) = 0 \} = \mathfrak{m}_{f(x)},
  \]
  da $\Gamma(f)(g) =g \circ f$.
\end{proof}
