Später sehen wir: Varietät = ,,separierte Prävarietät``. Affine
Varietäten sind stets ,,separiert``, daher braucht man nicht von
,,affinen Prävarietäten`` zu reden. Ist $X$ eine affine Varietät,
so schreiben wir oft $\Gamma(X)$ für $\mathcal{O}_{X}(X)$ (vgl. Satz
\ref{prop:fkt-auf-basis}).

Unter einer \textbf{offenen affinen Überdeckung} einer Prävarietät
$X$ verstehen wir eine Famile von offenen affinen Unterräumen mit Funktionen
$U_{i}\subseteq X$, $i\in I$ die affine Varietäten sind, d.d. $X=\bigcup_i U_{i}$.

\section{Vergleich mit differenzierbaren/komplexen Mannigfaltigkeiten}
\label{sec:vergleich-mit-mannigfaltigkeiten}

\paragraph{Differential/Komplexe Geometrie}

Mannigfaltigkeiten werden via Kartenabbildungen mit differenzierbaren/holomorphen
Übergangsabbildungen definiert (hier problematisch, da offene Teile
affiner algebraischer Mengen i.A. keine solche Struktur besitzen.)
Jedoch:
\begin{align*}
  \text{\{differenzierbare Mfgkt.\}} & \longrightarrow\text{\{Räume mit Fkt.}/\mathbb{R}\}\\
  X & \longmapsto(X,\mathcal{O}_{X})\\
                                     & \phantom{\longmapsto}\mathcal{O}_{X}(U):=C^{\infty}(U,\mathbb{R}),\ U\subseteq X\text{ offen}
\end{align*}

ist ein volltreuer Funktor. Daher kann man differenzierbare Mannigfaltigkeiten
auch als diejenigen Räume mit Funktionen über $\mathbb{R}$ definieren,
für die $X$ Hausdorff ist, und so dass eine offene Überdeckung durch
solche Räume mit Funktionen über $\mathbb{R}$ existiert, die in obiger
Weise offene Teilmengen von $\mathbb{R}^{n}$ zugeordnet sind. (Analog
bei komplexen Mannigfaltigkeiten.)
