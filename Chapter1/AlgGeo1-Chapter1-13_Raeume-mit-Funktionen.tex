
\section{Räume mit Funktionen}
\label{sec:raeume-mit-funktionen}

(Prototyp eines geometrischen Objektes, Spezialfall eines ``geringten
Raumes'' vgl. später.) Sei $K$ ein nicht notwendigerweise algebraisch abgeschlossener
Körper.
\begin{defn}
  \label{def:raum-mit-funktionen}
  \mbox{}
  \begin{enumerate}
  \item Ein \textbf{Raum mit Funktionen}\index{Raum mit Funktionen} besteht
    aus den folgenden Daten:
    \begin{itemize}
    \item ein topologischer Raum $X$;
    \item eine Familie von Unter-$K$-Algebren
      \[
        \mathcal{O}_X(U)\leq\text{Abb}(U,K),\quad\forall U\subseteq X\text{ offen }d.d
      \]

      \begin{enumerate}
      \item Sind $U'\subseteq U\subseteq X$ offen und $f\in\mathcal{O}_X(U)$ so ist
        $f|_{U'}\in \mathcal{O}_X(U')$.
      \item (\textbf{Verklebungsaxiom}\index{Verklebungsaxiom}) Sind $U_{i}\subseteq X$
        offen, $i\in I$, und $U=\bigcup_{i}U_{i}$, $f_{i}\in\mathcal{O}_X(U_{i})$,
        $i\in I$ gegeben mit
        \[
          f_{i}|_{U_{i}\cap U_{j}}=f_{j}|_{U_{i}\cap U_{j}}\quad\forall i,j\in I
        \]
        dann ist die eindeutige Abbildung
        \[
          f:U\rightarrow K\text{ mit }f|_{U_{i}}=f_{i}
        \]
        in $\mathcal{O}_X(U)$, bzw. $\exists!f\in\mathcal{O}(U)$ mit $f|_{U_{i}}=f_{i}$ für alle $i \in I$.
      \end{enumerate}
    \end{itemize}
    Bezeichne $\mathcal{O}_X$ oder auch $\mathcal{O}$ die oben genannte
    Familie $\{\mathcal{O}_X(U) \mid U \subseteq X \text{offen}\}$. Das Tupel $(X,\mathcal{O}_{X})$ heißt $\textbf{Raum mit Funktionen}$.
  \item Ein \textbf{Morphismus}\index{Raum mit Funktionen!Morphismus} $(X,\mathcal{O}_{X})\rightarrow(Y,\mathcal{O}_{Y})$
    von Räumen von Funktionen ist eine stetige Abbildung $\varphi:X\rightarrow Y$,
    so dass für alle $V\subseteq Y$ offen und $f\in\mathcal{O}_{Y}$
    gilt:
    \[
      f\circ \varphi|_{\varphi^{-1}(V)}:\varphi^{-1}(V)\rightarrow K
    \]
    liegt in $\mathcal{O}_{X}(\varphi^{-1}(V))$.
    \[
      \xymatrix{X\ar[r]^{\varphi} & Y\\
        \varphi^{-1}(V)\ar[r]^{\varphi|}\ar[d]_{f\circ \varphi|_{\varphi^{-1}(V)}}\ar@{^{(}->}[u] & V\ar[d]^{f}\ar@{^{(}->}[u]_{\text{offen}}\\
        K\ar@{=}[r] & K
      }
    \]
  \end{enumerate}
\end{defn}
Wir erhalten die Kategorie der $\emph{Räume mit Funktionen über K}$.
\begin{defn}[offene Unterräume von Räumen mit Funktionen]
  \label{def:raeume-mit-fkt-offener-unterraum}
  Für $(X,\mathcal{O}_{X})$ einen Raum mit Funktionen und $U\subseteq X$ offen bezeichne $(U,\mathcal{O}_{X}|_{U})$ den
  Raum mit Funktionen gegeben durch den topologischen Raum $U$ mit
  Funktionen $\mathcal{O}_{X}|_{U}(V):=\mathcal{O}_{X}(V)$ für $V\underset{\text{offen}}{\subseteq}U\subseteq X$.
\end{defn}
\textbf{Ab jetzt} betrachten wir Räume von Funktionen über einem festen, algebraisch abgeschlossenen Grundkörper $k$.

