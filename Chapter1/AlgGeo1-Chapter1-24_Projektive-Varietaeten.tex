
\section{Projektive Varietäten}
\label{sec:projektive-varietaeten}
\begin{defn}[orig. 54]
  \label{def:projektive-varietaeten}
  Abgeschlossene Unterprävarietäten eines projektiven Raumes $\mathbb{P}^{n}(k)$
  heißen \textbf{projektive Varietäten}.
\end{defn}
Vorsicht: für $x=(x_{0}:\ldots:x_{n})\in\mathbb{P}^{n}$, $f\in k[X_{0},\ldots,X_{n}]$
ist $f(x_{1},\ldots,x_{n})$ \emph{nicht} wohldefiniert, da von Repräsentaten
abhängig, d.h. $f$ kann \emph{nicht}\textbf{ }als Funktion auf $\mathbb{P}^{n}$
aufgefasst werden. Für \emph{homogene}\textbf{ }Polynome $f_{1},\ldots,f_{n}\in k[X_{0},\ldots X_{n}]$
(nicht notwendig vom selben Grad) können wir dennoch Verschwindungsmengen
definieren:
\[
  V_{+}(f_{1},\ldots,f_{n})=\{(x_{0}:\ldots:x_{n})\in\mathbb{P}^{n}\mid f_{j}(x_{0},\ldots,x_{n})=0\ \forall j\}
\]

Da $V_{+}(f_{1},\ldots,f_{n})\cap U_{i}=V(\Phi_{i}(f_{1}),\ldots,\Phi_{i}(f_{m}))$
ist $V_{+}(f_{1},\ldots,f_{m})$ abgeschlossen in $\mathbb{P}^{n}$.
Ist $V_{+}(f_{1},\ldots,f_{n})$ irreduzibel, so erhalten wir eine
projektive Varietät. In der Tat entstehen alle projektiven Varietäten
auf diese Weise, wie der folgende Satz zeigt:
\begin{prop}[orig. 55]
  \label{prop:charakterisierung-projektive-varietaeten}
  Sei $Z\subseteq\mathbb{P}^{n}(k)$ eine projektive Varietät. Dann
  existieren homogene Polynome $f_{1},\ldots,f_{n}\in k[X_{0},\ldots,X_{n}]$,
  so dass
  \[
    Z=V_{+}(f_{1},\ldots,f_{n})
  \]

  gilt.
\end{prop}
\begin{proof}
  Betrachte: 
  \[
    \begin{array}{cc}
      \\
      \\
    \end{array}
  \]

  $f| :f^{-1}(U_{i})\longrightarrow U_{i}$ ist Morphismus
  von Prävarietäten. Dann ist $f$ selbst ein Morphismus von Prävarietäten: $\emph{lokal}$ ist die Aussage klar, $\emph{global}$ verklebt man.
  \begin{align*}
    \overline{Y}:= & Y\cup\{0\}\text{, der Abschluss von }Y\text{ in }\mathbb{A}^{n+1}(k)\\
    \mathfrak{a}:= & I(\overline{Y})\subseteq k[X_{0},\ldots,X_{n}]
  \end{align*}

  Behauptung: $\mathfrak{a}$ wird von homogenen Polynomen erzeugt\emph{.
    Denn:} Sei für $g\in\mathfrak{a}$, $g=\sum_{d}g_{d}$ die Zerlegung in homogene
  Bestandteile vom Grad $d$. $\overline{Y}$ ist Vereinigung von Ursprungsgeraden
  im $k^{n+1}$, d.h. $\forall\lambda\in k^{\times}$ gilt:
  \[
    g(x_{0},\ldots,x_{n})=0\ \Leftrightarrow\ g(\lambda x_{0},\ldots,\lambda x_{n})=0
  \]

  Beweis durch Widerspruch: \emph{Angenommen} nicht alle $g_{d}$ liegen in $\mathfrak{a}$.

  $\Rightarrow\exists(x_{0},\ldots,x_{n})\in\mathbb{A}^{n+1}(k)$, so
  dass $g(x_{0},\ldots,x_{n})=0$, aber $g_{d_{0}}(x_{0},\ldots,x_{n})\neq0$.

  $\Rightarrow0\,\neq\sum_{d}g_{d}(x_{0},\ldots,x_{n})T^{d}\in k[T]$

  $\Rightarrow\exists\lambda\in k^{\times}$: $0\neq\sum_{d}g_{d}(x_{0},\ldots,x_{n})\lambda^{d}=\sum_{d}g_{d}(\lambda x_{0},\ldots,\lambda x_{n})=g(\lambda x_{0},\ldots,\lambda x_{n})=0$.
  Widerspruch.

  $\Rightarrow\mathfrak{a}=(f_{1},\ldots,f_{m})$, mit $f_{j}$ homogen.

  $\Rightarrow Z=V_{+}(f_{1},\ldots,f_{m})$. 

  \begin{align*}
    Z\ni(x_{0}:\ldots:x_{n}) & \Leftrightarrow(\lambda x_{0},\ldots,\lambda x_{n})\in\overline{Y}\ \forall\lambda\in k^{\times}\text{ und }\neq0\\
                             & \Leftrightarrow f_{i}(x_{0},\ldots,x_{n})=0\ \forall1\leq i\leq n,\ (x_{0},\ldots,x_{n})\in\mathbb{P}^{n}
  \end{align*}

  \rule[0.5ex]{1\columnwidth}{1pt}
\end{proof}
Zu Bemerkung \ref{rem:charakterisierung-homogen}:

Nach Satz \ref{prop:charakterisierung-reg-fkt-projektiver-raum} und Definition von $\mathcal{O}_{Z}'$ folgt: Ist $X$
eine projektive Varietät und $U\subseteq X$ offen, so erhalten wir 


($\dagger$) \ $\mathcal{O}_{X}(U)=\{f:U\rightarrow k\mid\forall x\in U\ \exists x\in V\underset{\text{offen}}{\subseteq}U,\ g,h\in k[X_{0},\ldots,X_{n}]$
homogen vom gleichen Grad mit $h(v)\neq0, \ f(v)=\frac{g(v)}{h(v)},\ \forall v\in V\}$.


Insbesondere gilt:
\begin{prop}[orig. 56]
  \label{prop:charakterisierung-morphismen-proj-varietaeten}
  Seien $V\subseteq\mathbb{P}^{m}(k)$, $W\subseteq\mathbb{P}^{n}(k)$
  projektive Varietäten und
  \[
    \phi: V \longrightarrow W
  \]

  eine Abbildung. Dann ist $\phi$ eine Morphismus genau dann, wenn
  zu jedem $x\in V$ eine offene Umgebung $x\in U_{x}\subseteq V$ und
  homogene Polynome $f_{0},\ldots,f_{n}\subseteq k[X_{0},\ldots,X_{m}]$
  vom selben Grad existieren mit
  \[
    \phi(y)=(f_{0}(y),\ldots,f_{n}(y))\quad\forall y\in U_{x}
  \]
\end{prop}
\begin{proof}
  \mbox{}
  \begin{itemize}
  \item ``$\Rightarrow$'': Übung.
  \item ``$\Leftarrow$'':
    \begin{enumerate}
    \item $\phi$ stetig: Sei $Z\subseteq W$ abgeschlossen. Ohne Einschränkung
      $Z=V_{+}(g)\cap W$ für ein homogenes Polynom $g$. Dann berechnet
      sich das Urbild
      \[
        \phi^{-1}(Z)=V_{+}(g\circ\phi)\cap V.
      \]
      Auf $U_{x}$, $x\in V$, ist $g\circ\phi$ als homogenes Polynom in
      $X_{0},\ldots,X_{n}$ gegeben. 

      $\Rightarrow V(g\circ\phi)\cap U_{x}=\phi^{-1}(Z)\cap U_{x}$ abgeschlossen
      in $U_{x}$ für alle $x$.

      $\Rightarrow\phi^{-1}(Z)\subseteq V$ abgeschlossen.
    \item Zu zeigen: $\forall W'\subseteq W$ offen, $g\in\mathcal{O}_{W}(W')$
      ist $g\circ\phi\in\mathcal{O}_{V}(\phi^{-1}(W'))$.

      ($\dagger$) $\Rightarrow$ Es ex. eine offene Umgebung $W_{y}$ in $W'$
      mit $g=\frac{h}{q}$ auf $W_{y}$, $h,q$ homogen vom Grad $d$.

      $\Rightarrow\phi_{|U_{x}\cap\phi^{-1}(W_{y}):=\tilde{U}_{x}}$ ist
      auch von dieser Gestalt.

      $\Rightarrow$ $\frac{h(f_{0},\ldots,f_{n})}{q(f_{0},\ldots,f_{n})}=g\circ\phi_{|\tilde{U}_{x}}\in\mathcal{O}_{V}(\tilde{U}_{x})$.
    \end{enumerate}
    Verklebungsaxiom $\Rightarrow$ $g\circ\phi\in\mathcal{O}_{V}(\phi^{-1}(V))$.
  \end{itemize}
\end{proof}

