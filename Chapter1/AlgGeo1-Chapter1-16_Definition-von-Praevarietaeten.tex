
\chapter*{Prävarietäten}

\textbf{Ziel.} Klasse der affin-algebraischen Mengen, aufgefasst
als Räume mit Funktionen durch Verkleben vergrößern.

$(X,\mathcal{O}_{X})$ heißt \textbf{zusammenhängend}, falls $X$
als topologischer Raum zusammenhängend ist.

\section{Definition von Prävarietäten}
\label{sec:def-praevarietaet}
\begin{defn}[orig. 37]
  \label{def:affine-varietaet}
  Eine \textbf{affine Varietät}\index{affine Varietät} ist ein Raum
  mit Funktionen, der isomorph zu dem Raum mit Funktionen assoziiert zu einer irreduziblen affin-algebraischen Menge ist.
\end{defn}

\begin{defn}[orig. 38]
  \label{def:praevarietaet}
  Eine \textbf{Prävarietät} ist ein zusammenhängender Raum mit Funktionen
  $(X,\mathcal{O}_{X})$, für den eine \emph{endliche }Überdeckung $X=\bigcup_{i=1}^{n}U_{i}$ durch offene Teilmengen $U_i \subseteq X$
  existiert, d.d. $\forall i=1,\ldots,n$ $(U_{i},\mathcal{O}_{X|_{U_{i}}})$
  eine affine Varietät ist.  Insbesondere sind affine Varietäten Prävarietäten!

  Ein \textbf{Morphismus von Prävarietäten} ist ein Morphismus der entsprechenden Räume mit Funktionen.
\end{defn}

