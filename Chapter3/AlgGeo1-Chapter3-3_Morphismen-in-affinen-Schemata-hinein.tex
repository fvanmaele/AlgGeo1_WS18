
\section{Morphismen in affinen Schemata hinein}
\begin{prop}[4]
  Sei $X$ ein Schemata, $Y=\Spec B$ ein affines Schemata. Dann ist
  die Abbildung
  \begin{align*}
    \Hom(X,Y) & \overset{\cong}{\longrightarrow}\Hom_{\ring}(B,\Gamma(X,\mathcal{O}_{X})),\\
    (f,f^{\flat}) & \longmapsto f_{Y}^{\flat}
  \end{align*}

  eine Bijektion.
\end{prop}

\begin{prop}[5, Verkleben von Morphismen]
  Seien $X,Y$ lokal geringte Räume. Für $U\subset X$ offen definiert
  \[
    \mathcal{F}:U\mapsto\Hom(U,Y)=\{(U,\mathcal{O}_{X|U})\rightarrow(Y,\mathcal{O}_{Y})\ \text{Morph. lokal ger. Räume\}}
  \]

  eine Garbe von Mengen auf $X$, d.h.
  \begin{enumerate}
  \item für eine offene Überdeckung $X=\bigcup_{i}U$, eine Familie $f_{i}:U_{i}\rightarrow Y_{i}$
    verkleben zu Morphismen
    \[
      f:X\rightarrow Y\Longleftrightarrow f_{i}|_{U_{i}\cap U_{j}}=f_{j}|_{U_{i}\cap U_{j}}
    \]
  \item $f$ ist eindeutig bestimmt.
  \end{enumerate}
\end{prop}

\begin{rem*}
  $\mathcal{G}:U\mapsto\Hom_{\ring}(B,\Gamma(U,\mathcal{O}_{X}))$ ist
  Garbe von Mengen.
\end{rem*}
\begin{proof}[Beweis von Proposition 5]
  Verkleben topologischer Räume + stetige Abbildung klar. $\checkmark$

  $\mathcal{O}_{Y}\rightarrow f_{\ast}\mathcal{O}_{X}$ lässt sich ebenfalls
  verkleben.
\end{proof}
% 
\begin{proof}[Beweis von Proposition 4]
  $X=\bigcup_{i}U_{i}$ sei eine affine offene Überdeckung. Nach Proposition
  2.35 ist $\Hom(U,Y)\rightarrow\Hom(B,\Gamma(U,\mathcal{O}_{X}))$
  eine Bijektion. Für $V\subset U_{i}\cap U_{j}$ kommutiert das Diagramm
  \[
    \xymatrix{\Hom(U,Y)\ar[r]^{\cong}\ar[d] & \Hom(B,\Gamma(U,\mathcal{O}_{X})\ar[d]\\
      \Hom(V,Y)\ar[r]^{\cong} & \Hom(B,\Gamma(V,\mathcal{O}_{X}))
    }
  \]

  da $\Gamma(-)$ funktoriell ist. Es folgt, dass $\mathcal{F}\rightarrow\mathcal{G}$
  ein Morphismus von Garben ist mit $\varphi_{U}:\mathcal{F}(U)\overset{\cong}{\rightarrow}\mathcal{G}(U)$
  für alle $U\in\mathcal{B}$, und $F\overset{\cong}{\rightarrow}\mathcal{G}$
  als Garbe. Somit $\mathcal{F}(X)\cong\mathcal{G}(X)$.
\end{proof}
$V\subset X$ offen beliebig, $\varphi_{V}=\underset{\underset{U\in B_{V}}{\longleftarrow}}{\lim}\varphi_{U}$.

Da $\mathbb{Z}$ kofinales Objekt in der Kategorie der Ringe ist ($\mathbb{Z}\overset{\exists_{1}}{\rightarrow}R$
für beliebige Ringe $R$), gilt:
\begin{cor}[6]
  Sei $X$ ein Schemata. $X$ besitzt einen eindeutig bestimmten Morphismus
  $X\rightarrow\Spec(\mathbb{Z})$, d.h. $\Spec(\mathbb{Z})$ ist ein
  terminales Objekt in der Kategorie der Schemata: Jedes Schemata ist
  ein $\mathbb{Z}$-Schemata.
\end{cor}

Weiterhin:
\begin{align*}
  \Hom(X,\underbrace{\Spec\mathbb{Z}[T]}_{\mathbb{A}_{\mathbb{Z}}^{1}}) & =\Hom_{\ring}(\mathbb{Z}[T],\mathcal{O}_{X}(X))=\Gamma(X,\mathcal{O}_{X})\\
  \Hom_{R}(X,\underbrace{\Spec R[T]}_{\mathbb{A}_{R}^{1}}) & =\Gamma(X,\mathcal{O}_{X})\text{ als }R\text{-Algebra für }R\text{-Schemata }X
\end{align*}
