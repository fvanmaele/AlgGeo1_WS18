\section*{Prävarietäten als Schema}

\textbf{Ziel: }``$X$ affine Varietät $\mapsto\Spec\Gamma(X,\mathcal{O}_{X})$``
(kein generischer Punkt, $\neq$ Schema). Verklebe zu: ``$X$ Prävarietät
über $k$ $\mapsto k$-Schema``. Welches Bild hat dieser Funktor?

\section{Schemata von endlichem Typ über $k$}

Sei $X$ affine Varietät über $k$, $k$ algebraisch Abgeschlossen.
Dann ist $A=\Gamma(X,\mathcal{O}_{X})$ eine endlich erzeugte $k$-Algebra.
\begin{defn}[25]
Sei $k$ Körper, $X\rightarrow\Spec(k)$ $k$-Schema. $X$ heißt:
\begin{itemize}
\item \textbf{lokal von endlichem Typ}, (l.v.e.T./$k$), falls eine affine
offene Überdeckung $X=\bigcup_{i\in I}U_{i}$ existiert, $U_{i}=\Spec A_{i}$,
mit $A_{i}$ endlich erzeugte $k$-Algebra für alle $i$.
\item \textbf{von endlichem Typ} (v.e.T$/k$), falls $X$ lokal von endlichem
Typ und quasi-kompakt ist.
\end{itemize}
\end{defn}

\begin{rem*}
Jedes $k$-Schema welches (lokal) von endlichem Typ ist, ist (lokal)
noethersch. (Da jede endlich erzeugte $k$-Algebra noethersch ist.)
\end{rem*}
\begin{prop}[26]
Sei $X$ $l.v.e.T/k$, $U\subset X$ offen affin. Dann ist $B:=\Gamma(U,\mathcal{O}_{X})$
eine endlich erzeugte $k$-Algebra.
\end{prop}

\begin{proof}
$U=\bigcup_{i=1}^{n}D(f_{i})$, $f_{i}\in B$ geeignet nach Lemma
3.3. $B$ ist endlich erzeugte $k$-Algebra $\Longrightarrow B_{f_{i}}=B\left[\frac{1}{f_{i}}\right]$
ist endlich erzeugte $k$-Algebra. Mit dem folgenden Lemma für $A=k$
folgt die Behauptung.
\end{proof}
\begin{lem}[28]
Sei $A$ ein Ring und $B$ eine $A$-Algebra, $\mathcal{L}:A\rightarrow B$
Ringhomomrphismus, $f_{1},\ldots,f_{n}\in B$ mit $(f_{1},\ldots,f_{n})=(1)$,
und so dass $B_{f_{i}}$ eine endlich erzeugte $A$-Algebra ist $\forall i$.
Dann ist $B$ eine endlich erzeugte $A$-Algebra.
\end{lem}

\begin{proof}
(Vergleiche Lemma 16.) Nach Voraussetzung gibt es ein $g_{i}\in B$
mit $\sum_{i}g_{i}f_{i}=1$. Da $B_{f_{i}}$ endlich erzeugt $\forall i$,
gibt es $b_{ij}$, $j\in J$ endlich, welche $B_{f_{i}}$ als $A$-Algebra
erzeugen. Setze $b_{ij}=c_{ij}/f_{i}^{m}$ mit $c_{ij}\in B$ für
$m\geq0$ geeignet (unabhäng von $i,j$).

Sei $C:=A$-Unteralgebra von $B$, erzeugt von $g_{i},f_{i},c_{ij}$,
d.h. endlich erzeugt über $A$. \emph{Behauptung}: $C=B$.

Sei $b\in B$. $\Longrightarrow$ $\exists N\gg0$ mit $f_{i}^{N}b\in C$
für alle $i$. Da $\sum_{i}g_{i}f_{i}=1$, ist $(f_{1},\ldots,f_{n})_{C}=(1)$.
Lemma 2.4 $\Longrightarrow$ $(f_{i}^{N},\ldots,f_{n}^{N})_{C}=(1)$.
$\Longrightarrow$ $\exists u_{1},\ldots,u_{n}\in C$ sodass $\sum_{i}u_{i}f_{i}^{N}=1$.
$\Longrightarrow b=\sum_{i}u_{i}\underbrace{f_{i}^{N}b}_{\in C}\in C$.
\end{proof}
\begin{prop}[28]
Sei $k$ algebraisch abgeschlossen, $X$ $k$-Schema l.v.e.T.$/k$.
Dann besteht die Menge der abgeschlossenen Punkte $|X|$ genau aus
den Punkten mit $\kappa(x)=k$, d.h. nach Proposition 7 gilt
\[
|X|=X(k)=\Hom_{k}(\Spec k,X).
\]
\end{prop}

\begin{proof}
Hilbert'scher Nullstellensatz $\Longrightarrow(x\in X$ abgeschlossen
$\Rightarrow\kappa(x)=k$). Daher reicht es zu zeigen: ($x\in X$
\emph{nicht} abgeschlossen, d.h. $\mathfrak{p}_{x}$ maximal $\Rightarrow\kappa(x)\neq k$).
Dazu: $\exists x\in U=\Spec(A)\subset X$ offen, mit $x$ \emph{nicht
abgeschlossen} in $U$. $\Longleftrightarrow\mathfrak{p}=\mathfrak{p}_{x}\in\Spec(A)$
\emph{nicht} maximal, d.h. $A/\mathfrak{p}_{x}$ ist \emph{kein} Körper.
$\Longrightarrow k\rightarrow(A/\mathfrak{p}_{x})\hookrightarrow\Quot(A/\mathfrak{p}_{x})=\kappa(x)$
ist \emph{echte} Inklusion.

Behauptung: $\kappa(x)$ ist nicht algebraisch abgeschlossen, d.h.
nicht abstrakt isomorph zu $k$. Denn: \emph{Noether-Normalisierung}:

$A/\mathfrak{p}$ ist endlich über $k[X_{1},\ldots,X_{n}]$. Nach
Lemma I.9 folgt $n>0$ (mit $A/\mathfrak{p}$ über $k$ ganz $\Longrightarrow A/\mathfrak{p}$
Körper). $\Longrightarrow\kappa(x)$ ist endliche Erweiterung von
$k(X_{1},\ldots,X_{n})$, $n>0$. $\Longrightarrow k$ nicht algebraisch
abgeschlossen ($[k(X_{1},\ldots,X_{n})(\sqrt[n]{X_{1}}):k(X)]\rightarrow\infty$,
$n\rightarrow\infty$)
\end{proof}
\begin{rem*}
Im Allgemeinen $\exists x\in U\subset X$ offen mit $\{x\}\subset U$
abgeschlossen, aber $\{x\}\subset X$ \emph{nicht} abgeschlossen ($X=\Spec\mathcal{O}$,
$\mathcal{O}$ DVR, $U=\{x\}$, $x=\eta$ generischer Punkt.) Für
$X$ lokal von endlichem Typ über $k$ (nicht notwendig algebraisch
abgeschlossen) kann nach der Proposition \emph{nicht} geschehen.
\end{rem*}
