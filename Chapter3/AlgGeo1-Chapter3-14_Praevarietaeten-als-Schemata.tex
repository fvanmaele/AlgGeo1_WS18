\section{Prävarietäten als Schemata}

Wir wollen einen Funktor von der Kategorie der Prävarietäten in die
Kategorie der Schemata sodass, wenn wir eine geweissen Unterkategorie
von $\sh$ betrachten, eine Äquivalenz von Kategorien entsteht.\medskip{}

\textbf{Erinnerung: }$k=\overline{k}$: $\mathbb{A}_{k}^{2}=\Spec(k[X,Y])$
besteht aus
\begin{itemize}
\item Punte des $\mathbb{A}^{2}(k)$ $\leadsto$ maximale Ideale, 0-dimensionale
  Teilmengen.
\item Irreduzible Kurve $f(x,y)=0$ $\leadsto$ Primideale, 1-dimensionale
  Teilmengen.
\item Generischer Punkt 0 $\leadsto$ 2-dimensionale Teilmengen.
\end{itemize}
Wie können wir die zusätzlichen Punkte für den Funktor
\[
  \pres\longrightarrow\schs
\]

präzisieren? Sei $X$ ein topologischer Raum, in dem alle Punkte abgeschlossen
sind. Betrachte 
\[
  t(X)=\{Z\subset X\mid Z\text{ irreduzibel abgeschlossen}\},
\]

versehen mit der Topologie: I$t(X)\supset t(Z)$, $Z\subseteq_{\text{abg.}}X$
bilden die abgeschlossenen Mengen. Überprüfe: $Z_{1},Z_{2},Z_{i}\subset X$
abgeschlossen $\Longrightarrow t(\cap_{i}Z_{i})=\cap_{i}t(Z_{i})$,
$t(Z_{1}\cup Z_{2})=t(Z_{i})\cup t(Z_{2})$. Ist $f:X\rightarrow Y$
stetig, so auch
\begin{align*}
  t(f):t(X) & \longrightarrow t(Y)\\
  Z & \longmapsto\overline{f(Z)}
\end{align*}

denn:
\begin{enumerate}
\item $\overline{f(Z)}$ irreduzibel: Sei $\overline{f(Z)}=A_{1}\cup A_{2}$,
  $A_{i}\neq\emptyset$. Dann existiert $z_{1},z_{2}\in Z$ mit $f(z_{i})\in A_{i}$,
  denn sonst gilt $f(z)\subseteq A_{1}$. $Z\subseteq f^{-1}(f(Z))$
  abgeschlossen. $\Longrightarrow Z=(f^{-1}(A_{1})\cap Z)\cup(f^{-1}(A_{2}\cap Z))$,
  Widerspruch.
\item Sei $t(Y')\subseteq t(Y)$ abgeschlossen. $t(f)^{-1}(t(Y))=\{Z\in t(X)$,
  $\overline{f(Z)}\in t(Y')\}$ denn:
  \begin{itemize}
  \item[,,$\subseteq$``] $\overline{f(Z)}\subset Y'$ $\Longrightarrow Z\in f^{-1}(\overline{f(Z))}\subset f^{-1}(Y)=t(f^{-1}(Y))$
  \item[,,$\supseteq$``] $z\in f^{-1}(Y)$ abgeschlossen $\Longrightarrow f(Z)\in\overline{f(Z)}\subset\overline{Y'}\subset Y'$.
  \end{itemize}
\end{enumerate}
Wir erhalten einen Funktor
\[
  t:\topcp\longrightarrow\Top
\]

Die irreduziblen Mengen von $f(X)$ sind gerade die $t(X)$, $Z\subseteq X$
irreduzibel. $Z\in t(Z)$ ist der eindeutige generische Punkt. Sei
\begin{align*}
  \alpha_{X}:X & \longrightarrow t(X)\\
  x & \longmapsto\{x\}\text{ irred. abg.}
\end{align*}

So ist die Abbildung
\begin{align*}
  \text{\{abg. Tm. von }f(X)\} & \longrightarrow\text{\{abg. Tm. von }X\}\\
  A=t(Z) & \longmapsto\alpha_{X}^{-1}(A)=\{x\in X:\{x\}\in t(Z)\}=Z
\end{align*}

eine Bijektion. $\Longrightarrow\alpha_{X}$ ist Homömorphismus von
$X$ auf die abgeschlossenen Punkte $|t(X)|$ von $t(X)$ {[}irred.
abg. Teilmengen $Z$ von $X$, die in $t(X)$ abgeschlossen sind.{]}

Es ist $\{Z\}=t(Z')$ für ein $Z'\subset X$ abgeschlossen. $\Longrightarrow$
Nur ein Punkt $x\in X$ in $Z$, sonst $\{x\}\subsetneq Z\subset Z'$
beide in $t(Z')$.

Es ist $|t(X)|\subset t(Y)$ eine sehr dichte Menge (Bijektion oben).
\begin{thm}[31]
  Der Funktor $X\mapsto(t(X),(\alpha_{X})_{\ast}\mathcal{O}_{X})$
  induziert eine Äquivalenz von Kategorien:
  \begin{align*}
    t:\{\prek\} & \overset{1:1}{\longleftrightarrow}\text{\{integere }k\text{-Schemata v. endl. Typ\}}\\
    \{\affk\} & \overset{1:1}{\longleftrightarrow}\text{\{affine }k\text{-Schemata v. endl. Typ}\}
  \end{align*}
\end{thm}

\begin{proof}
  Ist $X$ eine affine Varietät über $k$ mit $\Gamma(X)=A$, so ist
  $X=\maxspec(A)$. $\Longrightarrow t(X)=\Spec A$ (vgl Kapitel I),
  $\mathcal{O}_{X}(D(f))=A_{f}$, $f\in A$. $\Longrightarrow$ Behauptung
  im affinen Fall.

  Ist $f:X\rightarrow Y$ Morphismus von Prävarietäten, so erhalten
  wir 
  \begin{align*}
    t(f):t(X) & \longrightarrow t(Y),\\
    (\alpha_{Y})_{\ast}\mathcal{O}_{Y} & \longrightarrow t(f)_{\ast}((\alpha_{X})_{\ast}\mathcal{O}_{X})
  \end{align*}

  Morphismus lokal geringter Räume, da ein Morphismus von Garben auf
  $X$ und $Y$ durch Komposition von Abbildungen gegeben ist!

  Quasi-inverser Funktor $(X,\mathcal{O}_{X})\mapsto(X(k),\mathcal{O}_{X(k)}=\alpha^{-1}\mathcal{O}_{X})$
  geringter Raum. \textbf{(1)} $\alpha^{-1}(U)=U\cap(X)\overset{1:1}{\longleftrightarrow}U$
  offene Teilmenge. \textbf{Behauptung}: Bild $(X(k),\mathcal{O}_{X(k)})$
  ist Raum mit Funktionen: Sei $V\subseteq U\subseteq X$ offen. \textbf{(2)}
  Das Diagramm
  \[
    \xymatrix{\mathcal{O}_{X(k)}(U\cap X(k))\ar[d]_{\res}\ar[r] & \Abb(U\cap X(\xi),\xi)\ar[d]^{\res}\\
      \mathcal{O}_{X(k)}(V\cap X(k))\ar@{^{(}->}[r] & \Abb(V\cap X(k),k)
    }
  \]

  kommutiert. Dazu $f\in\mathcal{O}_{X(k)}(U\cap X(k))\overset{(1)}{=}\mathcal{O}_{X}(U)$,
  wir assoziieren es der Abbildung 
  \[
    U\cap X(k)\longrightarrow k,\quad x\mapsto f(x):=\pi_{x}(f),
  \]

  mit
  \[
    \xymatrix{\pi_{x}:\mathcal{O}_{X}(U)\ar[r]\ar[d]^{\res} & \mathcal{O}_{X,x}\ar[r] & \kappa(x)=k\\
      \mathcal{O}_{X}(V)\ar[ur]
    }
  \]

  $\Longrightarrow(2)$. \textbf{(3)} $f,g$ mit derselben Funktion
  \[
    f\equiv g:U'\rightarrow k\overset{!}{\Longrightarrow}f=g
  \]

  Garbenaxiom $\Longrightarrow$ kann lokal überprüft werden: $U=\Spec A$
  und $\pi_{x}(f)=\pi_{x}(g)$ für alle $x\in\maxspec$ 
  \[
    \Longrightarrow f-g\in\bigcap_{\mathfrak{m}\in\maxspec(A)}\mathfrak{m}=\nil(A)=0,
  \]

  da $A$ lokal reduzierte $k$-Algebra. Da sich $X$ durch endlich
  viele affine Schemata der Form $\Spec A$, $A$ integer endlich erzeugte
  $k$-Algebra, überdecken lässt, ist der Raum mit Funktion $X(k)$
  eine Prävarietät. Die Konstruktion ist funktoriell, da jede Menge
  von Schemata von endlichem Typ über $K$ abgeschlossene Punkte auf
  abgeschlossene Punkt schickt nach Proposition 28.

  Um zu zeigen, dass beide Funktoren Quasi-Inverse zueinander sind,
  benutze den affinen (Varietät/Schemata) Fall, so wie die Garbenaxiome.
\end{proof}
\begin{rem*}[32]
  \mbox{}Es gilt:

  \begin{align*}
    \kappa(x) & =\kappa(X(k))\\
    \mathbb{A}_{k}^{n} & \longleftrightarrow\mathbb{A}(k)\\
    \mathbb{P}_{k}^{n} & \longleftrightarrow\mathbb{P}^{n}(k)
  \end{align*}
\end{rem*}
