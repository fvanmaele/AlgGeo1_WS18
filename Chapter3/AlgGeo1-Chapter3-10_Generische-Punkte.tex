\section{Generische Punkte }
\begin{prop}[18]
  Die Abbildung
  \begin{align*}
    X & \longrightarrow\{Z\subset X\mid\text{abg., irred.}\}\\
    x & \longmapsto\overline{\{x\}}
  \end{align*}

  ist eine Bijektion, d.h. jede irreduzible abgeschlossene Teilmenge
  enthält genau einen generischen Punkt.
\end{prop}

\begin{proof}
  Gilt für affine Schemata nach Korollar II.7. Sei $Z\subset X$ irreduzibel,
  abgeschlossen sowie $U\subset X$ affin offen mit $Z\cap U\neq\emptyset$.

  $\Longrightarrow\overline{Z\cap U}^{X}=Z$, da $Z$ irreduzibel.

  $\Longrightarrow Z\cap U$ irreduzibel mit generischen Punkt $x$,
  $\overline{\{x\}}^{Z\cap U}=Z\cap U$.

  $\Longrightarrow\overline{\{x\}}^{X}=Z$.

  Umgekehrt: Sei $z\in Z$ generischer Punkt.

  $\Longrightarrow[U\subset X$ offen mit $U\cap Z\neq\emptyset$ $\Rightarrow z\in U]$
  d.h. Eindeutigkeit im affinen Fall impliziert allgemeiner Fall.
\end{proof}
``Generische Punkte reduzieren gewisse Aussagen auf das Studium von
\emph{einem} Punkt''.
\begin{prop}[19]
  Sei $f:X\rightarrow Y$ offener Morphismus von Schemata. Sei $Y=\overline{\{\eta\}}^{Y}$
  irreduzibel. Dann:
  \[
    f^{-1}(\eta)\text{ irreduzibel}\Leftrightarrow X\text{ irreduzibel}
  \]
\end{prop}

\begin{proof}
  $f$ offen $\Rightarrow\overline{\{f^{-1}(x)\}}=f^{-1}(\overline{\{\eta\}})=f^{-1}(Y)=X$.
  Mit Lemma I.14: $Z$ irreduzibel $\Leftrightarrow\overline{Z}$ irreduzibel.
\end{proof}
Topologische Räume von Schemata sind fast nie Hausdorffsch, aber:
\begin{prop}[20]
  Sei $X$ Schema. Dann ist der unterliegende topologische Raum ein
  $T_{0}$-Raum, d.h.
  \[
    \forall x\neq y\in X\ \exists U\subset X\text{ offen, mit \textbf{entweder }}x\in U\text{ oder }y\in U.
  \]
\end{prop}

\begin{proof}
  Ohne Einschränkung: $X$ affin, $x=\mathfrak{p}_{x}$, $y=\mathfrak{p}_{y}\in\Spec(\Gamma(X,\mathcal{O}_{X}))$.
  Falls $\mathfrak{p}_{x}\subsetneq\mathfrak{p}_{y}$ wähle
  \[
    \mathfrak{p}_{x}\in U=X\backslash\underbrace{V(\mathfrak{p}_{y})}_{\ni\mathfrak{p}_{y}}
  \]

  andernfalls $\exists f\in\mathfrak{p}_{x}\backslash\mathfrak{p}_{y}$,
  d.h. $U=D(f)$ enthält $y$ aber nicht $x$.
\end{proof}
Später: Separiertheit von Schemata als ``Hausdorffsch''-Ersatz.
