\section{Reduzierte und ganze Schemata }
\begin{defn}[21]
  Ein Schema $X$ heißt 
  \begin{enumerate}
  \item \textbf{reduziert}, falls alle $\mathcal{O}_{X,x}$, $x\in X$, reduzierte
    Ringe sind.
  \item \textbf{ganz}, falls $X$ reduziert und irreduzibel ist.
  \end{enumerate}
\end{defn}

\begin{prop}[22]
  \mbox{}
  \begin{enumerate}
  \item $X$ schema ist reduziert (ganz) $\Leftrightarrow\Gamma(U,\mathcal{O}_{X})$
    reduziert (integer) für alle $U\subseteq X$ offen.
  \item Sei $X$ ganz. Dann ist der Halm $\mathcal{O}_{X,x}$ integer $\forall x\in X$.
    (Die Umkehrung ist im Allgemeinen falsch!)
  \end{enumerate}
\end{prop}

\begin{proof}
  \mbox{}
  \begin{enumerate}
  \item \textbf{reduzibel,} ``$\Rightarrow$''. $f\in\Gamma(U,\mathcal{O}_{X})$
    mit $f^{n}=0$. Angenommen, $f\neq0$. Dann gibt es ein $x\in U$
    mit $f_{x}\neq0$ in $\mathcal{O}_{X,x}$, $f_{x}^{n}=0$. Widerspruch

    \textbf{reduzibel,} ``$\Leftarrow$''. Sei $\overline{f}\in\mathcal{O}_{X,x}$
    nilpotent. Dann gibt es ein $x\in U\subset X$ offen und $f\in\Gamma(U,\mathcal{O}_{X})$
    mit $f_{x}=\overline{f}$. Ohne Einschränkng: $f$ nilpotent (mit
    $U$ verkleinern sodass $f^{n}|_{U}=0$). Nach Voraussetzung ist dann
    $f=0$, also $\overline{f}=0$.

    \textbf{ganz,} ``$\Rightarrow$''. Sei $X$ ganz. Dann ist $U\subset X$
    offen ganz nach den Definitionen. Daher reicht es zu zeigen, dass
    $\Gamma(X,\mathcal{O}_{X})$ integer ist. Seien $f,g\in\mathcal{O}_{X}(X)$
    mit $fg=0$. Dann ist $X=V(f)\cup V(g)$. $X$ ist irreduzibel, also
    etwa $X=V(f)$. \emph{Behauptung}: $f=0$.

    Da Verschwinden aufgrund des Garbenaxioms eine lokale Frage ist, setze
    ohne Einschränking $X=\Spec A$ affin. Es folgt: $f\in\bigcap_{\Spec A}\mathfrak{p}=\sqrt{(0)}=0$.

    \textbf{ganz, }``$\Leftarrow$''. $\Gamma(U,\mathcal{O}_{X})$ integer,
    also reduziert. Nach (1, reduziert) ist $X$ reduziert. Angenommen
    es gibt $\emptyset\neq U_{1},U_{2}\subset X$ offen mit $\emptyset=U_{1}\cap U_{2}$.
    Nach den Garbenaxiomen enthält dann $\Gamma(U_{1}\cup U_{2},\mathcal{O}_{X})=\Gamma(U_{1},\mathcal{O}_{X})\times\Gamma(U_{2},\mathcal{O}_{X}$)
    Nullteiler $(1,0)\cdot(0,1)=0$. Widerspruch.
  \item Folgt aus 1, da $A$ integer, $0\notin S$. Es folgt: $A_{S}$ integer
    ($\subseteq\Quot(A)$).
  \end{enumerate}
\end{proof}
\begin{rem*}
  $X=\Spec A$ ganz $\Leftrightarrow A$ integer, $\eta\in X$ generischer
  Punkt $\Leftrightarrow(0)\subset A$. Es ist $\mathcal{O}_{X,\eta}=A_{(0)}=\Quot(A)$,
  d.h. für jedes ganze Schema $X$ gilt: $\mathcal{O}_{X,\eta}$ ist
  Körper (mit generischer Punkt $\eta$).
\end{rem*}
\begin{defn}[23]
  $X$ ganz, $\eta\in X$ generischer Punkt. Dann heißt $K(X):=\mathcal{O}_{X,\eta}$
  der \textbf{Funktionenkörper} von $X$.
\end{defn}

\begin{prop}[24]
  Sei $X$ noethersches irreduzibles Schema, $\eta\in X$ generischer
  Punkt. Dann sind äquivalent:
  \begin{enumerate}
  \item $\mathcal{O}_{X,\eta}$ ist reduziert.
  \item $\exists\emptyset\neq U\subset X$ reduziertes offenes Unterschema.
  \end{enumerate}
  Für $(ii)$ sagt man auch: $\mathcal{O}_{X,x}$ ist \textbf{generisch}
  reduziert.
\end{prop}

\begin{proof}
  \mbox{}
  \begin{itemize}
  \item[$``\Rightarrow"$] Sei ohne Einschränkung $X=\Spec A$ affin, $A$ nach Voraussetzung
    noethersch, $\eta\leftrightarrow\mathfrak{p}$ eindeutiges minimales
    Primideal ($A$ irreduzibel). $\Longrightarrow\mathfrak{p}=(f_{1},\ldots,f_{n})_{A}$
    endlich erzeugt. $\Longrightarrow\frac{f_{i}}{1}\in\nil(A_{\mathfrak{p}})=(0)\subset A_{\mathfrak{p}}=\mathcal{O}_{X,\eta}$,
    da $\mathcal{O}_{X,\eta}$ reduziert. $\exists g\in A\backslash\mathfrak{p}$
    $\Longrightarrow$ d.d. $\frac{f_{1}}{1}=0\sim A_{g}$. $\Longrightarrow0=\nil(A)_{g}=\nil(A_{g})$,
    d.h. $A_{g}$ ist reduziert, d.h. $U:=D(g)$.
  \item[$``\Leftarrow"$] $\emptyset\neq U\subset X$ reduziert offen. $\Longrightarrow\eta\in U$
    s.d. $\mathcal{O}_{X,x}=\mathcal{O}_{X,\eta}$ reduziert.
  \end{itemize}
\end{proof}
\begin{rem*}
  Analog zeigt man: $X$ noethersches Schema und $\mathcal{O}_{X,x}$
  reduzibel für ein $x\in X$ $\Longrightarrow\exists x\in U\subset X$
  offen, d.d. $U$ reduziert ist.
\end{rem*}
