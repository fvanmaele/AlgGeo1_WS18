\section{Verkleben von Schemata und disjunkte Vereinigung}
\begin{defn}
  Ein \textbf{Verklebe-Datum} von Schemata besteht aus:
  \begin{itemize}
  \item einer Indexierung $I$;
  \item ein Schemata $U_{i}$ für $i\in I$;
  \item ein affines Unterschemata $U_{ij}\subset U$ für alle $i,j\in I$;
  \item einen Isomorphismus $U_{ij}\stackrel[\cong]{\varphi_{ji}}{\longrightarrow}U_{ji}$
    für alle $(i,j)\in I\times I$, sodass:
    \begin{enumerate}
    \item $U_{ii}=U_{i}$ für alle $i\in I$;
    \item (Kozykel-Bedingung): $\varphi_{kj}\circ\varphi_{ji}=\varphi_{ki}$
      auf $U_{ij}\cap U_{ik}$, für alle $i,j,k\in I$.
    \end{enumerate}
  \end{itemize}
\end{defn}

Für die Kozykel-Bedingung soll implizit gelten:
\begin{align*}
  \varphi_{ji}(U_{ij}\cap U_{ik}) & \subseteq U_{jk}\\
  i=j=k & \Rightarrow\varphi_{ii}=\id_{U_{i}},\\
  \varphi_{ij}^{-1} & =\varphi_{ji},\text{ und}\\
  \varphi_{ji}:U_{ij}\cap U_{ik} & \overset{\cong}{\rightarrow}U_{ji}\cap U_{jk}
\end{align*}

\begin{prop}[9]
  Zu einem Verklebe-Datum $((U_{i})_{i\in I},(U_{ij})_{i,j\in I},(\varphi_{ij})_{i,j\in I})$
  gibt es ein Schemata $X$ zusammen mit Morphismen $\psi_{i}:U_{i}\rightarrow X$,
  sodass
  \begin{itemize}
  \item für alle $i\in I$ induziert $\psi_{i}$ einen Isomorphismus von $U_{i}$
    auf offene Unterschemata von $X$;
  \item $\psi_{j}\circ\varphi_{ji}=\psi_{i}$ auf $U_{ij}$ für alle $i,j\in I$;
  \item $X=\bigcup_{i}\psi_{i}(U)$;
  \item $\psi_{i}(U_{i})\cap\psi_{j}=\psi_{i}(U_{ij})=\psi_{j}(U_{ji})$ für
    alle $i,j\in I$.
  \end{itemize}
  $(X,\psi_{i\in I})$ ist eindeutig bis auf eindeutige Isomorphie bestimmt.
\end{prop}

Zusammen mit Proposition 5 folgt die universelle Eigenschaft: Für
$(T,\xi_{i}:U_{i}\rightarrow T)$ mit $\xi_{i}$ welche Isomorphismen
\[
  U_{i}\overset{\cong}{\rightarrow}\text{\{offenes Unterschemata von }T\}
\]

induzieren, sodass $\xi_{j}\circ\varphi_{ji}=\xi_{i}$ auf $U_{ij}$
für alle $i,j\in I$, dann gibt es einen eindeutigen Morphismus $\xi:X\rightarrow T$
mit $\xi\circ\psi_{i}=\xi_{i}$ für alle $i\in I$. ($\Longrightarrow$
Eindeutigkeit von Proposition 9)

Als topologischer Raum: $\coprod_{i\in I}U_{i}/\sim$ mit $x_{i}\in U_{i}\sim x_{j}\in U_{j}:\Leftrightarrow x_{i}\in U_{ij}$,
$x_{j}\in U_{ji}$ und $x_{j}=\varphi_{ji}(x_{i})$. Nach Eigenschaft
$(b)$ ist $\sim$ eine Äquivalenzrelation. Dann sind $\psi_{i}:U_{i}\rightarrow X$
injektiv. Ferner haben wir $\forall i,j\in I$ die Eigenschaft $\psi_{i}(U_{i})=\psi_{i}(U_{i})\cap\psi_{j}(U_{j})$.

$X$ hat also als topologischer Raum die Quotiententopologie, d.h.
die feinste Topologie sodass alle Abbildungen $\psi_{i}$ stetig sind.
$U\subset X$ offen genau dann, wenn $\psi_{i}^{-1}(U)\subset U_{i}$
dort offen sind $\forall i\in I$. Insbesondere sind $\psi_{i}(U_{i})$
und $\psi_{i}(U_{j})=\psi_{i}(U_{i})\cap\psi_{j}(U_{j})$ offen in
$X$.

Als lokal geringter Raum: ``Verkleben der Strukturgarben auf $U_{i}$''.
$\mathcal{O}_{X}$ ist eindeutig auf einer Basis $B$ der Topologie
definiert. Ohne Einschränkung reicht es hier, die Schnitte nur auf
$U\subset X$ offen mit $U\subset\psi_{i}(U_{i})$ für ein $i\in I$.
In dem Fall:
\[
  \mathcal{O}_{X}(U)=\mathcal{O}_{U_{i}}(\psi_{i}^{-1}(U))
\]

Für $U\subset\psi_{i}(U_{i})\cap\psi_{j}(U_{j})$
\[
  \xymatrix{U_{ij}\ar@{^{(}->}[r]\ar[d]_{\cong} & U_{i}\\
    U_{ji}\ar@{^{(}->}[r] & U_{i} & X\supset U
  }
\]

Dann gilt:

\[
  \mathcal{O}_{U_{i}}(\psi_{i}^{-1}(U))=\mathcal{O}_{U_{ij}}(\psi_{i}^{-1}(U))\cong\mathcal{O}_{U_{ji}}(\psi_{j}^{-1}(U))=\mathcal{O}_{U_{j}}(\psi_{j}^{-1}(U))
\]

Es folgt: $\mathcal{O}_{X}(U)$ unabhängig von Wahlen von $i$! Wir
halten damit $\mathcal{O}_{X}$ Ringgarbe auf $X$, sodass $(X,\mathcal{O}_{X})$
lokal geringter Raum da $(U_{i},\mathcal{O}_{U_{i}})$ lokal geringter
Raum $\forall i\in I$, $U_{i}\xrightarrow[\psi_{i}]{\cong}(\psi_{i}(U_{i}),\mathcal{O}_{X}|_{\psi_{i}(U_{i})})$
als lokal geringter Raum. Damit ist $X$ ein Schema und $X=\cup U_{i}$.

Spezialfall: $U_{ij}=\emptyset$ für alle $i\neq j\in I$, $\coprod U_{i}$
``disjunkte Vereinigung''.
\end{proof}
\begin{example}[10]
  $X_{1},\ldots,X_{n}$ affine Schemata, $X_{i}=\Spec A_{i}$. Dann
  ist
  \[
    \coprod X_{i}\cong\Spec\left(\prod_{i=1}^{n}A_{i}\right)\text{ offen}.
  \]

  (nicht für unendlich viele affin!)
\end{example}

\begin{example}[11]
  $I=\{1,2\}$, $U_{12}\subset U_{1}\xrightarrow{\varphi}U_{21}\subset U_{2}$.
  \[
    X\underset{\text{offen}}{\subset}V=U_{1}\cup_{\varphi}U_{2},
  \]

  $\Gamma(V,\mathcal{O}_{X})=\{(s_{1},s_{2})\in\Gamma(V\cap U_{1},\mathcal{O}_{U_{1}})\times\Gamma(V\cap U_{2},\mathcal{O}_{U_{2}})\}$,
  $\varphi^{\flat}(S_{2}|_{U_{21}\cap V})=S_{1}|_{U_{12}\cap V}$.
\end{example}

\textbf{Affine Gerade mit Doppelpunkt}: $k$ Körper.
\begin{align*}
  U_{1} & =U_{2}=\mathbb{A}_{k}^{1}=\Spec(k[T])\cong x\text{ abg.}\\
  U_{12} & :=U_{1}\backslash\{x\}\\
  U_{21} & :=U_{2}\backslash\{x\},\ \varphi=\id
\end{align*}

$X=U_{1}\cup_{\varphi}U_{2}$.  
\begin{xca*}
  $X$ ist \textbf{nicht} affin!
\end{xca*}
