\chapter*{Grundlegende Eigenschaften von Schemata und Morphismen}

\section{Topologische Eigenschaften }
\begin{defn}[12]
  Ein Schema $X$ heißt \textbf{zusammenhängend}, \textbf{quasi-kompakt}
  bzw. \textbf{irreduzibel}, falls der unterliegende topologische Raum
  diese Eigenschaft besitzt.
\end{defn}

\begin{itemize}
\item Nach Proposition II.5 ist jedes affine Schema quasi-kompakt.
\item $\coprod_{i=0}\Spec(R)$ ist \emph{nicht }quasi-kompakt.
\end{itemize}
\begin{defn}[13]
  $f:X\rightarrow Y$ heißt \textbf{injektiv} (surjektiv, bijektiv),
  falls die zugrendlegende stetige Abbildung diese Eigenschaft hat.
  Ebenso für ``offen'', ``abgeschlossen'', ``Homömorphismus''.
\end{defn}

Warnung: Homömorphismen von Schemata sind im Allgemeinen \emph{keine
}Isomorphismen!
