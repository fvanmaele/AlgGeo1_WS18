\section{Noethersche Schemata }
\begin{defn}[14]
  Ein Schema heißt \textbf{lokal noethersch}, falls eine affine offene
  Überdeckung $X=\bigcup U_{i}$ existiert, d.d. alle $\Gamma(U_{i},\mathcal{O}_{X})$
  (affine Koordinatenringe) \textbf{noethersch} sind. $X$ heißt \textbf{noethersch},
  falls zusätlich quasi-kompakt.
\end{defn}

Faktum: Lokalisierung noetherscher Ringe bleiben noethersch.

$\Rightarrow a)$ Jedes lokal noethersche Schema besitzt eine Basis
der Topologie aus noetherschen affin offenen Unterschemata.

b) $X$ lokal noethersch. Dann ist $\mathcal{O}_{X,x}$ noethersch
$\forall x\in X$.

Für offene Schemata gilt ferner: lokal noethersch $\Rightarrow$ noethersch.

\begin{prop}[15]
  Für $X\subset\Spec A$ affin gilt:
  \[
    X\text{ noethersch }\Leftrightarrow A\text{ noethersch}
  \]
\end{prop}

\begin{proof}
  \mbox{}
  \begin{itemize}
  \item[``$\Leftarrow$''] $X$ überdeckt sich selbst mit $\Gamma(X,\mathcal{O}_{X})=A$ noethersch.
  \item[``$\Rightarrow$''] Sei $I\subset A$ beliebiges Ideal. Zu zeigen: $I$ ist endlich erzeugt.
    Nach Voraussetzung ist
    \[
      X=\bigcup_{i=1}^{n}\Spec A_{i},\quad A_{i}\text{ noethersch}.
    \]
    Ohne Einschränking: $A_{i}=A_{f_{i}}$ und noethersch. Daraus folgt:
    $J_{i}=IA_{f_{i}}=I_{f_{i}}$ sind endlich erzeugt, Behauptung folgt
    aus Lemma 16.
  \end{itemize}
\end{proof}
\begin{lem}[16]
  $\Spec(A)=\cup_{i\in I}D(f_{i})$, $\#I<\infty$, $M$ $A$-Modul.
  Dann:
  \[
    M\text{ e.e. über }A\Leftrightarrow M_{f_{i}}\text{ e.e. }A_{f_{i}}\text{-Modul }\forall i\in I.
  \]
\end{lem}

\begin{proof}
  \mbox{}
  \begin{itemize}
  \item[``$\Rightarrow$'' ] Endlich erzeugt heißt $A^{n}\twoheadrightarrow M$, Lokalisierung
    exakt also $A_{f_{i}}^{n}\twoheadrightarrow M_{f_{i}}$ exakt.
  \item[``$\Leftarrow$''] $M_{f_{i}}$ werden von $\frac{m_{ij}}{f_{i}^{n_{ij}}}$, $j=1,\ldots,r_{i}$,
    $m_{ij}\in M$, $n_{ij}\in\mathbb{N}_{0}$ als $A_{f_{i}}$-Modul
    erzeugt.

    $\Rightarrow N:=\langle m_{ij}\rangle_{A}\subset M$ ist endlich erzeugt
    und $N_{f_{i}}=M_{f_{i}}$.

    $\Rightarrow(M/N)_{\mathfrak{p}}=(M_{f_{i}}/N_{f_{i}})_{\mathfrak{p}}=0$
    für alle Primideale $\mathfrak{p}\in\Spec A$.

    $\Rightarrow$ (Lokal-Global-Prinzip aus der kommutativen Algebra)
    $N=M$.

  \end{itemize}
\end{proof}
\begin{rem*}
  $X$ noethersches Schema. Dann ist $X$ als topologischer Raum noethersch.
\end{rem*}
\begin{proof}
  Für $X$ affin klar, sonst $X=\cup_{i=1}^{r}X_{i}$, $X_{i}=\Spec(A_{i})$
  noethersch. Sei
  \[
    X\supseteq Z_{1}\supseteq\cdots\supseteq Z_{n}\supseteq\cdots
  \]

  absteigende Kette abgeschlossener Teilmengen. $(Z_{j}\cap X_{i})_{j}$
  absteigende Kette abgeschlossener Teilmengen in $X_{i}$.

  $\Longrightarrow$ (endliche Überdeckung) $\exists N$ d.d. $Z_{j}\cap X_{i}=Z_{N}\cap X_{i}$
  für alle $j\geq N$.

  $\Longrightarrow Z_{j}=Z_{N}$.
\end{proof}
\begin{cor}[17]
  Sei $X$ (lokal) noethersches Schema, $U\subset X$ offenes Unterschema.
  Dann ist $U$ ein (lokal) noethersches Schema.
\end{cor}

\begin{proof}
  Lokal noethersch $X=\bigcup U_{i}$, $U_{i}\cap U=\bigcup D(f_{i})$.
  Sei $X$ noethersch. Dann ist der topologische Raum $X$ noethersch.
  Nach Lemma $I.20$ ist dann jede offene Teilmenge quasi-kompakt.
\end{proof}
