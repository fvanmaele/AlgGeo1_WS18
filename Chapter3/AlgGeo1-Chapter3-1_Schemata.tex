= Verkleben affiner Schemata

\section{Schemata}
\begin{defn}
  Ein Schemata ist ein lokal geringter Raum $(X,\mathcal{O}_{X})$,
  der eine offene Überdeckung $(U_{i})_{i\in I}$ besitzt, so dass alle
  lokal geringten Räume $(U_{i},\mathcal{O}_{X|U_{i}})$ affine Schemata
  sind. Für ein Schemata $S$ bezeichne $\schs$ die \textbf{Kategorie
    der Schemata über $S$} oder $S$-Schemata. Die Objekte dieser Kategorie
  sind Morphismen $X\rightarrow S$ von Schemata, und die Morphismen
  $\Hom(X\rightarrow S,Y\rightarrow S)$ sind Morphismen $X\rightarrow Y$
  von Schemata so dass
  \[
    \xymatrix{X\ar[rr]\ar[dr] &  & Y\ar[dl]\\
      & S
    }
  \]

  kommutiert. $X\rightarrow S$ heißt \textbf{Strukturmorphismus} des
  $S$-Schematas $X$. Ist $S=\Spec R$ affin, spricht man auch von
  $R$-Schemata oder Schemata über $R$. Die Menge der Morphismen $X\rightarrow Y$
  in $\schs$ bezeichnen wir mitt $\Hom_{S}(X,Y)$ bzw. $\Hom_{R}(X,Y)$
  falls $S=\Spec R$ affin ist.
\end{defn}
