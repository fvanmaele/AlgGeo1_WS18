
\section{Affine algebraische Mengen}

\subsection{Definition 4}
\begin{itemize}
\item $\mathbb{A}^{n}(k)$, der affine Raum der Dimension $n$ (�ber $k$),
bezeichne $k^{n}$ mit der Zariski-Topologie.
\item Abgeschlossene Teilmengen von $\mathbb{A}^{n}(k)$ hei�en affine abgeschlossene
Mengen.
\end{itemize}

\subsection{Beispiel 5}

Da $k[T]$ ein Hauptidealring ist, sind die abgeschlossen Mengen in
$\mathbb{A}^{1}(k)$: $\emptyset$, $\mathbb{A}^{1}$, Mengen der
Form $V(f)$, $f\in k[T]\backslash\{k\}$ (endliche Teilmengen). Insbesondere
sieht man, dass die Zariski-Topologie im Allgemeinen nicht Hausdorff
ist. 

\subsection{Beispiel 6}

$\mathbb{A}^{2}(k)$ hat zumindestens als abgeschlossene Mengen:
\begin{itemize}
\item $\emptyset$, $\mathbb{A}^{2}$;
\item Einpunktige Mengen: $\{(x_{1},x_{2})=V(T_{1}-x_{1},T_{2}-x_{2})$;
\item $V(f)$, $f\in k[T_{1},T_{2}]$ irreduzibel.
\end{itemize}
Ferner alle endliche Vereneigungen dieser Liste. (Dies sind in der
Tat alle, denn sp�ter sehen wir: ``irreduzibele'' abgeschlossene
Mengen entsprechen den \emph{Primidealen}, und $k[T_{1},T_{2}]$ hat
``Krull-Dimension $2$''.)
