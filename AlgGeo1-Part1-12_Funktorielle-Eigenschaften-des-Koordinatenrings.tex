\selectlanguage{english}%

\section{Funktorielle Eigenschaften von $\Gamma(X)$}

\subsection*{Satz 29}

F�r einen Morphismus $X\xrightarrow{f}Y$ affiner algebraischer Mengen
definiert 
\begin{align*}
\Gamma(f):\hom(Y=\Gamma(Y),\mathbb{A}^{1}(k)) & \rightarrow\hom(X=\Gamma(X),\mathbb{A}^{1}(k))\\
g & \mapsto g\circ f
\end{align*}

ein Homomorphismus von $k$-Algebren. Der so definierte \emph{kontravariante}
Funktor
\[
\Gamma:\{\text{affine algebraische Mengen\}}\rightarrow\{\text{red. endl. erz. }k\text{-Alg.}\}
\]

liefert eine Kategorien�quivalenz, der durch Einschr�nkung eine �quivalenz
\[
\Gamma:\{\text{irred. aff. alg. Meng.\}}\rightarrow\{\text{integere endl. erz. }k\text{-Alg.\}}
\]

induziert.

\subsubsection*{Beweis (Satz 29)}

$Y\xrightarrow{g}\mathbb{A}^{1}(k)\in\Gamma(Y)$ ist Morphismus. $\Rightarrow g\circ f:X\xrightarrow{f}Y\xrightarrow{g}\mathbb{A}^{1}(k)$
ist ..., d.h. $\in\Gamma(X)$. 

$\Gamma(f):\Gamma(Y)\rightarrow\Gamma(X)$ ist ein $k$-Alg.-Hom.
und Kompositions....  (nach Bemerkung 24) mit $\Gamma(\text{id}_{X})=\text{id}_{\Gamma(X)}$.
Da ferner $\Gamma(f_{1}\circ f_{2})=\Gamma(f_{2})\circ\Gamma(f_{1})$
aus der Definition folgt, ist $\Gamma$ ein kontravarianter Funktor.

\paragraph{Behauptung}

$\Gamma$ ist volltreu, d.h.
\begin{align*}
\Gamma:\hom(X,Y) & \rightarrow\hom(\Gamma(Y),\Gamma(X))\\
f & \mapsto\Gamma(f)
\end{align*}

ist \emph{bijektiv} f�r alle affinen algebraischen Mengen $X,Y$.

\paragraph{Beweis}

Wir konstruieren eine Umkehrabbildung wie folgt: Zu $\varphi:\Gamma(Y)\rightarrow\Gamma(X)$
f�r $X\subseteq\mathbb{A}^{n}$, $Y\subseteq\mathbb{A}^{n}$ existiert:
\[
\xymatrix{k[T_{1}',\ldots,T_{k}']\ar[r]^{\tilde{\varphi}}\ar@{->>}[d] & k[T_{1},\ldots,T_{m}]\ar@{->>}[d]\\
\Gamma(Y)\ar[r] & \Gamma(X)
}
\]
kommutiert ($\tilde{\varphi}(T_{1}'):=$ impede  liften $\varphi(\pi(T_{i}'))$
in  $k[\underline{T}]$). Definiere
\begin{align*}
f:X & \rightarrow Y\\
x=(x_{1},\ldots,x_{n}) & \mapsto(\tilde{\varphi}(T_{1}')(x_{1},\ldots,x_{n}),\ldots,\tilde{\varphi}(T_{n}')(x_{1},\ldots,x_{n}))
\end{align*}


\paragraph{Behauptung}

$\Gamma$ ist essentiell surjektiv, d.h. zu jeder reduzierten endlich
erzeugten $k$-Algebra $A$ existiert eine affine algebraische Menge
$X$ mit $A\cong\Gamma(X)$.

\paragraph{Beweis}

Da nach Voraussetzung $A\cong k[T]/\mathfrak{A}$ f�r Radikalideal
$\mathfrak{A}$, k�nnen wir etwa $X:=V(\mathfrak{A})\subseteq\mathbb{A}^{n}(k)$
setzen. Der Rest folgt aus Satz 28.

\subsection*{Satz 30}

Sei $f:X\rightarrow Y$ ein Morphismus und $\Gamma(f):\Gamma(Y)\rightarrow\Gamma(X)$
der zugeh�rige Homomorphismus der Koordinatenringe. Dann gilt $\forall x\in X$:
$\Gamma(f)^{-1}(\mathfrak{m}_{x})=\mathfrak{m}_{f}(x)$.

\subsubsection*{Beweis (Satz 30)}

Setting: 
\begin{align*}
\mathfrak{m}_{f(x)}=\{g\mid g(f(x))=0\}\subset\hom(Y,\mathbb{A}^{1})=\Gamma(Y) & \xrightarrow{\Gamma(f)}\Gamma(X)=\hom(X,\mathbb{A}^{1})\supset\{k\mid k(x))=0\}\\
g & \mapsto g\circ f
\end{align*}

$\Gamma(f)^{-1}(\mathfrak{m}_{x})=\{g\in\Gamma(Y)\mid g\circ f(x)\neq0\}=\mathfrak{m}_{f(x)}$,
da $\Gamma(f)(g)(x)=g(f(x))$.\selectlanguage{ngerman}%

