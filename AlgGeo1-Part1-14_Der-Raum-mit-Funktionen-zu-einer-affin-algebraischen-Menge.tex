\selectlanguage{english}%

\section{Der Raum mit Funktionen zu einer affin algebraischen Menge}
\begin{description}
\item [{Ziel.}] $X\subseteq\mathbb{A}^{n}(k)\mapsto(X,\mathcal{O}_{X})$
als irreduzibele affine algebraische Menge bzw. Zariski-Topologie.
D.h. wir m�ssen Menge von Funktionen $\mathcal{O}_{X}(U)$ auf $U$,
$U\subset X$ offen, definieren. Diese werden als Teilmengen des Funktionenk�rpers
$K(X)$ definiert (dazu $X$ irreduzibel, sp�ter bei Schemata f�llt
diese Bedingung weg!)
\end{description}

\subsection*{Definition 33}

$K(X):=\text{Quot}(\Gamma(X))$ hei�t \textbf{Funktionenk�rper} von
$X$ ($\Gamma(X)$ ist f�r $X$ irreduzibel nullteilerfrei).

Elemente $\frac{f}{g}\in K(X)$, $f,g\in\Gamma(X)=\hom(X,\mathbb{A}^{1}(k))$,
$g\neq0$ lassen sich zumindest als Funktion auf der offenen Menge
$\mathcal{D}(g)\subset X$ auffassen, wenn auch nicht i.A. auf ganz
$X$.

\subsection*{Lemma 34}

Gilt f�r $\frac{f_{1}}{g_{1}},\frac{f_{2}}{g_{2}}\in K(X)$ ($f_{i},g_{i}\in\Gamma(X))$
und einer offenen Teilmenge $\emptyset\neq U\subset\mathcal{D}(g_{1}g_{2})$
\[
\frac{f_{1}(x)}{g_{1}(x)}=\frac{f_{2}(x)}{g_{2}(x)}\qquad\forall x\in U,
\]

dann folgt $\frac{f_{1}}{g_{1}}=\frac{f_{2}}{g_{2}}$ in $K(X)$.

\subsubsection*{Beweis (Lemma 34)}

Ohne Einschr�nkung der Allgemeinheit: $g_{1}=g_{2}=g$ (sonst Erweitern!)

$\Rightarrow(f_{1}-f_{2})(x)=0$ $\forall x\in U$

$\Rightarrow\emptyset\neq U\subset V(f_{1}-f_{2})\subset X$ dicht

d.h. $V(f_{1}-f_{2})=X$.

$f_{1}-f_{2}\in IV(f_{1}-f_{2})=I(X)\equiv(0)$ in $\Gamma(X)$

$\Rightarrow f_{1}-f_{2}=0$.

\subsection*{Definition 35}

Sei $X$ eine irreduzibele affine algebraische Menge, $U\subset X$
offen. Sei $\Gamma(X)_{\mathfrak{m}_{x}}$ Lokalisierung von $\Gamma(X)$
bzgl. das maximale Ideal $\mathfrak{m}_{x}$ in $x\in X$.

$\mathcal{O}_{X}(U):=\bigcap_{x\in U}\Gamma(X)_{\mathfrak{m}_{x}}\subset K(X)$

d.h. f�r jedes $x\in U$ l�sst sich $f\in\mathcal{O}_{X}(U)$ schreiben
als $\frac{h}{g}$ mit $g(x)\neq0$.

(Wenn $f\in\Gamma(X)$ bezeichne $\Gamma(X)_{f}$ die Lokalisierung
von $\Gamma(X)$ bzgl. der multiplikativ abgeschlossenen Teilmenge
$\{1,f,f^{2},\ldots,f^{n}\ldots\}$.  Dann l�sst sich
\[
\Gamma(X)_{\mathfrak{m}_{x}}=\bigcup_{f\in\Gamma(X)\backslash\mathfrak{m}_{x}}\Gamma(X)_{f}\subset K(X)
\]

``$\supset$'' klar, ``$\subset$'' $\frac{g}{f}$ mit $f(x)\neq0$
d.h. $f\notin\mathfrak{m}_{x}$ $\Rightarrow\frac{g}{f}\in\Gamma(X)_{f}$.

\paragraph{Es gilt:}
\begin{enumerate}
\item $\mathcal{O}_{x}(U)\rightarrow\text{Abb}(U,k)$, $f\mapsto(x\mapsto f(x):=\frac{g(x)}{f(x)}\in k)$
ist injektiv (Lemma 34) und wohldefiniert (k�rzen/Erweitern) wobei
$g,h\in\Gamma(X)$ mit $h\notin\mathfrak{m}_{x}$ mit $f=\frac{g}{h}$
nach Definition von $\mathcal{O}_{X}(U)$ existiert.
\item F{[}r $V\subset U\subset X$ offen kommutiert das folgende Diagramm
\item \textbf{Verklebungseigenschaft.} Sei $U=\bigcup_{i\in I}U_{i}$. Nach
Definition ist 
\begin{align*}
\mathcal{O}_{X}(U) & =\bigcap_{i}\mathcal{O}_{X}(U_{i})\subset K(X)\\
\ni f:U\rightarrow k & \quad\ni f_{i}:U_{i}\rightarrow k
\end{align*}
. $\Rightarrow(X,\mathcal{O}_{X})$ ist Raum mit Funktionen, \textbf{der
zur irreduziblen affin algebraische Menge geh�rige Raum von Funktionen.} 
\end{enumerate}

\subsection*{Satz 36 (orig. 33)}

F�r $(X,\mathcal{O}_{X})$ zu $X$ wie oben und $f\in\Gamma(X)$ gilt:
\[
\mathcal{O}_{X}(D(f))=\Gamma(X)_{f},
\]

insbesondere $\mathcal{O}_{X}(X)=\Gamma(X)$.

\subsubsection*{Beweis (Satz 36)}

$\Gamma(X)\subset\mathcal{D}(f)$ klar, da $f(x)\neq0$ $\forall x\in\mathcal{D}(f)$
bzw. $f\in P(X)\backslash\mathfrak{m}_{x}$. 

Sei nun $g$ in $\mathcal{O}_{X}(\mathcal{D}(f))$ gegeben, $(*)$
und $\mathfrak{A}:=\{h\in\Gamma(X)\mid hg\in\Gamma(X)\}\subset\Gamma(X)$
Ideal.

Dazu: $g\in\Gamma(X)_{g}$

$\Leftrightarrow g=\frac{k}{g^{n}}$ f�r ein $n$ und $k\in\Gamma(X)$

$\Leftrightarrow f^{n}\in\mathfrak{A}$ f�r ein $n$.

d.h. zu zeigen: $f\in\text{rad}(\mathfrak{A})=IV(\mathfrak{A})$ (Hilbertsche
Nullstellensatz)

$\Leftrightarrow f(x)=0$ $\forall x\in V(\mathfrak{A})$

Ist dazu $x\in X$ mit $f(x)\neq0$, wo $x\in\mathcal{D}(f)$, so
existiert nach Voraussetzung $(*)$ $f_{1},f_{2}\in\Gamma(X)$, $f_{2}\notin\mathfrak{m}_{x}$
mit $g=\frac{f_{1}}{f_{2}}$

$\Rightarrow f_{2}\in\mathfrak{A}$. Da $f_{2}(x)\neq0$:

$\Rightarrow x\notin V(\mathfrak{A})$.

\subsection*{Bemerkung 37 (orig. 34)}
\begin{enumerate}
\item Im allgemeinen existierten f�r $f\in\mathcal{O}_{x}(U)$ \textbf{nicht}
$g,h\in\Gamma(X)$ mit $f=\frac{g}{h}$ und $h(x)\neq0$ $\forall x\in U$.
\item \textbf{Alternative Definition von $\mathcal{O}_{X}$ I.}

$\mathcal{O}_{X}(\mathcal{D}(f)):=\Gamma(X)_{f}$ $\forall f\in\Gamma(X)$

Da $\mathcal{D}(f)$ Basis der Topologie, kann es h�chstens einen
Raum mit Funktionen geben mit dieser Eigenschaft, es bleibt die Existenz
zu zeigen.
\item \textbf{Alternative Definition von $\mathcal{O}_{X}$ II.}

direkt von einer integeren endlich erzeugten $k$-Algebra $A$ ausgehend
(die $X$ bis auf Isomorphie festlegt), aber ohne ``Koordinaten''
zu w�hlen.

$X:=\{\mathfrak{m}\subseteq A\mid$ max. Ideale\}

$V(\mathfrak{A}):=\{\mathfrak{m}\subseteq A$ max. $\mid\mathfrak{m}\supseteq\mathfrak{A}\}$,
$\mathfrak{A}\subset A$ Ideal, sind die \textbf{abgeschlossenen}
Mengen.

$\mathcal{O}_{X}(U):=\bigcap_{\mathfrak{m}\in U}A_{\mathfrak{m}}\subset\text{Quot}(A)$
f�r $U\subset X$ offen (vgl. sp�ter Schemata).\selectlanguage{ngerman}%
\end{enumerate}

