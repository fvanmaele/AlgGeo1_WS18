\section{Allgemeine Schemata}
\begin{defn}[1]
  Für einen topologischen Raum $X$ ist die (Krull-)Dimension das Supremum
  der Länge aller Ketten
  \[
    Z_{0}\subsetneq Z_{1}\subsetneq\cdots\subsetneq Z_{n}\subseteq X
  \]

  irreduzibler abgeschlossener Teilmengen $Z_{i}$. $X$ sei \textbf{von
    Dimension $n$}, falls alle irreduzible Komponenten von $X$ die Dimension
  $n$ haben ($\dim\emptyset=-\infty$, sonst $\dim X\in\mathbb{N}\cup\{+\infty\}$).

  Die Dimension eines Schemas ist per Definition die Dimension des unterliegenden
  topologischen Raums, also $\dim X=\dim X_{\red}$.
\end{defn}

\begin{example}[2]
  \mbox{}
  \begin{enumerate}
  \item 
  \end{enumerate}
\end{example}
