\selectlanguage{english}%

\section{Morphismen von affinen algebraischen Mengen}

\subsection*{Definition 23}

Seien $X\subseteq\mathbb{A}^{m}(k)$, $Y\subseteq\mathbb{A}^{n}(k)$
affine algebraische Mengen. Ein \textbf{Morphismus} $X\rightarrow Y$
affiner algebraischer Mengen ist eine Abbildung $f:X\rightarrow Y$
der zugrundeliegenden Mengen, sodass $f_{1},\ldots,f_{n}\in k[T_{1},\ldots,T_{m}]$
existieren, derart dass $\forall x\in X$ gilt:
\[
f(x)=(f_{1}(x),\ldots,f_{n}(x)).
\]
\emph{Bezeichne }daf�r $\hom(X,Y)$ Menge der Morphismen $X\rightarrow Y$. 

\subsection*{Bemerkung 24}

$f:X\rightarrow Y$ l�sst sich immer fortsetzen zu einem Morphismus
\[
f:\mathbb{A}^{n}(k)\rightarrow\mathbb{A}^{m}(k)
\]
 (aber nicht eindeutig, es sei denn $X=\mathbb{A}^{m}(k)$.

\paragraph{Komposition}

\[
X\xrightarrow[f_{1},\ldots,f_{n}\in k[T_{1},\ldots,T_{m}]{]}{f}Y\xrightarrow[g_{1},\ldots,g_{r}\in k[T_{1}',\ldots,T_{m}']{]}{g}Z
\]
 mit $X\subseteq\mathbb{A}^{m}(k)$, $Y\subseteq\mathbb{A}^{n}(k)$,
$Z\subseteq\mathbb{A}^{r}(k)$. Es folgt:
\begin{align*}
g(f(x))=\, & (g_{1}(f_{1}(x),\ldots,f_{n}(x)),\ldots,g_{r}(f_{1}(x),\ldots,f_{n}(x))\\
:=\, & h_{1}(x),\ldots,h_{r}(x)
\end{align*}

d.h. $g\circ f$ ist durch Polynome $h_{i}\in k[T_{1},\ldots,T_{m}]$
gegeben, d.h. $g\circ f$ ist wieder ein Morphismus affiner algebrasischer
Mengen. Wir erhalten die \textbf{Kategorie affiner algebraischer Mengen}.

\subsection*{Beispiel 25}
\begin{enumerate}
\item Sei die Abbildung
\begin{align*}
\mathbb{A}^{1}(k) & \rightarrow V(T_{2}-T_{1}^{2})\subseteq\mathbb{A}^{2}(k)\\
x & \mapsto(x,x^{2}).
\end{align*}
Diese Abbildung ist sogar ein \emph{Isomorphismus }affiner algebraischer
Mengen, da die Umkehrabbildung
\[
(x,y)\mapsto x
\]
ebenfalls ein Morphismus ist.
\item Sei char$(k)\neq2$. Die Abbildung
\begin{align*}
\mathbb{A}^{1}(k) & \rightarrow V(T_{2}^{2}-T_{1}^{2}(T_{1}+1))\\
x & \mapsto(x^{2}-1,x(x^{2}-1))
\end{align*}
ist ein Morphismus, aber \emph{nicht }bijektiv, da $1,-1$ beide auf
$(0,0)$ abgebildet werden.\selectlanguage{ngerman}%
\end{enumerate}

