%% LyX 2.2.2 created this file.  For more info, see http://www.lyx.org/.
%% Do not edit unless you really know what you are doing.
\documentclass[english,ngerman]{article}
\usepackage[T1]{fontenc}
\usepackage[latin9]{inputenc}
\usepackage{enumitem}
\usepackage{amsmath}
\usepackage{amsthm}
\usepackage{amssymb}
\usepackage{stmaryrd}
\usepackage{stackrel}
\usepackage{makeidx}
\makeindex
\usepackage[all]{xy}

\makeatletter

%%%%%%%%%%%%%%%%%%%%%%%%%%%%%% LyX specific LaTeX commands.
%% Because html converters don't know tabularnewline
\providecommand{\tabularnewline}{\\}

%%%%%%%%%%%%%%%%%%%%%%%%%%%%%% Textclass specific LaTeX commands.
\newlength{\lyxlabelwidth}      % auxiliary length 

%%%%%%%%%%%%%%%%%%%%%%%%%%%%%% User specified LaTeX commands.
\usepackage{mathtools}
\usepackage{faktor}
\DeclareMathOperator{\rad}{rad}

\makeatother

\usepackage{babel}
\begin{document}

\title{Algebraische Geometrie}

\author{Prof. Dr. Venjakob}

\date{Vorlesung 17, 19 Oktober 2018}
\maketitle

\section*{Literatur}
\begin{itemize}
\item G�rtz, Wedhorn. \emph{Algebraic Geometry I}
\item Hartshorne. \emph{Algebraic Geometry}
\item Shajarevich. \emph{Basic Algebraic Geometry 1 u. 2}
\item Grothendieck. \emph{El�ments de g�ometrie alg�brique, EGA I-IV}
\end{itemize}

\paragraph{Kommutative Algebra}
\begin{itemize}
\item Br�ske, Ischebeck, Vogel. \emph{Kommutative Algebra}
\item Kunz. \emph{Einf�hrung in die kommutative Algebra und algebraische
Geometrie}
\end{itemize}
\tableofcontents{}

\newpage{}

\part{Pr�-Variet�ten}

\input{AlgGeo1-Part1-1_Einf�hrung.tex}


\section{Die Zariski-Topologie}
\label{sec:zariski-topologie}

\begin{defn}
\label{def:verschwindungsmenge}
Sei $M\subseteq k[T_{1},\ldots,T_{n}]=:k[\underline{T}]$ eine Teilmenge.
Mit
\[
V(M) :=\{(t_{1},\ldots,t_{n})\in k^n \mid f(t_{1},\ldots,t_{n})=0\ \boldsymbol{\forall f\in M}\}
\]

bezeichnen wir die gemeinsame \textbf{Nullstellen-(Verschwindungs-)Menge}\index{Nullstellen-Menge}
der Elemente aus $M$. (Manchmal auch $V(f_{i},i\in I)$ statt $V(\{f_{i},i\in I\})$.
\end{defn}

%% TODO:
%% das oben weg!!!
\paragraph{Notation}
Wir schreiben auch $V(f_i, i \in I)$ statt $V(\{f_i \mid i \in I\})$

\subsection{Eigenschaften}
\label{subsec:zariski-topologie-eigenschaften}
\begin{itemize}
\item $V(M)=V(\mathfrak{a})$, wenn $\mathfrak{a}=\langle M\rangle_{k[\underline{T}]}$ das
\emph{von} $M$ \emph{erzeugte Ideal} in $k[\underline{T}]$ bezeichnet.
\item Da $k[\underline{T}]$ noethersch (Hilbertscher Basissatz) ist, reichen
stets endlich viele $f_{1},\ldots,f_{n}\in M$:
\[
V(M)=V(f_{1},\ldots,f_{n})\qquad\text{falls }\mathfrak{a}=\langle f_{1},\ldots,f_{n}\rangle_{k[\underline{T}]}.
\]
\item $V(-)$ ist \textbf{inklusionsumkehrend}, $M'\subseteq M\implies V(M)\subseteq V(M')$.
\end{itemize}
\begin{prop}
\label{propdef:zariski-topologie}
Die Mengen $V(\mathfrak{a})$, $\mathfrak{a} \unlhd k[\underline{T}]$
ein Ideal, sind die \textbf{abgeschlossenen} Mengen einer Topologie
auf $k^{n}$, der sogenannten \textbf{Zariski-Topologie}\index{Zariski-Topologie}.
\begin{enumerate}
\item $\emptyset=V\left((1)\right)$, $k^{n}=V(0)$. 
\item $\bigcap_{i\in I}V(\mathfrak{a}_{i})=V\left(\sum_{i\in I}\mathfrak{a}_{i}\right)$
f�r beliebige Familien $(\mathfrak{a}_{i})_{i \in I}$ von Idealen.
\item $V(\mathfrak{a})\cup V(\mathfrak{a})=V(\mathfrak{ab})$ f�r $\mathfrak{a},\mathfrak{b}\unlhd k[\underline{T}]$
Ideale.
\end{enumerate}
\end{prop}
\begin{proof}
�bung / Algebra II. 

\-
\end{proof}




\section{Affine algebraische Mengen}
\label{sec:algebraische-mengen}
\begin{defn}
  \label{def:algebraische-mengen}
  \mbox{}
  \begin{itemize}
  \item $\mathbb{A}^{n}(k)$, der $\textbf{affine Raum der Dimension n}$ (�ber $k$),
    bezeichne $k^{n}$ mit der Zariski-Topologie.
  \item Abgeschlossene Teilmengen von $\mathbb{A}^{n}(k)$ hei�en affine abgeschlossene
    Mengen.
  \end{itemize}
\end{defn}
\begin{example}
  \label{bsp:algebraische-mengen-dim1}
  Da $k[T]$ ein Hauptidealring ist, sind die abgeschlossen Mengen in
  $\mathbb{A}^{1}(k)$: $\emptyset$, $\mathbb{A}^{1}$, Mengen der
  Form $V(f)$, $f\in k[T]\backslash\{k\}$ (endliche Teilmengen).%
  \begin{comment}
    F�r $f\in k$ ist $V(f)=\mathbb{A}^{1}$, denn die Einheiten im Polynomring
    $k[T]$ sind gegeben durch $k^{\times}$, und Ideale erzeugt von einer
    Einheit bilden den ganzen Ring. (siehe Algebra 1)
  \end{comment}
  {} Insbesondere sieht man, dass die Zariski-Topologie im Allgemeinen
  nicht Hausdorff ist. 
\end{example}
% 
\begin{example}
  \label{bsp:algebraische-mengen-dim2}
  $\mathbb{A}^{2}(k)$ hat zumindestens als abgeschlossene Mengen:
  \begin{itemize}
  \item $\emptyset$, $\mathbb{A}^{2}$;
  \item Einpunktige Mengen: $\{(x_{1},x_{2})\}=V(T_{1}-x_{1},T_{2}-x_{2})$;
  \item $V(f)$, $f\in k[T_{1},T_{2}]$ irreduzibel. 
  \end{itemize}
  Ferner alle endlichen Vereinigungen dieser Liste. (Dies sind in der
  Tat alle, denn sp�ter sehen wir: ``irreduzible'' abgeschlossene
  Mengen entsprechen den \emph{Primidealen}, und $k[T_{1},T_{2}]$ hat
  ``Krull-Dimension $2$''.)
\end{example}




\section{Der Hilbertsche Nullstellensatz}
\label{sec:nullstellensatz}
\begin{prop}
\label{prop:nullstellensatz}
Sei $K$ ein (nicht notwendigerweise algebraisch abgeschlossener) K�rper,
und $A$ eine endlich erzeugte $K$-Algebra. Dann ist $A$ Jacobson'sch,
d.h. f�r jedes Primideal $\mathfrak{p}\unlhd A$ gilt:
\[
\mathfrak{p}=\bigcap_{\mathfrak{m}\supseteq\mathfrak{p}}\mathfrak{m},\quad\mathfrak{m}\text{ maximales Ideal}
\]

Ist $\mathfrak{m}\unlhd A$ ein maximales Ideal, so ist die K�rpererweiterung
$K\subseteq A/\mathfrak{m}$ endlich.
\end{prop}
\begin{proof}
Algebra II / kommutative Algebra.
\end{proof}
\begin{cor}
\label{cor:nullstellensatz}
\mbox{}
\begin{enumerate}
\item Sei $A$ eine e.e. (endlich erzeugte) $k$-Algebra ($k$ sei algebraisch
abgeschlossen), $\mathfrak{m}\unlhd A$ ein maximales Ideal. Dann
ist $A/\mathfrak{m}=k$. 
\item Jedes maximale Ideal $\mathfrak{m}\unlhd k[\underline{T}]$ ist von der
Form $\mathfrak{m}=(T_{1}-x_{1},\ldots,T_{n}-x_{n})$ mit $x_{1},\ldots,x_{n}\in k$.
\item F�r ein Ideal  $\mathfrak{a}\unlhd k[\underline{T}]$ gilt:
\[
\rad(\mathfrak{a})=\sqrt{\mathfrak{a}}\overset{(i)}{=}\bigcap_{\mathfrak{a}\subseteq\mathfrak{p}\unlhd k[\underline{T}], \mathfrak{p} \text{prim}}\mathfrak{p}\overset{(ii)}{=}\bigcap_{\mathfrak{a}\subseteq\mathfrak{m}\unlhd k[\underline{T}], \mathfrak{m} \text{maximal}}\mathfrak{m}
\]
\end{enumerate}
\end{cor}
\begin{proof}
\mbox{}
\begin{enumerate}
\item $k\rightarrow A\rightarrow A/\mathfrak{m}$ ist Isomorphismus,  da
$k$ keine echte algebraische K�rpererweiterung besitzt.
\item Es ist
\begin{align*}
k[T_{1},\ldots,T_{n}] & \twoheadrightarrow k[\underline{T}]/\mathfrak{m}=k\\
T_{i} & \mapsto x_{i}
\end{align*}
surjektiv. Es folgt: $\mathfrak{m}=(T_{1}-x_{1},\ldots,T_{n}-x_{n})$, da letzteres
bereits maximal ist. ($\supseteq$ klar.)
\item (i) Algebra II. (ii) Theorem.
\end{enumerate}
\end{proof}




\section{Korrespondenz zwischen Radikalidealen und affinen algebraischen Mengen}

Sei $V(\mathfrak{A})\subseteq\mathbb{A}^{n}(k)$ affin algebraische
Menge, $\mathfrak{A}\subset k[\underline{T}]$.\textbf{ Es gilt:}
\[
V(\mathfrak{A})=V(\rad\mathfrak{A})
\]

mit $\rad\mathfrak{A}=\{f\in k[\underline{T}]\mid f^{n}\in\mathfrak{A}\text{ f�r }n>0\}$,
da
\[
f^{n}(x)=0\Leftrightarrow f(x)=0,
\]

d.h. verschiedene Ideale k�nnen dieselbe algebraische Menge beschreiben.
\begin{defn}
F�r eine Teilmenge $Z\subseteq\mathbb{A}^{n}(k)$ bezeichne
\[
I(Z)=\{f\in k[\underline{T}]\mid f(x)=0\ \forall x\in Z\}
\]

das Ideal aller auf $Z$ verschwindenden Polynomfunktionen.
\end{defn}
\begin{prop}
\mbox{}
\begin{enumerate}
\item Sei $\mathfrak{A}\subset k[\underline{T}]$ Ideal. Dann ist $I(V(\mathfrak{A}))=\rad(\mathfrak{A})$.
\item Sei $Z\subseteq\mathbb{A}^{n}(k)$ Teilmenge. Dann ist $V(I(Z))=\overline{Z}$,
der Abschluss von $Z$.
\end{enumerate}
\end{prop}
\begin{proof}
�bungsblatt 2.
\end{proof}
\medskip{}

$\mathfrak{A}$ hei�t \textbf{Radikalideal}\index{Radikalideal},
wenn $\mathfrak{A}=\rad(\mathfrak{A})$, oder �quivalent wenn $k[\underline{T}]/\mathfrak{A}$
\emph{reduziert} ist, d.h. keine nilpotente Elemente hat.
\begin{cor}
Wir erhalten eine 1-1 Korrespondenz
\begin{align*}
\{\text{abg. Mengen }\subseteq\mathbb{A}^{n}\} & \leftrightarrow\{\text{Radikalideale }\mathfrak{A}\subset k[\underline{T}]\}\\
Z & \mapsto I(Z)\\
V(\mathfrak{A}) & \mapsfrom\mathfrak{A}
\end{align*}

die sich zu einer 1-1 Korrespondenz
\begin{align*}
\left\{ \text{Punkte in }\mathbb{A}^{n}\right\}  & \leftrightarrow\left\{ \text{max. Ideale in }k[\underline{T}]\right\} \\
x=(x_{1},\ldots,x_{n}) & \mapsto\begin{array}{rl}
\mathfrak{m}_{x} & =I(\{x\})\\
 & =\ker(k[\underline{T}]\rightarrow k,\ T_{i}\mapsto x_{i})
\end{array}
\end{align*}

einschr�nkt.
\end{cor}



\input{AlgGeo1-Part1-6_Irreduzibele-topologische-R�ume.tex}


\section{Irreduzibele affine algebraische Mengen}

\subsection{Lemma 16}

Eine abgeschlossene Teilmenge $Z\subseteq\mathbb{A}^{n}(k)$ ist genau
dann irreduzibel, wenn $I(Z)$ ein Primideal ist. Insbesondere ist
$\mathbb{A}^{n}$ irreduzibel.  

\subsubsection{Beweis (Lemma 16)}

$Z$ irreduzibel $\Leftrightarrow(Z=\underbrace{V(\mathfrak{A})}_{\bigcap V(f_{i})}\cup\underbrace{V(\mathfrak{b})}_{\bigcap V(g_{j})}\Rightarrow V(\mathfrak{A})=Z$
oder $V(\mathfrak{b})=Z$) 

$\Leftrightarrow\forall f,g\in k[\underline{T}]$ ist $V(fg)=V(f)\cup V(g)\supseteq Z$
gilt $V(f)\supset Z$ oder $V(g)\supseteq Z$.

$\stackrel[I(V(\mathfrak{A}))=\rad(\mathfrak{A})]{V(I(Z))=Z}{\Leftrightarrow}\forall f,g\in k[\underline{T}]$
ist $fg\in I(V(fg)\subseteq I(Z)$ gilt $f\in I(Z)$ oder $g\in I(Z)$.

$\Leftrightarrow I(Z)$ ist Primideal.

\subsection{Bemerkung 17}

Die Korrespondenz aus Korollar 11 schr�nkt sich ein zu
\[
\{\text{irred. abg. Teilmengen }\subseteq\mathbb{A}^{n}\}\overset{1:1}{\leftrightarrow}\{\text{Primideale in }k[\underline{T}]\}
\]



\input{AlgGeo1-Part1-8_Quasikompakte-und-noethersche-topologische-R�ume.tex}

\selectlanguage{english}%

\section{Morphismen von affinen algebraischen Mengen}
\begin{defn}
Seien $X\subseteq\mathbb{A}^{m}(k)$, $Y\subseteq\mathbb{A}^{n}(k)$
affine algebraische Mengen. Ein \textbf{Morphismus} $X\rightarrow Y$
affiner algebraischer Mengen ist eine Abbildung $f:X\rightarrow Y$
der zugrundeliegenden Mengen, sodass $f_{1},\ldots,f_{n}\in k[T_{1},\ldots,T_{m}]$
existieren, derart dass $\forall x\in X$ gilt:
\[
f(x)=(f_{1}(x),\ldots,f_{n}(x)).
\]
\emph{Bezeichne }daf�r $\hom(X,Y)$ Menge der Morphismen $X\rightarrow Y$. 
\end{defn}
\begin{rem}
$f:X\rightarrow Y$ l�sst sich immer fortsetzen zu einem Morphismus
\[
f:\mathbb{A}^{n}(k)\rightarrow\mathbb{A}^{m}(k),
\]

aber nicht eindeutig, es sei denn $X=\mathbb{A}^{m}(k)$.
\end{rem}

\paragraph{Komposition}

\[
\xymatrix@C=9pc{X\ar[r]_{f}^{f_{1},\ldots,f_{n}\in k[T_{1},\ldots,T_{m}]} & Y\ar[r]_{g}^{g_{1},\ldots,g_{r}\in k[T_{1}',\ldots,T_{m}']} & Z}
\]

mit $X\subseteq\mathbb{A}^{m}(k)$, $Y\subseteq\mathbb{A}^{n}(k)$,
$Z\subseteq\mathbb{A}^{r}(k)$. Es folgt:
\begin{align*}
g(f(x))=\, & (g_{1}(f_{1}(x),\ldots,f_{n}(x)),\ldots,g_{r}(f_{1}(x),\ldots,f_{n}(x))\\
:=\, & h_{1}(x),\ldots,h_{r}(x)
\end{align*}

d.h. $g\circ f$ ist durch Polynome $h_{i}\in k[T_{1},\ldots,T_{m}]$
gegeben, d.h. $g\circ f$ ist wieder ein Morphismus affiner algebrasischer
Mengen. Wir erhalten die \textbf{Kategorie affiner algebraischer Mengen}.
\begin{example}
\mbox{}
\begin{enumerate}
\item Sei die Abbildung
\begin{align*}
\mathbb{A}^{1}(k) & \rightarrow V(T_{2}-T_{1}^{2})\subseteq\mathbb{A}^{2}(k)\\
x & \mapsto(x,x^{2}).
\end{align*}
Diese Abbildung ist sogar ein \emph{Isomorphismus }affiner algebraischer
Mengen, da die Umkehrabbildung
\[
(x,y)\mapsto x
\]
ebenfalls ein Morphismus ist.
\item Sei char$(k)\neq2$. Die Abbildung
\begin{align*}
\mathbb{A}^{1}(k) & \rightarrow V(T_{2}^{2}-T_{1}^{2}(T_{1}+1))\\
x & \mapsto(x^{2}-1,x(x^{2}-1))
\end{align*}
ist ein Morphismus, aber \emph{nicht }bijektiv, da $1,-1$ beide auf
$(0,0)$ abgebildet werden.\selectlanguage{ngerman}%
\end{enumerate}
\end{example}



\input{AlgGeo1-Part1-10_Unzul�nglichkeiten.tex}

\selectlanguage{english}%

\paragraph*{Affine algebraische Mengen als R�ume von Funktionen}

\section{Der affine Koordinatensatz}

Sei $X\subseteq\mathbb{A}^{n}(k)$ abgeschlossen. F�r den surjektiven
(Def. von Morphismen) $k$-Algebren-Homomorphismus
\begin{align*}
k[I] & \xrightarrow{\varphi}\hom(X,\mathbb{A}^{1}(k))\\
f & \mapsto(x\mapsto f(x)),
\end{align*}

wobei die Morphismen in folgende Weise eine $k$-Algebra bilden:
\begin{align*}
(f+g)(x) & :=f(x)+g(x)\\
(fg)(x) & :=f(x)g(x)\\
(\alpha f)(x) & :=\alpha f(x)
\end{align*}

mit $f,g\in\hom(X,\mathbb{A}^{1}(k))$, $\alpha\in k$. Es gilt:
\[
\ker\varphi=I(X)
\]


\subsection*{Definition 26}

$\Gamma(X):=k[I]/I(X)\cong\hom(X,\mathbb{A}^{1}(k))$ hei�t der \textbf{affine
Koordinatenring }von $X$.

F�r $x=(x_{1},\ldots,x_{n})\in X$ gilt:
\begin{align*}
\mathfrak{m}_{x}:=\  & \ker(\Gamma(X)\twoheadrightarrow k,\,f\mapsto f(x))\\
=\  & \{f\in\Gamma(X)\mid f(x)=0\}\\
=\  & \text{Bild von }(T_{1}-x_{1},\ldots,T_{n}-x_{n})\\
=\  & \ker(\Gamma(\mathbb{A}^{n}(k))\twoheadrightarrow k)
\end{align*}

unter der Projektion $\pi:k[\underline{T}]=\Gamma(\mathbb{A}^{n}(k))\twoheadrightarrow\Gamma(X)$.
Es ist $\mathfrak{m}_{x}$ ein maximales Ideal von $\Gamma(X)$ mit
$\Gamma(X)/\mathfrak{m}_{x}=k$.

F�r ein Ideal $\mathfrak{A}\subset\Gamma(X)$ setze
\[
V(\mathfrak{A})=\{x\in X\mid f(x)=0\ \forall f\in\mathfrak{A}\}=V(\pi^{-1}(\mathfrak{A}))\cap X.
\]

Dies sind genau die abgeschlossenen Mengen von $X$ als Teilraum in
$\mathbb{A}^{n}(k)$ mit der induzierten Topologie, diese wird auch
\textbf{Zariski-Topologie} genannt.

F�r $f\in\Gamma(X)$ setze:
\[
D(f)=\{x\in X\mid f(x)\neq0\}=X\backslash V(f).
\]


\subsection*{Lemma 27}

Die offenen Mengen $D(f)$, $f\in\Gamma(X)$, bilden eine Basis der
Topologie, d.h.
\[
\forall U\subset X\text{ offen }\exists f_{i}\in\Gamma(X),\,i\in I,\,\text{mit }U=\bigcup_{i\in I}D(f_{i})
\]


\subsubsection*{Beweis (Lemma 27)}

$U=X\backslash V(\mathfrak{A})$ f�r ein $\mathfrak{A}\subset\Gamma(X)$,
$\mathfrak{A}=\langle f_{1},\ldots,f_{n}\rangle$ . Wegen
\[
V(\mathfrak{A})=\bigcap_{i=1}^{n}V(f_{i})
\]

folgt
\[
U=\bigcup_{i=1}^{n}D(f_{i})
\]

Es reichen also sogar endlich viele $f_{i}$! 

\subsection*{Satz 28}

Der Koordinatenring $\Gamma(X)$ einer affinen algebraischen Menge
$X$ ist eine endlich erzeugte $k$-Algebra, die reduziert ist (d.h.
keine nilpotenten Elemente $\neq0$ enth�lt). Ferner:
\[
X\text{ irreduzibel}\Leftrightarrow\Gamma(X)\text{ ist integer}
\]


\subsubsection*{Beweis (Satz 28)}

$k[\underline{T}]\twoheadrightarrow\Gamma(X)$ impliziert ``endlich
erzeugte $k$-Algebra''. $\Gamma(X)$ irreduzibel ist �quivalent
dazu, dass $I(X)=\text{rad }I(X)$. 

(Satz 10. ii) + Korollar 11: $X=V(\mathfrak{A})$: $I(X)=\text{rad }\mathfrak{A}=I(X)$

$\Rightarrow\text{rad }I(X)=\text{rad rad }\mathfrak{A}=\text{rad }\mathfrak{A}=I(X)$).

$X$ irreduzibel $\overset{Lem.17}{\Leftrightarrow}I(X)$ Primideal
$\Leftrightarrow\Gamma(X)=k[T]/T(X)$ integer. \selectlanguage{ngerman}%




\section{Funktorielle Eigenschaften von $\Gamma(X)$}
\label{sec:koordinatenring-funktiorialitaet}
\begin{prop}
  \label{prop:koordinatenringfunktor}
  Für einen Morphismus $X\xrightarrow{f}Y$ affiner algebraischer Mengen
  definiert 
  \begin{align*}
    \Gamma(f):\quad\Gamma(Y) & \rightarrow\Gamma(X)\\
    g & \mapsto g\circ f
  \end{align*}
  ein Homomorphismus von $k$-Algebren. Der so definierte \emph{kontravariante}
  Funktor
  \[
    \Gamma:\{\text{affine algebraische Mengen}\}\rightarrow\{\text{reduzierte endl. erz. }k\text{-Algebren}\}
  \]
  liefert eine Kategorienäquivalenz, welche durch Einschränkung eine Äquivalenz
  \[
    \Gamma:\{\text{irred. aff. alg. Mengem\}}\rightarrow\{\text{integre endl. erz. }k\text{-Algebren\}}
  \]
  induziert.
\end{prop}
\begin{proof}
  Sei $Y\xrightarrow{g}\mathbb{A}^{1}(k)\in\Gamma(Y)$ ein Morphismus. Es
  folgt:
  \[
    g\circ f:X\xrightarrow{f}Y\xrightarrow{g}\mathbb{A}^{1}(k)
  \] 
  ist Morphismus,  d.h. $g \circ f\in\Gamma(X)$. $\Gamma(f):\Gamma(Y)\rightarrow\Gamma(X)$
  ist ein $k$-Algebren-Homomorphismus mit $\Gamma(\text{id}_{X})=\text{id}_{\Gamma(X)}$. Da ferner gilt, dass $\Gamma(f_{1}\circ f_{2})=\Gamma(f_{2})\circ\Gamma(f_{1})$ ist $\Gamma$ ein kontravarianter Funktor.
  \begin{claim*}
    $\Gamma$ ist volltreu, d.h.
    \begin{align*}
      \Gamma:\hom(X,Y) & \rightarrow\hom_{k\text{-Alg}}(\Gamma(Y),\Gamma(X))\\
      f & \mapsto\Gamma(f)
    \end{align*}
    ist \emph{bijektiv} für alle affinen algebraischen Mengen $X,Y$.
  \end{claim*}
  \begin{proof}
    Wir konstruieren eine Umkehrabbildung wie folgt: Zu $\varphi:\Gamma(Y)\rightarrow\Gamma(X)$
    für $X\subseteq\mathbb{A}^{m}(k)$, $Y\subseteq\mathbb{A}^{n}(k)$ existiert ein Lift $\tilde\varphi$, s.d.
    \[
      \xymatrix{k[T_{1}',\ldots,T_{n}']\ar[r]^{\tilde{\varphi}}\ar@{->>}[d] & k[T_{1},\ldots,T_{m}]\ar@{->>}[d]\\
        \Gamma(Y)\ar[r]^{\varphi} & \Gamma(X)
      }
    \]
    kommutiert; $\tilde{\varphi}(T_{i}'):= f_i$ mit $f_i \in \pi^{-1}(\varphi(T_{i}')) \subseteq k[T_1,...,T_n]$, wobei $\pi : k[\underline{T}] \to \Gamma(X)$ die kanonische Projektion bezeichne. 
    Definiere:
    \begin{align*}
      f:X & \rightarrow Y\\
      x=(x_{1},\ldots,x_{n}) & \mapsto(\tilde{\varphi}(T_{1}')(x_{1},\ldots,x_{n}),\ldots,\tilde{\varphi}(T_{n}')(x_{1},\ldots,x_{n}))
    \end{align*}
  \end{proof}
  \begin{claim*}
    $\Gamma$ ist essentiell surjektiv, d.h. zu jeder reduzierten endlich
    erzeugten $k$-Algebra $A$ existiert eine affine algebraische Menge
    $X$ mit $A\cong\Gamma(X)$.
  \end{claim*}
  \begin{proof}
    Da nach Voraussetzung $A\cong k[T]/\mathfrak{a}$ für ein Radikalideal
    $\mathfrak{a}$, können wir etwa $X:=V(\mathfrak{a})\subseteq\mathbb{A}^{n}(k)$
    setzen. Der Rest folgt aus Satz 28.
  \end{proof}
\end{proof}
\begin{prop}
  \label{prop:funktiorialitaet-specm}
  Sei $f:X\rightarrow Y$ ein Morphismus affiner algebraischer Mengen und $\Gamma(f):\Gamma(Y)\rightarrow\Gamma(X)$
  der zugehörige Homomorphismus der Koordinatenringe. Dann gilt $\forall x\in X$:
  $\Gamma(f)^{-1}(\mathfrak{m}_{x})=\mathfrak{m}_{f(x)}$.
\end{prop}
\begin{proof}
  \[
    \Gamma(f)^{-1}(\mathfrak{m}_{x})=\{g\in\Gamma(Y)\mid g\circ f \in \mathfrak{m}_{x}\}=\{g\in\Gamma(Y)\mid g(f(x)) = 0 \} = \mathfrak{m}_{f(x)},
  \]
  da $\Gamma(f)(g) =g \circ f$.
\end{proof}


\input{AlgGeo1-Part1-13_R�ume-mit-Funktionen.tex}


\section{Der Raum mit Funktionen zu einer affin algebraischen Menge}

\textbf{Ziel.} $X\subseteq\mathbb{A}^{n}(k)\mapsto(X,\mathcal{O}_{X})$
als irreduzibele affine algebraische Menge bzw. Zariski-Topologie.
D.h. wir m�ssen Mengen von Funktionen $\mathcal{O}_{X}(U)$ auf $U$,
$U\subset X$ offen, definieren. Diese werden als Teilmengen des Funktionenk�rpers
$K(X)$ definiert (dazu $X$ irreduzibel, sp�ter bei Schemata f�llt
diese Bedingung weg!)
\begin{defn}
$K(X):=\text{Quot}(\Gamma(X))$ hei�t \textbf{Funktionenk�rper} von
$X$. ($\Gamma(X)$ ist f�r $X$ irreduzibel nullteilerfrei.)

Elemente $\frac{f}{g}\in K(X)$, $f,g\in\Gamma(X)=\hom(X,\mathbb{A}^{1}(k))$,
$g\neq0$ lassen sich zumindest als Funktion auf der offenen Menge
$\mathcal{D}(g)\subset X$ auffassen, wenn auch nicht i.A. auf ganz
$X$.
\end{defn}
\begin{lem}
Gilt f�r $\frac{f_{1}}{g_{1}},\frac{f_{2}}{g_{2}}\in K(X)$, $f_{i},g_{i}\in\Gamma(X)$,
und einer offenen Teilmenge $\emptyset\neq U\subset\mathcal{D}(g_{1}g_{2})$
\[
\frac{f_{1}(x)}{g_{1}(x)}=\frac{f_{2}(x)}{g_{2}(x)}\qquad\forall x\in U,
\]

dann folgt $\frac{f_{1}}{g_{1}}=\frac{f_{2}}{g_{2}}$ in $K(X)$.
\end{lem}
\begin{proof}
Sei ohne Einschr�nkung der Allgemeinheit $g_{1}=g_{2}=g$. (Sonst
Erweitern!) 

$\Rightarrow(f_{1}-f_{2})(x)=0$ $\forall x\in U$.

$\Rightarrow\emptyset\neq U\subset V(f_{1}-f_{2})\subset X$ dicht,
d.h. $V(f_{1}-f_{2})=X$.

$\phantom{\Rightarrow\ }$$f_{1}-f_{2}\in IV(f_{1}-f_{2})=I(X)\equiv(0)$
in $\Gamma(X)$ 

$\Rightarrow f_{1}-f_{2}=0$.
\end{proof}
\begin{defn}
Sei $X$ eine irreduzibele affine algebraische Menge, $U\subset X$
offen. Sei $\Gamma(X)_{\mathfrak{m}_{x}}$ Lokalisierung von $\Gamma(X)$
bzgl. das maximale Ideal $\mathfrak{m}_{x}$ in $x\in X$.

\[
\mathcal{O}_{X}(U):=\bigcap_{x\in U}\Gamma(X)_{\mathfrak{m}_{x}}\subset K(X)
\]

d.h. f�r jedes $x\in U$ l�sst sich $f\in\mathcal{O}_{X}(U)$ schreiben
als $\frac{h}{g}$ mit $g(x)\neq0$.
\end{defn}
Wenn $f\in\Gamma(X)$ bezeichne $\Gamma(X)_{f}$ die Lokalisierung
von $\Gamma(X)$ bzgl. der multiplikativ abgeschlossenen Teilmenge
$\{1,f,f^{2},\ldots,f^{n}\ldots\}$. Dann l�sst sich
\[
\Gamma(X)_{\mathfrak{m}_{x}}=\bigcup_{f\in\Gamma(X)\backslash\mathfrak{m}_{x}}\Gamma(X)_{f}\subset K(X)
\]

schreiben. ``$\supset$'' klar, ``$\subset$'' $\frac{g}{f}$
mit $f(x)\neq0$ d.h. $f\notin\mathfrak{m}_{x}$ $\Rightarrow\frac{g}{f}\in\Gamma(X)_{f}$.


\paragraph{Es gilt:}
\begin{enumerate}
\item F�r $V\subset U\subset X$ offen kommutiert das folgende Diagramm:
\[
\xymatrix{\mathcal{O}_{X}(V)\ar@{^{(}->}[r] & \text{Abb}(V,k)\\
\mathcal{O}_{X}(U)\ar@{^{(}->}[r]\ar@{^{(}->}[u] & \text{Abb}(U,k)\ar[u]_{\text{Einschr�nkungsabb.}}
}
\]
mit $\mathcal{O}_{X}(U)\subset\mathcal{O}_{X}(V)$ nach Definition.
\item $\mathcal{O}_{X}(U)\rightarrow\text{Abb}(U,k)$, $f\mapsto(x\mapsto f(x):=\frac{g(x)}{f(x)}\in k)$
ist injektiv (Lemma 34) und wohldefiniert (k�rzen/Erweitern), wobei
$g,h\in\Gamma(X)$ mit $h\notin\mathfrak{m}_{x}$ mit $f=\frac{g}{h}$
nach Definition von $\mathcal{O}_{X}(U)$ existiert.
\item \textbf{Verklebungseigenschaft.} Sei $U=\bigcup_{i\in I}U_{i}$. Nach
Definition ist 
\begin{align*}
\mathcal{O}_{X}(U) & =\bigcap_{i}\mathcal{O}_{X}(U_{i})\subset K(X)\\
\ni f:U\rightarrow k & \quad\ni f_{i}:U_{i}\rightarrow k
\end{align*}
{[}Diagramm fehlt{]}. $\Rightarrow(X,\mathcal{O}_{X})$ ist Raum
mit Funktionen, \textbf{der zur irreduziblen affin algebraische Menge
geh�rige Raum von Funktionen.} 
\end{enumerate}
\begin{prop}[orig. 33]
F�r $(X,\mathcal{O}_{X})$ zu $X$ wie oben und $f\in\Gamma(X)$
gilt:
\[
\mathcal{O}_{X}(D(f))=\Gamma(X)_{f},
\]

insbesondere $\mathcal{O}_{X}(X)=\Gamma(X)$.
\end{prop}
\begin{proof}
$\Gamma(X)\subset\mathcal{D}(f)$ klar, da $f(x)\neq0$ $\forall x\in\mathcal{D}(f)$
bzw. $f\in P(X)\backslash\mathfrak{m}_{x}$. 

Sei nun $g$ in $\mathcal{O}_{X}(\mathcal{D}(f))$ gegeben, $(*)$
und $\mathfrak{A}:=\{h\in\Gamma(X)\mid hg\in\Gamma(X)\}\subset\Gamma(X)$
Ideal.

Dazu: $g\in\Gamma(X)_{g}$

$\Leftrightarrow g=\frac{k}{g^{n}}$ f�r ein $n$ und $k\in\Gamma(X)$

$\Leftrightarrow f^{n}\in\mathfrak{A}$ f�r ein $n$.

d.h. zu zeigen: $f\in\text{rad}(\mathfrak{A})=IV(\mathfrak{A})$ (Hilbertsche
Nullstellensatz)

$\Leftrightarrow f(x)=0$ $\forall x\in V(\mathfrak{A})$

Ist dazu $x\in X$ mit $f(x)\neq0$, wo $x\in\mathcal{D}(f)$, so
existiert nach Voraussetzung $(*)$ $f_{1},f_{2}\in\Gamma(X)$, $f_{2}\notin\mathfrak{m}_{x}$
mit $g=\frac{f_{1}}{f_{2}}$

$\Rightarrow f_{2}\in\mathfrak{A}$. Da $f_{2}(x)\neq0$:

$\Rightarrow x\notin V(\mathfrak{A})$.
\end{proof}
\begin{rem}[orig. 34]
\mbox{}
\begin{enumerate}
\item Im allgemeinen existieren f�r $f\in\mathcal{O}_{x}(U)$ \textbf{nicht}
$g,h\in\Gamma(X)$ mit $f=\frac{g}{h}$ und $h(x)\neq0$ $\forall x\in U$.
\item \textbf{Alternative Definition von $\mathcal{O}_{X}$, I.}
\[
\mathcal{O}_{X}(\mathcal{D}(f)):=\Gamma(X)_{f},\quad\forall f\in\Gamma(X).
\]
Da $\mathcal{D}(f)$ Basis der Topologie ist, kann es h�chstens einen
Raum mit Funktionen geben mit dieser Eigenschaft, es bleibt die Existenz
zu zeigen.
\item \textbf{Alternative Definition von $\mathcal{O}_{X}$, II.}

Direkt von einer integeren endlich erzeugten $k$-Algebra $A$ ausgehend
(die $X$ bis auf Isomorphie festlegt), aber ohne ``Koordinaten''
zu w�hlen.
\begin{align*}
X & :=\{\mathfrak{m}\subseteq A\mid\text{max. Ideale}\}
\end{align*}
Die \textbf{abgeschlossen Mengen} sind gegeben durch:

\[
V(\mathfrak{A}):=\{\mathfrak{m}\subseteq A\text{ max.}\mid\mathfrak{m}\supseteq\mathfrak{A}\},\quad\mathfrak{A}\subset A\text{ Ideal}.
\]

$\mathcal{O}_{X}(U):=\bigcap_{\mathfrak{m}\in U}A_{\mathfrak{m}}\subset\text{Quot}(A)$
f�r $U\subset X$ offen (vgl. sp�ter Schemata).
\end{enumerate}
\end{rem}



\input{AlgGeo1-Part1-15_Funktorialit�t-der-Konstruktion.tex}

\input{AlgGeo1-Part1-16_Definition-von-Pr�variet�ten.tex}

\newpage{}

\printindex{}

\end{document}
