\documentclass[12pt,a4paper]{book}
\usepackage[T1]{fontenc}
\usepackage[utf8]{inputenc}
\usepackage{geometry}
\geometry{verbose,tmargin=2cm,bmargin=2cm,lmargin=2cm,rmargin=2cm}
\pagestyle{headings}
\usepackage[ngerman]{babel}
\usepackage{verbatim}
\usepackage{amsmath}
\usepackage{amsthm}
\usepackage{amssymb}
\usepackage{stmaryrd}
\usepackage{makeidx}
\makeindex
\usepackage{setspace}
\usepackage[all]{xy}
\onehalfspacing
\usepackage[bookmarks=true]{hyperref}

%% Theorems (numbered by Part)
\newtheorem{thm}{Theorem}[chapter]
\theoremstyle{definition}
\newtheorem{example}[thm]{Beispiel}
\theoremstyle{definition}
\newtheorem{defn}[thm]{Definition}
\theoremstyle{plain}
\newtheorem{prop}[thm]{Satz}
\theoremstyle{plain}
\newtheorem{cor}[thm]{Korollar}
\theoremstyle{plain}
\newtheorem{lem}[thm]{Lemma}
\theoremstyle{remark}
\newtheorem{rem}[thm]{Bemerkung}
\theoremstyle{plain}

%% Theorems (unnumbered)
\newtheorem*{question*}{Frage}
\theoremstyle{remark}
\newtheorem*{claim*}{Behauptung}
\theoremstyle{definition}
\newtheorem*{example*}{Beispiel}
\theoremstyle{plain}
\newtheorem*{rem*}{Bemerkung}
\theoremstyle{remark}

%% User-specified commands
\DeclareMathOperator{\rad}{rad}
\DeclareMathOperator{\Spec}{Spec}
\DeclareMathOperator{\Quot}{Quot}
\DeclareMathOperator{\im}{\mathrm{im}}
\DeclareMathOperator{\Hom}{\mathrm{Hom}}
\DeclareMathOperator{\Mor}{\mathrm{Mor}}
\DeclareMathOperator{\id}{\mathrm{id}}

%%sets
\DeclareMathOperator{\CC}{\mathbb{C}}
\DeclareMathOperator{\RR}{\mathbb{R}}
\DeclareMathOperator{\QQ}{\mathbb{Q}}
\DeclareMathOperator{\ZZ}{\mathbb{Z}}
\DeclareMathOperator{\NN}{\mathbb{N}}

%% categories
\DeclareMathOperator{\ouv}{\mathcal{O}uv}
\DeclareMathOperator{\set}{\underline{Set}}
\DeclareMathOperator{\ab}{\underline{Ab}}
\DeclareMathOperator{\cattop}{\underline{Top}}
\DeclareMathOperator{\cring}{\underline{CRing}}
\DeclareMathOperator{\psh}{\underline{\mathcal{PS}h}}
\DeclareMathOperator{\sh}{\underline{\mathcal{S}h}}

%% commands
\newcommand{\cat}[1]{\mathcal{#1}}
\newcommand{\sheaf}[1]{\mathcal{#1}}
\renewcommand{\labelenumi}{(\roman{enumi})}
\renewcommand{\labelenumii}{\arabic{enumii}.}

\begin{document}

%% Title page
\title{Algebraische Geometrie I}
\author{Prof. Dr. Venjakob}
\maketitle

\tableofcontents{}
\newpage{}

\section*{Literatur}
\begin{itemize}
\item Görtz, Wedhorn. \emph{Algebraic Geometry I}
\item Hartshorne. \emph{Algebraic Geometry}
\item Shafarevich. \emph{Basic Algebraic Geometry 1 \& 2}
\item Grothendieck. \emph{Eléments de géometrie algébrique, EGA I-IV}
\end{itemize}

\paragraph{Kommutative Algebra}
\begin{itemize}
\item Brüske, Ischebeck, Vogel. \emph{Kommutative Algebra}
\item Kunz. \emph{Einführung in die kommutative Algebra und algebraische Geometrie}
\end{itemize}

\chapter{Prä-Varietäten}
\label{chap:prae-varietaeten}

\include{Chapter1/AlgGeo1-Chapter1-1_Einfuehrung}

\include{Chapter1/AlgGeo1-Chapter1-2_Die-Zariski-Topologie}

\include{Chapter1/AlgGeo1-Chapter1-3_Affine-algebraische-Mengen}

\include{Chapter1/AlgGeo1-Chapter1-4_Der-Hilbertsche-Nullstellensatz}

\include{Chapter1/AlgGeo1-Chapter1-5_Korrespondenz-zwischen-Radikalidealen}


\section{Irreduzible topologische Räume}
\label{sec:irreduzibilitaet-top}

Die folgenden topologischen Begriffe sind nur interessant, da $\mathbb{A}^{n}(k)$
($n>0$) kein Hausdorff'scher Raum ist.
\begin{defn}
  \label{def:irreduzibel}
  Ein topologischer Raum $X$ heißt \textbf{irreduzibel}\index{irreduzibel},
  falls $X\neq\emptyset$ und $X$ sich \emph{nicht} als Vereinigung
  zweier echter abgeschlossener Teilmengen darstellen lässt, d.h
  \[
    X=A_{1}\cup A_{2},\ A_{i}\ \text{abg.}\quad\implies\quad A_{1}=X\text{ oder }A_{2}=X.
  \]

  $Z\subseteq X$ heißt irreduzibel, falls $Z$ mit der induzierten Topologie
  irreduzibel ist.
\end{defn}
\begin{prop}
  \label{prop:charakterisierung-irreduzibel}
  Für einen topologischen Raum $X \neq \emptyset$ sind äquivalent:
  \begin{enumerate}
  \item $X$ ist irreduzibel.
  \item Je zwei nichtleere offene Teilmengen von $X$ haben nicht-leeren
    Durchschnitt.
  \item Jede nichtleere offene Teilmenge $U\subseteq X$ ist dicht in $X$.
  \item Jede nichtleere offene Teilmenge $U\subseteq X$ ist zusammenhängend.
  \item Jede nichtleere offene Teilmenge $U\subseteq X$ ist irreduzibel.
  \end{enumerate}
\end{prop}
\begin{proof}
  \mbox{}
  \begin{itemize}
  \item $(i)\Leftrightarrow(ii)$

    Komplementärmengen.
  \item $(ii)\Leftrightarrow(iii)$ 

    Es ist: $U\subseteq X$ dicht $\Leftrightarrow U\cap O\neq\emptyset$
    für jedes offene $\emptyset\neq O\subseteq X$.
  \item $(iii)\Rightarrow(iv)$

    Klar. 
  \item $(iv)\Rightarrow(iii)$

    Sei $\emptyset\neq U$ offen und zusammenhängend. Es folgt:
    \[
      U=U_{1}\sqcup U_{2},\qquad\emptyset\neq U_{i}\underset{\text{offen}}{\subseteq}U\underset{\text{offen}}{\subseteq}X
    \]
    Damit ist $U_{1}\cap U_{2}=\emptyset$, ein Widerspruch zu (iii).
  \item $(v)\Rightarrow(i)$ 

    Klar. $(U=X)$
  \item $(iii)\Rightarrow(v)$

    Sei $\emptyset\neq U\underset{\text{offen}}{\subseteq}X$. Ist $\emptyset\neq V\underset{\text{offen}}{\subseteq}U$,
    so ist $V\underset{\text{offen}}{\subseteq}X$. Es folgt: $V$ ist
    dicht in $X$ und irreduzibel in $U$. Mit $(iii)\Rightarrow(i)$
    folgt, dass $U$ irreduzibel ist. 

  \end{itemize}
\end{proof}
\begin{lem}
  \label{lem:irreduzibel-abschluss}
  Eine Teilmenge $Y$ ist genau dann irreduzibel, wenn ihr Abschluss $\overline{Y}$ dies ist.
\end{lem}
\begin{proof}
  $Y$ irreduzibel

  $\Leftrightarrow\forall U,V\subseteq X$ offen mit $U\cap Y\neq\emptyset\neq V\cap Y$,
  gilt $Y\cap(U\cap V)\neq\emptyset$.

  $\Leftrightarrow\overline{Y}$ irreduzibel 
\end{proof}
\begin{defn}
  \label{def:irreduzible-komponente}
  Eine maximale irreduzible Teilmenge eines topologischen Raumes $X$
  heißt \textbf{irreduzible Komponente}\index{irreduzible Komponente}
  von $X$.
\end{defn}
\begin{rem}
  \label{rem:irreduzibel}
  \mbox{}
  \begin{enumerate}
  \item Jede irreduzible Komponente ist abgeschlossen nach Lemma 14.
  \item $X$ ist Vereinigung seiner irreduziblen Komponenten, \emph{denn}: 

    die Menge der irreduziblen Teilmengen von $X$ ist \textbf{induktiv
      geordnet}: für jede aufsteigende Kette irreduzibler Teilmengen ist
    die Vereinigung wieder irreduzibel (Satz 13 (ii)). Mit dem \textbf{Lemma
      von Zorn} folgt: Jede irreduzible Teilmenge ist in einer irreduziblen
    Komponente enthalten. Damit ist jeder Punkt in einer irreduziblen
    Komponente enthalten.
  \end{enumerate}
\end{rem}



\include{Chapter1/AlgGeo1-Chapter1-7_Irreduzible-algebraische-Mengen}

\include{Chapter1/AlgGeo1-Chapter1-8_Quasikompakte-und-noethersche-topologische-Raeume}

\include{Chapter1/AlgGeo1-Chapter1-9_Morphismen-von-affinen-algebraischen-Mengen}

\include{Chapter1/AlgGeo1-Chapter1-10_Unzulaenglichkeiten}

\include{Chapter1/AlgGeo1-Chapter1-11_Der-affine-Koordinatenring}

\include{Chapter1/AlgGeo1-Chapter1-12_Funktorielle-Eigenschaften-des-Koordinatenrings}

\include{Chapter1/AlgGeo1-Chapter1-13_Raeume-mit-Funktionen}

\include{Chapter1/AlgGeo1-Chapter1-14_Der-Raum-mit-Funktionen-zu-einer-affin-algebraischen-Menge}

\include{Chapter1/AlgGeo1-Chapter1-15_Funktorialitaet-der-Konstruktion}

\include{Chapter1/AlgGeo1-Chapter1-16_Definition-von-Praevarietaeten}

\include{Chapter1/AlgGeo1-Chapter1-17_Vergleich-mit-differenzierbaren-komplexen-Mannigfaltigkeiten}


\section{Topologische Eigenschaften von Prävarietäten}
\label{sec:topologische-eigenschaften-von-praevarietaeten}
\begin{lem}
  \label{lem:bijektion-irred-teilraeume}
  Für einen topologischen Raum $X$ und $U\subseteq X$ offen haben
  wir eine Bijektion
  \begin{align*}
    \{Y\subseteq U\text{ irred. abg.}\} & \longleftrightarrow\{Z\subseteq X\text{ irred. abg. mit }Z\cap U\neq\emptyset\}\\
    Y & \longmapsto\overline{Y}\text{ (Abschluss in }X)\\
    Z\cap U & \longmapsfrom Z
  \end{align*}
\end{lem}
\begin{proof}
  Lemma \ref{lem:irreduzibel-abschluss}: $Y\subseteq X$ irreduzibel
  $\Leftrightarrow\overline{Y}\subseteq X$ irreduzibel.

  $Y\subseteq U$ abgeschlossen $\Leftrightarrow\exists A\subseteq X$
  abgeschlossen: $Y=U\cap A$.

  $\Rightarrow Y\subseteq\overline{Y}\subseteq A$ $\Rightarrow Y=U\cap\overline{Y}$

  $Y$ irreduzibel in $U$ $\Rightarrow Y$ irreduzibel in $X$

  $\Rightarrow$ $\overline{Y}$ irreduzibel nach \ref{lem:irreduzibel-abschluss}

  $\Rightarrow Y\mapsto\overline{Y}\mapsto\overline{Y}\cap U=Y$. $\checkmark$

  $\emptyset\neq Z\cap U \subseteq Z$
  damit dicht da $Z$ irreduzibel (Satz \ref{prop:charakterisierung-irreduzibel} ii. und v.)

  Also ist die Abbildung $\leftarrow$ wohldefiniert.

  $\Rightarrow\overline{Z\cap U}=Z$ 
\end{proof}
\begin{prop}
  \label{prop:praevarietaeten-noethersch-irreduzibel}
  Sei $(X,\mathcal{O}_{X})$ eine Prävarietät.

  Dann ist $X$ noethersch (insbesondere quasikompakt) und irreduzibel.
\end{prop}
\begin{proof}
  Sei $X=\bigcup_{i=1}^{n}$ endliche offene aff. Überdeckung und $X\supseteq Z_{1}\supseteq Z_{2}\supseteq\cdots$
  eine absteigende Kette abgeschlossener Teilmengen.

  $\Rightarrow U_{i}\cap Z_{1}\supseteq U_{i}\cap Z_{2}\supseteq\cdots$
  , ist eine absteigende Kette abgeschlossener Teilmengen von $U_{i}$

  $\Rightarrow\forall i$ $\exists n_{i} \in \mathbb{N}$: $U_{i}\cap Z_{n_{i}}=U_{i}\cap Z_{i+m}$ für alle $m \in \mathbb{N}$.
  Setzen wir $n:=\max n_{i}$, so folgt:

  $\forall i=1,\ldots,n$ $\forall m\geq n$: $U_{i}\cap Z_{m}=U_{i}\cap Z_{m+1}$

  $\Rightarrow(Z_{i})_{i}$ wird stationär da $Z_{m}=\bigcup_{i} U_{i}\cap Z_{m}$.

  $X$ ist demnach noethersch.

  $X$ ist weiter irreduzibel:

  Sei $X=X_{1}\cup\cdots\cup X_{n}$ die Zerlegung in irreduzible Komponenten.

  Angenommen es wäre $n\geq2$.

  $\Rightarrow\exists i_{0}\in\{2,\ldots,n\}$: $X_{1}\cap X_{i_{0}}\neq\emptyset$.
  (Andernfalls gilt: $X=X_{1}\sqcup\underbrace{X\backslash X_{1}}_{=X_{2}\cup\cdots\cup X_{n}\text{ abg.}}$, im Widerspruch dazu, dass $X$ zusammenhängend ist.)

  Sei ohne Einschränkung $i_{0}=2$. Sei $x\in X_{1}\cap X_{2}$, $x\in U\subseteq X$ offen, affin (d.h. affine Varietät).

  $U$ irreduzibel $\Rightarrow\overline{U}$ (Abschluss in $X$) $\subseteq X_{j}$
  für ein $j\in\{1,\ldots,n\}$

  \textbf{Jedoch}: Da $x\in X_{i}\cap U\subseteq U$ irreduzibel ist, ist $\underbrace{\overline{X_{i}\cap U}}_{\subseteq\overline{U}\subseteq X_{i}}=X_{i}$,
  $i=1,2$

  $\Rightarrow X_{1},X_{2}\subseteq X_{j}$. Widerspruch zu maximale
  Komponente.
\end{proof}



\include{Chapter1/AlgGeo1-Chapter1-19_Offene-Untervarietaeten}

\include{Chapter1/AlgGeo1-Chapter1-20_Funktionenkoerper-einer-Praevarietaet}

\include{Chapter1/AlgGeo1-Chapter1-21_Abgeschlossene-Unterpraevarietaeten}

\include{Chapter1/AlgGeo1-Chapter1-22_Homogene-Polynome}

\include{Chapter1/AlgGeo1-Chapter1-23_Definition-des-projektiven-Raumes}


\section{Projektive Varietäten}
\label{sec:projektive-varietaeten}
\begin{defn}[orig. 54]
  \label{def:projektive-varietaeten}
  Abgeschlossene Unterprävarietäten eines projektiven Raumes $\mathbb{P}^{n}(k)$
  heißen \textbf{projektive Varietäten}.
\end{defn}
Vorsicht: für $x=(x_{0}:\ldots:x_{n})\in\mathbb{P}^{n}$, $f\in k[X_{0},\ldots,X_{n}]$
ist $f(x_{1},\ldots,x_{n})$ \emph{nicht} wohldefiniert, da von Repräsentaten
abhängig, d.h. $f$ kann \emph{nicht}\textbf{ }als Funktion auf $\mathbb{P}^{n}$
aufgefasst werden. Für \emph{homogene}\textbf{ }Polynome $f_{1},\ldots,f_{n}\in k[X_{0},\ldots X_{n}]$
(nicht notwendig vom selben Grad) können wir dennoch Verschwindungsmengen
definieren:
\[
  V_{+}(f_{1},\ldots,f_{n})=\{(x_{0}:\ldots:x_{n})\in\mathbb{P}^{n}\mid f_{j}(x_{0},\ldots,x_{n})=0\ \forall j\}
\]

Da $V_{+}(f_{1},\ldots,f_{n})\cap U_{i}=V(\Phi_{i}(f_{1}),\ldots,\Phi_{i}(f_{m}))$
ist $V_{+}(f_{1},\ldots,f_{m})$ abgeschlossen in $\mathbb{P}^{n}$.
Ist $V_{+}(f_{1},\ldots,f_{n})$ irreduzibel, so erhalten wir eine
projektive Varietät. In der Tat entstehen alle projektiven Varietäten
auf diese Weise, wie der folgende Satz zeigt:
\begin{prop}[orig. 55]
  \label{prop:charakterisierung-projektive-varietaeten}
  Sei $Z\subseteq\mathbb{P}^{n}(k)$ eine projektive Varietät. Dann
  existieren homogene Polynome $f_{1},\ldots,f_{n}\in k[X_{0},\ldots,X_{n}]$,
  so dass
  \[
    Z=V_{+}(f_{1},\ldots,f_{n})
  \]

  gilt.
\end{prop}
\begin{proof}
  Betrachte: 
  \[
    \begin{array}{cc}
      \\
      \\
    \end{array}
  \]

  $f| :f^{-1}(U_{i})\longrightarrow U_{i}$ ist Morphismus
  von Prävarietäten. Dann ist $f$ selbst ein Morphismus von Prävarietäten: $\emph{lokal}$ ist die Aussage klar, $\emph{global}$ verklebt man.
  \begin{align*}
    \overline{Y}:= & Y\cup\{0\}\text{, der Abschluss von }Y\text{ in }\mathbb{A}^{n+1}(k)\\
    \mathfrak{a}:= & I(\overline{Y})\subseteq k[X_{0},\ldots,X_{n}]
  \end{align*}

  Behauptung: $\mathfrak{a}$ wird von homogenen Polynomen erzeugt\emph{.
    Denn:} Sei für $g\in\mathfrak{a}$, $g=\sum_{d}g_{d}$ die Zerlegung in homogene
  Bestandteile vom Grad $d$. $\overline{Y}$ ist Vereinigung von Ursprungsgeraden
  im $k^{n+1}$, d.h. $\forall\lambda\in k^{\times}$ gilt:
  \[
    g(x_{0},\ldots,x_{n})=0\ \Leftrightarrow\ g(\lambda x_{0},\ldots,\lambda x_{n})=0
  \]

  Beweis durch Widerspruch: \emph{Angenommen} nicht alle $g_{d}$ liegen in $\mathfrak{a}$.

  $\Rightarrow\exists(x_{0},\ldots,x_{n})\in\mathbb{A}^{n+1}(k)$, so
  dass $g(x_{0},\ldots,x_{n})=0$, aber $g_{d_{0}}(x_{0},\ldots,x_{n})\neq0$.

  $\Rightarrow0\,\neq\sum_{d}g_{d}(x_{0},\ldots,x_{n})T^{d}\in k[T]$

  $\Rightarrow\exists\lambda\in k^{\times}$: $0\neq\sum_{d}g_{d}(x_{0},\ldots,x_{n})\lambda^{d}=\sum_{d}g_{d}(\lambda x_{0},\ldots,\lambda x_{n})=g(\lambda x_{0},\ldots,\lambda x_{n})=0$.
  Widerspruch.

  $\Rightarrow\mathfrak{a}=(f_{1},\ldots,f_{m})$, mit $f_{j}$ homogen, also $Z=V_{+}(f_{1},\ldots,f_{m})$. 
  \begin{align*}
    Z\ni(x_{0}:\ldots:x_{n}) & \Leftrightarrow(\lambda x_{0},\ldots,\lambda x_{n})\in\overline{Y}\ \forall\lambda\in k^{\times}\text{ und }\neq0\\
                             & \Leftrightarrow f_{i}(x_{0},\ldots,x_{n})=0\ \forall1\leq i\leq n,\ (x_{0},\ldots,x_{n})\in\mathbb{P}^{n}
  \end{align*}

  %\rule[0.5ex]{1\columnwidth}{1pt}
\end{proof}
Zu Bemerkung \ref{rem:charakterisierung-homogen}:

Nach Satz \ref{prop:charakterisierung-reg-fkt-projektiver-raum} und Definition von $\mathcal{O}_{Z}'$ folgt: Ist $X$
eine projektive Varietät und $U\subseteq X$ offen, so erhalten wir 

($\dagger$) \ $\mathcal{O}_{X}(U)=\{f:U\rightarrow k\mid\forall x\in U\ \exists x\in V\underset{\text{offen}}{\subseteq}U,\ g,h\in k[X_{0},\ldots,X_{n}]$
homogen vom gleichen Grad mit $h(v)\neq0, \ f(v)=\frac{g(v)}{h(v)},\ \forall v\in V\}$.

Insbesondere gilt:
\begin{prop}[orig. 56]
  \label{prop:charakterisierung-morphismen-proj-varietaeten}
  Seien $V\subseteq\mathbb{P}^{m}(k)$, $W\subseteq\mathbb{P}^{n}(k)$
  projektive Varietäten und
  \[
    \phi: V \longrightarrow W
  \]

  eine Abbildung. Dann ist $\phi$ eine Morphismus genau dann, wenn
  zu jedem $x\in V$ eine offene Umgebung $x\in U_{x}\subseteq V$ und
  homogene Polynome $f_{0},\ldots,f_{n}\subseteq k[X_{0},\ldots,X_{m}]$
  vom selben Grad existieren mit
  \[
    \phi(y)=(f_{0}(y),\ldots,f_{n}(y))\quad\forall y\in U_{x}
  \]
\end{prop}
\begin{proof}
  \mbox{}
  \begin{itemize}
  \item ``$\Rightarrow$'': Übung.
  \item ``$\Leftarrow$'':
    \begin{enumerate}
    \item $\phi$ stetig: Sei $Z\subseteq W$ abgeschlossen. Ohne Einschränkung
      $Z=V_{+}(g)\cap W$ für ein homogenes Polynom $g$. Dann berechnet
      sich das Urbild
      \[
        \phi^{-1}(Z)=V_{+}(g\circ\phi)\cap V.
      \]
      Auf $U_{x}$, $x\in V$, ist $g\circ\phi$ als homogenes Polynom in
      $X_{0},\ldots,X_{n}$ gegeben. 

      $\Rightarrow V(g\circ\phi)\cap U_{x}=\phi^{-1}(Z)\cap U_{x}$ abgeschlossen
      in $U_{x}$ für alle $x$.

      $\Rightarrow\phi^{-1}(Z)\subseteq V$ abgeschlossen.
    \item Zu zeigen: $\forall W'\subseteq W$ offen, $g\in\mathcal{O}_{W}(W')$
      ist $g\circ\phi\in\mathcal{O}_{V}(\phi^{-1}(W'))$.

      ($\dagger$) $\Rightarrow$ Es ex. eine offene Umgebung $W_{y}$ in $W'$
      mit $g=\frac{h}{q}$ auf $W_{y}$, $h,q$ homogen vom Grad $d$.

      $\Rightarrow\phi_{|U_{x}\cap\phi^{-1}(W_{y}):=\tilde{U}_{x}}$ ist
      auch von dieser Gestalt, also $\frac{h(f_{0},\ldots,f_{n})}{q(f_{0},\ldots,f_{n})}=g\circ\phi_{|\tilde{U}_{x}}\in\mathcal{O}_{V}(\tilde{U}_{x})$.
    \end{enumerate}
    Verklebungsaxiom $\Rightarrow$ $g\circ\phi\in\mathcal{O}_{V}(\phi^{-1}(V))$.
  \end{itemize}
\end{proof}




\section{Koordinatenwechsel in $\mathbb{P}^{n}$}
\label{sec:koordinatenwechsel-projektiver-raum}

Sei $A=(a_{ij})\in GL_{n+1}(k)$ eine invertierbare, lineare Abbildung $k^{n+1}\rightarrow k^{n+1}$. Dann überführt $A$ Ursprungsgeraden in Ursprungsgeraden, respektiert also die Äquivalenzrelation des projektiven Raumes. Wir erhalten Abbildungen:
\begin{align*}
  \mathbb{P}^{n}(k) & \overset{\phi_{A}}{\longrightarrow}\mathbb{P}^{n}(k)\\
  (x_{0}:\ldots:x_{n}) & \longmapsto\left(\sum_{i=0}^{n}a_{0i}x_{i}:\ldots:\sum_{i=0}^{n}a_{ni}x_{i}\right),
\end{align*}

die nach Satz \ref{prop:charakterisierung-morphismen-proj-varietaeten} ein Morphismus von Prävarietäten ist. Offensichtlich
gilt für $A,B\in GL_{n+1}(k)$:
\[
  \varphi_{A\cdot B}=\varphi_{A}\circ\varphi_{B}
\]

d.h. $\varphi_{A}$ ist insbesondere wieder ein Isomorphismus, \textbf{der
  durch $A$ bestimmte Koordinatenwechsel des $\mathbb{P}^{n}(k)$}.
Es \emph{bezeichne} Aut$(\mathbb{P}^{n}(k))$ die Gruppe der Automorphismen
von $\mathbb{P}^{n}(k)$. Es folgt:
\[
  \varphi_{-}:GL_{n+1}(k)\rightarrow\text{Aut}(\mathbb{P}^{n}(k)), A \mapsto \varphi_{A}
\]

ist ein Gruppenhomomorphismus mit 
\[
  Z:=\ker\varphi_{-}=\{\lambda E_{n+1},\ \mid \lambda\in k^{\times}\}
\]

der Untergruppe der Skalarmatrizen. \emph{Später}:
\[
  PGL_{n+1}(k):=GL_{n+1}(k)/Z\overset{\sim}\longrightarrow\text{Aut}(\mathbb{P}^{n}(k)),\quad Z\cong k^{\times}
\]

die \textbf{projektive lineare Gruppe}.
\begin{example*}
Sei $n=1$. Es ist
\begin{align*}
PGL_{2}(\mathbb{C}) & =\left\{ \begin{array}{rl}
\mathbb{P}^{1}(\mathbb{C}) & \rightarrow\mathbb{P}^{1}(\mathbb{C})\\
(z:w) & \mapsto(az+bw,cz+dw)
\end{array}\right\} \\
 & \leftrightarrow\text{Möbiustransformationen }z\mapsto\frac{az+b}{cz+d}
\end{align*}
\end{example*}



\include{Chapter1/AlgGeo1-Chapter1-26_Lineare-Unterraeume-des-projektiven-Raums}

\section{Kegel}
\label{sec:Kegel}

Sei $H\subseteq\mathbb{P}^{n}(k)$ Hyperebene, $p\in\mathbb{P}^{n}(k)\setminus H$,
$X\subseteq H$ abgeschlossene Unterprävarietät.
\[
  \overline{X,p}:=\bigcup_{q\in X}\overline{qp}
\]

heißt \textbf{Kegel von $X$ über $p$}, es handelt sich um eine
abgeschlossenen Untervarietät von $\mathbb{P}^{n}(k)$. Ohne Einschränkung: $H=V_{+}(X_{n})$,
$p=(0:\ldots:0:1)$ (geeigneter Koordinatenwechsel)
Für  
\begin{align*}
  X=V_{+}(f_{1},\ldots,f_{m})\subseteq\mathbb{P}^{n-1}(k)=H, & \quad f_{i}\in k[X_{0},\ldots,X_{n-1}]\\
  \Rightarrow \overline{X,p}=V_{+}(\tilde{f}_{1},\ldots,\tilde{f}_{m})\subseteq\mathbb{P}^{n}(k), & \quad\tilde{f}_{i}\in k[X_{0},\ldots,X_{n}]
\end{align*}

Verallgemeinerung. Sei $\mathbb{P}^{m}(k)\cong\Lambda\subseteq\mathbb{P}^{n}(k)$
linearer Unterraum, $V\subseteq\mathbb{P}^{n}(k)$ komplementärer
linearer Unterraum, d.h. $\Lambda\cap V=\emptyset$ und $\mathbb{P}^{n}(k)$
ist der \emph{kleinste} lineare Unterraum von $\mathbb{P}^{n}(k)$, der $\Lambda$
und $V$ enthält. Für $X\subseteq V$ eine abgeschlossene Unterprävarietät definiert man den

\textbf{Kegel von $X$ über $\Lambda$} durch $\overline{X,\Lambda} :=\bigcup_{q\in X}\overline{q,\Lambda}$,
wobei der von $q$ und $\Lambda$ aufgespannte lineare Unterraum $\overline{q,\Lambda}$
der kleinste Unterraum sei, der $q$ und $\Lambda$ enthält.


\section{Quadriken}
\label{sec:quadriken}

Sei in diesem Abschnitt char$(k)\neq2$.
\begin{defn}[orig. 57]
  \label{def:quadrik}
  Eine abgeschlossene Unterprävarietät $Q\subseteq\mathbb{P}^{n}(k)$
  von der Form $V_{+}(q)$, $0 \neq q\in k[X_{0},\ldots,X_{n}]_{2}$
  heißt \textbf{Quadrik}.
  \[
    Q=V_{+}(q)
  \]

  Zur quadratischen Form $q$ gehört eine assoziierte Bilinearform $\beta$ auf
  $k^{n+1}$ (vgl. lineare Algebra), 
  \[
    \beta(v,w):=\frac{1}{2}(q(v+w)-q(v)-q(w)),\quad v,w\in k^{n+1}
  \]

Es gibt eine Basis von $k^{n+1}$, sodass die Strukturmatrix $B$
von $\beta$ die Gestalt
\[
  B=\begin{pmatrix}
    \begin{array}{ccc}
      1\\
      & \ddots\\
      &  & 1
    \end{array} & 0\\
    0 &
    \begin{array}{ccc}
      0\\
      & \ddots\\
      &  & 0
    \end{array}
  \end{pmatrix}
\]

hat, d.h. Koordinatenwechsel zur Basiswechselmatrix liefert einen
Isomorphismus
\[
  Q\xrightarrow{\sim}V_{+}(X_{0}^{2}+\cdots+X_{r-1}^{2}),\quad r=\text{rk }B
\]
\end{defn}
\begin{lem}[orig. 58]
\label{lem:irreduzibilitaet-quadriken}
	\begin{enumerate}
	
		\item $X_{0}^{2} + \ldots + X_{r-1}^{2}$ ist irreduzibel $\iff$ $r > 2$
		\item $V_{+}(X_{0}^{2} + \ldots + X_{r-1}^{2})$ ist irreduzibel $\iff$ $r \neq 2$
	\end{enumerate}
\end{lem}
\begin{proof}
	\begin{itemize}
		\item $r=0,1: X_{0}^2 = X_{0} \cdot X_{0} \Rightarrow V_{+}(X_{0}^2) = V_{+}(X_{0})$ irreduzibel
		\item $r=2: X_{0}^{2} + X_{1}^{2} = (X_{0} + i\cdot X_{1})\cdot(X_{0} - i \cdot X_{1})$ für $i = \sqrt{-1}$ 
		\item $r>2: $ Angenommen $\sum_{i}{a_{i} X_{i}} \cdot \sum_{j}{b_{j}X_{j}} = X_{0}^{2} + \ldots X_{r-1}^{2}$.\\
		Ausmultiplizieren $+$ Koeffizientenvergleich $\Rightarrow$ Widerspruch.
	\end{itemize}
\end{proof}

\begin{prop}[orig. 59]
  \label{prop:quadrik-in-normalform}
  Ist $r\neq s$, so sind $V_{+}(T_{0}^{2}+\cdots+T_{r-1}^{2})$ und
  $V_{+}(T_{0}^{2}+\cdots+T_{s-1}^{2})$ nicht isomorph.
\end{prop}
\begin{proof}
  Später: Es gibt keinen Koordinatenwechsel von $\mathbb{P}^{n}(k)$,
  der die beiden Mengen miteinander identifiziert, damit auch kein Automorphismus von
  $\mathbb{P}^{n}(k)$.
\end{proof}

\begin{defn}
  \label{def:dim-und-rang-einer-quadrik}
  Eine Quadrik $Q\subseteq\mathbb{P}^{n}(k)$ mit
  $Q\cong V_{+}(T_{0}^{2}+\cdots+T_{r-1}^{2})$, $r\geq1$, hat \textbf{Dimension $n-1$} und den \textbf{Rang $r$}. (nach Satz
  eindeutig!)
\end{defn}

\begin{cor}[orig. 61]
\label{cor:klassifikation-von-quadriken}
  Zwei Quadriken $Q_{1}$ und $Q_{2}$ sind genau dann isomorph als
  Prävarietäten, wenn sie dieselbe Dimension und denselben Rang haben.
\end{cor}
\begin{proof}
  \mbox{}
  \begin{itemize}
  \item[,,$\Leftarrow$``]
    $Q_{1}\cong V_{+}(T_{0}^{2}+\cdots+T_{n-1}^{2})\cong Q_{2}$ in
    dem selben $\mathbb{P}^{n}$.
  \item[,,$\Rightarrow$``] Für $Q\subseteq\mathbb{P}^{n}(k)$ berechne
    $K(Q)$. Ohne Einschränkung
    $Q=V_{+}(X_{0}^{2}+\cdots+X_{n-1}^{2})$.
    \begin{enumerate}
    \item $r=1$: $V_{+}(X_{0}^{2})=V_{+}(X_{0})=\mathbb{P}^{n-1}(k)$:
      $K(Q)=k(T_{1},\ldots,T_{n-1})$.
    \item $r=2$: reduzibel: Zerlegung in zwei Hyperebenen
      $Z\cong\mathbb{P}^{n-1}$

      $\Rightarrow K(Z)\cong k(T_{1},\ldots,T_{n-1})$.
    \item $r>2$:
      $U=V(1+T_{1}^{2}+\cdots+T_{n-1}^{2})\subseteq\mathbb{A}^{n}(k)$
      ist nichtleere offene affine Teilmenge von $Q$.

      $\Rightarrow K(Q)=K(U)=\text{Quot}(\Gamma(U))=\text{Quot}(k[T_{1},\ldots,T_{n}]/(1+T_{1}^{2}+\cdots+T_{n-1}^{2})$

      $\Rightarrow\text{trgrad}_{k}\ K(Q)=n-1$. 
    \end{enumerate}
  \end{itemize}
\end{proof}
\begin{example}
  \label{bsp:quadrik-joe-harris}
  $Q$ Quadrik in $\mathbb{P}^{n}$ (vgl. Joe Harris, Seite 34).
  \begin{enumerate}
  \item In $\mathbb{P}^{1}(k)$. 
    \begin{itemize}
    \item \emph{Rang 2:} 2 Punkte, reduzibel. 
    \item \emph{Rang 1:} 1 Punkt (Doppelpunkt). 
    \end{itemize}
  \item In $\mathbb{P}^{2}(k)$.
    \begin{itemize}
    \item \emph{Rang 3:} Glatter Kegel
      $\cong\mathbb{P}^{1}(k)$. $X_{0}^{2}+X_{1}^{2}-X_{2}^{2}=0$
    \item \emph{Rang 2:} 2 verschiedene Geraden, reduzibel. 
    \item \emph{Rang 1:} (Doppel)gerade.
    \end{itemize}
  \item In $\mathbb{P}^{3}(k)$.
    \begin{itemize}
    \item Rang 1: Doppelebene (2-dimensionaler linearer Unterraum)
    \item Rang 2: (insert image)
    \item Rang 3: (insert image)
    \item Rang 4: (insert image)
    \end{itemize}
  \end{enumerate}
\end{example}
Die Quadrik $Q\subseteq\mathbb{P}^{n}(k)$ heißt \textbf{glatt}, falls
$r=n+1$, d.h. falls die Matrix $B$ zu $q$ maximalen Rang hat. Für
$\text{rk}(Q)>3$, $\dim(Q)=d$, ist
$Q\cong\overline{\widetilde{Q},\Lambda}$ Kegel über einer
\textbf{glatten} Quadrik $\widetilde{Q}$, da Dimension $r-2$
bzgl. einer $(d-r+2)$-dimensionalen Unterraums 1.
\begin{itemize}
\item $r=1,2$ ausgeartet.
\item $r=1$. $Q=V_{+}(X_{0}^{2})=V_{+}(X_{0})$ Hyperebenen in
  $\mathbb{P}^{n}(k)$.  Der Unterschied zwischen $V_{+}(X_{0}^{2})$
  und $V_{+}(X_{0})$ ist für eine projektive Varietät $Q$ nicht
  sichtbar, jedoch in der Theorie der Schemata unterscheidbar!
\item $r=2$. $Q=V_{+}(X_{0}^{2}+X_{1}^{2})$ reduzibel, d.h. keine
  Prävarietät in unserem Sinne! Auch hier werden uns Schemata später helfen.
\end{itemize}
\medskip{}

$Q=V_{+}(X_{0}^{2}+X_{1}^{2}+\cdots+X_{n-1}^{2})\subseteq\mathbb{P}^{d+1}$,
$r\leq d+2$

$\tilde{Q}=V_{+}(X_{0}^{2}+\cdots+X_{n-1}^{2})\subseteq\mathbb{P}^{r-1}$
glatt.

$A=\mathbb{P}^{d+1-v}=V_{+}(X_{0},\ldots,X_{n-1})\subseteq\mathbb{P}^{d+1}$

$Q=\overline{\widetilde{Q},\Lambda}$


\chapter{Das Ringspektrum}
\label{chap:das-ringspektrum}

Prävarietäten sind Verklebungen. $k$ algebraisch abgeschlossen:

Affine Varietäten $\leftrightarrow$ integere endlich erzeugte
$k$-Algebren.  Punkte $\hat{=}$ maximale Ideale.

\textbf{Ziel}: Schemata sind Verklebungen.

Affine Schemta $\leftrightarrow$ (kommutative) Ringe. Punkte $\hat{=}$
Primideale.

\textbf{Ziel}: Wir wollen einen Funktor:
\begin{align*}
  A & \longmapsto(\Spec(A),\mathcal{O}_{\Spec(A)})\\
  \text{Ring} & \longrightarrow\text{top. Raum}
\end{align*}

,,Garbe von Funktionen`` verallgemeinert ,,Systeme von Funktionen``
für Raume von Funktionen.

(Insbesondere $k$-Algebren über beliebige Körper $k$!)

(Sei $\varphi:A\rightarrow B$ Ringhomomorphismus,
$\mathfrak{m}\subseteq B$ maximales Ideal. Dann folgt i.A. nicht, dass
$\varphi^{-1}(\mathfrak{m})$ maximal ist. Wir haben also zu wenige
maximale Ideale.)

\section*{Das Ringspektrum als topologischer Raum}

\section{Definition von Spec(A)}

Sei $A$ stets ein kommutativer Ring. Spec(A) =
$\{\mathfrak{p}\subseteq A$ Primideal\}. Sei $M\subset A$.
\begin{align*}
  V(M) & =\{\mathfrak{p}\in\Spec(A)\mid\mathfrak{p}\supset
         M\}=V\{\langle M\rangle\}\\
  V(f) & =V(\{f\})\text{\,für }f\in A
\end{align*}

\begin{lem}[1] Es ist
  \begin{align*}
    \{\text{Ideale in }A\} & \longrightarrow\text{\{Teilmengen in }\Spec(A)\}\\
    \mathfrak{A} & \longmapsto V(\mathfrak{A})
  \end{align*}

  ist eine inklusionsumkehrende Abbildung. Es gilt:
  \begin{enumerate}
  \item $V(0)=\Spec(A)$, $V(1)=\emptyset$.
  \item $V\left(\bigcup_{i\in
        I}\mathfrak{a}_{i}\right)=V\left(\sum_{i\in
        I}\mathfrak{a}_{i}\right)=\bigcap_{i\in I}V(\mathfrak{a}_{i})$
  \item
    $V(\mathfrak{a}\cap\mathfrak{a}')=V(\mathfrak{a}\mathfrak{a}')=V(\mathfrak{a})\cup
    V(\mathfrak{a}')$
  \end{enumerate}
\end{lem}
\begin{proof} \mbox{}
  \begin{itemize}
  \item (1), (2) klar.
  \item
    (3). $\mathfrak{p}\supset\mathfrak{a}\cap\mathfrak{a}'\supset\mathfrak{a}\mathfrak{a}'$.

    $\Rightarrow\mathfrak{p}\supset\mathfrak{a}\mathfrak{a}'$.

    $\Rightarrow$ (Primideal) $\mathfrak{p}\supset\mathfrak{a}$ oder
    $\mathfrak{p}\supset\mathfrak{a}'$.

    $\Rightarrow\mathfrak{p}\supset\mathfrak{a}\cap\mathfrak{a}'$

  \end{itemize}
\end{proof}
\begin{defn} Spec(A) mit der Topologie, dessen abgeschlossene Mengen
  gerade die Mengen der Form $V(\mathfrak{a})$,
  $\mathfrak{a}\subset A$ ein Ideal sind, heißt (Prim)Spektrum von $A$
  (mit der Zariski-Topologie).
  \begin{align*}
    x\in\Spec(A) & \leftrightarrow\mathfrak{p}_{x}\subset A\text{ Primideal}\\
    Y\subset\Spec(A), & \phantom{\leftrightarrow\
                              }I(Y):=\bigcap_{\mathfrak{p}\in Y}\mathfrak{p}
  \end{align*}

  $I(-)$ ist inklusionserhaltend, $I(\emptyset)=A$.
\end{defn}
\begin{prop} $\mathfrak{a}\subset A$ Ideal,
  $Y\subset\Spec(A)$. Dann gilt:
  \begin{enumerate}
  \item $\text{rad }I(Y)=I(Y)$, $V(\mathfrak{a})=V(\text{rad
    }\mathfrak{a})$
  \item $I(V(\mathfrak{a}))=\text{rad}(\mathfrak{a})$,
    $V(I(Y))=\overline{Y}$ (Abschluss in $\Spec(A)$).
  \item Wir haben eine 1:1-Korrespondenz:
    \begin{align*}
      \{\mathfrak{a}\subset A\mid\mathfrak{a}=\rad\mathfrak{a}\}
      & \longrightarrow\{\text{abg. Teilmengen }Y\text{ in }\Spec(A)\}
    \end{align*}
  \end{enumerate}
\end{prop}
\begin{proof} \mbox{}
  \begin{enumerate}
  \item $V(\mathfrak{a})=V(\rad\mathfrak{a})$.
    \begin{itemize}
    \item ,,$\supseteq$``. Klar, da rad
      $\mathfrak{a}\supseteq\mathfrak{a}$.
    \item ,,$\subseteq$``. Aus $f^{r}\in\mathfrak{a}\subseteq\mathfrak{a}$
      folgt $f\in\mathfrak{p}$, da $\mathfrak{p}$ Primideal. Damit:
      $\text{rad }\mathfrak{a}\subset\mathfrak{p}.$
    \end{itemize}
  \item $\text{rad }\mathfrak{a}=\bigcap_{\mathfrak{p}\in
      V(\mathfrak{a})}\mathfrak{p}=IV(\mathfrak{a})$.  Es ist:
    \begin{align*}
      V(\mathfrak{b})\supseteq Y & \Leftrightarrow(\forall\mathfrak{p}\in Y:\
                                   \mathfrak{p}\supset\mathfrak{b})\\
                                 & \Leftrightarrow
                                   I(Y)\supseteq\mathfrak{b}.
    \end{align*} Damit ist $V(I(Y))$ die kleinste abgeschlossene
    Teilmenge, die $Y$ umfasst, d.h. $V(I(Y))=\overline{Y}$.
  \item
  \end{enumerate}
\end{proof}


\section{Topologische Eigenschaften von Spec(A)}
\label{sec:topologische-eigenschaften-von-spec-A}

Definiere $D(f):=D_{A}(f):=\Spec(A)\setminus V(f)=\{x \in\Spec A \mid f\notin\mathfrak{p}_{x}\}$,
\begin{align*}
  \text{ev}_{x}:A & \longrightarrow A/\mathfrak{p}_{x}\subseteq \kappa_{x}(A) := \Quot(A/\mathfrak{p}_{x})\\
  f & \longmapsto f(x) := f(\mathfrak{p}_{x}) := f \mod \mathfrak{p}
\end{align*}

Für $x \in D(f)$ gilt dann $f(x) = \text{ev}_{x}(f) \neq 0$.

\textbf{Standard prinzipal offene Mengen}.
\begin{align*}
  D(0) & =\emptyset,\ D(1)=\Spec(A)=D(u),\ u\in A^{\times}\\    
  & D(f)\cap D(g) = D(fg)
\end{align*}

\begin{lem}
\label{lem:charakterisierung-ueberdeckungen-prinzipal}
Für $f_{i} \in A, i\in I$, $g\in A$ gilt:
  \begin{align*}
    D(g)\subseteq\bigcup_{i\in I}D(f_{i})
    & \Leftrightarrow g^{n}\in\mathfrak{a}=(f_{i},i\in I)\text{ für }n \in \mathbb{N} \text{ geeignet}\\
    & \Leftrightarrow g\in\rad(\mathfrak{a})
  \end{align*}
\end{lem}
\begin{proof} Es gilt:
  \begin{align*}
    D(g)\subseteq\bigcup_{i}D(f)
    & \Leftrightarrow V(g)\supseteq\bigcap_{i} V(f_{i})=V(\mathfrak{a})\\
    & \Leftrightarrow g\in\rad((g))\subseteq\rad(\mathfrak{a}) \text{ nach } \ref{prop:nullstellensatz-primspektrum}
  \end{align*}

  Für $g=1$, folgt:
  \[ \Spec(A)=\bigcup_{i\in I}D(f_{i})\Leftrightarrow\sum_{i\in
      I}Af_{i}=A
  \]
\end{proof}
\begin{prop}
\label{prop:prinzipal-offene-bilden-basis}
Die prinzipal offenen Mengen $D(f)$, $f\in A$, bilden
  eine Basis der Topologie von $\Spec(A)$, und sind
  quasikompakt. Insbesondere ist $\Spec(A)$ quasikompakt.
\end{prop}
\begin{proof} Nach Lemma \ref{lem:zariski-top-auf-spektrum}$.(ii)$ gilt:
  \[
    V(\mathfrak{a})=\bigcap_{f \in\mathfrak{a}}V(f)\Longrightarrow\Spec A\setminus
    V(\mathfrak{a})=\bigcup_{f\in\mathfrak{a}}D(f)\Rightarrow\text{Basis
      der Topologie}
  \]

  Sei $D(g)\subseteq\bigcup_{i}D(f_{i})$.

  \ref{lem:charakterisierung-ueberdeckungen-prinzipal} $\Rightarrow$ $g^{n}=\sum_{i\in I}a_{i}f_{i}$, $a_{i}\in A$
  fast alle 0.

  $\Rightarrow D(g)\subseteq\bigcup_{i\in J}D(f_{i})$ $\forall i\in J\subseteq I$ endlich

  $\Rightarrow D(g)$ quasikompakt.
\end{proof}

\begin{prop}
  $X\subseteq\text{Spec}(A)$ ist irreduzibel, genau dann, wenn
  $\varphi:=I(Y)\subset A$ prim ist. In diesem Fall ist
  $\{\mathfrak{p}\}\subset\overline{Y}$ dicht!
\end{prop}

\begin{proof} \mbox{}
  \begin{itemize}
  \item Sei $Y$ irreduzibel und $f,g\in A$ mit $fg\in\mathfrak{p}$.

    $\Rightarrow Y\subset\overline{Y}=VI(Y)\subseteq V(fg)=V(f)\cup V(g)$

    $\Rightarrow$ ($X$ irreduzibel) Ohne Einschränkung: $Y\subset V(f)$.

    $\Rightarrow
    f\in\bigcap_{f\in\mathfrak{q}}\mathfrak{q}=IV(f)\subset
    I(Y)=\mathfrak{p}$

    $\Rightarrow\mathfrak{p}$ Primideal.
  \item Sei umgekehrt $\mathfrak{p}=I(Y)$ ein Primideal.

    $\Rightarrow$ (Satz 3)
    $\overline{Y}=V(\mathfrak{p})=VI(\{\mathfrak{p}\})=\overline{\{\mathfrak{p}\}}$,
    d.h. $\overline{Y}$ ist der Abschluss der irred. Menge
    $\{\mathfrak{p}\}$ und daher selbst irreduzibel.

    $\Rightarrow$ (Lemma I.14): $Y$ ist auch irreduzibel, da dicht in
    $\overline{Y}$.
  \end{itemize}
\end{proof}
Warnung: im Allgemeinen ist $\mathfrak{p}$ nicht in $Y$!

\begin{cor}
  Die Abbildung
  \begin{align*}
    \text{Spec}(A) & \longrightarrow\text{\{abg. irred. Teilmengen von Spec }A\}\\
    \mathfrak{p} & \longmapsto V(\mathfrak{p})=\overline{\{\mathfrak{p}\}}
  \end{align*}
  ist eine Bijektion, unter der die maximalen Primideale von $A$ den
  irreduziblen Komponenten entsprechen.
\end{cor}

\begin{proof} Proposition 3 und 6.
\end{proof}

\begin{defn}
  Für ein topologischer Raum $X$ heißt $\eta\in X$ ein
  \textbf{generischer Punkt}, falls
  $\overline{\{\eta\}}=X$. Allgemeiner sagen wir für $x,x'\in X$, dass
  $x$ eine Verallgeimeinerung (eng. ,,generalization``) von $x'$ ist,
  bzw. $x'$ eine Spezialisierung von $x$, falls
  $x'\in\overline{\{x\}}$.
\end{defn}

\begin{rem}\mbox{}
  \begin{enumerate}
  \item $\eta\in X$ generisch $\Leftrightarrow\eta$ ist
    Verallgemeinerung von jedem Punkt von $X$.
  \item Existiert ein generischer Punkt in $X$, so ist $X$ als
    Abschluss einer irreduziblen Menge selbst irreduzibel.
  \item Für $X=\text{Spec}(A)$ gilt: $x'$ ist eine Spezialisierung von
    $x\Leftrightarrow\mathfrak{p}_{x}\subset\mathfrak{p}_{x'}$
    \begin{align*}
      \Leftrightarrow V(\mathfrak{p}_{x'}) & \subset V(\mathfrak{p}_{x})\\
      \shortparallel & \phantom{\subset\,}\shortparallel\\
      x'\in\overline{\{x')} & \in\overline{\{x\}}
    \end{align*}
    Ferner hat jede abgeschlossene irreduzible Teilmenge
    $Y\subset\text{Spec}(A)$ einen eindeutigen generischen Punkt (dies
    gilt nicht für beliebig irreduzible Teilmengen
    $Y\subset\text{Spec}(A)$).
  \end{enumerate}
\end{rem}


\section{Der Funktor $A\protect\mapsto\text{Spec}(A)$}
\label{sec:spec-als-funktor}
\textbf{Ziel:} Wir wollen einen kontravarianten Funktor
\begin{align*}
  \text{\underline{CRing}} & \longrightarrow\text{\underline{Top}}\\
  A & \longmapsto\Spec A
\end{align*}
definieren. Sei $\varphi:A\longrightarrow B$ ein Ringhomomorphismus, $\mathfrak{q}$ Primideal von $B$. Es folgt:
$\varphi^{-1}(\mathfrak{q})\unlhd A$ ist Primideal, denn $A/\varphi^{-1}(\mathfrak{q})\hookrightarrow B/\mathfrak{q}$ ist integer als Unterring eines integren Rings.
Wir erhalten also eine Abbildung
\begin{align*}
  ^{a}\varphi=\Spec\varphi:\ \Spec B & \longrightarrow\Spec A\\
  \mathfrak{q} & \longmapsto\varphi^{-1}(\mathfrak{q})
\end{align*}

\begin{prop} \mbox{}
\label{prop:abbildungen-auf-spektren-sind-stetig}	
  \begin{enumerate}
  \item $(^{a}\varphi)^{-1}(V(M))=V(\varphi(M))$ für
    $M\subseteq\Spec A$ Teilmenge, insbesondere gilt
    $(^{a}\varphi)^{-1}(D(f))=D(\varphi(f))$, $f\in A$.
  \item
    $V(\varphi^{-1}(\mathfrak{b}))=\overline{^{a}\varphi(V(\mathfrak{b}))}$
    für $\mathfrak{b}\unlhd B$ Ideal.
  \end{enumerate}
\end{prop}

\begin{proof} \mbox{}
  \begin{enumerate}
  \item Für $\mathfrak{q}\in\text{Spec }B$ gilt:
    \begin{align}
       \mathfrak{q}\in V(\varphi(M)) \iff \mathfrak{q}\supseteq\varphi(M)
      \iff \varphi^{-1}(\mathfrak{q})\supseteq M \iff \mathfrak{q}\in(^a\varphi)^{-1}(V(M))
    \end{align}
    Weiter:
    \begin{align}
      D(\varphi(f)) & =\Spec(B)\setminus V(\varphi(f)) \\
                    & =\Spec(B)\setminus (^a\varphi)^{-1} (V(f)) \\
                    & = (^a\varphi)^{-1} (D(f))
    \end{align}
     
  \item
    $\overline{^{a}\varphi(V(\mathfrak{b}))}=VI(^{a}\varphi(V(\mathfrak{b})))$
    nach Satz \ref{prop:nullstellensatz-primspektrum}. Nach Definition gilt:
    \begin{align*} I(^{a}\varphi(V(\mathfrak{b}))
      & =\bigcap_{\mathfrak{p}\in^{a}\varphi(V(\mathfrak{b}))}
        \mathfrak{p}=\bigcap_{\mathfrak{q}\in V(\mathfrak{b})}
        \varphi^{-1}(\mathfrak{q})\\ \text{komm. Algebra }
      & =\varphi^{-1}(\rad\mathfrak{b})\\
      & \overset{!}{=}\rad\varphi^{-1}(\mathfrak{b})
    \end{align*}
    Denn: Ohne Einschränkung gelte $\mathfrak{b}=0$,
    $\varphi^{-1}(\mathfrak{b})=\ker\varphi$
    (betrachte $A/\varphi^{-1}(\mathfrak{b})\hookrightarrow B/\mathfrak{b})$. Es
    ist:
    \begin{align*}
      a\in\varphi^{-1}(\sqrt{0})
      & \Leftrightarrow\varphi(a)^{n}=\varphi(a^{n})=0
        \text{ für }n\text{ geeignet}
    \end{align*} $V(\cdot )$
    liefert die Behauptung: $V(\rad\varphi^{-1}
    (\mathfrak{b}))=V(\varphi^{-1}(\mathfrak{b}))$ nach Satz \ref{prop:nullstellensatz-primspektrum}.
  \end{enumerate}
\end{proof}
Insbesondere ist $^{a}\varphi:\Spec B\rightarrow\Spec A$
\emph{stetig}.  Wegen
\[
  ^{a}(\psi\circ\varphi)=\ ^{a}\varphi\ \circ\ ^{a}\psi \text{ und } ^{a}\mathrm{id}_{A} = \mathrm{id}_{\Spec A}
\]
für einen weiteren Ringhomomorphismus $\psi:B\rightarrow C$ ist
$A\mapsto\Spec A$ der gesuchte kontravariante Funktor.

\begin{cor}
\label{cor:charakterisierung-dominanz-auf-spec}	
  $^{a}\varphi$ ist \textbf{dominant},
  d.h. $\im\ (^{a}\varphi)\subseteq\Spec A$ dicht
  $\iff$ Jedes Element in $\ker\varphi$ ist nilpotent:
  $\ker\varphi\subseteq\text{rad}(0)$.
\end{cor}

\begin{prop} \mbox{}
\label{prop:spec-quotienten-lokalisierung}	
  \begin{enumerate}
  \item Ist $\varphi:A\rightarrow B$ ein surjektiver
    Ringhomomorphismus mit $\ker\varphi=:\mathfrak{a}$, dann ist
    $^{a}\varphi$ ein Homöomorphismus von $\Spec B$ auf
    $V(\mathfrak{a})\underset{\text{abg.}}{\subseteq}\Spec A$.
  \item Ist $S$ eine multiplikativ abgeschlossene Teilmenge von $A$,
    und $\varphi:A\longrightarrow S^{-1}A=:B$ die kanonische
    Lokalisierungsabbildung, dann ist $^{a}\varphi$ ein
    Homöomorphismus, von $\Spec S^{-1}A$ auf $\{\mathfrak{p}\in\Spec A\mid
    S\cap\mathfrak{p}=\emptyset\}$.
  \end{enumerate}
\end{prop}

\begin{proof}
  $^{a}\varphi$ injektiv + $\im\ ^{a}\varphi$ ist bekannt aus kommutative Algebra.
  \emph{Ferner}: Für $\mathfrak{q}\in\text{Spec }B$,
  $\mathfrak{b}\unlhd B$ Ideal gilt
  $\mathfrak{q}\supseteq\mathfrak{b}\Leftrightarrow
  \varphi^{-1}(\mathfrak{q})\supseteq\varphi^{-1}(\mathfrak{b})$, also
  \begin{align*}
    ^{a}\varphi(V(\mathfrak{b})) & =V(\varphi^{-1}(\mathfrak{b})),
  \end{align*}
  d.h. $^{a}\varphi$ ist abgeschlossen.
\end{proof}


\section{Beispiele}
\begin{itemize}
\item Spec $A=\emptyset\Leftrightarrow A=\{0\}$.
\item $A$ Körper oder Ring mit einzigem Primideal: $\Spec
  A=\{\mathfrak{p}\}$.
\item $A$ Artinsch: $\Spec A$ endlich und diskret (da maximale
  Primidealen mit den minimalen Primidealen übereinstimmen)

  ($\Spec A=\Spec(A/\sqrt{0})$, $A/\sqrt{0}$ Produkt von Körpern.
  $\Spec(\prod A_{i})=\coprod\Spec(A_{i})$
\end{itemize}

\begin{example}
  Sei $A$ Hauptidealring (z.B. $\mathbb{Z}$ oder $K[X]$). Falls
  $\mathfrak{p}$ ein maximales Ideal ist, dann ist
  $\mathfrak{p}=(\pi)$, $\pi$ Primelement in $A$.

  Alle Primideale sind maximal oder 0.

  Abg. Punkte von $\Spec A\leftrightarrow$ Primelemente modulo $A^{\times}$

  $\overline{\{\eta\}}=\Spec A$ für $\eta\in\Spec A$ mit $\mathfrak{p}_{\eta}=(0)$.

  Abg. Mengen. $\Spec A\neq
  V(\mathfrak{a})\overset{0\neq\mathfrak{a}=(f)}{=}V(f)=\{(p_{1}),\ldots,(p_{n})\}$
  falls $f=p_{1}^{e_{1}}\cdots p_{n}^{e_{n}}$, $p_{i}$ paarweise
  verschieden, $e_{1}\geq1$, \emph{endliche Mengen abgeschlossener Punkte.}

  $g\neq0\neq f$:
  \begin{align*}
    V(f)\cap V(g) & =V(f,g)=V(d), & d=\text{ggT}(f,g)\\
    V(f)\cup V(g) & =V((f)\cap(g))=V(e), & e=\text{kgV}(f,g)
  \end{align*}

  Falls $A$ \emph{lokaler} Hauptidealring ist (also diskreter
  Bewertungsring, der kein Körper ist), dann:
  \[
    \Spec A=\{x,\eta\},\ \mathfrak{p}_{x}\text{ max. Ideal},\
    \mathfrak{p}_{\eta}=(0)
  \]

  $\{\eta\}$ einzige nicht-???  offene Menge.
\end{example}
 
\begin{example}
  Sei $k$ algebraisch abgeschlossener Körper. Affine Varietäten
  $V\leftrightarrow$ endlich erzeugte $k$-Algebren $A$.

  $V=$\{max. Ideale in $A$\} $\subseteq\Spec(A)$

  Topologie auf $V$ ist von $\Spec(A)$ induziert.
\end{example}
 
\begin{example}
  Sei $R$ Hauptidealring, $A=R[T]$, $X=\Spec(A)$. $R$ faktoriell: Gauß
  $\Rightarrow R[T]$ faktoriell mit Primidealen:
  \begin{enumerate}
  \item $p\in R$ prim
    \begin{proof}
      $p\in R$ prim $\Rightarrow R/pR$ Körper. Nach 12.1 gilt:
      \[
        \overline{\pR[T]}=V(pR[T])\cong\Spec\left(R/pR[T]\right)
      \]
      Hauptidealring mit unendlich vielen Elementen. Damit ist $pR[T]$
      \emph{nicht} maximal, sondern
      \[ V(pR[T])={pR[T], (f,p), f\in R[T]
          \text{ mit } \overline{f}\in R/p[T]
        \text{irreduzibel}}
      \]
    \end{proof}
  \item $f\in R[T]$ primitives Polynom, irreduzibel in $\Quot(R)[T]$
    \begin{proof}
      Sei $f$ prim, irreduzibles Polynom.
      \begin{casenv}
      \item $l(f)\in R^{\times} \Rightarrow$ (Division mit Rest)
        $R\subset R[T]/pR[T]$ ist eine ganze Ringerweiterung
        endlich erzeugter freier $R$-Moduln vom Rang $\deg(f)$.
        Angenommen, $fR[T]$ ist maximal. Dann ist $R[T]/fR[T]$ ein Körper,
        also $R$ ein Körper. Widerspruch.
      \item Andernfalls kann $fR[T]$ ein maximales Ideal sein. $R$ habe nur
        endlich viele Primelemente.
        \[
          0\neq a_{j} = \prod p\in R,\ f:=aT-1
        \]
        Es folgt:
        \[
          R[T]/fR[T]\cong R[a^{-1}]=\Quot(R)
        \]
        also $fR[T]$ maximal.
      \end{casenv}
    \end{proof}
  \end{enumerate}
\end{example}

%%% Local Variables:
%%% mode: latex
%%% TeX-master: "../AlgGeo1"
%%% End:


\section*{Exkursion über Garben}

Bisher: 
\begin{align}
X \text{affin alg. Menge} \longmapsto \Gamma(X) = \Hom(X, \mathbb{A}^{1})
\end{align}
Jetzt:
\begin{align}
\Spec A \longmapsfrom A
\end{align} 
d.h. $A$ soll den Funktionen auf $\Spec A$ entsprechen.
Für $x \in \Spec A$ definiert man $ev_{x} : A \to \kappa_A(x) := A_{\mathfrak{p}_x} / \mathfrak{p}_{x}A_{\mathfrak{p}_{x}} \cong \Quot(A/\mathfrak{p}_{x})$ durch $f \mapsto f(x) := ev_{x}(f) := f \mod \mathfrak{p}_{x}$.
Mit dieser Definition folgt insbesondere $D(f) = \{ x \in \Spec A \mid f(x) \neq 0\}$. Da $x \mapsto f(x)$ keine Funktion im engeren Sinne ist, können wir diese Konstruktion nicht als System von Funktionen auffassen.\\
Wichtige Aussagen: Restriktion $+$ Verklebung $\rightsquigarrow$ Garben.

\section{Prägarben und Garben}
\label{sec:garben}

\begin{defn}
\label{def:praegarbe}
Sei $X$ ein topologischer Raum. \\
$(i)$ Eine $\textbf{Prägarbe}\  \sheaf{F}$ auf X besteht aus den folgenden Daten:
\begin{itemize}
	\item eine Menge $\sheaf{F}(U)$ für jede offene Teilmenge $U \subseteq X$
	\item Eine $\textbf{Restriktionsabbildung} $ $res^{V}_{U} : \sheaf{F}(V) \to \sheaf{F}(U)$ für jedes Paar $U \subseteq V$ offen in $X$, so dass:
	\begin{itemize}
		\item $res^{U}_{U} = \id_{\sheaf{F}(U)}$
		\item $res^{W}_{U} = res^{V}_{U} \circ res^{W}_{V}$ für $U \subseteq V \subseteq W$ offen in $X$
	\end{itemize}
\end{itemize}
$(ii)$ Ein $\textbf{Morphismus von Prägarben}$ $\phi : \sheaf{F} \to \sheaf{G}$ ist eine Familie von Abbildungen $\{\phi_{U} : \sheaf{F}(U) \to \sheaf{G}(U) \mid U \subseteq X \text{ offen }\}$, so dass für alle Paare $U \subseteq V$ offen in $X$, das folgende Diagramm kommutiert:
\[
\xymatrix{\sheaf{F}(V)\ar^{\phi_V}@{->}[r]\ar^{res^{V}_{U}}@{->}[d]  & \sheaf{G}(V) \ar[d]^{res^{V}_{U}}\\
	\sheaf{F}(U)\ar^{\phi_{U}}@{->}[r] & \sheaf{G}(U) }
\]
\underline{Notation:} $U \subseteq V$, $s \in \mathcal{F}(V)$, dann: $s|_{U} := res^{V}_{U}(s)$.\\ Die Elemente in $\sheaf{F}(U)$ heißen $\textbf{Schnitte von } \sheaf{F} \textbf{ über }U$, $\Gamma(U, \sheaf{F}) := \sheaf{F}(U)$.
\end{defn}

Alternative Beschreibung:\\
$\ouv_{X}$: Kategorie offener Mengen von $X$ mit $\Hom(U, V) := \begin{cases} \emptyset, \text{ falls } U \not\subseteq V \\ \{ U \to V\}, \text{ falls } U \subseteq V \end{cases}$.
Eine $\textbf{Prägarbe auf }X$ ist ein kontravarianter Funktor $\mathcal{F} : \ouv_{X} \to \set$. Ersetzt man $\set$ durch eine Kategorie $\cat{C}$, so bekommt man Prägarben $\textbf{mit Werten in } \cat{C}$.
Ein Morphismus von Prägarben $\sheaf{F} \to \sheaf{G}$ ist eine natürliche Transformation $\sheaf{F} \implies \sheaf{G}$.\\
\\
Für eine Prägarbe $\sheaf{F}$ auf $X$, $U \subseteq X$ offen und $U = \bigcup_{i} U_{i}$ eine \underline{offene} Überdeckung von $U$, definiere:
\begin{align}
\rho : \sheaf{F}(U) \to \prod_{i}\sheaf{F}(U_{i}), s \mapsto (s|_{U_{i}})_{i}\\
b : \prod_{i} \sheaf{F}(U_{i}) \to \prod_{(i, j)} \sheaf{F}(U_{i} \cap U_{j}), (s_{i})_{i} \mapsto (s_{i}|_{U_{i} \cap U_{j}})_{(i,j)}\\
b' : \prod_{i} \sheaf{F}(U_{i}) \to \prod_{(i, j)} \sheaf{F}(U_{i} \cap U_{j}), (s_{i})_{i} \mapsto (s_{j}|_{U_{i} \cap U_{j}})_{(i,j)}
\end{align}

\begin{defn}
\label{def:garbe}
$(i)$ Eine Prägarbe $\sheaf{F}$ auf $X$ heißt $\textbf{Garbe}$, falls für alle offenen Teilmengen $U \subset X$ und alle offenen Überdeckungen $U = \bigcup_{i} U_{i}$ wie oben gilt:\\
\[
(Sh)\ \xymatrixcolsep{3pc}\xymatrix{\sheaf{F}(U)\ar^-\rho@{->}[r] & \prod_{i}\sheaf{F}(U_{i}) \ar^-b@<+.5ex>[r] \ar_-{b'}@<-.5ex>[r] & \prod_{(i, j)}\sheaf{F}(U_{i} \cap U_{j})}
\]d.h. $\rho$ ist injektiv und $\im\rho = \{ s \in \prod_{i} \sheaf{F}(U_{i}) \mid b(s) = b'(s)\}$, mit anderen Worten: $(\sheaf{F}(U), \rho)$ ist \textbf{Equalizer} von $b$ und $b'$.\\
Dabei ist $(Sh)$ äquivalent zu:
\begin{enumerate}
	\item[(Sh0)]   $\sheaf{F}(\emptyset)$ ist finales Objekt.
	\item[(Sh1)] Gilt für $s,s' \in \sheaf{F}(U)$  $s|_{U_{i}} = s'|_{U_{i}}$ für alle $i$, so folgt $s = s'$.
	\item[(Sh2)] Zu jeder Familie $(s_{i})_{i} \in \prod_{i}\sheaf{F}(U_{i})$ mit $s_{i}|_{U_{i} \cap U_{j}} = s_{j}|_{U_{i} \cap U_{j}}$ existiert ein $s \in \sheaf{F}(U)$ mit $s|_{U_{i}} = s_{i}$.
\end{enumerate}
$(ii)$ Ein \textbf{Morphismus von Garben} ist ein Morphismus der unterliegenden Prägarben.\\
\\
Wir erhalten die Kategorie $\psh_{X}(\set)$ der Mengenwertigen Prägarben auf $X$ und die \underline{volle} Unterkategorie $\sh_{X}(\set)$ der Mengenwertigen Garben auf $X$.
Analog erhalten wir Garben von abelschen Gruppen, Ringen, $R$-Moduln und $R$-Algebren.
\end{defn}

\begin{rem}
\label{rem:pathologien-garben}
\begin{enumerate}
	\item $\sheaf{F} \in \sh \implies \Gamma(\emptyset, \sheaf{F})$ ist einpunktig (wegen $(Sh)$ für die leere Überdeckung)
	\item $X = \{ pt \} \implies \sheaf{F}$ auf $X$ ist eindeutig durch $\sheaf{F}(X)$ bestimmt
\end{enumerate}
\end{rem}

%% TODO: Insert Abschnitt zu Limiten hier

\begin{example}
\label{bsp:beispiele-von-garben}	
\begin{enumerate}
	\item $\sheaf{F} \in \psh_{X}, U \subseteq X$ offen $\implies$ $\sheaf{F}|_{U} \in \psh_{U}$ mit $\Gamma(V, \sheaf{F}|_{U}) := \Gamma(V, \sheaf{F})$. Ist $\sheaf{F} \in \sh_{X}$, so ist $\sheaf{F}|_{U} \in \sh_{U}$. 
	\item Für $X,Y$ top. Räume definiert $\sheaf{F}$ gegeben durch $\Gamma(U, \sheaf{F}) := \mathcal{C}(U, Y) = \{ f : U \to Y \mid \ f \text{ stetig}\}$ und $res^{U}_{V} : f \mapsto f|_{V}$ eine Garbe.
	\item $k$ ein Körper, $(X,\sheaf{O}_X)$ ein Raum mit Funktionen$/_k$ $\implies$ $\sheaf{O}_{X}$ ist Garbe von $k$-Algebren auf $X$.
	\item Für einen top. Raum $X$ definiert $\sheaf{F}(U) := \{ f: U \to \mathbb R \mid f \text{ stetig und beschränkt}\}$ eine Prägarbe auf $X$, im Allgemeinen aber keine Garbe.
\end{enumerate}
\end{example}

Sei $\mathcal{B}$ eine Basis der Topologie von $X$ und $\sheaf{F} \in \sh_{X}$. Sei für $V \subseteq X$ offen $\mathcal{B}_{V} := \{ U \in \mathcal{B} \mid U \subseteq V\}$. Dann folgt wegen $(Sh)$:\\
\[
\sheaf{F}(V) \cong_{(\dagger)} \{ (s_{U})_{U} \in \prod_{U \in \mathcal{B}_{V}} \sheaf{F}(U) \mid \forall U' \subseteq U \in \mathcal{B}_{V}: s_{U}|_{U'} = s_{U'}\} \cong \varprojlim_{U \in \mathcal{B}_{V}} \sheaf{F}(U)
\]d.h. $\sheaf{F}$ ist bereits eindeutig durch die Schnitte auf einer Basis von $X$ bestimmt.\\
$(\dagger):$ einfache Folgerung aus $(Sh1)$.\\
\\
Eine \textbf{Prägarbe auf } $\mathcal{B}$ ist ein kontravarianter Funktor $\sheaf{F} : \mathcal{B} \to \set$. Jedes solche $\sheaf{F}$ induziert eine Prägarbe $\overline{\sheaf{F}}^{X}$ auf $X$ durch $\overline{\sheaf{F}}^{X}(V) := \varprojlim_{U \in \mathcal{B}_{V}} \sheaf{F}(U)$.
Für $U \in \mathcal{B}$ gilt dann $\overline{\sheaf{F}}^{X}(U) = \varprojlim_{U' \in \mathcal{B}_{U}}\sheaf{F}(U') = \sheaf{F}(U)$, da $U$ initial in $\mathcal{B}_{U}$.\\ Ein \textbf{Morphismus von Prägarben auf } $\mathcal{B}$ ist wieder ein Morphismus von Funktoren.

\begin{prop}
\label{prop:garbe-auf-basis}
$\overline{\sheaf{F}}^{X}$ ist eine Garbe $\iff \sheaf{F}$ erfüllt $(Sh)$  für alle $U \in \mathcal{B}$ und Überdeckungen $U = \bigcup_{i} U_{i}$ mit $U_{i} \in \mathcal{B}$.\\
In diesem Fall heißt $\sheaf{F}$ \textbf{Garbe auf} $\mathcal{B}$. 
\end{prop}
\begin{proof}
	Im Folgenden schreiben wir $\overline{\sheaf{F}} := \overline{\sheaf{F}}^{X}$.
\begin{itemize}
	\item [,,$\Rightarrow$``:] $\overline{\sheaf{F}}^{X}(U) = \sheaf{F}(U)$ für alle $U \in \mathcal{B}$.
	\item [,,$\Leftarrow$``:] Sei $U \subseteq X$ offen und $U = \bigcup_{i} U_{i}$ eine offene Überdeckung von $U$ in $X$.\\
	\underline{$(Sh1)$}:
	\[
	\xymatrix {
		\varprojlim_{B \in \mathcal{B}_{U}} \sheaf{F}(B) = \overline{\sheaf{F}}(U)\ar@{^{(}->}[r] & \prod_{i} \overline{\sheaf{F}}(U_{i}) \ar@{^{(}->}[r]  & \prod_{i} \prod_{B \in \mathcal{B}_{U_{i}}} \sheaf{F}(B)\\
		s = (s_{B})_{B}, s' = (s'_{B})_{B} \ar@{|->}[r] & s|_{U_i} = s'|_{U_i} \ar@{|->}[r] & ( (s_{B})_{B \in \mathcal{B}_{U_{i}}} )_{i} = ( (s'_{B})_{B \in \mathcal{B}_{U_{i}}} )_{i} \ \ \ \ \ (\dagger)
	}
	\]
	Behauptung: $\forall B \in \mathcal{B}_{U}: \ s_{B} = s'_{B}$, d.h. $s = s'$. \underline{denn:} \\
	\[
	\xymatrix {
	  \sheaf{F}(B) \ar@{^{(}->}[r] & \prod_{i} \prod_{B' \in \mathcal{B}_{U_i \cap B}} \sheaf{F}(B') 
    } \]  ist injektiv nach $(Sh1)$ für $\sheaf{F}$ auf $\mathcal{B}$.\\
    $s_{B}$ und $s'_{B}$ haben gleiches Bild wegen $(\dagger)$.\\
    \underline{$(Sh2)$}:
    \[ \xymatrix{
    \overline{\sheaf{F}}(U) \ar^{\rho}@{->}[r] & \prod_{i} \overline{\sheaf{F}}(U_{i}) \ar@{->}[r] & \prod_{(i, j)} \overline{\sheaf{F}}(U_{i} \cap U_{j})
    }\]
    \begin{itemize}
    	\item[,,$\overline{\sheaf{F}}(U) \subseteq \ker$``]: $\rho(s) = ((s_{B})_{B \in \mathcal{B}_{U_{i}}})_{i}$. Sei $T \in \mathcal{B}_{U_{i} \cap U_{j}} \subseteq \mathcal{B}_{U_{i}}$, $V \in \mathcal{B}_{U_{i}}$ und $W \in \mathcal{B}_{U_{j}}$. Dann folgt: \\
    	$s_{V}|_{T} = s_{T} = s_{W}|_{T} \implies$ Behauptung.
    	\item[,,$\overline{\sheaf{F}}(U)\supseteq \ker$``]: Sei $(s_{i})_{i} \in \prod_{i} \overline{\sheaf{F}}(U_{i})$ mit $b((s_{i})_{i}) = b'((s_{i})_{i})$.\\
    	\underline{Gesucht:} $s = (s_B)_{B} \in \overline{\sheaf{F}}(U)$ mit $s|_{U_{i}} = s_{i}$.\\
    	Es gilt:\\
    \[\xymatrixcolsep{3pc}\xymatrix{
     \sheaf{F}(B) \ar^-\rho@{->}[r] & \prod_{i} \prod_{B' \in \mathcal{B}_{U_{i} \cap B}} \sheaf{F}(B') \ar^-b@<+.5ex>[r] \ar_-{b'}@<-.5ex>[r] & \prod_{(i, j)} \prod_{\mathcal{B}_{U_{i} \cap B} \times \mathcal{B}_{U_{j} \cap B}} \sheaf{F}(V \cap W)
    }\] ist exakt, d.h. ein Equalizer-Diagramm. Konstruiere damit $s_{B} \in \sheaf{F}(B)$, welche kompatibles System bilden. $s = (s_B)_{B}$ ist dann das gesuchte Element in $\overline{\sheaf{F}}(U)$. 	
    \end{itemize}
\end{itemize}
\end{proof}



\section{Halme von Garben}
\label{sec:halme}

Für $x \in X$ und $\sheaf{F} \in \psh_{X}$ ist $(\sheaf{F}(V), res^{V}_{U})_{x \in U \subseteq X \text{offen}}$ ein \underline{filtriertes} induktives System.\\
\underline{filtriert}:\\
$\forall U,V \subseteq X$ offen $\exists W \subseteq U,V$ offen. (z.B. $W = U \cap V$).\\

\begin{defn}
\label{def:halm}
Der induktive Limes (oder auch Colimes) $\sheaf{F}_{x} := \varinjlim_{x \in U} \sheaf{F}(U)$ heißt \textbf{Halm} von $\sheaf{F}$ in $x$. Für $x \in U \subseteq X$ offen hat man einen kanonischen Morphismus $\pi_{U} : \sheaf{F}(U) \to \sheaf{F}_{x}$. Das Bild eines Schnittes $s \in \sheaf{F}(U)$ unter $\pi_{U}$ heißt \textbf{Keim} von $s$ in $x$ und wird mit $s_{x}$ bezeichnet.\\
\\
Ein Morphismus von Prägarben $\varphi: \sheaf{F} \to \sheaf{G}$ induziert eine Abbildung $\varphi_{x} = \varinjlim_{x \in U} \varphi_{U} : \sheaf{F}_{x} \to \sheaf{G}_{x}$ von Halmen in $x$.
\end{defn}

\begin{example}
\label{bsp:einige-halme}
$z_0 \in X := \CC$, $\sheaf{O}_{\CC}$: Garbe der holomorphen Funktionen auf $\CC$. Dann gilt: $(U, f) \sim (V,g) \iff$ $f$ und $g$ haben dieselbe Taylor-Entwicklung um $z_0$. \\$\implies$ $\sheaf{O}_{\CC, z_0} = \CC\{\{z_0 \}\}$ ist der Ring der Potenzreihen um $z_0$ mit positivem Konvergenzradius.
\end{example}

\begin{prop}
\label{prop:charakterisierung-morphismen-halme}
Seien $X$ ein top. Raum, $\sheaf{F}, \sheaf{G} \in \psh_{X}$ und $\xymatrix{\sheaf{F} \ar^{\varphi}@<+.5ex>[r] \ar_{\psi}@<-.5ex>[r] & \sheaf{G}}$ zwei Morphismen.
\begin{enumerate}
	\item[(1)] Ist $\sheaf{F}$ eine Garbe, so gilt $\varphi_{x} : \sheaf{F}_{x} \to \sheaf{G}_{x}$ ist injektiv für alle $x \in X \iff \varphi_{U} : \sheaf{F}(U) \to \sheaf{G}(U)$ ist injektiv für alle $U \subseteq X$ offen.
	\item[(2)] Sind $\sheaf{F}$ und $\sheaf{G}$ Garben, so gilt:
	\begin{enumerate}
		\item[(a)] $\varphi_{x}$ ist bijektiv für alle $x \in X \iff \varphi_{U}$ ist bijektiv für alle $U \subseteq X$ offen.
		\item[(b)] $\varphi = \psi \iff \varphi_{x} = \psi_{x}$ für alle $x \in X$.
	\end{enumerate}
\end{enumerate}
\end{prop}
\begin{proof}
Für $U \subseteq X$ offen ist
\[
\xymatrix{
  \sheaf{F}(U) \ar@{^{(}->}[r] & \prod_{x \in U} \sheaf{F}_{x}\\
  s \ar@{|->}[r] & (s_{x})_{x \in U}
}
\] injektiv, \underline{denn}:\\
Seien $s,t \in \sheaf{F}(U)$ mit $s_x = t_x$ für alle $x \in U$. Dann gibt es für jedes $x \in U$ eine offene Umgebung $x \in V_x \subseteq U$ s.d. $s|_{V_x} = t|_{V_x}$.\\
$(Sh1) \implies s = t$.\\
Wir erhalten ein kommutatives Diagramm \\
\[
\xymatrix
{
\sheaf{F}(U) \ar@{^{(}->}[r] \ar^{\varphi_U}@{->}[d] & \prod_{x \in U} \sheaf{F}_{x} \ar^{\prod_{x}\varphi_x}@{->}[d] \\
\sheaf{G}(U) \ar@{->}[r] & \prod_{x \in U} \sheaf{G}_{x}
}
\] welches $``(1) \Rightarrow``$ und $(2)(b)$ impliziert.\\
\\
\\
Allgemein gilt: 
\begin{enumerate}
	\item[$(i)$] Filtrierte Colimiten injektiver Abbildungen sind injektiv, d.h. $``(1) \Leftarrow ``$ gilt.
    \item[$(ii)$] Colimiten surjektiver Abbildungen sind surjektiv, d.h. $``(2)(a)\Leftarrow ``$ gilt
\end{enumerate}
Zu $``(2)(a)\Rightarrow ``$: reicht z.z.: Bijektivität von $\varphi_x$ impliziert Surjektivität von $\varphi_U$.\\
Sei dazu $t \in \sheaf{G}(U)$. Wähle für alle $x \in U$ eine offene Umgebung $x \in U^{x} \subseteq U$ und $s^{x} \in \sheaf{F}(U^x)$ so dass $(\varphi_{U^x}(s^x))_x = t_x$.\\
$\implies \exists x \in V^x \subseteq U^x$ offen mit $\varphi_{V^x}(s^x|_{V^x}) = t|_{V^x}$. \\
Da $U = \bigcup_x{V^x}$ offene Überdeckung, gilt für alle $x,y \in U$:\\
\[
\varphi_{V^y \cap V^x}(s^x|_{V^y \cap V^y}) = t|_{V^y \cap V^x} = \varphi_{V^y \cap V^x}(s^y|_{V^y \cap V^x})
\]
$\varphi_{U}$ injektiv nach $(1)$ $\implies$ $s^x|_{V^y \cap V^y} = s^y|_{V^y \cap V^y}$.\\
$(Sh2) \implies \exists s \in \sheaf{F}(U)$ mit $s|_{V^x} = s^x$ für alle $x \in U$. \\
$\implies$ $\varphi_U(s)_x = [(V^x, \varphi_{V^x}(s|_{V^x}))] = [(V^x, t|_{V^x})] = t_x \implies \varphi_{U}(s) = t$.
\end{proof}

\begin{defn}
\label{def:injektive-und-surjektive-garbenmorphismen}
Ein Morphismus $\sheaf{F} \to \sheaf{G}$ von Garben heißt \textbf{injektiv/ surjektiv/ bijektiv} $:\iff$ $\forall x \in X: \sheaf{F}_x \to \sheaf{G}_x$ ist injektiv/ surjektiv/ bijektiv.
\end{defn}

\begin{rem}
\label{rem:charakterisierung-surjektiv}
$\varphi : \sheaf{F} \to \sheaf{G}$ ist surjektiv gdw. für alle $t \in \sheaf{F}(U)$ eine offene Überdeckung $U = \bigcup_i U_i$ existiert und $s_i \in \sheaf{F}(U_i)$ s.d. $\varphi_{U_i}(s_i) = t|_{U_i}$, d.h. \underline{lokal} findet man stets ein Urbild.\\
\textbf{Warnung:} Aus $\varphi$ surjektiv folgt nicht $\varphi_U$ surjektiv für alle $U \subseteq X$ offen.	
\end{rem}

\section{Die zu einer Prägarbe assoziierte Garbe}
\label{sec:vergarbung}

\begin{defn}
\label{def:vergarbung}
Sei $\sheaf{F}$ eine Prägarbe auf einem top. Raum $X$. Eine \textbf{Vergarbung} (auch Garbifizierung/ assoziierte Garbe) von $\sheaf{F}$ ist eine Garbe $\sheaf{F}^{sh}$ auf $X$ zusammen mit einem Morphismus $\iota : \sheaf{F} \to V(\sheaf{F}^{sh})$ von Prägarben, so dass gilt:\\
\[\xymatrix{  
\Mor_{\psh_{X}}(\sheaf{F}, V(\sheaf{G})) \ar^-\cong@{->}[r] & \Mor_{\sh_{X}}(\sheaf{F}^{sh}, \sheaf{G})  \\
\varphi \circ \iota & \ar@{|->}[l] \varphi
}\] Hierbei bezeichne $V : \sh_{X} \to \psh_{X}$ den Vergissfunktor.\\
Durch diese Eigenschaft ist $(\sheaf{F}^{sh}, \iota)$ eindeutig bis auf eindeutigen Isomorphismus bestimmt.\\
Ferner gilt:
\begin{enumerate}
	\item[(0)] Es existiert eine Vergarbung $\iota : \sheaf{F} \to \sheaf{F}^{sh}$
	\item[(1)] $\iota$ wie oben induziert einen Isomorphismus $\iota_x : \sheaf{F}_x \to \sheaf{F}^{sh}_x$ auf Halmen für alle $x\in X$.
	\item[(2)] Für jede Prägarbe $\sheaf{G}$ auf $X$ und jeden Morphismus $\varphi: \sheaf{F} \to \sheaf{G}$ existiert genau ein Morphismus $\varphi^{sh}: \sheaf{F}^{sh} \to \sheaf{G}^{sh}$ s.d. folgendes Diagramm kommutiert:
	\[
	\xymatrix
	{
	\sheaf{F} \ar^{\iota_{\sheaf{F}}}@{->}[r] \ar^{\varphi}@{->}[d] & \sheaf{F}^{sh} \ar^{\varphi^{sh}}@{->}[d] \\
	\sheaf{G} \ar^{\iota_{\sheaf{G}}}@{->}[r] & \sheaf{G}^{sh}
    }
	\] d.h. $\psh_{X} \to \sh_{X}, \sheaf{F} \mapsto \sheaf{F}^{sh}$ ist ein Funktor, linksadjungiert zum Vergissfunktor $V$.
\end{enumerate}
\end{defn}
\begin{proof}
\underline{Existenz}:\\
$\sheaf{F}^{sh}(U) := \{ (s_x)_{x} \in \prod_{x \in U} \sheaf{F}_{x} \mid \forall x \in U: \ \exists x \in U^x \subseteq U \text{ offen und } t \in \sheaf{F}(U^x): \ \forall y \in U^x: t_x = s_x \}$\\
``Keime, die lokal Schnitte von $\sheaf{F}$ sind`` - $(Sh2)$ erzwingt dies.\\
Für $U \subseteq V$ ist $res^V_U$ induziert von:\\
\[
\xymatrix
{
\sheaf{F}^{sh}(V) \ar^-{res^V_U}@{-->}[r] \ar@{^{(}->}[d] & \sheaf{F}^{sh}(U) \ar@{^{(}->}[d] \\
\prod_{x\in V} \sheaf{F}_x \ar^-{proj.}@{->}[r] & \prod_{x\in U}\sheaf{F}_x
}
\]
\end{proof}

%% TODO:
%% - Limiten einfügen in Garben

\section{Direktes und inverses Bild von Garben}
\label{sec:garben-direktes-inverses-bild}

Sei $f:X\rightarrow Y$ stetige Abbildung topologischer Räume, $\mathcal{F}$
eine Prägarbe auf $X$. Ziel: $f_{\ast}\mathcal{F}$ Prägarbe auf
$Y$, das direkte Bild von $\mathcal{F}$ unter $f$. Definiere $(f_{\ast}\mathcal{F})(V):=\mathcal{F}(f^{-1}(V))$
mit Restriktionsabbildung von $\mathcal{F}$ ($V_{1}\subseteq V_{2}:$
$s\in f_{\ast}\mathcal{F}(V_{2})\rightarrow s|_{V_{1}}=\mathcal{F}res_{f^{-1}(V_{1})}^{f^{-1}(V_{2})}$).
\begin{align*}
  f_{\ast}:PSh(X) & \longrightarrow PSh(Y)\\
  \mathcal{F} & \longmapsto f_{\ast}\mathcal{F}\\
  \mathcal{F}\overset{\varphi}{\rightarrow}\mathcal{G} & \longmapsto f_{\ast}(U):f_{\ast}\mathcal{F}\rightarrow f_{\ast}\mathcal{G}
\end{align*}

ist Funktor via $(f_{\ast}\varphi)_{V}=\varphi_{f^{-1}(V)}$.
\begin{rem}[28]
  \mbox{}
  \begin{enumerate}
  \item $\mathcal{F}$ Garbe auf $X$ $\Longrightarrow f_{\ast}\mathcal{F}$
    Garbe auf $X$, d.h. $f_{\ast}:Sh(X)\rightarrow Sh(Y)$.
  \item Ist $g:Y\rightarrow Z$ eine weitere stetige Abbildung topologischer
    Räume, so existiert ein offensichtlicher Isomorphismus $g_{\ast}\circ(f_{\ast}\mathcal{F})=(g\circ f)_{\ast}\mathcal{F}$,
    funktoriell in $\mathcal{F}$.
  \end{enumerate}
  \medskip{}
\end{rem}

Nun sei $\mathcal{G}$ eine Prägarbe auf $Y$.

\textbf{Ziel:} Definiere $f^{+}\mathcal{G}$ Prägarbe auf $X$. $f^{-1}\mathcal{G}=\widetilde{f^{+}\mathcal{G}}$
Garbe auf $X$, \textbf{Inverses Bild zu $\mathcal{G}$ unter $f$}
via 
\[
(f^{+}\mathcal{G})(U):=\underset{\underset{Y\supseteq V\supseteq f(U)}{\longrightarrow}}{\lim}\mathcal{G}(V)
\]

mit induzierte Restriktionsabbildung.\medskip{}

\textbf{Warnung:} $\mathcal{G}$ Garbe auf $Y$ $\leadsto f^{+}\mathcal{G}$
im Allgemeinen keine Garbe auf $X$. Falls $f:X\hookrightarrow Y$
Inklusion, $\mathcal{G}|_{X}:=f^{-1}\mathcal{G}$. Ist $X\subseteq Y$
offen stimmt $\mathcal{G}|_{X}$ mit der Einschränkung aus Beispiel
19 überein (cofinales Objekt). $\leadsto f^{-1}:PSh(Y)\rightarrow Sh(X)$
Funktor.

$g:Y\xrightarrow{\text{stetig}}Z$, $\mathcal{H}$ Prägarbe auf $Z$,
$U\subseteq X$ offen. 
\[
Z\underset{\text{offen}}{\supseteq}W\supseteq g(f(U))\Longleftrightarrow W\supseteq g(V)
\]

für ein $f(U)\subseteq V\subseteq Y$ offen. 
\begin{align*}
  \underset{\longrightarrow}{\lim}\underset{\longrightarrow}{\lim}=\underset{\longrightarrow}{\lim}\Longrightarrow f^{+}(g^{+}\mathcal{H}) & =(g\circ f)^{+}\mathcal{H}\quad(*)\\
  \Longrightarrow f^{-1}(g^{-1}\mathcal{H}) & =(g\circ f)^{-1}\mathcal{H}
\end{align*}

\begin{example}
  $\imath:\{x\}\rightarrow X$ Inklusion, $\mathcal{F}$ Prägarbe auf
  $X$. $\Longrightarrow\imath^{-1}(\mathcal{F})=\mathcal{F}_{x}$ per
  Definition. $(\ast)\Longrightarrow$
  \[
  \begin{array}{ccc}
    (f^{-1}\mathcal{G})_{x} & = & \mathcal{G}_{f(x)}\\
    \shortparallel &  & \shortparallel\\
    \imath^{-1}\circ(f^{-1}\mathcal{G}) & = & (f\circ\imath)^{-1}\mathcal{G}
  \end{array}
  \]
\end{example}

\begin{prop}[29]
  Für $f:X\rightarrow Y$ stetig sind die Funktionen $f_{\ast}$ und
  $f^{-1}$ zueinander adjungiert, d.h. für $\mathcal{F}$ Garbe auf
  $X$, $\mathcal{G}$ Prägarbe auf $Y$ existiert eine bijektion
  \begin{align*}
    \hom_{Sh(x)}(f^{-1}\mathcal{G},\mathcal{F}) & \longleftrightarrow\hom_{Psh(Y)}(\mathcal{G},f_{\ast}\mathcal{F})\\
    \varphi & \longmapsto\varphi^{\flat}\\
    \psi^{\sharp} & \longmapsfrom\psi
  \end{align*}

  funktoriell in $\mathcal{F}$ und $\mathcal{G}$.
\end{prop}

\begin{proof}
  $\varphi:f^{-1}\mathcal{G}\rightarrow\mathcal{F}$ Morphismus von
  Garben auf $X$. $t\in\mathcal{G}(V)$, $V\subseteq Y$ offen
  \begin{align*}
    \mathcal{G}(V) & \rightarrow f^{+}\mathcal{G}(f^{-1}(V))\xrightarrow{\imath_{f^{+}\mathcal{G}}}f^{-1}\mathcal{G}(f^{-1}(V))\xrightarrow{\varphi_{f^{-1}(V)}}\mathcal{F}(f^{-1}(V))=f_{\ast}\mathcal{F}(V)\\
    & \phantom{\rightarrow\ }\shortparallel\underset{\underset{Y\supseteq W\supseteq ff^{-1}(V)\subseteq V}{\longrightarrow}}{\lim}\\
    t & \mapsto\varphi_{V}^{\flat}(t)
  \end{align*}

  Definition von $\psi^{\#}$. $\mathcal{G}\xrightarrow{\psi}f_{\ast}\mathcal{F}$
  Morphismus von Prägarben auf . Wir definieren $\psi^{\#}:f^{+}\mathcal{G}\rightarrow\mathcal{F}$,
  welches dann $\psi^{\#}:f^{-1}\mathcal{G}\rightarrow\mathcal{F}$
  induziert. $U\subseteq X$ offen, $S\subseteq f^{+}\mathcal{G}(U)$,
  $s=[(V,s_{V})]$, $V\supseteq f(U)$, $s_{V}\in\mathcal{G}(V)$. $\Longrightarrow f^{-1}(V)\supseteq U$.
  \[
  \xymatrix{\psi_{V}(s_{V})\in f_{\ast}\mathcal{F}(V)\ar@{=}[r]\ar@{|->}[rd] & \mathcal{F}(f^{-1}(V))\ar[d]\\
    & \psi_{U}^{\#}(s)\in\mathcal{F}(U)
  }
  \]

  Überprüfe $\varphi^{\flat^{\#}}=\varphi$, $\psi^{\#^{\flat}}=\mathcal{H}$
  und Funktoriell.
\end{proof}
Definition + Proposition 29 verallgemeinern sich zu (Prä)Garben von
Ringen, $R$-Moduln, $R$-Algebren.

\textbf{Beschreibung} von:
\[
\mathcal{G}_{f(x)}=(f^{-1}\mathcal{G})_{x}\overset{\varphi_{x}}{\longmapsto}\mathcal{F}_{x},\ x\in X
\]

\[
\xymatrix{f(x)\in U\underset{\text{offen}}{\subseteq}Y & \mathcal{G}(U)\ar[r]^{\varphi_{U}^{\flat}}\ar@{-->}[d] & \mathcal{F}(f^{-1}(U))\ar[r] & \mathcal{F}_{x}\\
  \underset{\underset{U}{\longrightarrow}}{\lim} & \mathcal{G}_{f(x)}\ar@{-->}[rru]
}
\]


\chapter{Schemata}
\label{chap:schemata}

\section{Schemata}

\begin{defn}
	Ein Schemata ist ein lokal geringter Raum $(X,\cat{O}_X)$, der eine offene Überdeckung $(U_i)_{i\in I}$
	besitz derart alle lokal geringter Räume $(U_i,\cat{O}_{X|U_i})$ affine Schemata sind.\\
	Für ein Schemata $S$ besitz $\textbf{Sch}/_{S}$ ~~ \textbf{Kategorie der Schemata über} $S$ oder $S$-\textbf{Schemata}\\
	\begin{itemize}
		\item Morphismen $X \rightarrow S$ von Schemata
	\end{itemize}
\end{defn}


\chapter{Faserprodukte}
\label{chap:faserprod}

\section{Der 'Punkte-Funktor'}
(Kontra) Funktor $\forall X \in \textbf{Sch}$
\[
\xymatrix{
	h_X\colon (\textbf{Sch})^\text{op} \ar@{->}[r] \ar@{->}[r]  &  \textbf{Sets}&\\
	S  \ar@{|->}[r] & h_X(S):= \Hom_{\textbf{Sch}}(S,X)\ar@{->}[d]^{f*=h_X(f)}& g \ar@{|->}[d]\\
	T  \ar@{->}[u]^{f} & h_X(T) & g \circ f
}
\]
$h_X(S)$ heißen $S$-wertige Punkte in $X$.\\
\textbf{\underline{Notation}} $X(S)$, $X(R)$, falls $S= \Spec(R)$ 


\section{Yoneda Lemma}
\underline{Ziel}: $h_X$ beschreibt $X$ eindeutig.

\section{Faserprodukt in beliebiger Kategorie}



\newpage{}
\printindex{}
\end{document}
