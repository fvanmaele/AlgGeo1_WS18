\documentclass[12pt,a4paper]{book}
\usepackage[T1]{fontenc}
\usepackage[utf8]{inputenc}
\usepackage{geometry}
\geometry{verbose,tmargin=2cm,bmargin=2cm,lmargin=2cm,rmargin=2cm}
\pagestyle{headings}
\usepackage[ngerman]{babel}
\usepackage{verbatim}
\usepackage{amsmath}
\usepackage{amsthm}
\usepackage{amssymb}
\usepackage{stmaryrd}
\usepackage{makeidx}
\makeindex
\usepackage{setspace}
\usepackage[all]{xy}
\onehalfspacing
\usepackage[bookmarks=true]{hyperref}

%% Theorems (numbered by Part)
\newtheorem{thm}{Theorem}[chapter]
\theoremstyle{definition}
\newtheorem{example}[thm]{Beispiel}
\theoremstyle{definition}
\newtheorem{defn}[thm]{Definition}
\theoremstyle{plain}
\newtheorem{prop}[thm]{Satz}
\theoremstyle{plain}
\newtheorem{cor}[thm]{Korollar}
\theoremstyle{plain}
\newtheorem{lem}[thm]{Lemma}
\theoremstyle{remark}
\newtheorem{rem}[thm]{Bemerkung}
\theoremstyle{plain}

%% Theorems (unnumbered)
\newtheorem*{question*}{Frage}
\theoremstyle{remark}
\newtheorem*{claim*}{Behauptung}
\theoremstyle{definition}
\newtheorem*{example*}{Beispiel}
\theoremstyle{plain}

%% User-specified commands
\DeclareMathOperator{\rad}{rad}
\DeclareMathOperator{\Spec}{Spec}
\DeclareMathOperator{\Quot}{Quot}
\DeclareMathOperator{\im}{\mathrm{im}}
\renewcommand{\labelenumi}{(\roman{enumi})}
\renewcommand{\labelenumii}{\arabic{enumii}.}

\begin{document}

%% Title page
\title{Algebraische Geometrie I}
\author{Prof. Dr. Venjakob}
\maketitle

\tableofcontents{}
\newpage{}

\section*{Literatur}
\begin{itemize}
\item Görtz, Wedhorn. \emph{Algebraic Geometry I}
\item Hartshorne. \emph{Algebraic Geometry}
\item Shafarevich. \emph{Basic Algebraic Geometry 1 \& 2}
\item Grothendieck. \emph{Eléments de géometrie algébrique, EGA I-IV}
\end{itemize}

\paragraph{Kommutative Algebra}
\begin{itemize}
\item Brüske, Ischebeck, Vogel. \emph{Kommutative Algebra}
\item Kunz. \emph{Einführung in die kommutative Algebra und algebraische Geometrie}
\end{itemize}

\chapter{Prä-Varietäten}
\label{chap:prae-varietaeten}


\section{Einführung}
\label{sec:einfuehrung}

\textbf{Algebraische Geometrie}\index{Algebraische Geometrie} kann
man verstehen, als das Studium von Systemen polynomialer Gleichungen
(in mehreren Variabelen). Damit ist die algebraische Geometrie eine
Verallgemeinerung der \textbf{linearen Algebra}, also statt $X$ auch
$X^{n}$, und auch der \textbf{Algebra}, durch Polynome in \emph{mehreren
} Variablen.
\begin{question*}
  Seien $k$ ein (algebraisch abgeschlossener) Körper, und $f_{1},\ldots,f_{m}\in k[T_{1},\ldots,T_{n}]$
  gegeben. Was sind die ``geometrischen Eigenschaften'' der Nullstellenmenge
  \[
    V(f_{1},\ldots,f_{n}):=\{(t_{1},\ldots,t_{n})\in k^{n}\mid f_{i}(t_{1},\ldots,t_{n})=0\ \forall i\}
  \]
\end{question*}
\begin{example}
  \label{bsp:einfuehrung}
  Sei $f=T_{2}^{2}-T_{1}^{2}(T_{1}-1)\in k[T_{1},T_{2}]$. Die Nullstellenmenge
  für $k=\mathbb{R}$ (\emph{aber: }trügerisch, da $\mathbb{R}$ nicht
  algebraisch abgeschlossen!) ist gegeben durch:
\end{example}
\begin{figure}
  \caption{$T_{2}^{2}=T_{1}^{2}(T_{1}-1)=T_{1}^{3}-T_{1}^{2}$}
\end{figure}


\begin{itemize}
\item Dimension 1
\item $(0,0)$ ist singulärer Punkt
\item Alle anderen Punkte besitzen eine eindeutig bestimmte Tangente
\end{itemize}


\begin{figure}[h]
  \label{fig:einfuehrung-glattheit}
  \caption{\textbf{Spitze} und \textbf{Doppelpunkt}}
\end{figure}

Vergleiche mit dem \textbf{Satz über implizite Funktionen}: (Analysis,
Differentialgeometrie) 

$V(f)$ ist lokal diffeomorph zu $\mathbb{R}$ (= reelle Gerade) im
Punkt $(x_{1},x_{2})$ genau dann, wenn die Jacobi-Matrix 
\[
  \left(\frac{\partial f}{\partial T_{1}},\frac{\partial f}{\partial T_{2}}\right)=\left(T_{1}(3T_{1}-2),\ 2T_{2}\right)
\]
Rang 1 in $(x_{1},x_{2})$ hat. Das ist äquivalent dazu, dass $(x_{1},x_{2})\neq(0,0)$.
Dies lässt sich rein formal über beliebigen Grundkörpern \textbf{algebraisch}
formulieren.

\paragraph{Methoden.}

GAGA - Géometrie algébrique, géometrique analytique (Serre)\medskip{}

\begin{tabular}{|c|c|}
  \hline 
  Komplexe Geometrie ($\mathbb{C}$), Differentialgeometrie $(\mathbb{R})$ & Algebraische Geometrie\tabularnewline \\
  \hline
  \hline
  Analytische Hilfsmittel & Kommutative Algebra\tabularnewline \\
  \hline 
\end{tabular}



\section{Die Zariski-Topologie}
\label{sec:zariski-topologie}

\begin{defn}
  \label{def:verschwindungsmenge}
  Sei $M\subseteq k[T_{1},\ldots,T_{n}]=:k[\underline{T}]$ eine Teilmenge.
  Mit
  \[
    V(M) :=\{(t_{1},\ldots,t_{n})\in k^n \mid f(t_{1},\ldots,t_{n})=0\ \boldsymbol{\forall f\in M}\}
  \]

  bezeichnen wir die gemeinsame \textbf{Nullstellen-(Verschwindungs-)Menge}\index{Nullstellen-Menge}
  der Elemente aus $M$. (Manchmal auch $V(f_{i},i\in I)$ statt $V(\{f_{i},i\in I\})$.
\end{defn}

%% TODO:
%% das oben weg!!!
\paragraph{Notation}
Wir schreiben auch $V(f_i, i \in I)$ statt $V(\{f_i \mid i \in I\})$

\subsection{Eigenschaften}
\label{subsec:zariski-topologie-eigenschaften}
\begin{itemize}
\item $V(M)=V(\mathfrak{a})$, wenn $\mathfrak{a}=\langle M\rangle_{k[\underline{T}]}$ das
  \emph{von} $M$ \emph{erzeugte Ideal} in $k[\underline{T}]$ bezeichnet.
\item Da $k[\underline{T}]$ noethersch (Hilbertscher Basissatz) ist, reichen
  stets endlich viele $f_{1},\ldots,f_{n}\in M$:
  \[
    V(M)=V(f_{1},\ldots,f_{n})\qquad\text{falls }\mathfrak{a}=\langle f_{1},\ldots,f_{n}\rangle_{k[\underline{T}]}.
  \]
\item $V(-)$ ist \textbf{inklusionsumkehrend}, $M'\subseteq M\implies V(M)\subseteq V(M')$.
\end{itemize}
\begin{prop}
  \label{propdef:zariski-topologie}
  Die Mengen $V(\mathfrak{a})$, $\mathfrak{a} \unlhd k[\underline{T}]$
  ein Ideal, sind die \textbf{abgeschlossenen} Mengen einer Topologie
  auf $k^{n}$, der sogenannten \textbf{Zariski-Topologie}\index{Zariski-Topologie}.
  \begin{enumerate}
  \item $\emptyset=V\left((1)\right)$, $k^{n}=V(0)$. 
  \item $\bigcap_{i\in I}V(\mathfrak{a}_{i})=V\left(\sum_{i\in I}\mathfrak{a}_{i}\right)$
    für beliebige Familien $(\mathfrak{a}_{i})_{i \in I}$ von Idealen.
  \item $V(\mathfrak{a})\cup V(\mathfrak{a})=V(\mathfrak{ab})$ für $\mathfrak{a},\mathfrak{b}\unlhd k[\underline{T}]$
    Ideale.
  \end{enumerate}
\end{prop}
\begin{proof}
  Übung / Algebra II. 

  \-
\end{proof}




\section{Affine algebraische Mengen}
\label{sec:algebraische-mengen}
\begin{defn}
  \label{def:algebraische-mengen}
  \mbox{}
  \begin{itemize}
  \item $\mathbb{A}^{n}(k)$, der $\textbf{affine Raum der Dimension n}$ (über $k$),
    bezeichne $k^{n}$ mit der Zariski-Topologie.
  \item Abgeschlossene Teilmengen von $\mathbb{A}^{n}(k)$ heißen affine abgeschlossene
    Mengen.
  \end{itemize}
\end{defn}
\begin{example}
  \label{bsp:algebraische-mengen-dim1}
  Da $k[T]$ ein Hauptidealring ist, sind die abgeschlossen Mengen in
  $\mathbb{A}^{1}(k)$: $\emptyset$, $\mathbb{A}^{1}$, Mengen der
  Form $V(f)$, $f\in k[T]\backslash\{k\}$ (endliche Teilmengen).%
  \begin{comment}
    Für $f\in k$ ist $V(f)=\mathbb{A}^{1}$, denn die Einheiten im Polynomring
    $k[T]$ sind gegeben durch $k^{\times}$, und Ideale erzeugt von einer
    Einheit bilden den ganzen Ring. (siehe Algebra 1)
  \end{comment}
  {} Insbesondere sieht man, dass die Zariski-Topologie im Allgemeinen
  nicht Hausdorff ist. 
\end{example}
% 
\begin{example}
  \label{bsp:algebraische-mengen-dim2}
  $\mathbb{A}^{2}(k)$ hat zumindestens als abgeschlossene Mengen:
  \begin{itemize}
  \item $\emptyset$, $\mathbb{A}^{2}$;
  \item Einpunktige Mengen: $\{(x_{1},x_{2})\}=V(T_{1}-x_{1},T_{2}-x_{2})$;
  \item $V(f)$, $f\in k[T_{1},T_{2}]$ irreduzibel. 
  \end{itemize}
  Ferner alle endlichen Vereinigungen dieser Liste. (Dies sind in der
  Tat alle, denn später sehen wir: ``irreduzible'' abgeschlossene
  Mengen entsprechen den \emph{Primidealen}, und $k[T_{1},T_{2}]$ hat
  ``Krull-Dimension $2$''.)
\end{example}




\section{Der Hilbertsche Nullstellensatz}
\label{sec:nullstellensatz}
\begin{prop}
  \label{prop:nullstellensatz}
  Sei $K$ ein (nicht notwendigerweise algebraisch abgeschlossener) Körper,
  und $A$ eine endlich erzeugte $K$-Algebra. Dann ist $A$ Jacobson'sch,
  d.h. für jedes Primideal $\mathfrak{p}\unlhd A$ gilt:
  \[
    \mathfrak{p}=\bigcap_{\mathfrak{m}\supseteq\mathfrak{p}}\mathfrak{m},\quad\mathfrak{m}\text{ maximales Ideal}
  \]

  Ist $\mathfrak{m}\unlhd A$ ein maximales Ideal, so ist die Körpererweiterung
  $K\subseteq A/\mathfrak{m}$ endlich.
\end{prop}
\begin{proof}
  Algebra II / kommutative Algebra.
\end{proof}
\begin{cor}
  \label{cor:nullstellensatz}
  \mbox{}
  \begin{enumerate}
  \item Sei $A$ eine e.e. (endlich erzeugte) $k$-Algebra ($k$ sei algebraisch
    abgeschlossen), $\mathfrak{m}\unlhd A$ ein maximales Ideal. Dann
    ist $A/\mathfrak{m}=k$. 
  \item Jedes maximale Ideal $\mathfrak{m}\unlhd k[\underline{T}]$ ist von der
    Form $\mathfrak{m}=(T_{1}-x_{1},\ldots,T_{n}-x_{n})$ mit $x_{1},\ldots,x_{n}\in k$.
  \item Für ein Ideal  $\mathfrak{a}\unlhd k[\underline{T}]$ gilt:
    \[
      \rad(\mathfrak{a})=\sqrt{\mathfrak{a}}\overset{(i)}{=}\bigcap_{\mathfrak{a}\subseteq\mathfrak{p}\unlhd k[\underline{T}], \mathfrak{p} \text{prim}}\mathfrak{p}\overset{(ii)}{=}\bigcap_{\mathfrak{a}\subseteq\mathfrak{m}\unlhd k[\underline{T}], \mathfrak{m} \text{maximal}}\mathfrak{m}
    \]
  \end{enumerate}
\end{cor}
\begin{proof}
  \mbox{}
  \begin{enumerate}
  \item $k\rightarrow A\rightarrow A/\mathfrak{m}$ ist Isomorphismus,  da
    $k$ keine echte algebraische Körpererweiterung besitzt.
  \item Es ist
    \begin{align*}
      k[T_{1},\ldots,T_{n}] & \twoheadrightarrow k[\underline{T}]/\mathfrak{m}=k\\
      T_{i} & \mapsto x_{i}
    \end{align*}
    surjektiv. Es folgt: $\mathfrak{m}=(T_{1}-x_{1},\ldots,T_{n}-x_{n})$, da letzteres
    bereits maximal ist. ($\supseteq$ klar.)
  \item (i) Algebra II. (ii) Theorem.
  \end{enumerate}
\end{proof}




\section{Korrespondenz zwischen Radikalidealen und affinen algebraischen Mengen}
\label{sec:radikalideale-und-algebraische-mengen}

Sei $V(\mathfrak{a})\subseteq\mathbb{A}^{n}(k)$ affin algebraische
Menge, $\mathfrak{a}\unlhd k[\underline{T}]$ ein Ideal.\textbf{ Es gilt:}
\[
  V(\mathfrak{a})=V(\rad\mathfrak{a})
\]

mit $\rad\mathfrak{a}=\{f\in k[\underline{T}]\mid f^{n}\in\mathfrak{a}\text{ für ein }n>0\}$,
da
\[
  f^{n}(x)=0\Leftrightarrow f(x)=0,
\]

d.h. verschiedene Ideale können dieselbe algebraische Menge beschreiben.
\begin{defn}
  \label{def:verschwindungsideal}
  Für eine Teilmenge $Z\subseteq\mathbb{A}^{n}(k)$ bezeichne
  \[
    I(Z):=\{f\in k[\underline{T}]\mid f(x)=0\ \forall x\in Z\}
  \]

  das \textbf{Verschwindungsideal von Z}, das Ideal aller auf $Z$ verschwindenden Polynomfunktionen.
\end{defn}
\begin{prop}
  \label{prop:verschwindungsmenge-verschwindungsideal}
  \mbox{}
  \begin{enumerate}
  \item Sei $\mathfrak{a}\unlhd k[\underline{T}]$ Ideal. Dann ist $I(V(\mathfrak{a}))=\rad(\mathfrak{a})$.
  \item Sei $Z\subseteq\mathbb{A}^{n}(k)$ Teilmenge. Dann ist $V(I(Z))=\overline{Z}$,
    der Abschluss von $Z$ in $\mathbb{A}^{n}(k)$.
  \end{enumerate}
\end{prop}
\begin{proof}
  Übungsblatt 2.
\end{proof}
\medskip{}

$\mathfrak{a}$ heißt \textbf{Radikalideal}\index{Radikalideal},
falls $\mathfrak{a}=\rad(\mathfrak{a})$, oder äquivalent falls $k[\underline{T}]/\mathfrak{a}$
\emph{reduziert} ist, d.h. keine nilpotente Elemente ungleich $0$ hat.
\begin{cor}
  \label{korrespondenz-radikalideal-abgeschlossene-mengen}
  Wir erhalten eine 1-1 Korrespondenz
  \begin{align*}
    \{\text{abg. Mengen }\subseteq\mathbb{A}^{n}\} & \leftrightarrow\{\text{Radikalideale }\mathfrak{a}\unlhd k[\underline{T}]\}\\
    Z & \mapsto I(Z)\\
    V(\mathfrak{a}) & \mapsfrom\mathfrak{a}
  \end{align*}

  die sich zu einer 1-1 Korrespondenz
  \begin{align*}
    \left\{ \text{Punkte in }\mathbb{A}^{n}\right\}
    & \leftrightarrow\left\{ \text{max. Ideale in }k[\underline{T}]\right\} \\
    x=(x_{1},\ldots,x_{n})
    & \mapsto
      \begin{array}{rl}
        \mathfrak{m}_{x} & =I(\{x\})\\
                         & =\ker(k[\underline{T}]\rightarrow k,\ T_{i}\mapsto x_{i})
      \end{array}
  \end{align*}

  einschränkt.
\end{cor}



\section{Irreduzible topologische Räume}
\label{sec:irreduzibilitaet-top}

Die folgenden topologischen Begriffe sind nur interessant, da $\mathbb{A}^{n}(k)$
($n>0$) kein Hausdorff'scher Raum ist.
\begin{defn}
  \label{def:irreduzibel}
  Ein topologischer Raum $X$ heißt \textbf{irreduzibel}\index{irreduzibel},
  falls $X\neq\emptyset$ und $X$ sich \emph{nicht} als Vereinigung
  zweier echter abgeschlossener Teilmengen darstellen lässt, d.h
  \[
    X=A_{1}\cup A_{2},\ A_{i}\ \text{abg.}\quad\implies\quad A_{1}=X\text{ oder }A_{2}=X.
  \]

  $Z\subseteq X$ heißt irreduzibel, falls $Z$ mit der induzierten Topologie
  irreduzibel ist.
\end{defn}
\begin{prop}
  \label{prop:charakterisierung-irreduzibel}
  Für einen topologischen Raum $X \neq \emptyset$ sind äquivalent:
  \begin{enumerate}
  \item $X$ ist irreduzibel.
  \item Je zwei nichtleere offene Teilmengen von $X$ haben nicht-leeren
    Durchschnitt.
  \item Jede nichtleere offene Teilmenge $U\subseteq X$ ist dicht in $X$.
  \item Jede nichtleere offene Teilmenge $U\subseteq X$ ist zusammenhängend.
  \item Jede nichtleere offene Teilmenge $U\subseteq X$ ist irreduzibel.
  \end{enumerate}
\end{prop}
\begin{proof}
  \mbox{}
  \begin{itemize}
  \item $(i)\Leftrightarrow(ii)$

    Komplementärmengen.
  \item $(ii)\Leftrightarrow(iii)$ 

    Es ist: $U\subseteq X$ dicht $\Leftrightarrow U\cap O\neq\emptyset$
    für jedes offene $\emptyset\neq O\subseteq X$.
  \item $(iii)\Rightarrow(iv)$

    Klar. 
  \item $(iv)\Rightarrow(iii)$

    Sei $\emptyset\neq U$ offen und zusammenhängend. Es folgt:
    \[
      U=U_{1}\sqcup U_{2},\qquad\emptyset\neq U_{i}\underset{\text{offen}}{\subseteq}U\underset{\text{offen}}{\subseteq}X
    \]
    Damit ist $U_{1}\cap U_{2}=\emptyset$, ein Widerspruch zu (iii).
  \item $(v)\Rightarrow(i)$ 

    Klar. $(U=X)$
  \item $(iii)\Rightarrow(v)$

    Sei $\emptyset\neq U\underset{\text{offen}}{\subseteq}X$. Ist $\emptyset\neq V\underset{\text{offen}}{\subseteq}U$,
    so ist $V\underset{\text{offen}}{\subseteq}X$. Es folgt: $V$ ist
    dicht in $X$ und irreduzibel in $U$. Mit $(iii)\Rightarrow(i)$
    folgt, dass $U$ irreduzibel ist. 

  \end{itemize}
\end{proof}
\begin{lem}
  \label{lem:irreduzibel-abschluss}
  Eine Teilmenge $Y$ ist genau dann irreduzibel, wenn ihr Abschluss $\overline{Y}$ dies ist.
\end{lem}
\begin{proof}
  $Y$ irreduzibel

  $\Leftrightarrow\forall U,V\subseteq X$ offen mit $U\cap Y\neq\emptyset\neq V\cap Y$,
  gilt $Y\cap(U\cap V)\neq\emptyset$.

  $\Leftrightarrow\overline{Y}$ irreduzibel 
\end{proof}
\begin{defn}
  \label{def:irreduzible-komponente}
  Eine maximale irreduzible Teilmenge eines topologischen Raumes $X$
  heißt \textbf{irreduzible Komponente}\index{irreduzible Komponente}
  von $X$.
\end{defn}
\begin{rem}
  \label{rem:irreduzibel}
  \mbox{}
  \begin{enumerate}
  \item Jede irreduzible Komponente ist abgeschlossen nach Lemma \ref{lem:irreduzibel-abschluss}.
  \item $X$ ist Vereinigung seiner irreduziblen Komponenten, \emph{denn}: 

    die Menge der irreduziblen Teilmengen von $X$ ist \textbf{induktiv
      geordnet}: für jede aufsteigende Kette irreduzibler Teilmengen ist
    die Vereinigung wieder irreduzibel (Satz \ref{prop:charakterisierung-irreduzibel}.(ii)). Mit dem \textbf{Lemma
      von Zorn} folgt: Jede irreduzible Teilmenge ist in einer irreduziblen
    Komponente enthalten. Damit ist jeder Punkt in einer irreduziblen
    Komponente enthalten.
  \end{enumerate}
\end{rem}




\section{Irreduzible affine algebraische Mengen}
\label{sec:irreduzibilitaet-alg}
\begin{lem}
  \label{lem:charakterisierung-irreduzibel-alg}
  Eine abgeschlossene Teilmenge $Z\subseteq\mathbb{A}^{n}(k)$ ist genau
  dann irreduzibel, wenn $I(Z) \unlhd k[\underline{T}]$ ein Primideal ist. Insbesondere ist
  $\mathbb{A}^{n}(k)$ irreduzibel.  
\end{lem}
\begin{proof}
  $Z$ irreduzibel ist äquivalent zu 
  \begin{align*}
    & (Z=\underbrace{V(\mathfrak{a})}_{\bigcap_{i} V(f_{i})}\cup\underbrace{V(\mathfrak{b})}_{\bigcap_{j} V(g_{j})}\quad\Rightarrow\quad V(\mathfrak{a})=Z\text{ oder }V(\mathfrak{b})=Z).\\
    \Leftrightarrow\  & \forall f,g\in k[\underline{T}]:\ V(fg)=V(f)\cup V(g)\supseteq Z:\ V(f)\supseteq Z\text{ oder }V(g)\supseteq Z.\\
    (*)\Leftrightarrow\  & \forall f,g\in k[\underline{T}]:\ fg\in I(V(fg))\subseteq I(Z):\ f\in I(Z)\text{ oder }g\in I(Z).\\
    \Leftrightarrow\  & I(Z)\text{ ist Primideal.}
  \end{align*}

  ({*}): $V(I(Z))=Z$, $I(V(\mathfrak{a}))=\rad(\mathfrak{a})$. 
\end{proof}
\begin{rem}
  \label{rem:korrespondenz-irreduzibel-prim}
  Die Korrespondenz aus Korollar \ref{cor:korrespondenz-radikalideal-abgeschlossene-mengen} schränkt sich ein zu
  \[
    \{\text{irred. abg. Teilmengen }\subseteq\mathbb{A}^{n}\}\overset{1:1}{\leftrightarrow}\{\text{Primideale in }k[\underline{T}]\}
  \]
\end{rem}




\section{Quasikompakte und noethersche topologische Räume}
\label{sec:quasikompakt-noethersch}
\begin{defn}
  \label{def:quasikompakt/noethersch}
  Ein topologischer Raum $X$ heißt \textbf{quasikompakt}\index{quasikompakt},
  falls jede offene Überdeckung von $X$ eine \emph{endliche} Teilüberdeckung
  enthält. (,,quasi`` deutet an, dass $X$ in der Regel nicht Hausdorff'sch
  ist!). Er heißt \textbf{noethersch}\index{noethersch}, wenn jede
  absteigende Kette
  \[
    X\supseteq Z_{1}\supseteq Z_{2}\supseteq\cdots
  \]

  abgeschlossener Teilmengen von $X$ stationär wird ($\Leftrightarrow$
  jede aufsteigende Kette offener Teilmengen wird stationär).
\end{defn}
\begin{lem}
  \label{lem:eigenschaften-noethersch}
  Sei $X$ ein noetherscher topologischer Raum. Dann gilt:
  \begin{enumerate}
  \item Jede abgeschlossene Teilmenge $Z \subseteq X$ ist noethersch.
  \item Jede offene Teilmenge $U \subseteq X$ ist quasikompakt.
  \item Jeder abgeschlossene Teilraum $Z \subseteq X$ besitzt nur endlich viele
    irreduzible Komponenten.
  \end{enumerate}
\end{lem}
\begin{proof}
  \mbox{}
  \begin{enumerate}
  \item Nach Definition, da abgeschlossene Mengen von $Z$ auch solche von
    $X$ sind.
  \item $U=\bigcup_{i\in I}U_{i}$ offen; Angenommen $U$ wäre nicht quasikompakt.
    Dann gibt es eine Folge $I_{1}\subseteq I_{2}\subseteq\cdots\subseteq I$ von Teilmengen
    mit
    \[
      V_{1}\subsetneq V_{2}\subsetneq\cdots\neq U\quad\text{für }V_{j}=\bigcup_{i\in I_{j}}U_{i}.
    \]
    Widerspruch zu noethersch.
  \item Es reicht zu zeigen: Jeder noethersche Raum ist Vereinigung endlich
    vieler irreduzibler Teilmengen. Da $X$ noethersch ist, folgt mit
    dem \emph{Lemma von Zorn} dass jede nichtleere Menge von algebraischen
    Teilmengen in $X$ ein minimales Element besitzt. 
    \[
      \text{Angenommen:} \mathcal{M}:=\left\{ Z\subseteq X\text{ abg.}\mid Z\text{ ist \textbf{nicht} endl. Vereinigung irred. Mengen}\right\} \text{ wäre nichtleer.}
    \]
    $\Rightarrow\exists$ minimales Element, sagen wir $Z$, in $\mathcal{M}$.

    $\Rightarrow Z$ ist nicht irreduzibel.

    $\Rightarrow Z=Z_{1}\cup Z_{2}$ mit $Z_{1},Z_{2}\subsetneq Z$ abgeschlossen.

    $\Rightarrow$ ($Z$ minimal) $Z_{1},Z_{2}\notin\mathcal{M}$

    $\Rightarrow Z\notin\mathcal{M}$. Widerspruch.

  \end{enumerate}
\end{proof}
\begin{prop}
  \label{prop:algebraische-mengen-noethersch}
  Jeder abgeschlossene Teilraum $X\subseteq\mathbb{A}^{n}(k)$ ist noethersch.
\end{prop}
\begin{proof}
  Nach dem obigen Lemma ist nur zu zeigen, dass $\mathbb{A}^{n}(k)$
  noethersch ist.

  Absteigende Ketten abgeschlossener Teilmengen sind nach \emph{Korollar
    11} in 1-1 Korrespondenz mit aufsteigenden Ketten von (Radikal-)Idealen
  in $k[\underline{T}]$. Da $k[\underline{T}]$ nach dem Hilbertschen
  Basissatz noethersch ist, werden letzere Ketten stationär.
\end{proof}
\begin{cor}[Primärzerlegung]
  \label{cor:primaerzerlegung}
  Sei $\mathfrak{a}=\rad(\mathfrak{a})\unlhd k[\underline{T}]$
  ein Radikalideal. Dann gilt: $\mathfrak{a}$ ist Durchschnitt von
  endlich vielen Primidealen, die sich jeweils paarweise nicht enthalten; diese
  Darstellung ist eindeutig bis auf Reihenfolge.
\end{cor}
\begin{proof}
  $V(\mathfrak{a})=\bigcup_{i=1}^{n}V(\mathfrak{b}_{i})$, $\mathfrak{b}_{i}$
  Primideal.%
  \begin{comment}
    Stationäre Kette folgt aus noethersch (Satz 21); mit Bemerkung 16
    bzw. Lemma 17 folgt, dass die $\mathfrak{b}_{i}$ Primideale sind.
  \end{comment}
  {} [Anmerkung] Mit Satz 10 folgt: % color package gives errors with plastex
  \[
    \mathfrak{a}=\rad(\mathfrak{a})=I(V(\mathfrak{a}))=\bigcap_{i=1}^{n}\underbrace{I(V(\mathfrak{b}_{i}))}_{\mathfrak{b}_{i}\text{ minimale Primideale (\ref{lem:charakterisierung-irreduzibel-alg})}}
  \]
\end{proof}




\section{Morphismen von affinen algebraischen Mengen}
\label{sec:morphismen-alg-mengen}
\begin{defn}
  \label{def:morphismus-alg-mengen}
  Seien $X\subseteq\mathbb{A}^{m}(k)$, $Y\subseteq\mathbb{A}^{n}(k)$
  affine algebraische Mengen. Ein \textbf{Morphismus} $X\rightarrow Y$
  affiner algebraischer Mengen ist eine Abbildung $f:X\rightarrow Y$
  der zugrundeliegenden Mengen, sodass $f_{1},\ldots,f_{n}\in k[T_{1},\ldots,T_{m}]$
  existieren, derart dass $\forall x\in X$ gilt:
  \[
    f(x)=(f_{1}(x),\ldots,f_{n}(x)) \in Y.
  \]
  Es bezeichne $\hom(X,Y)$ die Menge der Morphismen $X \to Y$. 
\end{defn}
\begin{rem}
  \label{rem:morphismen-fortsetzbarkeit}
  $f:X\rightarrow Y$ lässt sich immer fortsetzen zu einem Morphismus
  \[
    f:\mathbb{A}^{m}(k)\rightarrow\mathbb{A}^{n}(k),
  \]

  aber nicht eindeutig, es sei denn $X=\mathbb{A}^{m}(k)$.
\end{rem}

\paragraph{Komposition}

\[
  \xymatrix@C=9pc{X\ar[r]^{f}_{f_{1},\ldots,f_{n}\in k[T_{1},\ldots,T_{m}]} & Y\ar[r]^{g}_{g_{1},\ldots,g_{r}\in k[T_{1}',\ldots,T_{m}']} & Z}
\]

mit $X\subseteq\mathbb{A}^{m}(k)$, $Y\subseteq\mathbb{A}^{n}(k)$,
$Z\subseteq\mathbb{A}^{r}(k)$. Es folgt:
\begin{align*}
  g(f(x))=\, & (g_{1}(f_{1}(x),\ldots,f_{n}(x)),\ldots,g_{r}(f_{1}(x),\ldots,f_{n}(x))\\
  =:\, & (h_{1}(x),\ldots,h_{r}(x))
\end{align*}

d.h. $g\circ f$ ist durch Polynome $h_{i}\in k[T_{1},\ldots,T_{m}]$
gegeben, also ist $g\circ f$ wieder ein Morphismus affiner algebraischer
Mengen. Wir erhalten die \textbf{Kategorie affiner algebraischer Mengen}.
\begin{example}
  \label{bsp:morphismen-alg-mengen}
  \mbox{}
  \begin{enumerate}
  \item Sei die Abbildung
    \begin{align*}
      \mathbb{A}^{1}(k) & \rightarrow V(T_{2}-T_{1}^{2})\subseteq\mathbb{A}^{2}(k)\\
      x & \mapsto(x,x^{2}).
    \end{align*}
    Diese Abbildung ist sogar ein \emph{Isomorphismus }affiner algebraischer
    Mengen, da die Umkehrabbildung
    \[
      (x,y)\mapsto x
    \]
    ebenfalls ein Morphismus ist.
  \item Sei char$(k)\neq2$. Die Abbildung
    \begin{align*}
      \mathbb{A}^{1}(k) & \rightarrow V(T_{2}^{2}-T_{1}^{2}(T_{1}+1))\\
      x & \mapsto(x^{2}-1,x(x^{2}-1))
    \end{align*}
    ist ein Morphismus, aber \emph{nicht }bijektiv, da $1,-1$ beide auf
    $(0,0)$ abgebildet werden.
  \end{enumerate}
\end{example}




\section{Unzulänglichkeiten des Begriffs der affinen algebraischen Mengen}
\label{sec:unzulaenglichkeiten-alg-mengen}
\begin{enumerate}
\item Offene Teilmengen affiner algebraischer Mengen tragen nicht in natürlicher
  Weise die Struktur einer affinen algebraischen Menge.
\item Insbesondere können wir affine algebraische Mengen nicht entlang offener
  Teilräume verkleben. (vgl. Mannigfaltigkeiten.)
\item Keine Unterscheidungsmöglichkeiten z.B. zwischen $\{(0,0)\}$,
  $V(T_{1})\cap V(T_{2})$ und
  $V(T_{2})\cap V(T_{1}^{2}-T_{2})\subseteq\mathbb{A}^{2}(k)$, obwohl
  die ``geometrische Situation'' offensichtlich verschieden ist.
\end{enumerate}
Um die Punkte 1 und 2 zu verbessern, gehen wir im Folgenden zu ``Räumen mit Funktionen'' über, und verzichten darauf,
dass sich diese in einen affinen Raum $\mathbb{A}^{n}(k)$ einbetten
lassen.

Der Punkt 3 ist eine Motivation dafür, später Schemata einzuführen.
(subtiler)




\paragraph*{Affine algebraische Mengen als Räume von Funktionen}

\section{Der affine Koordinatenring}
\label{sec:koordinatenring}

Sei $X\subseteq\mathbb{A}^{n}(k)$ abgeschlossen. Für den surjektiven
(Def. von Morphismen) $k$-Algebren-Homomorphismus
\begin{align*}
  k[\underline{T}] & \xrightarrow{\varphi}\hom(X,\mathbb{A}^{1}(k))\\
  f & \mapsto(x\mapsto f(x)),
\end{align*}

wobei die Morphismen in folgende Weise eine $k$-Algebra bilden:
\begin{align*}
  (f+g)(x) & :=f(x)+g(x)\\
  (fg)(x) & :=f(x)g(x)\\
  (\alpha f)(x) & :=\alpha f(x)
\end{align*}

mit $f,g\in\hom(X,\mathbb{A}^{1}(k))$, $\alpha\in k$, gilt:
\[
  \ker\varphi=I(X).
\]
\begin{defn}
  \label{def:koordinatenring}
  $\Gamma(X):=k[\underline{T}]/I(X)\cong_{k-\mathrm{Alg}}\hom(X,\mathbb{A}^{1}(k))$ heißt der \textbf{affine
    Koordinatenring }von $X$.

  Für $x=(x_{1},\ldots,x_{n})\in X$ gilt:
  \begin{align*}
    \mathfrak{m}_{x}:=\  & \ker(\Gamma(X)\twoheadrightarrow k,\,f\mapsto f(x))\\
    =\  & \{f\in\Gamma(X)\mid f(x)=0\}\\
    =\  & \pi((T_{1}-x_{1},\ldots,T_{n}-x_{n}))\\
    =\  & \ker(\Gamma(\mathbb{A}^{n}(k))\twoheadrightarrow k)
  \end{align*}

  unter der Projektion $\pi:k[\underline{T}]=\Gamma(\mathbb{A}^{n}(k))\twoheadrightarrow\Gamma(X)$.
  Es ist $\mathfrak{m}_{x}$ ein maximales Ideal von $\Gamma(X)$ mit
  $\Gamma(X)/\mathfrak{m}_{x}\cong k$. Für ein Ideal $\mathfrak{a}\unlhd\Gamma(X)$
  sei
  \[
    V(\mathfrak{a}):=\{x\in X\mid f(x)=0\ \forall f\in\mathfrak{a}\}=V(\pi^{-1}(\mathfrak{a}))\cap X.
  \]

  Dies sind genau die abgeschlossenen Mengen von $X$ als Teilraum von
  $\mathbb{A}^{n}(k)$ mit der induzierten Topologie, diese wird auch
  \textbf{Zariski-Topologie} genannt. Für $f\in\Gamma(X)$ setze:
  \[
    D_X(f) := D(f):=\{x\in X\mid f(x)\neq0\}=X\setminus V(f).
  \]
\end{defn}
\begin{lem}
  \label{lem:basis-zariski-topologie}
  Die offenen Mengen $D(f)$, $f\in\Gamma(X)$, bilden eine Basis der
  Topologie von $X$, d.h.
  \[
    \forall U\subseteq X\text{ offen }\exists f_{i}\in\Gamma(X),\,i\in I\quad\text{mit }U=\bigcup_{i\in I}D(f_{i})
  \]
\end{lem}
\begin{proof}
  $U=X\backslash V(\mathfrak{a})$ für ein $\mathfrak{a}\unlhd\Gamma(X)$,
  $\mathfrak{a}=\langle f_{1},\ldots,f_{n}\rangle_{\Gamma(X)}$ . Wegen
  \[
    V(\mathfrak{a})=\bigcap_{i=1}^{n}V(f_{i})\quad\Rightarrow\quad U=\bigcup_{i=1}^{n}D(f_{i})
  \]

  Es reichen also sogar endlich viele $f_{i} \in \Gamma(X)$! 
\end{proof}
\begin{prop}
  \label{prop:eigenschaften-koordinatenring}
  Der Koordinatenring $\Gamma(X)$ einer affinen algebraischen Menge
  $X$ ist eine endlich erzeugte $k$-Algebra, die reduziert ist (d.h.
  keine nilpotenten Elemente $\neq0$ enthält). Ferner ist $X$ irreduzibel
  genau dann, wenn $\Gamma(X)$ integer ist.
\end{prop}
\begin{proof}
  $k[\underline{T}]\twoheadrightarrow\Gamma(X)$ impliziert, dass $\Gamma(X)$ als $k$-Algebra endlich
  erzeugte ist. Es gilt:
  \[
    \Gamma(X)\text{ irreduzibel }\Leftrightarrow I(X)=\rad I(X).
  \]

  Denn mit Satz 10.ii) und Korollar 11 folgt:
  \begin{align*}
    X & =V(\mathfrak{a}):\,I(X)=\rad\mathfrak{a}\\
    \Rightarrow\rad I(X) & =\rad\rad\mathfrak{a}=\rad\mathfrak{a}=I(X).
  \end{align*}

  Mit Lemma 17 folgt: $X$ irreduzibel

  $\phantom{\quad}\Leftrightarrow I(X)$ prim

  $\phantom{\quad}\Leftrightarrow\Gamma(X)=k[\underline{T}]/I(X)$ integer.
\end{proof}




\section{Funktorielle Eigenschaften von $\Gamma(X)$}
\label{sec:koordinatenring-funktiorialitaet}
\begin{prop}
  \label{prop:koordinatenringfunktor}
  Für einen Morphismus $X\xrightarrow{f}Y$ affiner algebraischer Mengen
  definiert 
  \begin{align*}
    \Gamma(f):\quad\Gamma(Y) & \rightarrow\Gamma(X)\\
    g & \mapsto g\circ f
  \end{align*}
  ein Homomorphismus von $k$-Algebren. Der so definierte \emph{kontravariante}
  Funktor
  \[
    \Gamma:\{\text{affine algebraische Mengen}\}\rightarrow\{\text{reduzierte endl. erz. }k\text{-Algebren}\}
  \]
  liefert eine Kategorienäquivalenz, welche durch Einschränkung eine Äquivalenz
  \[
    \Gamma:\{\text{irred. aff. alg. Mengem\}}\rightarrow\{\text{integre endl. erz. }k\text{-Algebren\}}
  \]
  induziert.
\end{prop}
\begin{proof}
  Sei $Y\xrightarrow{g}\mathbb{A}^{1}(k)\in\Gamma(Y)$ ein Morphismus. Es
  folgt:
  \[
    g\circ f:X\xrightarrow{f}Y\xrightarrow{g}\mathbb{A}^{1}(k)
  \] 
  ist Morphismus,  d.h. $g \circ f\in\Gamma(X)$. $\Gamma(f):\Gamma(Y)\rightarrow\Gamma(X)$
  ist ein $k$-Algebren-Homomorphismus mit $\Gamma(\text{id}_{X})=\text{id}_{\Gamma(X)}$. Da ferner gilt, dass $\Gamma(f_{1}\circ f_{2})=\Gamma(f_{2})\circ\Gamma(f_{1})$ ist $\Gamma$ ein kontravarianter Funktor.
  \begin{claim*}
    $\Gamma$ ist volltreu, d.h.
    \begin{align*}
      \Gamma:\hom(X,Y) & \rightarrow\hom_{k\text{-Alg}}(\Gamma(Y),\Gamma(X))\\
      f & \mapsto\Gamma(f)
    \end{align*}
    ist \emph{bijektiv} für alle affinen algebraischen Mengen $X,Y$.
  \end{claim*}
  \begin{proof}
    Wir konstruieren eine Umkehrabbildung wie folgt: Zu $\varphi:\Gamma(Y)\rightarrow\Gamma(X)$
    für $X\subseteq\mathbb{A}^{m}(k)$, $Y\subseteq\mathbb{A}^{n}(k)$ existiert ein Lift $\tilde\varphi$, s.d.
    \[
      \xymatrix{k[T_{1}',\ldots,T_{n}']\ar[r]^{\tilde{\varphi}}\ar@{->>}[d] & k[T_{1},\ldots,T_{m}]\ar@{->>}[d]\\
        \Gamma(Y)\ar[r]^{\varphi} & \Gamma(X)
      }
    \]
    kommutiert; $\tilde{\varphi}(T_{i}'):= f_i$ mit $f_i \in \pi^{-1}(\varphi(T_{i}')) \subseteq k[T_1,...,T_n]$, wobei $\pi : k[\underline{T}] \to \Gamma(X)$ die kanonische Projektion bezeichne. 
    Definiere:
    \begin{align*}
      f:X & \rightarrow Y\\
      x=(x_{1},\ldots,x_{n}) & \mapsto(\tilde{\varphi}(T_{1}')(x_{1},\ldots,x_{n}),\ldots,\tilde{\varphi}(T_{n}')(x_{1},\ldots,x_{n}))
    \end{align*}
  \end{proof}
  \begin{claim*}
    $\Gamma$ ist essentiell surjektiv, d.h. zu jeder reduzierten endlich
    erzeugten $k$-Algebra $A$ existiert eine affine algebraische Menge
    $X$ mit $A\cong\Gamma(X)$.
  \end{claim*}
  \begin{proof}
    Da nach Voraussetzung $A\cong k[T]/\mathfrak{a}$ für ein Radikalideal
    $\mathfrak{a}$, können wir etwa $X:=V(\mathfrak{a})\subseteq\mathbb{A}^{n}(k)$
    setzen. Der Rest folgt aus Satz \ref{prop:eigenschaften-koordinatenring}.
  \end{proof}
\end{proof}
\begin{prop}
  \label{prop:funktiorialitaet-specm}
  Sei $f:X\rightarrow Y$ ein Morphismus affiner algebraischer Mengen und $\Gamma(f):\Gamma(Y)\rightarrow\Gamma(X)$
  der zugehörige Homomorphismus der Koordinatenringe. Dann gilt $\forall x\in X$:
  $\Gamma(f)^{-1}(\mathfrak{m}_{x})=\mathfrak{m}_{f(x)}$.
\end{prop}
\begin{proof}
  \[
    \Gamma(f)^{-1}(\mathfrak{m}_{x})=\{g\in\Gamma(Y)\mid g\circ f \in \mathfrak{m}_{x}\}=\{g\in\Gamma(Y)\mid g(f(x)) = 0 \} = \mathfrak{m}_{f(x)},
  \]
  da $\Gamma(f)(g) =g \circ f$.
\end{proof}



\section{Räume mit Funktionen}
\label{sec:raeume-mit-funktionen}
(Prototyp eines geometrischen Objektes, Spezialfall eines ``geringten
Raumes'' vgl. später.) Sei $K$ ein nicht notwendigerweise algebraisch abgeschlossener
Körper.
\begin{defn}
  \label{def:raum-mit-funktionen}
  \mbox{}
  \begin{enumerate}
  \item Ein \textbf{Raum mit Funktionen}$_{/K}$\index{Raum mit Funktionen} besteht
    aus den folgenden Daten:
    \begin{itemize}
    \item ein topologischer Raum $X$;
    \item eine Familie von Unter-$K$-Algebren
      \[
        \mathcal{O}_X(U)\leq\text{Abb}(U,K),\quad\forall U\subseteq X\text{ offen }d.d
      \]

      \begin{enumerate}
      \item Sind $U'\subseteq U\subseteq X$ offen und $f\in\mathcal{O}_X(U)$ so ist
        $f|_{U'}\in \mathcal{O}_X(U')$.
      \item (\textbf{Verklebungsaxiom}\index{Verklebungsaxiom}) Sind $U_{i}\subseteq X$
        offen, $i\in I$, und $U=\bigcup_{i}U_{i}$, $f_{i}\in\mathcal{O}_X(U_{i})$,
        $i\in I$ gegeben mit
        \[
          f_{i}|_{U_{i}\cap U_{j}}=f_{j}|_{U_{i}\cap U_{j}}\quad\forall i,j\in I
        \]
        dann ist die eindeutige Abbildung
        \[
          f:U\rightarrow K\text{ mit }f|_{U_{i}}=f_{i}
        \]
        in $\mathcal{O}_X(U)$, bzw. $\exists!f\in\mathcal{O}(U)$ mit $f|_{U_{i}}=f_{i}$ für alle $i \in I$.
      \end{enumerate}
    \end{itemize}
    Bezeichne $\mathcal{O}_X$ oder auch $\mathcal{O}$ die oben genannte
    Familie $\{\mathcal{O}_X(U) \mid U \subseteq X \text{offen}\}$. Das Tupel $(X,\mathcal{O}_{X})$ heißt $\textbf{Raum mit Funktionen}$.
  \item Ein \textbf{Morphismus}\index{Raum mit Funktionen!Morphismus} $(X,\mathcal{O}_{X})\rightarrow(Y,\mathcal{O}_{Y})$
    von Räumen von Funktionen ist eine stetige Abbildung $\varphi:X\rightarrow Y$,
    so dass für alle $V\subseteq Y$ offen und $f\in\mathcal{O}_{Y}$
    gilt:
    \[
      f\circ \varphi|_{\varphi^{-1}(V)}:\varphi^{-1}(V)\rightarrow K
    \]
    liegt in $\mathcal{O}_{X}(\varphi^{-1}(V))$.
    \[
      \xymatrix{X\ar[r]^{\varphi} & Y\\
        \varphi^{-1}(V)\ar[r]^{\varphi|}\ar[d]_{f\circ \varphi|_{\varphi^{-1}(V)}}\ar@{^{(}->}[u] & V\ar[d]^{f}\ar@{^{(}->}[u]_{\text{offen}}\\
        K\ar@{=}[r] & K
      }
    \]
  \end{enumerate}
\end{defn}
Wir erhalten die Kategorie der $\emph{Räume mit Funktionen über K}$.
\begin{defn}[offene Unterräume von Räumen mit Funktionen]
  \label{def:raeume-mit-fkt-offener-unterraum}
  Für $(X,\mathcal{O}_{X})$ einen Raum mit Funktionen und $U\subseteq X$ offen bezeichne $(U,\mathcal{O}_{X}|_{U})$ den
  Raum mit Funktionen gegeben durch den topologischen Raum $U$ mit
  Funktionen $\mathcal{O}_{X}|_{U}(V):=\mathcal{O}_{X}(V)$ für $V\underset{\text{offen}}{\subseteq}U\subseteq X$.
\end{defn}
\textbf{Ab jetzt} betrachten wir Räume von Funktionen über einem fixierten, algebraisch abgeschlossenen Grundkörper $k$.




\section{Der Raum mit Funktionen zu einer affin-algebraischen Menge}
\label{sec:alg-mengen-raeume-mit-fkt}

\textbf{Ziel.} Wir wollen jeder irreduziblen affin algebraischen Menge $X\subseteq\mathbb{A}^{n}(k)$ einen Raum mit Funktionen $(X,\mathcal{O}_{X})$ zuordnen.
D.h. wir müssen Mengen von Funktionen $\mathcal{O}_{X}(U) \leq \text{Abb}(U,k)$,
$U\subseteq X$ offen, definieren. Diese werden als Teilmengen des Funktionenkörpers
$K(X)$ definiert (dazu $X$ irreduzibel, später bei Schemata fällt
diese Bedingung weg!)
\begin{defn}
  \label{def:funktionenkoerper}
  Für eine irreduzible, affin-algebraische Menge $X$ heißt $K(X):=\text{Quot}(\Gamma(X))$ \textbf{Funktionenkörper} von $X$.

  Elemente $\frac{f}{g}\in K(X)$, $f,g\in\Gamma(X)=\hom(X,\mathbb{A}^{1}(k))$,
  $g\neq0$ lassen sich zumindest als Funktion auf der offenen Menge
  $D(g)\subseteq X$ auffassen, wenn auch i.A. nicht auf ganz
  $X$.
\end{defn}
\begin{lem}
  \label{lem:gleichheit-im-funktionenkoerper}
  Gilt für $\frac{f_{1}}{g_{1}},\frac{f_{2}}{g_{2}}\in K(X)$, $f_{i},g_{i}\in\Gamma(X)$,
  und einer offenen Teilmenge $\emptyset\neq U\subseteq D(g_{1}g_{2})$
  \[
    \frac{f_{1}(x)}{g_{1}(x)}=\frac{f_{2}(x)}{g_{2}(x)}\qquad\forall x\in U,
  \]

  dann folgt $\frac{f_{1}}{g_{1}}=\frac{f_{2}}{g_{2}}$ in $K(X)$.
\end{lem}
\begin{proof}
  Sei ohne Einschränkung der Allgemeinheit $g_{1}=g_{2}=g$. (Sonst
  Erweitern!) 

  $\Rightarrow(f_{1}-f_{2})(x)=0$ $\forall x\in U$.

  $\Rightarrow\emptyset\neq U\subseteq V(f_{1}-f_{2})\subseteq X$ dicht,
  d.h. $V(f_{1}-f_{2})=X$.

  $\phantom{\Rightarrow\ }$$f_{1}-f_{2}\in I()V(f_{1}-f_{2}))=I(X)\equiv(0)$
  in $\Gamma(X)$ 

  $\Rightarrow f_{1}-f_{2}=0$.
\end{proof}
\begin{defn}
  \label{def:alg-menge-als-raum-mit-fkt}
  Sei $X$ eine irreduzible affin-algebraische Menge, $U\subseteq X$
  offen. Für $x \in X$ bezeichne $\Gamma(X)_{\mathfrak{m}_{x}}$ die Lokalisierung von $\Gamma(X)$ an der multiplikativ abgeschlossenen Menge $S := \Gamma(X) \setminus \mathfrak{m}_{x}$.
  \[
    \mathcal{O}_{X}(U):=\bigcap_{x\in U}\Gamma(X)_{\mathfrak{m}_{x}}\subseteq K(X)
  \]

  d.h. für jedes $x\in U$ lässt sich $f\in\mathcal{O}_{X}(U)$ schreiben
  als $\frac{h}{g} \in K(X)$ mit $g(x)\neq0$.
\end{defn}
Für $f\in\Gamma(X)$ bezeichne $\Gamma(X)_{f}$ die Lokalisierung
von $\Gamma(X)$ an der multiplikativ abgeschlossenen Menge
$\{1,f,f^{2},\ldots,f^{n}\ldots\}$. Dann lässt sich
\[
  \Gamma(X)_{\mathfrak{m}_{x}}=\bigcup_{f\in\Gamma(X)\backslash\mathfrak{m}_{x}}\Gamma(X)_{f}\subseteq K(X)
\]

schreiben. ``$\supseteq$'': klar, ``$\subseteq$'': $\frac{g}{f}$
mit $f(x)\neq0$ d.h. $f\notin\mathfrak{m}_{x}$ $\Rightarrow\frac{g}{f}\in\Gamma(X)_{f}$.


\paragraph{Es gilt:}
\begin{enumerate}
\item Für $V\subseteq U\subseteq X$ offen kommutiert das folgende Diagramm:
  \[
    \xymatrix{\mathcal{O}_{X}(V)\ar@{^{(}->}[r] & \text{Abb}(V,k)\\
      \mathcal{O}_{X}(U)\ar@{^{(}->}[r]\ar@{^{(}->}[u] & \text{Abb}(U,k)\ar[u]_{\text{Einschränkungsabb.}}
    }
  \]
  mit $\mathcal{O}_{X}(U) \hookrightarrow \mathcal{O}_{X}(V), f \mapsto f|_V$ nach Definition.
\item $\mathcal{O}_{X}(U)\rightarrow\text{Abb}(U,k)$, $f\mapsto(x\mapsto f(x):=\frac{g(x)}{f(x)}\in k)$
  ist injektiv (Lemma 34) und wohldefiniert (kürzen/erweitern), wobei
  $g,h\in\Gamma(X)$ mit $h\notin\mathfrak{m}_{x}$ mit $f=\frac{g}{h}$
  nach Definition von $\mathcal{O}_{X}(U)$ existiert.
\item \textbf{Verklebungseigenschaft.} Sei $U=\bigcup_{i\in I}U_{i}$. Nach
  Definition ist 
  \begin{align*}
    \mathcal{O}_{X}(U) & =\bigcap_{i}\mathcal{O}_{X}(U_{i})\subseteq K(X)\\
    \ni f:U\rightarrow k & \quad\ni f_{i}:U_{i}\rightarrow k
  \end{align*}
  {[}Diagramm fehlt{]}. $(X,\mathcal{O}_{X})$ ist Raum
  mit Funktionen, \textbf{der zur irreduziblen affin algebraische Menge
    assoziierte Raum von Funktionen}. 
\end{enumerate}
\begin{prop}[orig. 33]
  \label{prop:fkt-auf-basis}
  Für $(X,\mathcal{O}_{X})$ zu $X$ wie oben und $f\in\Gamma(X)$
  gilt:
  \[
    \mathcal{O}_{X}(D(f))=\Gamma(X)_{f},
  \]

  insbesondere $\mathcal{O}_{X}(X)=\Gamma(X)$.
\end{prop}
\begin{proof}
  $\Gamma(X)_f\subseteq \mathcal{O}_{X}(D(f))$ klar, da $f(x)\neq0$ $\forall x\in D(f)$
  bzw. $f\in \Gamma(X)\setminus\mathfrak{m}_{x}$. 

  Sei nun $g$ in $\mathcal{O}_{X}(D(f))$ gegeben, $(*)$
  und $\mathfrak{a}:=\{h\in\Gamma(X)\mid hg\in\Gamma(X)\}\unlhd\Gamma(X)$.

  Dann gilt: $g\in\Gamma(X)_{f}$

  $\Leftrightarrow g=\frac{k}{f^{n}}$ für ein $n$ und $k\in\Gamma(X)$

  $\Leftrightarrow f^{n}\in\mathfrak{a}$ für ein $n$.

  d.h. zu zeigen: $f\in\text{rad}(\mathfrak{a})=I(V(\mathfrak{a}))$ (Hilbertscher
  Nullstellensatz)

  $\Leftrightarrow f(x)=0$ $\forall x\in V(\mathfrak{a})$

  Ist dazu $x\in X$ mit $f(x)\neq0$, also $x\in D(f)$, so
  existieren wegen $g \in \mathcal{O}_{X}(D(f))$ \\ Funktionen $f_{1},f_{2}\in\Gamma(X)$, $f_{2}\notin\mathfrak{m}_{x}$
  mit $g=\frac{f_{1}}{f_{2}}$, also gilt $f_{2}\in\mathfrak{a}$. 

  Da $f_{2}(x)\neq0$ folgt weiter $x\notin V(\mathfrak{a})$.
\end{proof}
\begin{rem}[orig. 34]
  \label{rem:globale-darstellung-von-fkt}
  \mbox{}
  \begin{enumerate}
  \item Im Allgemeinen existieren für $f\in\mathcal{O}_{x}(U)$ \textbf{nicht notwendigerweise}
    $g,h\in\Gamma(X)$ mit $f=\frac{g}{h}$ und $h(x)\neq0$ $\forall x\in U$.
  \item \textbf{Alternative Definition von $\mathcal{O}_{X}$, I.}
    \[
      \mathcal{O}_{X}(D(f)):=\Gamma(X)_{f},\quad\forall f\in\Gamma(X).
    \]
    Da $(D(f))_{f \in \Gamma(X)}$ Basis der Topologie bildet, kann es höchstens einen
    Raum mit Funktionen mit dieser Eigenschaft geben, es bleibt die Existenz
    zu zeigen.
  \item \textbf{Alternative Definition von $\mathcal{O}_{X}$, II.}

    Direkt von einer integeren endlich erzeugten $k$-Algebra $A$ ausgehend
    (die $X$ bis auf Isomorphie festlegt), aber ohne ``Koordinaten''
    zu wählen.
    \begin{align*}
      X & :=\{\mathfrak{m}\unlhd A\mid\ \mathfrak{m} \text{ ist max. Ideal}\}
    \end{align*}
    Die \textbf{abgeschlossenen Mengen} sind gegeben durch:
    \[
      V(\mathfrak{a}):=\{\mathfrak{m} \in X \mid\mathfrak{m}\supseteq\mathfrak{a}\},\quad\mathfrak{a}\unlhd A\text{ Ideal}.
    \]

    $\mathcal{O}_{X}(U):=\bigcap_{\mathfrak{m}\in U}A_{\mathfrak{m}}\subseteq\text{Quot}(A)$
    für $U\subseteq X$ offen (vgl. später Schemata).
  \end{enumerate}
\end{rem}




\section{Funktorialität der Konstruktion}
\label{sec:funktorialitaet-affine-varietaet}
\begin{prop}[orig. 35]
  \label{prop:charakterisierung-morphismen-alg-mengen}
  Sei $f:X\rightarrow Y$ eine stetige Abbildung zwischen irreduziblen
  affin-algebraischen Mengen. Es sind äquivalent:
  \begin{enumerate}
  \item $f$ ist ein Morphismus affin-algebraischer Mengen.
  \item $\forall g\in\Gamma(Y)$ gilt $g\circ f\in\Gamma(X)$.
  \item $f$ ist ein Morphismus von Räumen von Funktionen, d.h. für alle $U\subseteq Y$
    offen und alle $g\in\mathcal{O}_{Y}(U)$ gilt $g\circ f\in\mathcal{O}_{X}(f^{-1}(U))$.
  \end{enumerate}
\end{prop}
\begin{proof}
  \mbox{}
  \begin{itemize}
  \item $(i)\Leftrightarrow(ii)$ 

    Folgt aus Satz $\ref{prop:koordinatenringfunktor}$.
  \item $(iii)\Rightarrow(ii)$ 

    $U:=Y$ und Satz $\ref{prop:fkt-auf-basis}$.
  \item $(ii)\Rightarrow(iii)$

    Betrachte $\Gamma(f):\Gamma(Y)\rightarrow\Gamma(X)$, $h\mapsto h\circ f$.
    Aufgrund des Verklebungsaxioms reicht es, die Bedingung für $U$ von
    der Form $D(g)$ zu zeigen; hier gilt:
    \[
      f^{-1}(D(g))=\{x\in X\mid\underbrace{g(f(x))}_{=\Gamma(f)(g)(x)}\neq0\}=D(g \circ f)
    \]
    Deswegen induziert $\Gamma(f)$:
    \begin{align*}
      h & \longmapsto h\circ f\\
      \mathcal{O}_{Y}(D(g)) & \longrightarrow\mathcal{O}_{X}(D(g\circ f))\\
        & \shortparallel\\
      \Gamma(Y)_{g} & \longrightarrow\Gamma(X)_{g \circ f}\\
      \frac{h}{g^{n}} & \longmapsto\frac{h\circ f}{(g\circ f)^{n}}
    \end{align*}
    mit $h\circ f, g\circ f\in\Gamma(X)$ nach Voraussetzung.

  \end{itemize}
\end{proof}
Insgesamt erhalten wir:
\begin{thm}[orig. 36]
  \label{thm:aequivalenz-alg-mengen-aff-varietaeten}
  Die obige Konstruktion definiert einen volltreuen Funktor
  \[
    \text{\{irreduzible aff. alg. Mengen über }k\}\rightarrow\{\text{Räume mit Funktionen über }k\}.
  \]
\end{thm}




\part*{Prävarietäten}

\textbf{Ziel.} Klasse der affin-algebraischen Mengen, aufgefasst
als Räume mit Funktionen durch Verkleben vergrößern.

$(X,\mathcal{O}_{X})$ heißt \textbf{zusammenhängend}, falls $X$
als topologischer Raum zusammenhängend ist.

\section{Definition von Prävarietäten}
\label{sec:def-praevarietaet}
\begin{defn}[orig. 37]
  \label{def:affine-varietaet}
  Eine \textbf{affine Varietät}\index{affine Varietät} ist ein Raum
  mit Funktionen, der isomorph zu dem Raum mit Funktionen assoziiert zu einer irreduziblen affin-algebraischen Menge ist.
\end{defn}
% 
\begin{defn}[orig. 38]
  \label{def:praevarietaet}
  Eine \textbf{Prävarietät} ist ein zusammenhängender Raum mit Funktionen
  $(X,\mathcal{O}_{X})$, für den eine \emph{endliche }Überdeckung $X=\bigcup_{i=1}^{n}U_{i}$ durch offene Teilmengen $U_i \subseteq X$
  existiert, d.d. $\forall i=1,\ldots,n$ $(U_{i},\mathcal{O}_{X|_{U_{i}}})$
  eine affine Varietät ist.  Insbesondere sind affine Varietäten Prävarietäten!

  Ein \textbf{Morphismus von Prävarietäten} ist ein Morphismus der entsprechenden Räume mit Funktionen.
\end{defn}



Später sehen wir: Varietät = ,,separierte Prävarietät``. Affine
Varietäten sind stets ,,separiert``, daher braucht man nicht von
,,affinen Prävarietäten`` zu reden. Ist $X$ eine affine Varietät,
so schreiben wir oft $\Gamma(X)$ für $\mathcal{O}_{X}(X)$ (vgl. Satz
\ref{prop:fkt-auf-basis}).

Unter einer \textbf{offenen affinen Überdeckung} einer Prävarietät
$X$ verstehen wir eine Famile von offenen affinen Unterräumen mit Funktionen
$U_{i}\subseteq X$, $i\in I$ die affine Varietäten sind, d.d. $X=\bigcup_i U_{i}$.

\section{Vergleich mit differenzierbaren/komplexen Mannigfaltigkeiten}
\label{sec:vergleich-mit-mannigfaltigkeiten}

\paragraph{Differential/Komplexe Geometrie}

Mannigfaltigkeiten werden via Kartenabbildungen mit differenzierbaren/holomorphen
Übergangsabbildungen definiert (hier problematisch, da offene Teile
affiner algebraischer Mengen i.A. keine solche Struktur besitzen.)
Jedoch:
\begin{align*}
  \text{\{differenzierbare Mfgkt.\}} & \longrightarrow\text{\{Räume mit Fkt.}/\mathbb{R}\}\\
  X & \longmapsto(X,\mathcal{O}_{X})\\
                                     & \phantom{\longmapsto}\mathcal{O}_{X}(U):=C^{\infty}(U,\mathbb{R}),\ U\subseteq X\text{ offen}
\end{align*}

ist ein volltreuer Funktor. Daher kann man differenzierbare Mannigfaltigkeiten
auch als diejenigen Räume mit Funktionen über $\mathbb{R}$ definieren,
für die $X$ Hausdorff ist, und so dass eine offene Überdeckung durch
solche Räume mit Funktionen über $\mathbb{R}$ existiert, die in obiger
Weise offene Teilmengen von $\mathbb{R}^{n}$ zugeordnet sind. (Analog
bei komplexen Mannigfaltigkeiten.)



\section{Topologische Eigenschaften von Prävarietäten}
\label{sec:topologische-eigenschaften-von-praevarietaeten}
\begin{lem}
  \label{lem:bijektion-irred-teilraeume}
  Für einen topologischen Raum $X$ und $U\subseteq X$ offen haben
  wir eine Bijektion
  \begin{align*}
    \{Y\subseteq U\text{ irred. abg.}\} & \longleftrightarrow\{Z\subseteq X\text{ irred. abg. mit }Z\cap U\neq\emptyset\}\\
    Y & \longmapsto\overline{Y}\text{ (Abschluss in }X)\\
    Z\cap U & \longmapsfrom Z
  \end{align*}
\end{lem}
\begin{proof}
  Lemma \ref{lem:irreduzibel-abschluss}: $Y\subseteq X$ irreduzibel
  $\Leftrightarrow\overline{Y}\subseteq X$ irreduzibel.

  $Y\subseteq U$ abgeschlossen $\Leftrightarrow\exists A\subseteq X$
  abgeschlossen: $Y=U\cap A$.

  $\Rightarrow Y\subseteq\overline{Y}\subseteq A$ $\Rightarrow Y=U\cap\overline{Y}$

  $Y$ irreduzibel in $U$ $\Rightarrow Y$ irreduzibel in $X$

  $\Rightarrow$ $\overline{Y}$ irreduzibel nach \ref{lem:irreduzibel-abschluss}

  $\Rightarrow Y\mapsto\overline{Y}\mapsto\overline{Y}\cap U=Y$. $\checkmark$

  $\emptyset\neq Z\cap U \subseteq Z$
  damit dicht da $Z$ irreduzibel (Satz \ref{prop:charakterisierung-irreduzibel} ii. und v.)

  Also ist die Abbildung $\leftarrow$ wohldefiniert.

  $\Rightarrow\overline{Z\cap U}=Z$ 
\end{proof}
\begin{prop}
  \label{prop:praevarietaeten-noethersch-irreduzibel}
  Sei $(X,\mathcal{O}_{X})$ eine Prävarietät.

  Dann ist $X$ noethersch (insbesondere quasikompakt) und irreduzibel.
\end{prop}
\begin{proof}
  Sei $X=\bigcup_{i=1}^{n}$ endliche offene aff. Überdeckung und $X\supseteq Z_{1}\supseteq Z_{2}\supseteq\cdots$
  eine absteigende Kette abgeschlossener Teilmengen.

  $\Rightarrow U_{i}\cap Z_{1}\supseteq U_{i}\cap Z_{2}\supseteq\cdots$
  , ist eine absteigende Kette abgeschlossener Teilmengen von $U_{i}$

  $\Rightarrow\forall i$ $\exists n_{i} \in \mathbb{N}$: $U_{i}\cap Z_{n_{i}}=U_{i}\cap Z_{i+m}$ für alle $m \in \mathbb{N}$.
  Setzen wir $n:=\max n_{i}$, so folgt:

  $\forall i=1,\ldots,n$ $\forall m\geq n$: $U_{i}\cap Z_{m}=U_{i}\cap Z_{m+1}$

  $\Rightarrow(Z_{i})_{i}$ wird stationär da $Z_{m}=\bigcup_{i} U_{i}\cap Z_{m}$.

  $X$ ist demnach noethersch.

  $X$ ist weiter irreduzibel:

  Sei $X=X_{1}\cup\cdots\cup X_{n}$ die Zerlegung in irreduzible Komponenten.

  Angenommen es wäre $n\geq2$.

  $\Rightarrow\exists i_{0}\in\{2,\ldots,n\}$: $X_{1}\cap X_{i_{0}}\neq\emptyset$.
  (Andernfalls gilt: $X=X_{1}\sqcup\underbrace{X\backslash X_{1}}_{=X_{2}\cup\cdots\cup X_{n}\text{ abg.}}$, im Widerspruch dazu, dass $X$ zusammenhängend ist.)

  Sei ohne Einschränkung $i_{0}=2$. Sei $x\in X_{1}\cap X_{2}$, $x\in U\subseteq X$ offen, affin (d.h. affine Varietät).

  $U$ irreduzibel $\Rightarrow\overline{U}$ (Abschluss in $X$) $\subseteq X_{j}$
  für ein $j\in\{1,\ldots,n\}$

  \textbf{Jedoch}: Da $x\in X_{i}\cap U\subseteq U$ irreduzibel ist, ist $\underbrace{\overline{X_{i}\cap U}}_{\subseteq\overline{U}\subseteq X_{i}}=X_{i}$,
  $i=1,2$

  $\Rightarrow X_{1},X_{2}\subseteq X_{j}$. Widerspruch zu maximale
  Komponente.
\end{proof}




\section{Offene Untervarietäten}
\label{sec:offene-untervarietaeten}

Offene Teilmengen von affinen Varietäten (und allgemeiner beliebigen
Prävarietäten) sind wieder Prävarietäten. (aber i.A. nicht affin!)
\begin{lem}[orig. 41]
  \label{lem:basisoffene-teilmengen-sind-affin}
  Sei $X$ eine affine Varietät, $f\in\mathcal{O}_{X}(X)$, $D(f)\subseteq X$. Die Lokalisierung
  von $\Gamma(X)=\mathcal{O}_{X}(X)$ an $f$,
  \[
    \Gamma(X)_{f}=\Gamma(X)[T]/(Tf-1)
  \]

  ist eine integre, endlich erzeugte $k$-Algebra. $(Y,\mathcal{O}_{Y})$
  bezeichne die zugehörige affine Varietät. Dann gilt:
  \[
    (D(f),\mathcal{O}_{X}|_{D(f)})\cong(Y,\mathcal{O}_{Y})
  \]

  als Räume mit Funktionen, d.h. $(D(f),\mathcal{O}_{X|_{D(f)}})$ ist
  selbst affine Varietät.
\end{lem}
\begin{proof}
  $\mathcal{O}_{X}(D(f))=\mathcal{O}_{X}(X)_{f}$ muss affiner
  Koordinatenring von $(D(f),\mathcal{O}_{X|_{D(f)}})$
  sein, wenn letzterer Raum von Funktionen affin ist. $X\subseteq\mathbb{A}^{n}(k)$
  korrespondiert zu dem Radikalideal:
  \begin{align*}
    \mathfrak{a} & :=I(X)\unlhd k[T_1, \ldots, T_n]\ \subseteq\ \mathfrak{a}':=(\mathfrak{a},fT_{n+1}-1)\subseteq k[T_{1},\ldots,T_{n+1}]
  \end{align*}

  mit Koordinatenringen:
  \begin{align*}
    \Gamma(X) & =k[T_{1},\ldots,T_{n}]/\mathfrak{a}\\
    \Gamma(Y) & =\Gamma(X)_{f}=(k[T_{1},\ldots,T_{n}]/\mathfrak{a})[T_{n+1}]/(T_{n+1}f-1)\\
              & \cong k[T_{1},\ldots,T_{n+1}]/\mathfrak{a}'
  \end{align*}

  Für $Y=V(\mathfrak{a}')\subseteq\mathbb{A}^{n+1}(k)$ induziert die
  Abbildung
  \[
    \xymatrix{Y\subseteq\mathbb{A}^{n+1}(k)\ar@{-->}[d] & (x_{1},\ldots,x_{n+1})\ar@{|->}[d] & T_{i}\\
      X\subseteq\mathbb{A}^{n}(k) & (x_{1},\ldots,x_{n}) & T_{i}\ar@{|->}[u]
    }
  \]

  eine Bijektion $Y\xrightarrow{j} D_{X}(f)$  mit Umkehrabbildung
  $(x_{0},\ldots,x_{n},\frac{1}{f(x_{0},\ldots,x_{n})})\mapsfrom(x_{0},\ldots,x_{n})$
  \begin{claim*}
    $j$ ist Isomorphismus von Räumen mit Funktionen:
    \begin{enumerate}
    \item $j$ ist \emph{stetig} (als Einschränkung einer stetigen Abbildung) $\checkmark$
    \item $j$ ist \emph{offen}: Für $\frac{g}{f^{n}}\in\Gamma(X)_{f} = \Gamma(Y)$ mit $g\in\Gamma(X)$ gilt
      \begin{align*}
        j\left(D_{Y}\left(\frac{g}{f^{n}}\right)\right) & =j\left(D_{Y}(gf)\right) & f\text{ Einheit}\\
                                                        & =D_{X}(gf)\text{ offen}
      \end{align*}

      $\Rightarrow j$ Homömorphismus.
    \item $j$ induziert $\forall g\in\Gamma(X)$ Isomorphismen:
      \begin{align*}
        \mathcal{O}_{X}(D(fg)) & \longrightarrow\Gamma(Y)_{g}\\
        s & \longmapsto s\circ j
      \end{align*}
      mit $\mathcal{O}_{X}(D(fg))=\Gamma(X)_{fg}=\Gamma(X)_{f})_{g}=\Gamma(Y)_{g}$.
      Mit dem Verklebungsaxiom folgt: $j$ ist Morphismus von Räumen mit Funktionen.
    \end{enumerate}
  \end{claim*}
\end{proof}
\begin{prop}[orig. 42]
  \label{prop:offener-teilraum-praevarietaet}
  Sei $(X,\mathcal{O}_{X})$ Prävarietät, $\emptyset\neq U\subseteq X$
  offen. Dann ist $(U,\mathcal{O}_{X}|_{U})$ eine Prävarietät und $U\hookrightarrow X$
  ist Morphismus von Prävarietäten.
\end{prop}
\begin{proof}
  $X$ ist irreduzibel, also folgt mit Satz \ref{prop:charakterisierung-irreduzibel}, dass $U$ zusammenhängend
  ist. Nach Voraussetzung besitzt $X=\bigcup_{i} X_{i}$ eine affine, offene
  Überdeckung. Es folgt:
  \[
    U=\bigcup_{i}(\underbrace{X_{i}\cap U}_{\text{offen in }X_{i}})=\bigcup_{i,j} D_{X_{i}}(f_{i,j})
  \]

  und $D_{X_{i}}(f_{i,j})$ ist eine affine Varietät nach
  Lemma \ref{lem:basisoffene-teilmengen-sind-affin}. Da $X$ noethersch
  ist, folgt mit Lemma \ref{lem:eigenschaften-noethersch}, dass $U$ quasikompakt ist.

  $\Rightarrow$ Es existiert eine endliche Teilüberdeckung, also ist $U$ Prävarietät. $\checkmark$

  Die kanonische Inklusion $i:U\hookrightarrow X$ ist sicher stetig. Für $f\in\mathcal{O}_{X}(V), V \subseteq X$ offen gilt mit dem Einschränkungsaxiom
  \[
    \mathcal{O}_{X}|_{U}(U\cap V)=\mathcal{O}_{X}(U\cap V)\ni f\circ i=f|_{U\cap V}
  \]

  Also ist $i$ Morphismus von Prävarietäten.
\end{proof}
Die offenen affinen Teilmengen einer Prävarietät $X$ ($\hat{=} U\subseteq X$
offen mit $(U,\mathcal{O}_{X}|_{U})$ affine Varietät) bilden eine
Basis der Topologie von $X$, da $X$ durch offene affine Untervarietäten
überdeckt wird und letzere diese Eigenschaft nach Lemma \ref{lem:basisoffene-teilmengen-sind-affin} haben.



\section{Funktionenkörper einer Prävarietät}
\label{sec:funktionenkorper-praevarietaet}
\begin{defn}[orig. 43]
  \label{def:funktionenkoerper-praevarietaet}
  Für eine Prävarietät $X$ sind die rationalen Funktionenkörper aller
  nicht-leeren affin-offenen Teilmengen in natürlicher Weise zu einander
  isomorph. Diesen Körper $K(X)$ nennen wir den \textbf{rationalen Funktionenkörper}von $X$.
\end{defn}
\begin{proof}
  $\emptyset\neq U$, $V\subseteq X$ affine, offene Untervarietäten. Da
  $X$ irreduzibel ist, gilt nach \emph{Satz \ref{prop:charakterisierung-irreduzibel}}:
  \[
    \emptyset\neq U\cap V\subseteq U\text{ offen}.
  \]

  Nach Definition von $\mathcal{O}_{X}$ ist 
  \[
    \mathcal{O}_{X}(U)\subseteq\mathcal{O}_{X}(U\cap V)\subseteq K(U)=\text{Quot}(\mathcal{O}_{X}(U)).
  \]

  Das impliziert $\text{Quot}(\mathcal{O}_{X}(U\cap V))=K(U)$. Aus
  Symmetriegründen ist aber damit auch bereits $K(V)=\text{Quot}(\mathcal{O}_{X}(U\cap V))$.
\end{proof}
\begin{rem}[orig. 44]
  \label{rem:funktionenkoerper-nicht-funktoriell}
  Bildung des des Funktionenkörpers $K(\cdot)$ ist \textbf{nicht} funktoriell!
  Für $X\rightarrow Y$ Morphismus affiner Varietäten ist die Abbildung
  auf den Koordinatenringen $\Gamma(Y)\rightarrow\Gamma(X)$ i.A. \textbf{nicht}
  injektiv, induziert also keine Abbildung $K(Y)\hookrightarrow K(X)$.

  \emph{Jedoch}: Eine Isomorphie $X\xrightarrow{\sim}Y$ induziert $K(Y)\xrightarrow{\sim}K(X)$.
  Allgemeiner sei $X\xrightarrow{\varphi} Y$ Morphismus mit $\text{im}(\varphi) \subseteq Y$
  offen ($\Rightarrow$ dicht. Später: $X\xrightarrow{\varphi} Y$ \textbf{dominant},
  gdw. $\text{im}(\varphi)\subseteq Y$ dicht) induziert in funktioreller Weise eine
  Abbildung $K(Y)\hookrightarrow K(X)$.
\end{rem}
\begin{prop}[orig. 45]
  \label{prop:charakterisierung-schnitte-praevarietaet}
  Sei $X$ eine Prävarietät, $V\subseteq U\subseteq X$ offen. Dann gilt:

  \begin{enumerate}
  \item $\mathcal{O}_{X}(U)\subseteq K(X)$ ist $k$-Unteralgebra.

  \item $\mathcal{O}_{X}(U)\rightarrow\mathcal{O}_{X}(V)$ ist Inklusion von Teilmengen des Funktionenkörpers $K(X)$.

  \item Insbesondere gilt für $U,V\subseteq X$ offen:
    \[
      \mathcal{O}_{X}(U\cup V)=\mathcal{O}_{X}(U)\cap\mathcal{O}_{X}(V).
    \]
  \end{enumerate}
\end{prop}
\begin{proof}
  \mbox{}
\item[(ii)] Sei $\mathcal{O}_{X}(X)\ni f:X\rightarrow k$. Dann ist $f^{-1}(0)\subseteq X$
  abgeschlossen, da für $W\subseteq X$ affin-offen beliebig gilt, dass
  \[
    f^{-1}(0)\cap W=V(f|_{W}).
  \]
  Dazu macht man sich klar: ,,abgeschlossen`` ist eine lokale Eigenschaft,
  affin-offene $W$ bilden eine Basis der Topologie. 

  $\Rightarrow\mathcal{O}_{X}(U)\hookrightarrow\mathcal{O}_{X}(V)$, $f\mapsto f|_{V}$
  ist injektiv für $\emptyset\neq V\subseteq U\subseteq X$ offen.

  $\Rightarrow V\subseteq f^{-1}(0)$ 

  $\Rightarrow f^{-1}(0)=U$ 

  $\Rightarrow f\equiv0$.
\item[(i)] $U\supseteq W$ affin-offene Untervarietät.
  \[
    \xymatrix{
      \mathcal{O}_{X}(W) \ar@{^(->}[r] & K(W) \text{ } k\text{-Algebren} \\
      \mathcal{O}_{X}(U) \ar@{^(->}[u] \ar@{^(-->}[ru] &
    }
  \]
\item[(iii)] Wir haben folgendes kommutatives Diagramm:
  \[
    \xymatrix{
      {} & \mathcal{O}_{X}(U) \ar@{^(->}[rd] & \\
      \mathcal{O}_X(U\cup V) \ar@{^{(}->}[rd] \ar@{^{(}->}[ru] & {} & \mathcal{O}_X(U\cap V) \\
      {} & \mathcal{O}_X(V) \ar@{^{(}->}[ru] &
    }
  \]
  Nach dem Verklebungsaxiom ist die Sequenz
  \[
    \xymatrix{
      0 \ar[r] & \mathcal{O}_{X}(U\cup V) \ar[r] & \mathcal{O}_{X}(U) \times \mathcal{O}_{X}(V) \ar[r] & \mathcal{O}_{X}(U \cap V) \\
      {} & f \ar@{|->}[r] & (f|_{U}, f|_{V}) & {} \\
      {} & {} & (g,h) \ar@{|->}[r] & g|_{U \cap V} - h|_{U \cap V}
    }
  \]
  exakt.


\end{proof}



\section{Abgeschlossene Unterprävarietäten}
\label{sec:abg-untervarietaeten}

Sei $X$ eine Prävarietät, $Z\subseteq X$ abgeschlossen und irreduzibel.

\textbf{Ziel.} $(Z,\mathcal{O}_{Z}')$ Raum von Funktionen erklären.
Definiere dazu für $U \subseteq Z$ offen:

\[
  \mathcal{O}_{Z}'(U):=\{f\in\text{Abb}(U,k)\mid\forall x\in U\ \exists x\in V\subseteq X\text{ offen},\ g\in\mathcal{O}_{X}(V) \text{ mit } f|_{U\cap V}=g|_{U\cap V}\}
\]

Damit ist $(Z,\mathcal{O}'_{Z})$ Raum von Funktionen (klar!) mit $\mathcal{O}'_{X}=\mathcal{O}_{X}$.
\begin{lem}[orig. 46]
  \label{lem:abg-untervarietaeten-affine-varietaeten}
  Seien $X\subseteq\mathbb{A}^{n}(k)$ eine irreduzible, affin-algebraische Menge
  und $Z\subseteq X$ ein irreduzibler abgeschlossener Teilraum. Dann
  ist $(Z,\mathcal{O}_{Z})=(Z,\mathcal{O}'_{Z})$.

  Bezeichne ab jetzt stets $\mathcal{O}_{Z}$ für $\mathcal{O}_{Z'}$.
\end{lem}
\begin{proof}
  $Z\subseteq X$ ist in beiden Fällen mit der Teilraumtopologie ausgestattet!
  Ferner wissen wir, dass der Morphismus $Z\hookrightarrow X$ affin-algebraischer Mengen einen Morphismus $(Z,\mathcal{O}_{Z})\rightarrow(X,\mathcal{O}_{X})$
  von Prävarietäten induziert. Nach Definition von $\mathcal{O}'$
  folgt dann:
  \[
    \mathcal{O}'_{Z}(U)\subseteq\mathcal{O}_{Z}(U)\quad\text{für }U\subseteq Z\ \text{offen, denn:}
  \] 
  Ist $f \in \mathcal{O}_{Z}'(U)$ und $x \in U$ so existieren nach Definition eine offene Umgebung $x \in V_{x} \subseteq X$ und ein $g \in \mathcal{O}_{X}(V_{x})$ d.d. $f|_{U \cap V_{x}} = g|_{U \cap V_{x}}$. Damit gilt $g|_{Z \cap V_x} \in \mathcal{O}_{Z}(Z \cap V_{x})$. Mit dem Verklebungsaxiom erhalten wir also $f \in \mathcal{O}_{Z}(U)$.


  Sei $f\in\mathcal{O}_{Z}(U)$ und $x\in U$ beliebig. Es folgt: $\exists h\in\Gamma(Z)$
  mit $x\in D(h)\subseteq U$ und
  \[
    f|_{D(h)}=\frac{g}{h^{n}}\in\Gamma(Z)_{h}=\mathcal{O}_{Z}(D(h))
  \]

  für $n\geq0$ und $g\in\Gamma(Z)$ geeignet. Lifte $g,h\in\Gamma(Z)\twoheadleftarrow\Gamma(X)$
  zu $\overline{g},\overline{h}\in\Gamma(X)$ und setze $V:=D(\overline{h})\subseteq X$.

  $\Rightarrow x\in V$, $\frac{\overline{g}}{\overline{h}^{n}}\in\mathcal{O}_{X}(D(\overline{h}))$
  und $f|_{U\cap V}=\frac{\overline{g}}{\overline{h}^{n}}|_{U\cap V}$.

  $\Rightarrow f\in\mathcal{O}'_{Z}(U)$.
\end{proof}
\begin{cor}[orig. 47]
  \label{cor:abg-untervarietaeten-sind-praevarietaeten}
  Wenn $X$ eine Prävarietät ist, und $Z\subseteq X$ irreduzibel und abgeschlossen, dann ist $(Z,\mathcal{O}_{Z})$ ebenfalls eine Prävarietät.
\end{cor}
\begin{proof}
  Es ist $X=\bigcup_{i}X_{i}$ für eine endliche affin-offene Überdeckung $(X_{i})_{i}$.
  Damit ist 
  \[
    Z=\bigcup_{i}\left(Z\cap X_{i}\right) :=\bigcup_{i} Z_{i}
  \]
  

  mit $(Z_{i}, \mathcal{O}_{Z_{i}})$ affine Varietät nach Lemma $\ref{lem:abg-untervarietaeten-affine-varietaeten}$.
\end{proof}




\section*{Beispiele (Projektiver Raum und projektive Varietäten)}

\section{Homogene Polynome}
\label{sec:homogene-polynome}
\begin{defn}[orig. 48]
  \label{def:homogen}
  Ein Polynom $f\in k[X_{0},\ldots,X_{n}]$ heißt \textbf{homogen vom
    Grad}\index{homogen} $d\in\mathbb{Z}_{\geq0}$, falls $f$ die Summe
  von Monomen von Grad $d$ ist. (Insbesondere ist für jedes $d$ das
  Nullpolynom homogen von Grad $d$.)

  \emph{Es bezeichne} $k[X_{0},\ldots,X_{n}]_{d}$ den $k$-Untervektorraum
  der Polynome \textbf{homogen vom Grad $d$}, $k[X_{0}, \ldots, X_{n}]_{\leq n}$ den $k$-Untervektorraum \textbf{aller Polynome vom Grad $\leq n$}.
\end{defn}
\begin{rem}[orig. 49]
  \label{rem:charakterisierung-homogen}
  Da \#$k$ unendlich ist, ist $f$ homogen vom Grad $d$
  $\Leftrightarrow f(\lambda x_{0},\ldots,\lambda x_{n})=\lambda^{d}f(x_{0},\ldots,x_{n})$
  $\forall x_{0},\ldots,x_{n}\in k$, $\lambda\in k^{\times}$. 

  Es gilt: $k[X_{0},\ldots X_{n}]=\bigoplus_{d\geq0}k[X_{0},\ldots,X_{n}]_{d}$.
\end{rem}
\begin{lem}[orig. 50]
  \label{lem:homogenisierung-dehomogenisierung}
  Für $i\in\{0,\ldots,n\}$ und $d\geq0$ haben wir bijektive $k$-lineare
  Abbildungen
  \begin{align*}
    k[X_{0},\ldots,X_{n}]_{d} & \longrightarrow k[T_{0},\ldots,\hat{T}_{i},\ldots,T_{n}]_{\leq d}\\
    f & \overset{\Phi_{i}^{d}}{\longmapsto}f(T_{0},\ldots,\underbrace{1}_{i},\ldots,T_{n})\\
    X_{i}^{d}g\left(\frac{X_{0}}{X_{i}},\ldots,\hat{\frac{X_{i}}{X_{i}}},\ldots,\frac{X_{n}}{X_{i}}\right) & \overset{\Psi_{i}^{d}}{\longmapsfrom}g
  \end{align*}

  \textbf{Dehomogenisierung }bzw. \textbf{Homogenisierung.}
\end{lem}
\begin{proof}
  Es reicht, $\Psi_{i}^{d}\circ\Phi_{i}^{d}=\text{id}$, $\Phi_{i}^{d}\circ\Psi_{i}^{d}=\text{id}$
  auf Monomen nachzurechnen, da alle Abbildungen $k$-linear sind. 
\end{proof}
Oft ist es nützlich, 
$k[T_{0},\ldots,\hat{T_{i}},\ldots,T_{n}]$ mit $k\left[\frac{X_{0}}{X_{i}},\ldots,\hat{\frac{X_{i}}{X_{i}}},\ldots,\frac{X_{n}}{X_{i}}\right] \hookrightarrow k(X_{0},\ldots,X_{n})$ zu identifizieren.




\section{Definition des projektiven Raumes}
\label{sec:def-projektiver-raum}

Seien $X_{1}=X_{2}=\mathbb{A}^{1}$, $\tilde{U}_{1}\subseteq X_{1}, \tilde{U}_{2}\subseteq X_{2}$ mit $\tilde{U}_{1} = \tilde{U}_{2} = \mathbb{A}^{1}\setminus\{0\}$.
\begin{align*}
  \tilde{U}_{1} & \overset{\sim}{\longrightarrow}\tilde{U}_{2}\\
  x & \longmapsto\frac{1}{x}
\end{align*}

Verkleben von $X_{1}$ und $X_{2}$ entlang $\tilde{U}_{1} \overset{\sim}\longrightarrow \tilde{U}_{2}$ liefert die \textbf{projektive Gerade}

\[
  \mathbb{P}^{1}=\mathbb{A}^{1}\cup\{\infty\}=U_{1}\cup U_{2}.
\]

Allgemein: 
\[
  \mathbb{P}^{n}=\bigcup_{i=1}^{n+1}U_{i}=\mathbb{A}^{n}\cup\mathbb{P}^{n-1}=\mathbb{A}^{n}\sqcup\mathbb{A}^{n-1}\sqcup\cdots\sqcup\mathbb{A}^{1}\sqcup\mathbb{A}^{0}
\]

\textbf{Idee}: $\mathbb{P}^{2}\supseteq\mathbb{A}^{2}$: Zwei verschiedene
Geraden in $\mathbb{P}^{2}$ schneiden sich genau in einem Punkt.

\textbf{Als Menge}:
\begin{align*}
  \mathbb{P}^{n}(k): & =\{\text{Ursprungsgeraden in }k^{n+1}\}=\{1\text{-dim. }k\text{-Unterräume}\}\\
                     & =(k^{n+1}\backslash\{0\})/k^{\times}
\end{align*}

Man schreibt meist kurz $(x_{0}:\ldots:x_{n})$ für den Repräsentanten der Klasse von $\langle(x_{0},\ldots x_{n})\rangle_{k}$ und nennt $(x_{0}:\ldots:x_{n})$ \textbf{homogene Koordinaten} auf $\mathbb{P}^{n}$.


\emph{Äquivalenzrelation}: 
\[
  (x_{0},\ldots,x_{n})\sim(x_{0}',\ldots,x_{n}')\Leftrightarrow\exists\lambda\in k^{\times}\ \text{mit}\ x_{i}=\lambda x_{i}'\ \forall i.
\]
Die Mengen
\[
  U_{i}:=\{(x_{0}:\ldots:x_{n})\in\mathbb{P}^{n}\mid x_{i}\neq0\}\subseteq\mathbb{P}^{n}(k),\ 0\leq i\leq n
\]

sind wohldefiniert und überdecken $\mathbb{P}^{n}(k)$:
\[
  \mathbb{P}^{n}(k)=\bigcup_{i=0}^{n}U_{i}
\]

Weiter hat man eine Bijektion
\begin{align*}
  U_{i} & \stackrel[1:1]{\kappa_{i}}{\longrightarrow}\mathbb{A}^{n}(k)\\
  (x_{0}:\ldots:x_{n}) & \longmapsto\left(\frac{x_{0}}{x_{i}},\ldots,\frac{\hat{x}_{i}}{x_{i}},\ldots,\frac{x_{n}}{x_{i}}\right)\\
  (t_{0}:\cdots t_{i-1}:1:t_{i+1}:\cdots t_{n}) & \longmapsfrom(t_{0},\ldots,\hat{t}_{i},\ldots,t_{n})
\end{align*}

Über die $\kappa_{i}$ definiert man nun eine Topologie auf $\mathbb{P}^{n}(k)$ durch:

$U\subseteq\mathbb{P}^{n}(k)$ ist genau dann offen, wenn $\kappa_{i}(U\cap U_{i})\subseteq\mathbb{A}^{n}(k)$
offen ist für alle $i$. 

Es gilt: 
\[
  U_{i}\cap U_{j}= D(T_{j})\subseteq U_{i}\text{ offen},\ i\neq j
\]

wenn auf $U_{i}\cong\mathbb{A}^{n}$ die Koordinaten $T_{0},\ldots,\hat{T}_{i},\ldots,T_{n}$
verwendet werden. Damit wird $\mathbb{P}^{n}(k)$ zu einem topologischen
Raum, der durch die $U_{i}$, $0\leq i\leq n$, offen überdeckt wird.

\subsection{Reguläre Funktionen}
\label{subsec:regulaere-fkt-auf-projektivem-raum}

Sei $U\subseteq\mathbb{P}^{n}(k)$ eine beliebige offene Teilmenge.
Die regularären Funktionen auf $U$ sind definiert als
\[
  \mathcal{O}_{\mathbb{P}^{n}}(U) :=\{f\in\text{Abb}(U,k)\mid f|_{U\cap U_{i}}\in\mathcal{O}_{U_{i}}(U\cap U_{i})\}\qquad\forall i\in\{0,\ldots,n\}
\]

wobei wir die $U_{i}$ via $\kappa_{i}$ implizit als Raum mit Funktionen auffassen. Insgesamt erhalten wir:
\[
  \mathbb{P}^{n}(k)=(\mathbb{P}^{n}(k),\mathcal{O}_{\mathbb{P}^{n}})
\]

als Raum mit Funktionen.
\begin{prop}[orig 51]
  \label{prop:charakterisierung-reg-fkt-projektiver-raum}
  Für $U\subseteq\mathbb{P}^{n}$ offen gilt: $\mathcal{O}_{\mathbb{P}^{n}}(U)=\{f:U\rightarrow k\mid\forall x\in U$:
  $\exists x\in V\subseteq U$ offen, $d\geq 0$ und $g,h\in k[X_{0},\ldots,X_{n}]_{d}$
  homogen vom selben Grad $d$, d.d. $\forall v\in V$: $h(v)\neq0$ und
  $f(v)=\frac{g(v)}{h(v)}\}$ 
\end{prop}
Wohldefiniertheit: Sei $v=(x_{0}:\ldots:x_{n})$.
\[
  f(\lambda x_{0},\ldots,\lambda x_{n})=\frac{g(\lambda x_{0},\ldots,\lambda x_{n})}{h(\lambda x_{0},\ldots,\lambda x_{n})}=\frac{\lambda^{d}g(x_{0},\ldots,x_{n})}{\lambda^{d}h(x_{0},\ldots,x_{n})}=f(x_{0},\ldots,x_{n})
\]

\begin{proof}
  \mbox{}
  \begin{itemize}
  \item[,,$\subseteq$``:] Sei $f\in\mathcal{O}_{\mathbb{P}^{n}}(U)$. Dann ist $f|_{U\cap U_{i}}\in\mathcal{O}_{U_{i}}(U\cap U_{i})$.
    Es folgt:
    \[
      f=\frac{\tilde{g}}{\tilde{h}},\ \tilde{g},\tilde{h}\in k[T_{0},\ldots,\hat{T}_{i},\ldots,T_{n}]
    \]
    Definiere $d:=\max\{\deg(\tilde{g}),\deg(\tilde{h})\}$. Homogenisiere:
    \[
      g:=\psi_{i}^{d}(\tilde{g}),\ h:=\psi_{i}^{d}(\tilde{h})
    \]
    $\Rightarrow f=\frac{g}{h}$ lokal. 
    \begin{align*}
      f(x) & =\frac{\tilde{g}}{\tilde{h}}(\kappa_{i}(x))\\
      f((x_{0}:\cdots:x_{n})) & =\frac{\tilde{g}\left(\frac{x_{0}}{x_{i}},\ldots,\frac{\hat{x_{i}}}{x_{i}},\ldots,\frac{x_{n}}{x_{i}}\right)}{\tilde{h}\left(\frac{x_{0}}{x_{i}},\ldots,\frac{\hat{x_{i}}}{x_{i}},\ldots,\frac{x_{n}}{x_{i}}\right)}\\
           & =\frac{x_{i}^{d}\tilde{g}(\ldots)}{x_{i}^{d}\tilde{h}(\ldots)}\\
           & =\frac{\psi_{i}^{d}(\tilde{g})(\ldots)}{\psi_{i}^{d}(\tilde{h})(\ldots)}=\frac{g}{h}(x_{0}:\ldots:x_{n})
    \end{align*}
  \item[,,$\supseteq$``:] Sei $f$ in der rechten Menge, fixiere $i\in\{0,\ldots,n\}$. Nach Voraussetzung ist $f$ lokal
    auf $U\cap U_{i}$ von der Form $f = \frac{g}{h}$, $g,h\in k[X_{0},\ldots,X_{n}]_{d}$, $d\geq 0$ geeignet. Definiere:
    \[
      \tilde{g}_{i}:=\frac{g}{X_{i}^{d}},\ \tilde{h}:=\frac{h}{X_{i}^{d}}\in k\left[\frac{X_{0}}{X_{i}},\ldots\hat{\frac{X_{i}}{X_{i}}},\ldots,\frac{X_{n}}{X_{i}}\right]
    \]
    $\Rightarrow f$ ist lokal von der Form: $\frac{\tilde{g}}{\tilde{h}}$,
    $\tilde{g},$$\tilde{h}\in k[T_{0},\ldots,\hat{T_{i}},\ldots T_{n}]$.

    $\Rightarrow f|_{U\cap U_{i}}\in\mathcal{O}_{U_{i}}(U\cap U_{i})$, also $f \in \mathcal{O}_{\mathbb{P}^{n}}(U)$.

  \end{itemize}
\end{proof}
\begin{cor}[orig. 52]
  \label{cor:affine-ueberdeckung-des-projektiven-raumes}
  Für $i\in\{0,\ldots,n\}$ induziert
  \[
    U\xrightarrow[\cong]{\kappa_{i}}\mathbb{A}^{n}(k)
  \]

  einen Isomorphismus
  \[
    (U_{i},\mathcal{O}_{\mathbb{P}^{n}|_{U_{i}}})\xrightarrow{\cong}\mathbb{A}^{n}(k)
  \]

  von Räumen mit Funktionen. Insbesondere ist $\mathbb{P}^{n}(k)$ eine
  Prävarietät.
\end{cor}
\begin{proof}
  Zu zeigen: $\forall U\subseteq U_{i}$ offen gilt
  \[
    \mathcal{O}_{\mathbb{P}^{n}(k)}(U)=\mathcal{O}_{U_{i}}(U)=\{f:U\rightarrow k\mid f\in\mathcal{O}_{U_{i}}(U)\}
  \]

  d.h. auf der rechten Seite muss die Bedingung nur für das fixierte
  $i$ überprüft werden. Dies folgt aus dem Beweis von Satz \ref{prop:charakterisierung-reg-fkt-projektiver-raum}.
\end{proof}
Damit identifizieren sich die Funktionenkörper 
\[
  K(\mathbb{P}^{n}(k))=K(U_{i})=k\left(\frac{X_{0}}{X_{i}},\ldots,\frac{X_{n}}{X_{i}}\right)
\]

\begin{prop}[orig. 53]
  \label{prop:globale-schnitte-des-proj-raums}
  $\mathcal{O}_{\mathbb{P}^{n}(k)}(\mathbb{P}^{n}(k))=k$. Insbesondere
  ist $\mathbb{P}^{n}$ für $n\geq1$ \textbf{keine} affine Varietät.
  (Da der $k$-Algebra $A = k$ ja $\mathbb{A}^{0}(k)=\{\text{pt}\}$ als
  affine Varietät entspricht.)
\end{prop}
\begin{proof}
  $k\subseteq\mathcal{O}_{\mathbb{P}^{n}(k)}(\mathbb{P}^{n}(k))$ klar, da konstante Funktionen. Nach Satz \ref{prop:charakterisierung-schnitte-praevarietaet} $(iii)$ gilt:
  \begin{align*}
    \mathcal{O}_{\mathbb{P}^{n}}(\mathbb{P}^{n}) & =\bigcap_{i=0}^{n}\mathcal{O}_{\mathbb{P}^{n}}(U_{i})\subseteq K(\mathbb{P}^{n}(k))\\
                                                 & =\bigcap_{i=0}^{n}k[t_{0},\ldots,\hat{t_{i}},\ldots,t_{n}]=k
  \end{align*}
\end{proof}




\section{Projektive Varietäten}
\label{sec:projektive-varietaeten}
\begin{defn}[orig. 54]
  \label{def:projektive-varietaeten}
  Abgeschlossene Unterprävarietäten eines projektiven Raumes $\mathbb{P}^{n}(k)$
  heißen \textbf{projektive Varietäten}.
\end{defn}
Vorsicht: für $x=(x_{0}:\ldots:x_{n})\in\mathbb{P}^{n}$, $f\in k[X_{0},\ldots,X_{n}]$
ist $f(x_{1},\ldots,x_{n})$ \emph{nicht} wohldefiniert, da von Repräsentaten
abhängig, d.h. $f$ kann \emph{nicht}\textbf{ }als Funktion auf $\mathbb{P}^{n}$
aufgefasst werden. Für \emph{homogene}\textbf{ }Polynome $f_{1},\ldots,f_{n}\in k[X_{0},\ldots X_{n}]$
(nicht notwendig vom selben Grad) können wir demnoch Verschwindungsmengen
definieren:
\[
  V_{+}(f_{1},\ldots,f_{n})=\{(x_{0}:\ldots:x_{n})\in\mathbb{P}^{n}\mid f_{j}(x_{0},\ldots,x_{n})=0\ \forall j\}
\]

Da $V_{+}(f_{1},\ldots,f_{n})\cap U_{i}=V(\Phi_{i}(f_{1}),\ldots,\Phi_{i}(f_{m}))$
ist $V_{+}(f_{1},\ldots,f_{m})$ abgeschlossen in $\mathbb{P}^{n}$.
Ist $V_{+}(f_{1},\ldots,f_{n})$ irreduzibel, so erhalten wir eine
projektive Varietät. In der Tat entstehen alle projektiven Varietäten
auf diese Weise, wie der folgende Satz zeigt:
\begin{prop}[orig. 55]
  \label{prop:charakterisierung-projektive-varietaeten}
  Sei $Z\subseteq\mathbb{P}^{n}(k)$ eine projektive Varietät. Dann
  existieren homogene Polynome $f_{1},\ldots,f_{n}\in k[X_{0},\ldots,X_{n}]$,
  so dass
  \[
    Z=V_{+}(f_{1},\ldots,f_{n})
  \]

  gilt.
\end{prop}
\begin{proof}
  Betrachte: 
  \[
    \begin{array}{cc}
      \\
      \\
    \end{array}
  \]

  $f|_{f^{-1}(U_{i})}:f^{-1}(U_{i})\longrightarrow U_{i}$ ist Morphismus
  von Prävarietäten. Dann ist $f$ selber ein Morphismus von Prävarietäten.
  \begin{align*}
    \overline{Y}:= & Y\cup\{0\}\text{ Abschluss von }Y\text{ in }\mathbb{A}^{n+1}(k)\\
    \mathfrak{A}:= & I(\overline{Y})\subseteq k[X_{0},\ldots,X_{n}]
  \end{align*}

  Behauptung: $\mathfrak{A}$ wird von homogenen Polynomen erzeugt\emph{.
    Denn:} für $g\in\mathfrak{A}$, $g=\sum_{d}g_{d}$ Zerlegung in homogene
  Bestandteile vom Grad $d$. $\overline{Y}$ ist Vereinigung von Ursprungsgeraden
  im $k^{n+1}$, d.h. $\forall\lambda\in k^{\times}$ gilt:
  \[
    g(x_{0},\ldots,x_{n})=0\ \Leftrightarrow\ g(\lambda x_{0},\ldots,\lambda x_{n})=0
  \]

  Beweis durch Widerspruch. Nicht alle $g_{d}$ liegen in $\mathfrak{A}$.

  $\Rightarrow\exists(x_{0},\ldots,x_{n})\in\mathbb{A}^{n+1}(k)$, so
  dass $g(x_{0},\ldots,x_{n})=0$, aber $g_{d_{0}}(x_{0},\ldots,x_{n})\neq0$.

  $\Rightarrow0\,\not\equiv,\sum_{d}g_{d}(x_{0},\ldots,x_{n})T^{d}\in k[T]$

  $\Rightarrow(\exists\lambda\in k^{\times})$ $0\neq\sum_{d}g_{d}(x_{0},\ldots,x_{n})\lambda^{d}=\sum_{d}g_{d}(\lambda x_{0},\ldots,\lambda x_{n})=g(\lambda x_{0},\ldots,\lambda x_{n})=0$.
  Widerspruch.

  $\Rightarrow\mathfrak{A}=(f_{1},\ldots,f_{m})$, $f_{j}$ homogen.

  $\Rightarrow Z=V_{+}(f_{1},\ldots,f_{m})$. 

  \begin{align*}
    Z\ni(x_{0}:\cdots:x_{n}) & \Leftrightarrow(\lambda x_{0},\ldots,\lambda x_{n})\in\overline{Y}\ \forall\lambda\in k^{\times}\text{ und }\neq0\\
                             & \Leftrightarrow f_{i}(x_{0},\ldots,x_{n})=0\ \forall1\leq i\leq n,\ (x_{0},\ldots,x_{n})\in\mathbb{P}^{n}
  \end{align*}

  \rule[0.5ex]{1\columnwidth}{1pt}
\end{proof}
Zu Bemerkung 49 

Nach Satz 51 und Definition von $\mathcal{O}_{Z}'$ folgt: Ist $X$
eine projektive Varietät und $U\subset X$ offen, so können wir 

$\mathcal{O}_{X}(U)=\{f:U\rightarrow k\mid\forall x\in U\ \exists x\in V\underset{\text{offen}}{\subset}U,\ g,h\in k[X_{0},\ldots,X_{n}]$
homogen vom gleichen Grad mit $h(v)\neq0,\ f(v)=\frac{g(v)}{h(v)},\ \forall v\in V\}$.
({*}) 

Insbesondere gilt:
\begin{prop}[orig. 56]
  \label{prop:charakterisierung-morphismen-proj-varietaeten}
  Seien $V\subseteq\mathbb{P}^{m}(k)$, $W\subset\mathbb{P}^{n}(k)$
  projektive Varietäten und
  \[
    V\subseteq\mathbb{P}^{m}(k)\xrightarrow{\phi}W\subseteq\mathbb{P}^{n}(k)
  \]

  eine Abbildung. Dann ist $\phi$ eine Morphismus genau dann, wenn
  es zu jedem $x\in V$ eine offene Menge $x\in U_{x}\subset V$ und
  homogene Polynome $f_{0},\ldots,f_{n}\subseteq k[X_{0},\ldots,X_{m}]$
  vom selben Grad existiert mit
  \[
    \phi(y)=(f_{0}(y),\ldots,f_{n}(y))\quad\forall y\in U_{x}
  \]
\end{prop}
\begin{proof}
  \mbox{}
  \begin{itemize}
  \item ``$\Rightarrow$'', Übung.
  \item ``$\Leftarrow$''.
    \begin{enumerate}
    \item $\phi$ stetig: Sei $Z\subseteq W$ abgeschlossen. Ohne Einschränkung
      $Z=V_{+}(g)\cap W$ für ein homogenes Polynom $g$. Dann berechnet
      sich das Urbild
      \[
        \phi^{-1}(Z)=V_{+}(g\circ\phi)\cap V.
      \]
      Auf $U_{x}$, $x\in V$, ist $g\circ\phi$ als homogenes Polynom in
      $X_{0},\ldots,X_{n}$ gegeben. 

      $\Rightarrow V(g\circ\phi)\cap U_{x}=\phi^{-1}(Z)\cap U_{x}$ abgeschlossen
      in $U_{x}$ für alle $x$.

      $\Rightarrow\phi^{-1}(Z)\subseteq V$ abgeschlossen.
    \item Zu zeigen: $\forall W'\subseteq W$ offen, $g\in\mathcal{O}_{W}(W')$
      ist $g\circ\phi\in\mathcal{O}_{V}(\phi^{-1}(W'))$.

      $\Rightarrow$ ({*}) Es ex. eine offene Umgebung $W_{y}$ in $W'$
      mit $g=\frac{h}{q}$ auf $W_{y}$, $h,q$ homogen vom Grad $d$.

      $\Rightarrow\phi_{|U_{x}\cap\phi^{-1}(W_{y}):=\tilde{U}_{x}}$ ist
      auch von dieser Gestalt.

      $\Rightarrow$ ({*}) $\frac{h(f_{0},\ldots,f_{n})}{q(f_{0},\ldots,f_{n})}=g\circ\phi_{|\tilde{U}_{x}}\in\mathcal{O}_{V}(\tilde{U}_{x})$.
    \end{enumerate}
    $\Rightarrow$ (Verkleben) $g\circ\phi\in\mathcal{O}_{V}(\phi^{-1}(V))$.
  \end{itemize}
\end{proof}




\section{Koordinatenwechsel in $\mathbb{P}^{n}$}
\label{sec:koordinatenwechsel-projektiver-raum}

$A=(a_{ij})\in GL_{n+1}(k)$ eine invertierbare $k^{n+1}\rightarrow k^{n+1}$
lineare Abbildung, die Ursprungsgeraden in solche überführt, bzw.
die Äquivalenzrelation respektiert. Wir erhalten Abbildungen:
\begin{align*}
  \mathbb{P}^{n}(k) & \overset{\phi_{A}}{\longrightarrow}\mathbb{P}^{n}(k)\\
  (x_{0}:\ldots:x_{n}) & \longmapsto\left(\sum_{i=0}^{n}a_{0_{i}}x_{i}:\cdots:\sum_{i=0}^{n}a_{n_{i}}x_{i}\right),
\end{align*}

die nach Satz 56 ein Morphismus von Prävarietäten ist. Offensichtlich
gilt für $A,B\in GL_{n+1}(k)$:
\[
  \varphi_{A\cdot B}=\varphi_{A}\circ\varphi_{B}
\]

d.h. $\varphi_{A}$ ist insbesondere wieder ein Isomorphismus, \textbf{der
  durch $A$ bestimmte Koordinatenwechsel des $\mathbb{P}^{n}(k)$}.
\emph{Bezeichne} Aut$(\mathbb{P}^{n}(k))$ die Gruppe der Automorphismen
von $\mathbb{P}^{n}(k)$. Es folgt:
\[
  \varphi_{-}:GL_{n+1}(k)\rightarrow\text{Aut}(\mathbb{P}^{n}(k))
\]

ist ein Gruppenhomomorphismus mit 
\[
  Z:=\ker\varphi=\{\lambda E_{n+1},\ \lambda\in k^{\times}\}
\]

die Untergruppe der Skalarmatrizen. \emph{Später}:
\[
  PGL_{n+1}(k):=GL_{n+1}(k)/Z\twoheadrightarrow\text{Aut}(\mathbb{P}^{n}(k)),\quad Z\cong k^{\times}
\]

die \textbf{projektive lineare Gruppe}.
\begin{example*}
Sei $n=1$. Es ist
\begin{align*}
PGL_{2}(\mathbb{C}) & =\left\{ \begin{array}{rl}
\mathbb{P}^{1}(\mathbb{C}) & \rightarrow\mathbb{P}^{1}(\mathbb{C})\\
(z:w) & \mapsto(az+bw,cz+dw)
\end{array}\right\} \\
 & \leftrightarrow\text{Möbiustransformationen }z\mapsto\frac{az+b}{cz+d}
\end{align*}
\end{example*}




\section{Lineare Unterräume von $\mathbb{P}^{n}$}
\label{sec:lineare-unterraeume-von-pn}

Sei $\varphi:k^{m+1}\rightarrow k^{n+1}$ ein \emph{injektiver} Homomorphismus
von $k$-Vektorräumen. $\varphi$ induziert eine injektive Abbildung
\[
  \imath:\mathbb{P}^{m}(k)\rightarrow\mathbb{P}^{n}(k)
\]

die nach Satz \ref{prop:charakterisierung-morphismen-proj-varietaeten} ein Morphismus von Prävarietäten ist. Das Bild von
$\imath$ ist eine abgeschlossene Untervarietät. Ist $A=(a_{ij})\in M_{l\times(n+1)}$
mit $\text{im}(\varphi)=\ker(k^{n+1}\xrightarrow{A}k^{l})$ und
\[
  f_{i}:=\sum_{j=0}^{n}a_{ij}X_{j}\in k[X_{0},\ldots,X_{n}], \text{ für } i = 1, \ldots, l
\]

so identifiziert $\imath$ den projektiven Raum $\mathbb{P}^{m}(k)$ mit $V_{+}(f_{1},\ldots,f_{l}) \subseteq \mathbb{P}^{n}(k)$.
(Die Abbildung $\imath:\mathbb{P}^{m}(k)\rightarrow V_{+}(f_{1},\ldots,f_{l})$
ist ein Isomorphismus von Prävarietäten, mit Umkehrabbildung induziert von $\varphi^{-1}:\varphi(k^{m+1})\rightarrow k^{m+1}$)
\begin{example*}
  $\mathbb{P}^{m}=V_{+}(X_{m+1},\ldots,X_{n})\subseteq\mathbb{P}^{n}$.
  Solche Unterräume heißen \textbf{lineare Unterräume} (der Dimension
  $m$).

  $m=0$: Punkte

  $m=1$: Geraden

  $m=2$: Ebenen

  $m=n-1$: Hyperebenen in $\mathbb{P}^{n}(k)$.
  \begin{itemize}
  \item Zu zwei Punkten $p\neq q\in\mathbb{P}^{n}(k)$ existiert genau eine
    gerade $\overline{pq}$ in $\mathbb{P}^{n}(k)$, die $p$ und $q$
    enthält, da zu zwei verschiedenen Ursprungsgeraden im $k^{n+1}$ genau
    eine Ebene (in $k^{n+1})$ existiert, die beide Geraden enthält.
  \end{itemize}
\end{example*}
\begin{itemize}
\item Je zwei verschiedene Geraden in $\mathbb{P}^{2}(k)$ schneiden sich
  in genau einem Punkt, da Geraden in $\mathbb{P}^{2}$ Ebenen in $k^{3}$
  entsprechen, und zwei Ebenen sich dort genau in einer Geraden, d.h.
  einem Punkt des $\mathbb{P}^{2}$, schneiden. Dimensionsformel (lineare
  Algebra):
  \[
    \dim _{k}E_{1}\cap E_{2}=-\underbrace{\dim_{k} (E_{1}+E_{2})}_{3}+\underbrace{\dim_{k} E_{1}}_{2}+\underbrace{\dim_{k} E_{2}}_{2}=1
  \]
  \emph{Später}: Verallgemeinerung durch den \emph{Satz von Bézout} für allgemeine Unterprävarietäten
  $V_{+}(f)$.
\end{itemize}



\section{Kegel}
\label{sec:Kegel}

Sei $H\subseteq\mathbb{P}^{n}(k)$ Hyperebene, $p\in\mathbb{P}^{n}(k)\backslash H$,
$X\subseteq H$ abgeschlossene Unterprävarietät.
\[
  \overline{X,p}:=\bigcup_{q\in X}\overline{qp}
\]

heißt \textbf{Kegel von $X$ über $p$}, es handelt sich um einen
abgeschlossenen Untervarietät von $\mathbb{P}^{n}(k)$. Ohne Einschränkung:$H=V_{+}(X_{n})$,
$p=(0:\cdots:1)$ (nach Koordinatenwechsel: $H\cong k^{n}\oplus p\cong k\}=k^{n+1}$.)
Für  
\begin{align*}
  X=V_{+}(f_{1},\ldots,f_{m})\subseteq\mathbb{P}^{n-1}(k)=H, & \quad f_{i}\in k[X_{0},\ldots,X_{n-1}]\\
  \Rightarrow X,p=V_{+}(\tilde{f}_{1},\ldots,\tilde{f}_{m})\subseteq\mathbb{P}^{n}(k), & \quad\tilde{f}_{i}\in k[X_{0},\ldots,X_{n}]
\end{align*}

Verallgemeinerung. Sei $\mathbb{P}^{n}(k)\cong\Lambda\subseteq\mathbb{P}^{n}(k)$
linearer Unterraum, $\psi\subseteq\mathbb{P}^{n}(k)$ komplementärer
linearer Unterraum, d.h. $\Lambda\cap\psi=\emptyset$ und $\mathbb{P}^{n}(k)$
ist der bekannte lineare Unterraum von $\mathbb{P}^{n}(k)$, der $\Lambda$
und $\psi$ enthält. $X\subseteq\psi$ abgeschlossene Unterprävarietät.

\textbf{Kegel von $X$ über $\Lambda$}: $\overline{X,\Lambda}=\bigcup_{q\in X}\overline{q,\Lambda}$,
wobei der von $q$ und $\Lambda$ aufgespannte lineare Unterraum $\overline{q,\Lambda}$
der kleinste Unterraum sei, der $q$ und $\Lambda$ enthält.


\section{Quadriken}
\label{sec:quadriken}

Sei in diesem Abschnitt char$(k)\neq2$.
\begin{defn}[orig. 57]
  \label{def:quadrik}
  Eine abgeschlossene Unterprävarietät $Q\subseteq\mathbb{P}^{n}(k)$
  von der Form $V_{+}(q)$, $0 \neq q\in k[X_{0},\ldots,X_{n}]_{2}$
  heißt \textbf{Quadrik}.
  \[
    Q=V_{+}(q)
  \]

  Zur quadratischen Form $q$ gehört eine assoziierte Bilinearform $\beta$ auf
  $k^{n+1}$ (vgl. lineare Algebra), 
  \[
    \beta(v,w):=\frac{1}{2}(q(v+w)-q(v)-q(w)),\quad v,w\in k^{n+1}
  \]

Es gibt eine Basis von $k^{n+1}$, sodass die Strukturmatrix $B$
von $\beta$ die Gestalt
\[
  B=\begin{pmatrix}
    \begin{array}{ccc}
      1\\
      & \ddots\\
      &  & 1
    \end{array} & 0\\
    0 &
    \begin{array}{ccc}
      0\\
      & \ddots\\
      &  & 0
    \end{array}
  \end{pmatrix}
\]

hat, d.h. Koordinatenwechsel zur Basiswechselmatrix liefert einen
Isomorphismus
\[
  Q\xrightarrow{\sim}V_{+}(X_{0}^{2}+\cdots+X_{r-1}^{2}),\quad r=\text{rk }B
\]
\end{defn}
\begin{lem}[orig. 58]
\label{lem:irreduzibilitaet-quadriken}
	\begin{enumerate}
	
		\item $X_{0}^{2} + \ldots + X_{r-1}^{2}$ ist irreduzibel $\iff$ $r > 2$
		\item $V_{+}(X_{0}^{2} + \ldots + X_{r-1}^{2})$ ist irreduzibel $\iff$ $r \neq 2$
	\end{enumerate}
\end{lem}
\begin{proof}
	\begin{itemize}
		\item $r=0,1: X_{0}^2 = X_{0} \cdot X_{0} \Rightarrow V_{+}(X_{0}^2) = V_{+}(X_{0})$ irreduzibel
		\item $r=2: X_{0}^{2} + X_{1}^{2} = (X_{0} + i\cdot X_{1})\cdot(X_{0} - i \cdot X_{1})$ für $i = \sqrt{-1}$ 
		\item $r>2: $ Angenommen $\sum_{i}{a_{i} X_{i}} \cdot \sum_{j}{b_{j}X_{j}} = X_{0}^{2} + \ldots X_{r-1}^{2}$.\\
		Ausmultiplizieren $+$ Koeffizientenvergleich $\Rightarrow$ Widerspruch.
	\end{itemize}
\end{proof}

\begin{prop}[orig. 59]
  \label{prop:quadrik-in-normalform}
  Ist $r\neq s$, so sind $V_{+}(T_{0}^{2}+\cdots+T_{r-1}^{2})$ und
  $V_{+}(T_{0}^{2}+\cdots+T_{s-1}^{2})$ nicht isomorph.
\end{prop}
\begin{proof}
  Später: Es gibt keinen Koordinatenwechsel von $\mathbb{P}^{n}(k)$,
  der die beiden Mengen miteinander identifiziert, damit auch kein Automorphismus von
  $\mathbb{P}^{n}(k)$.
\end{proof}

\begin{defn}
  \label{def:dim-und-rang-einer-quadrik}
  Eine Quadrik $Q\subseteq\mathbb{P}^{n}(k)$ mit
  $Q\cong V_{+}(T_{0}^{2}+\cdots+T_{r-1}^{2})$, $r\geq1$, hat \textbf{Dimension $n-1$} und den \textbf{Rang $r$}. (nach Satz
  eindeutig!)
\end{defn}

\begin{cor}[orig. 61]
\label{cor:klassifikation-von-quadriken}	
  Zwei Quadriken $Q_{1}$ und $Q_{2}$ sind genau dann isomorph als
  Prävarietäten, wenn sie dieselbe Dimension und denselben Rang haben.
\end{cor}
\begin{proof}
  \mbox{}
  \begin{itemize}
  \item[,,$\Leftarrow$``]
    $Q_{1}\cong V_{+}(T_{0}^{2}+\cdots+T_{n-1}^{2})\cong Q_{2}$ in
    dem selben $\mathbb{P}^{n}$.
  \item[,,$\Rightarrow$``] Für $Q\subseteq\mathbb{P}^{n}(k)$ berechne
    $K(Q)$. Ohne Einschränkung
    $Q=V_{+}(X_{0}^{2}+\cdots+X_{n-1}^{2})$.
    \begin{enumerate}
    \item $r=1$: $V_{+}(X_{0}^{2})=V_{+}(X_{0})=\mathbb{P}^{n-1}(k)$:
      $K(Q)=k(T_{1},\ldots,T_{n-1})$.
    \item $r=2$: reduzibel: Zerlegung in zwei Hyperebenen
      $Z\cong\mathbb{P}^{n-1}$

      $\Rightarrow K(Z)\cong k(T_{1},\ldots,T_{n-1})$.
    \item $r>2$:
      $U=V(1+T_{1}^{2}+\cdots+T_{n-1}^{2})\subseteq\mathbb{A}^{n}(k)$
      ist nichtleere offene affine Teilmenge von $Q$.

      $\Rightarrow K(Q)=K(U)=\text{Quot}(\Gamma(U))=\text{Quot}(k[T_{1},\ldots,T_{n}]/(1+T_{1}^{2}+\cdots+T_{n-1}^{2})$

      $\Rightarrow\text{trgrad}_{k}\ K(Q)=n-1$. 
    \end{enumerate}
  \end{itemize}
\end{proof}
\begin{example}
  $Q$ Quadrik in $\mathbb{P}^{n}$ (vgl. Joe Harris, Seite 34).
  \begin{enumerate}
  \item In $\mathbb{P}^{1}(k)$. 
    \begin{itemize}
    \item \emph{Rang 2:} 2 Punkte, reduzibel. 
    \item \emph{Rang 1:} 1 Punkt (Doppelpunkt). 
    \end{itemize}
  \item In $\mathbb{P}^{2}(k)$.
    \begin{itemize}
    \item \emph{Rang 3:} Glatter Kegel
      $\cong\mathbb{P}^{1}(k)$. $X_{0}^{2}+X_{1}^{2}-X_{2}^{2}=0$
    \item \emph{Rang 2:} 2 verschiedene Geraden, reduzibel. 
    \item \emph{Rang 1:} (Doppel)gerade.
    \end{itemize}
  \item In $\mathbb{P}^{3}(k)$.
    \begin{itemize}
    \item Rang 1: Doppelebene (2-dimensionaler linearer Unterraum)
    \item Rang 2: (insert image)
    \item Rang 3: (insert image)
    \item Rang 4: (insert image)
    \end{itemize}
  \end{enumerate}
\end{example}
Die Quadrik $Q\subseteq\mathbb{P}^{n}(k)$ heißt \textbf{glatt}, falls
$r=n+1$, d.h. falls die Matrix $B$ zu $q$ maximalen Rang hat. Für
$\text{rk}(Q)>3$, $\dim(Q)=d$, ist
$Q\cong\overline{\widetilde{Q},\Lambda}$ Kegel über einer
\textbf{glatten} Quadrik $\widetilde{Q}$, da Dimension $r-2$
bzgl. einer $(d-r+2)$-dimensionalen Unterraums 1.
\begin{itemize}
\item $r=1,2$ ausgeartet.
\item $r=1$. $Q=V_{+}(X_{0}^{2})=V_{+}(X_{0})$ Hyperebenen in
  $\mathbb{P}^{n}(k)$.  Der Unterschied zwischen $V_{+}(X_{0}^{2})$
  und $V_{+}(X_{0})$ ist für eine projektive Varietät $Q$ nicht
  sichtbar, jedoch in der Theorie der Schemata unterscheidbar!
\item $r=2$. $Q=V_{+}(X_{0}^{2}+X_{1}^{2})$ reduzibel, d.h. keine
  Prävarietät in unserem Sinne! Auch hier werden uns Schemata später helfen.
\end{itemize}
\medskip{}

$Q=V_{+}(X_{0}^{2}+X_{1}^{2}+\cdots+X_{n-1}^{2})\subseteq\mathbb{P}^{d+1}$,
$r\leq d+2$

$\tilde{Q}=V_{+}(X_{0}^{2}+\cdots+X_{n-1}^{2})\subseteq\mathbb{P}^{r-1}$
glatt.

$A=\mathbb{P}^{d+1-v}=V_{+}(X_{0},\ldots,X_{n-1})\subseteq\mathbb{P}^{d+1}$

$Q=\overline{\widetilde{Q},\Lambda}$


\chapter{Das Ringspektrum}
\label{chap:das-ringspektrum}

Prävarietäten sind Verklebungen. $k$ algebraisch abgeschlossen:

Affine Varietäten $\leftrightarrow$ integere endlich erzeugte
$k$-Algebren.  Punkte $\hat{=}$ maximale Ideale.

\textbf{Ziel}: Schemata sind Verklebungen.

Affine Schemta $\leftrightarrow$ (kommutative) Ringe. Punkte $\hat{=}$
Primideale.

\textbf{Ziel}: Wir wollen einen Funktor:
\begin{align*}
  A & \longmapsto(\Spec(A),\mathcal{O}_{\Spec(A)})\\
  \text{Ring} & \longrightarrow\text{top. Raum}
\end{align*}

,,Garbe von Funktionen`` verallgemeinert ,,Systeme von Funktionen``
für Raume von Funktionen.

(Insbesondere $k$-Algebren über beliebige Körper $k$!)

(Sei $\varphi:A\rightarrow B$ Ringhomomorphismus,
$\mathfrak{m}\subseteq B$ maximales Ideal. Dann folgt i.A. nicht, dass
$\varphi^{-1}(\mathfrak{m})$ maximal ist. Wir haben also zu wenige
maximale Ideale.)

\section*{Das Ringspektrum als topologischer Raum}

\section{Definition von Spec(A)}

Sei $A$ stets ein kommutativer Ring. Spec(A) =
$\{\mathfrak{p}\subseteq A$ Primideal\}. Sei $M\subset A$.
\begin{align*}
  V(M) & =\{\mathfrak{p}\in\Spec(A)\mid\mathfrak{p}\supset
         M\}=V\{\langle M\rangle\}\\
  V(f) & =V(\{f\})\text{\,für }f\in A
\end{align*}

\begin{lem}[1] Es ist
  \begin{align*}
    \{\text{Ideale in }A\} & \longrightarrow\text{\{Teilmengen in }\Spec(A)\}\\
    \mathfrak{A} & \longmapsto V(\mathfrak{A})
  \end{align*}

  ist eine inklusionsumkehrende Abbildung. Es gilt:
  \begin{enumerate}
  \item $V(0)=\Spec(A)$, $V(1)=\emptyset$.
  \item $V\left(\bigcup_{i\in
        I}\mathfrak{a}_{i}\right)=V\left(\sum_{i\in
        I}\mathfrak{a}_{i}\right)=\bigcap_{i\in I}V(\mathfrak{a}_{i})$
  \item
    $V(\mathfrak{a}\cap\mathfrak{a}')=V(\mathfrak{a}\mathfrak{a}')=V(\mathfrak{a})\cup
    V(\mathfrak{a}')$
  \end{enumerate}
\end{lem}
\begin{proof} \mbox{}
  \begin{itemize}
  \item (1), (2) klar.
  \item
    (3). $\mathfrak{p}\supset\mathfrak{a}\cap\mathfrak{a}'\supset\mathfrak{a}\mathfrak{a}'$.

    $\Rightarrow\mathfrak{p}\supset\mathfrak{a}\mathfrak{a}'$.

    $\Rightarrow$ (Primideal) $\mathfrak{p}\supset\mathfrak{a}$ oder
    $\mathfrak{p}\supset\mathfrak{a}'$.

    $\Rightarrow\mathfrak{p}\supset\mathfrak{a}\cap\mathfrak{a}'$

  \end{itemize}
\end{proof}
\begin{defn} Spec(A) mit der Topologie, dessen abgeschlossene Mengen
  gerade die Mengen der Form $V(\mathfrak{a})$,
  $\mathfrak{a}\subset A$ ein Ideal sind, heißt (Prim)Spektrum von $A$
  (mit der Zariski-Topologie).
  \begin{align*}
    x\in\Spec(A) & \leftrightarrow\mathfrak{p}_{x}\subset A\text{ Primideal}\\
    Y\subset\Spec(A), & \phantom{\leftrightarrow\
                              }I(Y):=\bigcap_{\mathfrak{p}\in Y}\mathfrak{p}
  \end{align*}

  $I(-)$ ist inklusionserhaltend, $I(\emptyset)=A$.
\end{defn}
\begin{prop} $\mathfrak{a}\subset A$ Ideal,
  $Y\subset\Spec(A)$. Dann gilt:
  \begin{enumerate}
  \item $\text{rad }I(Y)=I(Y)$, $V(\mathfrak{a})=V(\text{rad
    }\mathfrak{a})$
  \item $I(V(\mathfrak{a}))=\text{rad}(\mathfrak{a})$,
    $V(I(Y))=\overline{Y}$ (Abschluss in $\Spec(A)$).
  \item Wir haben eine 1:1-Korrespondenz:
    \begin{align*}
      \{\mathfrak{a}\subset A\mid\mathfrak{a}=\rad\mathfrak{a}\}
      & \longrightarrow\{\text{abg. Teilmengen }Y\text{ in }\Spec(A)\}
    \end{align*}
  \end{enumerate}
\end{prop}
\begin{proof} \mbox{}
  \begin{enumerate}
  \item $V(\mathfrak{a})=V(\rad\mathfrak{a})$.
    \begin{itemize}
    \item ,,$\supseteq$``. Klar, da rad
      $\mathfrak{a}\supseteq\mathfrak{a}$.
    \item ,,$\subseteq$``. Aus $f^{r}\in\mathfrak{a}\subseteq\mathfrak{a}$
      folgt $f\in\mathfrak{p}$, da $\mathfrak{p}$ Primideal. Damit:
      $\text{rad }\mathfrak{a}\subset\mathfrak{p}.$
    \end{itemize}
  \item $\text{rad }\mathfrak{a}=\bigcap_{\mathfrak{p}\in
      V(\mathfrak{a})}\mathfrak{p}=IV(\mathfrak{a})$.  Es ist:
    \begin{align*}
      V(\mathfrak{b})\supseteq Y & \Leftrightarrow(\forall\mathfrak{p}\in Y:\
                                   \mathfrak{p}\supset\mathfrak{b})\\
                                 & \Leftrightarrow
                                   I(Y)\supseteq\mathfrak{b}.
    \end{align*} Damit ist $V(I(Y))$ die kleinste abgeschlossene
    Teilmenge, die $Y$ umfasst, d.h. $V(I(Y))=\overline{Y}$.
  \item
  \end{enumerate}
\end{proof}


\section{Topologische Eigenschaften von Spec(A)}

Definiere $D(f):=D_{A}(f):=\Spec(A)\setminus V(f)=\{x \in\Spec A \mid f\notin\mathfrak{p}_{x}\}$,
\begin{align*}
  \text{ev}_{x}:A & \longrightarrow A/\mathfrak{p}_{x}\subseteq \kappa_{x}(A) := \Quot(A/\mathfrak{p}_{x})\\
  f & \longmapsto f(x) := f(\mathfrak{p}_{x}) := f \mod \mathfrak{p}
\end{align*}

Für $x \in D(f)$ gilt dann $f(x) = \text{ev}_{x}(f) \neq 0$.

\textbf{Standard prinzipal offene Mengen}.
\begin{align*}
  D(0) & =\emptyset,\ D(1)=\Spec(A)=D(u),\ u\in A^{\times}\\    
  & D(f)\cap D(g) = D(fg)
\end{align*}

\begin{lem}
\label{lem:charakterisierung-ueberdeckungen-prinzipal}	
Für $f_{i} \in A, i\in I$, $g\in A$ gilt:
  \begin{align*}
    D(g)\subseteq\bigcup_{i\in I}D(f_{i})
    & \Leftrightarrow g^{n}\in\mathfrak{a}=(f_{i},i\in I)\text{ für }n \in \mathbb{N} \text{ geeignet}\\
    & \Leftrightarrow g\in\rad(\mathfrak{a})
  \end{align*}
\end{lem}
\begin{proof} Es gilt:
  \begin{align*}
    D(g)\subseteq\bigcup_{i}D(f)
    & \Leftrightarrow V(g)\supseteq\bigcap_{i} V(f_{i})=V(\mathfrak{a})\\
    & \Leftrightarrow g\in\rad((g))\subseteq\rad(\mathfrak{a}) \text{ nach } \ref{prop:nullstellensatz-primspektrum}
  \end{align*}

  Für $g=1$, folgt:
  \[ \Spec(A)=\bigcup_{i\in I}D(f_{i})\Leftrightarrow\sum_{i\in
      I}Af_{i}=A
  \]
\end{proof}
\begin{prop}
\label{prop:prinzipal-offene-bilden-basis}
Die prinzipal offenen Mengen $D(f)$, $f\in A$, bilden
  eine Basis der Topologie von $\Spec(A)$, und sind
  quasikompakt. Insbesondere ist $\Spec(A)$ quasikompakt.
\end{prop}
\begin{proof} Nach Lemma \ref{lem:zariski-top-auf-spektrum}$.(ii)$ gilt:
  \[
    V(\mathfrak{a})=\bigcap_{f \in\mathfrak{a}}V(f)\Longrightarrow\Spec A\setminus
    V(\mathfrak{a})=\bigcup_{f\in\mathfrak{a}}D(f)\Rightarrow\text{Basis
      der Topologie}
  \]

  Sei $D(g)\subseteq\bigcup_{i}D(f_{i})$.

  \ref{lem:charakterisierung-ueberdeckungen-prinzipal} $\Rightarrow$ $g^{n}=\sum_{i\in I}a_{i}f_{i}$, $a_{i}\in A$
  fast alle 0.

  $\Rightarrow D(g)\subseteq\bigcup_{i\in J}D(f_{i})$ $\forall i\in J\subseteq I$ endlich

  $\Rightarrow D(g)$ quasikompakt.
\end{proof}


\section{Der Funktor $A\protect\mapsto\text{Spec}(A)$}
\label{sec:spec-als-funktor}
\textbf{Ziel:} Wir wollen einen kontravarianten Funktor
\begin{align*}
  \text{\underline{CRing}} & \longrightarrow\text{\underline{Top}}\\
  A & \longmapsto\Spec A
\end{align*}
definieren. Sei $\varphi:A\longrightarrow B$ ein Ringhomomorphismus, $\mathfrak{q}$ Primideal von $B$. Es folgt:
$\varphi^{-1}(\mathfrak{q})\unlhd A$ ist Primideal, denn $A/\varphi^{-1}(\mathfrak{q})\hookrightarrow B/\mathfrak{q}$ ist integer als Unterring eines integren Rings.
Wir erhalten also eine Abbildung
\begin{align*}
  ^{a}\varphi=\Spec\varphi:\ \Spec B & \longrightarrow\Spec A\\
  \mathfrak{q} & \longmapsto\varphi^{-1}(\mathfrak{q})
\end{align*}

\begin{prop} \mbox{}
\label{prop:abbildungen-auf-spektren-sind-stetig}	
  \begin{enumerate}
  \item $(^{a}\varphi)^{-1}(V(M))=V(\varphi(M))$ für
    $M\subseteq\Spec A$ Teilmenge, insbesondere gilt
    $(^{a}\varphi)^{-1}(D(f))=D(\varphi(f))$, $f\in A$.
  \item
    $V(\varphi^{-1}(\mathfrak{b}))=\overline{^{a}\varphi(V(\mathfrak{b}))}$
    für $\mathfrak{b}\unlhd B$ Ideal.
  \end{enumerate}
\end{prop}

\begin{proof} \mbox{}
  \begin{enumerate}
  \item Für $\mathfrak{q}\in\text{Spec }B$ gilt:
    \begin{align}
       \mathfrak{q}\in V(\varphi(M)) \iff \mathfrak{q}\supseteq\varphi(M)
      \iff \varphi^{-1}(\mathfrak{q})\supseteq M \iff \mathfrak{q}\in(^a\varphi)^{-1}(V(M))
    \end{align}
    Weiter:
    \begin{align}
      D(\varphi(f)) & =\Spec(B)\setminus V(\varphi(f)) \\
                    & =\Spec(B)\setminus (^a\varphi)^{-1} (V(f)) \\
                    & = (^a\varphi)^{-1} (D(f))
    \end{align}
     
  \item
    $\overline{^{a}\varphi(V(\mathfrak{b}))}=VI(^{a}\varphi(V(\mathfrak{b})))$
    nach Satz \ref{prop:nullstellensatz-primspektrum}. Nach Definition gilt:
    \begin{align*} I(^{a}\varphi(V(\mathfrak{b}))
      & =\bigcap_{\mathfrak{p}\in^{a}\varphi(V(\mathfrak{b}))}
        \mathfrak{p}=\bigcap_{\mathfrak{q}\in V(\mathfrak{b})}
        \varphi^{-1}(\mathfrak{q})\\ \text{komm. Algebra }
      & =\varphi^{-1}(\rad\mathfrak{b})\\
      & \overset{!}{=}\rad\varphi^{-1}(\mathfrak{b})
    \end{align*}
    Denn: Ohne Einschränkung gelte $\mathfrak{b}=0$,
    $\varphi^{-1}(\mathfrak{b})=\ker\varphi$
    (betrachte $A/\varphi^{-1}(\mathfrak{b})\hookrightarrow B/\mathfrak{b})$. Es
    ist:
    \begin{align*}
      a\in\varphi^{-1}(\sqrt{0})
      & \Leftrightarrow\varphi(a)^{n}=\varphi(a^{n})=0
        \text{ für }n\text{ geeignet}
    \end{align*} $V(\cdot )$
    liefert die Behauptung: $V(\rad\varphi^{-1}
    (\mathfrak{b}))=V(\varphi^{-1}(\mathfrak{b}))$ nach Satz \ref{prop:nullstellensatz-primspektrum}.
  \end{enumerate}
\end{proof}
Insbesondere ist $^{a}\varphi:\Spec B\rightarrow\Spec A$
\emph{stetig}.  Wegen
\[
  ^{a}(\psi\circ\varphi)=\ ^{a}\varphi\ \circ\ ^{a}\psi \text{ und } ^{a}\mathrm{id}_{A} = \mathrm{id}_{\Spec A}
\]
für einen weiteren Ringhomomorphismus $\psi:B\rightarrow C$ ist
$A\mapsto\Spec A$ der gesuchte kontravariante Funktor.

\begin{cor}
\label{cor:charakterisierung-dominanz-auf-spec}	
  $^{a}\varphi$ ist \textbf{dominant},
  d.h. $\im\ (^{a}\varphi)\subseteq\Spec A$ dicht
  $\iff$ Jedes Element in $\ker\varphi$ ist nilpotent:
  $\ker\varphi\subseteq\text{rad}(0)$.
\end{cor}

\begin{prop} \mbox{}
\label{prop:spec-quotienten-lokalisierung}	
  \begin{enumerate}
  \item Ist $\varphi:A\rightarrow B$ ein surjektiver
    Ringhomomorphismus mit $\ker\varphi=:\mathfrak{a}$, dann ist
    $^{a}\varphi$ ein Homöomorphismus von $\Spec B$ auf
    $V(\mathfrak{a})\underset{\text{abg.}}{\subseteq}\Spec A$.
  \item Ist $S$ eine multiplikativ abgeschlossene Teilmenge von $A$,
    und $\varphi:A\longrightarrow S^{-1}A=:B$ die kanonische
    Lokalisierungsabbildung, dann ist $^{a}\varphi$ ein
    Homöomorphismus, von $\Spec S^{-1}A$ auf $\{\mathfrak{p}\in\Spec A\mid
    S\cap\mathfrak{p}=\emptyset\}$.
  \end{enumerate}
\end{prop}

\begin{proof}
  $^{a}\varphi$ injektiv + $\im\ ^{a}\varphi$ ist bekannt aus kommutative Algebra.
  \emph{Ferner}: Für $\mathfrak{q}\in\text{Spec }B$,
  $\mathfrak{b}\unlhd B$ Ideal gilt
  $\mathfrak{q}\supseteq\mathfrak{b}\Leftrightarrow
  \varphi^{-1}(\mathfrak{q})\supseteq\varphi^{-1}(\mathfrak{b})$, also
  \begin{align*}
    ^{a}\varphi(V(\mathfrak{b})) & =V(\varphi^{-1}(\mathfrak{b})),
  \end{align*}
  d.h. $^{a}\varphi$ ist abgeschlossen.
\end{proof}


\section{Beispiele}
\label{sec:beispiele-spektren}
\begin{itemize}
\item $\Spec A=\emptyset\Leftrightarrow A=\{0\}$.
\item $A$ Körper oder Ring mit einem einzigem Primideal: $\Spec
  A=\{\mathfrak{p}\}$.
\item $A$ Artinsch: $\Spec A$ endlich und diskret (da maximale
  Primideale mit den minimalen Primidealen übereinstimmen)

  ($\Spec A=\Spec(A/\sqrt{0})$, $A/\sqrt{0}$ Produkt von Körpern.
  $\Spec(\prod_{i} A_{i})=\coprod_{i}\Spec(A_{i})$
\end{itemize}

\begin{example}
\label{bsp:spec-von-hauptidealring}	
  Sei $A$ Hauptidealring (z.B. $\mathbb{Z}$ oder $K[X]$). Falls
  $\mathfrak{p}$ ein maximales Ideal ist, dann ist
  $\mathfrak{p}=(\pi)$, $\pi$ Primelement in $A$.

  Alle Primideale sind maximal oder 0.

  Abg. Punkte von $\Spec A\leftrightarrow$ Primelemente modulo $A^{\times}$

  $\overline{\{\eta\}}=\Spec A$ für $\eta\in\Spec A$ mit $\mathfrak{p}_{\eta}=(0)$.

  Abgeschlossene Mengen $\Spec A\neq
  V(\mathfrak{a})\overset{0\neq\mathfrak{a}=(f)}{=}V(f)=\{(p_{1}),\ldots,(p_{n})\}$
  falls $f=p_{1}^{e_{1}}\cdots p_{n}^{e_{n}}$, $p_{i}$ paarweise
  verschieden, $e_{i}\geq1$, sind genau \emph{endliche Mengen abgeschlossener Punkte.}

  $g\neq0\neq f$:
  \begin{align*}
    V(f)\cap V(g) & =V(f,g)=V(d), & d=\text{ggT}(f,g)\\
    V(f)\cup V(g) & =V((f)\cap(g))=V(e), & e=\text{kgV}(f,g)
  \end{align*}

  Falls $A$ \emph{lokaler} Hauptidealring ist (also diskreter
  Bewertungsring, der kein Körper ist), dann:
  \[
    \Spec A=\{x,\eta\},\ \mathfrak{p}_{x}\text{ max. Ideal},\
    \mathfrak{p}_{\eta}=(0)
  \]

  $\{\eta\}$ einzige nicht-triviale  offene Menge.
\end{example}
 
\begin{example}
\label{bsp:zusammenhang-affine-varietaeten}	
  Sei $k$ algebraisch abgeschlossener Körper. Affine Varietäten
  $V\leftrightarrow$ endlich erzeugte $k$-Algebren $A$.

  $V=$\{max. Ideale in $A$\} $\subseteq\Spec(A)$

  Topologie auf $V$ ist die Unterraumtopologie von $\Spec(A)$.
\end{example}
 
\begin{example}
\label{bsp:spec-polynomring-ueber-hauptidealring}	
  Sei $R$ Hauptidealring, $A=R[T]$, $X=\Spec(A)$. $R$ faktoriell $\Rightarrow R[T]$ faktoriell, nach dem Satz von Gauß, mit Primidealen:
  \begin{enumerate}
  \item $p\in R$ prim
    \begin{proof}
      $p\in R$ prim $\Rightarrow R/pR$ Körper. Nach Proposition \ref{prop:spec-quotienten-lokalisierung} gilt:
      \[
        \overline{pR[T]}=V(pR[T])\cong\Spec\left(R/pR[T]\right)
      \]
      ein Hauptidealring mit unendlich vielen Elementen. Damit ist $pR[T]$
      \emph{nicht} maximal, sondern
      \[ V(pR[T])=\{pR[T], (f,p), f\in R[T]
          \text{ mit } \overline{f}\in R/p[T]
        \text{ irreduzibel}\}
      \]
    \end{proof}
  \item $f\in R[T]$ primitives Polynom, irreduzibel in $\Quot(R)[T]$
    \begin{proof}
      Sei $f$ primitives, irreduzibles Polynom.
      \begin{itemize}
      \item $l(f)\in R^{\times} \Rightarrow$ (Division mit Rest)
        $R\subseteq R[T]/pR[T]$ ist eine ganze Ringerweiterung
        und ein endl.-erz. freier $R$-Modul vom Rang $\deg(f)$.
        Angenommen, $fR[T]$ ist maximal. Dann ist $R[T]/fR[T]$ ein Körper,
        also $R$ ein Körper (da ganze Ringerweiterung). Widerspruch.
      \item Andernfalls kann $fR[T]$ ein maximales Ideal sein: $R$ habe nur
        endlich viele Primelemente.
        \[
          0\neq a := \prod_{p} p\in R,\ f:=aT-1
        \]
        Es folgt:
        \[
          R[T]/fR[T]\cong R[a^{-1}]=\Quot(R)
        \]
        also ist $fR[T]$ maximal.
      \end{itemize}
    \end{proof}
  \end{enumerate}
\end{example}

%%% Local Variables:
%%% mode: latex
%%% TeX-master: "../AlgGeo1"
%%% End:


\newpage{}
\printindex{}
\end{document}
