%% LyX 2.2.2 created this file.  For more info, see http://www.lyx.org/.
%% Do not edit unless you really know what you are doing.
\documentclass[12pt,english,ngerman]{article}
\usepackage[T1]{fontenc}
\usepackage[latin9]{inputenc}
\usepackage{geometry}
\geometry{verbose,tmargin=2cm,bmargin=2cm,lmargin=2cm,rmargin=2cm}
\pagestyle{headings}
\usepackage{color}
\usepackage{verbatim}
\usepackage{enumitem}
\usepackage{amsmath}
\usepackage{amsthm}
\usepackage{amssymb}
\usepackage{stmaryrd}
\usepackage{stackrel}
\usepackage{makeidx}
\makeindex
\usepackage{setspace}
\usepackage[all]{xy}
\onehalfspacing

\makeatletter

%%%%%%%%%%%%%%%%%%%%%%%%%%%%%% LyX specific LaTeX commands.
%% Because html converters don't know tabularnewline
\providecommand{\tabularnewline}{\\}

%%%%%%%%%%%%%%%%%%%%%%%%%%%%%% Textclass specific LaTeX commands.
\newlength{\lyxlabelwidth}      % auxiliary length 
  \theoremstyle{plain}
  \newtheorem*{question*}{\protect\questionname}
\theoremstyle{plain}
\newtheorem{thm}{\protect\theoremname}
  \theoremstyle{definition}
  \newtheorem{example}[thm]{\protect\examplename}
  \theoremstyle{definition}
  \newtheorem{defn}[thm]{\protect\definitionname}
  \theoremstyle{plain}
  \newtheorem{prop}[thm]{\protect\propositionname}
  \theoremstyle{plain}
  \newtheorem{cor}[thm]{\protect\corollaryname}
  \theoremstyle{plain}
  \newtheorem{lem}[thm]{\protect\lemmaname}
  \theoremstyle{remark}
  \newtheorem{rem}[thm]{\protect\remarkname}
  \theoremstyle{remark}
  \newtheorem*{claim*}{\protect\claimname}

%%%%%%%%%%%%%%%%%%%%%%%%%%%%%% User specified LaTeX commands.
\usepackage{mathtools}
\usepackage{faktor}
\DeclareMathOperator{\rad}{rad}
\renewcommand{\labelenumi}{(\roman{enumi})}
\renewcommand{\labelenumii}{\arabic{enumii}.}

\makeatother

\usepackage{babel}
  \addto\captionsenglish{\renewcommand{\claimname}{Claim}}
  \addto\captionsenglish{\renewcommand{\corollaryname}{Corollary}}
  \addto\captionsenglish{\renewcommand{\definitionname}{Definition}}
  \addto\captionsenglish{\renewcommand{\examplename}{Example}}
  \addto\captionsenglish{\renewcommand{\lemmaname}{Lemma}}
  \addto\captionsenglish{\renewcommand{\propositionname}{Proposition}}
  \addto\captionsenglish{\renewcommand{\questionname}{Question}}
  \addto\captionsenglish{\renewcommand{\remarkname}{Remark}}
  \addto\captionsenglish{\renewcommand{\theoremname}{Theorem}}
  \addto\captionsngerman{\renewcommand{\claimname}{Behauptung}}
  \addto\captionsngerman{\renewcommand{\corollaryname}{Korollar}}
  \addto\captionsngerman{\renewcommand{\definitionname}{Definition}}
  \addto\captionsngerman{\renewcommand{\examplename}{Beispiel}}
  \addto\captionsngerman{\renewcommand{\lemmaname}{Lemma}}
  \addto\captionsngerman{\renewcommand{\propositionname}{Satz}}
  \addto\captionsngerman{\renewcommand{\questionname}{Frage}}
  \addto\captionsngerman{\renewcommand{\remarkname}{Bemerkung}}
  \addto\captionsngerman{\renewcommand{\theoremname}{Theorem}}
  \providecommand{\claimname}{Behauptung}
  \providecommand{\corollaryname}{Korollar}
  \providecommand{\definitionname}{Definition}
  \providecommand{\examplename}{Beispiel}
  \providecommand{\lemmaname}{Lemma}
  \providecommand{\propositionname}{Satz}
  \providecommand{\questionname}{Frage}
  \providecommand{\remarkname}{Bemerkung}
\providecommand{\theoremname}{Theorem}

\begin{document}

\title{Algebraische Geometrie}

\author{Prof. Dr. Venjakob}

\date{Vorlesung 17, 19 Oktober 2018}
\maketitle

\section*{Literatur}
\begin{itemize}
\item G�rtz, Wedhorn. \emph{Algebraic Geometry I}
\item Hartshorne. \emph{Algebraic Geometry}
\item Shajarevich. \emph{Basic Algebraic Geometry 1 u. 2}
\item Grothendieck. \emph{El�ments de g�ometrie alg�brique, EGA I-IV}
\end{itemize}

\paragraph{Kommutative Algebra}
\begin{itemize}
\item Br�ske, Ischebeck, Vogel. \emph{Kommutative Algebra}
\item Kunz. \emph{Einf�hrung in die kommutative Algebra und algebraische
Geometrie}
\end{itemize}
\tableofcontents{}

\newpage{}

\part{Pr�-Variet�ten}


\section{Einführung}
\label{sec:einfuehrung}

\textbf{Algebraische Geometrie}\index{Algebraische Geometrie} kann
man verstehen, als das Studium von Systemen polynomialer Gleichungen
(in mehreren Variabelen). Damit ist die algebraische Geometrie eine
Verallgemeinerung der \textbf{linearen Algebra}, also statt $X$ auch
$X^{n}$, und auch der \textbf{Algebra}, durch Polynome in \emph{mehreren
} Variablen.
\begin{question*}
  Seien $k$ ein (algebraisch abgeschlossener) Körper, und $f_{1},\ldots,f_{m}\in k[T_{1},\ldots,T_{n}]$
  gegeben. Was sind die ``geometrischen Eigenschaften'' der Nullstellenmenge
  \[
    V(f_{1},\ldots,f_{n}):=\{(t_{1},\ldots,t_{n})\in k^{n}\mid f_{i}(t_{1},\ldots,t_{n})=0\ \forall i\}
  \]
\end{question*}
\begin{example}
  \label{bsp:einfuehrung}
  Sei $f=T_{2}^{2}-T_{1}^{2}(T_{1}-1)\in k[T_{1},T_{2}]$. Die Nullstellenmenge
  für $k=\mathbb{R}$ (\emph{aber: }trügerisch, da $\mathbb{R}$ nicht
  algebraisch abgeschlossen!) ist gegeben durch:
\end{example}
\begin{figure}
  \caption{$T_{2}^{2}=T_{1}^{2}(T_{1}-1)=T_{1}^{3}-T_{1}^{2}$}
\end{figure}


\begin{itemize}
\item Dimension 1
\item $(0,0)$ ist singulärer Punkt
\item Alle anderen Punkte besitzen eine eindeutig bestimmte Tangente
\end{itemize}


\begin{figure}[h]
  \label{fig:einfuehrung-glattheit}
  \caption{\textbf{Spitze} und \textbf{Doppelpunkt}}
\end{figure}

Vergleiche mit dem \textbf{Satz über implizite Funktionen}: (Analysis,
Differentialgeometrie) 

$V(f)$ ist lokal diffeomorph zu $\mathbb{R}$ (= reelle Gerade) im
Punkt $(x_{1},x_{2})$ genau dann, wenn die Jacobi-Matrix 
\[
  \left(\frac{\partial f}{\partial T_{1}},\frac{\partial f}{\partial T_{2}}\right)=\left(T_{1}(3T_{1}-2),\ 2T_{2}\right)
\]
Rang 1 in $(x_{1},x_{2})$ hat. Das ist äquivalent dazu, dass $(x_{1},x_{2})\neq(0,0)$.
Dies lässt sich rein formal über beliebigen Grundkörpern \textbf{algebraisch}
formulieren.

\paragraph{Methoden.}

GAGA - Géometrie algébrique, géometrique analytique (Serre)\medskip{}

\begin{tabular}{|c|c|}
  \hline 
  Komplexe Geometrie ($\mathbb{C}$), Differentialgeometrie $(\mathbb{R})$ & Algebraische Geometrie\tabularnewline \\
  \hline
  \hline
  Analytische Hilfsmittel & Kommutative Algebra\tabularnewline \\
  \hline 
\end{tabular}



\section{Die Zariski-Topologie}
\begin{defn}
Sei $M\subset k[T_{1},\ldots,T_{n}]=:k[\underline{T}]$ eine Teilmenge.
Mit
\[
V(M)=\{(t_{1},\ldots,t_{n})\in k\mid f(t_{1},\ldots,t_{n})=0\ \boldsymbol{\forall f\in M}\}
\]

bezeichnen wir die gemeinsame \textbf{Nullstellen-(Verschwindungs-)Menge}\index{Nullstellen-Menge}
der Elemente aus $M$. (Manchmal auch $V(f_{i},i\in I)$ statt $V(\{f_{i},i\in I\})$.
\end{defn}

\subsection{Eigenschaften}
\begin{itemize}
\item $V(M)=V(\mathfrak{A})$, wenn $\mathfrak{A}=\langle M\rangle$ das
\emph{von} $M$ \emph{erzeugte Ideal} in $k[I]$ bezeichnet.
\item Da $k[\underline{T}]$ noethersch (Hilbertscher Basissatz) ist, reichen
stets endlich viele $f_{1},\ldots,f_{n}\in M$:
\[
V(M)=V(f_{1},\ldots,f_{n})\qquad\text{falls }\mathfrak{A}=\langle f_{1},\ldots,f_{n}\rangle.
\]
\item $V(-)$ ist \textbf{inklusionsumkehrend}, $M'\subset M\Rightarrow V(M)\subseteq V(M')$.
\end{itemize}
\begin{prop}
Die Mengen $V(\mathfrak{A})$, $\mathfrak{A}\subset k[\underline{T}]$
ein Ideal, sind die \textbf{abgeschlossenen} Mengen einer Topologie
auf $k^{n}$, der sogenannten \textbf{Zariski-Topologie}\index{Zariski-Topologie}.
\begin{enumerate}
\item $\emptyset=V\left((1)\right)$, $k^{n}=V(0)$. 
\item $\bigcap_{i\in I}V(\mathfrak{A}_{i})=V\left(\sum_{i\in I}\mathfrak{A}_{i}\right)$
f�r beliebige Familien $(\mathfrak{A}_{i})$ von Idealen.
\item $V(\mathfrak{A})\cup V(\mathfrak{B})=V(\mathfrak{AB})$ f�r $\mathfrak{A},\mathfrak{B}\subset k[\underline{T}]$
Ideale.
\end{enumerate}
\end{prop}
\begin{proof}
�bung / Algebra II. 

\-
\end{proof}




\section{Affine algebraische Mengen}
\label{sec:algebraische-mengen}
\begin{defn}
\label{def:algebraische-mengen}
\mbox{}
\begin{itemize}
\item $\mathbb{A}^{n}(k)$, der $\textbf{affine Raum der Dimension n}$ (�ber $k$),
bezeichne $k^{n}$ mit der Zariski-Topologie.
\item Abgeschlossene Teilmengen von $\mathbb{A}^{n}(k)$ hei�en affine abgeschlossene
Mengen.
\end{itemize}
\end{defn}
\begin{example}
\label{bsp:algebraische-mengen-dim1}
Da $k[T]$ ein Hauptidealring ist, sind die abgeschlossen Mengen in
$\mathbb{A}^{1}(k)$: $\emptyset$, $\mathbb{A}^{1}$, Mengen der
Form $V(f)$, $f\in k[T]\backslash\{k\}$ (endliche Teilmengen).%
\begin{comment}
F�r $f\in k$ ist $V(f)=\mathbb{A}^{1}$, denn die Einheiten im Polynomring
$k[T]$ sind gegeben durch $k^{\times}$, und Ideale erzeugt von einer
Einheit bilden den ganzen Ring. (siehe Algebra 1)
\end{comment}
{} Insbesondere sieht man, dass die Zariski-Topologie im Allgemeinen
nicht Hausdorff ist. 
\end{example}
%
\begin{example}
\label{bsp:algebraische-mengen-dim2}
$\mathbb{A}^{2}(k)$ hat zumindestens als abgeschlossene Mengen:
\begin{itemize}
\item $\emptyset$, $\mathbb{A}^{2}$;
\item Einpunktige Mengen: $\{(x_{1},x_{2})\}=V(T_{1}-x_{1},T_{2}-x_{2})$;
\item $V(f)$, $f\in k[T_{1},T_{2}]$ irreduzibel. 
\end{itemize}
Ferner alle endlichen Vereinigungen dieser Liste. (Dies sind in der
Tat alle, denn sp�ter sehen wir: ``irreduzible'' abgeschlossene
Mengen entsprechen den \emph{Primidealen}, und $k[T_{1},T_{2}]$ hat
``Krull-Dimension $2$''.)
\end{example}




\section{Der Hilbertsche Nullstellensatz}\label{sec:nullstellensatz}
\begin{prop}\label{prop:nullstellensatz}
Sei $K$ ein (nicht notwendigerweise algebraisch abgeschlossener) K�rper,
und $A$ eine endlich erzeugte $K$-Algebra. Dann ist $A$ Jacobson'sch,
d.h. f�r jedes Primideal $\mathfrak{p}\unlhd A$ gilt:
\[
\mathfrak{p}=\bigcap_{\mathfrak{m}\supseteq\mathfrak{p}}\mathfrak{m},\quad\mathfrak{m}\text{ maximales Ideal}
\]

Ist $\mathfrak{m}\unlhd A$ ein maximales Ideal, so ist die K�rpererweiterung
$K\subseteq A/\mathfrak{m}$ endlich.
\end{prop}
\begin{proof}
Algebra II / kommutative Algebra.
\end{proof}
\begin{cor}\label{cor:nullstellensatz}
\mbox{}
\begin{enumerate}
\item Sei $A$ eine e.e. (endlich erzeugte) $k$-Algebra ($k$ sei algebraisch
abgeschlossen), $\mathfrak{m}\unlhd A$ ein maximales Ideal. Dann
ist $A/\mathfrak{m}=k$. 
\item Jedes maximale Ideal $\mathfrak{m}\unlhd k[\underline{T}]$ ist von der
Form $\mathfrak{m}=(T_{1}-x_{1},\ldots,T_{n}-x_{n})$ mit $x_{1},\ldots,x_{n}\in k$.
\item F�r ein Ideal  $\mathfrak{a}\unlhd k[\underline{T}]$ gilt:
\[
\rad(\mathfrak{a})=\sqrt{\mathfrak{a}}\overset{(i)}{=}\bigcap_{\mathfrak{a}\subseteq\mathfrak{p}\unlhd k[\underline{T}], \mathfrak{p} \text{prim}}\mathfrak{p}\overset{(ii)}{=}\bigcap_{\mathfrak{a}\subseteq\mathfrak{m}\unlhd k[\underline{T}], \mathfrak{m} \text{maximal}}\mathfrak{m}
\]
\end{enumerate}
\end{cor}
\begin{proof}
\mbox{}
\begin{enumerate}
\item $k\rightarrow A\rightarrow A/\mathfrak{m}$ ist Isomorphismus,  da
$k$ keine echte algebraische K�rpererweiterung besitzt.
\item Es ist
\begin{align*}
k[T_{1},\ldots,T_{n}] & \twoheadrightarrow k[\underline{T}]/\mathfrak{m}=k\\
T_{i} & \mapsto x_{i}
\end{align*}
surjektiv. Es folgt: $\mathfrak{m}=(T_{1}-x_{1},\ldots,T_{n}-x_{n})$, da letzteres
bereits maximal ist. ($\supseteq$ klar.)
\item (i) Algebra II. (ii) Theorem.
\end{enumerate}
\end{proof}




\section{Korrespondenz zwischen Radikalidealen und affinen algebraischen Mengen}
\label{sec:radikalideale-und-algebraische-mengen}

Sei $V(\mathfrak{a})\subseteq\mathbb{A}^{n}(k)$ affin algebraische
Menge, $\mathfrak{a}\unlhd k[\underline{T}]$ ein Ideal.\textbf{ Es gilt:}
\[
  V(\mathfrak{a})=V(\rad\mathfrak{a})
\]

mit $\rad\mathfrak{a}=\{f\in k[\underline{T}]\mid f^{n}\in\mathfrak{a}\text{ f�r ein }n>0\}$,
da
\[
  f^{n}(x)=0\Leftrightarrow f(x)=0,
\]

d.h. verschiedene Ideale k�nnen dieselbe algebraische Menge beschreiben.
\begin{defn}
  \label{def:verschwindungsideal}
  F�r eine Teilmenge $Z\subseteq\mathbb{A}^{n}(k)$ bezeichne
  \[
    I(Z):=\{f\in k[\underline{T}]\mid f(x)=0\ \forall x\in Z\}
  \]

  das \textbf{Verschwindungsideal von Z}, das Ideal aller auf $Z$ verschwindenden Polynomfunktionen.
\end{defn}
\begin{prop}
  \label{prop:verschwindungsmenge-verschwindungsideal}
  \mbox{}
  \begin{enumerate}
  \item Sei $\mathfrak{a}\unlhd k[\underline{T}]$ Ideal. Dann ist $I(V(\mathfrak{a}))=\rad(\mathfrak{a})$.
  \item Sei $Z\subseteq\mathbb{A}^{n}(k)$ Teilmenge. Dann ist $V(I(Z))=\overline{Z}$,
    der Abschluss von $Z$ in $\mathbb{A}^{n}(k)$.
  \end{enumerate}
\end{prop}
\begin{proof}
  �bungsblatt 2.
\end{proof}
\medskip{}

$\mathfrak{a}$ hei�t \textbf{Radikalideal}\index{Radikalideal},
falls $\mathfrak{a}=\rad(\mathfrak{a})$, oder �quivalent falls $k[\underline{T}]/\mathfrak{a}$
\emph{reduziert} ist, d.h. keine nilpotente Elemente ungleich $0$ hat.
\begin{cor}
  \label{korrespondenz-radikalideal-abgeschlossene-mengen}
  Wir erhalten eine 1-1 Korrespondenz
  \begin{align*}
    \{\text{abg. Mengen }\subseteq\mathbb{A}^{n}\} & \leftrightarrow\{\text{Radikalideale }\mathfrak{a}\unlhd k[\underline{T}]\}\\
    Z & \mapsto I(Z)\\
    V(\mathfrak{a}) & \mapsfrom\mathfrak{a}
  \end{align*}

  die sich zu einer 1-1 Korrespondenz
  \begin{align*}
    \left\{ \text{Punkte in }\mathbb{A}^{n}\right\}  & \leftrightarrow\left\{ \text{max. Ideale in }k[\underline{T}]\right\} \\
    x=(x_{1},\ldots,x_{n}) & \mapsto\begin{array}{rl}
                                      \mathfrak{m}_{x} & =I(\{x\})\\
                                                       & =\ker(k[\underline{T}]\rightarrow k,\ T_{i}\mapsto x_{i})
                                    \end{array}
  \end{align*}

  einschr�nkt.
\end{cor}


%%% Local Variables:
%%% mode: latex
%%% TeX-master: "../AlgGeo1"
%%% End:



\section{Irreduzibele topologische R�ume}

Die folgenden topologische Begriffe sind nur interessant, da $\mathbb{A}^{n}(k)$
($n>0$) kein Hausdorff'scher Raum ist.
\begin{defn}
Ein topologischer Raum $X$ hei�t \textbf{irreduzibel}\index{irreduzibel},
wenn $X\neq\emptyset$ und $X$ sich \emph{nicht} als Vereinigung
zweier echter abgeschlossenen Teilmengen darstellen l�sst, d.h
\[
X=A_{1}\cup A_{2},\ A_{i}\ \text{abg.}\quad\Rightarrow\quad A_{1}=X\text{ oder }A_{2}=X.
\]

$Z\subset X$ hei�t irreduzibel, falls $Z$ mit der induzierten Topologie
irreduzibel ist.
\end{defn}
\begin{prop}
\label{prop:irreduzibel-top}F�r einen topologischen Raum $X$ sind
�quivalent:
\begin{enumerate}
\item $X$ ist irreduzibel.
\item Je zwei nichtleere offenen Teilmengen von $X$ haben nicht-leeren
Durchschnitt.
\item Jede nichtleere offene Teilmenge $U\subset X$ ist dicht in $X$.
\item Jede nichtleere offene Teilmenge $U\subset X$ ist zusammenh�ngend.
\item Jede nichtleere offene Teilmenge $U\subset X$ ist irreduzibel.
\end{enumerate}
\end{prop}
\begin{proof}
\mbox{}
\begin{itemize}
\item $(i)\Leftrightarrow(ii)$

Komplementsmengen.
\item $(ii)\Leftrightarrow(iii)$ 

Es ist: $U\subset X$ dicht $\Leftrightarrow U\cap O\neq\emptyset$
f�r jedes offene $\emptyset\neq O\subset X$.
\item $(iii)\Rightarrow(iv)$

Klar. 
\item $(iv)\Rightarrow(iii)$

Sei $\emptyset\neq U$ offen und zusammenh�ngend. Es folgt:
\[
U=U_{1}\sqcup U_{2},\qquad\emptyset\neq U_{i}\underset{\text{offen}}{\subset}U\underset{\text{offen}}{\subset}X
\]
Damit ist $U_{1}\cap U_{2}=\emptyset$, ein Widerspruch zu (iii).
\item $(v)\Rightarrow(i)$ 

Klar. $(U=X)$
\item $(iii)\Rightarrow(v)$

Sei $\emptyset\neq U\underset{\text{offen}}{\subset}X$. Ist $\emptyset\neq V\underset{\text{offen}}{\subset}U$,
so ist $V\underset{\text{offen}}{\subseteq}X$. Es folgt: $V$ ist
dicht in $X$ und irreduzibel in $U$. Mit $(iii)\Rightarrow(i)$
folgt, dass $U$ irreduzibel ist. 

\end{itemize}
\end{proof}
\begin{lem}
\label{lem:irreduzibel-abschluss}Eine Teilmenge $Y$ ist genau dann
irreduzibel, wenn ihr Abschluss $\overline{Y}$ dies ist.
\end{lem}
\begin{proof}
$Y$ irreduzibel.

$\Leftrightarrow\forall U,V\subset X$ offen mit $U\cap Y\neq\emptyset\neq V\cap Y$,
gilt $Y\cap(U\cap V)\neq\emptyset$.

$\Leftrightarrow\overline{Y}$ irreduzibel. 
\end{proof}
\begin{defn}
Eine maximale irreduzibele Teilmenge eines topologischen Raumes $X$
hei�t \textbf{irreduzibele Komponente}\index{irreduzibele Komponente}
von $X$.
\end{defn}
\begin{rem}
\mbox{}
\begin{enumerate}
\item Jede irreduzibele Komponente ist abgeschlossen nach Lemma 14.
\item $X$ ist Vereinigung seiner irreduzibelen Komponenten, \emph{denn}: 

die Menge der irreduzibelen Teilmengen von $X$ ist \textbf{induktiv
geordnet}: f�r jede aufsteigende Kette irreduzibeler Teilmengen ist
die Vereinigung wieder irreduzibel. (Satz 13 (ii)). Mit dem \textbf{Lemma
von Zorn} folgt: Jede irreduzibele Teilmenge ist in einer irreduzibelen
Komponente enthalten. Damit ist jeder Punkt in einer irreduzibelen
Komponente enthalten.
\end{enumerate}
\end{rem}




\section{Irreduzibele affine algebraische Mengen}

\subsection{Lemma 16}

Eine abgeschlossene Teilmenge $Z\subseteq\mathbb{A}^{n}(k)$ ist genau
dann irreduzibel, wenn $I(Z)$ ein Primideal ist. Insbesondere ist
$\mathbb{A}^{n}$ irreduzibel.  

\subsubsection{Beweis (Lemma 16)}

$Z$ irreduzibel $\Leftrightarrow(Z=\underbrace{V(\mathfrak{A})}_{\bigcap V(f_{i})}\cup\underbrace{V(\mathfrak{b})}_{\bigcap V(g_{j})}\Rightarrow V(\mathfrak{A})=Z$
oder $V(\mathfrak{b})=Z$) 

$\Leftrightarrow\forall f,g\in k[\underline{T}]$ ist $V(fg)=V(f)\cup V(g)\supseteq Z$
gilt $V(f)\supset Z$ oder $V(g)\supseteq Z$.

$\stackrel[I(V(\mathfrak{A}))=\rad(\mathfrak{A})]{V(I(Z))=Z}{\Leftrightarrow}\forall f,g\in k[\underline{T}]$
ist $fg\in I(V(fg)\subseteq I(Z)$ gilt $f\in I(Z)$ oder $g\in I(Z)$.

$\Leftrightarrow I(Z)$ ist Primideal.

\subsection{Bemerkung 17}

Die Korrespondenz aus Korollar 11 schr�nkt sich ein zu
\[
\{\text{irred. abg. Teilmengen }\subseteq\mathbb{A}^{n}\}\overset{1:1}{\leftrightarrow}\{\text{Primideale in }k[\underline{T}]\}
\]




\section{Quasikompakte und noethersche topologische Räume}
\label{sec:quasikompakt-noethersch}
\begin{defn}
  \label{def:quasikompakt/noethersch}
  Ein topologischer Raum $X$ heißt \textbf{quasikompakt}\index{quasikompakt},
  falls jede offene Überdeckung von $X$ eine \emph{endliche} Teilüberdeckung
  enthält. (,,quasi`` deutet an, dass $X$ in der Regel nicht Hausdorff'sch
  ist!). Er heißt \textbf{noethersch}\index{noethersch}, wenn jede
  absteigende Kette
  \[
    X\supseteq Z_{1}\supseteq Z_{2}\supseteq\cdots
  \]

  abgeschlossener Teilmengen von $X$ stationär wird ($\Leftrightarrow$
  jede aufsteigende Kette offener Teilmengen wird stationär).
\end{defn}
\begin{lem}
  \label{lem:eigenschaften-noethersch}
  Sei $X$ ein noetherscher topologischer Raum. Dann gilt:
  \begin{enumerate}
  \item Jede abgeschlossene Teilmenge $Z \subseteq X$ ist noethersch.
  \item Jede offene Teilmenge $U \subseteq X$ ist quasikompakt.
  \item Jeder abgeschlossene Teilraum $Z \subseteq X$ besitzt nur endlich viele
    irreduzible Komponenten.
  \end{enumerate}
\end{lem}
\begin{proof}
  \mbox{}
  \begin{enumerate}
  \item Nach Definition, da abgeschlossene Mengen von $Z$ auch solche von
    $X$ sind.
  \item $U=\bigcup_{i\in I}U_{i}$ offen; Angenommen $U$ wäre nicht quasikompakt.
    Dann gibt es eine Folge $I_{1}\subseteq I_{2}\subseteq\cdots\subseteq I$ von Teilmengen
    mit
    \[
      V_{1}\subsetneq V_{2}\subsetneq\cdots\neq U\quad\text{für }V_{j}=\bigcup_{i\in I_{j}}U_{i}.
    \]
    Widerspruch zu noethersch.
  \item Es reicht zu zeigen: Jeder noethersche Raum ist Vereinigung endlich
    vieler irreduzibler Teilmengen. Da $X$ noethersch ist, folgt mit
    dem \emph{Lemma von Zorn} dass jede nichtleere Menge von algebraischen
    Teilmengen in $X$ ein minimales Element besitzt. 
    \[
      \text{Angenommen:} \mathcal{M}:=\left\{ Z\subseteq X\text{ abg.}\mid Z\text{ ist \textbf{nicht} endl. Vereinigung irred. Mengen}\right\} \text{ wäre nichtleer.}
    \]
    $\Rightarrow\exists$ minimales Element, sagen wir $Z$, in $\mathcal{M}$.

    $\Rightarrow Z$ ist nicht irreduzibel.

    $\Rightarrow Z=Z_{1}\cup Z_{2}$ mit $Z_{1},Z_{2}\subsetneq Z$ abgeschlossen.

    $\Rightarrow$ ($Z$ minimal) $Z_{1},Z_{2}\notin\mathcal{M}$

    $\Rightarrow Z\notin\mathcal{M}$. Widerspruch.

  \end{enumerate}
\end{proof}
\begin{prop}
  \label{prop:algebraische-mengen-noethersch}
  Jeder abgeschlossene Teilraum $X\subseteq\mathbb{A}^{n}(k)$ ist noethersch.
\end{prop}
\begin{proof}
  Nach dem obigen Lemma ist nur zu zeigen, dass $\mathbb{A}^{n}(k)$
  noethersch ist.

  Absteigende Ketten abgeschlossener Teilmengen sind nach \emph{Korollar
    11} in 1-1 Korrespondenz mit aufsteigenden Ketten von (Radikal-)Idealen
  in $k[\underline{T}]$. Da $k[\underline{T}]$ nach dem Hilbertschen
  Basissatz noethersch ist, werden letzere Ketten stationär.
\end{proof}
\begin{cor}[Primärzerlegung]
  \label{cor:primaerzerlegung}
  Sei $\mathfrak{a}=\rad(\mathfrak{a})\unlhd k[\underline{T}]$
  ein Radikalideal. Dann gilt: $\mathfrak{a}$ ist Durchschnitt von
  endlich vielen Primidealen, die sich jeweils paarweise nicht enthalten; diese
  Darstellung ist eindeutig bis auf Reihenfolge.
\end{cor}
\begin{proof}
  $V(\mathfrak{a})=\bigcup_{i=1}^{n}V(\mathfrak{b}_{i})$, $\mathfrak{b}_{i}$
  Primideal.%
  \begin{comment}
    Stationäre Kette folgt aus noethersch (Satz 21); mit Bemerkung 16
    bzw. Lemma 17 folgt, dass die $\mathfrak{b}_{i}$ Primideale sind.
  \end{comment}
  {} [Anmerkung] Mit Satz 10 folgt: % color package gives errors with plastex
  \[
    \mathfrak{a}=\rad(\mathfrak{a})=I(V(\mathfrak{a}))=\bigcap_{i=1}^{n}\underbrace{I(V(\mathfrak{b}_{i}))}_{\mathfrak{b}_{i}\text{ minimale Primideale (\ref{lem:charakterisierung-irreduzibel-alg})}}
  \]
\end{proof}




\section{Morphismen von affinen algebraischen Mengen}
\label{sec:morphismen-alg-mengen}
\begin{defn}
  \label{def:morphismus-alg-mengen}
  Seien $X\subseteq\mathbb{A}^{m}(k)$, $Y\subseteq\mathbb{A}^{n}(k)$
  affine algebraische Mengen. Ein \textbf{Morphismus} $X\rightarrow Y$
  affiner algebraischer Mengen ist eine Abbildung $f:X\rightarrow Y$
  der zugrundeliegenden Mengen, sodass $f_{1},\ldots,f_{n}\in k[T_{1},\ldots,T_{m}]$
  existieren, derart dass $\forall x\in X$ gilt:
  \[
    f(x)=(f_{1}(x),\ldots,f_{n}(x)) \in Y.
  \]
  Es bezeichne $\hom(X,Y)$ die Menge der Morphismen $X \to Y$. 
\end{defn}
\begin{rem}
  \label{rem:morphismen-fortsetzbarkeit}
  $f:X\rightarrow Y$ lässt sich immer fortsetzen zu einem Morphismus
  \[
    f:\mathbb{A}^{m}(k)\rightarrow\mathbb{A}^{n}(k),
  \]

  aber nicht eindeutig, es sei denn $X=\mathbb{A}^{m}(k)$.
\end{rem}

\paragraph{Komposition}

\[
  \xymatrix@C=9pc{X\ar[r]^{f}_{f_{1},\ldots,f_{n}\in k[T_{1},\ldots,T_{m}]} & Y\ar[r]^{g}_{g_{1},\ldots,g_{r}\in k[T_{1}',\ldots,T_{m}']} & Z}
\]

mit $X\subseteq\mathbb{A}^{m}(k)$, $Y\subseteq\mathbb{A}^{n}(k)$,
$Z\subseteq\mathbb{A}^{r}(k)$. Es folgt:
\begin{align*}
  g(f(x))=\, & (g_{1}(f_{1}(x),\ldots,f_{n}(x)),\ldots,g_{r}(f_{1}(x),\ldots,f_{n}(x))\\
  =:\, & (h_{1}(x),\ldots,h_{r}(x))
\end{align*}

d.h. $g\circ f$ ist durch Polynome $h_{i}\in k[T_{1},\ldots,T_{m}]$
gegeben, also ist $g\circ f$ wieder ein Morphismus affiner algebraischer
Mengen. Wir erhalten die \textbf{Kategorie affiner algebraischer Mengen}.
\begin{example}
  \label{bsp:morphismen-alg-mengen}
  \mbox{}
  \begin{enumerate}
  \item Sei die Abbildung
    \begin{align*}
      \mathbb{A}^{1}(k) & \rightarrow V(T_{2}-T_{1}^{2})\subseteq\mathbb{A}^{2}(k)\\
      x & \mapsto(x,x^{2}).
    \end{align*}
    Diese Abbildung ist sogar ein \emph{Isomorphismus }affiner algebraischer
    Mengen, da die Umkehrabbildung
    \[
      (x,y)\mapsto x
    \]
    ebenfalls ein Morphismus ist.
  \item Sei char$(k)\neq2$. Die Abbildung
    \begin{align*}
      \mathbb{A}^{1}(k) & \rightarrow V(T_{2}^{2}-T_{1}^{2}(T_{1}+1))\\
      x & \mapsto(x^{2}-1,x(x^{2}-1))
    \end{align*}
    ist ein Morphismus, aber \emph{nicht }bijektiv, da $1,-1$ beide auf
    $(0,0)$ abgebildet werden.
  \end{enumerate}
\end{example}




\section{Unzul�nglichkeiten des Begriffs der affinen algebraischen Mengen}
\label{sec:unzulaenglichkeiten-alg-mengen}
\begin{enumerate}
\item Offene Teilmengen affiner algebraischer Mengen tragen nicht in nat�rlicher
  Weise die Struktur einer affinen algebraischen Menge.
\item Insbesondere k�nnen wir affine algebraische Mengen nicht entlang offener
  Teilr�ume verkleben. (vgl. Mannigfaltigkeiten.)
\item Keine Unterscheidungsm�glichkeiten z.B. zwischen $\{(0,0)\}$, $V(T_{1})\cap V(T_{2})$
  und $V(T_{2})\cap V(T_{1}^{2}-T_{2})\subseteq\mathbb{A}^{2}(k)$,
  obwohl die ``geometrische Situation'' offensichtlich verschieden
  ist.
\end{enumerate}
Um die Punkte 1 und 2 zu verbessern, gehen wir im Folgenden zu ``R�umen mit Funktionen'' �ber, und verzichten darauf,
dass sich diese in einen affinen Raum $\mathbb{A}^{n}(k)$ einbetten
lassen.

Der Punkt 3 ist eine Motivation daf�r, sp�ter Schemata einzuf�hren.
(subtiler)




\paragraph*{Affine algebraische Mengen als R�ume von Funktionen}

\section{Der affine Koordinatenring}\label{sec:koordinatenring}

Sei $X\subseteq\mathbb{A}^{n}(k)$ abgeschlossen. F�r den surjektiven
(Def. von Morphismen) $k$-Algebren-Homomorphismus
\begin{align*}
k[\underline{T}] & \xrightarrow{\varphi}\hom(X,\mathbb{A}^{1}(k))\\
f & \mapsto(x\mapsto f(x)),
\end{align*}

wobei die Morphismen in folgende Weise eine $k$-Algebra bilden:
\begin{align*}
(f+g)(x) & :=f(x)+g(x)\\
(fg)(x) & :=f(x)g(x)\\
(\alpha f)(x) & :=\alpha f(x)
\end{align*}

mit $f,g\in\hom(X,\mathbb{A}^{1}(k))$, $\alpha\in k$, gilt:
\[
\ker\varphi=I(X).
\]
\begin{defn}\label{def:koordinatenring}
$\Gamma(X):=k[\underline{T}]/I(X)\cong_{k-\mathrm{Alg}}\hom(X,\mathbb{A}^{1}(k))$ hei�t der \textbf{affine
Koordinatenring }von $X$.

F�r $x=(x_{1},\ldots,x_{n})\in X$ gilt:
\begin{align*}
\mathfrak{m}_{x}:=\  & \ker(\Gamma(X)\twoheadrightarrow k,\,f\mapsto f(x))\\
=\  & \{f\in\Gamma(X)\mid f(x)=0\}\\
=\  & \pi((T_{1}-x_{1},\ldots,T_{n}-x_{n}))\\
=\  & \ker(\Gamma(\mathbb{A}^{n}(k))\twoheadrightarrow k)
\end{align*}

unter der Projektion $\pi:k[\underline{T}]=\Gamma(\mathbb{A}^{n}(k))\twoheadrightarrow\Gamma(X)$.
Es ist $\mathfrak{m}_{x}$ ein maximales Ideal von $\Gamma(X)$ mit
$\Gamma(X)/\mathfrak{m}_{x}\cong k$. F�r ein Ideal $\mathfrak{a}\unlhd\Gamma(X)$
sei
\[
V(\mathfrak{a}):=\{x\in X\mid f(x)=0\ \forall f\in\mathfrak{a}\}=V(\pi^{-1}(\mathfrak{a}))\cap X.
\]

Dies sind genau die abgeschlossenen Mengen von $X$ als Teilraum von
$\mathbb{A}^{n}(k)$ mit der induzierten Topologie, diese wird auch
\textbf{Zariski-Topologie} genannt. F�r $f\in\Gamma(X)$ setze:
\[
D_X(f) := D(f):=\{x\in X\mid f(x)\neq0\}=X\setminus V(f).
\]
\end{defn}
\begin{lem}\label{lem:basis-zariski-topologie}
Die offenen Mengen $D(f)$, $f\in\Gamma(X)$, bilden eine Basis der
Topologie von $X$, d.h.
\[
\forall U\subseteq X\text{ offen }\exists f_{i}\in\Gamma(X),\,i\in I\quad\text{mit }U=\bigcup_{i\in I}D(f_{i})
\]
\end{lem}
\begin{proof}
$U=X\backslash V(\mathfrak{a})$ f�r ein $\mathfrak{a}\unlhd\Gamma(X)$,
$\mathfrak{a}=\langle f_{1},\ldots,f_{n}\rangle_{\Gamma(X)}$ . Wegen
\[
V(\mathfrak{a})=\bigcap_{i=1}^{n}V(f_{i})\quad\Rightarrow\quad U=\bigcup_{i=1}^{n}D(f_{i})
\]

Es reichen also sogar endlich viele $f_{i} \in \Gamma(X)$! 
\end{proof}
\begin{prop}\label{prop:eigenschaften-koordinatenring}
Der Koordinatenring $\Gamma(X)$ einer affinen algebraischen Menge
$X$ ist eine endlich erzeugte $k$-Algebra, die reduziert ist (d.h.
keine nilpotenten Elemente $\neq0$ enth�lt). Ferner ist $X$ irreduzibel
genau dann, wenn $\Gamma(X)$ integer ist.
\end{prop}
\begin{proof}
$k[\underline{T}]\twoheadrightarrow\Gamma(X)$ impliziert, dass $\Gamma(X)$ als $k$-Algebra endlich
erzeugte ist. Es gilt:
\[
\Gamma(X)\text{ irreduzibel }\Leftrightarrow I(X)=\rad I(X).
\]

Denn mit Satz 10.ii) und Korollar 11 folgt:
\begin{align*}
X & =V(\mathfrak{a}):\,I(X)=\rad\mathfrak{a}\\
\Rightarrow\rad I(X) & =\rad\rad\mathfrak{a}=\rad\mathfrak{a}=I(X).
\end{align*}

Mit Lemma 17 folgt: $X$ irreduzibel

$\phantom{\quad}\Leftrightarrow I(X)$ prim

$\phantom{\quad}\Leftrightarrow\Gamma(X)=k[\underline{T}]/I(X)$ integer.
\end{proof}




\section{Funktorielle Eigenschaften von $\Gamma(X)$}
\begin{prop}
F�r einen Morphismus $X\xrightarrow{f}Y$ affiner algebraischer Mengen
definiert 
\begin{align*}
\Gamma(f):\hom(Y=\Gamma(Y),\mathbb{A}^{1}(k)) & \rightarrow\hom(X=\Gamma(X),\mathbb{A}^{1}(k))\\
g & \mapsto g\circ f
\end{align*}

ein Homomorphismus von $k$-Algebren. Der so definierte \emph{kontravariante}
Funktor
\[
\Gamma:\{\text{affine algebraische Mengen\}}\rightarrow\{\text{red. endl. erz. }k\text{-Alg.}\}
\]

liefert eine Kategorien�quivalenz, der durch Einschr�nkung eine �quivalenz
\[
\Gamma:\{\text{irred. aff. alg. Meng.\}}\rightarrow\{\text{integere endl. erz. }k\text{-Alg.\}}
\]

induziert.
\end{prop}
\begin{proof}
$Y\xrightarrow{g}\mathbb{A}^{1}(k)\in\Gamma(Y)$ ist Morphismus. Es
folgt:
\[
g\circ f:X\xrightarrow{f}Y\xrightarrow{g}\mathbb{A}^{1}(k)
\]
 

ist ...,  d.h. $\in\Gamma(X)$. $\Gamma(f):\Gamma(Y)\rightarrow\Gamma(X)$
ist ein $k$-Alg.-Hom. und Kompositions....  (nach Bemerkung 24)
mit $\Gamma(\text{id}_{X})=\text{id}_{\Gamma(X)}$. Da ferner $\Gamma(f_{1}\circ f_{2})=\Gamma(f_{2})\circ\Gamma(f_{1})$
aus der Definition folgt, ist $\Gamma$ ein kontravarianter Funktor.
\begin{claim*}
$\Gamma$ ist volltreu, d.h.
\begin{align*}
\Gamma:\hom(X,Y) & \rightarrow\hom(\Gamma(Y),\Gamma(X))\\
f & \mapsto\Gamma(f)
\end{align*}
ist \emph{bijektiv} f�r alle affinen algebraischen Mengen $X,Y$.
\end{claim*}
\begin{proof}
Wir konstruieren eine Umkehrabbildung wie folgt: Zu $\varphi:\Gamma(Y)\rightarrow\Gamma(X)$
f�r $X\subseteq\mathbb{A}^{n}$, $Y\subseteq\mathbb{A}^{n}$ existiert:
\[
\xymatrix{k[T_{1}',\ldots,T_{k}']\ar[r]^{\tilde{\varphi}}\ar@{->>}[d] & k[T_{1},\ldots,T_{m}]\ar@{->>}[d]\\
\Gamma(Y)\ar[r] & \Gamma(X)
}
\]
kommutiert ($\tilde{\varphi}(T_{1}'):=$  liften $\varphi(\pi(T_{i}'))$
in  $k[\underline{T}]$). Definiere:
\begin{align*}
f:X & \rightarrow Y\\
x=(x_{1},\ldots,x_{n}) & \mapsto(\tilde{\varphi}(T_{1}')(x_{1},\ldots,x_{n}),\ldots,\tilde{\varphi}(T_{n}')(x_{1},\ldots,x_{n}))
\end{align*}
\end{proof}
\begin{claim*}
$\Gamma$ ist essentiell surjektiv, d.h. zu jeder reduzierten endlich
erzeugten $k$-Algebra $A$ existiert eine affine algebraische Menge
$X$ mit $A\cong\Gamma(X)$.
\end{claim*}
\begin{proof}
Da nach Voraussetzung $A\cong k[T]/\mathfrak{A}$ f�r Radikalideal
$\mathfrak{A}$, k�nnen wir etwa $X:=V(\mathfrak{A})\subseteq\mathbb{A}^{n}(k)$
setzen. Der Rest folgt aus Satz 28.
\end{proof}
\end{proof}
\begin{prop}
Sei $f:X\rightarrow Y$ ein Morphismus und $\Gamma(f):\Gamma(Y)\rightarrow\Gamma(X)$
der zugeh�rige Homomorphismus der Koordinatenringe. Dann gilt $\forall x\in X$:
$\Gamma(f)^{-1}(\mathfrak{m}_{x})=\mathfrak{m}_{f}(x)$.
\end{prop}
\begin{proof}
Setting:
\begin{align*}
\mathfrak{m}_{f(x)} & =\{g\mid g(f(x))=0\}\subset\hom(Y,\mathbb{A}^{1})\\
 & =\Gamma(Y)\xrightarrow{\Gamma(f)}\Gamma(X)\\
 & =\hom(X,\mathbb{A}^{1})\supset\{k\mid k(x))=0\}\\
g & \mapsto g\circ f
\end{align*}

Es ist:

\[
\Gamma(f)^{-1}(\mathfrak{m}_{x})=\{g\in\Gamma(Y)\mid g\circ f(x)\neq0\}=\mathfrak{m}_{f(x)},
\]

da $\Gamma(f)(g)(x)=g(f(x))$.
\end{proof}



%\selectlanguage{english}%

\section{R�ume mit Funktionen}
\label{sec:raeume-mit-funktionen}

(Prototyp eines geometrischen Objektes, Spezialfall eines ``geringten
Raumes'' vgl. sp�ter.) Sei $K$ ein nicht notwendigerweise algebraisch abgeschlossener
K�rper.
\begin{defn}
\label{def:raum-mit-funktionen}\mbox{}
\begin{enumerate}
\item Ein \textbf{Raum mit Funktionen}\index{Raum mit Funktionen} besteht
aus den folgenden Daten:
\begin{itemize}
\item ein topologischer Raum $X$;
\item eine Familie von Unter-$K$-Algebren
\[
\mathcal{O}_X(U)\leq\text{Abb}(U,K),\quad\forall U\subseteq X\text{ offen }d.d
\]

\begin{enumerate}
\item Sind $U'\subseteq U\subseteq X$ offen und $f\in\mathcal{O}_X(U)$ so ist
$f|_{U'}\in \mathcal{O}_X(U')$.
\item (\textbf{Verklebungsaxiom}\index{Verklebungsaxiom}) Sind $U_{i}\subseteq X$
offen, $i\in I$, und $U=\bigcup_{i}U_{i}$, $f_{i}\in\mathcal{O}_X(U_{i})$,
$i\in I$ gegeben mit
\[
f_{i}|_{U_{i}\cap U_{j}}=f_{j}|_{U_{i}\cap U_{j}}\quad\forall i,j\in I
\]
dann ist die eindeutige Abbildung
\[
f:U\rightarrow K\text{ mit }f|_{U_{i}}=f_{i}
\]
in $\mathcal{O}_X(U)$, bzw. $\exists!f\in\mathcal{O}(U)$ mit $f|_{U_{i}}=f_{i}$ f�r alle $i \in I$.
\end{enumerate}
\end{itemize}
Bezeichne $\mathcal{O}_X$ oder auch $\mathcal{O}$ die oben genannte
Familie $\{\mathcal{O}_X(U) \mid U \subseteq X \text{offen}\}$. Das Tupel $(X,\mathcal{O}_{X})$ hei�t $\textbf{Raum mit Funktionen}$.
\item Ein \textbf{Morphismus}\index{Raum mit Funktionen!Morphismus} $(X,\mathcal{O}_{X})\rightarrow(Y,\mathcal{O}_{Y})$
von R�umen von Funktionen ist eine stetige Abbildung $\varphi:X\rightarrow Y$,
so dass f�r alle $V\subseteq Y$ offen und $f\in\mathcal{O}_{Y}$
gilt:
\[
f\circ \varphi|_{\varphi^{-1}(V)}:\varphi^{-1}(V)\rightarrow K
\]
liegt in $\mathcal{O}_{X}(\varphi^{-1}(V))$.
\[
\xymatrix{X\ar[r]^{\varphi} & Y\\
\varphi^{-1}(V)\ar[r]^{\varphi|}\ar[d]_{f\circ \varphi|_{\varphi^{-1}(V)}}\ar@{^{(}->}[u] & V\ar[d]^{f}\ar@{^{(}->}[u]_{\text{offen}}\\
K\ar@{=}[r] & K
}
\]
\end{enumerate}
\end{defn}
Wir erhalten die Kategorie der $\emph{R�ume mit Funktionen �ber K}$.
\begin{defn}[offene Unterr�ume von R�umen mit Funktionen]\label{def:raeume-mit-fkt-offener-unterraum}
F�r $(X,\mathcal{O}_{X})$ einen Raum mit Funktionen und $U\subseteq X$ offen bezeichne $(U,\mathcal{O}_{X}|_{U})$ den
Raum mit Funktionen gegeben durch den topologischen Raum $U$ mit
Funktionen $\mathcal{O}_{X}|_{U}(V):=\mathcal{O}_{X}(V)$ f�r $V\underset{\text{offen}}{\subseteq}U\subseteq X$.
\end{defn}
\textbf{Ab jetzt} betrachten wir R�ume von Funktionen �ber einem festen, algebraisch abgeschlossenen Grundk�rper $k$.\selectlanguage{ngerman}%



\selectlanguage{english}%

\section{Der Raum mit Funktionen zu einer affin algebraischen Menge}
\begin{description}
\item [{Ziel.}] $X\subseteq\mathbb{A}^{n}(k)\mapsto(X,\mathcal{O}_{X})$
als irreduzibele affine algebraische Menge bzw. Zariski-Topologie.
D.h. wir m�ssen Menge von Funktionen $\mathcal{O}_{X}(U)$ auf $U$,
$U\subset X$ offen, definieren. Diese werden als Teilmengen des Funktionenk�rpers
$K(X)$ definiert (dazu $X$ irreduzibel, sp�ter bei Schemata f�llt
diese Bedingung weg!)
\end{description}

\subsection*{Definition 33}

$K(X):=\text{Quot}(\Gamma(X))$ hei�t \textbf{Funktionenk�rper} von
$X$ ($\Gamma(X)$ ist f�r $X$ irreduzibel nullteilerfrei).

Elemente $\frac{f}{g}\in K(X)$, $f,g\in\Gamma(X)=\hom(X,\mathbb{A}^{1}(k))$,
$g\neq0$ lassen sich zumindest als Funktion auf der offenen Menge
$\mathcal{D}(g)\subset X$ auffassen, wenn auch nicht i.A. auf ganz
$X$.

\subsection*{Lemma 34}

Gilt f�r $\frac{f_{1}}{g_{1}},\frac{f_{2}}{g_{2}}\in K(X)$ ($f_{i},g_{i}\in\Gamma(X))$
und einer offenen Teilmenge $\emptyset\neq U\subset\mathcal{D}(g_{1}g_{2})$
\[
\frac{f_{1}(x)}{g_{1}(x)}=\frac{f_{2}(x)}{g_{2}(x)}\qquad\forall x\in U,
\]

dann folgt $\frac{f_{1}}{g_{1}}=\frac{f_{2}}{g_{2}}$ in $K(X)$.

\subsubsection*{Beweis (Lemma 34)}

Ohne Einschr�nkung der Allgemeinheit: $g_{1}=g_{2}=g$ (sonst Erweitern!)

$\Rightarrow(f_{1}-f_{2})(x)=0$ $\forall x\in U$

$\Rightarrow\emptyset\neq U\subset V(f_{1}-f_{2})\subset X$ dicht

d.h. $V(f_{1}-f_{2})=X$.

$f_{1}-f_{2}\in IV(f_{1}-f_{2})=I(X)\equiv(0)$ in $\Gamma(X)$

$\Rightarrow f_{1}-f_{2}=0$.

\subsection*{Definition 35}

Sei $X$ eine irreduzibele affine algebraische Menge, $U\subset X$
offen. Sei $\Gamma(X)_{\mathfrak{m}_{x}}$ Lokalisierung von $\Gamma(X)$
bzgl. das maximale Ideal $\mathfrak{m}_{x}$ in $x\in X$.

$\mathcal{O}_{X}(U):=\bigcap_{x\in U}\Gamma(X)_{\mathfrak{m}_{x}}\subset K(X)$

d.h. f�r jedes $x\in U$ l�sst sich $f\in\mathcal{O}_{X}(U)$ schreiben
als $\frac{h}{g}$ mit $g(x)\neq0$.

(Wenn $f\in\Gamma(X)$ bezeichne $\Gamma(X)_{f}$ die Lokalisierung
von $\Gamma(X)$ bzgl. der multiplikativ abgeschlossenen Teilmenge
$\{1,f,f^{2},\ldots,f^{n}\ldots\}$.  Dann l�sst sich
\[
\Gamma(X)_{\mathfrak{m}_{x}}=\bigcup_{f\in\Gamma(X)\backslash\mathfrak{m}_{x}}\Gamma(X)_{f}\subset K(X)
\]

``$\supset$'' klar, ``$\subset$'' $\frac{g}{f}$ mit $f(x)\neq0$
d.h. $f\notin\mathfrak{m}_{x}$ $\Rightarrow\frac{g}{f}\in\Gamma(X)_{f}$.

\paragraph{Es gilt:}
\begin{enumerate}
\item $\mathcal{O}_{x}(U)\rightarrow\text{Abb}(U,k)$, $f\mapsto(x\mapsto f(x):=\frac{g(x)}{f(x)}\in k)$
ist injektiv (Lemma 34) und wohldefiniert (k�rzen/Erweitern) wobei
$g,h\in\Gamma(X)$ mit $h\notin\mathfrak{m}_{x}$ mit $f=\frac{g}{h}$
nach Definition von $\mathcal{O}_{X}(U)$ existiert.
\item F{[}r $V\subset U\subset X$ offen kommutiert das folgende Diagramm
\item \textbf{Verklebungseigenschaft.} Sei $U=\bigcup_{i\in I}U_{i}$. Nach
Definition ist 
\begin{align*}
\mathcal{O}_{X}(U) & =\bigcap_{i}\mathcal{O}_{X}(U_{i})\subset K(X)\\
\ni f:U\rightarrow k & \quad\ni f_{i}:U_{i}\rightarrow k
\end{align*}
. $\Rightarrow(X,\mathcal{O}_{X})$ ist Raum mit Funktionen, \textbf{der
zur irreduziblen affin algebraische Menge geh�rige Raum von Funktionen.} 
\end{enumerate}

\subsection*{Satz 36 (orig. 33)}

F�r $(X,\mathcal{O}_{X})$ zu $X$ wie oben und $f\in\Gamma(X)$ gilt:
\[
\mathcal{O}_{X}(D(f))=\Gamma(X)_{f},
\]

insbesondere $\mathcal{O}_{X}(X)=\Gamma(X)$.

\subsubsection*{Beweis (Satz 36)}

$\Gamma(X)\subset\mathcal{D}(f)$ klar, da $f(x)\neq0$ $\forall x\in\mathcal{D}(f)$
bzw. $f\in P(X)\backslash\mathfrak{m}_{x}$. 

Sei nun $g$ in $\mathcal{O}_{X}(\mathcal{D}(f))$ gegeben, $(*)$
und $\mathfrak{A}:=\{h\in\Gamma(X)\mid hg\in\Gamma(X)\}\subset\Gamma(X)$
Ideal.

Dazu: $g\in\Gamma(X)_{g}$

$\Leftrightarrow g=\frac{k}{g^{n}}$ f�r ein $n$ und $k\in\Gamma(X)$

$\Leftrightarrow f^{n}\in\mathfrak{A}$ f�r ein $n$.

d.h. zu zeigen: $f\in\text{rad}(\mathfrak{A})=IV(\mathfrak{A})$ (Hilbertsche
Nullstellensatz)

$\Leftrightarrow f(x)=0$ $\forall x\in V(\mathfrak{A})$

Ist dazu $x\in X$ mit $f(x)\neq0$, wo $x\in\mathcal{D}(f)$, so
existiert nach Voraussetzung $(*)$ $f_{1},f_{2}\in\Gamma(X)$, $f_{2}\notin\mathfrak{m}_{x}$
mit $g=\frac{f_{1}}{f_{2}}$

$\Rightarrow f_{2}\in\mathfrak{A}$. Da $f_{2}(x)\neq0$:

$\Rightarrow x\notin V(\mathfrak{A})$.

\subsection*{Bemerkung 37 (orig. 34)}
\begin{enumerate}
\item Im allgemeinen existierten f�r $f\in\mathcal{O}_{x}(U)$ \textbf{nicht}
$g,h\in\Gamma(X)$ mit $f=\frac{g}{h}$ und $h(x)\neq0$ $\forall x\in U$.
\item \textbf{Alternative Definition von $\mathcal{O}_{X}$ I.}

$\mathcal{O}_{X}(\mathcal{D}(f)):=\Gamma(X)_{f}$ $\forall f\in\Gamma(X)$

Da $\mathcal{D}(f)$ Basis der Topologie, kann es h�chstens einen
Raum mit Funktionen geben mit dieser Eigenschaft, es bleibt die Existenz
zu zeigen.
\item \textbf{Alternative Definition von $\mathcal{O}_{X}$ II.}

direkt von einer integeren endlich erzeugten $k$-Algebra $A$ ausgehend
(die $X$ bis auf Isomorphie festlegt), aber ohne ``Koordinaten''
zu w�hlen.

$X:=\{\mathfrak{m}\subseteq A\mid$ max. Ideale\}

$V(\mathfrak{A}):=\{\mathfrak{m}\subseteq A$ max. $\mid\mathfrak{m}\supseteq\mathfrak{A}\}$,
$\mathfrak{A}\subset A$ Ideal, sind die \textbf{abgeschlossenen}
Mengen.

$\mathcal{O}_{X}(U):=\bigcap_{\mathfrak{m}\in U}A_{\mathfrak{m}}\subset\text{Quot}(A)$
f�r $U\subset X$ offen (vgl. sp�ter Schemata).\selectlanguage{ngerman}%
\end{enumerate}




\section{Funktorialität der Konstruktion}
\label{sec:funktorialitaet-affine-varietaet}
\begin{prop}[orig. 35]
  \label{prop:charakterisierung-morphismen-alg-mengen}
  Sei $f:X\rightarrow Y$ eine stetige Abbildung zwischen irreduziblen
  affin-algebraischen Mengen. Es sind äquivalent:
  \begin{enumerate}
  \item $f$ ist ein Morphismus affin-algebraischer Mengen.
  \item $\forall g\in\Gamma(Y)$ gilt $g\circ f\in\Gamma(X)$.
  \item $f$ ist ein Morphismus von Räumen von Funktionen, d.h. für alle $U\subseteq Y$
    offen und alle $g\in\mathcal{O}_{Y}(U)$ gilt $g\circ f\in\mathcal{O}_{X}(f^{-1}(U))$.
  \end{enumerate}
\end{prop}
\begin{proof}
  \mbox{}
  \begin{itemize}
  \item $(i)\Leftrightarrow(ii)$ 

    Folgt aus Satz $\ref{prop:koordinatenringfunktor}$.
  \item $(iii)\Rightarrow(ii)$ 

    $U:=Y$ und Satz $\ref{prop:fkt-auf-basis}$.
  \item $(ii)\Rightarrow(iii)$

    Betrachte $\Gamma(f):\Gamma(Y)\rightarrow\Gamma(X)$, $h\mapsto h\circ f$.
    Aufgrund des Verklebungsaxioms reicht es, die Bedingung für $U$ von
    der Form $D(g)$ zu zeigen; hier gilt:
    \[
      f^{-1}(D(g))=\{x\in X\mid\underbrace{g(f(x))}_{=\Gamma(f)(g)(x)}\neq0\}=D(g \circ f)
    \]
    Deswegen induziert $\Gamma(f)$:
    \begin{align*}
      h & \longmapsto h\circ f\\
      \mathcal{O}_{Y}(D(g)) & \longrightarrow\mathcal{O}_{X}(D(g\circ f))\\
        & \shortparallel\\
      \Gamma(Y)_{g} & \longrightarrow\Gamma(X)_{g \circ f}\\
      \frac{h}{g^{n}} & \longmapsto\frac{h\circ f}{(g\circ f)^{n}}
    \end{align*}
    mit $h\circ f, g\circ f\in\Gamma(X)$ nach Voraussetzung.

  \end{itemize}
\end{proof}
Insgesamt erhalten wir:
\begin{thm}[orig. 36]
  \label{thm:aequivalenz-alg-mengen-aff-varietaeten}
  Die obige Konstruktion definiert einen volltreuen Funktor
  \[
    \text{\{irreduzible aff. alg. Mengen über }k\}\rightarrow\{\text{Räume mit Funktionen über }k\}.
  \]
\end{thm}



\selectlanguage{english}%

\part*{Pr�variet�ten}
\begin{description}
\item [{Ziel.}] Klasse der affinen abgeschlossenen Mengen, aufgefasst als
R�ume mit Funktionen durch Verkleben vergr��ern.
\end{description}
$(X,\mathcal{O}_{X})$ hei�t \textbf{zusammenh�ngend}, falls $X$
als topologischer Raum zusammenh�ngend ist.

\section{Definition von Pr�variet�ten}

\subsection*{Definition 40 (orig. 37)}

Eine affine Variet�t ist ein Raum mit Funktionen, der isomorph ist
zu den Raum mit Funktionen einer irreduziblen affinen algebraischen
Menge.

\subsection*{Definition 41 (orig. 38)}

Eine \textbf{Pr�variet�t} ist ein zusammenh�ngender Raum mit Funktionen
$(X,\mathcal{O}_{X})$, f�r den eine \emph{endliche }�berdeckung $X=\bigcup_{i=1}^{n}U_{i}$
existiert, d.d. $\forall i=1,\ldots,n$ $(U_{i},\mathcal{O}_{X|_{U_{i}}})$
eine affine Variet�t ist.

Ein \textbf{Morphismus von Pr�variet�ten} ist ein Morphismus der entsprechend
R�ume mit Funktionen. Insbesondere sind ... affine Variet�ten Pr�variet�ten!\selectlanguage{ngerman}%



Sp�ter sehen wir: Variet�t = ,,separierte Pr�variet�t``. Affine
Variet�ten sind stets ,,separiert``, daher braucht man nicht von
,,affinen Pr�variet�ten`` zu reden. Ist $X$ eine affine Variet�t,
so schreiben wir oft $\Gamma(X)$ f�r $\mathcal{O}_{X}(X)$ (vgl. Satz
\ref{prop:fkt-auf-basis}).

Unter einer \textbf{offenen affinen �berdeckung} einer Pr�variet�t
$X$ verstehen wir eine Famile von offenen affinen Unterr�umen mit Funktionen
$U_{i}\subseteq X$, $i\in I$ die affine Variet�ten sind, d.d. $X=\bigcup_i U_{i}$.

\section{Vergleich mit differenzierbaren/komplexen Mannigfaltigkeiten}
\label{sec:vergleich-mit-mannigfaltigkeiten}

\paragraph{Differential/Komplexe Geometrie}

Mannigfaltigkeiten werden via Kartenabbildungen mit differenzierbaren/holomorphen
�bergangsabbildungen definiert (hier problematisch, da offene Teile
affiner algebraischer Mengen i.A. keine solche Struktur besitzen.)
Jedoch:
\begin{align*}
  \text{\{differenzierbare Mfgkt.\}} & \longrightarrow\text{\{R�ume mit Fkt.}/\mathbb{R}\}\\
  X & \longmapsto(X,\mathcal{O}_{X})\\
                                     & \phantom{\longmapsto}\mathcal{O}_{X}(U):=C^{\infty}(U,\mathbb{R}),\ U\subseteq X\text{ offen}
\end{align*}

ist ein volltreuer Funktor. Daher kann man differenzierbare Mannigfaltigkeiten
auch als diejenigen R�ume mit Funktionen �ber $\mathbb{R}$ definieren,
f�r die $X$ Hausdorff ist, und so dass eine offene �berdeckung durch
solche R�ume mit Funktionen �ber $\mathbb{R}$ existiert, die in obiger
Weise offene Teilmengen von $\mathbb{R}^{n}$ zugeordnet sind. (Analog
bei komplexen Mannigfaltigkeiten.)



\section{Topologische Eigenschaften von Pr�variet�ten}
\begin{lem}
F�r einen topologischen Raum $X$ und $U\subseteq X$ offen haben
wir eine Bijektion
\begin{align*}
\{Y\subseteq U\text{ irred. abg.}\} & \longleftrightarrow\{Z\subseteq X\text{ irred. abg. mit }Z\cap U\neq\emptyset\}\\
Y & \longmapsto\overline{Y}\text{ (Abschluss in }X)\\
Z\cap U & \longmapsfrom Z
\end{align*}
\end{lem}
\begin{proof}
Lemma 14: $Y\subseteq X$ irreduzibel $\Leftrightarrow\overline{Y}\subseteq X$
irreduzibel.

$Y\subseteq U$ abgeschlossen $\Leftrightarrow\exists A\subset X$
abgecshlossen: $Y=U\cap A$.

$\Rightarrow Y\subseteq\overline{Y}\subseteq A$ $\Rightarrow Y=U\cap\overline{Y}$

$Y$ irreduzibel in $U$ $\Rightarrow Y$ irreduzibel in $X$

$\Rightarrow$ (14) $\overline{Y}$ irreduzibel

$\Rightarrow Y\mapsto\overline{Y}\mapsto\overline{Y}\cap U=Y$. $\checkmark$

$\emptyset\neq\underbrace{Z\cap U}_{\text{irred. (S. 13v.)}}\underset{\text{offen}}{\subset}Z$
damit dicht da $Z$ irreduzibel (Satz 13.ii)

$\Rightarrow$ Abbildung $\leftarrow$ wohldefiniert 

$\Rightarrow\overline{Z\cap U}=Z$ 
\end{proof}
\begin{prop}
Sei $(X,\mathcal{O}_{X})$ eine Pr�variet�t.

$\Rightarrow X$ noethersch (insbesondere quasikompakt) und irreduzibel.
\end{prop}
\begin{proof}
Sei $X=\bigcup_{i=1}^{n}$ endliche eff. aff. �berdeckung und $X\supseteq Z_{1}\supseteq Z_{2}\supseteq\cdots$
eine absteigende Kette abgeschlossener Teilmengen.

$\Rightarrow U_{i}\cap Z_{1}\supseteq U_{i}\cap Z_{2}\supseteq\cdots$

$\Rightarrow$ abgeschlossene Teilmengen in $U_{i}$

$\Rightarrow\forall i$ $\exists n_{i}$: $U_{i}\cap Z_{n_{i}}=U_{i}\cap Z_{i+m}$.
 Setzen von $n:=\max n_{i}$ liefert:

$\forall i=1,\ldots,n$ $\forall m\geq n$: $U_{i}\cap Z_{m}=U_{i}\cap Z_{m+1}$

$\Rightarrow(Z_{i})$ wird station�r da $Z_{m}=\bigcup U_{i}\cap Z_{m}$.

$\Rightarrow X$ noethersch.

Zeige, $X$ ist irreduzibel:

Sei $X=X_{1}\cup\cdots\cup X_{n}$ die Zerlegung in irreduzibele Komponenten.

$\mathbb{A}$ $n\geq2$

$\Rightarrow\exists i_{0}\in\{2,\ldots,n\}$: $X_{1}\cap X_{i_{0}}\neq\emptyset$.
(Andernfalls gilt: $X=X_{1}\sqcup\underbrace{X\backslash X_{1}}_{=X_{2}\cup\cdots\cup X_{n}\text{ abg.}}$.
Widerspruch zu $X$ zusammenh�ngend.)

Sei ohne Einschr�nkung $i_{0}=2$. Sei $x\in X_{1}\cap X_{2}$, $x\in U\subset X$
offene affine �berdeckung (d.h. affine Variet�t).

$U$ irreduzibel $\Rightarrow\overline{U}$ (Abschluss in $X$) $\subseteq X_{j}$
f�r ein $j\in\{1,\ldots,n\}$

\textbf{Jedoch}: $x\in X_{i}\cap U\subseteq U$ irreduzibel ist $\underbrace{\overline{X_{i}\cap U}}_{\subset\overline{U}\subset X_{i}}=X_{i}$,
$i=1,2$

$\Rightarrow X_{1},X_{2}\subseteq X_{j}$. Widerspruch zu maximale
Komponente.
\end{proof}




\section{Offene Untervariet�ten}

,,Offene Teilmengen von affinen Variet�ten (abgeschlossene beliebige
 Pr�variet�ten) sind wieder Pr�variet�ten`` (aber i.A. nicht affin!)
\begin{lem}[orig. 41]
Sei $X$ affine Variet�t, $f\in\mathcal{O}_{X}(X)$, $\mathcal{D}(f)\subseteq X$.
Die Lokalisierung von $\Gamma(X)=\mathcal{O}_{X}(X)$ in $f$,
\[
\Gamma(X)_{f}=\Gamma(X)[T]/(Tf-1)
\]

ist eine integere endlich erzeugte $k$-Algebra. Dabei $(Y,\mathcal{O}_{Y})$
bezeichnet die zugeh�rige Variet�t. Es folgt:
\[
(D(f),\mathcal{O}_{X|_{D(f)}})\cong(Y,\mathcal{O}_{Y})
\]

als R�ume mit Funktionen, d.h. $(D(f),\mathcal{O}_{X|_{D(f)}})$ ist
affine Variet�t.
\end{lem}
\begin{proof}
$\mathcal{O}_{X}(\mathcal{D}(f))=\mathcal{O}_{X}(X)_{f}$ muss affiner
Koordinatenring von $(\mathcal{D}(f),\mathcal{O}_{X|_{\mathcal{D}(f)}})$
sein, wenn letzterer Raum von Funktionen affin ist. $X\subseteq\mathbb{A}^{n}(k)$
korrespondiert zu dem Radikalideal:
\begin{align*}
\mathfrak{A} & =I(X)\subseteq k[T_{1},\ldots,T_{n}]\ \subset\ \mathfrak{A}'=(\mathfrak{A},fT_{n+1}-1)\subseteq k[T_{1},\ldots,T_{n+1}]
\end{align*}

mit Koordinatenringen:
\begin{align*}
\Gamma(X) & =k[T_{1},\ldots,T_{n}]/\mathfrak{A}\\
\Gamma(Y) & =\Gamma(X)_{f}=(k[T_{1},\ldots,T_{n}]/\mathfrak{A})[T_{n+1}]/(T_{n+1}f-1)\\
 & =k[T_{1},\ldots,T_{n+1}]/\mathfrak{A}'
\end{align*}

F�r $Y=V(\mathfrak{A}')\subseteq\mathbb{A}^{n+1}(k)$ induziert die
Abbildung
\[
\xymatrix{Y\subseteq\mathbb{A}^{n+1}(k)\ar@{-->}[d] & (x_{1},\ldots,x_{n+1})\ar@{|->}[d] & T_{i}\\
X\subseteq\mathbb{A}^{n}(k) & (x_{1},\ldots,x_{n}) & T_{i}\ar@{|->}[u]
}
\]

eine Bijketion $Y\xrightarrow{j}\mathcal{D}_{X}(f)$  mit Umkehrabbildung
\begin{claim*}
$j$ ist Isomorphismus von R�umen mit Funktionen:
\begin{enumerate}
\item $j$ ist \emph{stetig} (als Einschr�nkung stetiger Funktionen) $\checkmark$
\item $j$ ist \emph{offen}: $g\in\Gamma(X)$, $\Gamma(Y)=\Gamma(X)_{f}$,
$\frac{g}{f^{n}}\in\Gamma(X)_{f}$,
\begin{align*}
j\left(D_{Y}\left(\frac{g}{f^{n}}\right)\right) & =j\left(\mathcal{D}_{Y}(gf)\right) & f\text{ Einheit}\\
 & =\mathcal{D}_{X}(gf)\text{ offen}
\end{align*}

$\Rightarrow j$ Hom�morphismus.
\item $j$ induziert $\forall g\in\Gamma(X)$ Isomorphismen:
\begin{align*}
\mathcal{O}_{X}(\mathcal{D}(fg)) & \longrightarrow\Gamma(Y)_{g}\\
s & \longmapsto s\circ j
\end{align*}
\end{enumerate}
\end{claim*}
\end{proof}




\section{Funktionenk�rper einer Pr�variet�t}
\begin{defn}[orig. 43]
F�r eine Pr�variet�t $X$ sind die rationalen Funktionenk�rper aller
nicht-leeren affinen offenen Teilmengen in nat�rlicher Weise zu einander
isomorph. Dieser K�rper nennen wir den \textbf{rationalen Funktionenk�rper
}von $X$: $K(X)$.
\end{defn}
\begin{proof}
$\emptyset\neq U$, $V\subset X$ affine offene Untervariet�t. Da
$X$ irreduzibel ist, gilt nach \emph{Satz 40}:
\[
\emptyset\neq U\cap V\subset U\text{ offen}.
\]

Nach Definition von $\mathcal{O}_{X}$ ist 
\[
\mathcal{O}_{X}(U)\subseteq\mathcal{O}_{X}(U\cap V)\subset K(U)=\text{Quot}(\mathcal{O}_{X}(U)).
\]

Das impliziert $\text{Quot}(\mathcal{O}_{X}(U\cap V))=K(U)$. Aus
Symmetriegr�nden ist aber $K(V)=\text{Quot}(\mathcal{O}_{X}(U\cap V))$.
\end{proof}
\begin{rem}[orig. 44]
Das Bild $K(\ )$ des Funktionenk�rpers ist \textbf{nicht} funktoriell!
F�r $X\xrightarrow{f}Y$ Morphismus affiner Variet�ten ist die Abbildung
auf den Koordinatenringen $\Gamma(Y)\rightarrow\Gamma(X)$ i.A. \textbf{nicht}
injektiv, also gibt es kein $K(Y)\hookrightarrow K(X)$.

\emph{Jedoch}: Eine Isomorphie $X\xrightarrow{\sim}Y$ induziert $K(Y)\xrightarrow{\sim}K(X)$.
Allgemeiner sei $X\rightarrow Y$ Morphismus mit Bild $\subset Y$
offen ($\Rightarrow$ dicht. Sp�ter $X\rightarrow Y$ \textbf{dominant},
d.h. Bild $\subset Y$ dicht.) induziert in funktioreller Weise eine
Abbildung $K(Y)\hookrightarrow K(X)$.
\end{rem}
\begin{prop}[orig. 45]
Sei $X$ eine Pr�variet�t, $V\subseteq U\subseteq X$ offen. Es folgt:

$\mathcal{O}_{X}(U)\subset K(X)$ $k$-Unteralgebra.

$\mathcal{O}(U)\rightarrow\mathcal{O}(V)$ ist Inklusion von Teilmengen
des Funktionenk�rpers $K(X)$.

Insbesondere gilt f�r $U,V\subset X$ offen:
\[
\mathcal{O}_{X}(U\cup V)=\mathcal{O}_{X}(U)\cap\mathcal{O}_{X}(V).
\]
\end{prop}
\begin{proof}
\mbox{}
\begin{enumerate}
\item[2.] Sei $\mathcal{O}(X)\ni f:X\rightarrow k$. Dann ist $f^{-1}(0)\subseteq X$
abgeschlossen, da f�r $W\subseteq X$ offen affin beliebig gilt:
\[
f^{-1}(0)\cap W=V(f_{|_{W}}).
\]
Dazu macht man sich klar: ,,abgeschlossen`` ist eine lokale Eigenschaft,
und die $W$ bilden eine Basis der Topologie. 

$\Rightarrow\mathcal{O}(U)\hookrightarrow\mathcal{O}(V)$, $f\mapsto\sigma$
injektiv f�r $\emptyset\neq V\subseteq U\subset X$ offen.

$\Rightarrow V\subset f^{-1}(0)$ 

$\Rightarrow f^{-1}(0)=U$ 

$\Rightarrow f\equiv0$.
\item[1.] (i) $U\supset W$ offen affine Variet�t. $\Rightarrow$
\[
\xymatrix{\mathcal{O}(W)\ar[r]\subset & K(W)\ k\text{-Variet�t}\\
\mathcal{O}(U)\ar@{^{(}->}[u]
}
\]
\item Verklebungsaxiom: 
\end{enumerate}
\end{proof}




\section{Abgeschlossene Unterpr�variet�ten}
\label{sec:abg-untervarietaeten}

Sei $X$ eine Pr�variet�t, $Z\subseteq X$ abgeschlossen und irreduzibel.

\textbf{Ziel.} $(Z,\mathcal{O}_{Z}')$ Raum von Funktionen erkl�ren.
Definiere dazu f�r $U \subseteq Z$ offen:

\[
  \mathcal{O}_{Z}'(U):=\{f\in\text{Abb}(U,k)\mid\forall x\in U\ \exists x\in V\subseteq X\text{ offen},\ g\in\mathcal{O}_{X}(V) \text{ mit } f|_{U\cap V}=g|_{U\cap V}\}
\]

Damit ist $(Z,\mathcal{O}'_{Z})$ Raum von Funktionen (klar!) mit $\mathcal{O}'_{X}=\mathcal{O}_{X}$.
\begin{lem}[orig. 46]
  \label{lem:abg-untervarietaeten-affine-varietaeten}
  Seien $X\subseteq\mathbb{A}^{n}(k)$ eine irreduzible, affin-algebraische Menge
  und $Z\subseteq X$ ein irreduzibler abgeschlossener Teilraum. Dann
  ist $(Z,\mathcal{O}_{Z})=(Z,\mathcal{O}'_{Z})$.

  Bezeichne ab jetzt stets $\mathcal{O}_{Z}$ f�r $\mathcal{O}_{Z'}$.
\end{lem}
\begin{proof}
  $Z\subseteq X$ ist in beiden F�llen mit der Teilraumtopologie ausgestattet!
  Ferner wissen wir, dass der Morphismus $Z\hookrightarrow X$ affin-algebraischer Mengen einen Morphismus $(Z,\mathcal{O}_{Z})\rightarrow(X,\mathcal{O}_{X})$
  von Pr�variet�ten induziert. Nach Definition von $\mathcal{O}'$
  folgt dann:
  \[
    \mathcal{O}'_{Z}(U)\subseteq\mathcal{O}_{Z}(U)\quad\text{f�r }U\subseteq Z\ \text{offen, denn:}
  \] 
  Ist $f \in \mathcal{O}_{Z}'(U)$ und $x \in U$ so existieren nach Definition eine offene Umgebung $x \in V_{x} \subseteq X$ und ein $g \in \mathcal{O}_{X}(V_{x})$ d.d. $f|_{U \cap V_{x}} = g|_{U \cap V_{x}}$. Damit gilt $g|_{Z \cap V_x} \in \mathcal{O}_{Z}(Z \cap V_{x})$. Mit dem Verklebungsaxiom erhalten wir also $f \in \mathcal{O}_{Z}(U)$.


  Sei $f\in\mathcal{O}_{Z}(U)$ und $x\in U$ beliebig. Es folgt: $\exists h\in\Gamma(Z)$
  mit $x\in D(h)\subseteq U$ und
  \[
    f|_{D(h)}=\frac{g}{h^{n}}\in\Gamma(Z)_{h}=\mathcal{O}_{Z}(D(h))
  \]

  f�r $n\geq0$ und $g\in\Gamma(Z)$ geeignet. Lifte $g,h\in\Gamma(Z)\twoheadleftarrow\Gamma(X)$
  zu $\overline{g},\overline{h}\in\Gamma(X)$ und setze $V:=D(\overline{h})\subseteq X$.

  $\Rightarrow x\in V$, $\frac{\overline{g}}{\overline{h}^{n}}\in\mathcal{O}_{X}(D(\overline{h}))$
  und $f|_{U\cap V}=\frac{\overline{g}}{\overline{h}^{n}}|_{U\cap V}$.

  $\Rightarrow f\in\mathcal{O}'_{Z}(U)$.
\end{proof}
\begin{cor}[orig. 47]
  \label{cor:abg-untervarietaeten-sind-praevarietaeten}
  Wenn $X$ eine Pr�variet�t ist, und $Z\subseteq X$ irreduzibel und abgeschlossen, dann ist $(Z,\mathcal{O}_{Z})$ ebenfalls eine Pr�variet�t.
\end{cor}
\begin{proof}
  Es ist $X=\bigcup_{i}X_{i}$ f�r eine endliche affin-offene �berdeckung $(X_{i})_{i}$.
  Damit ist 
  \[
    Z=\bigcup_{i}\left(Z\cap X_{i}\right) :=\bigcup_{i} Z_{i}
  \]
  

  mit $(Z_{i}, \mathcal{O}_{Z_{i}})$ affine Variet�t nach Lemma $\ref{lem:abg-untervarietaeten-affine-varietaeten}$.
\end{proof}




\section*{Beispiele (Projektiver Raum und projektive Variet�ten)}

\section{Homogene Polynome}
\begin{defn}[orig. 48]
Ein Polynom $f\in k[X_{0},\ldots,X_{n}]$ hei�t \textbf{homogen vom
Grad}\index{homogen} $d\in\mathbb{Z}_{\geq0}$, wenn $f$ die Summe
von Monomen von Grad $d$ ist. (Insbesondere ist f�r jedes $d$ das
Nullpolynom homogen von Grad $d$.)

\emph{Bezeichne} $k[X_{0},\ldots,X_{n}]_{d}$ der Untervektorraum
der Polynome vom Grad $d$.
\end{defn}
\begin{rem}[orig. 49]
Da \#$k$ unendlich ist, ist $f$ homogen vom Grad $d$. 

$\Leftrightarrow f(\lambda x_{0},\ldots,\lambda x_{n})=\lambda^{d}f(x_{0},\ldots,x_{n})$
$\forall x_{0},\ldots,x_{n}\in k$, $\lambda\in k^{\times}$. 

Es gilt: $k[X_{0},\ldots X_{n}]=\bigoplus_{d\geq0}k[X_{0},\ldots,X_{n}]_{d}$.
\end{rem}
\begin{lem}[orig. 50]
F�r $i\in\{0,\ldots,n\}$ und $d\geq0$ haben wir bijektive $k$-lineare
Abbildungen
\begin{align*}
k[X_{0},\ldots,X_{n}]_{d} & \longrightarrow\text{Polynome in }k[T_{0},\ldots,\hat{T}_{i},\ldots,T_{n}]\text{ v. Grad }\leq d\\
f & \overset{\Phi_{i}^{d}}{\longmapsto}f(T_{0},\ldots,\underbrace{1}_{i},\ldots,T_{n})\\
X_{i}^{d}g\left(\frac{X_{0}}{X_{i}},\ldots,\frac{\hat{X_{i}}}{X_{i}},\ldots,\frac{X_{n}}{X_{i}}\right) & \overset{\Psi_{i}^{d}}{\longmapsfrom}g
\end{align*}

\textbf{Dehomogenisierung }bzw. \textbf{Homogenisierung.}
\end{lem}
\begin{proof}
Es reicht, $\Psi_{i}^{d}\circ\Phi_{i}^{d}=\text{id}$, $\Phi_{i}^{d}\circ\Psi_{i}^{d}=\text{id}$
auf Monomen nachzurechnen, da alle Abbildungen $k$-linear sind. 
\end{proof}
Oft ist es n�tzlich, 
\[
k[T_{0},\ldots,\hat{T_{i}},\ldots,T_{n}]\text{ mit }
\]
 mit $k\left[\frac{X_{0}}{X_{i}},\ldots,\frac{\hat{X_{i}}}{X_{i}},\ldots,\frac{X_{n}}{X_{i}}\right]\underset{\text{Unterring}}{\subset}k(X_{0},\ldots,X_{n})$.



\newpage{}

\printindex{}
\end{document}
