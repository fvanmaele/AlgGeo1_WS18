\documentclass[12pt,a4paper]{book}
\usepackage[T1]{fontenc}
\usepackage[utf8]{inputenc}
\usepackage{geometry}
\geometry{verbose,tmargin=2cm,bmargin=2cm,lmargin=2cm,rmargin=2cm}
\pagestyle{headings}
\usepackage[ngerman]{babel}
\usepackage{verbatim}
\usepackage{amsmath}
\usepackage{amsthm}
\usepackage{amssymb}
\usepackage{stmaryrd}
\usepackage{makeidx}
\makeindex
\usepackage{setspace}
\usepackage[all]{xy}
\onehalfspacing
\usepackage[bookmarks=true]{hyperref}

%% Theorems (numbered by Part)
\newtheorem{thm}{Theorem}[chapter]
\theoremstyle{definition}
\newtheorem{example}[thm]{Beispiel}
\theoremstyle{definition}
\newtheorem{defn}[thm]{Definition}
\theoremstyle{plain}
\newtheorem{prop}[thm]{Satz}
\theoremstyle{plain}
\newtheorem{cor}[thm]{Korollar}
\theoremstyle{plain}
\newtheorem{lem}[thm]{Lemma}
\theoremstyle{remark}
\newtheorem{rem}[thm]{Bemerkung}
\theoremstyle{plain}

%% Theorems (unnumbered)
\newtheorem*{question*}{Frage}
\theoremstyle{remark}
\newtheorem*{claim*}{Behauptung}
\theoremstyle{definition}
\newtheorem*{example*}{Beispiel}
\theoremstyle{plain}

%% User-specified commands
\DeclareMathOperator{\rad}{rad}
\DeclareMathOperator{\Spec}{Spec}
\DeclareMathOperator{\Quot}{Quot}
\renewcommand{\labelenumi}{(\roman{enumi})}
\renewcommand{\labelenumii}{\arabic{enumii}.}

\begin{document}

%% Title page
\title{Algebraische Geometrie I}
\author{Prof. Dr. Venjakob}
\maketitle

\tableofcontents{}
\newpage{}

\section*{Literatur}
\begin{itemize}
\item Görtz, Wedhorn. \emph{Algebraic Geometry I}
\item Hartshorne. \emph{Algebraic Geometry}
\item Shafarevich. \emph{Basic Algebraic Geometry 1 \& 2}
\item Grothendieck. \emph{Eléments de géometrie algébrique, EGA I-IV}
\end{itemize}

\paragraph{Kommutative Algebra}
\begin{itemize}
\item Brüske, Ischebeck, Vogel. \emph{Kommutative Algebra}
\item Kunz. \emph{Einführung in die kommutative Algebra und algebraische Geometrie}
\end{itemize}

\chapter{Prä-Varietäten}
\label{chap:prae-varietaeten}

\include{Chapter1/AlgGeo1-Chapter1-1_Einfuehrung}

\include{Chapter1/AlgGeo1-Chapter1-2_Die-Zariski-Topologie}

\include{Chapter1/AlgGeo1-Chapter1-3_Affine-algebraische-Mengen}

\include{Chapter1/AlgGeo1-Chapter1-4_Der-Hilbertsche-Nullstellensatz}

\include{Chapter1/AlgGeo1-Chapter1-5_Korrespondenz-zwischen-Radikalidealen}


\section{Irreduzible topologische Räume}
\label{sec:irreduzibilitaet-top}

Die folgenden topologischen Begriffe sind nur interessant, da $\mathbb{A}^{n}(k)$
($n>0$) kein Hausdorff'scher Raum ist.
\begin{defn}
  \label{def:irreduzibel}
  Ein topologischer Raum $X$ heißt \textbf{irreduzibel}\index{irreduzibel},
  falls $X\neq\emptyset$ und $X$ sich \emph{nicht} als Vereinigung
  zweier echter abgeschlossener Teilmengen darstellen lässt, d.h
  \[
    X=A_{1}\cup A_{2},\ A_{i}\ \text{abg.}\quad\implies\quad A_{1}=X\text{ oder }A_{2}=X.
  \]

  $Z\subseteq X$ heißt irreduzibel, falls $Z$ mit der induzierten Topologie
  irreduzibel ist.
\end{defn}
\begin{prop}
  \label{prop:charakterisierung-irreduzibel}
  Für einen topologischen Raum $X \neq \emptyset$ sind äquivalent:
  \begin{enumerate}
  \item $X$ ist irreduzibel.
  \item Je zwei nichtleere offene Teilmengen von $X$ haben nicht-leeren
    Durchschnitt.
  \item Jede nichtleere offene Teilmenge $U\subseteq X$ ist dicht in $X$.
  \item Jede nichtleere offene Teilmenge $U\subseteq X$ ist zusammenhängend.
  \item Jede nichtleere offene Teilmenge $U\subseteq X$ ist irreduzibel.
  \end{enumerate}
\end{prop}
\begin{proof}
  \mbox{}
  \begin{itemize}
  \item $(i)\Leftrightarrow(ii)$

    Komplementärmengen.
  \item $(ii)\Leftrightarrow(iii)$ 

    Es ist: $U\subseteq X$ dicht $\Leftrightarrow U\cap O\neq\emptyset$
    für jedes offene $\emptyset\neq O\subseteq X$.
  \item $(iii)\Rightarrow(iv)$

    Klar. 
  \item $(iv)\Rightarrow(iii)$

    Sei $\emptyset\neq U$ offen und zusammenhängend. Es folgt:
    \[
      U=U_{1}\sqcup U_{2},\qquad\emptyset\neq U_{i}\underset{\text{offen}}{\subseteq}U\underset{\text{offen}}{\subseteq}X
    \]
    Damit ist $U_{1}\cap U_{2}=\emptyset$, ein Widerspruch zu (iii).
  \item $(v)\Rightarrow(i)$ 

    Klar. $(U=X)$
  \item $(iii)\Rightarrow(v)$

    Sei $\emptyset\neq U\underset{\text{offen}}{\subseteq}X$. Ist $\emptyset\neq V\underset{\text{offen}}{\subseteq}U$,
    so ist $V\underset{\text{offen}}{\subseteq}X$. Es folgt: $V$ ist
    dicht in $X$ und irreduzibel in $U$. Mit $(iii)\Rightarrow(i)$
    folgt, dass $U$ irreduzibel ist. 

  \end{itemize}
\end{proof}
\begin{lem}
  \label{lem:irreduzibel-abschluss}
  Eine Teilmenge $Y$ ist genau dann irreduzibel, wenn ihr Abschluss $\overline{Y}$ dies ist.
\end{lem}
\begin{proof}
  $Y$ irreduzibel

  $\Leftrightarrow\forall U,V\subseteq X$ offen mit $U\cap Y\neq\emptyset\neq V\cap Y$,
  gilt $Y\cap(U\cap V)\neq\emptyset$.

  $\Leftrightarrow\overline{Y}$ irreduzibel 
\end{proof}
\begin{defn}
  \label{def:irreduzible-komponente}
  Eine maximale irreduzible Teilmenge eines topologischen Raumes $X$
  heißt \textbf{irreduzible Komponente}\index{irreduzible Komponente}
  von $X$.
\end{defn}
\begin{rem}
  \label{rem:irreduzibel}
  \mbox{}
  \begin{enumerate}
  \item Jede irreduzible Komponente ist abgeschlossen nach Lemma 14.
  \item $X$ ist Vereinigung seiner irreduziblen Komponenten, \emph{denn}: 

    die Menge der irreduziblen Teilmengen von $X$ ist \textbf{induktiv
      geordnet}: für jede aufsteigende Kette irreduzibler Teilmengen ist
    die Vereinigung wieder irreduzibel (Satz 13 (ii)). Mit dem \textbf{Lemma
      von Zorn} folgt: Jede irreduzible Teilmenge ist in einer irreduziblen
    Komponente enthalten. Damit ist jeder Punkt in einer irreduziblen
    Komponente enthalten.
  \end{enumerate}
\end{rem}



\include{Chapter1/AlgGeo1-Chapter1-7_Irreduzible-algebraische-Mengen}

\include{Chapter1/AlgGeo1-Chapter1-8_Quasikompakte-und-noethersche-topologische-Raeume}

\include{Chapter1/AlgGeo1-Chapter1-9_Morphismen-von-affinen-algebraischen-Mengen}

\include{Chapter1/AlgGeo1-Chapter1-10_Unzulaenglichkeiten}

\include{Chapter1/AlgGeo1-Chapter1-11_Der-affine-Koordinatenring}

\include{Chapter1/AlgGeo1-Chapter1-12_Funktorielle-Eigenschaften-des-Koordinatenrings}

\include{Chapter1/AlgGeo1-Chapter1-13_Raeume-mit-Funktionen}

\include{Chapter1/AlgGeo1-Chapter1-14_Der-Raum-mit-Funktionen-zu-einer-affin-algebraischen-Menge}

\include{Chapter1/AlgGeo1-Chapter1-15_Funktorialitaet-der-Konstruktion}

\include{Chapter1/AlgGeo1-Chapter1-16_Definition-von-Praevarietaeten}

\include{Chapter1/AlgGeo1-Chapter1-17_Vergleich-mit-differenzierbaren-komplexen-Mannigfaltigkeiten}


\section{Topologische Eigenschaften von Prävarietäten}
\label{sec:topologische-eigenschaften-von-praevarietaeten}
\begin{lem}
  \label{lem:bijektion-irred-teilraeume}
  Für einen topologischen Raum $X$ und $U\subseteq X$ offen haben
  wir eine Bijektion
  \begin{align*}
    \{Y\subseteq U\text{ irred. abg.}\} & \longleftrightarrow\{Z\subseteq X\text{ irred. abg. mit }Z\cap U\neq\emptyset\}\\
    Y & \longmapsto\overline{Y}\text{ (Abschluss in }X)\\
    Z\cap U & \longmapsfrom Z
  \end{align*}
\end{lem}
\begin{proof}
  Lemma \ref{lem:irreduzibel-abschluss}: $Y\subseteq X$ irreduzibel
  $\Leftrightarrow\overline{Y}\subseteq X$ irreduzibel.

  $Y\subseteq U$ abgeschlossen $\Leftrightarrow\exists A\subseteq X$
  abgeschlossen: $Y=U\cap A$.

  $\Rightarrow Y\subseteq\overline{Y}\subseteq A$ $\Rightarrow Y=U\cap\overline{Y}$

  $Y$ irreduzibel in $U$ $\Rightarrow Y$ irreduzibel in $X$

  $\Rightarrow$ $\overline{Y}$ irreduzibel nach \ref{lem:irreduzibel-abschluss}

  $\Rightarrow Y\mapsto\overline{Y}\mapsto\overline{Y}\cap U=Y$. $\checkmark$

  $\emptyset\neq Z\cap U \subseteq Z$
  damit dicht da $Z$ irreduzibel (Satz \ref{prop:charakterisierung-irreduzibel} ii. und v.)

  Also ist die Abbildung $\leftarrow$ wohldefiniert.

  $\Rightarrow\overline{Z\cap U}=Z$ 
\end{proof}
\begin{prop}
  \label{prop:praevarietaeten-noethersch-irreduzibel}
  Sei $(X,\mathcal{O}_{X})$ eine Prävarietät.

  Dann ist $X$ noethersch (insbesondere quasikompakt) und irreduzibel.
\end{prop}
\begin{proof}
  Sei $X=\bigcup_{i=1}^{n}$ endliche offene aff. Überdeckung und $X\supseteq Z_{1}\supseteq Z_{2}\supseteq\cdots$
  eine absteigende Kette abgeschlossener Teilmengen.

  $\Rightarrow U_{i}\cap Z_{1}\supseteq U_{i}\cap Z_{2}\supseteq\cdots$
  , ist eine absteigende Kette abgeschlossener Teilmengen von $U_{i}$

  $\Rightarrow\forall i$ $\exists n_{i} \in \mathbb{N}$: $U_{i}\cap Z_{n_{i}}=U_{i}\cap Z_{i+m}$ für alle $m \in \mathbb{N}$.
  Setzen wir $n:=\max n_{i}$, so folgt:

  $\forall i=1,\ldots,n$ $\forall m\geq n$: $U_{i}\cap Z_{m}=U_{i}\cap Z_{m+1}$

  $\Rightarrow(Z_{i})_{i}$ wird stationär da $Z_{m}=\bigcup_{i} U_{i}\cap Z_{m}$.

  $X$ ist demnach noethersch.

  $X$ ist weiter irreduzibel:

  Sei $X=X_{1}\cup\cdots\cup X_{n}$ die Zerlegung in irreduzible Komponenten.

  Angenommen es wäre $n\geq2$.

  $\Rightarrow\exists i_{0}\in\{2,\ldots,n\}$: $X_{1}\cap X_{i_{0}}\neq\emptyset$.
  (Andernfalls gilt: $X=X_{1}\sqcup\underbrace{X\backslash X_{1}}_{=X_{2}\cup\cdots\cup X_{n}\text{ abg.}}$, im Widerspruch dazu, dass $X$ zusammenhängend ist.)

  Sei ohne Einschränkung $i_{0}=2$. Sei $x\in X_{1}\cap X_{2}$, $x\in U\subseteq X$ offen, affin (d.h. affine Varietät).

  $U$ irreduzibel $\Rightarrow\overline{U}$ (Abschluss in $X$) $\subseteq X_{j}$
  für ein $j\in\{1,\ldots,n\}$

  \textbf{Jedoch}: Da $x\in X_{i}\cap U\subseteq U$ irreduzibel ist, ist $\underbrace{\overline{X_{i}\cap U}}_{\subseteq\overline{U}\subseteq X_{i}}=X_{i}$,
  $i=1,2$

  $\Rightarrow X_{1},X_{2}\subseteq X_{j}$. Widerspruch zu maximale
  Komponente.
\end{proof}



\include{Chapter1/AlgGeo1-Chapter1-19_Offene-Untervarietaeten}

\include{Chapter1/AlgGeo1-Chapter1-20_Funktionenkoerper-einer-Praevarietaet}

\include{Chapter1/AlgGeo1-Chapter1-21_Abgeschlossene-Unterpraevarietaeten}

\include{Chapter1/AlgGeo1-Chapter1-22_Homogene-Polynome}

\include{Chapter1/AlgGeo1-Chapter1-23_Definition-des-projektiven-Raumes}


\section{Projektive Varietäten}
\label{sec:projektive-varietaeten}
\begin{defn}[orig. 54]
  \label{def:projektive-varietaeten}
  Abgeschlossene Unterprävarietäten eines projektiven Raumes $\mathbb{P}^{n}(k)$
  heißen \textbf{projektive Varietäten}.
\end{defn}
Vorsicht: für $x=(x_{0}:\ldots:x_{n})\in\mathbb{P}^{n}$, $f\in k[X_{0},\ldots,X_{n}]$
ist $f(x_{1},\ldots,x_{n})$ \emph{nicht} wohldefiniert, da von Repräsentaten
abhängig, d.h. $f$ kann \emph{nicht}\textbf{ }als Funktion auf $\mathbb{P}^{n}$
aufgefasst werden. Für \emph{homogene}\textbf{ }Polynome $f_{1},\ldots,f_{n}\in k[X_{0},\ldots X_{n}]$
(nicht notwendig vom selben Grad) können wir dennoch Verschwindungsmengen
definieren:
\[
  V_{+}(f_{1},\ldots,f_{n})=\{(x_{0}:\ldots:x_{n})\in\mathbb{P}^{n}\mid f_{j}(x_{0},\ldots,x_{n})=0\ \forall j\}
\]

Da $V_{+}(f_{1},\ldots,f_{n})\cap U_{i}=V(\Phi_{i}(f_{1}),\ldots,\Phi_{i}(f_{m}))$
ist $V_{+}(f_{1},\ldots,f_{m})$ abgeschlossen in $\mathbb{P}^{n}$.
Ist $V_{+}(f_{1},\ldots,f_{n})$ irreduzibel, so erhalten wir eine
projektive Varietät. In der Tat entstehen alle projektiven Varietäten
auf diese Weise, wie der folgende Satz zeigt:
\begin{prop}[orig. 55]
  \label{prop:charakterisierung-projektive-varietaeten}
  Sei $Z\subseteq\mathbb{P}^{n}(k)$ eine projektive Varietät. Dann
  existieren homogene Polynome $f_{1},\ldots,f_{n}\in k[X_{0},\ldots,X_{n}]$,
  so dass
  \[
    Z=V_{+}(f_{1},\ldots,f_{n})
  \]

  gilt.
\end{prop}
\begin{proof}
  Betrachte: 
  \[
    \begin{array}{cc}
      \\
      \\
    \end{array}
  \]

  $f| :f^{-1}(U_{i})\longrightarrow U_{i}$ ist Morphismus
  von Prävarietäten. Dann ist $f$ selbst ein Morphismus von Prävarietäten: $\emph{lokal}$ ist die Aussage klar, $\emph{global}$ verklebt man.
  \begin{align*}
    \overline{Y}:= & Y\cup\{0\}\text{, der Abschluss von }Y\text{ in }\mathbb{A}^{n+1}(k)\\
    \mathfrak{a}:= & I(\overline{Y})\subseteq k[X_{0},\ldots,X_{n}]
  \end{align*}

  Behauptung: $\mathfrak{a}$ wird von homogenen Polynomen erzeugt\emph{.
    Denn:} Sei für $g\in\mathfrak{a}$, $g=\sum_{d}g_{d}$ die Zerlegung in homogene
  Bestandteile vom Grad $d$. $\overline{Y}$ ist Vereinigung von Ursprungsgeraden
  im $k^{n+1}$, d.h. $\forall\lambda\in k^{\times}$ gilt:
  \[
    g(x_{0},\ldots,x_{n})=0\ \Leftrightarrow\ g(\lambda x_{0},\ldots,\lambda x_{n})=0
  \]

  Beweis durch Widerspruch: \emph{Angenommen} nicht alle $g_{d}$ liegen in $\mathfrak{a}$.

  $\Rightarrow\exists(x_{0},\ldots,x_{n})\in\mathbb{A}^{n+1}(k)$, so
  dass $g(x_{0},\ldots,x_{n})=0$, aber $g_{d_{0}}(x_{0},\ldots,x_{n})\neq0$.

  $\Rightarrow0\,\neq\sum_{d}g_{d}(x_{0},\ldots,x_{n})T^{d}\in k[T]$

  $\Rightarrow\exists\lambda\in k^{\times}$: $0\neq\sum_{d}g_{d}(x_{0},\ldots,x_{n})\lambda^{d}=\sum_{d}g_{d}(\lambda x_{0},\ldots,\lambda x_{n})=g(\lambda x_{0},\ldots,\lambda x_{n})=0$.
  Widerspruch.

  $\Rightarrow\mathfrak{a}=(f_{1},\ldots,f_{m})$, mit $f_{j}$ homogen, also $Z=V_{+}(f_{1},\ldots,f_{m})$. 
  \begin{align*}
    Z\ni(x_{0}:\ldots:x_{n}) & \Leftrightarrow(\lambda x_{0},\ldots,\lambda x_{n})\in\overline{Y}\ \forall\lambda\in k^{\times}\text{ und }\neq0\\
                             & \Leftrightarrow f_{i}(x_{0},\ldots,x_{n})=0\ \forall1\leq i\leq n,\ (x_{0},\ldots,x_{n})\in\mathbb{P}^{n}
  \end{align*}

  %\rule[0.5ex]{1\columnwidth}{1pt}
\end{proof}
Zu Bemerkung \ref{rem:charakterisierung-homogen}:

Nach Satz \ref{prop:charakterisierung-reg-fkt-projektiver-raum} und Definition von $\mathcal{O}_{Z}'$ folgt: Ist $X$
eine projektive Varietät und $U\subseteq X$ offen, so erhalten wir 

($\dagger$) \ $\mathcal{O}_{X}(U)=\{f:U\rightarrow k\mid\forall x\in U\ \exists x\in V\underset{\text{offen}}{\subseteq}U,\ g,h\in k[X_{0},\ldots,X_{n}]$
homogen vom gleichen Grad mit $h(v)\neq0, \ f(v)=\frac{g(v)}{h(v)},\ \forall v\in V\}$.

Insbesondere gilt:
\begin{prop}[orig. 56]
  \label{prop:charakterisierung-morphismen-proj-varietaeten}
  Seien $V\subseteq\mathbb{P}^{m}(k)$, $W\subseteq\mathbb{P}^{n}(k)$
  projektive Varietäten und
  \[
    \phi: V \longrightarrow W
  \]

  eine Abbildung. Dann ist $\phi$ eine Morphismus genau dann, wenn
  zu jedem $x\in V$ eine offene Umgebung $x\in U_{x}\subseteq V$ und
  homogene Polynome $f_{0},\ldots,f_{n}\subseteq k[X_{0},\ldots,X_{m}]$
  vom selben Grad existieren mit
  \[
    \phi(y)=(f_{0}(y),\ldots,f_{n}(y))\quad\forall y\in U_{x}
  \]
\end{prop}
\begin{proof}
  \mbox{}
  \begin{itemize}
  \item ``$\Rightarrow$'': Übung.
  \item ``$\Leftarrow$'':
    \begin{enumerate}
    \item $\phi$ stetig: Sei $Z\subseteq W$ abgeschlossen. Ohne Einschränkung
      $Z=V_{+}(g)\cap W$ für ein homogenes Polynom $g$. Dann berechnet
      sich das Urbild
      \[
        \phi^{-1}(Z)=V_{+}(g\circ\phi)\cap V.
      \]
      Auf $U_{x}$, $x\in V$, ist $g\circ\phi$ als homogenes Polynom in
      $X_{0},\ldots,X_{n}$ gegeben. 

      $\Rightarrow V(g\circ\phi)\cap U_{x}=\phi^{-1}(Z)\cap U_{x}$ abgeschlossen
      in $U_{x}$ für alle $x$.

      $\Rightarrow\phi^{-1}(Z)\subseteq V$ abgeschlossen.
    \item Zu zeigen: $\forall W'\subseteq W$ offen, $g\in\mathcal{O}_{W}(W')$
      ist $g\circ\phi\in\mathcal{O}_{V}(\phi^{-1}(W'))$.

      ($\dagger$) $\Rightarrow$ Es ex. eine offene Umgebung $W_{y}$ in $W'$
      mit $g=\frac{h}{q}$ auf $W_{y}$, $h,q$ homogen vom Grad $d$.

      $\Rightarrow\phi_{|U_{x}\cap\phi^{-1}(W_{y}):=\tilde{U}_{x}}$ ist
      auch von dieser Gestalt, also $\frac{h(f_{0},\ldots,f_{n})}{q(f_{0},\ldots,f_{n})}=g\circ\phi_{|\tilde{U}_{x}}\in\mathcal{O}_{V}(\tilde{U}_{x})$.
    \end{enumerate}
    Verklebungsaxiom $\Rightarrow$ $g\circ\phi\in\mathcal{O}_{V}(\phi^{-1}(V))$.
  \end{itemize}
\end{proof}




\section{Koordinatenwechsel in $\mathbb{P}^{n}$}
\label{sec:koordinatenwechsel-projektiver-raum}

Sei $A=(a_{ij})\in GL_{n+1}(k)$ eine invertierbare, lineare Abbildung $k^{n+1}\rightarrow k^{n+1}$. Dann überführt $A$ Ursprungsgeraden in Ursprungsgeraden, respektiert also die Äquivalenzrelation des projektiven Raumes. Wir erhalten Abbildungen:
\begin{align*}
  \mathbb{P}^{n}(k) & \overset{\phi_{A}}{\longrightarrow}\mathbb{P}^{n}(k)\\
  (x_{0}:\ldots:x_{n}) & \longmapsto\left(\sum_{i=0}^{n}a_{0i}x_{i}:\ldots:\sum_{i=0}^{n}a_{ni}x_{i}\right),
\end{align*}

die nach Satz \ref{prop:charakterisierung-morphismen-proj-varietaeten} ein Morphismus von Prävarietäten ist. Offensichtlich
gilt für $A,B\in GL_{n+1}(k)$:
\[
  \varphi_{A\cdot B}=\varphi_{A}\circ\varphi_{B}
\]

d.h. $\varphi_{A}$ ist insbesondere wieder ein Isomorphismus, \textbf{der
  durch $A$ bestimmte Koordinatenwechsel des $\mathbb{P}^{n}(k)$}.
Es \emph{bezeichne} Aut$(\mathbb{P}^{n}(k))$ die Gruppe der Automorphismen
von $\mathbb{P}^{n}(k)$. Es folgt:
\[
  \varphi_{-}:GL_{n+1}(k)\rightarrow\text{Aut}(\mathbb{P}^{n}(k)), A \mapsto \varphi_{A}
\]

ist ein Gruppenhomomorphismus mit 
\[
  Z:=\ker\varphi_{-}=\{\lambda E_{n+1},\ \mid \lambda\in k^{\times}\}
\]

der Untergruppe der Skalarmatrizen. \emph{Später}:
\[
  PGL_{n+1}(k):=GL_{n+1}(k)/Z\overset{\sim}\longrightarrow\text{Aut}(\mathbb{P}^{n}(k)),\quad Z\cong k^{\times}
\]

die \textbf{projektive lineare Gruppe}.
\begin{example*}
Sei $n=1$. Es ist
\begin{align*}
PGL_{2}(\mathbb{C}) & =\left\{ \begin{array}{rl}
\mathbb{P}^{1}(\mathbb{C}) & \rightarrow\mathbb{P}^{1}(\mathbb{C})\\
(z:w) & \mapsto(az+bw,cz+dw)
\end{array}\right\} \\
 & \leftrightarrow\text{Möbiustransformationen }z\mapsto\frac{az+b}{cz+d}
\end{align*}
\end{example*}



\include{Chapter1/AlgGeo1-Chapter1-26_Lineare-Unterraeume-des-projektiven-Raums}

\section{Kegel}
\label{sec:Kegel}

Sei $H\subseteq\mathbb{P}^{n}(k)$ Hyperebene, $p\in\mathbb{P}^{n}(k)\setminus H$,
$X\subseteq H$ abgeschlossene Unterprävarietät.
\[
  \overline{X,p}:=\bigcup_{q\in X}\overline{qp}
\]

heißt \textbf{Kegel von $X$ über $p$}, es handelt sich um eine
abgeschlossenen Untervarietät von $\mathbb{P}^{n}(k)$. Ohne Einschränkung: $H=V_{+}(X_{n})$,
$p=(0:\ldots:0:1)$ (geeigneter Koordinatenwechsel)
Für  
\begin{align*}
  X=V_{+}(f_{1},\ldots,f_{m})\subseteq\mathbb{P}^{n-1}(k)=H, & \quad f_{i}\in k[X_{0},\ldots,X_{n-1}]\\
  \Rightarrow \overline{X,p}=V_{+}(\tilde{f}_{1},\ldots,\tilde{f}_{m})\subseteq\mathbb{P}^{n}(k), & \quad\tilde{f}_{i}\in k[X_{0},\ldots,X_{n}]
\end{align*}

Verallgemeinerung. Sei $\mathbb{P}^{m}(k)\cong\Lambda\subseteq\mathbb{P}^{n}(k)$
linearer Unterraum, $V\subseteq\mathbb{P}^{n}(k)$ komplementärer
linearer Unterraum, d.h. $\Lambda\cap V=\emptyset$ und $\mathbb{P}^{n}(k)$
ist der \emph{kleinste} lineare Unterraum von $\mathbb{P}^{n}(k)$, der $\Lambda$
und $V$ enthält. Für $X\subseteq V$ eine abgeschlossene Unterprävarietät definiert man den

\textbf{Kegel von $X$ über $\Lambda$} durch $\overline{X,\Lambda} :=\bigcup_{q\in X}\overline{q,\Lambda}$,
wobei der von $q$ und $\Lambda$ aufgespannte lineare Unterraum $\overline{q,\Lambda}$
der kleinste Unterraum sei, der $q$ und $\Lambda$ enthält.


\section{Quadriken}
\label{sec:quadriken}

Sei in diesem Abschnitt char$(k)\neq2$.
\begin{defn}[orig. 57]
  \label{def:quadrik}
  Eine abgeschlossene Unterprävarietät $Q\subseteq\mathbb{P}^{n}(k)$
  von der Form $V_{+}(q)$, $0 \neq q\in k[X_{0},\ldots,X_{n}]_{2}$
  heißt \textbf{Quadrik}.
  \[
    Q=V_{+}(q)
  \]

  Zur quadratischen Form $q$ gehört eine assoziierte Bilinearform $\beta$ auf
  $k^{n+1}$ (vgl. lineare Algebra), 
  \[
    \beta(v,w):=\frac{1}{2}(q(v+w)-q(v)-q(w)),\quad v,w\in k^{n+1}
  \]

Es gibt eine Basis von $k^{n+1}$, sodass die Strukturmatrix $B$
von $\beta$ die Gestalt
\[
  B=\begin{pmatrix}
    \begin{array}{ccc}
      1\\
      & \ddots\\
      &  & 1
    \end{array} & 0\\
    0 &
    \begin{array}{ccc}
      0\\
      & \ddots\\
      &  & 0
    \end{array}
  \end{pmatrix}
\]

hat, d.h. Koordinatenwechsel zur Basiswechselmatrix liefert einen
Isomorphismus
\[
  Q\xrightarrow{\sim}V_{+}(X_{0}^{2}+\cdots+X_{r-1}^{2}),\quad r=\text{rk }B
\]
\end{defn}
\begin{lem}[orig. 58]
\label{lem:irreduzibilitaet-quadriken}
	\begin{enumerate}
	
		\item $X_{0}^{2} + \ldots + X_{r-1}^{2}$ ist irreduzibel $\iff$ $r > 2$
		\item $V_{+}(X_{0}^{2} + \ldots + X_{r-1}^{2})$ ist irreduzibel $\iff$ $r \neq 2$
	\end{enumerate}
\end{lem}
\begin{proof}
	\begin{itemize}
		\item $r=0,1: X_{0}^2 = X_{0} \cdot X_{0} \Rightarrow V_{+}(X_{0}^2) = V_{+}(X_{0})$ irreduzibel
		\item $r=2: X_{0}^{2} + X_{1}^{2} = (X_{0} + i\cdot X_{1})\cdot(X_{0} - i \cdot X_{1})$ für $i = \sqrt{-1}$ 
		\item $r>2: $ Angenommen $\sum_{i}{a_{i} X_{i}} \cdot \sum_{j}{b_{j}X_{j}} = X_{0}^{2} + \ldots X_{r-1}^{2}$.\\
		Ausmultiplizieren $+$ Koeffizientenvergleich $\Rightarrow$ Widerspruch.
	\end{itemize}
\end{proof}

\begin{prop}[orig. 59]
  \label{prop:quadrik-in-normalform}
  Ist $r\neq s$, so sind $V_{+}(T_{0}^{2}+\cdots+T_{r-1}^{2})$ und
  $V_{+}(T_{0}^{2}+\cdots+T_{s-1}^{2})$ nicht isomorph.
\end{prop}
\begin{proof}
  Später: Es gibt keinen Koordinatenwechsel von $\mathbb{P}^{n}(k)$,
  der die beiden Mengen miteinander identifiziert, damit auch kein Automorphismus von
  $\mathbb{P}^{n}(k)$.
\end{proof}

\begin{defn}
  \label{def:dim-und-rang-einer-quadrik}
  Eine Quadrik $Q\subseteq\mathbb{P}^{n}(k)$ mit
  $Q\cong V_{+}(T_{0}^{2}+\cdots+T_{r-1}^{2})$, $r\geq1$, hat \textbf{Dimension $n-1$} und den \textbf{Rang $r$}. (nach Satz
  eindeutig!)
\end{defn}

\begin{cor}[orig. 61]
\label{cor:klassifikation-von-quadriken}
  Zwei Quadriken $Q_{1}$ und $Q_{2}$ sind genau dann isomorph als
  Prävarietäten, wenn sie dieselbe Dimension und denselben Rang haben.
\end{cor}
\begin{proof}
  \mbox{}
  \begin{itemize}
  \item[,,$\Leftarrow$``]
    $Q_{1}\cong V_{+}(T_{0}^{2}+\cdots+T_{n-1}^{2})\cong Q_{2}$ in
    dem selben $\mathbb{P}^{n}$.
  \item[,,$\Rightarrow$``] Für $Q\subseteq\mathbb{P}^{n}(k)$ berechne
    $K(Q)$. Ohne Einschränkung
    $Q=V_{+}(X_{0}^{2}+\cdots+X_{n-1}^{2})$.
    \begin{enumerate}
    \item $r=1$: $V_{+}(X_{0}^{2})=V_{+}(X_{0})=\mathbb{P}^{n-1}(k)$:
      $K(Q)=k(T_{1},\ldots,T_{n-1})$.
    \item $r=2$: reduzibel: Zerlegung in zwei Hyperebenen
      $Z\cong\mathbb{P}^{n-1}$

      $\Rightarrow K(Z)\cong k(T_{1},\ldots,T_{n-1})$.
    \item $r>2$:
      $U=V(1+T_{1}^{2}+\cdots+T_{n-1}^{2})\subseteq\mathbb{A}^{n}(k)$
      ist nichtleere offene affine Teilmenge von $Q$.

      $\Rightarrow K(Q)=K(U)=\text{Quot}(\Gamma(U))=\text{Quot}(k[T_{1},\ldots,T_{n}]/(1+T_{1}^{2}+\cdots+T_{n-1}^{2})$

      $\Rightarrow\text{trgrad}_{k}\ K(Q)=n-1$. 
    \end{enumerate}
  \end{itemize}
\end{proof}
\begin{example}
  \label{bsp:quadrik-joe-harris}
  $Q$ Quadrik in $\mathbb{P}^{n}$ (vgl. Joe Harris, Seite 34).
  \begin{enumerate}
  \item In $\mathbb{P}^{1}(k)$. 
    \begin{itemize}
    \item \emph{Rang 2:} 2 Punkte, reduzibel. 
    \item \emph{Rang 1:} 1 Punkt (Doppelpunkt). 
    \end{itemize}
  \item In $\mathbb{P}^{2}(k)$.
    \begin{itemize}
    \item \emph{Rang 3:} Glatter Kegel
      $\cong\mathbb{P}^{1}(k)$. $X_{0}^{2}+X_{1}^{2}-X_{2}^{2}=0$
    \item \emph{Rang 2:} 2 verschiedene Geraden, reduzibel. 
    \item \emph{Rang 1:} (Doppel)gerade.
    \end{itemize}
  \item In $\mathbb{P}^{3}(k)$.
    \begin{itemize}
    \item Rang 1: Doppelebene (2-dimensionaler linearer Unterraum)
    \item Rang 2: (insert image)
    \item Rang 3: (insert image)
    \item Rang 4: (insert image)
    \end{itemize}
  \end{enumerate}
\end{example}
Die Quadrik $Q\subseteq\mathbb{P}^{n}(k)$ heißt \textbf{glatt}, falls
$r=n+1$, d.h. falls die Matrix $B$ zu $q$ maximalen Rang hat. Für
$\text{rk}(Q)>3$, $\dim(Q)=d$, ist
$Q\cong\overline{\widetilde{Q},\Lambda}$ Kegel über einer
\textbf{glatten} Quadrik $\widetilde{Q}$, da Dimension $r-2$
bzgl. einer $(d-r+2)$-dimensionalen Unterraums 1.
\begin{itemize}
\item $r=1,2$ ausgeartet.
\item $r=1$. $Q=V_{+}(X_{0}^{2})=V_{+}(X_{0})$ Hyperebenen in
  $\mathbb{P}^{n}(k)$.  Der Unterschied zwischen $V_{+}(X_{0}^{2})$
  und $V_{+}(X_{0})$ ist für eine projektive Varietät $Q$ nicht
  sichtbar, jedoch in der Theorie der Schemata unterscheidbar!
\item $r=2$. $Q=V_{+}(X_{0}^{2}+X_{1}^{2})$ reduzibel, d.h. keine
  Prävarietät in unserem Sinne! Auch hier werden uns Schemata später helfen.
\end{itemize}
\medskip{}

$Q=V_{+}(X_{0}^{2}+X_{1}^{2}+\cdots+X_{n-1}^{2})\subseteq\mathbb{P}^{d+1}$,
$r\leq d+2$

$\tilde{Q}=V_{+}(X_{0}^{2}+\cdots+X_{n-1}^{2})\subseteq\mathbb{P}^{r-1}$
glatt.

$A=\mathbb{P}^{d+1-v}=V_{+}(X_{0},\ldots,X_{n-1})\subseteq\mathbb{P}^{d+1}$

$Q=\overline{\widetilde{Q},\Lambda}$


\chapter{Das Ringspektrum}
\label{chap:das-ringspektrum}

Prävarietäten sind Verklebungen. $k$ algebraisch abgeschlossen:

Affine Varietäten $\leftrightarrow$ integere endlich erzeugte
$k$-Algebren.  Punkte $\hat{=}$ maximale Ideale.

\textbf{Ziel}: Schemata sind Verklebungen.

Affine Schemta $\leftrightarrow$ (kommutative) Ringe. Punkte $\hat{=}$
Primideale.

\textbf{Ziel}: Wir wollen einen Funktor:
\begin{align*}
  A & \longmapsto(\Spec(A),\mathcal{O}_{\Spec(A)})\\
  \text{Ring} & \longrightarrow\text{top. Raum}
\end{align*}

,,Garbe von Funktionen`` verallgemeinert ,,Systeme von Funktionen``
für Raume von Funktionen.

(Insbesondere $k$-Algebren über beliebige Körper $k$!)

(Sei $\varphi:A\rightarrow B$ Ringhomomorphismus,
$\mathfrak{m}\subseteq B$ maximales Ideal. Dann folgt i.A. nicht, dass
$\varphi^{-1}(\mathfrak{m})$ maximal ist. Wir haben also zu wenige
maximale Ideale.)

\section*{Das Ringspektrum als topologischer Raum}

\section{Definition von Spec(A)}

Sei $A$ stets ein kommutativer Ring. Spec(A) =
$\{\mathfrak{p}\subseteq A$ Primideal\}. Sei $M\subset A$.
\begin{align*}
  V(M) & =\{\mathfrak{p}\in\Spec(A)\mid\mathfrak{p}\supset
         M\}=V\{\langle M\rangle\}\\
  V(f) & =V(\{f\})\text{\,für }f\in A
\end{align*}

\begin{lem}[1] Es ist
  \begin{align*}
    \{\text{Ideale in }A\} & \longrightarrow\text{\{Teilmengen in }\Spec(A)\}\\
    \mathfrak{A} & \longmapsto V(\mathfrak{A})
  \end{align*}

  ist eine inklusionsumkehrende Abbildung. Es gilt:
  \begin{enumerate}
  \item $V(0)=\Spec(A)$, $V(1)=\emptyset$.
  \item $V\left(\bigcup_{i\in
        I}\mathfrak{a}_{i}\right)=V\left(\sum_{i\in
        I}\mathfrak{a}_{i}\right)=\bigcap_{i\in I}V(\mathfrak{a}_{i})$
  \item
    $V(\mathfrak{a}\cap\mathfrak{a}')=V(\mathfrak{a}\mathfrak{a}')=V(\mathfrak{a})\cup
    V(\mathfrak{a}')$
  \end{enumerate}
\end{lem}
\begin{proof} \mbox{}
  \begin{itemize}
  \item (1), (2) klar.
  \item
    (3). $\mathfrak{p}\supset\mathfrak{a}\cap\mathfrak{a}'\supset\mathfrak{a}\mathfrak{a}'$.

    $\Rightarrow\mathfrak{p}\supset\mathfrak{a}\mathfrak{a}'$.

    $\Rightarrow$ (Primideal) $\mathfrak{p}\supset\mathfrak{a}$ oder
    $\mathfrak{p}\supset\mathfrak{a}'$.

    $\Rightarrow\mathfrak{p}\supset\mathfrak{a}\cap\mathfrak{a}'$

  \end{itemize}
\end{proof}
\begin{defn} Spec(A) mit der Topologie, dessen abgeschlossene Mengen
  gerade die Mengen der Form $V(\mathfrak{a})$,
  $\mathfrak{a}\subset A$ ein Ideal sind, heißt (Prim)Spektrum von $A$
  (mit der Zariski-Topologie).
  \begin{align*}
    x\in\Spec(A) & \leftrightarrow\mathfrak{p}_{x}\subset A\text{ Primideal}\\
    Y\subset\Spec(A), & \phantom{\leftrightarrow\
                              }I(Y):=\bigcap_{\mathfrak{p}\in Y}\mathfrak{p}
  \end{align*}

  $I(-)$ ist inklusionserhaltend, $I(\emptyset)=A$.
\end{defn}
\begin{prop} $\mathfrak{a}\subset A$ Ideal,
  $Y\subset\Spec(A)$. Dann gilt:
  \begin{enumerate}
  \item $\text{rad }I(Y)=I(Y)$, $V(\mathfrak{a})=V(\text{rad
    }\mathfrak{a})$
  \item $I(V(\mathfrak{a}))=\text{rad}(\mathfrak{a})$,
    $V(I(Y))=\overline{Y}$ (Abschluss in $\Spec(A)$).
  \item Wir haben eine 1:1-Korrespondenz:
    \begin{align*}
      \{\mathfrak{a}\subset A\mid\mathfrak{a}=\rad\mathfrak{a}\}
      & \longrightarrow\{\text{abg. Teilmengen }Y\text{ in }\Spec(A)\}
    \end{align*}
  \end{enumerate}
\end{prop}
\begin{proof} \mbox{}
  \begin{enumerate}
  \item $V(\mathfrak{a})=V(\rad\mathfrak{a})$.
    \begin{itemize}
    \item ,,$\supseteq$``. Klar, da rad
      $\mathfrak{a}\supseteq\mathfrak{a}$.
    \item ,,$\subseteq$``. Aus $f^{r}\in\mathfrak{a}\subseteq\mathfrak{a}$
      folgt $f\in\mathfrak{p}$, da $\mathfrak{p}$ Primideal. Damit:
      $\text{rad }\mathfrak{a}\subset\mathfrak{p}.$
    \end{itemize}
  \item $\text{rad }\mathfrak{a}=\bigcap_{\mathfrak{p}\in
      V(\mathfrak{a})}\mathfrak{p}=IV(\mathfrak{a})$.  Es ist:
    \begin{align*}
      V(\mathfrak{b})\supseteq Y & \Leftrightarrow(\forall\mathfrak{p}\in Y:\
                                   \mathfrak{p}\supset\mathfrak{b})\\
                                 & \Leftrightarrow
                                   I(Y)\supseteq\mathfrak{b}.
    \end{align*} Damit ist $V(I(Y))$ die kleinste abgeschlossene
    Teilmenge, die $Y$ umfasst, d.h. $V(I(Y))=\overline{Y}$.
  \item
  \end{enumerate}
\end{proof}


\section{Topologische Eigenschaften von Spec(A)}
\label{sec:topologische-eigenschaften-von-spec-A}

Definiere $D(f):=D_{A}(f):=\Spec(A)\setminus V(f)=\{x \in\Spec A \mid f\notin\mathfrak{p}_{x}\}$,
\begin{align*}
  \text{ev}_{x}:A & \longrightarrow A/\mathfrak{p}_{x}\subseteq \kappa_{x}(A) := \Quot(A/\mathfrak{p}_{x})\\
  f & \longmapsto f(x) := f(\mathfrak{p}_{x}) := f \mod \mathfrak{p}
\end{align*}

Für $x \in D(f)$ gilt dann $f(x) = \text{ev}_{x}(f) \neq 0$.

\textbf{Standard prinzipal offene Mengen}.
\begin{align*}
  D(0) & =\emptyset,\ D(1)=\Spec(A)=D(u),\ u\in A^{\times}\\    
  & D(f)\cap D(g) = D(fg)
\end{align*}

\begin{lem}
\label{lem:charakterisierung-ueberdeckungen-prinzipal}
Für $f_{i} \in A, i\in I$, $g\in A$ gilt:
  \begin{align*}
    D(g)\subseteq\bigcup_{i\in I}D(f_{i})
    & \Leftrightarrow g^{n}\in\mathfrak{a}=(f_{i},i\in I)\text{ für }n \in \mathbb{N} \text{ geeignet}\\
    & \Leftrightarrow g\in\rad(\mathfrak{a})
  \end{align*}
\end{lem}
\begin{proof} Es gilt:
  \begin{align*}
    D(g)\subseteq\bigcup_{i}D(f)
    & \Leftrightarrow V(g)\supseteq\bigcap_{i} V(f_{i})=V(\mathfrak{a})\\
    & \Leftrightarrow g\in\rad((g))\subseteq\rad(\mathfrak{a}) \text{ nach } \ref{prop:nullstellensatz-primspektrum}
  \end{align*}

  Für $g=1$, folgt:
  \[ \Spec(A)=\bigcup_{i\in I}D(f_{i})\Leftrightarrow\sum_{i\in
      I}Af_{i}=A
  \]
\end{proof}
\begin{prop}
\label{prop:prinzipal-offene-bilden-basis}
Die prinzipal offenen Mengen $D(f)$, $f\in A$, bilden
  eine Basis der Topologie von $\Spec(A)$, und sind
  quasikompakt. Insbesondere ist $\Spec(A)$ quasikompakt.
\end{prop}
\begin{proof} Nach Lemma \ref{lem:zariski-top-auf-spektrum}$.(ii)$ gilt:
  \[
    V(\mathfrak{a})=\bigcap_{f \in\mathfrak{a}}V(f)\Longrightarrow\Spec A\setminus
    V(\mathfrak{a})=\bigcup_{f\in\mathfrak{a}}D(f)\Rightarrow\text{Basis
      der Topologie}
  \]

  Sei $D(g)\subseteq\bigcup_{i}D(f_{i})$.

  \ref{lem:charakterisierung-ueberdeckungen-prinzipal} $\Rightarrow$ $g^{n}=\sum_{i\in I}a_{i}f_{i}$, $a_{i}\in A$
  fast alle 0.

  $\Rightarrow D(g)\subseteq\bigcup_{i\in J}D(f_{i})$ $\forall i\in J\subseteq I$ endlich

  $\Rightarrow D(g)$ quasikompakt.
\end{proof}

\begin{prop}
  $X\subseteq\text{Spec}(A)$ ist irreduzibel, genau dann, wenn
  $\varphi:=I(Y)\subset A$ prim ist. In diesem Fall ist
  $\{\mathfrak{p}\}\subset\overline{Y}$ dicht!
\end{prop}

\begin{proof} \mbox{}
  \begin{itemize}
  \item Sei $Y$ irreduzibel und $f,g\in A$ mit $fg\in\mathfrak{p}$.

    $\Rightarrow Y\subset\overline{Y}=VI(Y)\subseteq V(fg)=V(f)\cup V(g)$

    $\Rightarrow$ ($X$ irreduzibel) Ohne Einschränkung: $Y\subset V(f)$.

    $\Rightarrow
    f\in\bigcap_{f\in\mathfrak{q}}\mathfrak{q}=IV(f)\subset
    I(Y)=\mathfrak{p}$

    $\Rightarrow\mathfrak{p}$ Primideal.
  \item Sei umgekehrt $\mathfrak{p}=I(Y)$ ein Primideal.

    $\Rightarrow$ (Satz 3)
    $\overline{Y}=V(\mathfrak{p})=VI(\{\mathfrak{p}\})=\overline{\{\mathfrak{p}\}}$,
    d.h. $\overline{Y}$ ist der Abschluss der irred. Menge
    $\{\mathfrak{p}\}$ und daher selbst irreduzibel.

    $\Rightarrow$ (Lemma I.14): $Y$ ist auch irreduzibel, da dicht in
    $\overline{Y}$.
  \end{itemize}
\end{proof}
Warnung: im Allgemeinen ist $\mathfrak{p}$ nicht in $Y$!

\begin{cor}
  Die Abbildung
  \begin{align*}
    \text{Spec}(A) & \longrightarrow\text{\{abg. irred. Teilmengen von Spec }A\}\\
    \mathfrak{p} & \longmapsto V(\mathfrak{p})=\overline{\{\mathfrak{p}\}}
  \end{align*}
  ist eine Bijektion, unter der die maximalen Primideale von $A$ den
  irreduziblen Komponenten entsprechen.
\end{cor}

\begin{proof} Proposition 3 und 6.
\end{proof}

\begin{defn}
  Für ein topologischer Raum $X$ heißt $\eta\in X$ ein
  \textbf{generischer Punkt}, falls
  $\overline{\{\eta\}}=X$. Allgemeiner sagen wir für $x,x'\in X$, dass
  $x$ eine Verallgeimeinerung (eng. ,,generalization``) von $x'$ ist,
  bzw. $x'$ eine Spezialisierung von $x$, falls
  $x'\in\overline{\{x\}}$.
\end{defn}

\begin{rem}\mbox{}
  \begin{enumerate}
  \item $\eta\in X$ generisch $\Leftrightarrow\eta$ ist
    Verallgemeinerung von jedem Punkt von $X$.
  \item Existiert ein generischer Punkt in $X$, so ist $X$ als
    Abschluss einer irreduziblen Menge selbst irreduzibel.
  \item Für $X=\text{Spec}(A)$ gilt: $x'$ ist eine Spezialisierung von
    $x\Leftrightarrow\mathfrak{p}_{x}\subset\mathfrak{p}_{x'}$
    \begin{align*}
      \Leftrightarrow V(\mathfrak{p}_{x'}) & \subset V(\mathfrak{p}_{x})\\
      \shortparallel & \phantom{\subset\,}\shortparallel\\
      x'\in\overline{\{x')} & \in\overline{\{x\}}
    \end{align*}
    Ferner hat jede abgeschlossene irreduzible Teilmenge
    $Y\subset\text{Spec}(A)$ einen eindeutigen generischen Punkt (dies
    gilt nicht für beliebig irreduzible Teilmengen
    $Y\subset\text{Spec}(A)$).
  \end{enumerate}
\end{rem}


\newpage{}
\printindex{}
\end{document}
