\documentclass[12pt,a4paper]{book}
\usepackage[T1]{fontenc}
\usepackage[utf8]{inputenc}
\usepackage{geometry}
\geometry{verbose,tmargin=2cm,bmargin=2cm,lmargin=2cm,rmargin=2cm}
\pagestyle{headings}
\usepackage[ngerman]{babel}
\usepackage{verbatim}
\usepackage{amsmath}
\usepackage{amsthm}
\usepackage{amssymb}
\usepackage{stmaryrd}
\usepackage{makeidx}
\makeindex
\usepackage{setspace}
\usepackage[all]{xy}
\onehalfspacing
\usepackage[bookmarks=true]{hyperref}

%% Theorems (numbered by Part)
\newtheorem{thm}{Theorem}[chapter]
\theoremstyle{definition}
\newtheorem{example}[thm]{Beispiel}
\theoremstyle{definition}
\newtheorem{notation}[thm]{Notation}
\theoremstyle{definition}
\newtheorem{defn}[thm]{Definition}
\theoremstyle{plain}
\newtheorem{prop}[thm]{Satz}
\theoremstyle{plain}
\newtheorem{cor}[thm]{Korollar}
\theoremstyle{plain}
\newtheorem{lem}[thm]{Lemma}
\theoremstyle{remark}
\newtheorem{rem}[thm]{Bemerkung}
\theoremstyle{plain}
\newtheorem{sprw}[thm]{Sprechweise}
\theoremstyle{plain}

%% Theorems (unnumbered)
\newtheorem*{defn*}{Definition}
\theoremstyle{definition}
\newtheorem*{question*}{Frage}
\theoremstyle{remark}
\newtheorem*{claim*}{Behauptung}
\theoremstyle{definition}
\newtheorem*{notation*}{Notation}
\theoremstyle{definition}
\newtheorem*{example*}{Beispiel}
\theoremstyle{plain}
\newtheorem*{rem*}{Bemerkung}
\theoremstyle{remark}
\newtheorem*{lem*}{Lemma}
\theoremstyle{remark}

%% User-specified commands
\DeclareMathOperator{\rad}{rad}
\DeclareMathOperator{\Spec}{Spec}
\DeclareMathOperator{\maxspec}{MaxSpec}
\DeclareMathOperator{\Quot}{Quot}
\DeclareMathOperator{\im}{\mathrm{im}}
\DeclareMathOperator{\Hom}{\mathrm{Hom}}
\DeclareMathOperator{\Mor}{\mathrm{Mor}}
\DeclareMathOperator{\id}{\mathrm{id}}
\DeclareMathOperator{\res}{res}
\DeclareMathOperator{\Abb}{Abb}
\DeclareMathOperator{\supp}{supp}
\DeclareMathOperator{\red}{red}
\DeclareMathOperator{\op}{op}
\DeclareMathOperator{\obj}{Obj}
\DeclareMathOperator{\nil}{nil}
\DeclareMathOperator{\codim}{codim}

%% sets
\DeclareMathOperator{\CC}{\mathbb{C}}
\DeclareMathOperator{\RR}{\mathbb{R}}
\DeclareMathOperator{\QQ}{\mathbb{Q}}
\DeclareMathOperator{\ZZ}{\mathbb{Z}}
\DeclareMathOperator{\NN}{\mathbb{N}}

%% categories
\DeclareMathOperator{\ouv}{\mathcal{O}uv}
\DeclareMathOperator{\set}{\underline{Set}}
\DeclareMathOperator{\ab}{\underline{Ab}}
\DeclareMathOperator{\cattop}{\underline{Top}}
\DeclareMathOperator{\cring}{\underline{CRing}}
\DeclareMathOperator{\ring}{\underline{Ring}}
\DeclareMathOperator{\psh}{\underline{\mathcal{PS}h}}
\DeclareMathOperator{\sh}{\underline{\mathcal{S}h}}
\DeclareMathOperator{\aff}{\underline{Aff}}
\DeclareMathOperator{\sch}{\underline{Sch}}
\DeclareMathOperator{\schk}{\underline{Sch/k}}
\DeclareMathOperator{\schs}{\underline{Sch/S}}
\DeclareMathOperator{\schsn}{\underline{Sch/S_{0}}}
\DeclareMathOperator{\schz}{\underline{Sch/\mathbb{Z}}}
\DeclareMathOperator{\pres}{\underline{Prevar/S}}
\DeclareMathOperator{\prek}{\underline{Prevar/k}}
\DeclareMathOperator{\affk}{\underline{AffVar/k}}
\DeclareMathOperator{\topcp}{\underline{TopCP}}
\DeclareMathOperator{\Top}{\underline{Top}}
\DeclareMathOperator{\grp}{\underline{Grp}}
\DeclareMathOperator{\func}{\underline{Func}}

%% commands
\newcommand{\cat}[1]{\mathcal{#1}}
\newcommand{\sheaf}[1]{\mathcal{#1}}
\renewcommand{\labelenumi}{(\roman{enumi})}
\renewcommand{\labelenumii}{\arabic{enumii}.}

\begin{document}

%% Title page
\title{Algebraische Geometrie I}
\author{Prof. Dr. Venjakob}
\maketitle

\tableofcontents{}
\newpage{}

\section*{Literatur}
\begin{itemize}
\item Görtz, Wedhorn. \emph{Algebraic Geometry I}
\item Hartshorne. \emph{Algebraic Geometry}
\item Shafarevich. \emph{Basic Algebraic Geometry 1 \& 2}
\item Grothendieck. \emph{Eléments de géometrie algébrique, EGA I-IV}
\end{itemize}

\paragraph{Kommutative Algebra}
\begin{itemize}
\item Brüske, Ischebeck, Vogel. \emph{Kommutative Algebra}
\item Kunz. \emph{Einführung in die kommutative Algebra und algebraische Geometrie}
\end{itemize}

\chapter{Prä-Varietäten}
\label{chap:prae-varietaeten}


\section{Einführung}
\label{sec:einfuehrung}

\textbf{Algebraische Geometrie}\index{Algebraische Geometrie} kann
man verstehen, als das Studium von Systemen polynomialer Gleichungen
(in mehreren Variabelen). Damit ist die algebraische Geometrie eine
Verallgemeinerung der \textbf{linearen Algebra}, also statt $X$ auch
$X^{n}$, und auch der \textbf{Algebra}, durch Polynome in \emph{mehreren
} Variablen.
\begin{question*}
  Seien $k$ ein (algebraisch abgeschlossener) Körper, und $f_{1},\ldots,f_{m}\in k[T_{1},\ldots,T_{n}]$
  gegeben. Was sind die ``geometrischen Eigenschaften'' der Nullstellenmenge
  \[
    V(f_{1},\ldots,f_{n}):=\{(t_{1},\ldots,t_{n})\in k^{n}\mid f_{i}(t_{1},\ldots,t_{n})=0\ \forall i\}
  \]
\end{question*}
\begin{example}
  \label{bsp:einfuehrung}
  Sei $f=T_{2}^{2}-T_{1}^{2}(T_{1}-1)\in k[T_{1},T_{2}]$. Die Nullstellenmenge
  für $k=\mathbb{R}$ (\emph{aber: }trügerisch, da $\mathbb{R}$ nicht
  algebraisch abgeschlossen!) ist gegeben durch:
\end{example}
\begin{figure}
  \caption{$T_{2}^{2}=T_{1}^{2}(T_{1}-1)=T_{1}^{3}-T_{1}^{2}$}
\end{figure}


\begin{itemize}
\item Dimension 1
\item $(0,0)$ ist singulärer Punkt
\item Alle anderen Punkte besitzen eine eindeutig bestimmte Tangente
\end{itemize}


\begin{figure}[h]
  \label{fig:einfuehrung-glattheit}
  \caption{\textbf{Spitze} und \textbf{Doppelpunkt}}
\end{figure}

Vergleiche mit dem \textbf{Satz über implizite Funktionen}: (Analysis,
Differentialgeometrie) 

$V(f)$ ist lokal diffeomorph zu $\mathbb{R}$ (= reelle Gerade) im
Punkt $(x_{1},x_{2})$ genau dann, wenn die Jacobi-Matrix 
\[
  \left(\frac{\partial f}{\partial T_{1}},\frac{\partial f}{\partial T_{2}}\right)=\left(T_{1}(3T_{1}-2),\ 2T_{2}\right)
\]
Rang 1 in $(x_{1},x_{2})$ hat. Das ist äquivalent dazu, dass $(x_{1},x_{2})\neq(0,0)$.
Dies lässt sich rein formal über beliebigen Grundkörpern \textbf{algebraisch}
formulieren.

\paragraph{Methoden.}

GAGA - Géometrie algébrique, géometrique analytique (Serre)\medskip{}

\begin{tabular}{|c|c|}
  \hline 
  Komplexe Geometrie ($\mathbb{C}$), Differentialgeometrie $(\mathbb{R})$ & Algebraische Geometrie\tabularnewline \\
  \hline
  \hline
  Analytische Hilfsmittel & Kommutative Algebra\tabularnewline \\
  \hline 
\end{tabular}


\section{Die Zariski-Topologie}
\label{sec:zariski-topologie}

\begin{defn}
  \label{def:verschwindungsmenge}
  Sei $M\subseteq k[T_{1},\ldots,T_{n}]=:k[\underline{T}]$ eine Teilmenge.
  Mit
  \[
    V(M) :=\{(t_{1},\ldots,t_{n})\in k^n \mid f(t_{1},\ldots,t_{n})=0\ \boldsymbol{\forall f\in M}\}
  \]

  bezeichnen wir die gemeinsame \textbf{Nullstellen-(Verschwindungs-)Menge}\index{Nullstellen-Menge}
  der Elemente aus $M$. (Manchmal auch $V(f_{i},i\in I)$ statt $V(\{f_{i},i\in I\})$.
\end{defn}

%% TODO:
%% das oben weg!!!
\paragraph{Notation}
Wir schreiben auch $V(f_i, i \in I)$ statt $V(\{f_i \mid i \in I\})$

\subsection{Eigenschaften}
\label{subsec:zariski-topologie-eigenschaften}
\begin{itemize}
\item $V(M)=V(\mathfrak{a})$, wenn $\mathfrak{a}=\langle M\rangle_{k[\underline{T}]}$ das
  \emph{von} $M$ \emph{erzeugte Ideal} in $k[\underline{T}]$ bezeichnet.
\item Da $k[\underline{T}]$ noethersch (Hilbertscher Basissatz) ist, reichen
  stets endlich viele $f_{1},\ldots,f_{n}\in M$:
  \[
    V(M)=V(f_{1},\ldots,f_{n})\qquad\text{falls }\mathfrak{a}=\langle f_{1},\ldots,f_{n}\rangle_{k[\underline{T}]}.
  \]
\item $V(-)$ ist \textbf{inklusionsumkehrend}, $M'\subseteq M\implies V(M)\subseteq V(M')$.
\end{itemize}
\begin{prop}
  \label{propdef:zariski-topologie}
  Die Mengen $V(\mathfrak{a})$, $\mathfrak{a} \unlhd k[\underline{T}]$
  ein Ideal, sind die \textbf{abgeschlossenen} Mengen einer Topologie
  auf $k^{n}$, der sogenannten \textbf{Zariski-Topologie}\index{Zariski-Topologie}.
  \begin{enumerate}
  \item $\emptyset=V\left((1)\right)$, $k^{n}=V(0)$. 
  \item $\bigcap_{i\in I}V(\mathfrak{a}_{i})=V\left(\sum_{i\in I}\mathfrak{a}_{i}\right)$
    für beliebige Familien $(\mathfrak{a}_{i})_{i \in I}$ von Idealen.
  \item $V(\mathfrak{a})\cup V(\mathfrak{a})=V(\mathfrak{ab})$ für $\mathfrak{a},\mathfrak{b}\unlhd k[\underline{T}]$
    Ideale.
  \end{enumerate}
\end{prop}
\begin{proof}
  Übung / Algebra II. 

  \-
\end{proof}



\section{Affine algebraische Mengen}
\label{sec:algebraische-mengen}
\begin{defn}
  \label{def:algebraische-mengen}
  \mbox{}
  \begin{itemize}
  \item $\mathbb{A}^{n}(k)$, der $\textbf{affine Raum der Dimension n}$ (über $k$),
    bezeichne $k^{n}$ mit der Zariski-Topologie.
  \item Abgeschlossene Teilmengen von $\mathbb{A}^{n}(k)$ heißen affine abgeschlossene
    Mengen.
  \end{itemize}
\end{defn}
\begin{example}
  \label{bsp:algebraische-mengen-dim1}
  Da $k[T]$ ein Hauptidealring ist, sind die abgeschlossen Mengen in
  $\mathbb{A}^{1}(k)$: $\emptyset$, $\mathbb{A}^{1}$, Mengen der
  Form $V(f)$, $f\in k[T]\backslash\{k\}$ (endliche Teilmengen).%
  \begin{comment}
    Für $f\in k$ ist $V(f)=\mathbb{A}^{1}$, denn die Einheiten im Polynomring
    $k[T]$ sind gegeben durch $k^{\times}$, und Ideale erzeugt von einer
    Einheit bilden den ganzen Ring. (siehe Algebra 1)
  \end{comment}
  {} Insbesondere sieht man, dass die Zariski-Topologie im Allgemeinen
  nicht Hausdorff ist. 
\end{example}
% 
\begin{example}
  \label{bsp:algebraische-mengen-dim2}
  $\mathbb{A}^{2}(k)$ hat zumindestens als abgeschlossene Mengen:
  \begin{itemize}
  \item $\emptyset$, $\mathbb{A}^{2}$;
  \item Einpunktige Mengen: $\{(x_{1},x_{2})\}=V(T_{1}-x_{1},T_{2}-x_{2})$;
  \item $V(f)$, $f\in k[T_{1},T_{2}]$ irreduzibel. 
  \end{itemize}
  Ferner alle endlichen Vereinigungen dieser Liste. (Dies sind in der
  Tat alle, denn später sehen wir: ``irreduzible'' abgeschlossene
  Mengen entsprechen den \emph{Primidealen}, und $k[T_{1},T_{2}]$ hat
  ``Krull-Dimension $2$''.)
\end{example}



\section{Der Hilbertsche Nullstellensatz}
\label{sec:nullstellensatz}
\begin{prop}
  \label{prop:nullstellensatz}
  Sei $K$ ein (nicht notwendigerweise algebraisch abgeschlossener) Körper,
  und $A$ eine endlich erzeugte $K$-Algebra. Dann ist $A$ Jacobson'sch,
  d.h. für jedes Primideal $\mathfrak{p}\unlhd A$ gilt:
  \[
    \mathfrak{p}=\bigcap_{\mathfrak{m}\supseteq\mathfrak{p}}\mathfrak{m},\quad\mathfrak{m}\text{ maximales Ideal}
  \]

  Ist $\mathfrak{m}\unlhd A$ ein maximales Ideal, so ist die Körpererweiterung
  $K\subseteq A/\mathfrak{m}$ endlich.
\end{prop}
\begin{proof}
  Algebra II / kommutative Algebra.
\end{proof}
\begin{cor}
  \label{cor:nullstellensatz}
  \mbox{}
  \begin{enumerate}
  \item Sei $A$ eine e.e. (endlich erzeugte) $k$-Algebra ($k$ sei algebraisch
    abgeschlossen), $\mathfrak{m}\unlhd A$ ein maximales Ideal. Dann
    ist $A/\mathfrak{m}=k$. 
  \item Jedes maximale Ideal $\mathfrak{m}\unlhd k[\underline{T}]$ ist von der
    Form $\mathfrak{m}=(T_{1}-x_{1},\ldots,T_{n}-x_{n})$ mit $x_{1},\ldots,x_{n}\in k$.
  \item Für ein Ideal  $\mathfrak{a}\unlhd k[\underline{T}]$ gilt:
    \[
      \rad(\mathfrak{a})=\sqrt{\mathfrak{a}}\overset{(i)}{=}\bigcap_{\mathfrak{a}\subseteq\mathfrak{p}\unlhd k[\underline{T}], \mathfrak{p} \text{prim}}\mathfrak{p}\overset{(ii)}{=}\bigcap_{\mathfrak{a}\subseteq\mathfrak{m}\unlhd k[\underline{T}], \mathfrak{m} \text{maximal}}\mathfrak{m}
    \]
  \end{enumerate}
\end{cor}
\begin{proof}
  \mbox{}
  \begin{enumerate}
  \item $k\rightarrow A\rightarrow A/\mathfrak{m}$ ist Isomorphismus,  da
    $k$ keine echte algebraische Körpererweiterung besitzt.
  \item Es ist
    \begin{align*}
      k[T_{1},\ldots,T_{n}] & \twoheadrightarrow k[\underline{T}]/\mathfrak{m}=k\\
      T_{i} & \mapsto x_{i}
    \end{align*}
    surjektiv. Es folgt: $\mathfrak{m}=(T_{1}-x_{1},\ldots,T_{n}-x_{n})$, da letzteres
    bereits maximal ist. ($\supseteq$ klar.)
  \item (i) Algebra II. (ii) Theorem.
  \end{enumerate}
\end{proof}



\section{Korrespondenz zwischen Radikalidealen und affinen algebraischen Mengen}
\label{sec:radikalideale-und-algebraische-mengen}

Sei $V(\mathfrak{a})\subseteq\mathbb{A}^{n}(k)$ affin algebraische
Menge, $\mathfrak{a}\unlhd k[\underline{T}]$ ein Ideal.\textbf{ Es gilt:}
\[
  V(\mathfrak{a})=V(\rad\mathfrak{a})
\]

mit $\rad\mathfrak{a}=\{f\in k[\underline{T}]\mid f^{n}\in\mathfrak{a}\text{ für ein }n>0\}$,
da
\[
  f^{n}(x)=0\Leftrightarrow f(x)=0,
\]

d.h. verschiedene Ideale können dieselbe algebraische Menge beschreiben.
\begin{defn}
  \label{def:verschwindungsideal}
  Für eine Teilmenge $Z\subseteq\mathbb{A}^{n}(k)$ bezeichne
  \[
    I(Z):=\{f\in k[\underline{T}]\mid f(x)=0\ \forall x\in Z\}
  \]

  das \textbf{Verschwindungsideal von Z}, das Ideal aller auf $Z$ verschwindenden Polynomfunktionen.
\end{defn}
\begin{prop}
  \label{prop:verschwindungsmenge-verschwindungsideal}
  \mbox{}
  \begin{enumerate}
  \item Sei $\mathfrak{a}\unlhd k[\underline{T}]$ Ideal. Dann ist $I(V(\mathfrak{a}))=\rad(\mathfrak{a})$.
  \item Sei $Z\subseteq\mathbb{A}^{n}(k)$ Teilmenge. Dann ist $V(I(Z))=\overline{Z}$,
    der Abschluss von $Z$ in $\mathbb{A}^{n}(k)$.
  \end{enumerate}
\end{prop}
\begin{proof}
  Übungsblatt 2.
\end{proof}
\medskip{}

$\mathfrak{a}$ heißt \textbf{Radikalideal}\index{Radikalideal},
falls $\mathfrak{a}=\rad(\mathfrak{a})$, oder äquivalent falls $k[\underline{T}]/\mathfrak{a}$
\emph{reduziert} ist, d.h. keine nilpotente Elemente ungleich $0$ hat.
\begin{cor}
  \label{korrespondenz-radikalideal-abgeschlossene-mengen}
  Wir erhalten eine 1-1 Korrespondenz
  \begin{align*}
    \{\text{abg. Mengen }\subseteq\mathbb{A}^{n}\} & \leftrightarrow\{\text{Radikalideale }\mathfrak{a}\unlhd k[\underline{T}]\}\\
    Z & \mapsto I(Z)\\
    V(\mathfrak{a}) & \mapsfrom\mathfrak{a}
  \end{align*}

  die sich zu einer 1-1 Korrespondenz
  \begin{align*}
    \left\{ \text{Punkte in }\mathbb{A}^{n}\right\}
    & \leftrightarrow\left\{ \text{max. Ideale in }k[\underline{T}]\right\} \\
    x=(x_{1},\ldots,x_{n})
    & \mapsto
      \begin{array}{rl}
        \mathfrak{m}_{x} & =I(\{x\})\\
                         & =\ker(k[\underline{T}]\rightarrow k,\ T_{i}\mapsto x_{i})
      \end{array}
  \end{align*}

  einschränkt.
\end{cor}


\section{Irreduzible topologische Räume}
\label{sec:irreduzibilitaet-top}

Die folgenden topologischen Begriffe sind nur interessant, da $\mathbb{A}^{n}(k)$
($n>0$) kein Hausdorff'scher Raum ist.
\begin{defn}
  \label{def:irreduzibel}
  Ein topologischer Raum $X$ heißt \textbf{irreduzibel}\index{irreduzibel},
  falls $X\neq\emptyset$ und $X$ sich \emph{nicht} als Vereinigung
  zweier echter abgeschlossener Teilmengen darstellen lässt, d.h
  \[
    X=A_{1}\cup A_{2},\ A_{i}\ \text{abg.}\quad\implies\quad A_{1}=X\text{ oder }A_{2}=X.
  \]

  $Z\subseteq X$ heißt irreduzibel, falls $Z$ mit der induzierten Topologie
  irreduzibel ist.
\end{defn}
\begin{prop}
  \label{prop:charakterisierung-irreduzibel}
  Für einen topologischen Raum $X \neq \emptyset$ sind äquivalent:
  \begin{enumerate}
  \item $X$ ist irreduzibel.
  \item Je zwei nichtleere offene Teilmengen von $X$ haben nicht-leeren
    Durchschnitt.
  \item Jede nichtleere offene Teilmenge $U\subseteq X$ ist dicht in $X$.
  \item Jede nichtleere offene Teilmenge $U\subseteq X$ ist zusammenhängend.
  \item Jede nichtleere offene Teilmenge $U\subseteq X$ ist irreduzibel.
  \end{enumerate}
\end{prop}
\begin{proof}
  \mbox{}
  \begin{itemize}
  \item $(i)\Leftrightarrow(ii)$

    Komplementärmengen.
  \item $(ii)\Leftrightarrow(iii)$ 

    Es ist: $U\subseteq X$ dicht $\Leftrightarrow U\cap O\neq\emptyset$
    für jedes offene $\emptyset\neq O\subseteq X$.
  \item $(iii)\Rightarrow(iv)$

    Klar. 
  \item $(iv)\Rightarrow(iii)$

    Sei $\emptyset\neq U$ offen und zusammenhängend. Es folgt:
    \[
      U=U_{1}\sqcup U_{2},\qquad\emptyset\neq U_{i}\underset{\text{offen}}{\subseteq}U\underset{\text{offen}}{\subseteq}X
    \]
    Damit ist $U_{1}\cap U_{2}=\emptyset$, ein Widerspruch zu (iii).
  \item $(v)\Rightarrow(i)$ 

    Klar. $(U=X)$
  \item $(iii)\Rightarrow(v)$

    Sei $\emptyset\neq U\underset{\text{offen}}{\subseteq}X$. Ist $\emptyset\neq V\underset{\text{offen}}{\subseteq}U$,
    so ist $V\underset{\text{offen}}{\subseteq}X$. Es folgt: $V$ ist
    dicht in $X$ und irreduzibel in $U$. Mit $(iii)\Rightarrow(i)$
    folgt, dass $U$ irreduzibel ist. 

  \end{itemize}
\end{proof}
\begin{lem}
  \label{lem:irreduzibel-abschluss}
  Eine Teilmenge $Y$ ist genau dann irreduzibel, wenn ihr Abschluss $\overline{Y}$ dies ist.
\end{lem}
\begin{proof}
  $Y$ irreduzibel

  $\Leftrightarrow\forall U,V\subseteq X$ offen mit $U\cap Y\neq\emptyset\neq V\cap Y$,
  gilt $Y\cap(U\cap V)\neq\emptyset$.

  $\Leftrightarrow\overline{Y}$ irreduzibel 
\end{proof}
\begin{defn}
  \label{def:irreduzible-komponente}
  Eine maximale irreduzible Teilmenge eines topologischen Raumes $X$
  heißt \textbf{irreduzible Komponente}\index{irreduzible Komponente}
  von $X$.
\end{defn}
\begin{rem}
  \label{rem:irreduzibel}
  \mbox{}
  \begin{enumerate}
  \item Jede irreduzible Komponente ist abgeschlossen nach Lemma \ref{lem:irreduzibel-abschluss}.
  \item $X$ ist Vereinigung seiner irreduziblen Komponenten, \emph{denn}: 

    die Menge der irreduziblen Teilmengen von $X$ ist \textbf{induktiv
      geordnet}: für jede aufsteigende Kette irreduzibler Teilmengen ist
    die Vereinigung wieder irreduzibel (Satz \ref{prop:charakterisierung-irreduzibel}.(ii)). Mit dem \textbf{Lemma
      von Zorn} folgt: Jede irreduzible Teilmenge ist in einer irreduziblen
    Komponente enthalten. Damit ist jeder Punkt in einer irreduziblen
    Komponente enthalten.
  \end{enumerate}
\end{rem}



\section{Irreduzible affine algebraische Mengen}
\label{sec:irreduzibilitaet-alg}
\begin{lem}
  \label{lem:charakterisierung-irreduzibel-alg}
  Eine abgeschlossene Teilmenge $Z\subseteq\mathbb{A}^{n}(k)$ ist genau
  dann irreduzibel, wenn $I(Z) \unlhd k[\underline{T}]$ ein Primideal ist. Insbesondere ist
  $\mathbb{A}^{n}(k)$ irreduzibel.  
\end{lem}
\begin{proof}
  $Z$ irreduzibel ist äquivalent zu 
  \begin{align*}
    & (Z=\underbrace{V(\mathfrak{a})}_{\bigcap_{i} V(f_{i})}\cup\underbrace{V(\mathfrak{b})}_{\bigcap_{j} V(g_{j})}\quad\Rightarrow\quad V(\mathfrak{a})=Z\text{ oder }V(\mathfrak{b})=Z).\\
    \Leftrightarrow\  & \forall f,g\in k[\underline{T}]:\ V(fg)=V(f)\cup V(g)\supseteq Z:\ V(f)\supseteq Z\text{ oder }V(g)\supseteq Z.\\
    (*)\Leftrightarrow\  & \forall f,g\in k[\underline{T}]:\ fg\in I(V(fg))\subseteq I(Z):\ f\in I(Z)\text{ oder }g\in I(Z).\\
    \Leftrightarrow\  & I(Z)\text{ ist Primideal.}
  \end{align*}

  ({*}): $V(I(Z))=Z$, $I(V(\mathfrak{a}))=\rad(\mathfrak{a})$. 
\end{proof}
\begin{rem}
  \label{rem:korrespondenz-irreduzibel-prim}
  Die Korrespondenz aus Korollar \ref{cor:korrespondenz-radikalideal-abgeschlossene-mengen} schränkt sich ein zu
  \[
    \{\text{irred. abg. Teilmengen }\subseteq\mathbb{A}^{n}\}\overset{1:1}{\leftrightarrow}\{\text{Primideale in }k[\underline{T}]\}
  \]
\end{rem}



\section{Quasikompakte und noethersche topologische Räume}
\label{sec:quasikompakt-noethersch}
\begin{defn}
  \label{def:quasikompakt/noethersch}
  Ein topologischer Raum $X$ heißt \textbf{quasikompakt}\index{quasikompakt},
  falls jede offene Überdeckung von $X$ eine \emph{endliche} Teilüberdeckung
  enthält. (,,quasi`` deutet an, dass $X$ in der Regel nicht Hausdorff'sch
  ist!). Er heißt \textbf{noethersch}\index{noethersch}, wenn jede
  absteigende Kette
  \[
    X\supseteq Z_{1}\supseteq Z_{2}\supseteq\cdots
  \]

  abgeschlossener Teilmengen von $X$ stationär wird ($\Leftrightarrow$
  jede aufsteigende Kette offener Teilmengen wird stationär).
\end{defn}
\begin{lem}
  \label{lem:eigenschaften-noethersch}
  Sei $X$ ein noetherscher topologischer Raum. Dann gilt:
  \begin{enumerate}
  \item Jede abgeschlossene Teilmenge $Z \subseteq X$ ist noethersch.
  \item Jede offene Teilmenge $U \subseteq X$ ist quasikompakt.
  \item Jeder abgeschlossene Teilraum $Z \subseteq X$ besitzt nur endlich viele
    irreduzible Komponenten.
  \end{enumerate}
\end{lem}
\begin{proof}
  \mbox{}
  \begin{enumerate}
  \item Nach Definition, da abgeschlossene Mengen von $Z$ auch solche von
    $X$ sind.
  \item $U=\bigcup_{i\in I}U_{i}$ offen; Angenommen $U$ wäre nicht quasikompakt.
    Dann gibt es eine Folge $I_{1}\subseteq I_{2}\subseteq\cdots\subseteq I$ von Teilmengen
    mit
    \[
      V_{1}\subsetneq V_{2}\subsetneq\cdots\neq U\quad\text{für }V_{j}=\bigcup_{i\in I_{j}}U_{i}.
    \]
    Widerspruch zu noethersch.
  \item Es reicht zu zeigen: Jeder noethersche Raum ist Vereinigung endlich
    vieler irreduzibler Teilmengen. Da $X$ noethersch ist, folgt mit
    dem \emph{Lemma von Zorn} dass jede nichtleere Menge von algebraischen
    Teilmengen in $X$ ein minimales Element besitzt. 
    \[
      \text{Angenommen:} \mathcal{M}:=\left\{ Z\subseteq X\text{ abg.}\mid Z\text{ ist \textbf{nicht} endl. Vereinigung irred. Mengen}\right\} \text{ wäre nichtleer.}
    \]
    $\Rightarrow\exists$ minimales Element, sagen wir $Z$, in $\mathcal{M}$.

    $\Rightarrow Z$ ist nicht irreduzibel.

    $\Rightarrow Z=Z_{1}\cup Z_{2}$ mit $Z_{1},Z_{2}\subsetneq Z$ abgeschlossen.

    $\Rightarrow$ ($Z$ minimal) $Z_{1},Z_{2}\notin\mathcal{M}$

    $\Rightarrow Z\notin\mathcal{M}$. Widerspruch.

  \end{enumerate}
\end{proof}
\begin{prop}
  \label{prop:algebraische-mengen-noethersch}
  Jeder abgeschlossene Teilraum $X\subseteq\mathbb{A}^{n}(k)$ ist noethersch.
\end{prop}
\begin{proof}
  Nach dem obigen Lemma ist nur zu zeigen, dass $\mathbb{A}^{n}(k)$
  noethersch ist.

  Absteigende Ketten abgeschlossener Teilmengen sind nach \emph{Korollar
    11} in 1-1 Korrespondenz mit aufsteigenden Ketten von (Radikal-)Idealen
  in $k[\underline{T}]$. Da $k[\underline{T}]$ nach dem Hilbertschen
  Basissatz noethersch ist, werden letzere Ketten stationär.
\end{proof}
\begin{cor}[Primärzerlegung]
  \label{cor:primaerzerlegung}
  Sei $\mathfrak{a}=\rad(\mathfrak{a})\unlhd k[\underline{T}]$
  ein Radikalideal. Dann gilt: $\mathfrak{a}$ ist Durchschnitt von
  endlich vielen Primidealen, die sich jeweils paarweise nicht enthalten; diese
  Darstellung ist eindeutig bis auf Reihenfolge.
\end{cor}
\begin{proof}
  $V(\mathfrak{a})=\bigcup_{i=1}^{n}V(\mathfrak{b}_{i})$, $\mathfrak{b}_{i}$
  Primideal.%
  \begin{comment}
    Stationäre Kette folgt aus noethersch (Satz 21); mit Bemerkung 16
    bzw. Lemma 17 folgt, dass die $\mathfrak{b}_{i}$ Primideale sind.
  \end{comment}
  {} [Anmerkung] Mit Satz 10 folgt: % color package gives errors with plastex
  \[
    \mathfrak{a}=\rad(\mathfrak{a})=I(V(\mathfrak{a}))=\bigcap_{i=1}^{n}\underbrace{I(V(\mathfrak{b}_{i}))}_{\mathfrak{b}_{i}\text{ minimale Primideale (\ref{lem:charakterisierung-irreduzibel-alg})}}
  \]
\end{proof}



\section{Morphismen von affinen algebraischen Mengen}
\label{sec:morphismen-alg-mengen}
\begin{defn}
  \label{def:morphismus-alg-mengen}
  Seien $X\subseteq\mathbb{A}^{m}(k)$, $Y\subseteq\mathbb{A}^{n}(k)$
  affine algebraische Mengen. Ein \textbf{Morphismus} $X\rightarrow Y$
  affiner algebraischer Mengen ist eine Abbildung $f:X\rightarrow Y$
  der zugrundeliegenden Mengen, sodass $f_{1},\ldots,f_{n}\in k[T_{1},\ldots,T_{m}]$
  existieren, derart dass $\forall x\in X$ gilt:
  \[
    f(x)=(f_{1}(x),\ldots,f_{n}(x)) \in Y.
  \]
  Es bezeichne $\hom(X,Y)$ die Menge der Morphismen $X \to Y$. 
\end{defn}
\begin{rem}
  \label{rem:morphismen-fortsetzbarkeit}
  $f:X\rightarrow Y$ lässt sich immer fortsetzen zu einem Morphismus
  \[
    f:\mathbb{A}^{m}(k)\rightarrow\mathbb{A}^{n}(k),
  \]

  aber nicht eindeutig, es sei denn $X=\mathbb{A}^{m}(k)$.
\end{rem}

\paragraph{Komposition}

\[
  \xymatrix@C=9pc{X\ar[r]^{f}_{f_{1},\ldots,f_{n}\in k[T_{1},\ldots,T_{m}]} & Y\ar[r]^{g}_{g_{1},\ldots,g_{r}\in k[T_{1}',\ldots,T_{m}']} & Z}
\]

mit $X\subseteq\mathbb{A}^{m}(k)$, $Y\subseteq\mathbb{A}^{n}(k)$,
$Z\subseteq\mathbb{A}^{r}(k)$. Es folgt:
\begin{align*}
  g(f(x))=\, & (g_{1}(f_{1}(x),\ldots,f_{n}(x)),\ldots,g_{r}(f_{1}(x),\ldots,f_{n}(x))\\
  =:\, & (h_{1}(x),\ldots,h_{r}(x))
\end{align*}

d.h. $g\circ f$ ist durch Polynome $h_{i}\in k[T_{1},\ldots,T_{m}]$
gegeben, also ist $g\circ f$ wieder ein Morphismus affiner algebraischer
Mengen. Wir erhalten die \textbf{Kategorie affiner algebraischer Mengen}.
\begin{example}
  \label{bsp:morphismen-alg-mengen}
  \mbox{}
  \begin{enumerate}
  \item Sei die Abbildung
    \begin{align*}
      \mathbb{A}^{1}(k) & \rightarrow V(T_{2}-T_{1}^{2})\subseteq\mathbb{A}^{2}(k)\\
      x & \mapsto(x,x^{2}).
    \end{align*}
    Diese Abbildung ist sogar ein \emph{Isomorphismus }affiner algebraischer
    Mengen, da die Umkehrabbildung
    \[
      (x,y)\mapsto x
    \]
    ebenfalls ein Morphismus ist.
  \item Sei char$(k)\neq2$. Die Abbildung
    \begin{align*}
      \mathbb{A}^{1}(k) & \rightarrow V(T_{2}^{2}-T_{1}^{2}(T_{1}+1))\\
      x & \mapsto(x^{2}-1,x(x^{2}-1))
    \end{align*}
    ist ein Morphismus, aber \emph{nicht }bijektiv, da $1,-1$ beide auf
    $(0,0)$ abgebildet werden.
  \end{enumerate}
\end{example}



\section{Unzulänglichkeiten des Begriffs der affinen algebraischen Mengen}
\label{sec:unzulaenglichkeiten-alg-mengen}
\begin{enumerate}
\item Offene Teilmengen affiner algebraischer Mengen tragen nicht in natürlicher
  Weise die Struktur einer affinen algebraischen Menge.
\item Insbesondere können wir affine algebraische Mengen nicht entlang offener
  Teilräume verkleben. (vgl. Mannigfaltigkeiten.)
\item Keine Unterscheidungsmöglichkeiten z.B. zwischen $\{(0,0)\}$,
  $V(T_{1})\cap V(T_{2})$ und
  $V(T_{2})\cap V(T_{1}^{2}-T_{2})\subseteq\mathbb{A}^{2}(k)$, obwohl
  die ``geometrische Situation'' offensichtlich verschieden ist.
\end{enumerate}
Um die Punkte 1 und 2 zu verbessern, gehen wir im Folgenden zu ``Räumen mit Funktionen'' über, und verzichten darauf,
dass sich diese in einen affinen Raum $\mathbb{A}^{n}(k)$ einbetten
lassen.

Der Punkt 3 ist eine Motivation dafür, später Schemata einzuführen.
(subtiler)



\paragraph*{Affine algebraische Mengen als Räume von Funktionen}

\section{Der affine Koordinatenring}
\label{sec:koordinatenring}

Sei $X\subseteq\mathbb{A}^{n}(k)$ abgeschlossen. Für den surjektiven
(Def. von Morphismen) $k$-Algebren-Homomorphismus
\begin{align*}
  k[\underline{T}] & \xrightarrow{\varphi}\hom(X,\mathbb{A}^{1}(k))\\
  f & \mapsto(x\mapsto f(x)),
\end{align*}

wobei die Morphismen in folgende Weise eine $k$-Algebra bilden:
\begin{align*}
  (f+g)(x) & :=f(x)+g(x)\\
  (fg)(x) & :=f(x)g(x)\\
  (\alpha f)(x) & :=\alpha f(x)
\end{align*}

mit $f,g\in\hom(X,\mathbb{A}^{1}(k))$, $\alpha\in k$, gilt:
\[
  \ker\varphi=I(X).
\]
\begin{defn}
  \label{def:koordinatenring}
  $\Gamma(X):=k[\underline{T}]/I(X)\cong_{k-\mathrm{Alg}}\hom(X,\mathbb{A}^{1}(k))$ heißt der \textbf{affine
    Koordinatenring }von $X$.

  Für $x=(x_{1},\ldots,x_{n})\in X$ gilt:
  \begin{align*}
    \mathfrak{m}_{x}:=\  & \ker(\Gamma(X)\twoheadrightarrow k,\,f\mapsto f(x))\\
    =\  & \{f\in\Gamma(X)\mid f(x)=0\}\\
    =\  & \pi((T_{1}-x_{1},\ldots,T_{n}-x_{n}))\\
    =\  & \ker(\Gamma(\mathbb{A}^{n}(k))\twoheadrightarrow k)
  \end{align*}

  unter der Projektion $\pi:k[\underline{T}]=\Gamma(\mathbb{A}^{n}(k))\twoheadrightarrow\Gamma(X)$.
  Es ist $\mathfrak{m}_{x}$ ein maximales Ideal von $\Gamma(X)$ mit
  $\Gamma(X)/\mathfrak{m}_{x}\cong k$. Für ein Ideal $\mathfrak{a}\unlhd\Gamma(X)$
  sei
  \[
    V(\mathfrak{a}):=\{x\in X\mid f(x)=0\ \forall f\in\mathfrak{a}\}=V(\pi^{-1}(\mathfrak{a}))\cap X.
  \]

  Dies sind genau die abgeschlossenen Mengen von $X$ als Teilraum von
  $\mathbb{A}^{n}(k)$ mit der induzierten Topologie, diese wird auch
  \textbf{Zariski-Topologie} genannt. Für $f\in\Gamma(X)$ setze:
  \[
    D_X(f) := D(f):=\{x\in X\mid f(x)\neq0\}=X\setminus V(f).
  \]
\end{defn}
\begin{lem}
  \label{lem:basis-zariski-topologie}
  Die offenen Mengen $D(f)$, $f\in\Gamma(X)$, bilden eine Basis der
  Topologie von $X$, d.h.
  \[
    \forall U\subseteq X\text{ offen }\exists f_{i}\in\Gamma(X),\,i\in I\quad\text{mit }U=\bigcup_{i\in I}D(f_{i})
  \]
\end{lem}
\begin{proof}
  $U=X\backslash V(\mathfrak{a})$ für ein $\mathfrak{a}\unlhd\Gamma(X)$,
  $\mathfrak{a}=\langle f_{1},\ldots,f_{n}\rangle_{\Gamma(X)}$ . Wegen
  \[
    V(\mathfrak{a})=\bigcap_{i=1}^{n}V(f_{i})\quad\Rightarrow\quad U=\bigcup_{i=1}^{n}D(f_{i})
  \]

  Es reichen also sogar endlich viele $f_{i} \in \Gamma(X)$! 
\end{proof}
\begin{prop}
  \label{prop:eigenschaften-koordinatenring}
  Der Koordinatenring $\Gamma(X)$ einer affinen algebraischen Menge
  $X$ ist eine endlich erzeugte $k$-Algebra, die reduziert ist (d.h.
  keine nilpotenten Elemente $\neq0$ enthält). Ferner ist $X$ irreduzibel
  genau dann, wenn $\Gamma(X)$ integer ist.
\end{prop}
\begin{proof}
  $k[\underline{T}]\twoheadrightarrow\Gamma(X)$ impliziert, dass $\Gamma(X)$ als $k$-Algebra endlich
  erzeugte ist. Es gilt:
  \[
    \Gamma(X)\text{ irreduzibel }\Leftrightarrow I(X)=\rad I(X).
  \]

  Denn mit Satz 10.ii) und Korollar 11 folgt:
  \begin{align*}
    X & =V(\mathfrak{a}):\,I(X)=\rad\mathfrak{a}\\
    \Rightarrow\rad I(X) & =\rad\rad\mathfrak{a}=\rad\mathfrak{a}=I(X).
  \end{align*}

  Mit Lemma 17 folgt: $X$ irreduzibel

  $\phantom{\quad}\Leftrightarrow I(X)$ prim

  $\phantom{\quad}\Leftrightarrow\Gamma(X)=k[\underline{T}]/I(X)$ integer.
\end{proof}



\section{Funktorielle Eigenschaften von $\Gamma(X)$}
\label{sec:koordinatenring-funktiorialitaet}
\begin{prop}
  \label{prop:koordinatenringfunktor}
  Für einen Morphismus $X\xrightarrow{f}Y$ affiner algebraischer Mengen
  definiert 
  \begin{align*}
    \Gamma(f):\quad\Gamma(Y) & \rightarrow\Gamma(X)\\
    g & \mapsto g\circ f
  \end{align*}
  ein Homomorphismus von $k$-Algebren. Der so definierte \emph{kontravariante}
  Funktor
  \[
    \Gamma:\{\text{affine algebraische Mengen}\}\rightarrow\{\text{reduzierte endl. erz. }k\text{-Algebren}\}
  \]
  liefert eine Kategorienäquivalenz, welche durch Einschränkung eine Äquivalenz
  \[
    \Gamma:\{\text{irred. aff. alg. Mengem\}}\rightarrow\{\text{integre endl. erz. }k\text{-Algebren\}}
  \]
  induziert.
\end{prop}
\begin{proof}
  Sei $Y\xrightarrow{g}\mathbb{A}^{1}(k)\in\Gamma(Y)$ ein Morphismus. Es
  folgt:
  \[
    g\circ f:X\xrightarrow{f}Y\xrightarrow{g}\mathbb{A}^{1}(k)
  \] 
  ist Morphismus,  d.h. $g \circ f\in\Gamma(X)$. $\Gamma(f):\Gamma(Y)\rightarrow\Gamma(X)$
  ist ein $k$-Algebren-Homomorphismus mit $\Gamma(\text{id}_{X})=\text{id}_{\Gamma(X)}$. Da ferner gilt, dass $\Gamma(f_{1}\circ f_{2})=\Gamma(f_{2})\circ\Gamma(f_{1})$ ist $\Gamma$ ein kontravarianter Funktor.
  \begin{claim*}
    $\Gamma$ ist volltreu, d.h.
    \begin{align*}
      \Gamma:\hom(X,Y) & \rightarrow\hom_{k\text{-Alg}}(\Gamma(Y),\Gamma(X))\\
      f & \mapsto\Gamma(f)
    \end{align*}
    ist \emph{bijektiv} für alle affinen algebraischen Mengen $X,Y$.
  \end{claim*}
  \begin{proof}
    Wir konstruieren eine Umkehrabbildung wie folgt: Zu $\varphi:\Gamma(Y)\rightarrow\Gamma(X)$
    für $X\subseteq\mathbb{A}^{m}(k)$, $Y\subseteq\mathbb{A}^{n}(k)$ existiert ein Lift $\tilde\varphi$, s.d.
    \[
      \xymatrix{k[T_{1}',\ldots,T_{n}']\ar[r]^{\tilde{\varphi}}\ar@{->>}[d] & k[T_{1},\ldots,T_{m}]\ar@{->>}[d]\\
        \Gamma(Y)\ar[r]^{\varphi} & \Gamma(X)
      }
    \]
    kommutiert; $\tilde{\varphi}(T_{i}'):= f_i$ mit $f_i \in \pi^{-1}(\varphi(T_{i}')) \subseteq k[T_1,...,T_n]$, wobei $\pi : k[\underline{T}] \to \Gamma(X)$ die kanonische Projektion bezeichne. 
    Definiere:
    \begin{align*}
      f:X & \rightarrow Y\\
      x=(x_{1},\ldots,x_{n}) & \mapsto(\tilde{\varphi}(T_{1}')(x_{1},\ldots,x_{n}),\ldots,\tilde{\varphi}(T_{n}')(x_{1},\ldots,x_{n}))
    \end{align*}
  \end{proof}
  \begin{claim*}
    $\Gamma$ ist essentiell surjektiv, d.h. zu jeder reduzierten endlich
    erzeugten $k$-Algebra $A$ existiert eine affine algebraische Menge
    $X$ mit $A\cong\Gamma(X)$.
  \end{claim*}
  \begin{proof}
    Da nach Voraussetzung $A\cong k[T]/\mathfrak{a}$ für ein Radikalideal
    $\mathfrak{a}$, können wir etwa $X:=V(\mathfrak{a})\subseteq\mathbb{A}^{n}(k)$
    setzen. Der Rest folgt aus Satz \ref{prop:eigenschaften-koordinatenring}.
  \end{proof}
\end{proof}
\begin{prop}
  \label{prop:funktiorialitaet-specm}
  Sei $f:X\rightarrow Y$ ein Morphismus affiner algebraischer Mengen und $\Gamma(f):\Gamma(Y)\rightarrow\Gamma(X)$
  der zugehörige Homomorphismus der Koordinatenringe. Dann gilt $\forall x\in X$:
  $\Gamma(f)^{-1}(\mathfrak{m}_{x})=\mathfrak{m}_{f(x)}$.
\end{prop}
\begin{proof}
  \[
    \Gamma(f)^{-1}(\mathfrak{m}_{x})=\{g\in\Gamma(Y)\mid g\circ f \in \mathfrak{m}_{x}\}=\{g\in\Gamma(Y)\mid g(f(x)) = 0 \} = \mathfrak{m}_{f(x)},
  \]
  da $\Gamma(f)(g) =g \circ f$.
\end{proof}


\section{Räume mit Funktionen}
\label{sec:raeume-mit-funktionen}
(Prototyp eines geometrischen Objektes, Spezialfall eines ``geringten
Raumes'' vgl. später.) Sei $K$ ein nicht notwendigerweise algebraisch abgeschlossener
Körper.
\begin{defn}
  \label{def:raum-mit-funktionen}
  \mbox{}
  \begin{enumerate}
  \item Ein \textbf{Raum mit Funktionen}$_{/K}$\index{Raum mit Funktionen} besteht
    aus den folgenden Daten:
    \begin{itemize}
    \item ein topologischer Raum $X$;
    \item eine Familie von Unter-$K$-Algebren
      \[
        \mathcal{O}_X(U)\leq\text{Abb}(U,K),\quad\forall U\subseteq X\text{ offen }d.d
      \]

      \begin{enumerate}
      \item Sind $U'\subseteq U\subseteq X$ offen und $f\in\mathcal{O}_X(U)$ so ist
        $f|_{U'}\in \mathcal{O}_X(U')$.
      \item (\textbf{Verklebungsaxiom}\index{Verklebungsaxiom}) Sind $U_{i}\subseteq X$
        offen, $i\in I$, und $U=\bigcup_{i}U_{i}$, $f_{i}\in\mathcal{O}_X(U_{i})$,
        $i\in I$ gegeben mit
        \[
          f_{i}|_{U_{i}\cap U_{j}}=f_{j}|_{U_{i}\cap U_{j}}\quad\forall i,j\in I
        \]
        dann ist die eindeutige Abbildung
        \[
          f:U\rightarrow K\text{ mit }f|_{U_{i}}=f_{i}
        \]
        in $\mathcal{O}_X(U)$, bzw. $\exists!f\in\mathcal{O}(U)$ mit $f|_{U_{i}}=f_{i}$ für alle $i \in I$.
      \end{enumerate}
    \end{itemize}
    Bezeichne $\mathcal{O}_X$ oder auch $\mathcal{O}$ die oben genannte
    Familie $\{\mathcal{O}_X(U) \mid U \subseteq X \text{offen}\}$. Das Tupel $(X,\mathcal{O}_{X})$ heißt $\textbf{Raum mit Funktionen}$.
  \item Ein \textbf{Morphismus}\index{Raum mit Funktionen!Morphismus} $(X,\mathcal{O}_{X})\rightarrow(Y,\mathcal{O}_{Y})$
    von Räumen von Funktionen ist eine stetige Abbildung $\varphi:X\rightarrow Y$,
    so dass für alle $V\subseteq Y$ offen und $f\in\mathcal{O}_{Y}$
    gilt:
    \[
      f\circ \varphi|_{\varphi^{-1}(V)}:\varphi^{-1}(V)\rightarrow K
    \]
    liegt in $\mathcal{O}_{X}(\varphi^{-1}(V))$.
    \[
      \xymatrix{X\ar[r]^{\varphi} & Y\\
        \varphi^{-1}(V)\ar[r]^{\varphi|}\ar[d]_{f\circ \varphi|_{\varphi^{-1}(V)}}\ar@{^{(}->}[u] & V\ar[d]^{f}\ar@{^{(}->}[u]_{\text{offen}}\\
        K\ar@{=}[r] & K
      }
    \]
  \end{enumerate}
\end{defn}
Wir erhalten die Kategorie der $\emph{Räume mit Funktionen über K}$.
\begin{defn}[offene Unterräume von Räumen mit Funktionen]
  \label{def:raeume-mit-fkt-offener-unterraum}
  Für $(X,\mathcal{O}_{X})$ einen Raum mit Funktionen und $U\subseteq X$ offen bezeichne $(U,\mathcal{O}_{X}|_{U})$ den
  Raum mit Funktionen gegeben durch den topologischen Raum $U$ mit
  Funktionen $\mathcal{O}_{X}|_{U}(V):=\mathcal{O}_{X}(V)$ für $V\underset{\text{offen}}{\subseteq}U\subseteq X$.
\end{defn}
\textbf{Ab jetzt} betrachten wir Räume von Funktionen über einem fixierten, algebraisch abgeschlossenen Grundkörper $k$.



\section{Der Raum mit Funktionen zu einer affin-algebraischen Menge}
\label{sec:alg-mengen-raeume-mit-fkt}

\textbf{Ziel.} Wir wollen jeder irreduziblen affin algebraischen Menge $X\subseteq\mathbb{A}^{n}(k)$ einen Raum mit Funktionen $(X,\mathcal{O}_{X})$ zuordnen.
D.h. wir müssen Mengen von Funktionen $\mathcal{O}_{X}(U) \leq \text{Abb}(U,k)$,
$U\subseteq X$ offen, definieren. Diese werden als Teilmengen des Funktionenkörpers
$K(X)$ definiert (dazu $X$ irreduzibel, später bei Schemata fällt
diese Bedingung weg!)
\begin{defn}
  \label{def:funktionenkoerper}
  Für eine irreduzible, affin-algebraische Menge $X$ heißt $K(X):=\text{Quot}(\Gamma(X))$ \textbf{Funktionenkörper} von $X$.

  Elemente $\frac{f}{g}\in K(X)$, $f,g\in\Gamma(X)=\hom(X,\mathbb{A}^{1}(k))$,
  $g\neq0$ lassen sich zumindest als Funktion auf der offenen Menge
  $D(g)\subseteq X$ auffassen, wenn auch i.A. nicht auf ganz
  $X$.
\end{defn}
\begin{lem}
  \label{lem:gleichheit-im-funktionenkoerper}
  Gilt für $\frac{f_{1}}{g_{1}},\frac{f_{2}}{g_{2}}\in K(X)$, $f_{i},g_{i}\in\Gamma(X)$,
  und einer offenen Teilmenge $\emptyset\neq U\subseteq D(g_{1}g_{2})$
  \[
    \frac{f_{1}(x)}{g_{1}(x)}=\frac{f_{2}(x)}{g_{2}(x)}\qquad\forall x\in U,
  \]

  dann folgt $\frac{f_{1}}{g_{1}}=\frac{f_{2}}{g_{2}}$ in $K(X)$.
\end{lem}
\begin{proof}
  Sei ohne Einschränkung der Allgemeinheit $g_{1}=g_{2}=g$. (Sonst
  Erweitern!) 

  $\Rightarrow(f_{1}-f_{2})(x)=0$ $\forall x\in U$.

  $\Rightarrow\emptyset\neq U\subseteq V(f_{1}-f_{2})\subseteq X$ dicht,
  d.h. $V(f_{1}-f_{2})=X$.

  $\phantom{\Rightarrow\ }$$f_{1}-f_{2}\in I()V(f_{1}-f_{2}))=I(X)\equiv(0)$
  in $\Gamma(X)$ 

  $\Rightarrow f_{1}-f_{2}=0$.
\end{proof}
\begin{defn}
  \label{def:alg-menge-als-raum-mit-fkt}
  Sei $X$ eine irreduzible affin-algebraische Menge, $U\subseteq X$
  offen. Für $x \in X$ bezeichne $\Gamma(X)_{\mathfrak{m}_{x}}$ die Lokalisierung von $\Gamma(X)$ an der multiplikativ abgeschlossenen Menge $S := \Gamma(X) \setminus \mathfrak{m}_{x}$.
  \[
    \mathcal{O}_{X}(U):=\bigcap_{x\in U}\Gamma(X)_{\mathfrak{m}_{x}}\subseteq K(X)
  \]

  d.h. für jedes $x\in U$ lässt sich $f\in\mathcal{O}_{X}(U)$ schreiben
  als $\frac{h}{g} \in K(X)$ mit $g(x)\neq0$.
\end{defn}
Für $f\in\Gamma(X)$ bezeichne $\Gamma(X)_{f}$ die Lokalisierung
von $\Gamma(X)$ an der multiplikativ abgeschlossenen Menge
$\{1,f,f^{2},\ldots,f^{n}\ldots\}$. Dann lässt sich
\[
  \Gamma(X)_{\mathfrak{m}_{x}}=\bigcup_{f\in\Gamma(X)\backslash\mathfrak{m}_{x}}\Gamma(X)_{f}\subseteq K(X)
\]

schreiben. ``$\supseteq$'': klar, ``$\subseteq$'': $\frac{g}{f}$
mit $f(x)\neq0$ d.h. $f\notin\mathfrak{m}_{x}$ $\Rightarrow\frac{g}{f}\in\Gamma(X)_{f}$.


\paragraph{Es gilt:}
\begin{enumerate}
\item Für $V\subseteq U\subseteq X$ offen kommutiert das folgende Diagramm:
  \[
    \xymatrix{\mathcal{O}_{X}(V)\ar@{^{(}->}[r] & \text{Abb}(V,k)\\
      \mathcal{O}_{X}(U)\ar@{^{(}->}[r]\ar@{^{(}->}[u] & \text{Abb}(U,k)\ar[u]_{\text{Einschränkungsabb.}}
    }
  \]
  mit $\mathcal{O}_{X}(U) \hookrightarrow \mathcal{O}_{X}(V), f \mapsto f|_V$ nach Definition.
\item $\mathcal{O}_{X}(U)\rightarrow\text{Abb}(U,k)$, $f\mapsto(x\mapsto f(x):=\frac{g(x)}{f(x)}\in k)$
  ist injektiv (Lemma 34) und wohldefiniert (kürzen/erweitern), wobei
  $g,h\in\Gamma(X)$ mit $h\notin\mathfrak{m}_{x}$ mit $f=\frac{g}{h}$
  nach Definition von $\mathcal{O}_{X}(U)$ existiert.
\item \textbf{Verklebungseigenschaft.} Sei $U=\bigcup_{i\in I}U_{i}$. Nach
  Definition ist 
  \begin{align*}
    \mathcal{O}_{X}(U) & =\bigcap_{i}\mathcal{O}_{X}(U_{i})\subseteq K(X)\\
    \ni f:U\rightarrow k & \quad\ni f_{i}:U_{i}\rightarrow k
  \end{align*}
  {[}Diagramm fehlt{]}. $(X,\mathcal{O}_{X})$ ist Raum
  mit Funktionen, \textbf{der zur irreduziblen affin algebraische Menge
    assoziierte Raum von Funktionen}. 
\end{enumerate}
\begin{prop}[orig. 33]
  \label{prop:fkt-auf-basis}
  Für $(X,\mathcal{O}_{X})$ zu $X$ wie oben und $f\in\Gamma(X)$
  gilt:
  \[
    \mathcal{O}_{X}(D(f))=\Gamma(X)_{f},
  \]

  insbesondere $\mathcal{O}_{X}(X)=\Gamma(X)$.
\end{prop}
\begin{proof}
  $\Gamma(X)_f\subseteq \mathcal{O}_{X}(D(f))$ klar, da $f(x)\neq0$ $\forall x\in D(f)$
  bzw. $f\in \Gamma(X)\setminus\mathfrak{m}_{x}$. 

  Sei nun $g$ in $\mathcal{O}_{X}(D(f))$ gegeben, $(*)$
  und $\mathfrak{a}:=\{h\in\Gamma(X)\mid hg\in\Gamma(X)\}\unlhd\Gamma(X)$.

  Dann gilt: $g\in\Gamma(X)_{f}$

  $\Leftrightarrow g=\frac{k}{f^{n}}$ für ein $n$ und $k\in\Gamma(X)$

  $\Leftrightarrow f^{n}\in\mathfrak{a}$ für ein $n$.

  d.h. zu zeigen: $f\in\text{rad}(\mathfrak{a})=I(V(\mathfrak{a}))$ (Hilbertscher
  Nullstellensatz)

  $\Leftrightarrow f(x)=0$ $\forall x\in V(\mathfrak{a})$

  Ist dazu $x\in X$ mit $f(x)\neq0$, also $x\in D(f)$, so
  existieren wegen $g \in \mathcal{O}_{X}(D(f))$ \\ Funktionen $f_{1},f_{2}\in\Gamma(X)$, $f_{2}\notin\mathfrak{m}_{x}$
  mit $g=\frac{f_{1}}{f_{2}}$, also gilt $f_{2}\in\mathfrak{a}$. 

  Da $f_{2}(x)\neq0$ folgt weiter $x\notin V(\mathfrak{a})$.
\end{proof}
\begin{rem}[orig. 34]
  \label{rem:globale-darstellung-von-fkt}
  \mbox{}
  \begin{enumerate}
  \item Im Allgemeinen existieren für $f\in\mathcal{O}_{x}(U)$ \textbf{nicht notwendigerweise}
    $g,h\in\Gamma(X)$ mit $f=\frac{g}{h}$ und $h(x)\neq0$ $\forall x\in U$.
  \item \textbf{Alternative Definition von $\mathcal{O}_{X}$, I.}
    \[
      \mathcal{O}_{X}(D(f)):=\Gamma(X)_{f},\quad\forall f\in\Gamma(X).
    \]
    Da $(D(f))_{f \in \Gamma(X)}$ Basis der Topologie bildet, kann es höchstens einen
    Raum mit Funktionen mit dieser Eigenschaft geben, es bleibt die Existenz
    zu zeigen.
  \item \textbf{Alternative Definition von $\mathcal{O}_{X}$, II.}

    Direkt von einer integeren endlich erzeugten $k$-Algebra $A$ ausgehend
    (die $X$ bis auf Isomorphie festlegt), aber ohne ``Koordinaten''
    zu wählen.
    \begin{align*}
      X & :=\{\mathfrak{m}\unlhd A\mid\ \mathfrak{m} \text{ ist max. Ideal}\}
    \end{align*}
    Die \textbf{abgeschlossenen Mengen} sind gegeben durch:
    \[
      V(\mathfrak{a}):=\{\mathfrak{m} \in X \mid\mathfrak{m}\supseteq\mathfrak{a}\},\quad\mathfrak{a}\unlhd A\text{ Ideal}.
    \]

    $\mathcal{O}_{X}(U):=\bigcap_{\mathfrak{m}\in U}A_{\mathfrak{m}}\subseteq\text{Quot}(A)$
    für $U\subseteq X$ offen (vgl. später Schemata).
  \end{enumerate}
\end{rem}



\section{Funktorialität der Konstruktion}
\label{sec:funktorialitaet-affine-varietaet}
\begin{prop}[orig. 35]
  \label{prop:charakterisierung-morphismen-alg-mengen}
  Sei $f:X\rightarrow Y$ eine stetige Abbildung zwischen irreduziblen
  affin-algebraischen Mengen. Es sind äquivalent:
  \begin{enumerate}
  \item $f$ ist ein Morphismus affin-algebraischer Mengen.
  \item $\forall g\in\Gamma(Y)$ gilt $g\circ f\in\Gamma(X)$.
  \item $f$ ist ein Morphismus von Räumen von Funktionen, d.h. für alle $U\subseteq Y$
    offen und alle $g\in\mathcal{O}_{Y}(U)$ gilt $g\circ f\in\mathcal{O}_{X}(f^{-1}(U))$.
  \end{enumerate}
\end{prop}
\begin{proof}
  \mbox{}
  \begin{itemize}
  \item $(i)\Leftrightarrow(ii)$ 

    Folgt aus Satz $\ref{prop:koordinatenringfunktor}$.
  \item $(iii)\Rightarrow(ii)$ 

    $U:=Y$ und Satz $\ref{prop:fkt-auf-basis}$.
  \item $(ii)\Rightarrow(iii)$

    Betrachte $\Gamma(f):\Gamma(Y)\rightarrow\Gamma(X)$, $h\mapsto h\circ f$.
    Aufgrund des Verklebungsaxioms reicht es, die Bedingung für $U$ von
    der Form $D(g)$ zu zeigen; hier gilt:
    \[
      f^{-1}(D(g))=\{x\in X\mid\underbrace{g(f(x))}_{=\Gamma(f)(g)(x)}\neq0\}=D(g \circ f)
    \]
    Deswegen induziert $\Gamma(f)$:
    \begin{align*}
      h & \longmapsto h\circ f\\
      \mathcal{O}_{Y}(D(g)) & \longrightarrow\mathcal{O}_{X}(D(g\circ f))\\
        & \shortparallel\\
      \Gamma(Y)_{g} & \longrightarrow\Gamma(X)_{g \circ f}\\
      \frac{h}{g^{n}} & \longmapsto\frac{h\circ f}{(g\circ f)^{n}}
    \end{align*}
    mit $h\circ f, g\circ f\in\Gamma(X)$ nach Voraussetzung.

  \end{itemize}
\end{proof}
Insgesamt erhalten wir:
\begin{thm}[orig. 36]
  \label{thm:aequivalenz-alg-mengen-aff-varietaeten}
  Die obige Konstruktion definiert einen volltreuen Funktor
  \[
    \text{\{irreduzible aff. alg. Mengen über }k\}\rightarrow\{\text{Räume mit Funktionen über }k\}.
  \]
\end{thm}



\part*{Prävarietäten}

\textbf{Ziel.} Klasse der affin-algebraischen Mengen, aufgefasst
als Räume mit Funktionen durch Verkleben vergrößern.

$(X,\mathcal{O}_{X})$ heißt \textbf{zusammenhängend}, falls $X$
als topologischer Raum zusammenhängend ist.

\section{Definition von Prävarietäten}
\label{sec:def-praevarietaet}
\begin{defn}[orig. 37]
  \label{def:affine-varietaet}
  Eine \textbf{affine Varietät}\index{affine Varietät} ist ein Raum
  mit Funktionen, der isomorph zu dem Raum mit Funktionen assoziiert zu einer irreduziblen affin-algebraischen Menge ist.
\end{defn}
% 
\begin{defn}[orig. 38]
  \label{def:praevarietaet}
  Eine \textbf{Prävarietät} ist ein zusammenhängender Raum mit Funktionen
  $(X,\mathcal{O}_{X})$, für den eine \emph{endliche }Überdeckung $X=\bigcup_{i=1}^{n}U_{i}$ durch offene Teilmengen $U_i \subseteq X$
  existiert, d.d. $\forall i=1,\ldots,n$ $(U_{i},\mathcal{O}_{X|_{U_{i}}})$
  eine affine Varietät ist.  Insbesondere sind affine Varietäten Prävarietäten!

  Ein \textbf{Morphismus von Prävarietäten} ist ein Morphismus der entsprechenden Räume mit Funktionen.
\end{defn}


Später sehen wir: Varietät = ,,separierte Prävarietät``. Affine
Varietäten sind stets ,,separiert``, daher braucht man nicht von
,,affinen Prävarietäten`` zu reden. Ist $X$ eine affine Varietät,
so schreiben wir oft $\Gamma(X)$ für $\mathcal{O}_{X}(X)$ (vgl. Satz
\ref{prop:fkt-auf-basis}).

Unter einer \textbf{offenen affinen Überdeckung} einer Prävarietät
$X$ verstehen wir eine Famile von offenen affinen Unterräumen mit Funktionen
$U_{i}\subseteq X$, $i\in I$ die affine Varietäten sind, d.d. $X=\bigcup_i U_{i}$.

\section{Vergleich mit differenzierbaren/komplexen Mannigfaltigkeiten}
\label{sec:vergleich-mit-mannigfaltigkeiten}

\paragraph{Differential/Komplexe Geometrie}

Mannigfaltigkeiten werden via Kartenabbildungen mit differenzierbaren/holomorphen
Übergangsabbildungen definiert (hier problematisch, da offene Teile
affiner algebraischer Mengen i.A. keine solche Struktur besitzen.)
Jedoch:
\begin{align*}
  \text{\{differenzierbare Mfgkt.\}} & \longrightarrow\text{\{Räume mit Fkt.}/\mathbb{R}\}\\
  X & \longmapsto(X,\mathcal{O}_{X})\\
                                     & \phantom{\longmapsto}\mathcal{O}_{X}(U):=C^{\infty}(U,\mathbb{R}),\ U\subseteq X\text{ offen}
\end{align*}

ist ein volltreuer Funktor. Daher kann man differenzierbare Mannigfaltigkeiten
auch als diejenigen Räume mit Funktionen über $\mathbb{R}$ definieren,
für die $X$ Hausdorff ist, und so dass eine offene Überdeckung durch
solche Räume mit Funktionen über $\mathbb{R}$ existiert, die in obiger
Weise offene Teilmengen von $\mathbb{R}^{n}$ zugeordnet sind. (Analog
bei komplexen Mannigfaltigkeiten.)


\section{Topologische Eigenschaften von Prävarietäten}
\label{sec:topologische-eigenschaften-von-praevarietaeten}
\begin{lem}
  \label{lem:bijektion-irred-teilraeume}
  Für einen topologischen Raum $X$ und $U\subseteq X$ offen haben
  wir eine Bijektion
  \begin{align*}
    \{Y\subseteq U\text{ irred. abg.}\} & \longleftrightarrow\{Z\subseteq X\text{ irred. abg. mit }Z\cap U\neq\emptyset\}\\
    Y & \longmapsto\overline{Y}\text{ (Abschluss in }X)\\
    Z\cap U & \longmapsfrom Z
  \end{align*}
\end{lem}
\begin{proof}
  Lemma \ref{lem:irreduzibel-abschluss}: $Y\subseteq X$ irreduzibel
  $\Leftrightarrow\overline{Y}\subseteq X$ irreduzibel.

  $Y\subseteq U$ abgeschlossen $\Leftrightarrow\exists A\subseteq X$
  abgeschlossen: $Y=U\cap A$.

  $\Rightarrow Y\subseteq\overline{Y}\subseteq A$ $\Rightarrow Y=U\cap\overline{Y}$

  $Y$ irreduzibel in $U$ $\Rightarrow Y$ irreduzibel in $X$

  $\Rightarrow$ $\overline{Y}$ irreduzibel nach \ref{lem:irreduzibel-abschluss}

  $\Rightarrow Y\mapsto\overline{Y}\mapsto\overline{Y}\cap U=Y$. $\checkmark$

  $\emptyset\neq Z\cap U \subseteq Z$
  damit dicht da $Z$ irreduzibel (Satz \ref{prop:charakterisierung-irreduzibel} ii. und v.)

  Also ist die Abbildung $\leftarrow$ wohldefiniert.

  $\Rightarrow\overline{Z\cap U}=Z$ 
\end{proof}
\begin{prop}
  \label{prop:praevarietaeten-noethersch-irreduzibel}
  Sei $(X,\mathcal{O}_{X})$ eine Prävarietät.

  Dann ist $X$ noethersch (insbesondere quasikompakt) und irreduzibel.
\end{prop}
\begin{proof}
  Sei $X=\bigcup_{i=1}^{n}$ endliche offene aff. Überdeckung und $X\supseteq Z_{1}\supseteq Z_{2}\supseteq\cdots$
  eine absteigende Kette abgeschlossener Teilmengen.

  $\Rightarrow U_{i}\cap Z_{1}\supseteq U_{i}\cap Z_{2}\supseteq\cdots$
  , ist eine absteigende Kette abgeschlossener Teilmengen von $U_{i}$

  $\Rightarrow\forall i$ $\exists n_{i} \in \mathbb{N}$: $U_{i}\cap Z_{n_{i}}=U_{i}\cap Z_{i+m}$ für alle $m \in \mathbb{N}$.
  Setzen wir $n:=\max n_{i}$, so folgt:

  $\forall i=1,\ldots,n$ $\forall m\geq n$: $U_{i}\cap Z_{m}=U_{i}\cap Z_{m+1}$

  $\Rightarrow(Z_{i})_{i}$ wird stationär da $Z_{m}=\bigcup_{i} U_{i}\cap Z_{m}$.

  $X$ ist demnach noethersch.

  $X$ ist weiter irreduzibel:

  Sei $X=X_{1}\cup\cdots\cup X_{n}$ die Zerlegung in irreduzible Komponenten.

  Angenommen es wäre $n\geq2$.

  $\Rightarrow\exists i_{0}\in\{2,\ldots,n\}$: $X_{1}\cap X_{i_{0}}\neq\emptyset$.
  (Andernfalls gilt: $X=X_{1}\sqcup\underbrace{X\backslash X_{1}}_{=X_{2}\cup\cdots\cup X_{n}\text{ abg.}}$, im Widerspruch dazu, dass $X$ zusammenhängend ist.)

  Sei ohne Einschränkung $i_{0}=2$. Sei $x\in X_{1}\cap X_{2}$, $x\in U\subseteq X$ offen, affin (d.h. affine Varietät).

  $U$ irreduzibel $\Rightarrow\overline{U}$ (Abschluss in $X$) $\subseteq X_{j}$
  für ein $j\in\{1,\ldots,n\}$

  \textbf{Jedoch}: Da $x\in X_{i}\cap U\subseteq U$ irreduzibel ist, ist $\underbrace{\overline{X_{i}\cap U}}_{\subseteq\overline{U}\subseteq X_{i}}=X_{i}$,
  $i=1,2$

  $\Rightarrow X_{1},X_{2}\subseteq X_{j}$. Widerspruch zu maximale
  Komponente.
\end{proof}



\section{Offene Untervarietäten}
\label{sec:offene-untervarietaeten}

Offene Teilmengen von affinen Varietäten (und allgemeiner beliebigen
Prävarietäten) sind wieder Prävarietäten. (aber i.A. nicht affin!)
\begin{lem}[orig. 41]
  \label{lem:basisoffene-teilmengen-sind-affin}
  Sei $X$ eine affine Varietät, $f\in\mathcal{O}_{X}(X)$, $D(f)\subseteq X$. Die Lokalisierung
  von $\Gamma(X)=\mathcal{O}_{X}(X)$ an $f$,
  \[
    \Gamma(X)_{f}=\Gamma(X)[T]/(Tf-1)
  \]

  ist eine integre, endlich erzeugte $k$-Algebra. $(Y,\mathcal{O}_{Y})$
  bezeichne die zugehörige affine Varietät. Dann gilt:
  \[
    (D(f),\mathcal{O}_{X}|_{D(f)})\cong(Y,\mathcal{O}_{Y})
  \]

  als Räume mit Funktionen, d.h. $(D(f),\mathcal{O}_{X|_{D(f)}})$ ist
  selbst affine Varietät.
\end{lem}
\begin{proof}
  $\mathcal{O}_{X}(D(f))=\mathcal{O}_{X}(X)_{f}$ muss affiner
  Koordinatenring von $(D(f),\mathcal{O}_{X|_{D(f)}})$
  sein, wenn letzterer Raum von Funktionen affin ist. $X\subseteq\mathbb{A}^{n}(k)$
  korrespondiert zu dem Radikalideal:
  \begin{align*}
    \mathfrak{a} & :=I(X)\unlhd k[T_1, \ldots, T_n]\ \subseteq\ \mathfrak{a}':=(\mathfrak{a},fT_{n+1}-1)\subseteq k[T_{1},\ldots,T_{n+1}]
  \end{align*}

  mit Koordinatenringen:
  \begin{align*}
    \Gamma(X) & =k[T_{1},\ldots,T_{n}]/\mathfrak{a}\\
    \Gamma(Y) & =\Gamma(X)_{f}=(k[T_{1},\ldots,T_{n}]/\mathfrak{a})[T_{n+1}]/(T_{n+1}f-1)\\
              & \cong k[T_{1},\ldots,T_{n+1}]/\mathfrak{a}'
  \end{align*}

  Für $Y=V(\mathfrak{a}')\subseteq\mathbb{A}^{n+1}(k)$ induziert die
  Abbildung
  \[
    \xymatrix{Y\subseteq\mathbb{A}^{n+1}(k)\ar@{-->}[d] & (x_{1},\ldots,x_{n+1})\ar@{|->}[d] & T_{i}\\
      X\subseteq\mathbb{A}^{n}(k) & (x_{1},\ldots,x_{n}) & T_{i}\ar@{|->}[u]
    }
  \]

  eine Bijektion $Y\xrightarrow{j} D_{X}(f)$  mit Umkehrabbildung
  $(x_{0},\ldots,x_{n},\frac{1}{f(x_{0},\ldots,x_{n})})\mapsfrom(x_{0},\ldots,x_{n})$
  \begin{claim*}
    $j$ ist Isomorphismus von Räumen mit Funktionen:
    \begin{enumerate}
    \item $j$ ist \emph{stetig} (als Einschränkung einer stetigen Abbildung) $\checkmark$
    \item $j$ ist \emph{offen}: Für $\frac{g}{f^{n}}\in\Gamma(X)_{f} = \Gamma(Y)$ mit $g\in\Gamma(X)$ gilt
      \begin{align*}
        j\left(D_{Y}\left(\frac{g}{f^{n}}\right)\right) & =j\left(D_{Y}(gf)\right) & f\text{ Einheit}\\
                                                        & =D_{X}(gf)\text{ offen}
      \end{align*}

      $\Rightarrow j$ Homömorphismus.
    \item $j$ induziert $\forall g\in\Gamma(X)$ Isomorphismen:
      \begin{align*}
        \mathcal{O}_{X}(D(fg)) & \longrightarrow\Gamma(Y)_{g}\\
        s & \longmapsto s\circ j
      \end{align*}
      mit $\mathcal{O}_{X}(D(fg))=\Gamma(X)_{fg}=\Gamma(X)_{f})_{g}=\Gamma(Y)_{g}$.
      Mit dem Verklebungsaxiom folgt: $j$ ist Morphismus von Räumen mit Funktionen.
    \end{enumerate}
  \end{claim*}
\end{proof}
\begin{prop}[orig. 42]
  \label{prop:offener-teilraum-praevarietaet}
  Sei $(X,\mathcal{O}_{X})$ Prävarietät, $\emptyset\neq U\subseteq X$
  offen. Dann ist $(U,\mathcal{O}_{X}|_{U})$ eine Prävarietät und $U\hookrightarrow X$
  ist Morphismus von Prävarietäten.
\end{prop}
\begin{proof}
  $X$ ist irreduzibel, also folgt mit Satz \ref{prop:charakterisierung-irreduzibel}, dass $U$ zusammenhängend
  ist. Nach Voraussetzung besitzt $X=\bigcup_{i} X_{i}$ eine affine, offene
  Überdeckung. Es folgt:
  \[
    U=\bigcup_{i}(\underbrace{X_{i}\cap U}_{\text{offen in }X_{i}})=\bigcup_{i,j} D_{X_{i}}(f_{i,j})
  \]

  und $D_{X_{i}}(f_{i,j})$ ist eine affine Varietät nach
  Lemma \ref{lem:basisoffene-teilmengen-sind-affin}. Da $X$ noethersch
  ist, folgt mit Lemma \ref{lem:eigenschaften-noethersch}, dass $U$ quasikompakt ist.

  $\Rightarrow$ Es existiert eine endliche Teilüberdeckung, also ist $U$ Prävarietät. $\checkmark$

  Die kanonische Inklusion $i:U\hookrightarrow X$ ist sicher stetig. Für $f\in\mathcal{O}_{X}(V), V \subseteq X$ offen gilt mit dem Einschränkungsaxiom
  \[
    \mathcal{O}_{X}|_{U}(U\cap V)=\mathcal{O}_{X}(U\cap V)\ni f\circ i=f|_{U\cap V}
  \]

  Also ist $i$ Morphismus von Prävarietäten.
\end{proof}
Die offenen affinen Teilmengen einer Prävarietät $X$ ($\hat{=} U\subseteq X$
offen mit $(U,\mathcal{O}_{X}|_{U})$ affine Varietät) bilden eine
Basis der Topologie von $X$, da $X$ durch offene affine Untervarietäten
überdeckt wird und letzere diese Eigenschaft nach Lemma \ref{lem:basisoffene-teilmengen-sind-affin} haben.


\section{Funktionenkörper einer Prävarietät}
\label{sec:funktionenkorper-praevarietaet}
\begin{defn}[orig. 43]
  \label{def:funktionenkoerper-praevarietaet}
  Für eine Prävarietät $X$ sind die rationalen Funktionenkörper aller
  nicht-leeren affin-offenen Teilmengen in natürlicher Weise zu einander
  isomorph. Diesen Körper $K(X)$ nennen wir den \textbf{rationalen Funktionenkörper}von $X$.
\end{defn}
\begin{proof}
  $\emptyset\neq U$, $V\subseteq X$ affine, offene Untervarietäten. Da
  $X$ irreduzibel ist, gilt nach \emph{Satz \ref{prop:charakterisierung-irreduzibel}}:
  \[
    \emptyset\neq U\cap V\subseteq U\text{ offen}.
  \]

  Nach Definition von $\mathcal{O}_{X}$ ist 
  \[
    \mathcal{O}_{X}(U)\subseteq\mathcal{O}_{X}(U\cap V)\subseteq K(U)=\text{Quot}(\mathcal{O}_{X}(U)).
  \]

  Das impliziert $\text{Quot}(\mathcal{O}_{X}(U\cap V))=K(U)$. Aus
  Symmetriegründen ist aber damit auch bereits $K(V)=\text{Quot}(\mathcal{O}_{X}(U\cap V))$.
\end{proof}
\begin{rem}[orig. 44]
  \label{rem:funktionenkoerper-nicht-funktoriell}
  Bildung des des Funktionenkörpers $K(\cdot)$ ist \textbf{nicht} funktoriell!
  Für $X\rightarrow Y$ Morphismus affiner Varietäten ist die Abbildung
  auf den Koordinatenringen $\Gamma(Y)\rightarrow\Gamma(X)$ i.A. \textbf{nicht}
  injektiv, induziert also keine Abbildung $K(Y)\hookrightarrow K(X)$.

  \emph{Jedoch}: Eine Isomorphie $X\xrightarrow{\sim}Y$ induziert $K(Y)\xrightarrow{\sim}K(X)$.
  Allgemeiner sei $X\xrightarrow{\varphi} Y$ Morphismus mit $\text{im}(\varphi) \subseteq Y$
  offen ($\Rightarrow$ dicht. Später: $X\xrightarrow{\varphi} Y$ \textbf{dominant},
  gdw. $\text{im}(\varphi)\subseteq Y$ dicht) induziert in funktioreller Weise eine
  Abbildung $K(Y)\hookrightarrow K(X)$.
\end{rem}
\begin{prop}[orig. 45]
  \label{prop:charakterisierung-schnitte-praevarietaet}
  Sei $X$ eine Prävarietät, $V\subseteq U\subseteq X$ offen. Dann gilt:

  \begin{enumerate}
  \item $\mathcal{O}_{X}(U)\subseteq K(X)$ ist $k$-Unteralgebra.

  \item $\mathcal{O}_{X}(U)\rightarrow\mathcal{O}_{X}(V)$ ist Inklusion von Teilmengen des Funktionenkörpers $K(X)$.

  \item Insbesondere gilt für $U,V\subseteq X$ offen:
    \[
      \mathcal{O}_{X}(U\cup V)=\mathcal{O}_{X}(U)\cap\mathcal{O}_{X}(V).
    \]
  \end{enumerate}
\end{prop}
\begin{proof}
  \mbox{}
\item[(ii)] Sei $\mathcal{O}_{X}(X)\ni f:X\rightarrow k$. Dann ist $f^{-1}(0)\subseteq X$
  abgeschlossen, da für $W\subseteq X$ affin-offen beliebig gilt, dass
  \[
    f^{-1}(0)\cap W=V(f|_{W}).
  \]
  Dazu macht man sich klar: ,,abgeschlossen`` ist eine lokale Eigenschaft,
  affin-offene $W$ bilden eine Basis der Topologie. 

  $\Rightarrow\mathcal{O}_{X}(U)\hookrightarrow\mathcal{O}_{X}(V)$, $f\mapsto f|_{V}$
  ist injektiv für $\emptyset\neq V\subseteq U\subseteq X$ offen.

  $\Rightarrow V\subseteq f^{-1}(0)$ 

  $\Rightarrow f^{-1}(0)=U$ 

  $\Rightarrow f\equiv0$.
\item[(i)] $U\supseteq W$ affin-offene Untervarietät.
  \[
    \xymatrix{
      \mathcal{O}_{X}(W) \ar@{^(->}[r] & K(W) \text{ } k\text{-Algebren} \\
      \mathcal{O}_{X}(U) \ar@{^(->}[u] \ar@{^(-->}[ru] &
    }
  \]
\item[(iii)] Wir haben folgendes kommutatives Diagramm:
  \[
    \xymatrix{
      {} & \mathcal{O}_{X}(U) \ar@{^(->}[rd] & \\
      \mathcal{O}_X(U\cup V) \ar@{^{(}->}[rd] \ar@{^{(}->}[ru] & {} & \mathcal{O}_X(U\cap V) \\
      {} & \mathcal{O}_X(V) \ar@{^{(}->}[ru] &
    }
  \]
  Nach dem Verklebungsaxiom ist die Sequenz
  \[
    \xymatrix{
      0 \ar[r] & \mathcal{O}_{X}(U\cup V) \ar[r] & \mathcal{O}_{X}(U) \times \mathcal{O}_{X}(V) \ar[r] & \mathcal{O}_{X}(U \cap V) \\
      {} & f \ar@{|->}[r] & (f|_{U}, f|_{V}) & {} \\
      {} & {} & (g,h) \ar@{|->}[r] & g|_{U \cap V} - h|_{U \cap V}
    }
  \]
  exakt.


\end{proof}


\section{Abgeschlossene Unterprävarietäten}
\label{sec:abg-untervarietaeten}

Sei $X$ eine Prävarietät, $Z\subseteq X$ abgeschlossen und irreduzibel.

\textbf{Ziel.} $(Z,\mathcal{O}_{Z}')$ Raum von Funktionen erklären.
Definiere dazu für $U \subseteq Z$ offen:

\[
  \mathcal{O}_{Z}'(U):=\{f\in\text{Abb}(U,k)\mid\forall x\in U\ \exists x\in V\subseteq X\text{ offen},\ g\in\mathcal{O}_{X}(V) \text{ mit } f|_{U\cap V}=g|_{U\cap V}\}
\]

Damit ist $(Z,\mathcal{O}'_{Z})$ Raum von Funktionen (klar!) mit $\mathcal{O}'_{X}=\mathcal{O}_{X}$.
\begin{lem}[orig. 46]
  \label{lem:abg-untervarietaeten-affine-varietaeten}
  Seien $X\subseteq\mathbb{A}^{n}(k)$ eine irreduzible, affin-algebraische Menge
  und $Z\subseteq X$ ein irreduzibler abgeschlossener Teilraum. Dann
  ist $(Z,\mathcal{O}_{Z})=(Z,\mathcal{O}'_{Z})$.

  Bezeichne ab jetzt stets $\mathcal{O}_{Z}$ für $\mathcal{O}_{Z'}$.
\end{lem}
\begin{proof}
  $Z\subseteq X$ ist in beiden Fällen mit der Teilraumtopologie ausgestattet!
  Ferner wissen wir, dass der Morphismus $Z\hookrightarrow X$ affin-algebraischer Mengen einen Morphismus $(Z,\mathcal{O}_{Z})\rightarrow(X,\mathcal{O}_{X})$
  von Prävarietäten induziert. Nach Definition von $\mathcal{O}'$
  folgt dann:
  \[
    \mathcal{O}'_{Z}(U)\subseteq\mathcal{O}_{Z}(U)\quad\text{für }U\subseteq Z\ \text{offen, denn:}
  \] 
  Ist $f \in \mathcal{O}_{Z}'(U)$ und $x \in U$ so existieren nach Definition eine offene Umgebung $x \in V_{x} \subseteq X$ und ein $g \in \mathcal{O}_{X}(V_{x})$ d.d. $f|_{U \cap V_{x}} = g|_{U \cap V_{x}}$. Damit gilt $g|_{Z \cap V_x} \in \mathcal{O}_{Z}(Z \cap V_{x})$. Mit dem Verklebungsaxiom erhalten wir also $f \in \mathcal{O}_{Z}(U)$.


  Sei $f\in\mathcal{O}_{Z}(U)$ und $x\in U$ beliebig. Es folgt: $\exists h\in\Gamma(Z)$
  mit $x\in D(h)\subseteq U$ und
  \[
    f|_{D(h)}=\frac{g}{h^{n}}\in\Gamma(Z)_{h}=\mathcal{O}_{Z}(D(h))
  \]

  für $n\geq0$ und $g\in\Gamma(Z)$ geeignet. Lifte $g,h\in\Gamma(Z)\twoheadleftarrow\Gamma(X)$
  zu $\overline{g},\overline{h}\in\Gamma(X)$ und setze $V:=D(\overline{h})\subseteq X$.

  $\Rightarrow x\in V$, $\frac{\overline{g}}{\overline{h}^{n}}\in\mathcal{O}_{X}(D(\overline{h}))$
  und $f|_{U\cap V}=\frac{\overline{g}}{\overline{h}^{n}}|_{U\cap V}$.

  $\Rightarrow f\in\mathcal{O}'_{Z}(U)$.
\end{proof}
\begin{cor}[orig. 47]
  \label{cor:abg-untervarietaeten-sind-praevarietaeten}
  Wenn $X$ eine Prävarietät ist, und $Z\subseteq X$ irreduzibel und abgeschlossen, dann ist $(Z,\mathcal{O}_{Z})$ ebenfalls eine Prävarietät.
\end{cor}
\begin{proof}
  Es ist $X=\bigcup_{i}X_{i}$ für eine endliche affin-offene Überdeckung $(X_{i})_{i}$.
  Damit ist 
  \[
    Z=\bigcup_{i}\left(Z\cap X_{i}\right) :=\bigcup_{i} Z_{i}
  \]
  

  mit $(Z_{i}, \mathcal{O}_{Z_{i}})$ affine Varietät nach Lemma $\ref{lem:abg-untervarietaeten-affine-varietaeten}$.
\end{proof}



\section*{Beispiele (Projektiver Raum und projektive Varietäten)}

\section{Homogene Polynome}
\label{sec:homogene-polynome}
\begin{defn}[orig. 48]
  \label{def:homogen}
  Ein Polynom $f\in k[X_{0},\ldots,X_{n}]$ heißt \textbf{homogen vom
    Grad}\index{homogen} $d\in\mathbb{Z}_{\geq0}$, falls $f$ die Summe
  von Monomen von Grad $d$ ist. (Insbesondere ist für jedes $d$ das
  Nullpolynom homogen von Grad $d$.)

  \emph{Es bezeichne} $k[X_{0},\ldots,X_{n}]_{d}$ den $k$-Untervektorraum
  der Polynome \textbf{homogen vom Grad $d$}, $k[X_{0}, \ldots, X_{n}]_{\leq n}$ den $k$-Untervektorraum \textbf{aller Polynome vom Grad $\leq n$}.
\end{defn}
\begin{rem}[orig. 49]
  \label{rem:charakterisierung-homogen}
  Da \#$k$ unendlich ist, ist $f$ homogen vom Grad $d$
  $\Leftrightarrow f(\lambda x_{0},\ldots,\lambda x_{n})=\lambda^{d}f(x_{0},\ldots,x_{n})$
  $\forall x_{0},\ldots,x_{n}\in k$, $\lambda\in k^{\times}$. 

  Es gilt: $k[X_{0},\ldots X_{n}]=\bigoplus_{d\geq0}k[X_{0},\ldots,X_{n}]_{d}$.
\end{rem}
\begin{lem}[orig. 50]
  \label{lem:homogenisierung-dehomogenisierung}
  Für $i\in\{0,\ldots,n\}$ und $d\geq0$ haben wir bijektive $k$-lineare
  Abbildungen
  \begin{align*}
    k[X_{0},\ldots,X_{n}]_{d} & \longrightarrow k[T_{0},\ldots,\hat{T}_{i},\ldots,T_{n}]_{\leq d}\\
    f & \overset{\Phi_{i}^{d}}{\longmapsto}f(T_{0},\ldots,\underbrace{1}_{i},\ldots,T_{n})\\
    X_{i}^{d}g\left(\frac{X_{0}}{X_{i}},\ldots,\hat{\frac{X_{i}}{X_{i}}},\ldots,\frac{X_{n}}{X_{i}}\right) & \overset{\Psi_{i}^{d}}{\longmapsfrom}g
  \end{align*}

  \textbf{Dehomogenisierung }bzw. \textbf{Homogenisierung.}
\end{lem}
\begin{proof}
  Es reicht, $\Psi_{i}^{d}\circ\Phi_{i}^{d}=\text{id}$, $\Phi_{i}^{d}\circ\Psi_{i}^{d}=\text{id}$
  auf Monomen nachzurechnen, da alle Abbildungen $k$-linear sind. 
\end{proof}
Oft ist es nützlich, 
$k[T_{0},\ldots,\hat{T_{i}},\ldots,T_{n}]$ mit $k\left[\frac{X_{0}}{X_{i}},\ldots,\hat{\frac{X_{i}}{X_{i}}},\ldots,\frac{X_{n}}{X_{i}}\right] \hookrightarrow k(X_{0},\ldots,X_{n})$ zu identifizieren.



\section{Definition des projektiven Raumes}
\label{sec:def-projektiver-raum}

Seien $X_{1}=X_{2}=\mathbb{A}^{1}$, $\tilde{U}_{1}\subseteq X_{1}, \tilde{U}_{2}\subseteq X_{2}$ mit $\tilde{U}_{1} = \tilde{U}_{2} = \mathbb{A}^{1}\setminus\{0\}$.
\begin{align*}
  \tilde{U}_{1} & \overset{\sim}{\longrightarrow}\tilde{U}_{2}\\
  x & \longmapsto\frac{1}{x}
\end{align*}

Verkleben von $X_{1}$ und $X_{2}$ entlang $\tilde{U}_{1} \overset{\sim}\longrightarrow \tilde{U}_{2}$ liefert die \textbf{projektive Gerade}

\[
  \mathbb{P}^{1}=\mathbb{A}^{1}\cup\{\infty\}=U_{1}\cup U_{2}.
\]

Allgemein: 
\[
  \mathbb{P}^{n}=\bigcup_{i=1}^{n+1}U_{i}=\mathbb{A}^{n}\cup\mathbb{P}^{n-1}=\mathbb{A}^{n}\sqcup\mathbb{A}^{n-1}\sqcup\cdots\sqcup\mathbb{A}^{1}\sqcup\mathbb{A}^{0}
\]

\textbf{Idee}: $\mathbb{P}^{2}\supseteq\mathbb{A}^{2}$: Zwei verschiedene
Geraden in $\mathbb{P}^{2}$ schneiden sich genau in einem Punkt.

\textbf{Als Menge}:
\begin{align*}
  \mathbb{P}^{n}(k): & =\{\text{Ursprungsgeraden in }k^{n+1}\}=\{1\text{-dim. }k\text{-Unterräume}\}\\
                     & =(k^{n+1}\backslash\{0\})/k^{\times}
\end{align*}

Man schreibt meist kurz $(x_{0}:\ldots:x_{n})$ für den Repräsentanten der Klasse von $\langle(x_{0},\ldots x_{n})\rangle_{k}$ und nennt $(x_{0}:\ldots:x_{n})$ \textbf{homogene Koordinaten} auf $\mathbb{P}^{n}$.


\emph{Äquivalenzrelation}: 
\[
  (x_{0},\ldots,x_{n})\sim(x_{0}',\ldots,x_{n}')\Leftrightarrow\exists\lambda\in k^{\times}\ \text{mit}\ x_{i}=\lambda x_{i}'\ \forall i.
\]
Die Mengen
\[
  U_{i}:=\{(x_{0}:\ldots:x_{n})\in\mathbb{P}^{n}\mid x_{i}\neq0\}\subseteq\mathbb{P}^{n}(k),\ 0\leq i\leq n
\]

sind wohldefiniert und überdecken $\mathbb{P}^{n}(k)$:
\[
  \mathbb{P}^{n}(k)=\bigcup_{i=0}^{n}U_{i}
\]

Weiter hat man eine Bijektion
\begin{align*}
  U_{i} & \stackrel[1:1]{\kappa_{i}}{\longrightarrow}\mathbb{A}^{n}(k)\\
  (x_{0}:\ldots:x_{n}) & \longmapsto\left(\frac{x_{0}}{x_{i}},\ldots,\frac{\hat{x}_{i}}{x_{i}},\ldots,\frac{x_{n}}{x_{i}}\right)\\
  (t_{0}:\cdots t_{i-1}:1:t_{i+1}:\cdots t_{n}) & \longmapsfrom(t_{0},\ldots,\hat{t}_{i},\ldots,t_{n})
\end{align*}

Über die $\kappa_{i}$ definiert man nun eine Topologie auf $\mathbb{P}^{n}(k)$ durch:

$U\subseteq\mathbb{P}^{n}(k)$ ist genau dann offen, wenn $\kappa_{i}(U\cap U_{i})\subseteq\mathbb{A}^{n}(k)$
offen ist für alle $i$. 

Es gilt: 
\[
  U_{i}\cap U_{j}= D(T_{j})\subseteq U_{i}\text{ offen},\ i\neq j
\]

wenn auf $U_{i}\cong\mathbb{A}^{n}$ die Koordinaten $T_{0},\ldots,\hat{T}_{i},\ldots,T_{n}$
verwendet werden. Damit wird $\mathbb{P}^{n}(k)$ zu einem topologischen
Raum, der durch die $U_{i}$, $0\leq i\leq n$, offen überdeckt wird.

\subsection{Reguläre Funktionen}
\label{subsec:regulaere-fkt-auf-projektivem-raum}

Sei $U\subseteq\mathbb{P}^{n}(k)$ eine beliebige offene Teilmenge.
Die regularären Funktionen auf $U$ sind definiert als
\[
  \mathcal{O}_{\mathbb{P}^{n}}(U) :=\{f\in\text{Abb}(U,k)\mid f|_{U\cap U_{i}}\in\mathcal{O}_{U_{i}}(U\cap U_{i})\}\qquad\forall i\in\{0,\ldots,n\}
\]

wobei wir die $U_{i}$ via $\kappa_{i}$ implizit als Raum mit Funktionen auffassen. Insgesamt erhalten wir:
\[
  \mathbb{P}^{n}(k)=(\mathbb{P}^{n}(k),\mathcal{O}_{\mathbb{P}^{n}})
\]

als Raum mit Funktionen.
\begin{prop}[orig 51]
  \label{prop:charakterisierung-reg-fkt-projektiver-raum}
  Für $U\subseteq\mathbb{P}^{n}$ offen gilt: $\mathcal{O}_{\mathbb{P}^{n}}(U)=\{f:U\rightarrow k\mid\forall x\in U$:
  $\exists x\in V\subseteq U$ offen, $d\geq 0$ und $g,h\in k[X_{0},\ldots,X_{n}]_{d}$
  homogen vom selben Grad $d$, d.d. $\forall v\in V$: $h(v)\neq0$ und
  $f(v)=\frac{g(v)}{h(v)}\}$ 
\end{prop}
Wohldefiniertheit: Sei $v=(x_{0}:\ldots:x_{n})$.
\[
  f(\lambda x_{0},\ldots,\lambda x_{n})=\frac{g(\lambda x_{0},\ldots,\lambda x_{n})}{h(\lambda x_{0},\ldots,\lambda x_{n})}=\frac{\lambda^{d}g(x_{0},\ldots,x_{n})}{\lambda^{d}h(x_{0},\ldots,x_{n})}=f(x_{0},\ldots,x_{n})
\]

\begin{proof}
  \mbox{}
  \begin{itemize}
  \item[,,$\subseteq$``:] Sei $f\in\mathcal{O}_{\mathbb{P}^{n}}(U)$. Dann ist $f|_{U\cap U_{i}}\in\mathcal{O}_{U_{i}}(U\cap U_{i})$.
    Es folgt:
    \[
      f=\frac{\tilde{g}}{\tilde{h}},\ \tilde{g},\tilde{h}\in k[T_{0},\ldots,\hat{T}_{i},\ldots,T_{n}]
    \]
    Definiere $d:=\max\{\deg(\tilde{g}),\deg(\tilde{h})\}$. Homogenisiere:
    \[
      g:=\psi_{i}^{d}(\tilde{g}),\ h:=\psi_{i}^{d}(\tilde{h})
    \]
    $\Rightarrow f=\frac{g}{h}$ lokal. 
    \begin{align*}
      f(x) & =\frac{\tilde{g}}{\tilde{h}}(\kappa_{i}(x))\\
      f((x_{0}:\cdots:x_{n})) & =\frac{\tilde{g}\left(\frac{x_{0}}{x_{i}},\ldots,\frac{\hat{x_{i}}}{x_{i}},\ldots,\frac{x_{n}}{x_{i}}\right)}{\tilde{h}\left(\frac{x_{0}}{x_{i}},\ldots,\frac{\hat{x_{i}}}{x_{i}},\ldots,\frac{x_{n}}{x_{i}}\right)}\\
           & =\frac{x_{i}^{d}\tilde{g}(\ldots)}{x_{i}^{d}\tilde{h}(\ldots)}\\
           & =\frac{\psi_{i}^{d}(\tilde{g})(\ldots)}{\psi_{i}^{d}(\tilde{h})(\ldots)}=\frac{g}{h}(x_{0}:\ldots:x_{n})
    \end{align*}
  \item[,,$\supseteq$``:] Sei $f$ in der rechten Menge, fixiere $i\in\{0,\ldots,n\}$. Nach Voraussetzung ist $f$ lokal
    auf $U\cap U_{i}$ von der Form $f = \frac{g}{h}$, $g,h\in k[X_{0},\ldots,X_{n}]_{d}$, $d\geq 0$ geeignet. Definiere:
    \[
      \tilde{g}_{i}:=\frac{g}{X_{i}^{d}},\ \tilde{h}:=\frac{h}{X_{i}^{d}}\in k\left[\frac{X_{0}}{X_{i}},\ldots\hat{\frac{X_{i}}{X_{i}}},\ldots,\frac{X_{n}}{X_{i}}\right]
    \]
    $\Rightarrow f$ ist lokal von der Form: $\frac{\tilde{g}}{\tilde{h}}$,
    $\tilde{g},$$\tilde{h}\in k[T_{0},\ldots,\hat{T_{i}},\ldots T_{n}]$.

    $\Rightarrow f|_{U\cap U_{i}}\in\mathcal{O}_{U_{i}}(U\cap U_{i})$, also $f \in \mathcal{O}_{\mathbb{P}^{n}}(U)$.

  \end{itemize}
\end{proof}
\begin{cor}[orig. 52]
  \label{cor:affine-ueberdeckung-des-projektiven-raumes}
  Für $i\in\{0,\ldots,n\}$ induziert
  \[
    U\xrightarrow[\cong]{\kappa_{i}}\mathbb{A}^{n}(k)
  \]

  einen Isomorphismus
  \[
    (U_{i},\mathcal{O}_{\mathbb{P}^{n}|_{U_{i}}})\xrightarrow{\cong}\mathbb{A}^{n}(k)
  \]

  von Räumen mit Funktionen. Insbesondere ist $\mathbb{P}^{n}(k)$ eine
  Prävarietät.
\end{cor}
\begin{proof}
  Zu zeigen: $\forall U\subseteq U_{i}$ offen gilt
  \[
    \mathcal{O}_{\mathbb{P}^{n}(k)}(U)=\mathcal{O}_{U_{i}}(U)=\{f:U\rightarrow k\mid f\in\mathcal{O}_{U_{i}}(U)\}
  \]

  d.h. auf der rechten Seite muss die Bedingung nur für das fixierte
  $i$ überprüft werden. Dies folgt aus dem Beweis von Satz \ref{prop:charakterisierung-reg-fkt-projektiver-raum}.
\end{proof}
Damit identifizieren sich die Funktionenkörper 
\[
  K(\mathbb{P}^{n}(k))=K(U_{i})=k\left(\frac{X_{0}}{X_{i}},\ldots,\frac{X_{n}}{X_{i}}\right)
\]

\begin{prop}[orig. 53]
  \label{prop:globale-schnitte-des-proj-raums}
  $\mathcal{O}_{\mathbb{P}^{n}(k)}(\mathbb{P}^{n}(k))=k$. Insbesondere
  ist $\mathbb{P}^{n}$ für $n\geq1$ \textbf{keine} affine Varietät.
  (Da der $k$-Algebra $A = k$ ja $\mathbb{A}^{0}(k)=\{\text{pt}\}$ als
  affine Varietät entspricht.)
\end{prop}
\begin{proof}
  $k\subseteq\mathcal{O}_{\mathbb{P}^{n}(k)}(\mathbb{P}^{n}(k))$ klar, da konstante Funktionen. Nach Satz \ref{prop:charakterisierung-schnitte-praevarietaet} $(iii)$ gilt:
  \begin{align*}
    \mathcal{O}_{\mathbb{P}^{n}}(\mathbb{P}^{n}) & =\bigcap_{i=0}^{n}\mathcal{O}_{\mathbb{P}^{n}}(U_{i})\subseteq K(\mathbb{P}^{n}(k))\\
                                                 & =\bigcap_{i=0}^{n}k[t_{0},\ldots,\hat{t_{i}},\ldots,t_{n}]=k
  \end{align*}
\end{proof}



\section{Projektive Varietäten}
\label{sec:projektive-varietaeten}
\begin{defn}[orig. 54]
  \label{def:projektive-varietaeten}
  Abgeschlossene Unterprävarietäten eines projektiven Raumes $\mathbb{P}^{n}(k)$
  heißen \textbf{projektive Varietäten}.
\end{defn}
Vorsicht: für $x=(x_{0}:\ldots:x_{n})\in\mathbb{P}^{n}$, $f\in k[X_{0},\ldots,X_{n}]$
ist $f(x_{1},\ldots,x_{n})$ \emph{nicht} wohldefiniert, da von Repräsentaten
abhängig, d.h. $f$ kann \emph{nicht}\textbf{ }als Funktion auf $\mathbb{P}^{n}$
aufgefasst werden. Für \emph{homogene}\textbf{ }Polynome $f_{1},\ldots,f_{n}\in k[X_{0},\ldots X_{n}]$
(nicht notwendig vom selben Grad) können wir demnoch Verschwindungsmengen
definieren:
\[
  V_{+}(f_{1},\ldots,f_{n})=\{(x_{0}:\ldots:x_{n})\in\mathbb{P}^{n}\mid f_{j}(x_{0},\ldots,x_{n})=0\ \forall j\}
\]

Da $V_{+}(f_{1},\ldots,f_{n})\cap U_{i}=V(\Phi_{i}(f_{1}),\ldots,\Phi_{i}(f_{m}))$
ist $V_{+}(f_{1},\ldots,f_{m})$ abgeschlossen in $\mathbb{P}^{n}$.
Ist $V_{+}(f_{1},\ldots,f_{n})$ irreduzibel, so erhalten wir eine
projektive Varietät. In der Tat entstehen alle projektiven Varietäten
auf diese Weise, wie der folgende Satz zeigt:
\begin{prop}[orig. 55]
  \label{prop:charakterisierung-projektive-varietaeten}
  Sei $Z\subseteq\mathbb{P}^{n}(k)$ eine projektive Varietät. Dann
  existieren homogene Polynome $f_{1},\ldots,f_{n}\in k[X_{0},\ldots,X_{n}]$,
  so dass
  \[
    Z=V_{+}(f_{1},\ldots,f_{n})
  \]

  gilt.
\end{prop}
\begin{proof}
  Betrachte: 
  \[
    \begin{array}{cc}
      \\
      \\
    \end{array}
  \]

  $f|_{f^{-1}(U_{i})}:f^{-1}(U_{i})\longrightarrow U_{i}$ ist Morphismus
  von Prävarietäten. Dann ist $f$ selber ein Morphismus von Prävarietäten.
  \begin{align*}
    \overline{Y}:= & Y\cup\{0\}\text{ Abschluss von }Y\text{ in }\mathbb{A}^{n+1}(k)\\
    \mathfrak{A}:= & I(\overline{Y})\subseteq k[X_{0},\ldots,X_{n}]
  \end{align*}

  Behauptung: $\mathfrak{A}$ wird von homogenen Polynomen erzeugt\emph{.
    Denn:} für $g\in\mathfrak{A}$, $g=\sum_{d}g_{d}$ Zerlegung in homogene
  Bestandteile vom Grad $d$. $\overline{Y}$ ist Vereinigung von Ursprungsgeraden
  im $k^{n+1}$, d.h. $\forall\lambda\in k^{\times}$ gilt:
  \[
    g(x_{0},\ldots,x_{n})=0\ \Leftrightarrow\ g(\lambda x_{0},\ldots,\lambda x_{n})=0
  \]

  Beweis durch Widerspruch. Nicht alle $g_{d}$ liegen in $\mathfrak{A}$.

  $\Rightarrow\exists(x_{0},\ldots,x_{n})\in\mathbb{A}^{n+1}(k)$, so
  dass $g(x_{0},\ldots,x_{n})=0$, aber $g_{d_{0}}(x_{0},\ldots,x_{n})\neq0$.

  $\Rightarrow0\,\not\equiv,\sum_{d}g_{d}(x_{0},\ldots,x_{n})T^{d}\in k[T]$

  $\Rightarrow(\exists\lambda\in k^{\times})$ $0\neq\sum_{d}g_{d}(x_{0},\ldots,x_{n})\lambda^{d}=\sum_{d}g_{d}(\lambda x_{0},\ldots,\lambda x_{n})=g(\lambda x_{0},\ldots,\lambda x_{n})=0$.
  Widerspruch.

  $\Rightarrow\mathfrak{A}=(f_{1},\ldots,f_{m})$, $f_{j}$ homogen.

  $\Rightarrow Z=V_{+}(f_{1},\ldots,f_{m})$. 

  \begin{align*}
    Z\ni(x_{0}:\cdots:x_{n}) & \Leftrightarrow(\lambda x_{0},\ldots,\lambda x_{n})\in\overline{Y}\ \forall\lambda\in k^{\times}\text{ und }\neq0\\
                             & \Leftrightarrow f_{i}(x_{0},\ldots,x_{n})=0\ \forall1\leq i\leq n,\ (x_{0},\ldots,x_{n})\in\mathbb{P}^{n}
  \end{align*}

  \rule[0.5ex]{1\columnwidth}{1pt}
\end{proof}
Zu Bemerkung 49 

Nach Satz 51 und Definition von $\mathcal{O}_{Z}'$ folgt: Ist $X$
eine projektive Varietät und $U\subset X$ offen, so können wir 

$\mathcal{O}_{X}(U)=\{f:U\rightarrow k\mid\forall x\in U\ \exists x\in V\underset{\text{offen}}{\subset}U,\ g,h\in k[X_{0},\ldots,X_{n}]$
homogen vom gleichen Grad mit $h(v)\neq0,\ f(v)=\frac{g(v)}{h(v)},\ \forall v\in V\}$.
({*}) 

Insbesondere gilt:
\begin{prop}[orig. 56]
  \label{prop:charakterisierung-morphismen-proj-varietaeten}
  Seien $V\subseteq\mathbb{P}^{m}(k)$, $W\subset\mathbb{P}^{n}(k)$
  projektive Varietäten und
  \[
    V\subseteq\mathbb{P}^{m}(k)\xrightarrow{\phi}W\subseteq\mathbb{P}^{n}(k)
  \]

  eine Abbildung. Dann ist $\phi$ eine Morphismus genau dann, wenn
  es zu jedem $x\in V$ eine offene Menge $x\in U_{x}\subset V$ und
  homogene Polynome $f_{0},\ldots,f_{n}\subseteq k[X_{0},\ldots,X_{m}]$
  vom selben Grad existiert mit
  \[
    \phi(y)=(f_{0}(y),\ldots,f_{n}(y))\quad\forall y\in U_{x}
  \]
\end{prop}
\begin{proof}
  \mbox{}
  \begin{itemize}
  \item ``$\Rightarrow$'', Übung.
  \item ``$\Leftarrow$''.
    \begin{enumerate}
    \item $\phi$ stetig: Sei $Z\subseteq W$ abgeschlossen. Ohne Einschränkung
      $Z=V_{+}(g)\cap W$ für ein homogenes Polynom $g$. Dann berechnet
      sich das Urbild
      \[
        \phi^{-1}(Z)=V_{+}(g\circ\phi)\cap V.
      \]
      Auf $U_{x}$, $x\in V$, ist $g\circ\phi$ als homogenes Polynom in
      $X_{0},\ldots,X_{n}$ gegeben. 

      $\Rightarrow V(g\circ\phi)\cap U_{x}=\phi^{-1}(Z)\cap U_{x}$ abgeschlossen
      in $U_{x}$ für alle $x$.

      $\Rightarrow\phi^{-1}(Z)\subseteq V$ abgeschlossen.
    \item Zu zeigen: $\forall W'\subseteq W$ offen, $g\in\mathcal{O}_{W}(W')$
      ist $g\circ\phi\in\mathcal{O}_{V}(\phi^{-1}(W'))$.

      $\Rightarrow$ ({*}) Es ex. eine offene Umgebung $W_{y}$ in $W'$
      mit $g=\frac{h}{q}$ auf $W_{y}$, $h,q$ homogen vom Grad $d$.

      $\Rightarrow\phi_{|U_{x}\cap\phi^{-1}(W_{y}):=\tilde{U}_{x}}$ ist
      auch von dieser Gestalt.

      $\Rightarrow$ ({*}) $\frac{h(f_{0},\ldots,f_{n})}{q(f_{0},\ldots,f_{n})}=g\circ\phi_{|\tilde{U}_{x}}\in\mathcal{O}_{V}(\tilde{U}_{x})$.
    \end{enumerate}
    $\Rightarrow$ (Verkleben) $g\circ\phi\in\mathcal{O}_{V}(\phi^{-1}(V))$.
  \end{itemize}
\end{proof}



\section{Koordinatenwechsel in $\mathbb{P}^{n}$}
\label{sec:koordinatenwechsel-projektiver-raum}

$A=(a_{ij})\in GL_{n+1}(k)$ eine invertierbare $k^{n+1}\rightarrow k^{n+1}$
lineare Abbildung, die Ursprungsgeraden in solche überführt, bzw.
die Äquivalenzrelation respektiert. Wir erhalten Abbildungen:
\begin{align*}
  \mathbb{P}^{n}(k) & \overset{\phi_{A}}{\longrightarrow}\mathbb{P}^{n}(k)\\
  (x_{0}:\ldots:x_{n}) & \longmapsto\left(\sum_{i=0}^{n}a_{0_{i}}x_{i}:\cdots:\sum_{i=0}^{n}a_{n_{i}}x_{i}\right),
\end{align*}

die nach Satz 56 ein Morphismus von Prävarietäten ist. Offensichtlich
gilt für $A,B\in GL_{n+1}(k)$:
\[
  \varphi_{A\cdot B}=\varphi_{A}\circ\varphi_{B}
\]

d.h. $\varphi_{A}$ ist insbesondere wieder ein Isomorphismus, \textbf{der
  durch $A$ bestimmte Koordinatenwechsel des $\mathbb{P}^{n}(k)$}.
\emph{Bezeichne} Aut$(\mathbb{P}^{n}(k))$ die Gruppe der Automorphismen
von $\mathbb{P}^{n}(k)$. Es folgt:
\[
  \varphi_{-}:GL_{n+1}(k)\rightarrow\text{Aut}(\mathbb{P}^{n}(k))
\]

ist ein Gruppenhomomorphismus mit 
\[
  Z:=\ker\varphi=\{\lambda E_{n+1},\ \lambda\in k^{\times}\}
\]

die Untergruppe der Skalarmatrizen. \emph{Später}:
\[
  PGL_{n+1}(k):=GL_{n+1}(k)/Z\twoheadrightarrow\text{Aut}(\mathbb{P}^{n}(k)),\quad Z\cong k^{\times}
\]

die \textbf{projektive lineare Gruppe}.
\begin{example*}
Sei $n=1$. Es ist
\begin{align*}
PGL_{2}(\mathbb{C}) & =\left\{ \begin{array}{rl}
\mathbb{P}^{1}(\mathbb{C}) & \rightarrow\mathbb{P}^{1}(\mathbb{C})\\
(z:w) & \mapsto(az+bw,cz+dw)
\end{array}\right\} \\
 & \leftrightarrow\text{Möbiustransformationen }z\mapsto\frac{az+b}{cz+d}
\end{align*}
\end{example*}



\section{Lineare Unterräume von $\mathbb{P}^{n}$}
\label{sec:lineare-unterraeume-von-pn}

Sei $\varphi:k^{m+1}\rightarrow k^{n+1}$ ein \emph{injektiver} Homomorphismus
von $k$-Vektorräumen. $\varphi$ induziert eine injektive Abbildung
\[
  \imath:\mathbb{P}^{m}(k)\rightarrow\mathbb{P}^{n}(k)
\]

die nach Satz \ref{prop:charakterisierung-morphismen-proj-varietaeten} ein Morphismus von Prävarietäten ist. Das Bild von
$\imath$ ist eine abgeschlossene Untervarietät. Ist $A=(a_{ij})\in M_{l\times(n+1)}$
mit $\text{im}(\varphi)=\ker(k^{n+1}\xrightarrow{A}k^{l})$ und
\[
  f_{i}:=\sum_{j=0}^{n}a_{ij}X_{j}\in k[X_{0},\ldots,X_{n}], \text{ für } i = 1, \ldots, l
\]

so identifiziert $\imath$ den projektiven Raum $\mathbb{P}^{m}(k)$ mit $V_{+}(f_{1},\ldots,f_{l}) \subseteq \mathbb{P}^{n}(k)$.
(Die Abbildung $\imath:\mathbb{P}^{m}(k)\rightarrow V_{+}(f_{1},\ldots,f_{l})$
ist ein Isomorphismus von Prävarietäten, mit Umkehrabbildung induziert von $\varphi^{-1}:\varphi(k^{m+1})\rightarrow k^{m+1}$)
\begin{example*}
  $\mathbb{P}^{m}=V_{+}(X_{m+1},\ldots,X_{n})\subseteq\mathbb{P}^{n}$.
  Solche Unterräume heißen \textbf{lineare Unterräume} (der Dimension
  $m$).

  $m=0$: Punkte

  $m=1$: Geraden

  $m=2$: Ebenen

  $m=n-1$: Hyperebenen in $\mathbb{P}^{n}(k)$.
  \begin{itemize}
  \item Zu zwei Punkten $p\neq q\in\mathbb{P}^{n}(k)$ existiert genau eine
    gerade $\overline{pq}$ in $\mathbb{P}^{n}(k)$, die $p$ und $q$
    enthält, da zu zwei verschiedenen Ursprungsgeraden im $k^{n+1}$ genau
    eine Ebene (in $k^{n+1})$ existiert, die beide Geraden enthält.
  \end{itemize}
\end{example*}
\begin{itemize}
\item Je zwei verschiedene Geraden in $\mathbb{P}^{2}(k)$ schneiden sich
  in genau einem Punkt, da Geraden in $\mathbb{P}^{2}$ Ebenen in $k^{3}$
  entsprechen, und zwei Ebenen sich dort genau in einer Geraden, d.h.
  einem Punkt des $\mathbb{P}^{2}$, schneiden. Dimensionsformel (lineare
  Algebra):
  \[
    \dim _{k}E_{1}\cap E_{2}=-\underbrace{\dim_{k} (E_{1}+E_{2})}_{3}+\underbrace{\dim_{k} E_{1}}_{2}+\underbrace{\dim_{k} E_{2}}_{2}=1
  \]
  \emph{Später}: Verallgemeinerung durch den \emph{Satz von Bézout} für allgemeine Unterprävarietäten
  $V_{+}(f)$.
\end{itemize}


\section{Kegel}
\label{sec:Kegel}

Sei $H\subseteq\mathbb{P}^{n}(k)$ Hyperebene, $p\in\mathbb{P}^{n}(k)\backslash H$,
$X\subseteq H$ abgeschlossene Unterprävarietät.
\[
  \overline{X,p}:=\bigcup_{q\in X}\overline{qp}
\]

heißt \textbf{Kegel von $X$ über $p$}, es handelt sich um einen
abgeschlossenen Untervarietät von $\mathbb{P}^{n}(k)$. Ohne Einschränkung:$H=V_{+}(X_{n})$,
$p=(0:\cdots:1)$ (nach Koordinatenwechsel: $H\cong k^{n}\oplus p\cong k\}=k^{n+1}$.)
Für  
\begin{align*}
  X=V_{+}(f_{1},\ldots,f_{m})\subseteq\mathbb{P}^{n-1}(k)=H, & \quad f_{i}\in k[X_{0},\ldots,X_{n-1}]\\
  \Rightarrow X,p=V_{+}(\tilde{f}_{1},\ldots,\tilde{f}_{m})\subseteq\mathbb{P}^{n}(k), & \quad\tilde{f}_{i}\in k[X_{0},\ldots,X_{n}]
\end{align*}

Verallgemeinerung. Sei $\mathbb{P}^{n}(k)\cong\Lambda\subseteq\mathbb{P}^{n}(k)$
linearer Unterraum, $\psi\subseteq\mathbb{P}^{n}(k)$ komplementärer
linearer Unterraum, d.h. $\Lambda\cap\psi=\emptyset$ und $\mathbb{P}^{n}(k)$
ist der bekannte lineare Unterraum von $\mathbb{P}^{n}(k)$, der $\Lambda$
und $\psi$ enthält. $X\subseteq\psi$ abgeschlossene Unterprävarietät.

\textbf{Kegel von $X$ über $\Lambda$}: $\overline{X,\Lambda}=\bigcup_{q\in X}\overline{q,\Lambda}$,
wobei der von $q$ und $\Lambda$ aufgespannte lineare Unterraum $\overline{q,\Lambda}$
der kleinste Unterraum sei, der $q$ und $\Lambda$ enthält.

\section{Quadriken}
\label{sec:quadriken}

Sei in diesem Abschnitt char$(k)\neq2$.
\begin{defn}[orig. 57]
  \label{def:quadrik}
  Eine abgeschlossene Unterprävarietät $Q\subseteq\mathbb{P}^{n}(k)$
  von der Form $V_{+}(q)$, $0 \neq q\in k[X_{0},\ldots,X_{n}]_{2}$
  heißt \textbf{Quadrik}.
  \[
    Q=V_{+}(q)
  \]

  Zur quadratischen Form $q$ gehört eine assoziierte Bilinearform $\beta$ auf
  $k^{n+1}$ (vgl. lineare Algebra), 
  \[
    \beta(v,w):=\frac{1}{2}(q(v+w)-q(v)-q(w)),\quad v,w\in k^{n+1}
  \]

Es gibt eine Basis von $k^{n+1}$, sodass die Strukturmatrix $B$
von $\beta$ die Gestalt
\[
  B=\begin{pmatrix}
    \begin{array}{ccc}
      1\\
      & \ddots\\
      &  & 1
    \end{array} & 0\\
    0 &
    \begin{array}{ccc}
      0\\
      & \ddots\\
      &  & 0
    \end{array}
  \end{pmatrix}
\]

hat, d.h. Koordinatenwechsel zur Basiswechselmatrix liefert einen
Isomorphismus
\[
  Q\xrightarrow{\sim}V_{+}(X_{0}^{2}+\cdots+X_{r-1}^{2}),\quad r=\text{rk }B
\]
\end{defn}
\begin{lem}[orig. 58]
\label{lem:irreduzibilitaet-quadriken}
	\begin{enumerate}
	
		\item $X_{0}^{2} + \ldots + X_{r-1}^{2}$ ist irreduzibel $\iff$ $r > 2$
		\item $V_{+}(X_{0}^{2} + \ldots + X_{r-1}^{2})$ ist irreduzibel $\iff$ $r \neq 2$
	\end{enumerate}
\end{lem}
\begin{proof}
	\begin{itemize}
		\item $r=0,1: X_{0}^2 = X_{0} \cdot X_{0} \Rightarrow V_{+}(X_{0}^2) = V_{+}(X_{0})$ irreduzibel
		\item $r=2: X_{0}^{2} + X_{1}^{2} = (X_{0} + i\cdot X_{1})\cdot(X_{0} - i \cdot X_{1})$ für $i = \sqrt{-1}$ 
		\item $r>2: $ Angenommen $\sum_{i}{a_{i} X_{i}} \cdot \sum_{j}{b_{j}X_{j}} = X_{0}^{2} + \ldots X_{r-1}^{2}$.\\
		Ausmultiplizieren $+$ Koeffizientenvergleich $\Rightarrow$ Widerspruch.
	\end{itemize}
\end{proof}

\begin{prop}[orig. 59]
  \label{prop:quadrik-in-normalform}
  Ist $r\neq s$, so sind $V_{+}(T_{0}^{2}+\cdots+T_{r-1}^{2})$ und
  $V_{+}(T_{0}^{2}+\cdots+T_{s-1}^{2})$ nicht isomorph.
\end{prop}
\begin{proof}
  Später: Es gibt keinen Koordinatenwechsel von $\mathbb{P}^{n}(k)$,
  der die beiden Mengen miteinander identifiziert, damit auch kein Automorphismus von
  $\mathbb{P}^{n}(k)$.
\end{proof}

\begin{defn}
  \label{def:dim-und-rang-einer-quadrik}
  Eine Quadrik $Q\subseteq\mathbb{P}^{n}(k)$ mit
  $Q\cong V_{+}(T_{0}^{2}+\cdots+T_{r-1}^{2})$, $r\geq1$, hat \textbf{Dimension $n-1$} und den \textbf{Rang $r$}. (nach Satz
  eindeutig!)
\end{defn}

\begin{cor}[orig. 61]
\label{cor:klassifikation-von-quadriken}	
  Zwei Quadriken $Q_{1}$ und $Q_{2}$ sind genau dann isomorph als
  Prävarietäten, wenn sie dieselbe Dimension und denselben Rang haben.
\end{cor}
\begin{proof}
  \mbox{}
  \begin{itemize}
  \item[,,$\Leftarrow$``]
    $Q_{1}\cong V_{+}(T_{0}^{2}+\cdots+T_{n-1}^{2})\cong Q_{2}$ in
    dem selben $\mathbb{P}^{n}$.
  \item[,,$\Rightarrow$``] Für $Q\subseteq\mathbb{P}^{n}(k)$ berechne
    $K(Q)$. Ohne Einschränkung
    $Q=V_{+}(X_{0}^{2}+\cdots+X_{n-1}^{2})$.
    \begin{enumerate}
    \item $r=1$: $V_{+}(X_{0}^{2})=V_{+}(X_{0})=\mathbb{P}^{n-1}(k)$:
      $K(Q)=k(T_{1},\ldots,T_{n-1})$.
    \item $r=2$: reduzibel: Zerlegung in zwei Hyperebenen
      $Z\cong\mathbb{P}^{n-1}$

      $\Rightarrow K(Z)\cong k(T_{1},\ldots,T_{n-1})$.
    \item $r>2$:
      $U=V(1+T_{1}^{2}+\cdots+T_{n-1}^{2})\subseteq\mathbb{A}^{n}(k)$
      ist nichtleere offene affine Teilmenge von $Q$.

      $\Rightarrow K(Q)=K(U)=\text{Quot}(\Gamma(U))=\text{Quot}(k[T_{1},\ldots,T_{n}]/(1+T_{1}^{2}+\cdots+T_{n-1}^{2})$

      $\Rightarrow\text{trgrad}_{k}\ K(Q)=n-1$. 
    \end{enumerate}
  \end{itemize}
\end{proof}
\begin{example}
  $Q$ Quadrik in $\mathbb{P}^{n}$ (vgl. Joe Harris, Seite 34).
  \begin{enumerate}
  \item In $\mathbb{P}^{1}(k)$. 
    \begin{itemize}
    \item \emph{Rang 2:} 2 Punkte, reduzibel. 
    \item \emph{Rang 1:} 1 Punkt (Doppelpunkt). 
    \end{itemize}
  \item In $\mathbb{P}^{2}(k)$.
    \begin{itemize}
    \item \emph{Rang 3:} Glatter Kegel
      $\cong\mathbb{P}^{1}(k)$. $X_{0}^{2}+X_{1}^{2}-X_{2}^{2}=0$
    \item \emph{Rang 2:} 2 verschiedene Geraden, reduzibel. 
    \item \emph{Rang 1:} (Doppel)gerade.
    \end{itemize}
  \item In $\mathbb{P}^{3}(k)$.
    \begin{itemize}
    \item Rang 1: Doppelebene (2-dimensionaler linearer Unterraum)
    \item Rang 2: (insert image)
    \item Rang 3: (insert image)
    \item Rang 4: (insert image)
    \end{itemize}
  \end{enumerate}
\end{example}
Die Quadrik $Q\subseteq\mathbb{P}^{n}(k)$ heißt \textbf{glatt}, falls
$r=n+1$, d.h. falls die Matrix $B$ zu $q$ maximalen Rang hat. Für
$\text{rk}(Q)>3$, $\dim(Q)=d$, ist
$Q\cong\overline{\widetilde{Q},\Lambda}$ Kegel über einer
\textbf{glatten} Quadrik $\widetilde{Q}$, da Dimension $r-2$
bzgl. einer $(d-r+2)$-dimensionalen Unterraums 1.
\begin{itemize}
\item $r=1,2$ ausgeartet.
\item $r=1$. $Q=V_{+}(X_{0}^{2})=V_{+}(X_{0})$ Hyperebenen in
  $\mathbb{P}^{n}(k)$.  Der Unterschied zwischen $V_{+}(X_{0}^{2})$
  und $V_{+}(X_{0})$ ist für eine projektive Varietät $Q$ nicht
  sichtbar, jedoch in der Theorie der Schemata unterscheidbar!
\item $r=2$. $Q=V_{+}(X_{0}^{2}+X_{1}^{2})$ reduzibel, d.h. keine
  Prävarietät in unserem Sinne! Auch hier werden uns Schemata später helfen.
\end{itemize}
\medskip{}

$Q=V_{+}(X_{0}^{2}+X_{1}^{2}+\cdots+X_{n-1}^{2})\subseteq\mathbb{P}^{d+1}$,
$r\leq d+2$

$\tilde{Q}=V_{+}(X_{0}^{2}+\cdots+X_{n-1}^{2})\subseteq\mathbb{P}^{r-1}$
glatt.

$A=\mathbb{P}^{d+1-v}=V_{+}(X_{0},\ldots,X_{n-1})\subseteq\mathbb{P}^{d+1}$

$Q=\overline{\widetilde{Q},\Lambda}$


\chapter{Das Ringspektrum}
\label{chap:das-ringspektrum}

Prävarietäten sind Verklebungen. $k$ algebraisch abgeschlossen:

Affine Varietäten $\leftrightarrow$ integere endlich erzeugte
$k$-Algebren.  Punkte $\hat{=}$ maximale Ideale.

\textbf{Ziel}: Schemata sind Verklebungen.

Affine Schemta $\leftrightarrow$ (kommutative) Ringe. Punkte $\hat{=}$
Primideale.

\textbf{Ziel}: Wir wollen einen Funktor:
\begin{align*}
  A & \longmapsto(\Spec(A),\mathcal{O}_{\Spec(A)})\\
  \text{Ring} & \longrightarrow\text{top. Raum}
\end{align*}

,,Garbe von Funktionen`` verallgemeinert ,,Systeme von Funktionen``
für Raume von Funktionen.

(Insbesondere $k$-Algebren über beliebige Körper $k$!)

(Sei $\varphi:A\rightarrow B$ Ringhomomorphismus,
$\mathfrak{m}\subseteq B$ maximales Ideal. Dann folgt i.A. nicht, dass
$\varphi^{-1}(\mathfrak{m})$ maximal ist. Wir haben also zu wenige
maximale Ideale.)

\section*{Das Ringspektrum als topologischer Raum}

\section{Definition von Spec(A)}

Sei $A$ stets ein kommutativer Ring. Spec(A) =
$\{\mathfrak{p}\subseteq A$ Primideal\}. Sei $M\subset A$.
\begin{align*}
  V(M) & =\{\mathfrak{p}\in\Spec(A)\mid\mathfrak{p}\supset
         M\}=V\{\langle M\rangle\}\\
  V(f) & =V(\{f\})\text{\,für }f\in A
\end{align*}

\begin{lem}[1] Es ist
  \begin{align*}
    \{\text{Ideale in }A\} & \longrightarrow\text{\{Teilmengen in }\Spec(A)\}\\
    \mathfrak{A} & \longmapsto V(\mathfrak{A})
  \end{align*}

  ist eine inklusionsumkehrende Abbildung. Es gilt:
  \begin{enumerate}
  \item $V(0)=\Spec(A)$, $V(1)=\emptyset$.
  \item $V\left(\bigcup_{i\in
        I}\mathfrak{a}_{i}\right)=V\left(\sum_{i\in
        I}\mathfrak{a}_{i}\right)=\bigcap_{i\in I}V(\mathfrak{a}_{i})$
  \item
    $V(\mathfrak{a}\cap\mathfrak{a}')=V(\mathfrak{a}\mathfrak{a}')=V(\mathfrak{a})\cup
    V(\mathfrak{a}')$
  \end{enumerate}
\end{lem}
\begin{proof} \mbox{}
  \begin{itemize}
  \item (1), (2) klar.
  \item
    (3). $\mathfrak{p}\supset\mathfrak{a}\cap\mathfrak{a}'\supset\mathfrak{a}\mathfrak{a}'$.

    $\Rightarrow\mathfrak{p}\supset\mathfrak{a}\mathfrak{a}'$.

    $\Rightarrow$ (Primideal) $\mathfrak{p}\supset\mathfrak{a}$ oder
    $\mathfrak{p}\supset\mathfrak{a}'$.

    $\Rightarrow\mathfrak{p}\supset\mathfrak{a}\cap\mathfrak{a}'$

  \end{itemize}
\end{proof}
\begin{defn} Spec(A) mit der Topologie, dessen abgeschlossene Mengen
  gerade die Mengen der Form $V(\mathfrak{a})$,
  $\mathfrak{a}\subset A$ ein Ideal sind, heißt (Prim)Spektrum von $A$
  (mit der Zariski-Topologie).
  \begin{align*}
    x\in\Spec(A) & \leftrightarrow\mathfrak{p}_{x}\subset A\text{ Primideal}\\
    Y\subset\Spec(A), & \phantom{\leftrightarrow\
                              }I(Y):=\bigcap_{\mathfrak{p}\in Y}\mathfrak{p}
  \end{align*}

  $I(-)$ ist inklusionserhaltend, $I(\emptyset)=A$.
\end{defn}
\begin{prop} $\mathfrak{a}\subset A$ Ideal,
  $Y\subset\Spec(A)$. Dann gilt:
  \begin{enumerate}
  \item $\text{rad }I(Y)=I(Y)$, $V(\mathfrak{a})=V(\text{rad
    }\mathfrak{a})$
  \item $I(V(\mathfrak{a}))=\text{rad}(\mathfrak{a})$,
    $V(I(Y))=\overline{Y}$ (Abschluss in $\Spec(A)$).
  \item Wir haben eine 1:1-Korrespondenz:
    \begin{align*}
      \{\mathfrak{a}\subset A\mid\mathfrak{a}=\rad\mathfrak{a}\}
      & \longrightarrow\{\text{abg. Teilmengen }Y\text{ in }\Spec(A)\}
    \end{align*}
  \end{enumerate}
\end{prop}
\begin{proof} \mbox{}
  \begin{enumerate}
  \item $V(\mathfrak{a})=V(\rad\mathfrak{a})$.
    \begin{itemize}
    \item ,,$\supseteq$``. Klar, da rad
      $\mathfrak{a}\supseteq\mathfrak{a}$.
    \item ,,$\subseteq$``. Aus $f^{r}\in\mathfrak{a}\subseteq\mathfrak{a}$
      folgt $f\in\mathfrak{p}$, da $\mathfrak{p}$ Primideal. Damit:
      $\text{rad }\mathfrak{a}\subset\mathfrak{p}.$
    \end{itemize}
  \item $\text{rad }\mathfrak{a}=\bigcap_{\mathfrak{p}\in
      V(\mathfrak{a})}\mathfrak{p}=IV(\mathfrak{a})$.  Es ist:
    \begin{align*}
      V(\mathfrak{b})\supseteq Y & \Leftrightarrow(\forall\mathfrak{p}\in Y:\
                                   \mathfrak{p}\supset\mathfrak{b})\\
                                 & \Leftrightarrow
                                   I(Y)\supseteq\mathfrak{b}.
    \end{align*} Damit ist $V(I(Y))$ die kleinste abgeschlossene
    Teilmenge, die $Y$ umfasst, d.h. $V(I(Y))=\overline{Y}$.
  \item
  \end{enumerate}
\end{proof}

\section{Topologische Eigenschaften von Spec(A)}

Definiere $D(f):=D_{A}(f):=\Spec(A)\setminus V(f)=\{x \in\Spec A \mid f\notin\mathfrak{p}_{x}\}$,
\begin{align*}
  \text{ev}_{x}:A & \longrightarrow A/\mathfrak{p}_{x}\subseteq \kappa_{x}(A) := \Quot(A/\mathfrak{p}_{x})\\
  f & \longmapsto f(x) := f(\mathfrak{p}_{x}) := f \mod \mathfrak{p}
\end{align*}

Für $x \in D(f)$ gilt dann $f(x) = \text{ev}_{x}(f) \neq 0$.

\textbf{Standard prinzipal offene Mengen}.
\begin{align*}
  D(0) & =\emptyset,\ D(1)=\Spec(A)=D(u),\ u\in A^{\times}\\    
  & D(f)\cap D(g) = D(fg)
\end{align*}

\begin{lem}
\label{lem:charakterisierung-ueberdeckungen-prinzipal}	
Für $f_{i} \in A, i\in I$, $g\in A$ gilt:
  \begin{align*}
    D(g)\subseteq\bigcup_{i\in I}D(f_{i})
    & \Leftrightarrow g^{n}\in\mathfrak{a}=(f_{i},i\in I)\text{ für }n \in \mathbb{N} \text{ geeignet}\\
    & \Leftrightarrow g\in\rad(\mathfrak{a})
  \end{align*}
\end{lem}
\begin{proof} Es gilt:
  \begin{align*}
    D(g)\subseteq\bigcup_{i}D(f)
    & \Leftrightarrow V(g)\supseteq\bigcap_{i} V(f_{i})=V(\mathfrak{a})\\
    & \Leftrightarrow g\in\rad((g))\subseteq\rad(\mathfrak{a}) \text{ nach } \ref{prop:nullstellensatz-primspektrum}
  \end{align*}

  Für $g=1$, folgt:
  \[ \Spec(A)=\bigcup_{i\in I}D(f_{i})\Leftrightarrow\sum_{i\in
      I}Af_{i}=A
  \]
\end{proof}
\begin{prop}
\label{prop:prinzipal-offene-bilden-basis}
Die prinzipal offenen Mengen $D(f)$, $f\in A$, bilden
  eine Basis der Topologie von $\Spec(A)$, und sind
  quasikompakt. Insbesondere ist $\Spec(A)$ quasikompakt.
\end{prop}
\begin{proof} Nach Lemma \ref{lem:zariski-top-auf-spektrum}$.(ii)$ gilt:
  \[
    V(\mathfrak{a})=\bigcap_{f \in\mathfrak{a}}V(f)\Longrightarrow\Spec A\setminus
    V(\mathfrak{a})=\bigcup_{f\in\mathfrak{a}}D(f)\Rightarrow\text{Basis
      der Topologie}
  \]

  Sei $D(g)\subseteq\bigcup_{i}D(f_{i})$.

  \ref{lem:charakterisierung-ueberdeckungen-prinzipal} $\Rightarrow$ $g^{n}=\sum_{i\in I}a_{i}f_{i}$, $a_{i}\in A$
  fast alle 0.

  $\Rightarrow D(g)\subseteq\bigcup_{i\in J}D(f_{i})$ $\forall i\in J\subseteq I$ endlich

  $\Rightarrow D(g)$ quasikompakt.
\end{proof}

\section{Der Funktor $A\protect\mapsto\text{Spec}(A)$}
\label{sec:spec-als-funktor}
\textbf{Ziel:} Wir wollen einen kontravarianten Funktor
\begin{align*}
  \text{\underline{CRing}} & \longrightarrow\text{\underline{Top}}\\
  A & \longmapsto\Spec A
\end{align*}
definieren. Sei $\varphi:A\longrightarrow B$ ein Ringhomomorphismus, $\mathfrak{q}$ Primideal von $B$. Es folgt:
$\varphi^{-1}(\mathfrak{q})\unlhd A$ ist Primideal, denn $A/\varphi^{-1}(\mathfrak{q})\hookrightarrow B/\mathfrak{q}$ ist integer als Unterring eines integren Rings.
Wir erhalten also eine Abbildung
\begin{align*}
  ^{a}\varphi=\Spec\varphi:\ \Spec B & \longrightarrow\Spec A\\
  \mathfrak{q} & \longmapsto\varphi^{-1}(\mathfrak{q})
\end{align*}

\begin{prop} \mbox{}
\label{prop:abbildungen-auf-spektren-sind-stetig}	
  \begin{enumerate}
  \item $(^{a}\varphi)^{-1}(V(M))=V(\varphi(M))$ für
    $M\subseteq\Spec A$ Teilmenge, insbesondere gilt
    $(^{a}\varphi)^{-1}(D(f))=D(\varphi(f))$, $f\in A$.
  \item
    $V(\varphi^{-1}(\mathfrak{b}))=\overline{^{a}\varphi(V(\mathfrak{b}))}$
    für $\mathfrak{b}\unlhd B$ Ideal.
  \end{enumerate}
\end{prop}

\begin{proof} \mbox{}
  \begin{enumerate}
  \item Für $\mathfrak{q}\in\text{Spec }B$ gilt:
    \begin{align}
       \mathfrak{q}\in V(\varphi(M)) \iff \mathfrak{q}\supseteq\varphi(M)
      \iff \varphi^{-1}(\mathfrak{q})\supseteq M \iff \mathfrak{q}\in(^a\varphi)^{-1}(V(M))
    \end{align}
    Weiter:
    \begin{align}
      D(\varphi(f)) & =\Spec(B)\setminus V(\varphi(f)) \\
                    & =\Spec(B)\setminus (^a\varphi)^{-1} (V(f)) \\
                    & = (^a\varphi)^{-1} (D(f))
    \end{align}
     
  \item
    $\overline{^{a}\varphi(V(\mathfrak{b}))}=VI(^{a}\varphi(V(\mathfrak{b})))$
    nach Satz \ref{prop:nullstellensatz-primspektrum}. Nach Definition gilt:
    \begin{align*} I(^{a}\varphi(V(\mathfrak{b}))
      & =\bigcap_{\mathfrak{p}\in^{a}\varphi(V(\mathfrak{b}))}
        \mathfrak{p}=\bigcap_{\mathfrak{q}\in V(\mathfrak{b})}
        \varphi^{-1}(\mathfrak{q})\\ \text{komm. Algebra }
      & =\varphi^{-1}(\rad\mathfrak{b})\\
      & \overset{!}{=}\rad\varphi^{-1}(\mathfrak{b})
    \end{align*}
    Denn: Ohne Einschränkung gelte $\mathfrak{b}=0$,
    $\varphi^{-1}(\mathfrak{b})=\ker\varphi$
    (betrachte $A/\varphi^{-1}(\mathfrak{b})\hookrightarrow B/\mathfrak{b})$. Es
    ist:
    \begin{align*}
      a\in\varphi^{-1}(\sqrt{0})
      & \Leftrightarrow\varphi(a)^{n}=\varphi(a^{n})=0
        \text{ für }n\text{ geeignet}
    \end{align*} $V(\cdot )$
    liefert die Behauptung: $V(\rad\varphi^{-1}
    (\mathfrak{b}))=V(\varphi^{-1}(\mathfrak{b}))$ nach Satz \ref{prop:nullstellensatz-primspektrum}.
  \end{enumerate}
\end{proof}
Insbesondere ist $^{a}\varphi:\Spec B\rightarrow\Spec A$
\emph{stetig}.  Wegen
\[
  ^{a}(\psi\circ\varphi)=\ ^{a}\varphi\ \circ\ ^{a}\psi \text{ und } ^{a}\mathrm{id}_{A} = \mathrm{id}_{\Spec A}
\]
für einen weiteren Ringhomomorphismus $\psi:B\rightarrow C$ ist
$A\mapsto\Spec A$ der gesuchte kontravariante Funktor.

\begin{cor}
\label{cor:charakterisierung-dominanz-auf-spec}	
  $^{a}\varphi$ ist \textbf{dominant},
  d.h. $\im\ (^{a}\varphi)\subseteq\Spec A$ dicht
  $\iff$ Jedes Element in $\ker\varphi$ ist nilpotent:
  $\ker\varphi\subseteq\text{rad}(0)$.
\end{cor}

\begin{prop} \mbox{}
\label{prop:spec-quotienten-lokalisierung}	
  \begin{enumerate}
  \item Ist $\varphi:A\rightarrow B$ ein surjektiver
    Ringhomomorphismus mit $\ker\varphi=:\mathfrak{a}$, dann ist
    $^{a}\varphi$ ein Homöomorphismus von $\Spec B$ auf
    $V(\mathfrak{a})\underset{\text{abg.}}{\subseteq}\Spec A$.
  \item Ist $S$ eine multiplikativ abgeschlossene Teilmenge von $A$,
    und $\varphi:A\longrightarrow S^{-1}A=:B$ die kanonische
    Lokalisierungsabbildung, dann ist $^{a}\varphi$ ein
    Homöomorphismus, von $\Spec S^{-1}A$ auf $\{\mathfrak{p}\in\Spec A\mid
    S\cap\mathfrak{p}=\emptyset\}$.
  \end{enumerate}
\end{prop}

\begin{proof}
  $^{a}\varphi$ injektiv + $\im\ ^{a}\varphi$ ist bekannt aus kommutative Algebra.
  \emph{Ferner}: Für $\mathfrak{q}\in\text{Spec }B$,
  $\mathfrak{b}\unlhd B$ Ideal gilt
  $\mathfrak{q}\supseteq\mathfrak{b}\Leftrightarrow
  \varphi^{-1}(\mathfrak{q})\supseteq\varphi^{-1}(\mathfrak{b})$, also
  \begin{align*}
    ^{a}\varphi(V(\mathfrak{b})) & =V(\varphi^{-1}(\mathfrak{b})),
  \end{align*}
  d.h. $^{a}\varphi$ ist abgeschlossen.
\end{proof}

\section{Beispiele}
\label{sec:beispiele-spektren}
\begin{itemize}
\item $\Spec A=\emptyset\Leftrightarrow A=\{0\}$.
\item $A$ Körper oder Ring mit einem einzigem Primideal: $\Spec
  A=\{\mathfrak{p}\}$.
\item $A$ Artinsch: $\Spec A$ endlich und diskret (da maximale
  Primideale mit den minimalen Primidealen übereinstimmen)

  ($\Spec A=\Spec(A/\sqrt{0})$, $A/\sqrt{0}$ Produkt von Körpern.
  $\Spec(\prod_{i} A_{i})=\coprod_{i}\Spec(A_{i})$
\end{itemize}

\begin{example}
\label{bsp:spec-von-hauptidealring}	
  Sei $A$ Hauptidealring (z.B. $\mathbb{Z}$ oder $K[X]$). Falls
  $\mathfrak{p}$ ein maximales Ideal ist, dann ist
  $\mathfrak{p}=(\pi)$, $\pi$ Primelement in $A$.

  Alle Primideale sind maximal oder 0.

  Abg. Punkte von $\Spec A\leftrightarrow$ Primelemente modulo $A^{\times}$

  $\overline{\{\eta\}}=\Spec A$ für $\eta\in\Spec A$ mit $\mathfrak{p}_{\eta}=(0)$.

  Abgeschlossene Mengen $\Spec A\neq
  V(\mathfrak{a})\overset{0\neq\mathfrak{a}=(f)}{=}V(f)=\{(p_{1}),\ldots,(p_{n})\}$
  falls $f=p_{1}^{e_{1}}\cdots p_{n}^{e_{n}}$, $p_{i}$ paarweise
  verschieden, $e_{i}\geq1$, sind genau \emph{endliche Mengen abgeschlossener Punkte.}

  $g\neq0\neq f$:
  \begin{align*}
    V(f)\cap V(g) & =V(f,g)=V(d), & d=\text{ggT}(f,g)\\
    V(f)\cup V(g) & =V((f)\cap(g))=V(e), & e=\text{kgV}(f,g)
  \end{align*}

  Falls $A$ \emph{lokaler} Hauptidealring ist (also diskreter
  Bewertungsring, der kein Körper ist), dann:
  \[
    \Spec A=\{x,\eta\},\ \mathfrak{p}_{x}\text{ max. Ideal},\
    \mathfrak{p}_{\eta}=(0)
  \]

  $\{\eta\}$ einzige nicht-triviale  offene Menge.
\end{example}
 
\begin{example}
\label{bsp:zusammenhang-affine-varietaeten}	
  Sei $k$ algebraisch abgeschlossener Körper. Affine Varietäten
  $V\leftrightarrow$ endlich erzeugte $k$-Algebren $A$.

  $V=$\{max. Ideale in $A$\} $\subseteq\Spec(A)$

  Topologie auf $V$ ist die Unterraumtopologie von $\Spec(A)$.
\end{example}
 
\begin{example}
\label{bsp:spec-polynomring-ueber-hauptidealring}	
  Sei $R$ Hauptidealring, $A=R[T]$, $X=\Spec(A)$. $R$ faktoriell $\Rightarrow R[T]$ faktoriell, nach dem Satz von Gauß, mit Primidealen:
  \begin{enumerate}
  \item $p\in R$ prim
    \begin{proof}
      $p\in R$ prim $\Rightarrow R/pR$ Körper. Nach Proposition \ref{prop:spec-quotienten-lokalisierung} gilt:
      \[
        \overline{pR[T]}=V(pR[T])\cong\Spec\left(R/pR[T]\right)
      \]
      ein Hauptidealring mit unendlich vielen Elementen. Damit ist $pR[T]$
      \emph{nicht} maximal, sondern
      \[ V(pR[T])=\{pR[T], (f,p), f\in R[T]
          \text{ mit } \overline{f}\in R/p[T]
        \text{ irreduzibel}\}
      \]
    \end{proof}
  \item $f\in R[T]$ primitives Polynom, irreduzibel in $\Quot(R)[T]$
    \begin{proof}
      Sei $f$ primitives, irreduzibles Polynom.
      \begin{itemize}
      \item $l(f)\in R^{\times} \Rightarrow$ (Division mit Rest)
        $R\subseteq R[T]/pR[T]$ ist eine ganze Ringerweiterung
        und ein endl.-erz. freier $R$-Modul vom Rang $\deg(f)$.
        Angenommen, $fR[T]$ ist maximal. Dann ist $R[T]/fR[T]$ ein Körper,
        also $R$ ein Körper (da ganze Ringerweiterung). Widerspruch.
      \item Andernfalls kann $fR[T]$ ein maximales Ideal sein: $R$ habe nur
        endlich viele Primelemente.
        \[
          0\neq a := \prod_{p} p\in R,\ f:=aT-1
        \]
        Es folgt:
        \[
          R[T]/fR[T]\cong R[a^{-1}]=\Quot(R)
        \]
        also ist $fR[T]$ maximal.
      \end{itemize}
    \end{proof}
  \end{enumerate}
\end{example}

%%% Local Variables:
%%% mode: latex
%%% TeX-master: "../AlgGeo1"
%%% End:

\section*{Exkursion über Garben}

Bisher: 
\begin{align}
X \text{affin alg. Menge} \longmapsto \Gamma(X) = \Hom(X, \mathbb{A}^{1})
\end{align}
Jetzt:
\begin{align}
\Spec A \longmapsfrom A
\end{align} 
d.h. $A$ soll den Funktionen auf $\Spec A$ entsprechen.
Für $x \in \Spec A$ definiert man $ev_{x} : A \to \kappa_A(x) := A_{\mathfrak{p}_x} / \mathfrak{p}_{x}A_{\mathfrak{p}_{x}} \cong \Quot(A/\mathfrak{p}_{x})$ durch $f \mapsto f(x) := ev_{x}(f) := f \mod \mathfrak{p}_{x}$.
Mit dieser Definition folgt insbesondere $D(f) = \{ x \in \Spec A \mid f(x) \neq 0\}$. Da $x \mapsto f(x)$ keine Funktion im engeren Sinne ist, können wir diese Konstruktion nicht als System von Funktionen auffassen.\\
Wichtige Aussagen: Restriktion $+$ Verklebung $\rightsquigarrow$ Garben.

\section{Prägarben und Garben}
\label{sec:garben}

\begin{defn}
\label{def:praegarbe}
Sei $X$ ein topologischer Raum. \\
$(i)$ Eine $\textbf{Prägarbe}\  \sheaf{F}$ auf X besteht aus den folgenden Daten:
\begin{itemize}
	\item eine Menge $\sheaf{F}(U)$ für jede offene Teilmenge $U \subseteq X$
	\item Eine $\textbf{Restriktionsabbildung} $ $res^{V}_{U} : \sheaf{F}(V) \to \sheaf{F}(U)$ für jedes Paar $U \subseteq V$ offen in $X$, so dass:
	\begin{itemize}
		\item $res^{U}_{U} = \id_{\sheaf{F}(U)}$
		\item $res^{W}_{U} = res^{V}_{U} \circ res^{W}_{V}$ für $U \subseteq V \subseteq W$ offen in $X$
	\end{itemize}
\end{itemize}
$(ii)$ Ein $\textbf{Morphismus von Prägarben}$ $\phi : \sheaf{F} \to \sheaf{G}$ ist eine Familie von Abbildungen $\{\phi_{U} : \sheaf{F}(U) \to \sheaf{G}(U) \mid U \subseteq X \text{ offen }\}$, so dass für alle Paare $U \subseteq V$ offen in $X$, das folgende Diagramm kommutiert:
\[
\xymatrix{\sheaf{F}(V)\ar^{\phi_V}@{->}[r]\ar^{res^{V}_{U}}@{->}[d]  & \sheaf{G}(V) \ar[d]^{res^{V}_{U}}\\
	\sheaf{F}(U)\ar^{\phi_{U}}@{->}[r] & \sheaf{G}(U) }
\]
\underline{Notation:} $U \subseteq V$, $s \in \mathcal{F}(V)$, dann: $s|_{U} := res^{V}_{U}(s)$.\\ Die Elemente in $\sheaf{F}(U)$ heißen $\textbf{Schnitte von } \sheaf{F} \textbf{ über }U$, $\Gamma(U, \sheaf{F}) := \sheaf{F}(U)$.
\end{defn}

Alternative Beschreibung:\\
$\ouv_{X}$: Kategorie offener Mengen von $X$ mit $\Hom(U, V) := \begin{cases} \emptyset, \text{ falls } U \not\subseteq V \\ \{ U \to V\}, \text{ falls } U \subseteq V \end{cases}$.
Eine $\textbf{Prägarbe auf }X$ ist ein kontravarianter Funktor $\mathcal{F} : \ouv_{X} \to \set$. Ersetzt man $\set$ durch eine Kategorie $\cat{C}$, so bekommt man Prägarben $\textbf{mit Werten in } \cat{C}$.
Ein Morphismus von Prägarben $\sheaf{F} \to \sheaf{G}$ ist eine natürliche Transformation $\sheaf{F} \implies \sheaf{G}$.\\
\\
Für eine Prägarbe $\sheaf{F}$ auf $X$, $U \subseteq X$ offen und $U = \bigcup_{i} U_{i}$ eine \underline{offene} Überdeckung von $U$, definiere:
\begin{align}
\rho : \sheaf{F}(U) \to \prod_{i}\sheaf{F}(U_{i}), s \mapsto (s|_{U_{i}})_{i}\\
b : \prod_{i} \sheaf{F}(U_{i}) \to \prod_{(i, j)} \sheaf{F}(U_{i} \cap U_{j}), (s_{i})_{i} \mapsto (s_{i}|_{U_{i} \cap U_{j}})_{(i,j)}\\
b' : \prod_{i} \sheaf{F}(U_{i}) \to \prod_{(i, j)} \sheaf{F}(U_{i} \cap U_{j}), (s_{i})_{i} \mapsto (s_{j}|_{U_{i} \cap U_{j}})_{(i,j)}
\end{align}

\begin{defn}
\label{def:garbe}
$(i)$ Eine Prägarbe $\sheaf{F}$ auf $X$ heißt $\textbf{Garbe}$, falls für alle offenen Teilmengen $U \subset X$ und alle offenen Überdeckungen $U = \bigcup_{i} U_{i}$ wie oben gilt:\\
\[
(Sh)\ \xymatrixcolsep{3pc}\xymatrix{\sheaf{F}(U)\ar^-\rho@{->}[r] & \prod_{i}\sheaf{F}(U_{i}) \ar^-b@<+.5ex>[r] \ar_-{b'}@<-.5ex>[r] & \prod_{(i, j)}\sheaf{F}(U_{i} \cap U_{j})}
\]d.h. $\rho$ ist injektiv und $\im\rho = \{ s \in \prod_{i} \sheaf{F}(U_{i}) \mid b(s) = b'(s)\}$, mit anderen Worten: $(\sheaf{F}(U), \rho)$ ist \textbf{Equalizer} von $b$ und $b'$.\\
Dabei ist $(Sh)$ äquivalent zu:
\begin{enumerate}
	\item[(Sh0)]   $\sheaf{F}(\emptyset)$ ist finales Objekt.
	\item[(Sh1)] Gilt für $s,s' \in \sheaf{F}(U)$  $s|_{U_{i}} = s'|_{U_{i}}$ für alle $i$, so folgt $s = s'$.
	\item[(Sh2)] Zu jeder Familie $(s_{i})_{i} \in \prod_{i}\sheaf{F}(U_{i})$ mit $s_{i}|_{U_{i} \cap U_{j}} = s_{j}|_{U_{i} \cap U_{j}}$ existiert ein $s \in \sheaf{F}(U)$ mit $s|_{U_{i}} = s_{i}$.
\end{enumerate}
$(ii)$ Ein \textbf{Morphismus von Garben} ist ein Morphismus der unterliegenden Prägarben.\\
\\
Wir erhalten die Kategorie $\psh_{X}(\set)$ der Mengenwertigen Prägarben auf $X$ und die \underline{volle} Unterkategorie $\sh_{X}(\set)$ der Mengenwertigen Garben auf $X$.
Analog erhalten wir Garben von abelschen Gruppen, Ringen, $R$-Moduln und $R$-Algebren.
\end{defn}

\begin{rem}
\label{rem:pathologien-garben}
\begin{enumerate}
	\item $\sheaf{F} \in \sh \implies \Gamma(\emptyset, \sheaf{F})$ ist einpunktig (wegen $(Sh)$ für die leere Überdeckung)
	\item $X = \{ pt \} \implies \sheaf{F}$ auf $X$ ist eindeutig durch $\sheaf{F}(X)$ bestimmt
\end{enumerate}
\end{rem}

%% TODO: Insert Abschnitt zu Limiten hier

\begin{example}
\label{bsp:beispiele-von-garben}	
\begin{enumerate}
	\item $\sheaf{F} \in \psh_{X}, U \subseteq X$ offen $\implies$ $\sheaf{F}|_{U} \in \psh_{U}$ mit $\Gamma(V, \sheaf{F}|_{U}) := \Gamma(V, \sheaf{F})$. Ist $\sheaf{F} \in \sh_{X}$, so ist $\sheaf{F}|_{U} \in \sh_{U}$. 
	\item Für $X,Y$ top. Räume definiert $\sheaf{F}$ gegeben durch $\Gamma(U, \sheaf{F}) := \mathcal{C}(U, Y) = \{ f : U \to Y \mid \ f \text{ stetig}\}$ und $res^{U}_{V} : f \mapsto f|_{V}$ eine Garbe.
	\item $k$ ein Körper, $(X,\sheaf{O}_X)$ ein Raum mit Funktionen$/_k$ $\implies$ $\sheaf{O}_{X}$ ist Garbe von $k$-Algebren auf $X$.
	\item Für einen top. Raum $X$ definiert $\sheaf{F}(U) := \{ f: U \to \mathbb R \mid f \text{ stetig und beschränkt}\}$ eine Prägarbe auf $X$, im Allgemeinen aber keine Garbe.
\end{enumerate}
\end{example}

Sei $\mathcal{B}$ eine Basis der Topologie von $X$ und $\sheaf{F} \in \sh_{X}$. Sei für $V \subseteq X$ offen $\mathcal{B}_{V} := \{ U \in \mathcal{B} \mid U \subseteq V\}$. Dann folgt wegen $(Sh)$:\\
\[
\sheaf{F}(V) \cong_{(\dagger)} \{ (s_{U})_{U} \in \prod_{U \in \mathcal{B}_{V}} \sheaf{F}(U) \mid \forall U' \subseteq U \in \mathcal{B}_{V}: s_{U}|_{U'} = s_{U'}\} \cong \varprojlim_{U \in \mathcal{B}_{V}} \sheaf{F}(U)
\]d.h. $\sheaf{F}$ ist bereits eindeutig durch die Schnitte auf einer Basis von $X$ bestimmt.\\
$(\dagger):$ einfache Folgerung aus $(Sh1)$.\\
\\
Eine \textbf{Prägarbe auf } $\mathcal{B}$ ist ein kontravarianter Funktor $\sheaf{F} : \mathcal{B} \to \set$. Jedes solche $\sheaf{F}$ induziert eine Prägarbe $\overline{\sheaf{F}}^{X}$ auf $X$ durch $\overline{\sheaf{F}}^{X}(V) := \varprojlim_{U \in \mathcal{B}_{V}} \sheaf{F}(U)$.
Für $U \in \mathcal{B}$ gilt dann $\overline{\sheaf{F}}^{X}(U) = \varprojlim_{U' \in \mathcal{B}_{U}}\sheaf{F}(U') = \sheaf{F}(U)$, da $U$ initial in $\mathcal{B}_{U}$.\\ Ein \textbf{Morphismus von Prägarben auf } $\mathcal{B}$ ist wieder ein Morphismus von Funktoren.

\begin{prop}
\label{prop:garbe-auf-basis}
$\overline{\sheaf{F}}^{X}$ ist eine Garbe $\iff \sheaf{F}$ erfüllt $(Sh)$  für alle $U \in \mathcal{B}$ und Überdeckungen $U = \bigcup_{i} U_{i}$ mit $U_{i} \in \mathcal{B}$.\\
In diesem Fall heißt $\sheaf{F}$ \textbf{Garbe auf} $\mathcal{B}$. 
\end{prop}
\begin{proof}
	Im Folgenden schreiben wir $\overline{\sheaf{F}} := \overline{\sheaf{F}}^{X}$.
\begin{itemize}
	\item [,,$\Rightarrow$``:] $\overline{\sheaf{F}}^{X}(U) = \sheaf{F}(U)$ für alle $U \in \mathcal{B}$.
	\item [,,$\Leftarrow$``:] Sei $U \subseteq X$ offen und $U = \bigcup_{i} U_{i}$ eine offene Überdeckung von $U$ in $X$.\\
	\underline{$(Sh1)$}:
	\[
	\xymatrix {
		\varprojlim_{B \in \mathcal{B}_{U}} \sheaf{F}(B) = \overline{\sheaf{F}}(U)\ar@{^{(}->}[r] & \prod_{i} \overline{\sheaf{F}}(U_{i}) \ar@{^{(}->}[r]  & \prod_{i} \prod_{B \in \mathcal{B}_{U_{i}}} \sheaf{F}(B)\\
		s = (s_{B})_{B}, s' = (s'_{B})_{B} \ar@{|->}[r] & s|_{U_i} = s'|_{U_i} \ar@{|->}[r] & ( (s_{B})_{B \in \mathcal{B}_{U_{i}}} )_{i} = ( (s'_{B})_{B \in \mathcal{B}_{U_{i}}} )_{i} \ \ \ \ \ (\dagger)
	}
	\]
	Behauptung: $\forall B \in \mathcal{B}_{U}: \ s_{B} = s'_{B}$, d.h. $s = s'$. \underline{denn:} \\
	\[
	\xymatrix {
	  \sheaf{F}(B) \ar@{^{(}->}[r] & \prod_{i} \prod_{B' \in \mathcal{B}_{U_i \cap B}} \sheaf{F}(B') 
    } \]  ist injektiv nach $(Sh1)$ für $\sheaf{F}$ auf $\mathcal{B}$.\\
    $s_{B}$ und $s'_{B}$ haben gleiches Bild wegen $(\dagger)$.\\
    \underline{$(Sh2)$}:
    \[ \xymatrix{
    \overline{\sheaf{F}}(U) \ar^{\rho}@{->}[r] & \prod_{i} \overline{\sheaf{F}}(U_{i}) \ar@{->}[r] & \prod_{(i, j)} \overline{\sheaf{F}}(U_{i} \cap U_{j})
    }\]
    \begin{itemize}
    	\item[,,$\overline{\sheaf{F}}(U) \subseteq \ker$``]: $\rho(s) = ((s_{B})_{B \in \mathcal{B}_{U_{i}}})_{i}$. Sei $T \in \mathcal{B}_{U_{i} \cap U_{j}} \subseteq \mathcal{B}_{U_{i}}$, $V \in \mathcal{B}_{U_{i}}$ und $W \in \mathcal{B}_{U_{j}}$. Dann folgt: \\
    	$s_{V}|_{T} = s_{T} = s_{W}|_{T} \implies$ Behauptung.
    	\item[,,$\overline{\sheaf{F}}(U)\supseteq \ker$``]: Sei $(s_{i})_{i} \in \prod_{i} \overline{\sheaf{F}}(U_{i})$ mit $b((s_{i})_{i}) = b'((s_{i})_{i})$.\\
    	\underline{Gesucht:} $s = (s_B)_{B} \in \overline{\sheaf{F}}(U)$ mit $s|_{U_{i}} = s_{i}$.\\
    	Es gilt:\\
    \[\xymatrixcolsep{3pc}\xymatrix{
     \sheaf{F}(B) \ar^-\rho@{->}[r] & \prod_{i} \prod_{B' \in \mathcal{B}_{U_{i} \cap B}} \sheaf{F}(B') \ar^-b@<+.5ex>[r] \ar_-{b'}@<-.5ex>[r] & \prod_{(i, j)} \prod_{\mathcal{B}_{U_{i} \cap B} \times \mathcal{B}_{U_{j} \cap B}} \sheaf{F}(V \cap W)
    }\] ist exakt, d.h. ein Equalizer-Diagramm. Konstruiere damit $s_{B} \in \sheaf{F}(B)$, welche kompatibles System bilden. $s = (s_B)_{B}$ ist dann das gesuchte Element in $\overline{\sheaf{F}}(U)$. 	
    \end{itemize}
\end{itemize}
\end{proof}


\section{Halme von Garben}
\label{sec:halme}

Für $x \in X$ und $\sheaf{F} \in \psh_{X}$ ist $(\sheaf{F}(V), res^{V}_{U})_{x \in U \subseteq X \text{offen}}$ ein \underline{filtriertes} induktives System.\\
\underline{filtriert}:\\
$\forall U,V \subseteq X$ offen $\exists W \subseteq U,V$ offen. (z.B. $W = U \cap V$).\\

\begin{defn}
\label{def:halm}
Der induktive Limes (oder auch Colimes) $\sheaf{F}_{x} := \varinjlim_{x \in U} \sheaf{F}(U)$ heißt \textbf{Halm} von $\sheaf{F}$ in $x$. Für $x \in U \subseteq X$ offen hat man einen kanonischen Morphismus $\pi_{U} : \sheaf{F}(U) \to \sheaf{F}_{x}$. Das Bild eines Schnittes $s \in \sheaf{F}(U)$ unter $\pi_{U}$ heißt \textbf{Keim} von $s$ in $x$ und wird mit $s_{x}$ bezeichnet.\\
\\
Ein Morphismus von Prägarben $\varphi: \sheaf{F} \to \sheaf{G}$ induziert eine Abbildung $\varphi_{x} = \varinjlim_{x \in U} \varphi_{U} : \sheaf{F}_{x} \to \sheaf{G}_{x}$ von Halmen in $x$.
\end{defn}

\begin{example}
\label{bsp:einige-halme}
$z_0 \in X := \CC$, $\sheaf{O}_{\CC}$: Garbe der holomorphen Funktionen auf $\CC$. Dann gilt: $(U, f) \sim (V,g) \iff$ $f$ und $g$ haben dieselbe Taylor-Entwicklung um $z_0$. \\$\implies$ $\sheaf{O}_{\CC, z_0} = \CC\{\{z_0 \}\}$ ist der Ring der Potenzreihen um $z_0$ mit positivem Konvergenzradius.
\end{example}

\begin{prop}
\label{prop:charakterisierung-morphismen-halme}
Seien $X$ ein top. Raum, $\sheaf{F}, \sheaf{G} \in \psh_{X}$ und $\xymatrix{\sheaf{F} \ar^{\varphi}@<+.5ex>[r] \ar_{\psi}@<-.5ex>[r] & \sheaf{G}}$ zwei Morphismen.
\begin{enumerate}
	\item[(1)] Ist $\sheaf{F}$ eine Garbe, so gilt $\varphi_{x} : \sheaf{F}_{x} \to \sheaf{G}_{x}$ ist injektiv für alle $x \in X \iff \varphi_{U} : \sheaf{F}(U) \to \sheaf{G}(U)$ ist injektiv für alle $U \subseteq X$ offen.
	\item[(2)] Sind $\sheaf{F}$ und $\sheaf{G}$ Garben, so gilt:
	\begin{enumerate}
		\item[(a)] $\varphi_{x}$ ist bijektiv für alle $x \in X \iff \varphi_{U}$ ist bijektiv für alle $U \subseteq X$ offen.
		\item[(b)] $\varphi = \psi \iff \varphi_{x} = \psi_{x}$ für alle $x \in X$.
	\end{enumerate}
\end{enumerate}
\end{prop}
\begin{proof}
Für $U \subseteq X$ offen ist
\[
\xymatrix{
  \sheaf{F}(U) \ar@{^{(}->}[r] & \prod_{x \in U} \sheaf{F}_{x}\\
  s \ar@{|->}[r] & (s_{x})_{x \in U}
}
\] injektiv, \underline{denn}:\\
Seien $s,t \in \sheaf{F}(U)$ mit $s_x = t_x$ für alle $x \in U$. Dann gibt es für jedes $x \in U$ eine offene Umgebung $x \in V_x \subseteq U$ s.d. $s|_{V_x} = t|_{V_x}$.\\
$(Sh1) \implies s = t$.\\
Wir erhalten ein kommutatives Diagramm \\
\[
\xymatrix
{
\sheaf{F}(U) \ar@{^{(}->}[r] \ar^{\varphi_U}@{->}[d] & \prod_{x \in U} \sheaf{F}_{x} \ar^{\prod_{x}\varphi_x}@{->}[d] \\
\sheaf{G}(U) \ar@{->}[r] & \prod_{x \in U} \sheaf{G}_{x}
}
\] welches $``(1) \Rightarrow``$ und $(2)(b)$ impliziert.\\
\\
\\
Allgemein gilt: 
\begin{enumerate}
	\item[$(i)$] Filtrierte Colimiten injektiver Abbildungen sind injektiv, d.h. $``(1) \Leftarrow ``$ gilt.
    \item[$(ii)$] Colimiten surjektiver Abbildungen sind surjektiv, d.h. $``(2)(a)\Leftarrow ``$ gilt
\end{enumerate}
Zu $``(2)(a)\Rightarrow ``$: reicht z.z.: Bijektivität von $\varphi_x$ impliziert Surjektivität von $\varphi_U$.\\
Sei dazu $t \in \sheaf{G}(U)$. Wähle für alle $x \in U$ eine offene Umgebung $x \in U^{x} \subseteq U$ und $s^{x} \in \sheaf{F}(U^x)$ so dass $(\varphi_{U^x}(s^x))_x = t_x$.\\
$\implies \exists x \in V^x \subseteq U^x$ offen mit $\varphi_{V^x}(s^x|_{V^x}) = t|_{V^x}$. \\
Da $U = \bigcup_x{V^x}$ offene Überdeckung, gilt für alle $x,y \in U$:\\
\[
\varphi_{V^y \cap V^x}(s^x|_{V^y \cap V^y}) = t|_{V^y \cap V^x} = \varphi_{V^y \cap V^x}(s^y|_{V^y \cap V^x})
\]
$\varphi_{U}$ injektiv nach $(1)$ $\implies$ $s^x|_{V^y \cap V^y} = s^y|_{V^y \cap V^y}$.\\
$(Sh2) \implies \exists s \in \sheaf{F}(U)$ mit $s|_{V^x} = s^x$ für alle $x \in U$. \\
$\implies$ $\varphi_U(s)_x = [(V^x, \varphi_{V^x}(s|_{V^x}))] = [(V^x, t|_{V^x})] = t_x \implies \varphi_{U}(s) = t$.
\end{proof}

\begin{defn}
\label{def:injektive-und-surjektive-garbenmorphismen}
Ein Morphismus $\sheaf{F} \to \sheaf{G}$ von Garben heißt \textbf{injektiv/ surjektiv/ bijektiv} $:\iff$ $\forall x \in X: \sheaf{F}_x \to \sheaf{G}_x$ ist injektiv/ surjektiv/ bijektiv.
\end{defn}

\begin{rem}
\label{rem:charakterisierung-surjektiv}
$\varphi : \sheaf{F} \to \sheaf{G}$ ist surjektiv gdw. für alle $t \in \sheaf{F}(U)$ eine offene Überdeckung $U = \bigcup_i U_i$ existiert und $s_i \in \sheaf{F}(U_i)$ s.d. $\varphi_{U_i}(s_i) = t|_{U_i}$, d.h. \underline{lokal} findet man stets ein Urbild.\\
\textbf{Warnung:} Aus $\varphi$ surjektiv folgt nicht $\varphi_U$ surjektiv für alle $U \subseteq X$ offen.	
\end{rem}
\section{Die zu einer Prägarbe assoziierte Garbe}
\label{sec:vergarbung}

\begin{defn}
\label{def:vergarbung}
Sei $\sheaf{F}$ eine Prägarbe auf einem top. Raum $X$. Eine \textbf{Vergarbung} (auch Garbifizierung/ assoziierte Garbe) von $\sheaf{F}$ ist eine Garbe $\sheaf{F}^{sh}$ auf $X$ zusammen mit einem Morphismus $\iota : \sheaf{F} \to V(\sheaf{F}^{sh})$ von Prägarben, so dass gilt:\\
\[\xymatrix{  
\Mor_{\psh_{X}}(\sheaf{F}, V(\sheaf{G})) \ar^-\cong@{->}[r] & \Mor_{\sh_{X}}(\sheaf{F}^{sh}, \sheaf{G})  \\
\varphi \circ \iota & \ar@{|->}[l] \varphi
}\] Hierbei bezeichne $V : \sh_{X} \to \psh_{X}$ den Vergissfunktor.\\
Durch diese Eigenschaft ist $(\sheaf{F}^{sh}, \iota)$ eindeutig bis auf eindeutigen Isomorphismus bestimmt.\\
Ferner gilt:
\begin{enumerate}
	\item[(0)] Es existiert eine Vergarbung $\iota : \sheaf{F} \to \sheaf{F}^{sh}$
	\item[(1)] $\iota$ wie oben induziert einen Isomorphismus $\iota_x : \sheaf{F}_x \to \sheaf{F}^{sh}_x$ auf Halmen für alle $x\in X$.
	\item[(2)] Für jede Prägarbe $\sheaf{G}$ auf $X$ und jeden Morphismus $\varphi: \sheaf{F} \to \sheaf{G}$ existiert genau ein Morphismus $\varphi^{sh}: \sheaf{F}^{sh} \to \sheaf{G}^{sh}$ s.d. folgendes Diagramm kommutiert:
	\[
	\xymatrix
	{
	\sheaf{F} \ar^{\iota_{\sheaf{F}}}@{->}[r] \ar^{\varphi}@{->}[d] & \sheaf{F}^{sh} \ar^{\varphi^{sh}}@{->}[d] \\
	\sheaf{G} \ar^{\iota_{\sheaf{G}}}@{->}[r] & \sheaf{G}^{sh}
    }
	\] d.h. $\psh_{X} \to \sh_{X}, \sheaf{F} \mapsto \sheaf{F}^{sh}$ ist ein Funktor, linksadjungiert zum Vergissfunktor $V$.
\end{enumerate}
\end{defn}
\begin{proof}
\underline{Existenz}:\\
$\sheaf{F}^{sh}(U) := \{ (s_x)_{x} \in \prod_{x \in U} \sheaf{F}_{x} \mid \forall x \in U: \ \exists x \in U^x \subseteq U \text{ offen und } t \in \sheaf{F}(U^x): \ \forall y \in U^x: t_x = s_x \}$\\
``Keime, die lokal Schnitte von $\sheaf{F}$ sind`` - $(Sh2)$ erzwingt dies.\\
Für $U \subseteq V$ ist $res^V_U$ induziert von:\\
\[
\xymatrix
{
\sheaf{F}^{sh}(V) \ar^-{res^V_U}@{-->}[r] \ar@{^{(}->}[d] & \sheaf{F}^{sh}(U) \ar@{^{(}->}[d] \\
\prod_{x\in V} \sheaf{F}_x \ar^-{proj.}@{->}[r] & \prod_{x\in U}\sheaf{F}_x
}
\]
\end{proof}
\section{Direktes und inverses Bild von Garben}
\label{sec:garben-direktes-inverses-bild}

Sei $f:X\rightarrow Y$ stetige Abbildung topologischer Räume, $\mathcal{F}$
eine Prägarbe auf $X$. Ziel: $f_{\ast}\mathcal{F}$ Prägarbe auf
$Y$, das direkte Bild von $\mathcal{F}$ unter $f$. Definiere $(f_{\ast}\mathcal{F})(V):=\mathcal{F}(f^{-1}(V))$
mit Restriktionsabbildung von $\mathcal{F}$ ($V_{1}\subseteq V_{2}:$
$s\in f_{\ast}\mathcal{F}(V_{2})\rightarrow s|_{V_{1}}=\mathcal{F}res_{f^{-1}(V_{1})}^{f^{-1}(V_{2})}$).
\begin{align*}
  f_{\ast}:PSh(X) & \longrightarrow PSh(Y)\\
  \mathcal{F} & \longmapsto f_{\ast}\mathcal{F}\\
  \mathcal{F}\overset{\varphi}{\rightarrow}\mathcal{G} & \longmapsto f_{\ast}(U):f_{\ast}\mathcal{F}\rightarrow f_{\ast}\mathcal{G}
\end{align*}

ist Funktor via $(f_{\ast}\varphi)_{V}=\varphi_{f^{-1}(V)}$.
\begin{rem}[28]
  \mbox{}
  \begin{enumerate}
  \item $\mathcal{F}$ Garbe auf $X$ $\Longrightarrow f_{\ast}\mathcal{F}$
    Garbe auf $X$, d.h. $f_{\ast}:Sh(X)\rightarrow Sh(Y)$.
  \item Ist $g:Y\rightarrow Z$ eine weitere stetige Abbildung topologischer
    Räume, so existiert ein offensichtlicher Isomorphismus $g_{\ast}\circ(f_{\ast}\mathcal{F})=(g\circ f)_{\ast}\mathcal{F}$,
    funktoriell in $\mathcal{F}$.
  \end{enumerate}
  \medskip{}
\end{rem}

Nun sei $\mathcal{G}$ eine Prägarbe auf $Y$.

\textbf{Ziel:} Definiere $f^{+}\mathcal{G}$ Prägarbe auf $X$. $f^{-1}\mathcal{G}=\widetilde{f^{+}\mathcal{G}}$
Garbe auf $X$, \textbf{Inverses Bild zu $\mathcal{G}$ unter $f$}
via 
\[
(f^{+}\mathcal{G})(U):=\underset{\underset{Y\supseteq V\supseteq f(U)}{\longrightarrow}}{\lim}\mathcal{G}(V)
\]

mit induzierte Restriktionsabbildung.\medskip{}

\textbf{Warnung:} $\mathcal{G}$ Garbe auf $Y$ $\leadsto f^{+}\mathcal{G}$
im Allgemeinen keine Garbe auf $X$. Falls $f:X\hookrightarrow Y$
Inklusion, $\mathcal{G}|_{X}:=f^{-1}\mathcal{G}$. Ist $X\subseteq Y$
offen stimmt $\mathcal{G}|_{X}$ mit der Einschränkung aus Beispiel
19 überein (cofinales Objekt). $\leadsto f^{-1}:PSh(Y)\rightarrow Sh(X)$
Funktor.

$g:Y\xrightarrow{\text{stetig}}Z$, $\mathcal{H}$ Prägarbe auf $Z$,
$U\subseteq X$ offen. 
\[
Z\underset{\text{offen}}{\supseteq}W\supseteq g(f(U))\Longleftrightarrow W\supseteq g(V)
\]

für ein $f(U)\subseteq V\subseteq Y$ offen. 
\begin{align*}
  \underset{\longrightarrow}{\lim}\underset{\longrightarrow}{\lim}=\underset{\longrightarrow}{\lim}\Longrightarrow f^{+}(g^{+}\mathcal{H}) & =(g\circ f)^{+}\mathcal{H}\quad(*)\\
  \Longrightarrow f^{-1}(g^{-1}\mathcal{H}) & =(g\circ f)^{-1}\mathcal{H}
\end{align*}

\begin{example}
  $\imath:\{x\}\rightarrow X$ Inklusion, $\mathcal{F}$ Prägarbe auf
  $X$. $\Longrightarrow\imath^{-1}(\mathcal{F})=\mathcal{F}_{x}$ per
  Definition. $(\ast)\Longrightarrow$
  \[
  \begin{array}{ccc}
    (f^{-1}\mathcal{G})_{x} & = & \mathcal{G}_{f(x)}\\
    \shortparallel &  & \shortparallel\\
    \imath^{-1}\circ(f^{-1}\mathcal{G}) & = & (f\circ\imath)^{-1}\mathcal{G}
  \end{array}
  \]
\end{example}

\begin{prop}[29]
  Für $f:X\rightarrow Y$ stetig sind die Funktionen $f_{\ast}$ und
  $f^{-1}$ zueinander adjungiert, d.h. für $\mathcal{F}$ Garbe auf
  $X$, $\mathcal{G}$ Prägarbe auf $Y$ existiert eine bijektion
  \begin{align*}
    \hom_{Sh(x)}(f^{-1}\mathcal{G},\mathcal{F}) & \longleftrightarrow\hom_{Psh(Y)}(\mathcal{G},f_{\ast}\mathcal{F})\\
    \varphi & \longmapsto\varphi^{\flat}\\
    \psi^{\sharp} & \longmapsfrom\psi
  \end{align*}

  funktoriell in $\mathcal{F}$ und $\mathcal{G}$.
\end{prop}

\begin{proof}
  $\varphi:f^{-1}\mathcal{G}\rightarrow\mathcal{F}$ Morphismus von
  Garben auf $X$. $t\in\mathcal{G}(V)$, $V\subseteq Y$ offen
  \begin{align*}
    \mathcal{G}(V) & \rightarrow f^{+}\mathcal{G}(f^{-1}(V))\xrightarrow{\imath_{f^{+}\mathcal{G}}}f^{-1}\mathcal{G}(f^{-1}(V))\xrightarrow{\varphi_{f^{-1}(V)}}\mathcal{F}(f^{-1}(V))=f_{\ast}\mathcal{F}(V)\\
    & \phantom{\rightarrow\ }\shortparallel\underset{\underset{Y\supseteq W\supseteq ff^{-1}(V)\subseteq V}{\longrightarrow}}{\lim}\\
    t & \mapsto\varphi_{V}^{\flat}(t)
  \end{align*}

  Definition von $\psi^{\#}$. $\mathcal{G}\xrightarrow{\psi}f_{\ast}\mathcal{F}$
  Morphismus von Prägarben auf . Wir definieren $\psi^{\#}:f^{+}\mathcal{G}\rightarrow\mathcal{F}$,
  welches dann $\psi^{\#}:f^{-1}\mathcal{G}\rightarrow\mathcal{F}$
  induziert. $U\subseteq X$ offen, $S\subseteq f^{+}\mathcal{G}(U)$,
  $s=[(V,s_{V})]$, $V\supseteq f(U)$, $s_{V}\in\mathcal{G}(V)$. $\Longrightarrow f^{-1}(V)\supseteq U$.
  \[
  \xymatrix{\psi_{V}(s_{V})\in f_{\ast}\mathcal{F}(V)\ar@{=}[r]\ar@{|->}[rd] & \mathcal{F}(f^{-1}(V))\ar[d]\\
    & \psi_{U}^{\#}(s)\in\mathcal{F}(U)
  }
  \]

  Überprüfe $\varphi^{\flat^{\#}}=\varphi$, $\psi^{\#^{\flat}}=\mathcal{H}$
  und Funktoriell.
\end{proof}
Definition + Proposition 29 verallgemeinern sich zu (Prä)Garben von
Ringen, $R$-Moduln, $R$-Algebren.

\textbf{Beschreibung} von:
\[
\mathcal{G}_{f(x)}=(f^{-1}\mathcal{G})_{x}\overset{\varphi_{x}}{\longmapsto}\mathcal{F}_{x},\ x\in X
\]

\[
\xymatrix{f(x)\in U\underset{\text{offen}}{\subseteq}Y & \mathcal{G}(U)\ar[r]^{\varphi_{U}^{\flat}}\ar@{-->}[d] & \mathcal{F}(f^{-1}(U))\ar[r] & \mathcal{F}_{x}\\
  \underset{\underset{U}{\longrightarrow}}{\lim} & \mathcal{G}_{f(x)}\ar@{-->}[rru]
}
\]

\section{Lokal geringte Räume}
\label{sec:lokal-geringte-raeume}
\begin{defn}
  Ein geringter Raum ist ein Paar $(X,\mathcal{O}_{X})$ bestehend aus
  einem topologischen Raum $X$ und einer Garbe $\mathcal{O}_{X}$ (kommutativer)
  Ringe. Ein Morphismus $(X,\mathcal{O}_{X})\rightarrow(Y,\mathcal{O}_{Y})$
  geringter Räume ist wiederum ein Paar $(f,f^{\flat})$ bestehend aus
  einer stetigen Abbildung $f:X\rightarrow Y$ und einem Homomorphismus
  $f^{\flat}:\mathcal{O}_{Y}\rightarrow f_{\ast}\mathcal{O}_{X}$ von
  Ringgarben auf $Y$. Dieses Datum ist gleichbedeutend (Proposition
  29) mit $(f,f^{\sharp})$, wobei nun $f^{\sharp}:f^{-1}\mathcal{O}_{Y}\rightarrow\mathcal{O}_{X}$
  ein Garbenhomomorphismus auf $X$ ist.

  Bezeichne: $f$ oder $(f,f^{\flat})$ oder $(f,f^{\sharp})$. Damit
  haben wir eine \textbf{Kategorie der geringten Räume}. $\mathcal{O}_{X}$
  heißt Strukturgarbe\index{Strukturgarbe} von $X$, oft schreiben
  wir $X$ für $\mathcal{O}_{X}$.

  Idee: $\mathcal{O}_{X}$ beschreibt die zulässigen Funktionen auf
  $U\subset X$, d.h. etwa stetige, differenzierbare, holomorphe, rigid
  analytische usw. Funktionen. Solche Funktionen auf $V\subset Y$ sollen
  beim ``Zurückziehen'' unter $f$ in dieselbe Klasse überführt werden.
  Dies wird formal durch das Datum $f^{\flat}$ sichergestellt.
\end{defn}

\begin{notation*}
  Wenn $A$ ein lokaler Ring ist, $\mathfrak{m}_{A}$ das maximale Ideal,
  und $\kappa(A)=A/\mathfrak{m}_{A}$ Restklassenkörper. Ein Homomorphismus
  $\varphi:A\rightarrow B$ lokaler Ringe heißt \textbf{lokal}, falls
  $\varphi(\mathfrak{m}_{A})\subset\mathfrak{m}_{B}$. $(f,f^{\flat})=(f,f^{\sharp})=X\rightarrow Y$
  Morphismus geringter Räume induziert:
  \begin{align*}
    & \mathcal{O}_{Y,f(x)}=(f^{-1}\mathcal{O}_{Y})_{x}\xrightarrow{f_{x}^{\sharp}}\mathcal{O}_{X,x}\\
    \text{oder } & \xymatrix{\mathcal{O}_{Y}(U)\ar[r]^{f_{U}^{\flat}}\ar[d] & \mathcal{O}_{X}(f^{-1}(U))\ar[d] & f(x)\in U\subset Y\text{ offen}\\
      \mathcal{O}_{Y,f(x)}=\lim\mathcal{O}_{Y}(U)\ar@{-->}[r] & \mathcal{O}_{X,x}
    }
  \end{align*}
\end{notation*}
\begin{defn}[orig. 31]
  Ein lokal geringter Raum ist ein geringter Raum $(X,\mathcal{O}_{X})$,
  für der $\mathcal{O}_{X,x}$ für alle $x\in X$ ein \emph{lokaler
    Ring} ist. Ein Morphismus $(X,\mathcal{O}_{X})\rightarrow(Y,\mathcal{O}_{Y})$
  lokal geringter Räume ist ein Morphismus geringter Räume $(f,f^{\flat})$,
  so dass die induzierte Abbildung 
  \[
  f_{x}^{\sharp}:\mathcal{O}_{Y,f(x)}\rightarrow\mathcal{O}_{X,x}
  \]

  ein lokaler Ringhomomorphismus ist für alle $x\in X$. Dies führt
  zu einer Unterkategorie der Kategorie geringter Räume, die im Allgemeinen
  \emph{nicht }voll ist, d.h. es gibt Morphismen $f$ geringter Räume
  zwischen lokal geringten Räume, die nicht lokal sind!
\end{defn}

Bezeichne: 
\begin{itemize}
\item $\mathcal{O}_{X,x}$ der ``lokale Ring von $X$ in $x$'';
\item $\mathfrak{m}_{x}$ maximales Ideal;
\item $\kappa(x):=\mathcal{O}_{X,x}/\mathfrak{m}_{x}$ Restklassenkörper
  (bei $x$). \texttt{
  \[
  \xymatrix@R=0pt{\mathcal{O}_{X}(U)\ar[r] & \mathcal{O}_{X,x}\ar[r] & \kappa(x)\\
    f\ar[rr] &  & f(x)
  }
  \]
}
\end{itemize}
Warum \textbf{lokal} geringte Räume? Heuristik:
\begin{align*}
  \mathcal{O}_{X}(U) & \leftrightarrow\text{Funktionen auf }U\\
  \mathcal{O}_{X,x} & \leftrightarrow\text{Funktionen auf Umgebung }U\text{ von }x
\end{align*}

\emph{Wunsch}: $f(x)\neq0\overset{!}{\Rightarrow}f$ ist invertierbar
auf einer kleinen Umgebung $V$ von $x$, d.h. 
\[
\mathcal{O}_{X,x}\backslash\underbrace{\{f\mid f(x)=0\}}_{=\mathfrak{m}_{x}}\subset\mathcal{O}_{X,x}^{\times},
\]

also $\mathcal{O}_{X,x}$ lokal. \emph{Ferner}: $g\mathcal{O}_{Y,f(x)}$
mit $g(f(x))=0$ sollte implizieren: $(g\circ f)(x)=0$. Übersetzt:
\[
f_{x}^{\sharp}(\mathfrak{m}_{f(x)})\subset\mathfrak{m}_{x},\quad f_{x}^{\sharp}(g)="g\circ f"
\]

\begin{example}[orig. 32]
  $\varphi_{X}$ Garbe der $\mathbb{R}$-wertiger stetiger Funktionen
  auf einem topologischen Raum $X$. $\varphi_{X,x}$ Ring der Keime
  $[s]$ stetiger Funktionen in einer Umgebung von $X$:
  \[
  \mathfrak{m}_{x}=\{[s]\in\varphi_{X,x}\mid0=s(x)\}
  \]

  ist einziges maximale Ideal, d.h. $(X,\varphi_{X})$ ist lokal geringter
  Raum. 

  \emph{Denn}: Sei $[s]\in\varphi_{X,x}\backslash\mathfrak{m}_{x}$
  gegeben.

  $\Rightarrow s(x)\neq0$ für alle $s\in[s]$. 

  $\Rightarrow(s$ stetig) $\exists x\in U\subset X$ offen mit $s(u)\neq0$
  für alle $u\in U$.

  $\Rightarrow\frac{1}{s|_{U}}\in\varphi_{X}(U)$ existiert.

  $\Rightarrow\varphi_{X,x}\backslash\mathfrak{m}_{x}=\varphi_{X,x}^{\times}$
  Einheitengruppe. Es ist: 
  \[
  \varphi_{X,x}\rightarrow\mathbb{R},\ [s]\mapsto s(x)
  \]

  ein surjektiver Ringhomomorphismus mit $\ker=\mathfrak{m}_{x}$.

  $\Rightarrow\kappa(x)\cong\mathbb{R}$. Sei $f:X\rightarrow Y$ stetig,
  $V\subset Y$ offen.
  \begin{align*}
    f_{x}^{\flat}:\varphi_{Y}(V) & \longrightarrow\varphi_{X}(f^{-1}(V))=f_{\ast}\varphi_{X}(V)\\
    t & \longmapsto t\circ f
  \end{align*}

  Es folgt:
  \begin{align*}
    \varphi_{Y,f(x)} & \longrightarrow\varphi_{X,x}\\{}
           [t] & \longmapsto[t\circ f]
  \end{align*}

  ist ein Morphismus lokal geringter Räume. Ebenso lassen sich Prävarietäten
  über lokal geringte Räume interpretieren!
\end{example}

\chapter*{Das Ringsprektrum als lokal geringter Raum}
\label{chap:ringspektrum-lokal-geringter-raum}
Ziel: volltreuer Funktor
\begin{align*}
  \text{Ringe} & \longrightarrow\text{Kategorie lokal geringter Räume}\\
  A & \longmapsto(\Spec A,\mathcal{O}_{\Spec A})
\end{align*}

\section{Die Strukturgarbe auf Spec A}
\label{sec:strukturgarbe-auf-spec-A}

Sei $X:=\Spec(A)$, $\mathcal{B}=\{D(f)\mid f\in A\}$ Basis der Topologie.

\textbf{Vorgegeben:} Definiere Prägarbe $\mathcal{O}_{X}$ auf $\mathcal{B}$,
die Garbenaxiome bzgl. $\mathcal{B}$ erfüllt.

\textbf{Wähle:} $\mathcal{O}_{X}(X)=A$ (vgl. Prävarietäten) bzw.
$\mathcal{O}_{X}(D(f))=A_{f}$, da
\begin{align*}
  \imath_{j}:A & \longrightarrow A_{f}\\
  a & \longmapsto\frac{a}{1}
\end{align*}

einen Homöomorphismus $D(f)\xrightarrow{\sim}\Spec(A_{f})$ induziert.
(``Funktionen mit möglichen Polen in $V(f)$).

\subsection{Wohldefiniertheit}
\label{subsec:strukturgarbe-wohldefiniertheit}

$D(f)=D(g)\Rightarrow A_{f}=A_{g}$ kanonisch. Dazu: 
\begin{align*}
  D(f)\subset D(g) & \Leftrightarrow\exists n\geq1\text{ d.d. }f^{n}\in A_{g}
\end{align*}


\subsection{Induzierte Abbildung}
\label{subsec:strukturgarbe-induzierte-abbildung}

\[
\mathcal{O}_{X}(D(g))\rightarrow\mathcal{O}_{X}(D(f)),\ \rho_{f,g}=:\text{res}_{D(f)}^{D(g)}
\]

Dies definiert eine Prägarbe auf $\mathcal{B}$.
\begin{thm}[orig. 33]
  Die Prägarbe $\mathcal{O}_{X}$ ist eine Garbe auf $\mathcal{B}$.
  Die induzierte Garbe auf $X$ (Proposition 20) werde auch mit $\mathcal{O}_{X}$
  bezeichnet. Da
  \[
  \mathcal{O}_{X,x}:=\underset{\underset{D(f)\ni x}{\longrightarrow}}{\lim}\mathcal{O}_{X}(D(f))=\underset{\underset{f\in\mathfrak{p}_{x}}{\longrightarrow}}{\lim}A_{f}=A_{\mathfrak{p}_{x}}
  \]
  mit $(X,\mathcal{O}_{X})=(\Spec A,\mathcal{O}_{\Spec A})$ (kurz $\Spec A$)
  ein lokal geringter Raum.
\end{thm}

\begin{proof}
  Sei $D(f)=\bigcup_{i\in I}D(f_{i})$ Überdeckung in $\mathcal{B}$.
  Zu zeigen:
  \begin{enumerate}
  \item $s\in\mathcal{O}_{X}(D(f))$ mit $s|_{D(f_{i})}=0$, $i\in I$.

    $\overset{!}{\Rightarrow}s=0$.
  \item $s_{i}\in\mathcal{O}_{X}(D(f_{i}))$, $i\in I$, mit $s_{i}|_{D(f_{i})\cap D(f_{j})}=s_{j}|_{D(f_{i})\cap D(f_{j})}$
    $\forall i,j\in I$.

    $\overset{!}{\Rightarrow}\exists s\in\mathcal{O}_{X}(D(f))$ mit $s|_{D(f_{i})}=s_{i}$
    $\forall i\in I$.
  \end{enumerate}
  Ohne Einschränkung:
  \begin{itemize}
  \item $I$ endlich, da $D(f)$ quasi-kompakt.
  \item $f=1$, $D(f)=X$ (mit $(A_{f},\mathcal{O}_{X}|_{D(f)})$ statt $(A,\mathcal{O}_{X})$
    betrachtet) 
    \[
    X=\bigcup_{i\in I}D(f_{i})\Leftrightarrow(f_{i}\mid i\in I)=A
    \]
    Es folgt: $b_{i}=b_{i}(n)\in A$ d.d. $\sum_{i\in I}b_{i}f_{i}^{n}=1$
    \textbf{Zerlegung der 1}. (z)
  \item[Zu 1.] Sei $s=a\in A$ d.d. $0=\frac{a}{1}\in A_{f}$, $\forall i\in I$.
    $I$ endlich, also $\exists n\geq1$ d.d. $f_{i}^{n}a=0$ $\forall i\in I$.
    Mit $(z)$ folgt
    \[
    a=\left(\sum_{i\in I}b_{i}f_{i}^{n}\right)a=0
    \]
  \item[Zu 2.] $s_{i}=\frac{a_{i}}{f_{i}^{n}}$ für $n$ geeignet, unabhängig von
    $i\in I$ (endlich). Nach Voraussetzung:
    \[
    \frac{a_{i}}{f_{i}^{n}}=\frac{a_{j}}{f_{j}^{n}}\in A_{f_{i}f_{j}},\quad D(f_{i})\cap D(f_{j})=D(f_{i}f_{j})
    \]
    Es folgt: $\exists m\geq1$ d.d. $(f_{i}f_{j})^{m}(f_{j}^{n}a_{i}-f_{i}^{n}a_{j})=0$
    $\forall i,j$.
    \begin{align*}
      \frac{a_{i}}{f_{i}^{n}} & =\frac{f_{i}^{m}a_{i}}{f_{i}^{n+m}}=:\frac{a_{i}'}{f_{i}^{n'}},\quad n'=n+m
    \end{align*}
    Ohne Einschränkung: $f_{j}^{n}a_{i}=f_{i}^{n}a_{j}$ $\forall i,j\in I$,
    ({*}) denn:
    \begin{align*}
      f_{j}^{m+n}f_{i}a_{i} & =f_{i}^{m+n}f_{j}^{m}a_{j}\\
      f_{j}^{n'}a_{i}' & =f_{i}^{n'}a_{j}'
    \end{align*}
    Setze $s:=\sum_{j\in I}b_{j}a_{j}\in A$ ($(z)$). Es folgt:
    \[
    f_{i}^{n}s=f_{i}^{n}\sum b_{i}a_{j}=\sum b_{j}(f_{i}^{n}a_{j})\overset{(*)}{=}\left(\sum b_{i}f_{i}^{n}\right)a_{i}\overset{(z)}{=}a_{i}
    \]
    also $\frac{s}{1}=\frac{a_{i}}{f^{n}}=s_{i}$.
  \end{itemize}
\end{proof}

\end{document}

\section{Der Funktor $A\protect\mapsto(\Spec A,\mathcal{O}_{\Spec A})$}
\begin{defn}[34]
  Ein lokal geringter Raum $(X,\mathcal{O}_{X})$ heißt \textbf{affines
    Schema}, falls ein Ring $A$ existiert d.d
  \[
    (X,\mathcal{O}_{X})\cong(\Spec A,\mathcal{O}_{\Spec A})
  \]

  Ein \textbf{Morphismus affiner Schemata} ist ein Morphismus lokal
  geringter Räume. Bezeichne $\aff$ die Kategorie der affinen Schemata.
  \begin{align*}
    \varphi:A & \longrightarrow B & \text{Ringhom.}\\
    f:X:=\Spec A & \longrightarrow Y:=\Spec A & \text{stetige Abb.}
  \end{align*}
\end{defn}

\textbf{Ziel: }Definiere $(f,f^{\flat}):X\rightarrow Y$ mit $f:=^{a}\varphi$
Morphismus von lokal geringter Räume und
\[
  f_{\Spec A}^{\flat}=\varphi:A=\mathcal{O}_{\Spec A}(\Spec A)\rightarrow f_{\ast}\mathcal{O}_{\Spec B}(\Spec B)=B
\]

Dazu: Für $s\in A$ gilt $f^{-1}(D(s))=D(\varphi(s))$ nach Proposition
2.10. Definiere 
\[
  f_{D(s)}^{\flat}:\mathcal{O}_{Y}(D(s))=A_{s}\rightarrow B_{\varphi(s)}=f_{\ast}\mathcal{O}_{X}(D(s))
\]

als die von $\varphi$ induzierte Abbildung. $f^{\flat}$ ist kompatibel
mit $\res_{D(t)}^{D(s)}$ für prinzipal offene Mengen $D(t)\subseteq D(s)$.
$B$ Basis $\Longrightarrow f^{\flat}:\mathcal{O}_{Y}\rightarrow f_{\ast}\mathcal{O}_{X}$
Homomorphismus von Ringgarben. Für $s=1$ erhalten wir $f_{\Spec A}^{\flat}=\varphi$!

Für $x\in X$ gilt:
\[
  \xymatrix{f^{\sharp}:\mathcal{O}_{Y,f(x)}=A_{\varphi^{-1}(\mathfrak{p}_{x})=\mathfrak{p}_{f(x)}}\ar[r] & B_{\mathfrak{p}_{x}}=\mathcal{O}_{X,x}\\
    A\ar[u]\ar[r]^{\varphi} & B\ar[u]
  }
  \qquad(*)
\]

ist der von $\varphi$ induzierte Homomophismus. $f_{x}^{\sharp}$
is lokal:
\[
  \varphi(\varphi^{-1}(\mathfrak{p}_{x}))\subseteq\mathfrak{p}_{x}
\]

\textbf{Bezeichne}: $^{a}\varphi$ für $\Spec(\varphi)=(f,f^{\flat})$,
$^{a}(\psi\circ\varphi)=^{a}\varphi\circ^{a}\psi$. Wir erhalten einen
kontravarianten Funktor
\[
  \Spec:\underline{Ring}\longrightarrow\aff.
\]

Für $f:(X,\mathcal{O}_{X})\rightarrow(Y,\mathcal{O}_{Y})$ Morphismus
von geringten Räumen erhalten wir einen Ringhomomorphismus
\[
  \Gamma(f):=f_{Y}^{\flat}:\Gamma(Y,\mathcal{O}_{Y})=\mathcal{O}_{Y}(Y)\rightarrow\Gamma(X,\mathcal{O}_{X})=(f_{\ast}\mathcal{O}_{X})(Y)=\mathcal{O}_{X}(X).
\]

So erhalten wir einen kontravarianten Funktor
\[
  \Gamma:\aff\longrightarrow\underline{Ring}.
\]

\begin{thm}[35]
  Die Funktoren $\Spec$ und $\Gamma$ definieren eine Anti-Äquivalenz
  zwischen der Kategorie der Ringe und der Kategorie der affinen Schemata.
\end{thm}

\begin{proof}
  $\Spec$ ist essentiell surjektiv per Definition. $\Gamma\circ\Spec$
  ist isomorph zu $\id_{\underline{Ring}}$ nach Konstruktion.Zu zeigen:
  \[
    \Hom_{\ring}(A,B)\stackrel[\Gamma]{\Spec}{\rightleftharpoons}\Hom_{\aff}(\Spec B,\Spec A)
  \]

  sind zueinander invers. Es fehlt die Verkettung $\Spec\circ\Gamma=\id_{\aff}$.
  Sei $f\in\Hom_{\aff}(\Spec B,\Spec A)$, $\varphi:=\Gamma(f)$, $^{a}\varphi=f$.
  Für $\mathfrak{p}_{x}\in\Spec B=X$ ist $f_{x}^{\sharp}$ der eindeutig
  bestimmte Ringhomomorphismus, welcher das Diagramm
  \[
    \xymatrix{A\ar[r]^{f_{\Spec A}^{\flat}=\Gamma(f)=\varphi}\ar[d]_{\imath_{A}} & B\ar[d]^{\imath_{B}}\supset\imath_{B}^{-1}(\mathfrak{p}_{x}B_{\mathfrak{p}_{x}})=\mathfrak{p}_{x}\\
      A_{\mathfrak{p}_{f(x)}}\ar[r]_{f_{x}^{\#}\text{ lokal}} & B_{\mathfrak{p}_{x}}\underset{\max}{\supset}\mathfrak{p}_{x}B_{\mathfrak{p}_{x}}
    }
    \qquad(**)
  \]

  kommutieren lässt. Es gilt:
  \begin{align*}
    \mathfrak{p}_{f(x)}A_{\mathfrak{p}_{f(x)}} & \supseteq(f_{x}^{\sharp})^{-1}(\mathfrak{p}_{x}B\mathfrak{p}_{x})=\mathfrak{p}_{f(x)}A_{\mathfrak{p}_{f(x)}}\\
    \mathfrak{p}_{f(x)} & =\imath_{A}^{-1}(\mathfrak{p}_{f(x)}A\mathfrak{p}_{f(x)})\subset A
  \end{align*}

  $f_{x}^{\#}$ lokal $\Longrightarrow f_{x}^{\#}(\mathfrak{p}_{f(x)}A_{\mathfrak{p}_{f(x)}})\subset\mathfrak{p}_{x}B_{\mathfrak{p}_{x}}$
  $\Longrightarrow^{a}\varphi=f$ als stetige Abbildung. Wegen $(*)$
  lässt auch $(^{a}\varphi)_{x}^{\#}$ das Diagramm $(**)$ kommutieren.
  Proposition $\Longrightarrow(^{a}\varphi)^{\#}=f^{\#}$.
\end{proof}

\section{Beispiele}
\begin{example}[36, Integritätsbereiche]
  Sei $A$ integer, $K=\Quot(A)$. Sei $X=\Spec A$, $\eta=(0)$. Dann
  ist$\overline{\{\eta\}}=\Spec X$, d.h. jede nicht-leere offene Menge
  $U\subset X$ enthält $\eta$. Es folgt: $\mathcal{O}_{X,y}=A_{(0)}=K$.
  Für alle $f\in A$ gilt nach Definition 
  \[
    \mathcal{O}_{X}(D(f))=A_{f}\subset U.
  \]

  Sei $U\subset X$ beliebig offen. Es folgt:
  \[
    \mathcal{O}_{X}(U)=\underset{\underset{D(f)\subset U}{\longleftarrow}}{\lim}\mathcal{O}_{X}(D(f)=\bigcap_{\underset{D(f)\subset U}{f\in A}}A_{f}\subseteq K.
  \]

  Wie im Beweis von Satz 1.37 ist $a_{F}=\bigcap_{\mathfrak{p}\in D(f)}A_{\mathfrak{p}},$
  also $\mathcal{O}_{X}(U)=\bigcap_{x\in U}\mathcal{O}_{X,x}$.
\end{example}

\begin{example}[37, Prinzipal offene Unterschemata affiner Schemata]
  Sei $X=\Spec A$, $f\in A$. Sei $j:\Spec A_{f}\rightarrow\Spec A$
  induziert von $A\rightarrow A_{f}$. $\Longrightarrow j:\Spec A_{j}\rightarrow D(f)$
  ist Homoömorphismus (Proposition 2.12). Für alle $x\in D(f)$ ist
  $j_{x}^{\#}$ der kanonische Isomorphismus $A_{\mathfrak{p}_{x}}\overset{\cong}{\rightarrow}(A_{f})_{\mathfrak{p}_{x}}$.
  $\Longrightarrow(j,j^{\#})$ induziert einen Isomorphismus $\Spec A_{f}\cong(D(f),\mathcal{O}_{X|D(f)})$.
\end{example}

\begin{example}[38, Abgeschlossene Unterschemata affiner Schemata]
  Sei $X=\Spec A$ und $\mathfrak{a}$ ein Ideal von $A$. Sei $\imath:\Spec A/\mathfrak{a}\rightarrow\Spec A$
  der von $A\rightarrow A/\mathfrak{a}$ induzierte Morphismus affiner
  Schemata. Nach Proposition 2.12 induziert $\imath$ einen Homöomorphismus
  $\Spec A/\mathfrak{a}\overset{\cong}{\rightarrow}V(\mathfrak{a})\subseteq\Spec A$.
  Sei $\overline{\mathfrak{p}_{x}}$ das Bild von $\mathfrak{p}_{x}$
  in $A/\mathfrak{a}$. Für alle $x\in V(\mathfrak{a})$ ist der Morphismus
  $i_{x}^{\flat}$ der kanonische Homomorphismus $A_{\mathfrak{p}_{x}}\rightarrow(A/\mathfrak{a})_{\overline{\mathfrak{p_{x}}}}$.
  ($=0$, falls $x\in V(\mathfrak{a})$, also $\mathfrak{a}\notin f_{x}$.)
  Schreibe kurz $V(\mathfrak{a})$ für den lokal geringten Raum 
  \begin{align*}
    \left(V(\mathfrak{a}),\imath_{x}(\mathcal{O}_{\Spec A/\mathfrak{a}})|_{V(\mathfrak{a})}\right) & \stackrel[\imath]{\cong}{\longleftarrow}\Spec(A/\mathfrak{a})
  \end{align*}

  Da $x\in V(\mathfrak{a})$, ist $\imath_{x}(\mathcal{O}_{\Spec A/\mathfrak{a}})|_{V(\mathfrak{a})}\overset{\cong}{\longrightarrow}i_{x}\mathcal{O}_{\Spec A/\mathfrak{a}}$.
\end{example}

\begin{example}[39]
  Sei $B$ ein Ring und $\mathfrak{b}\subset B$ Ideal. $V(\mathfrak{b}^{n})=V(\mathfrak{b})\subset\Spec B$
  als abgeschlossene Teilmenge hängt \emph{nicht }von $n\geq1$ ab,
  aber $\Spec(B/\mathfrak{b}^{n})=V(\mathfrak{b}^{n})$ als offenes
  Schemata sehr wohl!

  Etwa: $B=k[T]$, $b=(T)$ mit $k$ ein algebraisch abgeschlossener
  Körper. Für die abgeschlossenen Punkte von $\mathbb{A}_{k}^{1}=\Spec k[T]$
  gilt:
  \begin{align*}
    \Spec(k[T]) & \longleftrightarrow k\\
    \mathfrak{b} & \longleftrightarrow0
  \end{align*}

  Sei $A=k[T]/(T^{n})$, $X=\Spec A=\{x\}$. Es gilt: 
  \begin{align*}
    \mathcal{O}_{X}(X) & =\mathcal{O}_{X,x}=A\\
    \kappa(x) & =k\\
    n>1:\ 0\neq\mathfrak{m}_{x} & =T\mod T^{n}
  \end{align*}

  Betrachte $X\subset\mathbb{A}_{k}^{1}$ als abgeschlossenes ,,Unterschemata``
  welches ,,konzentriert in einem Punkt`` ist. $B=k[T]$ ist $k$-Algebra
  von Funktionen auf $\mathbb{A}^{1}(k)$. (vgl. Beispiel 2.14.) Die
  Einschränkung einer solchen Funktion $f\in k[T]$ auf $X$ ist gegeben
  durch $k[T]\rightarrow k[T]/(T^{n})$. Wir unterscheiden:
  \begin{itemize}
  \item[$n=1$.] $k[T]/(T^{n})=k$, $f\mapsto f(0)$.
  \item[$n>1$.] $k[T]/(T^{n})\neq k$, $f\mapsto$ (,,Taylor-Entwicklung`` von
    $A$ um 0 der Länge $n-1$). $\{x\}\subset\mathbb{A}_{k}^{1}$ hat
    ,,infinitesimale Ausdehnung der Länge $n-1$ in $\mathbb{A}_{k}^{1}$``
  \end{itemize}
  Sei nun $\mathbb{A}_{k}^{2}:=\Spec(k[T,U])$ betrachtet als $\{(u,t)\mid u,f\in k\}=k^{2}$.
  Sei $\mathfrak{a}_{1}=(U)$, $\mathfrak{a}_{2}=(U-T^{n})$. Diese
  definieren:
  \[
    X_{1}=\{u,t)\in\mathbb{A}^{2}(k)\mid u=0\},\quad X_{2}=\{(u,t)\in\mathbb{A}^{2}(k)\mid u=t^{n}\}.
  \]

  Es ist $X_{1}\cap X_{2}=\{(0,0)\}$ als Menge. Aber für $n>1$ treffen
  sich beide Mengen \emph{nicht} transversal! Als affine Schemata wird
  später der Schnitt als $\Spec k[T,U]/(\mathfrak{a}_{1}+\mathfrak{a}_{2})$
  definiert, also eine präzisere Beschreibung als Durchschnitt.
\end{example}


\chapter{Schemata}
\label{chap:schemata}

= Verkleben affiner Schemata

\section{Schemata}
\begin{defn}
  Ein Schemata ist ein lokal geringter Raum $(X,\mathcal{O}_{X})$,
  der eine offene Überdeckung $(U_{i})_{i\in I}$ besitzt, so dass alle
  lokal geringten Räume $(U_{i},\mathcal{O}_{X|U_{i}})$ affine Schemata
  sind. Für ein Schemata $S$ bezeichne $\schs$ die \textbf{Kategorie
    der Schemata über $S$} oder $S$-Schemata. Die Objekte dieser Kategorie
  sind Morphismen $X\rightarrow S$ von Schemata, und die Morphismen
  $\Hom(X\rightarrow S,Y\rightarrow S)$ sind Morphismen $X\rightarrow Y$
  von Schemata so dass
  \[
    \xymatrix{X\ar[rr]\ar[dr] &  & Y\ar[dl]\\
      & S
    }
  \]

  kommutiert. $X\rightarrow S$ heißt \textbf{Strukturmorphismus} des
  $S$-Schematas $X$. Ist $S=\Spec R$ affin, spricht man auch von
  $R$-Schemata oder Schemata über $R$. Die Menge der Morphismen $X\rightarrow Y$
  in $\schs$ bezeichnen wir mitt $\Hom_{S}(X,Y)$ bzw. $\Hom_{R}(X,Y)$
  falls $S=\Spec R$ affin ist.
\end{defn}

\section{Offene Unterschemata}

Erinnerung: $X=\Spec A$ affin $\Longrightarrow(D(f),\mathcal{O}_{X|D(f)})$
auch affin, und $D(f)$ Basis der Topologie.
\begin{prop}[2]
  Sei $X$ ein Schemata.
  \begin{enumerate}
  \item Ist $U\subset X$ eine offene Teilmenge, dann ist der lokal geringte
    Raum $(U,\mathcal{O}_{X\mid U})$ wieder ein Schemata. $U$ heißt
    ein \textbf{offenes Unterschemata}. Ist $U$ affin, dann heißt $U$
    \textbf{affines offenes Schemata}.
  \item Die zugrundlegenden topologische Räume der affinen offene Unterschemata
    bilden eine Basis der Topologie.
  \end{enumerate}
\end{prop}

\begin{proof}
  Es gibt eine Überdeckung $(U_{i})$ von $X$, d.d. $(U_{i},\mathcal{O}_{X|U_{i}})$
  affine Schemata, $\cong\Spec A$. Es gilt:
  \[
    U=\bigcup_{i}(U\cap U_{i})=\bigcup_{i,j\in I_{i}}D(f_{ij})
  \]

  wobei die letzte Gleichheit gilt wegen $\Spec(A_{i})\supset U\cap U_{i}=\bigcup_{j\in I_{i}}D(f_{ij})$,
  $f_{ij}\in A_{i}$.
\end{proof}
Zu $U\subset X$ offen gibt es einen kanonischen Morphismus von Schemata
\[
  (j,j^{\flat}):(U,\mathcal{O}_{X|U)}\longrightarrow(X,\mathcal{O}_{X})
\]

via der Inklusion $j:U\hookrightarrow X$ und $j^{\flat}:\mathcal{O}_{X}\rightarrow j_{\ast}(\mathcal{O}_{X\mid U})$.
Für $V\subseteq X$ offen ergibt $\res_{V\cap U}^{V}$ einen Ringhomomorphismus:
\[
  \Gamma(V,\mathcal{O}_{X})\rightarrow\Gamma(V\cap U,\mathcal{O}_{X})=\Gamma(j^{-1}(V),\mathcal{O}_{X|U})=\Gamma(V,j_{\ast}\mathcal{O}_{X|U}).
\]

Eine affine offene Überdeckung eines Schematas $X$ ist eine Überdeckung
$X=\bigcup U_{i}$, sodass alle $U_{i}$ affine offene Unterschemata
sind.
\begin{lem}[3]
  Sei $X$ ein Schemata, und seien $U,V\subset X$ affin offene Unterschemata.
  Dann existiert für jedes $x\in U\cap V$ eine Umgebung $x\in W\subset U\cap V$
  offenes Unterschema, welches gleichzeitig prinzipal offen in $U$
  \emph{und} $V$ ist.
\end{lem}

\begin{proof}
  Sei ohne Einschränkung $V\subset U$ (sonst ersetze $V$ durch eine
  prinzipiale offene Teilmenge von $V$ welche $x$ enthält). Wähle:
  \[
    \xymatrix{f\in\Gamma(U,\mathcal{O}_{X})\ar[d]^{\res} & \text{d.d. }x\in D(f)\subset V\\
      f|_{V}\in\Gamma(V,\mathcal{O}_{X}) & D_{U}(f)=D_{V}(f|_{V})
    }
  \]

  denn $\Gamma(U,\mathcal{O}_{X})_{f}=\mathcal{O}_{X}(D_{U}(f)$, $\Gamma(V,\mathcal{O}_{X})_{f|_{V}}=\mathcal{O}_{X}(D_{V}|f_{|_{V}})$.
\end{proof}


\section{Morphismen in affinen Schemata hinein}
\begin{prop}[4]
  Sei $X$ ein Schemata, $Y=\Spec B$ ein affines Schemata. Dann ist
  die Abbildung
  \begin{align*}
    \Hom(X,Y) & \overset{\cong}{\longrightarrow}\Hom_{\ring}(B,\Gamma(X,\mathcal{O}_{X})),\\
    (f,f^{\flat}) & \longmapsto f_{Y}^{\flat}
  \end{align*}

  eine Bijektion.
\end{prop}

\begin{prop}[5, Verkleben von Morphismen]
  Seien $X,Y$ lokal geringte Räume. Für $U\subset X$ offen definiert
  \[
    \mathcal{F}:U\mapsto\Hom(U,Y)=\{(U,\mathcal{O}_{X|U})\rightarrow(Y,\mathcal{O}_{Y})\ \text{Morph. lokal ger. Räume\}}
  \]

  eine Garbe von Mengen auf $X$, d.h.
  \begin{enumerate}
  \item für eine offene Überdeckung $X=\bigcup_{i}U$, eine Familie $f_{i}:U_{i}\rightarrow Y_{i}$
    verkleben zu Morphismen
    \[
      f:X\rightarrow Y\Longleftrightarrow f_{i}|_{U_{i}\cap U_{j}}=f_{j}|_{U_{i}\cap U_{j}}
    \]
  \item $f$ ist eindeutig bestimmt.
  \end{enumerate}
\end{prop}

\begin{rem*}
  $\mathcal{G}:U\mapsto\Hom_{\ring}(B,\Gamma(U,\mathcal{O}_{X}))$ ist
  Garbe von Mengen.
\end{rem*}
\begin{proof}[Beweis von Proposition 5]
  Verkleben topologischer Räume + stetige Abbildung klar. $\checkmark$

  $\mathcal{O}_{Y}\rightarrow f_{\ast}\mathcal{O}_{X}$ lässt sich ebenfalls
  verkleben.
\end{proof}
% 
\begin{proof}[Beweis von Proposition 4]
  $X=\bigcup_{i}U_{i}$ sei eine affine offene Überdeckung. Nach Proposition
  2.35 ist $\Hom(U,Y)\rightarrow\Hom(B,\Gamma(U,\mathcal{O}_{X}))$
  eine Bijektion. Für $V\subset U_{i}\cap U_{j}$ kommutiert das Diagramm
  \[
    \xymatrix{\Hom(U,Y)\ar[r]^{\cong}\ar[d] & \Hom(B,\Gamma(U,\mathcal{O}_{X})\ar[d]\\
      \Hom(V,Y)\ar[r]^{\cong} & \Hom(B,\Gamma(V,\mathcal{O}_{X}))
    }
  \]

  da $\Gamma(-)$ funktoriell ist. Es folgt, dass $\mathcal{F}\rightarrow\mathcal{G}$
  ein Morphismus von Garben ist mit $\varphi_{U}:\mathcal{F}(U)\overset{\cong}{\rightarrow}\mathcal{G}(U)$
  für alle $U\in\mathcal{B}$, und $F\overset{\cong}{\rightarrow}\mathcal{G}$
  als Garbe. Somit $\mathcal{F}(X)\cong\mathcal{G}(X)$.
\end{proof}
$V\subset X$ offen beliebig, $\varphi_{V}=\underset{\underset{U\in B_{V}}{\longleftarrow}}{\lim}\varphi_{U}$.

Da $\mathbb{Z}$ kofinales Objekt in der Kategorie der Ringe ist ($\mathbb{Z}\overset{\exists_{1}}{\rightarrow}R$
für beliebige Ringe $R$), gilt:
\begin{cor}[6]
  Sei $X$ ein Schemata. $X$ besitzt einen eindeutig bestimmten Morphismus
  $X\rightarrow\Spec(\mathbb{Z})$, d.h. $\Spec(\mathbb{Z})$ ist ein
  terminales Objekt in der Kategorie der Schemata: Jedes Schemata ist
  ein $\mathbb{Z}$-Schemata.
\end{cor}

Weiterhin:
\begin{align*}
  \Hom(X,\underbrace{\Spec\mathbb{Z}[T]}_{\mathbb{A}_{\mathbb{Z}}^{1}}) & =\Hom_{\ring}(\mathbb{Z}[T],\mathcal{O}_{X}(X))=\Gamma(X,\mathcal{O}_{X})\\
  \Hom_{R}(X,\underbrace{\Spec R[T]}_{\mathbb{A}_{R}^{1}}) & =\Gamma(X,\mathcal{O}_{X})\text{ als }R\text{-Algebra für }R\text{-Schemata }X
\end{align*}

\section{Morphismen der Form $\Spec(K)\rightarrow X$}

Sei $X$ ein Schemata und sei $x\in U\subset X$ offene affine Umgebung
von $x$, z.B. $U=\Spec A$. Sei $\mathfrak{p}=\mathfrak{p}_{x}\subset A$.
Es folgt: $\mathcal{O}_{X,x}=\mathcal{O}_{U,x}=A_{\mathfrak{p}}$,
und der Homomorphismus $A\rightarrow A_{\mathfrak{p}}$ induziert
\[
  j_{x}:\Spec\mathcal{O}_{X,x}=\Spec A_{\mathfrak{p}}\longrightarrow\Spec A=U\subset X
\]

Morphismus von Schemata, welcher nach Proposition 2 unabhängig von
$U$ ist. Nach Proposition 2.22 ist
\begin{align*}
  j_{x}:\Spec\mathcal{O}_{X,x}\overset{\cong}{\longrightarrow}Z & =\{x'\in X\mid x'\text{ Verallgemeinerung von }x\}\\
  (x\in\{x'\}\Leftrightarrow\mathfrak{p}_{x'}\subset\mathfrak{p}_{x})\  & =\bigcap_{x\in U\subseteq_{\text{off.}}X}U
\end{align*}

Sei $\kappa(x)=\mathcal{O}_{X,x}/\mathfrak{m}_{x}$. Die Abbildung
$\mathcal{O}_{X,x}\rightarrow\kappa(x)$ induziert einen Morphismus
von Schemata
\begin{align*}
  i_{x}:\Spec\kappa(x) & \longrightarrow\Spec\mathcal{O}_{X,x}\longrightarrow X\\
  \{\text{pt}\} & \longmapsto x
\end{align*}

Nun sei $K$ ein beliebiger Körper, und $f:\Spec K\rightarrow X$
ein beliebiger Morphismus mit $f(\text{\{pt\}})=x\in X$. Dieser induziert
einen lokalen Homomorphismus
\[
  \xymatrix{\mathcal{O}_{X,x}\ar[r]\ar[d] & K=\mathcal{O}_{\Spec(K),(0)}\\
    \kappa(x)\ar[ur]_{\imath}
  }
\]

d.h. $f$ faktorisiert als $f=i_{x}\circ(\Spec\imath):\Spec K\rightarrow\Spec\kappa(x)\rightarrow X$.
Damit haben wir:
\begin{prop}[7]
  Die Abbildung
  \[
    \Hom(\Spec K,X)\longrightarrow\{(x,\imath):x\in X,\ \imath:\kappa(x)\rightarrow K\}
  \]

  ist eine Bijektion.
\end{prop}

\begin{proof}
  Umgekehrt bilden wir:
  \[
    (x,\imath:\kappa(x)\rightarrow K)\longrightarrow(\Spec K\overset{\Spec\imath}{\rightarrow}\Spec\kappa(x)\overset{i_{x}}{\rightarrow}X).
  \]
\end{proof}

\section{Verkleben von Schemata und disjunkte Vereinigung}
\begin{defn}
  Ein \textbf{Verklebe-Datum} von Schemata besteht aus:
  \begin{itemize}
  \item einer Indexierung $I$;
  \item ein Schemata $U_{i}$ für $i\in I$;
  \item ein affines Unterschemata $U_{ij}\subset U$ für alle $i,j\in I$;
  \item einen Isomorphismus $U_{ij}\stackrel[\cong]{\varphi_{ji}}{\longrightarrow}U_{ji}$
    für alle $(i,j)\in I\times I$, sodass:
    \begin{enumerate}
    \item $U_{ii}=U_{i}$ für alle $i\in I$;
    \item (Kozykel-Bedingung): $\varphi_{kj}\circ\varphi_{ji}=\varphi_{ki}$
      auf $U_{ij}\cap U_{ik}$, für alle $i,j,k\in I$.
    \end{enumerate}
  \end{itemize}
\end{defn}

Für die Kozykel-Bedingung soll implizit gelten:
\begin{align*}
  \varphi_{ji}(U_{ij}\cap U_{ik}) & \subseteq U_{jk}\\
  i=j=k & \Rightarrow\varphi_{ii}=\id_{U_{i}},\\
  \varphi_{ij}^{-1} & =\varphi_{ji},\text{ und}\\
  \varphi_{ji}:U_{ij}\cap U_{ik} & \overset{\cong}{\rightarrow}U_{ji}\cap U_{jk}
\end{align*}

\begin{prop}[9]
  Zu einem Verklebe-Datum $((U_{i})_{i\in I},(U_{ij})_{i,j\in I},(\varphi_{ij})_{i,j\in I})$
  gibt es ein Schemata $X$ zusammen mit Morphismen $\psi_{i}:U_{i}\rightarrow X$,
  sodass
  \begin{itemize}
  \item für alle $i\in I$ induziert $\psi_{i}$ einen Isomorphismus von $U_{i}$
    auf offene Unterschemata von $X$;
  \item $\psi_{j}\circ\varphi_{ji}=\psi_{i}$ auf $U_{ij}$ für alle $i,j\in I$;
  \item $X=\bigcup_{i}\psi_{i}(U)$;
  \item $\psi_{i}(U_{i})\cap\psi_{j}=\psi_{i}(U_{ij})=\psi_{j}(U_{ji})$ für
    alle $i,j\in I$.
  \end{itemize}
  $(X,\psi_{i\in I})$ ist eindeutig bis auf eindeutige Isomorphie bestimmt.
\end{prop}

Zusammen mit Proposition 5 folgt die universelle Eigenschaft: Für
$(T,\xi_{i}:U_{i}\rightarrow T)$ mit $\xi_{i}$ welche Isomorphismen
\[
  U_{i}\overset{\cong}{\rightarrow}\text{\{offenes Unterschemata von }T\}
\]

induzieren, sodass $\xi_{j}\circ\varphi_{ji}=\xi_{i}$ auf $U_{ij}$
für alle $i,j\in I$, dann gibt es einen eindeutigen Morphismus $\xi:X\rightarrow T$
mit $\xi\circ\psi_{i}=\xi_{i}$ für alle $i\in I$. ($\Longrightarrow$
Eindeutigkeit von Proposition 9)

\begin{proof}
Als topologischer Raum: $\coprod_{i\in I}U_{i}/\sim$ mit $x_{i}\in U_{i}\sim x_{j}\in U_{j}:\Leftrightarrow x_{i}\in U_{ij}$,
$x_{j}\in U_{ji}$ und $x_{j}=\varphi_{ji}(x_{i})$. Nach Eigenschaft
$(b)$ ist $\sim$ eine Äquivalenzrelation. Dann sind $\psi_{i}:U_{i}\rightarrow X$
injektiv. Ferner haben wir $\forall i,j\in I$ die Eigenschaft $\psi_{i}(U_{i})=\psi_{i}(U_{i})\cap\psi_{j}(U_{j})$.

$X$ hat also als topologischer Raum die Quotiententopologie, d.h.
die feinste Topologie sodass alle Abbildungen $\psi_{i}$ stetig sind.
$U\subset X$ offen genau dann, wenn $\psi_{i}^{-1}(U)\subset U_{i}$
dort offen sind $\forall i\in I$. Insbesondere sind $\psi_{i}(U_{i})$
und $\psi_{i}(U_{j})=\psi_{i}(U_{i})\cap\psi_{j}(U_{j})$ offen in
$X$.

Als lokal geringter Raum: ``Verkleben der Strukturgarben auf $U_{i}$''.
$\mathcal{O}_{X}$ ist eindeutig auf einer Basis $B$ der Topologie
definiert. Ohne Einschränkung reicht es hier, die Schnitte nur auf
$U\subset X$ offen mit $U\subset\psi_{i}(U_{i})$ für ein $i\in I$.
In dem Fall:
\[
  \mathcal{O}_{X}(U)=\mathcal{O}_{U_{i}}(\psi_{i}^{-1}(U))
\]

Für $U\subset\psi_{i}(U_{i})\cap\psi_{j}(U_{j})$
\[
  \xymatrix{U_{ij}\ar@{^{(}->}[r]\ar[d]_{\cong} & U_{i}\\
    U_{ji}\ar@{^{(}->}[r] & U_{i} & X\supset U
  }
\]

Dann gilt:

\[
  \mathcal{O}_{U_{i}}(\psi_{i}^{-1}(U))=\mathcal{O}_{U_{ij}}(\psi_{i}^{-1}(U))\cong\mathcal{O}_{U_{ji}}(\psi_{j}^{-1}(U))=\mathcal{O}_{U_{j}}(\psi_{j}^{-1}(U))
\]

Es folgt: $\mathcal{O}_{X}(U)$ unabhängig von Wahlen von $i$! Wir
halten damit $\mathcal{O}_{X}$ Ringgarbe auf $X$, sodass $(X,\mathcal{O}_{X})$
lokal geringter Raum da $(U_{i},\mathcal{O}_{U_{i}})$ lokal geringter
Raum $\forall i\in I$, $U_{i}\xrightarrow[\psi_{i}]{\cong}(\psi_{i}(U_{i}),\mathcal{O}_{X}|_{\psi_{i}(U_{i})})$
als lokal geringter Raum. Damit ist $X$ ein Schema und $X=\cup U_{i}$.

Spezialfall: $U_{ij}=\emptyset$ für alle $i\neq j\in I$, $\coprod U_{i}$
``disjunkte Vereinigung''.
\end{proof}
\begin{example}[10]
  $X_{1},\ldots,X_{n}$ affine Schemata, $X_{i}=\Spec A_{i}$. Dann
  ist
  \[
    \coprod X_{i}\cong\Spec\left(\prod_{i=1}^{n}A_{i}\right)\text{ offen}.
  \]

  (nicht für unendlich viele affin!)
\end{example}

\begin{example}[11]
  $I=\{1,2\}$, $U_{12}\subset U_{1}\xrightarrow{\varphi}U_{21}\subset U_{2}$.
  \[
    X\underset{\text{offen}}{\subset}V=U_{1}\cup_{\varphi}U_{2},
  \]

  $\Gamma(V,\mathcal{O}_{X})=\{(s_{1},s_{2})\in\Gamma(V\cap U_{1},\mathcal{O}_{U_{1}})\times\Gamma(V\cap U_{2},\mathcal{O}_{U_{2}})\}$,
  $\varphi^{\flat}(S_{2}|_{U_{21}\cap V})=S_{1}|_{U_{12}\cap V}$.
\end{example}

\textbf{Affine Gerade mit Doppelpunkt}: $k$ Körper.
\begin{align*}
  U_{1} & =U_{2}=\mathbb{A}_{k}^{1}=\Spec(k[T])\cong x\text{ abg.}\\
  U_{12} & :=U_{1}\backslash\{x\}\\
  U_{21} & :=U_{2}\backslash\{x\},\ \varphi=\id
\end{align*}

$X=U_{1}\cup_{\varphi}U_{2}$.  
\textbf{Aufgabe.} $X$ ist \textbf{nicht} affin!

\section{Der projektive Raum als Schema }

Sei $R$ Ring. $\mathbb{P}_{R}^{n}$ Verklebung von $(n+1)$-Kopien
des 
\begin{align*}
  \mathbb{A}_{R}^{n} & =\Spec(R[T_{1},\ldots,T_{n}])\\
  \shortparallel\\
  \Spec\left(R\left[\frac{X_{0}}{X_{i}},\ldots,\frac{\hat{X}_{i}}{X_{i}},\ldots,\frac{X_{n}}{X_{i}}\right]\right)=U_{i}, & i=0,\ldots,n
\end{align*}

Verklebungs Datum:
\begin{align*}
  B:= & R[X_{0},\ldots,X_{n},X_{0}^{-1},\ldots,X_{n}^{-1}]\\
  U_{ij} & :=D\left(\frac{X_{j}}{X_{i}}\right)\subset U_{i},\ \text{OE}\ i\neq j\leq n,\\
  U_{ii} & =U_{i},\ \varphi_{ii}=\id
\end{align*}

$\varphi_{ji}:U_{ij}\rightarrow U_{ji}$ definiert durch
\[
  \xymatrix{R\left[\frac{X_{0}}{X_{i}},\ldots,\frac{\hat{X}_{i}}{X_{i}},\ldots,\frac{X_{n}}{X_{i}}\right]_{\frac{x_{j}}{x_{i}}}\ar@{=}[r]^{''\id''}\ar@{^{(}->}[dr] & R\left[\frac{x_{0}}{x_{j}},\ldots,\frac{\hat{x}_{j}}{x_{j}},\ldots,\frac{x_{n}}{x_{j}}\right]_{\frac{x_{i}}{x_{j}}}\ar@{^{(}->}[d]\\
    & B
  }
\]

Mit ``$\id$'' folgt: Kozykelbedingung automatisch, $U_{i}\rightarrow\Spec(R)$
verkleben von $\mathbb{P}_{R}^{n}\rightarrow\Spec(R)$ mit $\mathbb{P}_{R}^{n}:=\coprod U_{i}/\sim$
Schema (über $R$). ``Der projektive Raum relativer Dimension $n$
über $R$''.
\textbf{Aufgabe.}
$R\xrightarrow{\sim}\Gamma(\mathbb{P}_{R}^{n},\mathcal{O}_{\mathbb{P}_{R}^{n}})$
(Strukturgarbe) d.h. für $n>0$ ist $\mathbb{P}_{R}^{n}$ nicht affin (mit
$\mathbb{P}_{R}^{n}=\Spec(R)$).

\section{Nullstellenmenge im projektiven Raum }

Sei $I\subset R[X_{0},\ldots,X_{n}]$ homogenes Ideal, d.h. erzeugt
von homogenen Elementen.

\[
  \Leftrightarrow I=\bigoplus_{d}I\cap R[X_{1},\ldots,X_{n}]_{d})
\]

Ziel: $V_{+}(I)\rightarrow\mathbb{P}_{R}^{n}$ Morphismus von Schemata
\begin{align*}
  R[X_{0},\ldots,X_{n}] & \longrightarrow R\left[\frac{X_{0}}{X_{i}},\ldots,\frac{\hat{X_{i}}}{X_{i}},\ldots,\frac{X_{n}}{X_{i}}\right]=\Gamma(U_{i},\mathcal{O}_{U_{i}})\\
  I & \longmapsto\Phi_{i}(I)\text{ das vom Bild von }I\text{ erzeugte Ideal}
\end{align*}

Verklebe $V_{i}:=\Spec(\Gamma(U_{i},\mathcal{O}_{U_{i}})/\Phi_{i})\subseteq U_{i}$
entlang
\[
  V_{ij}=D_{V_{i}}\left(\frac{X_{j}}{X_{i}}\right)\xrightarrow{\cong}V_{ji}
\]

Beachte: $f\in I$, $\deg(f)=d$, $X_{i}^{d}\Phi_{i}(f)=X_{j}^{d}\Phi_{j}(f)$,
d.h. $\Phi_{i}(f)$ und $\Phi_{j}(f)$ unterscheiden sich in einer
Einheit auf $D\left(\frac{X_{i}}{X_{j}}\right)$.

$\Longrightarrow\Phi_{i}(I)=\Phi_{j}(I)$ in $\Gamma(U_{ij},\mathcal{O}_{U_{i}})$.
$\Longrightarrow V_{ij}=V_{ji}$ und Kozykelbedingungen überträgt
sich von $U_{ij}\subset U_{i}$.

$\Longrightarrow$ Verkleben liefert Schema $V_{+}(I)$ + 

D.h. jedes solche $I$ definiert ein Schema $V_{+}(I)\rightarrow\mathbb{P}_{R}^{n}$.

\chapter*{Grundlegende Eigenschaften von Schemata und Morphismen}

\section{Topologische Eigenschaften }
\begin{defn}[12]
  Ein Schema $X$ heißt \textbf{zusammenhängend}, \textbf{quasi-kompakt}
  bzw. \textbf{irreduzibel}, falls der unterliegende topologische Raum
  diese Eigenschaft besitzt.
\end{defn}

\begin{itemize}
\item Nach Proposition II.5 ist jedes affine Schema quasi-kompakt.
\item $\coprod_{i=0}\Spec(R)$ ist \emph{nicht }quasi-kompakt.
\end{itemize}
\begin{defn}[13]
  $f:X\rightarrow Y$ heißt \textbf{injektiv} (surjektiv, bijektiv),
  falls die zugrendlegende stetige Abbildung diese Eigenschaft hat.
  Ebenso für ``offen'', ``abgeschlossen'', ``Homömorphismus''.
\end{defn}

Warnung: Homömorphismen von Schemata sind im Allgemeinen \emph{keine
}Isomorphismen!

\section{Noethersche Schemata }
\begin{defn}[14]
  Ein Schema heißt \textbf{lokal noethersch}, falls eine affine offene
  Überdeckung $X=\bigcup U_{i}$ existiert, d.d. alle $\Gamma(U_{i},\mathcal{O}_{X})$
  (affine Koordinatenringe) \textbf{noethersch} sind. $X$ heißt \textbf{noethersch},
  falls zusätlich quasi-kompakt.
\end{defn}

Faktum: Lokalisierung noetherscher Ringe bleiben noethersch.

$\Rightarrow a)$ Jedes lokal noethersche Schema besitzt eine Basis
der Topologie aus noetherschen affin offenen Unterschemata.

b) $X$ lokal noethersch. Dann ist $\mathcal{O}_{X,x}$ noethersch
$\forall x\in X$.

Für offene Schemata gilt ferner: lokal noethersch $\Rightarrow$ noethersch.

\begin{prop}[15]
  Für $X\subset\Spec A$ affin gilt:
  \[
    X\text{ noethersch }\Leftrightarrow A\text{ noethersch}
  \]
\end{prop}

\begin{proof}
  \mbox{}
  \begin{casenv}
  \item[``$\Leftarrow$''] $X$ überdeckt sich selbst mit $\Gamma(X,\mathcal{O}_{X})=A$ noethersch.
  \item[``$\Rightarrow$''] Sei $I\subset A$ beliebiges Ideal. Zu zeigen: $I$ ist endlich erzeugt.
    Nach Voraussetzung ist
    \[
      X=\bigcup_{i=1}^{n}\Spec A_{i},\quad A_{i}\text{ noethersch}.
    \]
    Ohne Einschränking: $A_{i}=A_{f_{i}}$ und noethersch. Daraus folgt:
    $J_{i}=IA_{f_{i}}=I_{f_{i}}$ sind endlich erzeugt, Behauptung folgt
    aus Lemma 16.
  \end{casenv}
\end{proof}
\begin{lem}[16]
  $\Spec(A)=\cup_{i\in I}D(f_{i})$, $\#I<\infty$, $M$ $A$-Modul.
  Dann:
  \[
    M\text{ e.e. über }A\Leftrightarrow M_{f_{i}}\text{ e.e. }A_{f_{i}}\text{-Modul }\forall i\in I.
  \]
\end{lem}

\begin{proof}
  \mbox{}
  \begin{casenv}
  \item[``$\Rightarrow$'' ] Endlich erzeugt heißt $A^{n}\twoheadrightarrow M$, Lokalisierung
    exakt also $A_{f_{i}}^{n}\twoheadrightarrow M_{f_{i}}$ exakt.
  \item[``$\Leftarrow$''] $M_{f_{i}}$ werden von $\frac{m_{ij}}{f_{i}^{n_{ij}}}$, $j=1,\ldots,r_{i}$,
    $m_{ij}\in M$, $n_{ij}\in\mathbb{N}_{0}$ als $A_{f_{i}}$-Modul
    erzeugt.

    $\Rightarrow N:=\langle m_{ij}\rangle_{A}\subset M$ ist endlich erzeugt
    und $N_{f_{i}}=M_{f_{i}}$.

    $\Rightarrow(M/N)_{\mathfrak{p}}=(M_{f_{i}}/N_{f_{i}})_{\mathfrak{p}}=0$
    für alle Primideale $\mathfrak{p}\in\Spec A$.

    $\Rightarrow$ (Lokal-Global-Prinzip aus der kommutativen Algebra)
    $N=M$.

  \end{casenv}
\end{proof}
\begin{rem*}
  $X$ noethersches Schema. Dann ist $X$ als topologischer Raum noethersch.
\end{rem*}
\begin{proof}
  Für $X$ affin klar, sonst $X=\cup_{i=1}^{r}X_{i}$, $X_{i}=\Spec(A_{i})$
  noethersch. Sei
  \[
    X\supseteq Z_{1}\supseteq\cdots\supseteq Z_{n}\supseteq\cdots
  \]

  absteigende Kette abgeschlossener Teilmengen. $(Z_{j}\cap X_{i})_{j}$
  absteigende Kette abgeschlossener Teilmengen in $X_{i}$.

  $\Longrightarrow$ (endliche Überdeckung) $\exists N$ d.d. $Z_{j}\cap X_{i}=Z_{N}\cap X_{i}$
  für alle $j\geq N$.

  $\Longrightarrow Z_{j}=Z_{N}$.
\end{proof}
\begin{cor}[17]
  Sei $X$ (lokal) noethersches Schema, $U\subset X$ offenes Unterschema.
  Dann ist $U$ ein (lokal) noethersches Schema.
\end{cor}

\begin{proof}
  Lokal noethersch $X=\bigcup U_{i}$, $U_{i}\cap U=\bigcup D(f_{i})$.
  Sei $X$ noethersch. Dann ist der topologische Raum $X$ noethersch.
  Nach Lemma $I.20$ ist dann jede offene Teilmenge quasi-kompakt.
\end{proof}

\section{Generische Punkte }
\begin{prop}[18]
  Die Abbildung
  \begin{align*}
    X & \longrightarrow\{Z\subset X\mid\text{abg., irred.}\}\\
    x & \longmapsto\overline{\{x\}}
  \end{align*}

  ist eine Bijektion, d.h. jede irreduzible abgeschlossene Teilmenge
  enthält genau einen generischen Punkt.
\end{prop}

\begin{proof}
  Gilt für affine Schemata nach Korollar II.7. Sei $Z\subset X$ irreduzibel,
  abgeschlossen sowie $U\subset X$ affin offen mit $Z\cap U\neq\emptyset$.

  $\Longrightarrow\overline{Z\cap U}^{X}=Z$, da $Z$ irreduzibel.

  $\Longrightarrow Z\cap U$ irreduzibel mit generischen Punkt $x$,
  $\overline{\{x\}}^{Z\cap U}=Z\cap U$.

  $\Longrightarrow\overline{\{x\}}^{X}=Z$.

  Umgekehrt: Sei $z\in Z$ generischer Punkt.

  $\Longrightarrow[U\subset X$ offen mit $U\cap Z\neq\emptyset$ $\Rightarrow z\in U]$
  d.h. Eindeutigkeit im affinen Fall impliziert allgemeiner Fall.
\end{proof}
``Generische Punkte reduzieren gewisse Aussagen auf das Studium von
\emph{einem} Punkt''.
\begin{prop}[19]
  Sei $f:X\rightarrow Y$ offener Morphismus von Schemata. Sei $Y=\overline{\{\eta\}}^{Y}$
  irreduzibel. Dann:
  \[
    f^{-1}(\eta)\text{ irreduzibel}\Leftrightarrow X\text{ irreduzibel}
  \]
\end{prop}

\begin{proof}
  $f$ offen $\Rightarrow\overline{\{f^{-1}(x)\}}=f^{-1}(\overline{\{\eta\}})=f^{-1}(Y)=X$.
  Mit Lemma I.14: $Z$ irreduzibel $\Leftrightarrow\overline{Z}$ irreduzibel.
\end{proof}
Topologische Räume von Schemata sind fast nie Hausdorffsch, aber:
\begin{prop}[20]
  Sei $X$ Schema. Dann ist der unterliegende topologische Raum ein
  $T_{0}$-Raum, d.h.
  \[
    \forall x\neq y\in X\ \exists U\subset X\text{ offen, mit \textbf{entweder }}x\in U\text{ oder }y\in U.
  \]
\end{prop}

\begin{proof}
  Ohne Einschränkung: $X$ affin, $x=\mathfrak{p}_{x}$, $y=\mathfrak{p}_{y}\in\Spec(\Gamma(X,\mathcal{O}_{X}))$.
  Falls $\mathfrak{p}_{x}\subsetneq\mathfrak{p}_{y}$ wähle
  \[
    \mathfrak{p}_{x}\in U=X\backslash\underbrace{V(\mathfrak{p}_{y})}_{\ni\mathfrak{p}_{y}}
  \]

  andernfalls $\exists f\in\mathfrak{p}_{x}\backslash\mathfrak{p}_{y}$,
  d.h. $U=D(f)$ enthält $y$ aber nicht $x$.
\end{proof}
Später: Separiertheit von Schemata als ``Hausdorffsch''-Ersatz.

\section{Reduzierte und ganze Schemata }
\begin{defn}[21]
  Ein Schema $X$ heißt 
  \begin{enumerate}
  \item \textbf{reduziert}, falls alle $\mathcal{O}_{X,x}$, $x\in X$, reduzierte
    Ringe sind.
  \item \textbf{ganz}, falls $X$ reduziert und irreduzibel ist.
  \end{enumerate}
\end{defn}

\begin{prop}[22]
  \mbox{}
  \begin{enumerate}
  \item $X$ schema ist reduziert (ganz) $\Leftrightarrow\Gamma(U,\mathcal{O}_{X})$
    reduziert (integer) für alle $U\subseteq X$ offen.
  \item Sei $X$ ganz. Dann ist der Halm $\mathcal{O}_{X,x}$ integer $\forall x\in X$.
    (Die Umkehrung ist im Allgemeinen falsch!)
  \end{enumerate}
\end{prop}

\begin{proof}
  \mbox{}
  \begin{enumerate}
  \item \textbf{reduzibel,} ``$\Rightarrow$''. $f\in\Gamma(U,\mathcal{O}_{X})$
    mit $f^{n}=0$. Angenommen, $f\neq0$. Dann gibt es ein $x\in U$
    mit $f_{x}\neq0$ in $\mathcal{O}_{X,x}$, $f_{x}^{n}=0$. Widerspruch

    \textbf{reduzibel,} ``$\Leftarrow$''. Sei $\overline{f}\in\mathcal{O}_{X,x}$
    nilpotent. Dann gibt es ein $x\in U\subset X$ offen und $f\in\Gamma(U,\mathcal{O}_{X})$
    mit $f_{x}=\overline{f}$. Ohne Einschränkng: $f$ nilpotent (mit
    $U$ verkleinern sodass $f^{n}|_{U}=0$). Nach Voraussetzung ist dann
    $f=0$, also $\overline{f}=0$.

    \textbf{ganz,} ``$\Rightarrow$''. Sei $X$ ganz. Dann ist $U\subset X$
    offen ganz nach den Definitionen. Daher reicht es zu zeigen, dass
    $\Gamma(X,\mathcal{O}_{X})$ integer ist. Seien $f,g\in\mathcal{O}_{X}(X)$
    mit $fg=0$. Dann ist $X=V(f)\cup V(g)$. $X$ ist irreduzibel, also
    etwa $X=V(f)$. \emph{Behauptung}: $f=0$.

    Da Verschwinden aufgrund des Garbenaxioms eine lokale Frage ist, setze
    ohne Einschränking $X=\Spec A$ affin. Es folgt: $f\in\bigcap_{\Spec A}\mathfrak{p}=\sqrt{(0)}=0$.

    \textbf{ganz, }``$\Leftarrow$''. $\Gamma(U,\mathcal{O}_{X})$ integer,
    also reduziert. Nach (1, reduziert) ist $X$ reduziert. Angenommen
    es gibt $\emptyset\neq U_{1},U_{2}\subset X$ offen mit $\emptyset=U_{1}\cap U_{2}$.
    Nach den Garbenaxiomen enthält dann $\Gamma(U_{1}\cup U_{2},\mathcal{O}_{X})=\Gamma(U_{1},\mathcal{O}_{X})\times\Gamma(U_{2},\mathcal{O}_{X}$)
    Nullteiler $(1,0)\cdot(0,1)=0$. Widerspruch.
  \item Folgt aus 1, da $A$ integer, $0\notin S$. Es folgt: $A_{S}$ integer
    ($\subseteq\Quot(A)$).
  \end{enumerate}
\end{proof}
\begin{rem*}
  $X=\Spec A$ ganz $\Leftrightarrow A$ integer, $\eta\in X$ generischer
  Punkt $\Leftrightarrow(0)\subset A$. Es ist $\mathcal{O}_{X,\eta}=A_{(0)}=\Quot(A)$,
  d.h. für jedes ganze Schema $X$ gilt: $\mathcal{O}_{X,\eta}$ ist
  Körper (mit generischer Punkt $\eta$).
\end{rem*}
\begin{defn}[23]
  $X$ ganz, $\eta\in X$ generischer Punkt. Dann heißt $K(X):=\mathcal{O}_{X,\eta}$
  der \textbf{Funktionenkörper} von $X$.
\end{defn}

\begin{prop}[24]
  Sei $X$ noethersches irreduzibles Schema, $\eta\in X$ generischer
  Punkt. Dann sind äquivalent:
  \begin{enumerate}
  \item $\mathcal{O}_{X,\eta}$ ist reduziert.
  \item $\exists\emptyset\neq U\subset X$ reduziertes offenes Unterschema.
  \end{enumerate}
  Für $(ii)$ sagt man auch: $\mathcal{O}_{X,x}$ ist \textbf{generisch}
  reduziert.
\end{prop}

\begin{proof}
  \mbox{}
  \begin{itemize}
  \item[$``\Rightarrow"$] Sei ohne Einschränkung $X=\Spec A$ affin, $A$ nach Voraussetzung
    noethersch, $\eta\leftrightarrow\mathfrak{p}$ eindeutiges minimales
    Primideal ($A$ irreduzibel). $\Longrightarrow\mathfrak{p}=(f_{1},\ldots,f_{n})_{A}$
    endlich erzeugt. $\Longrightarrow\frac{f_{i}}{1}\in\nil(A_{\mathfrak{p}})=(0)\subset A_{\mathfrak{p}}=\mathcal{O}_{X,\eta}$,
    da $\mathcal{O}_{X,\eta}$ reduziert. $\exists g\in A\backslash\mathfrak{p}$
    $\Longrightarrow$ d.d. $\frac{f_{1}}{1}=0\sim A_{g}$. $\Longrightarrow0=\nil(A)_{g}=\nil(A_{g})$,
    d.h. $A_{g}$ ist reduziert, d.h. $U:=D(g)$.
  \item[$``\Leftarrow"$] $\emptyset\neq U\subset X$ reduziert offen. $\Longrightarrow\eta\in U$
    s.d. $\mathcal{O}_{X,x}=\mathcal{O}_{X,\eta}$ reduziert.
  \end{itemize}
\end{proof}
\begin{rem*}
  Analog zeigt man: $X$ noethersches Schema und $\mathcal{O}_{X,x}$
  reduzibel für ein $x\in X$ $\Longrightarrow\exists x\in U\subset X$
  offen, d.d. $U$ reduziert ist.
\end{rem*}

\section*{Prävarietäten als Schema}

\textbf{Ziel: }``$X$ affine Varietät $\mapsto\Spec\Gamma(X,\mathcal{O}_{X})$``
(kein generischer Punkt, $\neq$ Schema). Verklebe zu: ``$X$ Prävarietät
über $k$ $\mapsto k$-Schema``. Welches Bild hat dieser Funktor?

\section{Schemata von endlichem Typ über $k$}

Sei $X$ affine Varietät über $k$, $k$ algebraisch Abgeschlossen.
Dann ist $A=\Gamma(X,\mathcal{O}_{X})$ eine endlich erzeugte $k$-Algebra.
\begin{defn}[25]
Sei $k$ Körper, $X\rightarrow\Spec(k)$ $k$-Schema. $X$ heißt:
\begin{itemize}
\item \textbf{lokal von endlichem Typ}, (l.v.e.T./$k$), falls eine affine
offene Überdeckung $X=\bigcup_{i\in I}U_{i}$ existiert, $U_{i}=\Spec A_{i}$,
mit $A_{i}$ endlich erzeugte $k$-Algebra für alle $i$.
\item \textbf{von endlichem Typ} (v.e.T$/k$), falls $X$ lokal von endlichem
Typ und quasi-kompakt ist.
\end{itemize}
\end{defn}

\begin{rem*}
Jedes $k$-Schema welches (lokal) von endlichem Typ ist, ist (lokal)
noethersch. (Da jede endlich erzeugte $k$-Algebra noethersch ist.)
\end{rem*}
\begin{prop}[26]
Sei $X$ $l.v.e.T/k$, $U\subset X$ offen affin. Dann ist $B:=\Gamma(U,\mathcal{O}_{X})$
eine endlich erzeugte $k$-Algebra.
\end{prop}

\begin{proof}
$U=\bigcup_{i=1}^{n}D(f_{i})$, $f_{i}\in B$ geeignet nach Lemma
3.3. $B$ ist endlich erzeugte $k$-Algebra $\Longrightarrow B_{f_{i}}=B\left[\frac{1}{f_{i}}\right]$
ist endlich erzeugte $k$-Algebra. Mit dem folgenden Lemma für $A=k$
folgt die Behauptung.
\end{proof}
\begin{lem}[28]
Sei $A$ ein Ring und $B$ eine $A$-Algebra, $\mathcal{L}:A\rightarrow B$
Ringhomomrphismus, $f_{1},\ldots,f_{n}\in B$ mit $(f_{1},\ldots,f_{n})=(1)$,
und so dass $B_{f_{i}}$ eine endlich erzeugte $A$-Algebra ist $\forall i$.
Dann ist $B$ eine endlich erzeugte $A$-Algebra.
\end{lem}

\begin{proof}
(Vergleiche Lemma 16.) Nach Voraussetzung gibt es ein $g_{i}\in B$
mit $\sum_{i}g_{i}f_{i}=1$. Da $B_{f_{i}}$ endlich erzeugt $\forall i$,
gibt es $b_{ij}$, $j\in J$ endlich, welche $B_{f_{i}}$ als $A$-Algebra
erzeugen. Setze $b_{ij}=c_{ij}/f_{i}^{m}$ mit $c_{ij}\in B$ für
$m\geq0$ geeignet (unabhäng von $i,j$).

Sei $C:=A$-Unteralgebra von $B$, erzeugt von $g_{i},f_{i},c_{ij}$,
d.h. endlich erzeugt über $A$. \emph{Behauptung}: $C=B$.

Sei $b\in B$. $\Longrightarrow$ $\exists N\gg0$ mit $f_{i}^{N}b\in C$
für alle $i$. Da $\sum_{i}g_{i}f_{i}=1$, ist $(f_{1},\ldots,f_{n})_{C}=(1)$.
Lemma 2.4 $\Longrightarrow$ $(f_{i}^{N},\ldots,f_{n}^{N})_{C}=(1)$.
$\Longrightarrow$ $\exists u_{1},\ldots,u_{n}\in C$ sodass $\sum_{i}u_{i}f_{i}^{N}=1$.
$\Longrightarrow b=\sum_{i}u_{i}\underbrace{f_{i}^{N}b}_{\in C}\in C$.
\end{proof}
\begin{prop}[28]
Sei $k$ algebraisch abgeschlossen, $X$ $k$-Schema l.v.e.T.$/k$.
Dann besteht die Menge der abgeschlossenen Punkte $|X|$ genau aus
den Punkten mit $\kappa(x)=k$, d.h. nach Proposition 7 gilt
\[
|X|=X(k)=\Hom_{k}(\Spec k,X).
\]
\end{prop}

\begin{proof}
Hilbert'scher Nullstellensatz $\Longrightarrow(x\in X$ abgeschlossen
$\Rightarrow\kappa(x)=k$). Daher reicht es zu zeigen: ($x\in X$
\emph{nicht} abgeschlossen, d.h. $\mathfrak{p}_{x}$ maximal $\Rightarrow\kappa(x)\neq k$).
Dazu: $\exists x\in U=\Spec(A)\subset X$ offen, mit $x$ \emph{nicht
abgeschlossen} in $U$. $\Longleftrightarrow\mathfrak{p}=\mathfrak{p}_{x}\in\Spec(A)$
\emph{nicht} maximal, d.h. $A/\mathfrak{p}_{x}$ ist \emph{kein} Körper.
$\Longrightarrow k\rightarrow(A/\mathfrak{p}_{x})\hookrightarrow\Quot(A/\mathfrak{p}_{x})=\kappa(x)$
ist \emph{echte} Inklusion.

Behauptung: $\kappa(x)$ ist nicht algebraisch abgeschlossen, d.h.
nicht abstrakt isomorph zu $k$. Denn: \emph{Noether-Normalisierung}:

$A/\mathfrak{p}$ ist endlich über $k[X_{1},\ldots,X_{n}]$. Nach
Lemma I.9 folgt $n>0$ (mit $A/\mathfrak{p}$ über $k$ ganz $\Longrightarrow A/\mathfrak{p}$
Körper). $\Longrightarrow\kappa(x)$ ist endliche Erweiterung von
$k(X_{1},\ldots,X_{n})$, $n>0$. $\Longrightarrow k$ nicht algebraisch
abgeschlossen ($[k(X_{1},\ldots,X_{n})(\sqrt[n]{X_{1}}):k(X)]\rightarrow\infty$,
$n\rightarrow\infty$)
\end{proof}
\begin{rem*}
Im Allgemeinen $\exists x\in U\subset X$ offen mit $\{x\}\subset U$
abgeschlossen, aber $\{x\}\subset X$ \emph{nicht} abgeschlossen ($X=\Spec\mathcal{O}$,
$\mathcal{O}$ DVR, $U=\{x\}$, $x=\eta$ generischer Punkt.) Für
$X$ lokal von endlichem Typ über $k$ (nicht notwendig algebraisch
abgeschlossen) kann nach der Proposition \emph{nicht} geschehen.
\end{rem*}

\section{Sehr dichte Teilmengen}
\begin{defn}[29]
  Sei $X$ topologischer Raum. Eine Teilmenge $Y\subset X$ heißt \textbf{sehr
    dicht}, falls die folgenden äquivalenten Bedingungen gelten:
  \begin{enumerate}
  \item $U\mapsto U\cap Y$ definiert eine Bijektion:
    \[
      \{\text{offenen Teilmengen in }X\}\leftrightarrow\{\text{offene Teilmengen in }Y\}.
    \]
  \item $F\mapsto F\cap Y$ definiert eine Bijektion:
    \[
      \{\text{abgeschlossene Teilmengen in }X\}\leftrightarrow\{\text{abgeschlossene Teilmengen in }Y\}.
    \]
  \item Für alle $F\subseteq X$ abgeschlossen gilt: $F=\overline{F\cap Y}$.
  \item Jede lokal abgeschlossene Teilmenge $Z\neq\emptyset$ von $X$ enthält
    einen Punkt aus $Y$.
  \end{enumerate}
\end{defn}

\begin{proof}
  Die Äquivalenz von $(i)$, $(ii)$ und $(iii)$ ist klar.
  \begin{itemize}
  \item $(iii)\Rightarrow(iv)$
  
  Für abgeschlossene Teilmengen $F'\subsetneq F$ von $X$ setze $Z:=F\backslash F'$.
    Angenommen $(F\cap Y)\backslash(F'\cap Y)=Z\cap Y=\emptyset$. $\Longrightarrow F\cap Y=F'\cap Y$.
    $(iii)\Longrightarrow F=F'$, Widerspruch.
  \item $(iv)\Rightarrow(ii)$
  
  Sei $F,F'\subset X$ abgeschlossen mit $F\cap Y=F'\cap Y$. $\Longleftrightarrow((F\cup F')\backslash(F\cap F'))\cap Y=\emptyset$.
    $\Longrightarrow(F\cup F')\backslash(F\cap F')=\emptyset$. $\Longrightarrow F=F'$.
  \end{itemize}
\end{proof}
\begin{prop}[30]
  Sei $X$ l.v.e.T über $k$ algebraisch abgeschlossen. Dann ist $|X|$
  sehr dicht in $X$.
\end{prop}

\begin{proof}
  Zeige: Bedingung $(iv)$. Sei $\emptyset\neq A\subset X$ lokal abgeschlossen.
  Ohne Einschränkung: 
  \[
    A\subset_{\text{abg. }}U=\Spec A\subset_{\text{off.}}X.
  \]

  Nach Voraussetzung ist $A$ endlich-erzeugte $k$-Algebra. $\emptyset\neq A=V(\mathfrak{a})$
  mit $\mathfrak{a}\subset\mathfrak{m}\subset A$ für ein maximales
  Ideal $\mathfrak{m}$. $\Longrightarrow V(\mathfrak{a})$ enthält
  abgeschlossenen Punkt $x\in\mathfrak{m}$. Proposition 28 $\Longrightarrow x$
  ist abgeschlossen in $X$, da $\kappa(x)=k$%
  \begin{comment}
    unlesbar
  \end{comment}
  . $\Longrightarrow A\cap|X|\neq\emptyset$.
\end{proof}

\section{Prävarietäten als Schemata}

Wir wollen einen Funktor von der Kategorie der Prävarietäten in die
Kategorie der Schemata sodass, wenn wir eine geweissen Unterkategorie
von $\sh$ betrachten, eine Äquivalenz von Kategorien entsteht.\medskip{}

\textbf{Erinnerung: }$k=\overline{k}$: $\mathbb{A}_{k}^{2}=\Spec(k[X,Y])$
besteht aus
\begin{itemize}
\item Punte des $\mathbb{A}^{2}(k)$ $\leadsto$ maximale Ideale, 0-dimensionale
  Teilmengen.
\item Irreduzible Kurve $f(x,y)=0$ $\leadsto$ Primideale, 1-dimensionale
  Teilmengen.
\item Generischer Punkt 0 $\leadsto$ 2-dimensionale Teilmengen.
\end{itemize}
Wie können wir die zusätzlichen Punkte für den Funktor
\[
  \pres\longrightarrow\schs
\]

präzisieren? Sei $X$ ein topologischer Raum, in dem alle Punkte abgeschlossen
sind. Betrachte 
\[
  t(X)=\{Z\subset X\mid Z\text{ irreduzibel abgeschlossen}\},
\]

versehen mit der Topologie: I$t(X)\supset t(Z)$, $Z\subseteq_{\text{abg.}}X$
bilden die abgeschlossenen Mengen. Überprüfe: $Z_{1},Z_{2},Z_{i}\subset X$
abgeschlossen $\Longrightarrow t(\cap_{i}Z_{i})=\cap_{i}t(Z_{i})$,
$t(Z_{1}\cup Z_{2})=t(Z_{i})\cup t(Z_{2})$. Ist $f:X\rightarrow Y$
stetig, so auch
\begin{align*}
  t(f):t(X) & \longrightarrow t(Y)\\
  Z & \longmapsto\overline{f(Z)}
\end{align*}

denn:
\begin{enumerate}
\item $\overline{f(Z)}$ irreduzibel: Sei $\overline{f(Z)}=A_{1}\cup A_{2}$,
  $A_{i}\neq\emptyset$. Dann existiert $z_{1},z_{2}\in Z$ mit $f(z_{i})\in A_{i}$,
  denn sonst gilt $f(z)\subseteq A_{1}$. $Z\subseteq f^{-1}(f(Z))$
  abgeschlossen. $\Longrightarrow Z=(f^{-1}(A_{1})\cap Z)\cup(f^{-1}(A_{2}\cap Z))$,
  Widerspruch.
\item Sei $t(Y')\subseteq t(Y)$ abgeschlossen. $t(f)^{-1}(t(Y))=\{Z\in t(X)$,
  $\overline{f(Z)}\in t(Y')\}$ denn:
  \begin{itemize}
  \item[,,$\subseteq$``] $\overline{f(Z)}\subset Y'$ $\Longrightarrow Z\in f^{-1}(\overline{f(Z))}\subset f^{-1}(Y)=t(f^{-1}(Y))$
  \item[,,$\supseteq$``] $z\in f^{-1}(Y)$ abgeschlossen $\Longrightarrow f(Z)\in\overline{f(Z)}\subset\overline{Y'}\subset Y'$.
  \end{itemize}
\end{enumerate}
Wir erhalten einen Funktor
\[
  t:\topcp\longrightarrow\Top
\]

Die irreduziblen Mengen von $f(X)$ sind gerade die $t(X)$, $Z\subseteq X$
irreduzibel. $Z\in t(Z)$ ist der eindeutige generische Punkt. Sei
\begin{align*}
  \alpha_{X}:X & \longrightarrow t(X)\\
  x & \longmapsto\{x\}\text{ irred. abg.}
\end{align*}

So ist die Abbildung
\begin{align*}
  \text{\{abg. Tm. von }f(X)\} & \longrightarrow\text{\{abg. Tm. von }X\}\\
  A=t(Z) & \longmapsto\alpha_{X}^{-1}(A)=\{x\in X:\{x\}\in t(Z)\}=Z
\end{align*}

eine Bijektion. $\Longrightarrow\alpha_{X}$ ist Homömorphismus von
$X$ auf die abgeschlossenen Punkte $|t(X)|$ von $t(X)$ {[}irred.
abg. Teilmengen $Z$ von $X$, die in $t(X)$ abgeschlossen sind.{]}

Es ist $\{Z\}=t(Z')$ für ein $Z'\subset X$ abgeschlossen. $\Longrightarrow$
Nur ein Punkt $x\in X$ in $Z$, sonst $\{x\}\subsetneq Z\subset Z'$
beide in $t(Z')$.

Es ist $|t(X)|\subset t(Y)$ eine sehr dichte Menge (Bijektion oben).
\begin{thm}[31]
  Der Funktor $X\mapsto(t(X),(\alpha_{X})_{\ast}\mathcal{O}_{X})$
  induziert eine Äquivalenz von Kategorien:
  \begin{align*}
    t:\{\prek\} & \overset{1:1}{\longleftrightarrow}\text{\{integere }k\text{-Schemata v. endl. Typ\}}\\
    \{\affk\} & \overset{1:1}{\longleftrightarrow}\text{\{affine }k\text{-Schemata v. endl. Typ}\}
  \end{align*}
\end{thm}

\begin{proof}
  Ist $X$ eine affine Varietät über $k$ mit $\Gamma(X)=A$, so ist
  $X=\maxspec(A)$. $\Longrightarrow t(X)=\Spec A$ (vgl Kapitel I),
  $\mathcal{O}_{X}(D(f))=A_{f}$, $f\in A$. $\Longrightarrow$ Behauptung
  im affinen Fall.

  Ist $f:X\rightarrow Y$ Morphismus von Prävarietäten, so erhalten
  wir 
  \begin{align*}
    t(f):t(X) & \longrightarrow t(Y),\\
    (\alpha_{Y})_{\ast}\mathcal{O}_{Y} & \longrightarrow t(f)_{\ast}((\alpha_{X})_{\ast}\mathcal{O}_{X})
  \end{align*}

  Morphismus lokal geringter Räume, da ein Morphismus von Garben auf
  $X$ und $Y$ durch Komposition von Abbildungen gegeben ist!

  Quasi-inverser Funktor $(X,\mathcal{O}_{X})\mapsto(X(k),\mathcal{O}_{X(k)}=\alpha^{-1}\mathcal{O}_{X})$
  geringter Raum. \textbf{(1)} $\alpha^{-1}(U)=U\cap(X)\overset{1:1}{\longleftrightarrow}U$
  offene Teilmenge. \textbf{Behauptung}: Bild $(X(k),\mathcal{O}_{X(k)})$
  ist Raum mit Funktionen: Sei $V\subseteq U\subseteq X$ offen. \textbf{(2)}
  Das Diagramm
  \[
    \xymatrix{\mathcal{O}_{X(k)}(U\cap X(k))\ar[d]_{\res}\ar[r] & \Abb(U\cap X(\xi),\xi)\ar[d]^{\res}\\
      \mathcal{O}_{X(k)}(V\cap X(k))\ar@{^{(}->}[r] & \Abb(V\cap X(k),k)
    }
  \]

  kommutiert. Dazu $f\in\mathcal{O}_{X(k)}(U\cap X(k))\overset{(1)}{=}\mathcal{O}_{X}(U)$,
  wir assoziieren es der Abbildung 
  \[
    U\cap X(k)\longrightarrow k,\quad x\mapsto f(x):=\pi_{x}(f),
  \]

  mit
  \[
    \xymatrix{\pi_{x}:\mathcal{O}_{X}(U)\ar[r]\ar[d]^{\res} & \mathcal{O}_{X,x}\ar[r] & \kappa(x)=k\\
      \mathcal{O}_{X}(V)\ar[ur]
    }
  \]

  $\Longrightarrow(2)$. \textbf{(3)} $f,g$ mit derselben Funktion
  \[
    f\equiv g:U'\rightarrow k\overset{!}{\Longrightarrow}f=g
  \]

  Garbenaxiom $\Longrightarrow$ kann lokal überprüft werden: $U=\Spec A$
  und $\pi_{x}(f)=\pi_{x}(g)$ für alle $x\in\maxspec$ 
  \[
    \Longrightarrow f-g\in\bigcap_{\mathfrak{m}\in\maxspec(A)}\mathfrak{m}=\nil(A)=0,
  \]

  da $A$ lokal reduzierte $k$-Algebra. Da sich $X$ durch endlich
  viele affine Schemata der Form $\Spec A$, $A$ integer endlich erzeugte
  $k$-Algebra, überdecken lässt, ist der Raum mit Funktion $X(k)$
  eine Prävarietät. Die Konstruktion ist funktoriell, da jede Menge
  von Schemata von endlichem Typ über $K$ abgeschlossene Punkte auf
  abgeschlossene Punkt schickt nach Proposition 28.

  Um zu zeigen, dass beide Funktoren Quasi-Inverse zueinander sind,
  benutze den affinen (Varietät/Schemata) Fall, so wie die Garbenaxiome.
\end{proof}
\begin{rem*}[32]
  \mbox{}Es gilt:

  \begin{align*}
    \kappa(x) & =\kappa(X(k))\\
    \mathbb{A}_{k}^{n} & \longleftrightarrow\mathbb{A}(k)\\
    \mathbb{P}_{k}^{n} & \longleftrightarrow\mathbb{P}^{n}(k)
  \end{align*}
\end{rem*}

\section*{Unterschemata und Immersion (Einbettungen)}

\section{Offene und abgeschlossene Einbettung}
\begin{defn}[33]
  Ein Morphismus $j:Y\rightarrow X$ von Schemata heißt \textbf{offene
    Einbettung}, falls die unterliegende stetige Abbildung ein Homöomorphismus
  von $Y$ auf eine \emph{offene} Menge $U\subset X$ ist, sowie der
  Garbenhomomorphismus $\mathcal{O}_{X}\rightarrow j_{\ast}\mathcal{O}_{Y}$
  einen Isomorphismus $\mathcal{O}_{X|U}\cong j_{\ast}\mathcal{O}_{Y}$
  von Garben über $U$ induziert.
\end{defn}

,,$j$ induziert Isomorphismus zu $Y$ und offenen Unterschemata
$U$``
\begin{defn}[34]
  Sei $(X,\mathcal{O}_{X})$ ein geringter Raum. Eine Untergarbe $\mathcal{I}\subset\mathcal{O}_{X}$
  heißt \textbf{Idealgarbe}, falls $\Gamma(U,\mathcal{I})\unlhd\Gamma(U,\mathcal{O}_{X})$
  Ideal ist für alle $U\subseteq X$ offen. Es bezeichne $\mathcal{O}_{X}/\mathcal{I}$
  die Quotientengarbe assoziiert von der Prägarbe $U\mapsto\mathcal{O}_{X}(U)/\mathcal{I}(U)$.
  Dies ist eine Prägarbe mit $\mathcal{O}_{X}\rightarrow\mathcal{O}_{X}/\mathcal{I}$
  surjektiv, denn auf Halme:
  \[
    \underset{\underset{x\in U}{\longrightarrow}}{\lim}(\mathcal{O}_{X}(U)\twoheadrightarrow\mathcal{O}_{X}(U)/\mathcal{I}(U))=\mathcal{O}_{X,x}\twoheadrightarrow(\mathcal{O}_{X}/\mathcal{I})_{x}.
  \]
\end{defn}

\begin{defn}[35]
  Sei $X$ ein Schemata.
  \begin{enumerate}
  \item Ein \textbf{abgeschlossenes Unterschemata von $X$ }ist gegeben durch
    eine abgeschlossene Menge $Z\subseteq X$ ($i:Z\rightarrow X$ Inklusion),
    sowie eine Garbe $\mathcal{O}_{Z}$ auf $Z$, sodass $(Z,\mathcal{O}_{Z})$
    ein Schemata und $i_{\ast}\mathcal{O}_{Z}\cong\mathcal{O}_{X}/I$
    für eine Idealgarbe $I\subset\mathcal{O}_{X}$.
  \item Ein Morphismus $i:Z\rightarrow X$ von Schemata heißt \textbf{abgeschlossene
      Einbettung}, falls die unterliegende stetige Abbildung einen Homöomorphismus
    zwischen $Z$ und eine abgeschlossene Teilmenge von $X$ ist, und
    der Garbenhomomorphismus $i^{\flat}:\mathcal{O}_{X}\rightarrow i_{\ast}\mathcal{O}_{X}$
    surjektiv ist.
  \end{enumerate}
  Ist $Z\subseteq X$ ein abgeschlossenes Unterschemata wie in (1),
  so ist $(i,i^{\flat})$ eine abgeschlossene Einbettung. Umgekehrt
  bestimmt jede abgeschlossene Einbettung einen Isomorphismus von seiner
  Quelle auf ein eindeutiges abgeschlossenes Unterschemata seines Ziels.

  \textbf{Warnung:} Nicht für jede Idealgarbe $\mathcal{I}$ ist
  \[
    (Z=\supp\mathcal{O}_{X}/\mathcal{I},\mathcal{O}_{X}/\mathcal{I})
  \]

  ein Schema. Später: gilt gdw. $\mathcal{I}$ quasi-kompakt ist.
\end{defn}

\begin{thm}[36]
  Sei $X=\Spec A$. Dann ist die Abbildung
  \begin{align*}
    \text{\{Ideale }A\} & \overset{1:1}{\longleftrightarrow}\{\text{abg. Unterschemata von }X\}\\
    \mathfrak{a} & \longmapsto V(\mathfrak{a})\cong\Spec(A/\mathfrak{a})
  \end{align*}

  eine Bijektion. Insbesondere ist jedes abgeschlossene Unterschemata
  eines affinen Schematas affin.
\end{thm}

\begin{proof}
  Sei $Z$ ein abgeschlossenes Unterschemata, $i:Z\hookrightarrow X$
  Inklusion. Definition $\Longrightarrow\mathcal{O}_{X}\twoheadrightarrow i_{\ast}\mathcal{O}_{Z}$
  surjektiv. Sei:
  \[
    \mathcal{I}_{Z}:=\ker(\mathcal{O}_{X}(X)\rightarrow\Gamma(X,i_{\ast}\mathcal{O}_{Z})=\Gamma(Z,\mathcal{O}_{Z}))\unlhd A
  \]

  Ideal. Falls $Z$ von der Form $V(\mathfrak{a})$ ist (was zu zeigen
  ist!) gilt $\mathcal{I}_{Z}=\mathfrak{a}$. Daher reicht z.z. $Z=V(\mathcal{I}_{Z})$.
  \textbf{Dazu:
    \[
      \xymatrix{A\ar[r]^{\varphi}\ar@{->>}[dr] & \Gamma(Z,\mathcal{O}_{Z})\\
        & A/\mathcal{I}_{Z}\ar@{^{(}->}[u]
      }
    \]
  }

  faktorisiert per Definition. $\Longrightarrow$ Das Diagramm
  \[
    \xymatrix{Z\ar@{^{(}->}[r]^{i}\ar[rd] & X\\
      & \Spec(A/\mathcal{I}_{Z})\ar@{^{(}->}[u]
    }
  \]

  kommutiert. Es ist $\Mor(Z,\Spec A)=\Hom(A,\Gamma(Z,\mathcal{O}_{Z}))$,
  ohne Einschränkung: $\mathcal{I}_{Z}=0$ (sonst ersetze $A$ durch
  $A/\mathcal{I}_{Z}$). Zu zeigen: $Z\hookrightarrow X=V(\mathfrak{a})$
  ist ein Isomorphismus.

  Wir wissen: die unterliegende stetige Abbildung topologischer Räume
  ist injektiv und abgeschlossen. ($A\subset_{\text{abg.}}Z\subset X$
  $\Longrightarrow A\subset X$ abg.) Bleibt zu zeigen: surjektiv.

  Sei $U\subseteq Z$ offen mit $(U,\mathcal{O}_{X|U})$ affin. So gilt:
  \begin{align*}
    U\subset U\backslash D(\varphi(s)|_{U}) & =V_{U}(\varphi(s)|_{U})\\
                                            & =\varphi(s)|_{U}\in\mathcal{O}_{Z}(U)\text{ nilpotent}.
  \end{align*}

  Endliche Überdeckung von $Z$ durch affine Schemata $\Longrightarrow\varphi(s^{N})=0$.
  $\varphi$ injektiv $\Longrightarrow s^{N}=0$ bzw. $V(s)=X$. $Z$
  abgeschlossen in $X$ $\Longrightarrow i(Z)=X$.

  \textbf{Behauptung: }Der Homomorphismus von Garben $\mathcal{O}_{X}\rightarrow\mathcal{O}_{Z}$
  ist bijektiv. Reicht zu zeigen: injektiv (da surjektiv nach Voraussetzung).

  Sei $x\in X$ beliebig, $\mathcal{O}_{X,x}=A_{\mathfrak{p}_{x}}$.
  Sei $\frac{g}{1}\in\ker(\mathcal{O}_{X,x}\rightarrow\mathcal{O}_{Z,x}$).
  Überdecke
  \[
    Z=U\cup\bigcup_{i\in I}U_{i},\quad\#I<\infty
  \]

  mit:
  \begin{enumerate}
  \item $(U,\mathcal{O}_{Z\mid U})$, $(U_{i},\mathcal{O}_{Z\mid U_{i})}$
    affin für alle $i\in I$;
  \item $x\in U$, $\varphi(g)|_{U}=0$.
  \end{enumerate}
  Wähle $s\in A$ mit $x\in D(s)\subseteq U$. \textbf{Behauptung: }$\varphi(s^{N}g)=0$
  für $N>0$. Mit $\varphi$ injektiv folgt dann $s^{N}g=0$, und $\frac{g}{1}=0$
  in $\mathcal{O}_{X,x}$ da $s$ eine Einheit ist in $\mathcal{O}_{X,x}$.
  \begin{itemize}
  \item Nach (2) ist $\varphi(g)=0$, d.h. $\varphi(s\cdot g)|_{U}=\varphi(s)|_{U}\cdot\underbrace{\varphi(g)|_{U}}_{=0}=0$.
  \item $D_{U_{i}}(\varphi(s)|_{U_{i}})=D(s)\cap U_{i}\subseteq U\cap U_{i}$,
    also $\varphi(g)|_{D_{U_{i}}(\varphi(s)|_{U_{i}})}=0$, d.h. $\frac{\varphi(g)}{1}=0$
    in $\mathcal{O}_{Z}(U_{i})_{\varphi(s)|_{U_{i}}}$. $\Longleftrightarrow\varphi(s)|_{U_{i}}^{N_{i}}\varphi(g)=\varphi(s^{N_{i}}g)=0$
    (Die Indexmenge $I$ ist endlich). Setze $N:=\max_{i\in I}\{1,N_{i}\}$.
  \end{itemize}
\end{proof}

\section{Unterschemata und Einbettung}

Offene und abgeschlossene Unterschemata sind Spezialfälle von \emph{lokal
  abgeschlossene} Unterschemata.
\begin{defn}[37]
  \mbox{}
  \begin{enumerate}
  \item Sei $X$ ein Schemata. Ein \textbf{Unterschemata} von $X$ ist ein
    Schemata $(Y,\mathcal{O}_{Y})$, so dass $Y\subset X$ eine lokal
    abgeschlossene Teilmenge von $X$ ist, und $Y$ ein abgeschlossenes
    Unterschemata von dem offenen Unterschemata $U=X\backslash(\overline{Y}\backslash Y)\subseteq X$
    ist. Wir haben dann einen natürlichen Morphismus $Y\rightarrow X$
    von Schemata.
  \item Eine \textbf{Einbettung} $i:Y\rightarrow X$ ist ein Morphismus von
    Schemata, dessen unterlegende stetige Abbildung ein Homöomorphismus
    von $Y$ auf eine lokale abgeschlossene Teilmenge von $X$ ist, und
    sodass für alle $y\in Y$ : 
    \[
      i_{y}^{\#}:\mathcal{O}_{X,i(y)}\rightarrow\mathcal{O}_{Y,y}
    \]
    surjektiv ist.
  \end{enumerate}
\end{defn}

\begin{rem}[38]
  \mbox{}
  \begin{enumerate}
  \item Ist $Y$ ein Unterschemata von $X$, dann ist $Y\hookrightarrow X$
    eine Einbettung. Umgekehrt bestimmt jede Einbettung einen Isomorphismus
    seiner Quelle mit einem eindeutigen Unterschemata seines Ziels.
  \item Ist $Y$ ein Unterschemata von $X$, wessen unterliegende Teilmenge
    abgeschlossen in $X$ ist, dann ist $Y$ ein abgeschlossenes Unterschemata
    von $X$.
  \item Das Analogon von $(ii)$ für offene Unterschemata ist i.A. falsch.
  \item Jede Einbettung $i:Y\hookrightarrow X$ faktorisiert als:
    \[
      \xymatrix{Y\ar@{^{(}->}[r]^{i}\ar@{^{(}->}[rd] & X\\
        & U=X\backslash(\overline{i(Y)}\backslash i(Y))\ar@{^{(}->}[u]
      }
    \]
  \end{enumerate}
\end{rem}

\begin{defn}[39]
  Sei $X$ ein Schemata und $Z,Z'$ Unterschemata. Wir sagen $Z'$
  \textbf{majorisiert} $Z$, wenn die Inklusion $Z\hookrightarrow X$
  faktorisiert als:
  \[
    \xymatrix{Z\ar@{^{(}->}[r]\ar[rd] & X\\
      & Z'\ar@{^{(}->}[u]
    }
    .
  \]
\end{defn}

\begin{rem}[40]
  Sei \textbf{P} die Eigenschaft eines Schemata-Morphismus, eine affine
  Einbettung, bzw. abgeschlossene Einbettung, bzw. Einbettung zu sein.
  Dann:
  \begin{enumerate}
  \item Die Eigenschaft \textbf{P} ist lokal auf der Basis, d.h. für $f:Z\rightarrow X$
    Morphismus, $X=\bigcup_{i\in I}U_{i}$ offene Überdeckung hat $f$
    die Eigenschaft \textbf{P} $\Longleftrightarrow\forall i$ hat $f^{-1}(U_{i})\rightarrow U_{i}$
    die Eigenschaft \textbf{P}.
  \item Die Komposition zweier Morphismen mit Eigenschaft \textbf{P} hat Eigenschaft
    \textbf{P}.
  \end{enumerate}
\end{rem}

\begin{example}[41]
  \mbox{}
  \begin{enumerate}
  \item Sei $I\subseteq R[T_{0},\ldots,T_{n}]$ homogenes Ideal. Dann ist
    $V_{+}(I)\subseteq\mathbb{P}_{R}^{n}$ ein abgeschlossenes Unterschemata
    von $\mathbb{P}_{R}^{n}$. (Nach Bemerkung 40.1, denn $V_{+}(I)\cap U_{i}\subseteq U_{i}$
    abgeschlossen.)
  \item Alle Unterschemata eines $k$-Schematas $X$ von endlichem Typ sind
    selbst von endlichem Typ. 
  \end{enumerate}
\end{example}

\section{Projektive und quasi-projektive Schemata über einen Körper}
\begin{defn}[42]
Sei $k$ ein Körper.
\begin{enumerate}
\item Ein $k$-Schemata $X$ heißt \textbf{projektiv} wenn es ein $n\geq0$
und eine abgeschlossene Einbettung $X\hookrightarrow\mathbb{P}_{k}^{n}$
gibt.
\item Ein $k$-Schemata $X$ heißt \textbf{quasi-projektiv }wenn es ein
$n\geq0$ und eine Einbettung $X\hookrightarrow\mathbb{P}_{k}^{n}$
gibt.
\end{enumerate}
\end{defn}

\begin{example}[43]
\mbox{}
\begin{enumerate}
\item Für ein homogenes Ideal $I$ sind $V_{+}(I)$ projektive Schemata
(Beispiel 41).
\item Sei $X=\Spec A$ affines $k$-Schemata von endlichem Typ. Dann ist
$X$ quasi-projektiv: $A\cong k[T_{1},\ldots,T_{n}]\backslash\mathfrak{a}$,
\[
\xymatrix{X\ar@{^{(}->}[r]\ar[rd] & \mathbb{A}^{n}\ar[d]^{j}\\
 & \mathbb{P}^{n}
}
\]
\end{enumerate}
\end{example}

\section{Reduzierte Unterschemata}

\[
  \Spec K[X,Y]\supset\Spec(K[X,Y]/Y^{2})\supset\Spec(K[X,Y]/Y)
\]

\begin{question*}
  Gibt es ein ,,kleinstes`` Unterschemata?
\end{question*}
Setze $\mathcal{N}_{X}\subset\mathcal{O}_{X}$, Garbifizierung der
Prägarben:
\[
  U\mapsto\nil(\Gamma(U,\mathcal{O}_{X})),\quad U\subseteq X\text{ offen}
\]

Definiere $X_{\red}:=(X,\mathcal{O}_{X}/\mathcal{N}_{X})$.

\begin{prop}[44]
  \mbox{}
  \begin{enumerate}
  \item Der geringte Raum $X_{\red}=(X,\mathcal{O}_{X}/\mathcal{N})$ ist
    ein Schema, also ein abgeschlossenes Unterschema von $X$ mit demselben
    topologischen Raum wie $X$.
  \item Falls $X'\subset X$ ein weiteres solches Unterschema ist, dann gibt
    es eine abgeschlossene Einbettung $f:X_{\red}\rightarrow X'$, sodass
    das Diagramm
    \[
      \xymatrix{X_{\red}\ar@{^{(}->}[r]\ar[d] & X\\
        X'\ar@{^{(}->}[ur]
      }
    \]
    kommutiert.
  \item $X_{\red}$ ist reduziert und heißt das \textbf{unterliegend reduzierte
      Unterschema von $X$.}
  \item Falls $X=\Spec A$ affin, gilt $X_{\red}=\Spec(A/\nil(A))$.
  \end{enumerate}
\end{prop}

\begin{proof}
  Ohne Einschränkung sei $X=\Spec A$ $\Longrightarrow U\mapsto\nil(\Gamma(U,\mathcal{O}_{X}))$
  ist bereits eine Garbe, da 
  \[
    \nil(\mathcal{O}_{X}(D(f))=\nil(A_{f})=\nil(A)A_{f}
  \]

  für alle $f\in A$. $\Longrightarrow X_{\red}=\Spec(A/\nil A)$ offensichtlich
  reduziert.\textbf{ Universelle Eigenschaft:} zu zeigen $\mathcal{O}_{X}\rightarrow\mathcal{O}_{X}/\mathcal{N}$
  faktorisiert:
  \[
    \xymatrix{\mathcal{O}_{X}\ar[r]\ar[rd] & \mathcal{O}_{X}/\mathcal{N}\\
      & \mathcal{O}_{X'}\ar[u]
    }
    ,
  \]

  d.h. $\ker(\mathcal{O}_{X}\rightarrow\mathcal{O}_{X'})\subset\mathcal{N}$.
  Es reicht zu zeigen:
  \[
    \ker(\mathcal{O}_{X}(U)\rightarrow\mathcal{O}_{X'}(U))\subset\Gamma(U,\mathcal{N})
  \]

  für alle $U$ offen affin. Ohne Einschränkung $X=\Spec A$, $X'$
  abgeschlossenes Unterschema $\Longrightarrow$ affin: $X'=\Spec B$.

  Zu zeigen: $\ker(A\rightarrow B)\subset\nil A$. Da nach Voraussetzung
  $\Spec B\rightarrow\Spec A$ bijektiv ist, folgt:
  \[
    \ker(A\rightarrow B)\subset\bigcap_{\mathfrak{g}\in\Spec A}\mathfrak{g}=\nil A
  \]
\end{proof}
\begin{cor}[45]
  $(X_{\red},i_{X}:X_{red}\rightarrow X)$ ist durch die universelle
  Eigenschaft eindeutig bis auf eindeutige Isomorphie bestimmt.
\end{cor}

\begin{lem}[46]
  Jede Einbettung $i:Z\rightarrow X$ ist ein Monomorphismus in $\sch$.
\end{lem}

\begin{proof}
  \mbox{}
  \begin{itemize}
  \item Stetige Abbildung $Z\hookrightarrow X$ klar.
  \item Die Garbenabbildung $i_{Y}^{\#}$ ist surjektiv.
  \end{itemize}
\end{proof}
% 
\begin{proof}[Beweis von Korollar 45]
  Sei $X'_{\red}$ ein weiteres Schema mit universeller Eigenschaft
  \[
    \exists f:X_{\red}\rightarrow X'_{\red},\quad g:X'_{\red}\rightarrow X_{\red}
  \]

  so dass
  \[
    \xymatrix{X_{\red}\ar[r]^{f}\ar[rd]_{i_{X}} & X'_{\red}\ar[r]^{g}\ar[d]^{i_{X'}} & X_{\red}\ar[ld]^{i_{X}}\\
      & X
    }
    ,\quad i_{X}\circ(g\circ f)=i_{X}\circ\id_{X_{\red}}
  \]

  $i_{X}$ Monomorphismus $\Longrightarrow g\circ f=\id_{X_{\red}}=f\circ g$.
  Auf $(X_{\red},i_{X})=\{\id\}$ $\Longrightarrow$ Eindeutig.
\end{proof}
$(\cdot)_{\red}$ ist ein Funktor, wie die folgende Proposition zeigt:
\begin{prop}[47]
  Sei $f:X\rightarrow Y$ ein Schemata-Morphismus. Dann gibt es:
  \[
    \xymatrix{X_{\red}\ar[r]^{f_{\red}} & Y_{\red}\\
      X_{\red}\ar@{^{(}->}[r]^{i_{X}}\ar[d]_{f_{\red}} & X\ar[d]^{f}\\
      Y_{\red}\ar@{^{(}->}[r]_{i_{Y}} & Y
    }
    ,\quad\text{d.d.}
  \]

  Für weitere Morphismen $g:Y\rightarrow Z$ gilt
  \[
    (g\circ f)_{\red}=g_{\red}\circ f_{\red}.
  \]
\end{prop}

\begin{proof}
  $i_{Y}$ Monomorphismus $\Longrightarrow f_{\red}$ eindeutig. \textbf{Existenz:
  }Nach Verklebungs-Lemma $\Longrightarrow$ ohne Einschränkung $X=\Spec A$,
  $Y=\Spec B$, $f\hat{=}\ \varphi:B\rightarrow A$.

  $\Longrightarrow\varphi(\nil(B))\subset\nil(A)$

  $\Longrightarrow\varphi_{\red}:B/\nil(B)\rightarrow A/\nil(A)$

  $\Longrightarrow f_{\red}:\Spec(A/\nil(A))\rightarrow\Spec(B/\nil(B))$.
\end{proof}
\begin{prop}[48]
  Sei $X$ Schemata, $Z\subset X$ lokal abgeschlossene Teilmenge.
  Dann existiert ein eindeutig bestimmtes reduziertes Unterschema mit
  topologischem Raum $Z$.
\end{prop}

\begin{proof}
  Eindeutigkeit: Korollar 45. Existenz: Verklebungslemma $\Longrightarrow$
  ohne Einschränkung $X=\Spec A$ affin und $Z\subset X$ abgeschlossen
  (sonst Überdeckung von $X$ zu $Z\subset_{\text{abg.}}U\subset_{\text{abg.}}X$)
  $\Longrightarrow\exists\mathfrak{a}\subset A$ so dass $Z=V(\mathfrak{A})$
  $\Longrightarrow Z'=\Spec(A/\mathfrak{a})$ ist abgeschlossenes Unterschema
  von $X$ mit topologischem Raum $Z$. Satz 44 $\Longrightarrow\exists Z'_{\red}\subset Z'\subset X$.
\end{proof}
Damit besitzt für ein lokal abgeschlossene Teilmenge die geordnete
Menge (bzgl. Inklusion) von Unterschema, denen topologischen Raum
$Z$ umfassen, ein eindeutiges minimales Element $Z_{\red}$, das
\textbf{reduzierte Unterschema} mit unterliegendem Raum $Z$.


\chapter{Faserprodukte}
\label{chap:faserprodukte}

\chapter{Faserprodukte}

\section{Der ,,Punkte-Funktor``}

Kontravarianter Funktor, $\forall X\in\sch$,
\begin{align*}
  h_{X}:(\sch)^{\op} & \longrightarrow\set\\
  S & \longmapsto h_{X}(S):=\Hom_{\sch}(S,X)\\
  (f:T\rightarrow S) & \longmapsto(\Hom(S,X)\overset{f^{\ast}}{\rightarrow}\Hom(T,X),\ g\mapsto g\circ f)
\end{align*}

$f^{\ast}=h_{X}(S)$ heißen $S$-wertige Punkte von $X$. \textbf{Notation:
}$X(S)$, $X(R)$, falls $S=\Spec R$.

\textbf{Relative Version:} $X\in\schs0$, $S_{0}$ fixes Schemata.
\begin{align*}
  \schs0 & \longrightarrow\set\\
  S & \longmapsto\Hom_{S_{0}}(S,X)
\end{align*}

\textbf{Notation: $X_{S_{0}}(S)$, $X_{R_{0}}(S)$, $X_{S_{0}}(R)$,
  $X_{R_{0}}(R)$}
\begin{example}
  Sei $k$ algebraisch abgeschlossen, $X/k$ von endlichem Typ, $x\in X_{k}(k)$.
  Dann ist
  \[
    \im(\Spec k\overset{x}{\longrightarrow}X)\in X
  \]
  abgeschlossener Punkt. $x\mapsto\im(x)$ liefert Bijektion, $X_{k}(k)\rightarrow|X|$
  Menge der abgeschlossenen Punkte.
\end{example}

\begin{example}
  Sei $X=\mathbb{A}^{n}=\Spec(\mathbb{Z}[T_{1},\ldots T_{n}])$. Dann:
  \begin{align*}
    \mathbb{A}^{n}(S) & =\Hom_{\sch}(S,\mathbb{A}^{n})=\Hom_{\ring}(\mathbb{Z}[T_{1},\ldots,T_{n}],\mathcal{O}_{S}(S))\\
                      & =\Gamma(S,\mathcal{O}_{S})^{n}
  \end{align*}
\end{example}

\begin{example}
  Sei $X=\Spec(R[T_{1},\ldots,T_{n})/(f_{1},\ldots,f_{m}))$, $S$ ein
  $R$-Schema. Dann:
  \begin{align*}
    X_{R}(S) & =\Hom_{R\text{-Alg}}(R[I]/(f),\mathcal{O}_{S}(S))\\
             & =\{s\in\mathcal{O}_{S}(S)^{n}\mid f_{1}(s)=\cdots=f_{m}(s)=0\}
  \end{align*}
\end{example}

\begin{example}
  Sei $X=\Spec\mathbb{Z}[T,T^{-1}]$. Dann:
  \[
    X(S)=\Hom(\mathbb{Z}[T,T^{-1}],\mathcal{O}_{S}(S))=\Gamma(S,\mathcal{O}_{S})^{\times}.
  \]
  Hier sogar $h_{X}:\sch\rightarrow\grp$. $X$ ist eine abelsche Gruppe.
\end{example}

\section{Yoneda-Lemma}

\textbf{Ziel:} $h_{X}$ beschreibt $X$ eindeutig.

Erinnerung: $\mathcal{F},\mathcal{G}:\mathcal{A}\rightarrow\mathcal{B}$
Funktoren, natürliche Transformation $f\in\Hom(\mathcal{F},\mathcal{G})$:
\[
  f=\{f(X):\mathcal{F}(X):\rightarrow\mathcal{G}(X))_{X\in\mathcal{A}}.
\]

Wir erhalten Kategorien: $\func(\mathcal{A},\mathcal{B})$, \textbf{hier:
}$\mathcal{C}=\schs0$, $\hat{C}=\func(\mathcal{C}^{\op},\set)$.
Wir erhalten einen Funktor
\begin{align*}
  \mathcal{C} & \longrightarrow\hat{\mathcal{C}},\\
  X & \longmapsto h_{X},\\
  f & \longmapsto\text{Pullback }f^{\ast}.
\end{align*}

\begin{prop}[5]
  Sei $X\in\mathcal{C}$, $\mathcal{F}\in\hat{\mathcal{C}}$. Dann
  ist die Abbildung
  \begin{align*}
    \Hom_{\hat{C}}(h_{X},\mathcal{F}) & \longrightarrow\mathcal{F}(X)\\
    \alpha & \longmapsto\alpha(X)(\id_{X})\in\Hom(h_{X}(X),\mathcal{F}(X))
  \end{align*}

  bijektiv und funktoriell.
\end{prop}

Insbesondere ist der obige Funktor $\mathcal{C}\rightarrow\hat{\mathcal{C}}$
volltreu (wähle $\mathcal{F}=h_{Y}$!)
\begin{proof}
  Umkehrabbildung:
  \begin{align*}
    \mathcal{F}(X) & \longrightarrow\Hom_{\hat{\mathcal{C}}}(h_{X},\mathcal{F})\\
    \xi & \longmapsto\alpha_{\xi}=(\alpha_{\xi}(Y))_{Y\in\schs0}
  \end{align*}

  mit
  \begin{align*}
    \alpha_{\xi}(Y):\Hom(Y,X)=h_{X}(Y) & \longrightarrow\mathcal{F}(Y)\\
    f & \longmapsto\mathcal{F}(f)(\xi)\in\Hom(\mathcal{F}(X),\mathcal{F}(Y))
  \end{align*}
\end{proof}

\section{Faserprodukte in beliebigen Kategorien}

Sei $\mathcal{C}$ eine Kategorie, $S\in\obj(\mathcal{C})$, $f:X\rightarrow S$,
$g:Y\rightarrow S$ Morphismen.
\begin{defn}[6]
  Ein Tupel $(Z,p,q)$ mit $Z\in\obj(\mathcal{C})$ und Morphismen
  $p:Z\rightarrow X$, $q:Z\rightarrow Y$, $f\circ p=g\circ q$, hei{\small{}ßt
  }\textbf{\small{}Faserprodukt}{\small{} von $X$ und $Y$ über $S$
    (bzw. von $f,g$), falls für jedes $T\in\obj(\mathcal{C})$ und Paare
    $(u:T\rightarrow X,v:T\rightarrow Y)$ von Morphismen mit $f\circ u=g\circ v$
    genau ein Morphismus $w:T\rightarrow Z$ existiert mit $p\circ w=u$,
    $g\circ w=v$.}{\small\par}
\end{defn}

\textbf{Notation:} $X\times_{S}Y$ oder $X\times_{f,S,g}Y:=Z$, $(u,v)_{S}:=w$.
\[
  \xymatrix{T\ar@/^{1pc}/[rrd]^{u}\ar@/_{1pc}/[ddr]_{v}\ar[dr]|-{\exists_{1}}\\
    & X\times_{S}Y\ar[r]^{p}\ar[d]^{q} & X\ar[d]^{f}\\
    & Y\ar[r]_{g} & S
  }
\]

Ist $S$ ein finales Objekt in $\mathcal{C}$, so ist $X\times_{S}Y=X\times Y$
das kategorielle Produkt.
\begin{example}[7]
  \mbox{}
  \begin{enumerate}
  \item $\mathcal{C}=\set$. $X\times_{S}Y=\{(x,y)\in X\times Y\mid f(x)=g(x)(\}$.
  \item $\mathcal{C}=\Top$, $f:X\rightarrow S$, $g:Y\rightarrow S$ stetige
    Abbildungen. Versehe $\{(x,y)\mid f(x)=g(y)\}\subset X\times Y$ mit
    der Topologie induziert von der Produkttopologie auf $X\times Y$.
    Dies ist ein Faserprodukt in $\Top$.
  \end{enumerate}
\end{example}

\textbf{Ab jetzt:} Alle Faserprodukte mögen in $\mathcal{C}$ existieren.

Sei Morphismus $h:T\rightarrow S$ in $\mathcal{C}$, ein \textbf{$S$-Objekt}.
(kurz $T$, \textbf{$h$ Strukturmorphismus} von $T$.)

Für $S$-Objekte $h:T\rightarrow S$ und $f:X\rightarrow S$ schreibe
$\Hom_{S}(T,X):=X_{S}(T)$ für Morphismen $w:T\rightarrow X$ mit
$f\circ w=h$, die \textbf{$S$-Morphismen}. Nenne $X_{S}(T)$ die
\textbf{Menge der $T$-wertigen Punkte von $X$ (über $S$)}. Dies
definiert eine Kategorie $\mathcal{C}/S$ mit finalem Objekt $\id_{S}$.

\textbf{Faserprodukt }$X\times_{S}Y$ ist Produkt von $S$-Objekten
$f$ und $g$ in $\mathcal{C}/S$.
\begin{align*}
  (X\times_{S}Y)(T) & =X_{S}(T)\times Y_{S}(T)\\
  \text{(UAE)\ }\Hom_{S}(T,X\times_{S}Y) & \overset{\sim}{\longrightarrow}\Hom_{S}(T,X)\times\Hom_{S}(T,Y)\\
  w & \longmapsto(p\circ w,q\circ w)
\end{align*}

für alle $h:T\rightarrow S$.

\subsection*{Funktorialität}

Seien $X,Y,X',Y'\in\mathcal{C}/S$, $u:X\rightarrow X'$, $v:Y\rightarrow Y'$
$S$-Morphismus. $\Longrightarrow\exists_{1}$ Morphismus $u\times_{S}v$
(oder nur $u\times v$): $X\times_{S}Y\rightarrow X'\times_{S}Y'$.
d.d.
\[
  \xymatrix{X\times_{S}Y\ar[rd]|-{u\times v}\ar[r]^{p}\ar[d]_{q} & X\ar[rd]^{u}\\
    Y\ar[rd]_{v} & X'\times Y'\ar[r]\ar[d] & X'\ar[d]\\
    & Y'\ar[r] & S
  }
  \qquad\text{kommutiert.}
\]

Setze $u\times v:=(u\circ p,v\circ q)_{S}$ (Universelle Eigenschaft
von $X'\times_{S}Y'$!)\medskip{}

\textbf{Yoneda-Lemma:}
\[
  f:X\rightarrow Y\in\Mor(\mathcal{C}/S)\leftrightarrow(f_{S}(T):X_{S}(T)\rightarrow Y_{S}(T))_{T\in\mathcal{C}/S}\text{ funktoriell in }T
\]

\begin{prop}[8, Eigenschaften des Faserprodukts]
  Sei $X,Y,Z\in\mathcal{C}/S$. Dann gibt es \textbf{kanonische Isomorphismen}
  (funktoriell in $X,Y,Z$),
  \begin{enumerate}
  \item $X\times_{S}S\xrightarrow{\sim}X$
  \item $X\times_{S}Y\xrightarrow{\sim}Y\times_{S}X$
  \item $(X\times_{S}Y)\times_{S}Z\xrightarrow{\sim}X\times_{S}(Y\times_{S}Z)$
  \end{enumerate}
  auf $T$-wertigen Punkten, für alle $h:T\rightarrow S$ $S$-Objekt,
  gegeben durch:
  \begin{align*}
    X_{S}(T)\times S_{S}(T) & \xrightarrow{\sim}X_{S}(T), & (x,h)\mapsto x,\\
    X_{S}(T)\times Y_{S}(T) & \xrightarrow{\sim}Y_{S}(T)\times X_{S}(T), & (x,y)\mapsto(y,x),\\
    (X_{S}(T)\times Y_{S}(T))\times Z_{S}(T) & \xrightarrow{\sim}X_{S}(T)\times(Y_{S}(T)\times Z_{S}(T)), & ((x,y),z)\mapsto(x,(y,z)).
  \end{align*}
\end{prop}

Sei $Z\in\mathcal{C}/S$. Ein kommutatives Diagramm in $\mathcal{C}$:
\[
  \xymatrix{Z\ar[r]^{u}\ar[d]_{v}\ar@{}[dr]|{\square} & X\ar[d]^{f}\\
    Y\ar[r]_{g} & S
  }
\]

heißt \textbf{kartesisch} falls $(u,v)_{S}:Z\rightarrow X\times_{S}Y$
ein Isomorphismus ist (und dabei automatisch in $\mathcal{C}/S$).
\[
  \xymatrix{Z\ar[d]_{u}\ar@{-->}[rd]|-{(u,v)_{S}}\ar[r]^{v} & Y\\
    X & X\times_{S}Y\ar[l]^{p}\ar[u]_{q}
  }
\]

\begin{rem}
Nach dem Yoneda-Lemma ist dies äquivalent dazu, dass das Diagramm:
\[
  \xymatrix{\Hom_{\mathcal{C}}(T,Z)\ar[r]^{u(T)}\ar[d]_{v(T)} & \Hom_{\mathcal{C}}(T,X)\ar[d]^{f(T)}\\
    \Hom_{\mathcal{C}}(T,Y)\ar[r]_{g(T)} & \Hom_{\mathcal{C}}(T,S)
  }
  \qquad(*)
\]

kartesisch ist für alle $T\in\mathcal{C}$ (Beispiel 4.7). Dies ist
äquivalent zu: %
\begin{comment}
  Pfeile klappen hier nicht besonders gut, evt. Isomorphismus $\Hom_{\mathcal{C}}(T,Z)\rightarrow\cdots$
  benennen.
\end{comment}
\[
  \xymatrix{\Hom_{\mathcal{C}}(T,Z)\overset{!}{\cong}\ar[r]\ar[d]_{s}^{(u,v)_{S}(T)} & \Hom_{\mathcal{C}}(T,X)\times_{\Hom_{\mathcal{C}}(T,S)}\Hom_{\mathcal{C}}(T,Y)\\
    \Hom_{\mathcal{C}}(T,X\times_{S}Y)\ar[ur]_{f\mapsto(p\circ h,q\circ h)}
  }
\]

mit $\Hom_{\mathcal{C}}(T,X)\times_{\Hom_{\mathcal{C}}(T,S)}\Hom_{\mathcal{C}}(T,Y):=\{(h_{1},h_{2})\mid f\circ f_{1}=g\circ h_{2}\}$.
\begin{comment}
  Hier ist noch ein weiteres Diagramm, mit sich kreuzenden Pfeilen.
\end{comment}
\end{rem}

\begin{prop}[10]
  Sei das Diagramm
  \[
    \xymatrix{X''\ar[r]^{g'}\ar[d] & X'\ar[r]^{g}\ar[d]\ar@{}[dr]|{\square} & X\ar[d]\\
      S''\ar[r]_{f'} & S'\ar[r]_{f} & S
    }
  \]

  kommutativ, mit rechts ein kartesisches Diagramm. Dann gilt:
  \[
    \xymatrix{X''\ar[r]\ar[d]\ar@{}[dr]|{\square} & X'\ar[d]\\
      S''\ar[r] & S'
    }
    \quad\Longleftrightarrow\quad\xymatrix{X''\ar[r]\ar[d]\ar@{}[dr]|{\square} & X\ar[d]\\
      S''\ar[r] & S
    }
  \]
\end{prop}

\begin{proof}
  Zeige in der Kategorie $\mathcal{C}=\set$, und wende das Yoneda-Lemma,
  $(*)$ an.
\end{proof}

\section{Faserprodukte von Schemata}
\begin{prop}[11]
\end{prop}

\begin{thm}[12]
\end{thm}

\begin{cor}[13]
\end{cor}

Sei $X,X'\in\schs$, $f:X'\rightarrow X$ Morphismus in $\schs$,
$g:=f\times_{S}\id_{Y}$.
\[
  \xymatrix{Z'=X'\times_{S}Y\ar[r]^{g}\ar[d]_{p'} & Z=X\times_{S}Y\ar[r]^{q}\ar[d]_{p'} & Y\ar[d]\\
    X'\ar[r]^{f} & X\ar[r] & S
  }
\]

kommutiert. Da $q\circ g=q'$ Projektion auf $Y$ ist, ist das große
und damit auch beide Diagramme kartesisch. (Proposition 10)
\begin{prop}[14]
  $f$ induziert einen Homömorphismus von $X'$ auf $f(X')$ und:
  \begin{enumerate}
  \item $f_{x'}^{\#}:\mathcal{O}_{X,f(x')}\rightarrow\mathcal{O}_{X',x'}$
    sei surjektiv $\forall x'\in X'$ und es existiert eine offene affine
    Umgebung $U'$ von $f(x')$, sodass $f^{-1}(U')$ quasi-kompakt ist,
    oder
  \item $f_{x'}^{\#}$ ist bijektiv $\forall x'\in X'$.
  \end{enumerate}
  Dann gilt:
  \begin{enumerate}
  \item $g$ ist ein Homöomorphismus von $Z'$ auf $g(Z')=p^{-1}(f(X'))$.
  \item $\forall z'\in Z'$ haben wir das induzierte Diagramm für lokale Ringe:
    \[
      \xymatrix{\mathcal{O}_{Z',z'}\ar[d] & \mathcal{O}_{Z,g(z')}\ar[l]^{g_{z'}^{\#}}\ar[d]^{p_{g(z)}^{\#}}\\
        \mathcal{O}_{X',p'(z')} & \mathcal{O}_{X,p(g(z'))}\ar[l]^{f_{p'(z')}}
      }
    \]
  \end{enumerate}
  \begin{itemize}
  \item $g_{z'}^{\#}$ ist surjektiv;
  \item $\ker(g_{z'}^{\#})$ ist von $p_{g(z')}^{\#}(\ker f_{p'(z')}^{\#})$
    erzeugt.
  \end{itemize}
\end{prop}

\begin{proof}
  (1), (2) lassen sich lokal bzgl. $S,Y,X$ verifizieren. Ohne Einschränkung
  sei $S=\Spec R$, $X=\Spec A$, $Y=\Spec B$ affin, $X'$ quasi-kompakt.
  \[
    f\leftrightarrow\xymatrix{A\ar[r]^{\varphi}\ar@{->>}[rd]_{\varphi_{1}} & \Gamma(X,\mathcal{O}_{X})\\
      & A/\ker\varphi\ar@{^{(}->}[u]^{\varphi_{2}}
    }
    \quad R\text{-Algebren}
  \]

  $f$ faktorisiert sich als
  \[
    \xymatrix{X'\ar[r]^{f_{1}} & \Spec(A/\ker\varphi)\ar[r]^{f_{2}} & \Spec(A)=X}
    .
  \]

  Dann ist $f_{2}$ eine abgeschlossene Immersion, also surjektiv auf
  Halmen $(f_{p})_{2}$, und ein Homeomorphismus auf einer abgeschlossenen
  Teilmenge in $X$. In Situation $I$ erfüllt daher mit $f$ auch $f_{1}$
  Voraussetzung (1). Daher reicht es die folgenden 2 Fälle zu beweisen.
  \begin{enumerate}
  \item $f$ ist eine abgeschlossene Immersion ($\hat{=}$ $f_{2}$ Voraussetzung
    (1))
  \item $f_{x'}^{\#}$ ist bijektiv für alle $x'\in X'$ ($\hat{=}$ $f_{1}$
    Voraussetzung (1) + Voraussetzung (2)).
  \end{enumerate}
  \begin{lem*}
    Sei:
    \[
      \xymatrix{A\ar@{^{(}->}[r] & \Gamma(X,\mathcal{O}_{X})\\
        X'\ar[r]^{f} & \Spec(A)
      }
      ,\quad X'\text{ quasi-kompakt}
    \]

    Dann ist $f_{x'}^{\#}$ injektiv für alle $x\in X'$.
  \end{lem*}
  \textbf{Vorüberlegung. }Sei $Z$ ein Schemata und $t\in\Gamma(Z,\mathcal{O}_{Z})$,
  $Z_{T}:=\{z\mid t(z)\neq0\}\subset Z$ offen. Die Einschränkung
  \[
    \xymatrix{\Gamma(Z,\mathcal{O}_{Z})\ar[r]\ar[d] & \Gamma(Z_{t},\mathcal{O}_{Z})\\
      \Gamma(Z,\mathcal{O}_{Z})_{t}\ar@{^{(}-->}[ur]_{\rho_{t}}
    }
  \]

  definiert einen Homomorphismus $\rho_{t}$. Dieser ist injektiv, falls
  $Z$ quasi-kompakt, \textbf{denn} $Z=\bigcup U_{i}$ ist endliche
  offene affine Überdeckung. Sei $C_{i}=\mathcal{O}_{Z}(U_{i})$, $t_{i}=t|_{U_{i}}$.
  $\Longrightarrow(\prod_{i}C_{i})_{t}=\prod_{i}(C_{i})_{t_{i}}$ da
  $i$ endlich. Wir erhalten das kommutative Diagramm:
  \[
    \xymatrix{\mathcal{O}_{Z}(Z)\ar[r]\ar@{^{(}-}[d]_{\text{Garbe}} & \Gamma(Z,\mathcal{cO}_{Z})_{t}\ar[r]^{\rho_{t}}\ar@{^{(}-}[d] & \Gamma(Z_{t},\mathcal{O}_{Z})\ar@{^{(}-}[d]^{\text{Garbe}}\\
      \prod_{i}C_{i}\ar[r] & \prod_{i}(C_{i})_{t_{i}}\ar[r]_{\cong} & \prod_{i}\Gamma(D(t_{i}),\mathcal{O}_{U_{i}})
    }
  \]
  \begin{proof}[Beweis (Lemma)]
    Sei $\mathfrak{p}\subset A\cong f(x')$. Für alle $s\in A\backslash\mathfrak{p}$
    sei
    \[
      \varphi_{s}:A_{s}\longrightarrow\Gamma(X',\mathcal{O}_{X'})_{\varphi(s)}
    \]

    der injektive Homomorphismus aus $\varphi$ durch Lokalisierung in
    $s$, und sei $\psi_{s}$ die injektive Komposition
    \[
      \xymatrix{\psi_{s}:A_{s}\ar[r]^{\varphi_{s}} & \Gamma(X',\mathcal{O}_{X'})_{\varphi(s)}\ar[r]^{\rho_{\varphi(s)}} & \Gamma(X'_{\varphi(s)},\mathcal{O}_{X'}).}
    \]

    Dann ist $X'_{\varphi(s)}=f^{-1}(D(s))$. Da für $s\in A\backslash\mathfrak{p}$
    die $D(s)$ eine offene Umgebungsbasis von $f(x)'$, und da $f$ ein
    Homeomorphismus auf sein Bild ist, bilden die $X'_{\varphi(s)}$ eine
    offene Umgebungsbasis von $x'$. $\Longrightarrow\underset{\underset{s}{\longrightarrow}}{\lim}\ \Gamma(X'_{\varphi(s)},\mathcal{O}_{X'})=\mathcal{O}_{X',x'}$
    und $\underset{\underset{s}{\longrightarrow}}{\lim}\psi_{s}=f_{x'}^{\#}$
    $\Longrightarrow f_{x'}^{\#}$ injektiv.
  \end{proof}
  Zu 1.) Sei $\xymatrix{X'=\Spec(A/\mathfrak{a})\ar[r]^{f} & \Spec(A)=X}
  ,$ $\mathfrak{a}\subset A$ Ideal. Proposition 11 $\Longrightarrow Z,Z'$
  affin, und $g$ entspricht $R$-Algebren
  \[
    \xymatrix{A\otimes_{R}B\ar@{->>}[r]^{``g``} & A/\mathfrak{a}\otimes_{R}B\\
      A\ar[u]^{``p``}\ar[r]_{``f``} & A/\mathfrak{a}\ar[u]_{p'}
    }
  \]

  und $``p``(\ker``f``)\subset A\otimes_{R}B=\ker``g``$ $\Longrightarrow g$
  ist Homöomorphismus auf $g(Z')=p^{-1}(f(x'))$, $x'\in X'$. $\checkmark$\medskip{}

  Zu 2.) Es ist $f^{\#}:f^{-1}\mathcal{O}_{X}\rightarrow\mathcal{O}_{X'}$
  ein Isomorphismus bzgl. $(X',\mathcal{O}_{X'})\cong(f(x'),\mathcal{O}_{X}|_{f(x')})$
  Isomorphismus lokal geringter Räume. Leicht zu verifizieren: $(p^{-1}(f(x')),\mathcal{O}_{Z}|_{p^{-1}(f(x'))})$
  ist ein Faserprodukt von $X'$ mit $Z$ über $X$ in der Kategorie
  lokal geringter Räume, also erst recht in $\sch$. (vgl. Zusatz in
  Theorem (Existenz $X\times_{S}Y$)). $\Longrightarrow g$ Isomorphismus
  \[
    \xymatrix{(Z',\mathcal{O}_{Z})\ar[r]^{\cong} & (p^{-1}(f(x')),\mathcal{O}_{Z}|_{p^{-1}(f(x'))})}
    .
  \]
\end{proof}
\begin{example*}
  Proposition 14 gilt in folgenden Situationen.
  \begin{enumerate}
  \item $f$ ist eine Immersion von Schemata.
  \item $f$ ist der kanonische Morphismus $\Spec\mathcal{O}_{X,x}\rightarrow X$
    für ein $x\in X$, vgl. (5.4), (2.11).
  \item $f$ ist der kanonische Morphismus $\Spec\kappa(x)\rightarrow X$
    für ein $x\in X$.
  \item Komposition von Morphismen, die Proposition 14 erfüllen (und Proposition
    10).
  \end{enumerate}
\end{example*}
\section{Beispiele}

\paragraph{Produkte affiner Räume}

Sei $R$ ein Ring, und $\mathbb{A}_{R}^{n}=\Spec(R[T_{1},\ldots,T_{n}])$
der affine Raum über $R$. Für $n,m\geq0$ haben wir 
\[
  R[T_{1},\ldots,T_{n}]\otimes_{R}R[T_{n+1},\ldots,T_{n+m}]\cong R[T_{1},\ldots,T_{n+m}]
\]

und deshalb nach Proposition 11
\[
  \mathbb{A}_{R}^{n}\times_{R}\mathbb{A}_{R}^{m}\cong\mathbb{A}_{R}^{n+m}.
\]


\paragraph{Produkte von Prävarietäten}

Sei $k$ ein algebraisch abgeschlossener Körper, und $X$ ein $k$-Schema
endlichen Typs. Nach 3.14 ist $X_{k}(k)=X_{0}$ (abgeschlossene Punkte
von $X$).
\begin{align*}
  x:\Spec k & \longrightarrow X\longrightarrow\text{Bild}\\
  \text{\{Integ. Sch. v.e.T./}k\} & \longleftrightarrow\text{\{Präv./}k\}\\
  X & \longmapsto\{X_{0},\mathcal{O}_{X}|_{X_{0}}\}
\end{align*}

\begin{lem}[15]
  Sei $k$ ein Körper und seien $X,Y$ integre $k$-Schemata. Dann
  ist $X\times_{k}Y$ ein integres $k$-Schemata.
\end{lem}

Beweis: später. Falls $X,Y$ integral von endlichem Typ über $k$
sind, dann ist auch $X\times_{k}Y$ integral von endlichem Typ über
$k$. Denn: $X=\bigcup_{\text{endl.}}X_{i}$, $Y=\bigcup_{\text{endl.}}Y_{j}$
$\Longrightarrow X\times_{k}Y=\bigcup_{i,j}X_{i}\times_{k}Y_{i}$.
$\Longrightarrow X=\Spec A$, $Y=\Spec B$ mit $A,B$ endlich erzeugte
$k$-Algebras. $\Longrightarrow X\times_{k}Y=\Spec A\otimes_{k}B$
endlich erzeugte $k$-Algebra.\medskip{}

Seien $X_{0}$ und $Y_{0}$ die Prävarietäten zu $X$ bzw. $Y$, und
$Z_{0}$ die Prävarietät zu $X\times_{k}Y$. Dann gilt nach der universellen
Eigenschaft des Faserprodukts:
\[
  Z_{0}=(X\times_{k}Y)_{k}(k)=X_{k}(k)\times Y_{k}(k)=X_{0}\times Y_{0},
\]

d.h. das Faserprodukt von 2 Prävarietäten $X_{0},Y_{0}$ ist wieder
eine Prävarietät $Z_{0}$ (als volle Unterkategorie von $\schk$)
mit $Z_{0}=X_{0}\times Y_{0}$ (als Mengen). Die Projektionen $Z_{0}\rightarrow X_{0}$
und $Z_{0}\rightarrow Y_{0}$ sind stetig, aber im Allgemeinen ist
die Topologie auf $Z_{0}$ \textbf{feiner }als die Produkttopologie
von $X_{0}$ und $Y_{0}$.

\section{Basiswechsel}

Sei $\mathcal{C}$ eine beliebige Kategorie mit Faserprodukten (z.B.
$\sch$), $u:S'\rightarrow S$ ein Morphismus in $\mathcal{C}$, $X\rightarrow S$
ein $S$-Objekt. $\Longrightarrow q:X\times_{S}S'\rightarrow S'$
ist ein $S'$-Objekt. Bezeichne $u^{\ast}(X)=:q$ oder $X_{(s')}$
\textbf{Urbild} oder \textbf{Basiswechsel} von $X$ bzgl. $u$.\medskip{}

Sei $f:X\rightarrow Y$ Morphismus von $S$-Objekten. $\Longrightarrow f\times_{S}\id_{S'}:X\times_{S}S'\rightarrow Y\times_{S}S'$
ist ein Morphismus von $S'$-Objekten. Bezeichne $f\times_{S}\id_{S'}=:u^{*}(f)=:f_{(s')}$
der \textbf{Basiswechsel von $f$ bzgl. $u$}. Wir erhalten einen
kontravarianten Funktor
\[
  u^{\ast}:\mathcal{C}/S\longrightarrow\mathcal{C}/S'
\]

der Kategorie von $S$-Objekten in $\mathcal{C}$ zu der Kategorie
der $S'$-Objekten in $\mathcal{C}$. Nenne $u^{\ast}$ den \textbf{Basiswechsel
  bzgl. $u$}.

\paragraph{Transitivität des Basiswechsels}

Sei $u':S''\rightarrow S'$ ein weiterer Morphismus in $\mathcal{C}$.
Nach Proposition 10 ist $(u\circ u')^{\ast}\cong u'^{\ast}\circ u^{\ast}$
ein Isomorphismus von Funktoren. Sei
\[
  \xymatrix{T\ar[r]^{h}\ar[rd] & S'\ar[d]^{u}\\
    & S
  }
  \in\mathcal{C}/S'.
\]

Wir können $T$ als $S$-Objekt auffassen durch $u\circ h$. Sei $p:X_{(S')}\rightarrow X$
die erste Projektion. Dann erhalten wir zueinander inverse Bijektionen,
funktoriell in $T$ und $X$:
\begin{align*}
  t & \longmapsto p\circ\\
  \hom_{S'}(T,X_{(S')}) & \longleftrightarrow\hom_{S}(T,X)\\
  (t,h)_{S'} & \longmapsfrom t
\end{align*}

\begin{defn}[16]
  Sei $\mathbb{P}$ eine Eigenschaft von Morphismen in $\mathcal{C}$,
  sodass $\id_{X}$ $\mathbb{P}$ erfüllt für alle $X\in\mathcal{C}$.
  \begin{enumerate}
  \item $\mathbb{P}$ heißt \textbf{stabil}
    \begin{enumerate}
    \item \textbf{unter Komposition}, wenn mit $f:X\rightarrow Y$ und $g:Y\rightarrow Z$
      auch $g\circ f$ $\mathbb{P}$ erfüllt.
    \item \textbf{unter Basiswechsel}, wenn mit $f:X\rightarrow S$ auch $f_{(S')}:X_{(S')}\rightarrow S'$
      für alle Morphismen $S'\rightarrow S$, $\mathbb{P}$ erfüllt.
    \end{enumerate}
  \item Wir sagen, dass $f:X\rightarrow S$ $\mathbb{P}$ \textbf{universell}
    erfüllt, falls $f_{(S')}$ $\mathbb{P}$ erfüllt für alle $S'\rightarrow S$.
  \end{enumerate}
\end{defn}

\begin{rem}[17]
  Sei $\mathbb{P}$ stabil unter Komposition. Dann sind äquivalent:
  \begin{enumerate}
  \item $\forall S\in\mathcal{C}$, $\forall S$-Morphismen $f:X'\rightarrow X$,
    $g:Y'\rightarrow Y$, die $\mathbb{P}$ erfüllen, erfüllt auch $f\times_{S}g$
    $\mathbb{P}$.
  \item $\mathbb{P}$ ist stabil unter Basiswechsel.
  \end{enumerate}
\end{rem}

\begin{proof}
  \mbox{}
  \begin{itemize}
  \item $(i)\Rightarrow(ii)$
  
  $f_{(S')}=f\times_{S}\id_{S'}$.

  \item $(ii)\Rightarrow(i)$
  
  Seien $f,g$ Morphismen (wie in 1) die $\mathbb{P}$ erfüllen. Da
    $f\times_{S}g=(f\times_{S}\id_{Y})\circ(\id_{X}\times_{S}g)$ sei
    ohne Einschränkung $g=\id_{Y}$.
    \begin{align*}
      f_{(X\times_{S}Y)}=f\times_{S}\id_{Y}: & X'\times_{S}Y=X'\underbrace{\times_{X}}_{\text{bzgl. }f}(X\times_{S}Y)\rightarrow X\times_{S}Y
    \end{align*}
    erfüllt $\mathbb{P}$.
  \end{itemize}
  In $\sch$ sind fast alle betrachteten Eigenschaften von Morphismen
  stabil unter Komposition, aber nicht unbedingt unter Basiswechsel,
  z.B. injektiv oder abgeschlossen.
\end{proof}
\begin{example*}
  Es ist:
  \begin{align*}
    f:X=\Spec\mathbb{Q}(\xi_{p}) & \longrightarrow\Spec\mathbb{Q}=S\\
    u:S' & \longrightarrow\Spec\mathbb{Q}
  \end{align*}

  Homöomorphismus, d.h. injektiv, aber
  \begin{align*}
    f_{(S')}:X\times_{S}S' & \longrightarrow\underbrace{S'=\Spec\mathbb{Q}(\xi_{p})}_{1\text{ Punkt}}
  \end{align*}

  ist nicht injektiv:
  \[
    \Spec(\mathbb{Q}(\xi_{p})\otimes\mathbb{Q}_{p})\cong\underbrace{\prod^{p-1}\mathbb{Q}(\xi_{p})}_{p-1\text{ Punkte}}.
  \]
\end{example*}
\textbf{Warnung.} Absolute Eigenschaften von Schemata sind oft nicht
kompatibel mit Basiswechsel. Sei $k=\mathbb{F}_{p}(t)$ (nicht perfekt!), $K=\bigcup_{n\geq1}\mathbb{F}_{p}(t^{\frac{1}{p^{n}}})$
perfekter Abschluss von $k$, $A:=K\otimes_{k}K$. Man kann zeigen:
$\nil(A)$ ist \emph{nicht} endlich erzeugt, d.h. $\Spec(A)$, $A$
ist nicht reduzibel und nicht noethersch.
\section{Fasern von Morphismen}

\textbf{Ziel:} $\xymatrix{X\ar[d]^{f}\supset f^{-1}(s)\\
  S\ni s
}
$ als Schema.
\begin{defn}[18]
  Für den kanonischen Morphismus $\Spec(\kappa(s))$ nennen wir
  \[
    X_{s}:=X\otimes_{S}\kappa(s)
  \]

  die \textbf{Faser von $f$ in $s$}, ein $\kappa(s)$-Schema. Proposition
  14 $\Longrightarrow$
  \[
    \xymatrix{X_{s}\ar[r]\ar[d] & S\times_{S}X\ar[r]^{q=\id_{X}}\ \ \ar[d]_{f} & X\ar[d]\\
      \Spec\kappa(s)\ar[r]_{\text{canon.}} & S\ar[r]_{\id_{S}} & S
    }
    \qquad\text{kommutativ,}
  \]

  besagt $X_{s}\cong f^{-1}(s)$ (=homoömorph zu Bild(canon)), d.h.
  wir können $f^{-1}(s)$ als Schema auffassen.

  \textbf{Denkweise}: $\xymatrix{X\ar[d]_{f}\\
    S
  }
  \cong$ Familie von $\kappa(s)$-Schemata $X_{s}$, parametrisiert durch
  Punkte von $S$.
\end{defn}

\begin{example}[19]
  Sei $k$ algebraisch abgeschlossen.
  \[
    X(k):=\{(u,t,s)\in\mathbb{A}^{3}(k)\mid ut=s\}
  \]

  Da $UT-s\in k[U,T,S]$ irreduzibel ist, ist $X(k)$ eine affine Varietät.
  $\leftrightarrow$ $X=\Spec(k[U,T,S]/(UT-S)$ ist ganzes $k$-Schema.
  Sei $S=\mathbb{A}^{1}$,
  \begin{align*}
    X & \longrightarrow S\\
    (u,t,s) & \longmapsto s
  \end{align*}

  Projektion. $s\in\mathbb{A}^{1}(k)$, $X_{s}=\Spec A_{s}$, 
  \begin{align*}
    A_{s} & =k[U,T,S]/(UT-S)\otimes_{k[S]}k[S]/(S-s)\\
          & \cong k[U,T]/(UT-S).
  \end{align*}

  $UT-s\in k[U,T]$ irreduzibel für $s\neq0$, reduzibel für $s=0$.
  $\Longrightarrow X\rightarrow S$ definiert Familie $X_{s}$ von $k$-Schemata,
  sodass $X_{0}$ reduzibel, $X_{s}$ irreduzibel für $s\neq0$.
\end{example}

\begin{lem}[20]
  Sei das Diagramm
  \[
    \xymatrix{ & X\times_{S}Y\ar[ld]_{p}\ar[rd]^{q}\\
      x\in X\ar[rd]_{f} &  & Y\ni y\ar[ld]^{g}\\
      \Spec\kappa(x)\ar[u]^{\xi} & S & \Spec\kappa(y)\ar[u]_{\psi}
    }
  \]

  Dann gilt:
  \begin{enumerate}
  \item Es gibt ein $z\in X\times_{S}Y$ mit $p(z)=x$, $q(z)=y$, genau dann
    wenn $f(x)=g(y).$
  \item Es gelte $(1)$, setze $s:=f(x)=g(y)$. Dann ist
    \begin{align*}
      \zeta:=\xi\times_{S}\psi:Z:=\Spec(\kappa(x)\otimes_{\kappa(s)}\kappa(y)) & \longrightarrow X\times_{S}Y
    \end{align*}
    ein Homöomorphismus von $Z$ auf Teilraum $\zeta(Z)=p^{-1}(x)\cap q^{-1}(y)$.
  \end{enumerate}
\end{lem}

\begin{proof}
  Setze $Z:=p^{-1}(x)\times_{(X\times_{S}Y)}q^{-1}(y)$, und betrachte:
  \[
    \xymatrix{ &  & Z\ar[rd]^{g'}\ar[ld]_{f'}\\
      & p^{-1}(x)\ar[rd]\ar[ld] &  & q^{-1}(y)\ar[ld]\ar[rd]\\
      \Spec\kappa(s)\ar[rd] &  & X\times_{S}Y\ar[ld]\ar[rd] &  & \Spec\kappa(y)\ar[ld]\\
      & X\ar[rd] &  & Y\ar[ld]\\
      &  & S
    }
  \]

  Wende Proposition 14 zweifach an $\Longrightarrow$
  \[
    g'(Z)=i^{-1}(p^{-1}(x))=q^{-1}(y)\cap p^{-1}(x).
  \]
\end{proof}

\section{Eigenschaften von Schemata-Morphismen}
\begin{notation}[21]
  Sei $\mathbb{P}$ Eigenschaft von Morphismen von Schemata.
  \begin{enumerate}
  \item $\mathbb{P}$ heißt \textbf{lokal im Ziel}, falls für alle Morphismen
    $f$ und alle offenen Überdeckungen $S=\bigcup_{j\in J}S_{j}$ gilt:
    \[
      f:X\rightarrow S\text{ erfüllt }\mathbb{P}\Longleftrightarrow f|_{f^{-1}(S_{j})}:f^{-1}(S_{j}):S_{j}\text{ erfüllt }\mathbb{P}\ \forall j\in J
    \]
  \item $\mathbb{P}$ heißt \textbf{lokal in der Quelle}, falls für alle $f:X\rightarrow Y$
    und alle offenen Überdeckungen $X=\bigcup_{i\in I}U_{i}$ gilt:
    \[
      f\text{ erfüllt }\mathbb{P}\Longleftrightarrow f|_{U_{i}}:U_{i}\rightarrow Y\text{ erfüllt }\mathbb{P}\ \forall i\in I
    \]
  \end{enumerate}
\end{notation}

\begin{defn*}
  Ein Morphismus $f:X\rightarrow Y$ von Schemata heißt \textbf{(treu)flach},
  falls $\forall x\in X$ die Abbildung
  \[
    \mathcal{O}_{Y,f(x)}\longrightarrow\mathcal{O}_{X,x}
  \]

  (treu)flach ist.
\end{defn*}
\begin{prop}[22]
  Die folgende Eigenschaft von Schemata-Morphismen sind:
  \begin{enumerate}
  \item stabil unter Komposition: ,,injektiv``, ,,surjektiv``, ,,bijektiv``,
    ,,homöomorph``, ,,flach``, ,,treuflach``, ,,offen``, ,,abgeschlossen``,
    ,,offene Immersion``, ,,abgeschlossene Immersion``, ,,Immersion``;
  \item stabil unter Basiswechsel: ,,surjektiv``, ,,offene Immersion``,
    ,,abgeschlossene Immersion``, ,,Immersion``, ,,flach``, ,,treuflach``;
  \item lokal bzgl. Ziel: ,,surjektiv``, ,,bijektiv``, ,,homöomorph``,
    ,,offen``, ,,abgeschlossen``, ,,offene Immersion``, ,,abgeschlossene
    Immersion``, ,,Immersion``, ,,flach``, ,,treuflach``;
  \item lokal bzgl. Quelle: ,,offen``, ,,flach``.
  \end{enumerate}
\end{prop}

\begin{proof}
  Die Fälle $(1)$, $(3)$, $(4)$ sind klar. Der Fall $(2)$ für offene/abgeschlossene
  Immersionen wird in Abschnitt 9 behandelt. Lemma 20 $\Longrightarrow$
  ,,surjektiv`` stabil unter Basiswechsel.
  \[
    \xymatrix{X\times_{S}Y\ar[r]^{q} & Y & y\ar@{|->}[d]\\
      X\ar@{->>}[r]^{f} & S & s\\
      \exists x\ar@/_{1pc}/@{|->}[rru]
    }
  \]

  Sei $X\rightarrow S$ (treu)flach und $S'\rightarrow S$ beliebig.
  Sei in Fall $(3)$, $(4)$ ohne Einschränkung $X=\Spec A$, $S=\Spec R$,
  $S'=\Spec R'$, $A$ (treu)flache $R$-Algebra. $\Longrightarrow A\otimes_{R}R'$
  (treu)flache $R'$-Algebra, d.h. $f_{(S')}$ ist treuflach.
  \[
    (A\otimes_{R}R')\otimes_{R'}M\cong A\otimes_{R}M
  \]
\end{proof}
\begin{cor}[23]
  Die folgende Eigenschaften sind stabil unter Komposition, stabil
  unter Basiswechsel und lokal bzgl. Ziel:

  ,,universal injektiv``, ,,universal bijektiv``, ,,universal homöomorph``,
  ,,universell offen``, ,,universell abgeschlossen``.
\end{cor}

\section{Urbilder und Schema-theoretische Schnitte von Unterschemata}

Sei $f:X\rightarrow Y$ ein Morphismus von Schemata und $i:Z\rightarrow Y$
eine Immersion.
\[
  \xymatrix{Z\times_{Y}X\ar[r]\ar[d]_{i_{(X)}} & Z\ar[d]^{i}\\
    X\ar[r]_{f} & Y
  }
\]

Proposition 14 $\Longrightarrow i_{(X)}$ ist surjektiv auf Halmen,
Homöomorphismus von $Z\times_{Y}X$ auf lokal abgeschlossene Teilmenge
$f^{-1}(Z)$ (genau $f^{-1}(i(Z))$), d.h. $i_{(X)}$ ist Immersion.
Fasse $Z\times_{Y}X$ als Unterschema von $X$ auf, das \textbf{Urbild
  von $Z$ unter $f$}.
\begin{rem*}
  \mbox{}
  \begin{enumerate}
  \item Ist $Z\subset Y$ offenes Unterschema, so auch $f^{-1}(Z)\subset X$.
  \item Ist $Z=V(\mathfrak{p})$ abgeschlossenes Unterschema, $\mathfrak{p}\subset\mathcal{O}_{Y}$
    Idealgarbe, so auch 
    \begin{align*}
      f^{-1}(Z) & =V(f^{\ast^{-1}}(\mathfrak{p})\mathcal{O}_{X})\\
                & =\text{Bild}(f^{\ast^{-1}}(\mathfrak{p})\rightarrow f^{\ast^{-1}}\mathcal{O}_{Y}\rightarrow\mathcal{O}_{X}).
    \end{align*}
  \end{enumerate}
  \textbf{Spezialfall}: Durchschnitt von 2 Unterschemata $i:Y\rightarrow X$,
  $j:Z\rightarrow X$:
  \[
    Y\cap Z:=Y\times_{X}Z=i^{-1}(Z)=j^{-1}(Y)
  \]

  heißt \textbf{(Schema-theoretischer) Durchschnitt von $Y$ und $Z$
    in $X$}.
\end{rem*}

\subsubsection*{Universelle Eigenschaft (aus univ. Eig. Faserprodukt)}

Ein Morphismus $h:T\rightarrow X$ faktorisiert durch $Y\cap Z$ genau
dann wenn $h$ faktorisiert durch $Y$ und $Z$. Sind $Y=V(\mathfrak{p})$,
$Z=V(\mathfrak{q})$ abgeschlossene Unterschemata, so folgt:
\begin{align*}
  V(\mathfrak{p})\cap V(\mathfrak{q}) & =V(\mathfrak{p}+\mathfrak{q})\\
  A/\mathfrak{p}\otimes_{A}A/\mathfrak{q} & \cong A/\mathfrak{q+}\mathfrak{p}
\end{align*}

\begin{example*}
  $f_{1},\ldots,f_{r},g_{1},\ldots,g_{s}\in R[X_{0},\ldots,X_{n}]$
  homogene Polynome. Dann ist:
  \[
    V_{+}(f_{1},\ldots,f_{r})\cap V_{+}(g_{1},\ldots,g_{s})=V_{+}(f_{1},\ldots,f_{r},g_{1},\ldots,g_{s})\subseteq\mathbb{P}_{R}^{n}
  \]
\end{example*}

\section{Affine und projektive Räume über beliebige Basen}
\begin{itemize}
\item[] $\mathbb{A}^{n}:=\mathbb{A}_{\mathbb{Z}}^{n}$, $S$ beliebiges Schema.
\item[] $\mathbb{A}_{S}^{n}:=\mathbb{A}^{n}\times_{\mathbb{Z}}S$ \textbf{affiner
    Raum der relativen Dimension $n$ über $S$}.
\item[] $\mathbb{P}_{S}^{n}:=\mathbb{P}^{n}\times_{\mathbb{Z}}S$ \textbf{projektiver
    Raum der relativen Dimension $n$ über $S$}.
\item[] $S=\Spec R$ affin:
  \begin{align*}
    \mathbb{A}_{S}^{n} & =\Spec(\mathbb{Z}[T_{1},\ldots,T_{n}]\otimes_{\mathbb{Z}}R)=\Spec(R[T_{1},\ldots,T_{n}])\\
                       & =\mathbb{A}_{R}^{n}\text{ wie zuvor!}\\
    \mathbb{P}_{S}^{n} & =\mathbb{P}_{R}^{n}\text{ analog.}
  \end{align*}
\item[] $\mathbb{A}_{n}^{S}\times_{S}S'=\mathbb{A}^{n}\times_{\mathbb{Z}}S\times_{S}S'=\mathbb{A}_{S'}^{n}$
\item[] $\mathbb{P}_{S}^{n}\times_{S}S'=\mathbb{P}_{S'}^{n}$ für einen beliebigen
  Basiswechsel $S'\rightarrow S$.
\end{itemize}
Sei $X$ ein beliebiges Schema.
\begin{align*}
  \Gamma(X,\mathcal{O}_{X}) & =\Hom_{\ring}(\mathbb{Z}[T],\Gamma(X,\mathcal{O}_{X}))\\
                            & =\Hom_{\schz}(X,\mathbb{A}_{\mathbb{Z}}^{1})\\
  \varphi(T) & \mapsfrom\varphi
\end{align*}

Sei $X$ ein $S$-Schema.
\[
  \Gamma(X,\mathcal{O}_{X})=\Hom_{\schs}(X,\mathbb{A}_{S}^{1})
\]

\section{Diagonal, Graph und Kern in beliebigen Kategorien}

Sei $\mathcal{C}$ Kategorie mit Faserprodukten, $S\in\mathcal{C}$,
$X,T\in\mathcal{C}/S$, $X_{S}(T)$ Menge der $S$-Morphismen.
\begin{defn}[24]
  Der Morphismus
  \begin{enumerate}
  \item $\Delta_{X/S}:=\Delta_{u}:=(\id_{X},\id_{X}):X\rightarrow X\times_{S}X$,
    $u:X\rightarrow S$, heißt \textbf{Diagonale }(diagonaler Morphismus)
    \textbf{von $X$ über $S$}.
  \item Sei $f:X\rightarrow Y\in$ Morph/$S$. Der Morphismus
    \[
      \Gamma_{j}:=(\id_{X},f)_{S}:X\longrightarrow X\times_{S}Y
    \]
    heißt der \textbf{Graph(morphismus) von $f$}.
  \item Seien $f,g:X\rightarrow Y\in$ Morph/$S$. Ein $S$-Monomorphismus
    $i:K\rightarrow X$ heißt \textbf{(Differenzen)kern} von $f$ und
    $g$, falls für alle $T\in\mathcal{C}/S$ die Abbildung $i(T):K_{S}(T)\rightarrow X_{S}(T)$
    injektiv ist mit
    \[
      \text{Bild}(i(T))=\{x\in X_{S}(T)\mid f(T)(x)=g(T)(x)\}.
    \]
    Beizchne $K(f,g)_{S}$ oder $\ker(f,g)$, $i$ ,,kanonisch``. Mit
    anderen Worten, $\ker(f,g)$ separiert den Funktor
    \begin{align*}
      \mathcal{C}/S & \longrightarrow\sch\\
      T & \longmapsto\{x\in X_{S}(T)\mid f(T)(x)=g(T)(x)\}
    \end{align*}
  \end{enumerate}
\end{defn}

\begin{example}[25]
  In der Kategorie $\mathcal{C}=\set$ gilt für:
  \[
    \xymatrix{X\ar[r]^{f,g}\ar[rd]_{u} & Y\ar[d]^{v}\\
      & S
    }
  \]
  \begin{itemize}
  \item[]
    $\begin{array}{rl}
       \Delta_{u}:X & \longrightarrow X\times X=\{(x,x')\in X\times X\mid u(x)=u(x')\}\\
       x & \longmapsto(x,x)
     \end{array}$
   \item[]
     $\begin{array}{rl}
        \Gamma_{j}:X & \longrightarrow X\times_{S}Y=\{(x,y)\in X\times Y\mid u(x)=v(y)\}\\
        x & \longmapsto(x,f(x))
      \end{array}$
    \item[]
      $\ker(f,g)=\{x\in X\mid f(x)=g(x)\}$
    \end{itemize}
    Da $p\circ\Gamma_{j}=\id_{X}$, sind $\Gamma_{j}$ und $\Delta_{X/S}=\Gamma_{\id_{X}}$
    Monomorphismen.
\end{example}

\begin{prop}[26]
  Seien $f,g:X\rightarrow Y$ $S$-Morphismen.
  \begin{enumerate}
  \item $\Delta_{X/S}=\Gamma_{\id_{X}}$,
    
    $\Gamma_{f}=\ker(\xymatrix{X\times_{S}Y\ar@<1ex>[r]^{q}\ar@<-1ex>[r]_{f\circ p} & Y}
    )\longrightarrow X\times_{S}Y$ kanonisch.
  \item Alle Rechtecke des folgenden kommutativen Diagramms sind kartesisch:
    \[
      \xymatrix{\ker(f,g)\ar[r]^{\text{canon.}}\ar[d] & X\ar[r]^{f}\ar[d]_{\Gamma_{f}} & Y\ar[d]^{\Delta_{Y/S}}\\
        X\ar[r]_{\Gamma_{g}} & X\times_{S}Y\ar[r]_{f\times\id_{S}} & Y\times_{S}Y
      }
    \]
  \item Sei $s:S\rightarrow X$ ein Schnitt von $f$ ($f\circ s=\id_{S}$).
    Dann ist das folgende Diagramm kartesisch:
    \[
      \xymatrix{S\ar[r]^{s}\ar[d]_{s} & X\ar[d]^{\Gamma_{s\circ f}}\\
        X\ar[r]_{\Delta_{X/S}} & \quad X\times_{S}X
      }
    \]
  \end{enumerate}
\end{prop}

\begin{proof}
  Nach dem Yoneda-Lemma reicht es, denn Fall $\mathcal{C}=\set$ zu
  verifizieren. Dies ist elementar aufgrund der Beschreibung in Beispiel
  25.
\end{proof}
Insbesondere existiert $\ker(f,g)$ stets!

\section{Diagonal für Schemata}
\begin{prop}[27]
  Für affine $S$-Schemata
  \begin{align*}
    X & =\Spec(B)\overset{u}{\longrightarrow}S=\Spec(R)\\
    Y & =\Spec(A)\overset{v}{\longrightarrow}S
  \end{align*}

  und $S$-Morphismus $f=\Spec\varphi:X\rightarrow Y$ zu einem $R$-Algebra
  Morphismus $\varphi:A\rightarrow B$, entsprechen $\Delta_{X/S}$
  und $\Gamma_{f}$ den folgenden surjektiven Ringhomomorphismen:
  \[
    \begin{array}{rl}
      \Delta_{B/R}:B\otimes_{R}B & \longrightarrow B\\
      b\otimes b' & \longmapsto bb'
    \end{array},\qquad
    \begin{array}{rl}
      \Gamma_{\varphi}:A\otimes_{R}B & \longrightarrow B\\
      a\otimes b & \longmapsto\varphi(a)b
    \end{array}.
  \]
  
  Insbesondere sind $\Delta_{X/S}$, $\Gamma_{j}$ abgeschlossene Immersionen.
\end{prop}

Im allgemeinen sind $\Delta_{X/S}$, $\Gamma_{f}$ Immersionen (nicht
notwendig abgeschlossen!): Seien $Z,Z'\subset X$ Unterschemata. $\Longrightarrow Z\times_{S}Z'\subset X\times_{S}X$
Unterschemata (Immersionen und stabil unter Basiswechsel und Komposition),
und
\[
  Z\cap Z'=\Delta_{X/S}^{-1}(Z\times_{S}Z')\qquad(*)
\]

\begin{prop}
  Seien $X,Y\in\schs$, $f,g:X\rightarrow Y$ $S$-Morphismen. Dann
  sind $\Delta_{X/S}$, $\Gamma_{f}$, $\ker(f,g)\rightarrow X$ Immersionen.
\end{prop}

\begin{proof}
  Es reicht zu zeigen: $\Delta_{X/S}$ ist eine Immersion (und $(2)$
  in Proposition 26, da ,,Immersion`` stabil ist unter Basiswechsel)
  lokal bzgl. Ziel. Sei also ohne Einschränkung $S$ affin. Falls $X=\bigcup_{i\in I}U_{i}$
  offene Überdeckung, dann ist $\Delta_{X/S}(X)=\bigcup_{i\in I}U_{i}\times_{S}U_{i}$
  offene Überdeckung. 
  \[
    \xymatrix{X\ar[r]_{\text{abg. Imm.}} & \bigcup_{i\in I}(U_{i}\times_{S}U_{i})\ar@{^{(}->}_{\text{off. Imm.}}[r] & X\times_{S}X}
  \]
  
  $(*)\Longrightarrow$ ohne Einschränkung, $X$ affin. Wende nun Proposition
  27 an.
\end{proof}
Das Unterschema
\begin{itemize}
\item $X\cong\Delta_{X/S}(X)\subset X\times_{S}X$ heißt die \textbf{Diagonale
    von $X\times_{S}X$}.
\item $\Gamma_{f}(X)\subset X\times_{S}Y$ heißt der \textbf{Graph von $f$}.
\end{itemize}

\begin{rem}[29]
  \mbox{}
  \begin{enumerate}
  \item Ein Unterschema $T\subset X\times_{S}Y$ ist der Graph eines $S$-Morphismus
    $f:X\rightarrow Y$ genau dann, wenn $p|_{T}:T\rightarrow X\times_{S}Y\underset{p}{\rightarrow}X$
    ein Isomorphismus ist, \emph{denn }$f=q\circ(p|_{T})^{-1}$.
  \item Im Allgemeinen ist die mengentheoretische Inklusion
    \[
      \Delta_{X/S}(X)\subset\{z\in X\times_{S}X\mid f(z)=g(z)\}
    \]
    \emph{keine} Gleichheit!
  \end{enumerate}
\end{rem}

\section{Separierte Morphismen}

Erinnerung: Für einen topologischen Raum $X$ sind äquivalent:
\begin{enumerate}
\item $X$ ist Hausdorff.
\item $\Delta\subset X\times X$ ist abgeschlossen bzgl. der Produkttopolgie.
\item Für jedes Paar von stetigen Abbildungen $f,g:Y\rightarrow X$ ist
  $\ker(f,g)\subset X$ abgeschlossen.
\item Für jeder stetige Morphismus $f:Y\rightarrow X$ ist $\Gamma_{f}\subset X\times Y$
  abgeschlossen.
\end{enumerate}
Für ein Schema $X$ ist der unterliegende topologische Raum selten
Hausdorff, aber $(2)-(4)$ geben sinnvolle Konzepte für Schemata (im
Allgemeinen ist die Produkttopologie ungleich der Faserprodukttopologie).
\begin{defn}[30]
  Ein Morphismus $v:Y\rightarrow S$ von Schemata heißt separiert,
  falls folgende äquivalente Bedingungen erfüllt sind.
  \begin{enumerate}
  \item $\Delta_{Y/S}$ ist eine \emph{abgeschlossene} Immersion.
  \item Für jedees Paar $f,g:X\rightarrow Y$ ist $\ker(f,g)\subset X$ ein
    abgeschlossenes Unterschema.
  \item Für jeden $S$-Morphismus $f:X\rightarrow Y$ ist $\Gamma_{f}$ eine
    abgeschlossene Immersion.
  \end{enumerate}
  Dann heißt auch $Y$ ist \textbf{separiert über $S$}. Ein Schema
  $Y$ heißt \textbf{separiert}, falls es separiert über $\mathbb{Z}$
  ist.
\end{defn}

\begin{proof}
  Die Äquivalenz folgt nach Proposition 26, und dass ,,abgeschlossene
  Immersion`` stabil unter Basiswechsel ist.
\end{proof}
Nach Proposition 27 ist jeder Morphismus zwischen affinen Schemata
separiert. Insbesondere ist jedes affine Schemata separiert.
\begin{prop}[31]
  Seien $X,Y\in\schs$, $Y$ separiert über $S$, $U\subset X$ offenes
  dichtes Unterschema, $f,g:X\rightarrow Y$ $S$-Morphismus mit $f|_{U}=g|_{U}$.
  Dann ist $f|_{X_{\red}}=g|_{X_{\red}}$.
\end{prop}

\begin{proof}
  Nach Voraussetzung ist $U\subseteq\ker(f,g)$. Da $Y$ separiert ist
  über $S$, ist $\ker(f,g)\subset X$ abgeschlossenes Unterschema.
  Da $U$ dicht ist in $X$, ist der unterliegende topologische Raum
  von $\ker(f,g)$ gleich $X$. $\Longrightarrow X_{\red}\subseteq\ker(f,g)$
  als Schema.
\end{proof}
\begin{example}[32]
  Sei
  \[
    \,\xymatrix{\ar@{-}[r] & :\ar@{-}[r] & \,}
  \]

  affine Gerade mit Doppelpunkt (siehe Beispiel 11, III.5) ist \emph{nicht}
  separiert: $V\subset U$ offen, nicht abgeschlossen.
  \[
    j,j':U\longrightarrow U\cup_{V}U\quad\Rightarrow\quad\ker(j,j')=V\subset U\subset X\text{\,nicht abg.!}
  \]
\end{example}

\begin{rem}[33]
  $\mathbb{P}$ sei eine Eigenschaft von Morphismen, sodass gilt:
  \begin{itemize}
  \item stabil unter Komposition und Basiswechsel;
  \item jede (abgeschlossene) Immersion erfült $\mathbb{P}$.
  \end{itemize}
  Für jedes kommutative Diagramm:
  \[
    \xymatrix{X\ar[r]^{f}\ar[rd]_{u} & Y\ar[d]^{v}\\
      & S
    }
  \]

  mit $u$ erfüllt $\mathbb{P}$ (und $v$ seperariert) $\Longrightarrow f$
  erfüllt $\mathbb{P}$, da:
  \[
    f:X\xrightarrow[\text{(abg.) Imm. erfüllt }\mathbb{P}]{\Gamma_{f}}X\times_{S}Y\xrightarrow[\text{Basisw. erfüllt }\mathbb{P}]{q}Y
  \]

  erfüllt $\mathbb{P}$, wegen stabil unter Komposition.
\end{rem}

\begin{prop}
  \mbox{}
  \begin{enumerate}
  \item Jeder Monomorphismsu von Schemata (insbesondere jede Immersion) ist
    separiert.
  \item Die Eigenschaft ,,separiert`` ist stabil unter Komposition, stabil
    unter Basiswechsel, und lokal bzgl. Ziel.
  \item Ist die Komposition $X\rightarrow Y\rightarrow Z$ zweier Morphismen
    separiert, so auch $X\rightarrow Y$.
  \item $f:X\rightarrow Y$ ist seperariert genau dann, wenn $f_{\red}:X_{\red}\rightarrow Y_{\red}$
    separiert ist.
  \end{enumerate}
\end{prop}

\begin{proof}
  \mbox{}
  \begin{enumerate}
  \item Wenn $f$ Monomorphismus ist (d.h. injektiv auf $T$-wertigen Punkten
    für alle Schemata $T$), dann ist $\Delta_{f}$ Isomorphismus (d.h.
    bijektiv auf allen $T$-wertigen Punkte für alle $T$). Insbesondere
    ist $\Delta_{f}$ eine abgeschlossene Immersion.
  \item Seien $f:X\rightarrow Y$, $g:Y\rightarrow Z$ separierte Schemata-Morphismen,
    $p,q:X\times_{Y}X\rightarrow X$ die zwei Projektionen. Das folgende
    Diagramm ist kommutativ, und das rechte Viereck ist kartesisch (überprüfe
    in $\set$):
    \[
      \xymatrix{X\ar[r]^{\Delta_{f}}\ar[rd]_{\Delta_{g\circ f}} & X\times_{Y}X\ar[r]^{f\circ p=f\circ q}\ar[d]|-{(p,q)_{Z}} & Y\ar[d]^{\Delta_{g}}\\
        & X\times_{Z}X\ar[r]_{f\times f} & Y\times_{Z}Y.
      }
    \]
    Da $\Delta_{g}$ abgeschlossene Immersion, ist $(p,q)_{Z}$ abgeschlossene
    Immersion $\Longrightarrow$ die Komposition $\Delta_{g\circ f}$
    ist abgeschlossene Immersion $\Longrightarrow g\circ f$ ist separiert.

    $\Delta_{f}$ abgeschlossene Immersion $\Longrightarrow\Delta_{f_{(S')}}$
    ist abgeschlossene Immersion. Dies zeigt das ,,separiert`` abgeschlossen
    ist unter Basiswechsel. Weiter ist ,,separiert`` lokal bzgl. Ziel,
    da dies gilt für ,,abgeschlossene Immersion``.
  \item Folgt aus (1), (2) nach Bemerkung 33. ($u=``\circ"$, $v=Y\rightarrow Z$,
    $f:X\rightarrow Y$)
  \item Sei $f:X\rightarrow S$ Morphismus, $i:X_{\red}\rightarrow X$ kanonische
    Immersion. Dann ist $i$ surjektive Immersion, also ein universeller
    Homöomorphismus. Identifizieren von $X_{\red}\times_{S_{\red}}X_{\red}$
    mit $X_{\red}\times_{S}X_{\red}$ liefert $\Delta_{f}\circ i=(i\times_{S}i)\circ\Delta_{f_{\red}}$.
    $\Longrightarrow\Delta_{f}$ ist abgeschlossene Immersion genau dann
    wenn $\Delta_{f_{\red}}$ abgeschlossene Immersion.
  \end{enumerate}
\end{proof}

\begin{example}[35]
  Sei $S$ beliebiges Schema, $n\in\mathbb{N}$. Dann ist $\mathbb{A}_{S}^{n}$
  separiert über $S$, ebenso jedes Unterschema, denn $\mathbb{A}_{S}^{n}=\mathbb{A}_{\mathbb{Z}}^{n}\times_{\Spec\mathbb{Z}}S$
  und ,,separiert`` ist stabil unter Basiswechsel nach Proposition
  34.
\end{example}

\begin{prop}[36]
  Sei $S=\Spec R$ affin und $X$ ein $S$-Schema. Dann sind äquivalent:
  \begin{enumerate}
  \item $X$ ist separiert.
  \item Für je zwei offene affine $U,V\subseteq X$ ist $U\cap V$ affin,
    und
    \begin{align*}
      \rho_{U,V}:\mathcal{O}_{X}(U)\otimes_{R}\mathcal{O}_{X}(V) & \longrightarrow\mathcal{O}_{X}(U\cap V),\\
      s\otimes t & \longmapsto s|_{U\cap V}\cdot t|_{V\cap U}.
    \end{align*}
  \item Es gibt eine offene affine Überdeckung $X=\bigcup_{i\in I}U_{i}$,
    sodass $\forall i,j\in I$: $\rho_{U,V}$ ist surjektiv.
  \end{enumerate}
\end{prop}

\begin{proof}
  Für alle offene affine $U,V\subseteq X$ gilt:
  \[
    U\cap V=\Delta_{X/S}^{-1}(U\times_{S}V).
  \]

  ,,abgeschlossene Immersion`` ist lokal auf dem Ziel, daher: $\Delta_{X/S}$
  ist abgeschlossene Immersion

  $\Longleftrightarrow$ für alle $U,V\subseteq X$ offen affin ist
  \[
    U\cap V\xrightarrow{\Delta_{X/S}|_{U\cap V}}U\times_{S}V
  \]

  abgeschlossene Immersion.

  $\Longleftrightarrow$ für jede offene affine Überdeckung $X=\bigcup_{i\in I}U_{i}$
  und alle $i,j\in I$ ist
  \[
    U_{i}\cap U_{j}\longrightarrow U_{i}\times_{S}U_{j}
  \]

  abgeschlossene Immersion. Sind $U=\Spec A$, $V=\Spec B$ affin, so
  ist auch
  \[
    U\times_{S}V=\Spec(A\otimes_{R}B)
  \]

  affin. Daher:
  \[
    U\cap V\longrightarrow U\times_{S}V
  \]

  abgeschlossene Immersion. $\Longleftrightarrow\rho_{U,V}$ surjektiv.
\end{proof}
\begin{example}[37]
  Für jedes Schema $S$ und $n\in\mathbb{N}$ ist $\mathbb{P}_{S}^{n}$
  separiert über $S$, denn ,,separiert`` ist lokal auf dem Ziel (Proposition
  34), daher ohne Einschränkung $S=\Spec R$ affin. Sei $\mathbb{P}_{R}^{n}=\bigcup_{i=0}^{n}U_{i}$
  mit $U_{i}=\Spec R\left[\frac{X_{0}}{X_{i}},\ldots,\frac{\widehat{X_{i}}}{X_{i}},\ldots,\frac{X_{n}}{X_{i}}\right]$
  und
  \begin{align*}
    \rho_{U_{i},U_{j}}:R & \left[\frac{X_{0}}{X_{i}},\ldots,\frac{\widehat{X_{i}}}{X_{i}},\ldots,\frac{X_{n}}{X_{i}}\right]\otimes_{R}R\left[\frac{X_{0}}{X_{j}},\ldots,\frac{\widehat{X_{j}}}{X_{j}},\ldots,\frac{X_{n}}{X_{j}}\right]\\
                         & \longrightarrow R\left[\frac{X_{0}}{X_{i}},\ldots,\frac{\widehat{X_{i}}}{X_{i}},\ldots,\frac{X_{n}}{X_{i}}\right]\left[\frac{X_{i}}{X_{j}}\right]
  \end{align*}

  ist surjektiv.
\end{example}

\begin{example}[38]
  Sei $k$ algebraisch abgeschlossener Körper, $X$ Prävarietät über
  $k$, d.h. ganzes Schema von endlichem Typ über $k$. $X$ heißt \textbf{Varietät},
  wenn $X$ separabel ist. Affine Prävarietäten sind also automatisch
  Varietäten. $\mathbb{P}^{n}(k)$ ist Varietät (Beispiel 37) $\Longrightarrow$
  Jede quasi-projektive Prävarietät ist eine Varietät!
\end{example}

\section{Eigentliche Morphismen}

(eng. ,,proper``)\medskip{}

Ist $f:X\rightarrow Y$ stetige Abbildung zwischen topologischen Räume,
dann heißt $f$ \textbf{eigentlich}, wenn die Urbilder kompakter Mengen
kompat sind.

Sei $X$ Hausdorff, $Y$ lokal kompakt. Dann ist $f$ eigentlich $\Longleftrightarrow f$
universell abgeschlossen. (Bourbaki, Topologie générale, $I$.10 nr.
3, Prop. 7)

$f:X\rightarrow Y$ heißt von \textbf{endlichem Typ}, wenn $\forall U\subset Y$
offen eine Überdeckung $f^{-1}(U)=\bigcup_{i=1}^{n}V_{i}$ existiert,
sodass $\forall i:\mathcal{O}_{X}(V_{i})$ ist endlich erzeugte $\mathcal{O}_{Y}(U)$-Algebra.%

\begin{defn}[39]
  Ein Morphismus $f:X\rightarrow Y$ von Schemata heißt \textbf{eigentlich},
  wenn:
  \begin{enumerate}
  \item $f$ von endlichem Typ.
  \item $f$ separiert.
  \item $f$ universell abgeschlossen.
  \end{enumerate}
  Ein $Y$-Schema heißt \textbf{eigentlich}, wenn der Strukturmorphismus
  eigentlich ist. ,,eigentlich`` ist lokal auf dem Ziel.\medskip{}
\end{defn}

\begin{defn}[40]
  Ein Morphismus $f:X\rightarrow Y$ heißt \textbf{projektiv}, wenn
  er sich faktorisieren lässt als
  \[
    \xymatrix{X\ar[rr]^{f}\ar[rd]_{\text{abg. Imm.}}^{g} &  & Y\\
      & \mathbb{P}_{Y}^{n}\ar[ur]_{\text{kan. Morph.}}
    }
  \]

  für ein $n\in\mathbb{N}$.
\end{defn}

\begin{prop}[41]
  Sei $\mathbb{P}$ eine der folgenden Eigenschaften:
  \begin{itemize}
  \item[I.] von endlichem Typ;
  \item[II.] eigentlich;
  \item[III.] projektiv.
  \item[IV.] (separiert)
  \end{itemize}
  Dann gilt:
  \begin{enumerate}
  \item Abgeschlossene Immersionen erfüllen $\mathbb{P}$.
  \item $\mathbb{P}$ ist stabil unter Komposition.
  \item $\mathbb{P}$ ist stabil unter Basiswechsel.
  \item Falls $f:X\rightarrow Z$, $g:Y\rightarrow Z$ $\mathbb{P}$ erfüllen,
    dann auch $X\times_{Z}Y\rightarrow Z$.
  \item Erfüllt $X\xrightarrow{f}Y\xrightarrow{g}Z$ $\mathbb{P}$, so auch
    $f$, falls:
    \begin{itemize}
    \item $f$ quasi-kompakt (d.h. $Y$ hat offene affine Überdeckung, deren
      Urbilder quasi-kompakt sind).
    \item $g$ separiert, im Falle II, III.
    \item (stetig im Fall IV)
    \end{itemize}
  \end{enumerate}
\end{prop}

Vergleiche: Lin, ,,Algebraic Geometry and Arithmetic curves``, Prop
24, 3.16, 3.32, (3.9).
\begin{proof}[Beweis (Skizze)]
\end{proof}
\begin{prop}[42]
  Sei $f:X\rightarrow Y$ surjektiver Morphismus von $S$-Schemata,
  und $Y$ separiert von endlichem Typ über $S$, sowie $X$ eigentlich
  über $S$. Dann ist $Y$ eigentlich über $S$.
\end{prop}

\begin{proof}
  $f$ surjektiv $\Longrightarrow\forall T\rightarrow S$ ist $f_{(T)}:X_{T}\rightarrow Y_{T}$
  surjektiv.

  $\Longrightarrow\xymatrix{A\underset{\text{abg.}}{\subset}Y_{T}\ar[r] & T\\
    & \varphi^{-1}(A)\underset{\text{abg.}}{\subset}X_{T}\ar@{->>}[ul]_{\varphi}\ar[u]_{X\text{ eig.}\Rightarrow\text{abg.}}
  }
  $

  $\Longrightarrow Y_{T}\rightarrow T$ abgeschlossen.

  $\Longrightarrow Y\rightarrow S$ universal abgeschlossen.
\end{proof}
\begin{prop}[43]
  Sei $X$ eigentliches Schema über $S=\Spec A$. Dann ist $\mathcal{O}_{X}(X)$
  ganz über $A$. Ist $X=\Spec B$ affin, so ist $B$ endlich über $A$.
\end{prop}

\begin{proof}
  Lin, 3.17/3.18.
\end{proof}
\begin{cor}
  Sei $X$ reduzierte eigentliche Varietät über $k$. Dann ist $\mathcal{O}_{X}(X)$
  endlich-dimensionaler $k$-Vektorraum.
\end{cor}

\begin{thm}[45]
  Sei $\mathcal{O}_{K}$ Bewertungsring, $K=\Quot(\mathcal{O}_{X})$,
  $X/\mathcal{O}_{K}$ eigentlich. Dann ist $X_{\mathcal{O}_{K}}(\mathcal{O}_{K})\rightarrow X_{K}(K)$
  bijektiv.
\end{thm}

\textbf{Bewertungskriterium }(vgl. Hartshorne, Lin 3.26)
\[
  \xymatrix{\Spec K\ar[r]\ar[d] & X\ar[d]^{f}\\
    \Spec\mathcal{O}_{K}\ar@{-->}[ur]|-{\exists!}\ar[r] & Y
  }
\]

$f$ eigentlich $\Longleftrightarrow$ universelle Eigenschaft oben
erfüllt. (Theorem auf $X\times_{Y}\Spec\mathcal{O}_{K}\rightarrow\Spec\mathcal{O}_{K}$).
\begin{thm}[46, Lin III 3.30]
  Jeder projektive Morphismus ist eigentlich.
\end{thm}

Zum Beispiel: abelsche Varietäten, etwa elliptische Kurven. Theorem
46 $\Longrightarrow E(\mathbb{Q}_{p})=E(\mathbb{Z}_{p})$.


\chapter{Dimensionen}
\label{chap:dimensionen}

\section{Allgemeine Schemata}
\begin{defn}[1]
  Für einen topologischen Raum $X$ ist die (Krull-)Dimension das Supremum
  der Länge aller Ketten
  \[
    Z_{0}\subsetneq Z_{1}\subsetneq\cdots\subsetneq Z_{n}\subseteq X
  \]

  irreduzibler abgeschlossener Teilmengen $Z_{i}$. $X$ sei \textbf{von
    Dimension $n$}, falls alle irreduzible Komponenten von $X$ die Dimension
  $n$ haben ($\dim\emptyset=-\infty$, sonst $\dim X\in\mathbb{N}\cup\{+\infty\}$).

  Die Dimension eines Schemas ist per Definition die Dimension des unterliegenden
  topologischen Raums, also $\dim X=\dim X_{\red}$.
\end{defn}

\begin{example}[2]
  \mbox{}
  \begin{enumerate}
  \item 
  \end{enumerate}
\end{example}


\newpage{}
\printindex{}
\end{document}
