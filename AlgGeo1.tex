%% LyX 2.2.2 created this file.  For more info, see http://www.lyx.org/.
%% Do not edit unless you really know what you are doing.
\documentclass[12pt,english,ngerman]{article}
\usepackage[T1]{fontenc}
\usepackage[latin9]{inputenc}
\usepackage{geometry}
\geometry{verbose,tmargin=2cm,bmargin=2cm,lmargin=2cm,rmargin=2cm}
\pagestyle{headings}
\usepackage{color}
\usepackage{babel}
\usepackage{verbatim}
\usepackage{enumitem}
\usepackage{amsmath}
\usepackage{amsthm}
\usepackage{amssymb}
\usepackage{cancel}
\usepackage{stmaryrd}
\usepackage{stackrel}
\usepackage{makeidx}
\makeindex
\usepackage{setspace}
\usepackage[all]{xy}
\usepackage{tikz-cd}
\onehalfspacing
\usepackage[unicode=true,
 bookmarks=true,bookmarksnumbered=false,bookmarksopen=false,
 breaklinks=false,pdfborder={0 0 1},backref=false,colorlinks=false]
 {hyperref}
\hypersetup{pdftitle={Algebraische Geometrie I},
 pdfauthor={Prof. Dr. Venjakob}}
\usepackage{breakurl}

\makeatletter

%%%%%%%%%%%%%%%%%%%%%%%%%%%%%% LyX specific LaTeX commands.
%% Because html converters don't know tabularnewline
\providecommand{\tabularnewline}{\\}

%%%%%%%%%%%%%%%%%%%%%%%%%%%%%% Textclass specific LaTeX commands.
\newlength{\lyxlabelwidth}      % auxiliary length 
  \theoremstyle{plain}
  \newtheorem*{question*}{\protect\questionname}
\theoremstyle{plain}
\newtheorem{thm}{\protect\theoremname}
  \theoremstyle{definition}
  \newtheorem{example}[thm]{\protect\examplename}
  \theoremstyle{definition}
  \newtheorem{defn}[thm]{\protect\definitionname}
  \theoremstyle{plain}
  \newtheorem{prop}[thm]{\protect\propositionname}
  \theoremstyle{plain}
  \newtheorem{cor}[thm]{\protect\corollaryname}
  \theoremstyle{plain}
  \newtheorem{lem}[thm]{\protect\lemmaname}
  \theoremstyle{remark}
  \newtheorem{rem}[thm]{\protect\remarkname}
  \theoremstyle{remark}
  \newtheorem*{claim*}{\protect\claimname}
  \theoremstyle{definition}
  \newtheorem*{example*}{\protect\examplename}

%%%%%%%%%%%%%%%%%%%%%%%%%%%%%% User specified LaTeX commands.
\usepackage{mathtools}
\usepackage{faktor}
\DeclareMathOperator{\rad}{rad}
\renewcommand{\labelenumi}{(\roman{enumi})}
\renewcommand{\labelenumii}{\arabic{enumii}.}

\makeatother

  \addto\captionsenglish{\renewcommand{\claimname}{Claim}}
  \addto\captionsenglish{\renewcommand{\corollaryname}{Corollary}}
  \addto\captionsenglish{\renewcommand{\definitionname}{Definition}}
  \addto\captionsenglish{\renewcommand{\examplename}{Example}}
  \addto\captionsenglish{\renewcommand{\lemmaname}{Lemma}}
  \addto\captionsenglish{\renewcommand{\propositionname}{Proposition}}
  \addto\captionsenglish{\renewcommand{\questionname}{Question}}
  \addto\captionsenglish{\renewcommand{\remarkname}{Remark}}
  \addto\captionsenglish{\renewcommand{\theoremname}{Theorem}}
  \addto\captionsngerman{\renewcommand{\claimname}{Behauptung}}
  \addto\captionsngerman{\renewcommand{\corollaryname}{Korollar}}
  \addto\captionsngerman{\renewcommand{\definitionname}{Definition}}
  \addto\captionsngerman{\renewcommand{\examplename}{Beispiel}}
  \addto\captionsngerman{\renewcommand{\lemmaname}{Lemma}}
  \addto\captionsngerman{\renewcommand{\propositionname}{Satz}}
  \addto\captionsngerman{\renewcommand{\questionname}{Frage}}
  \addto\captionsngerman{\renewcommand{\remarkname}{Bemerkung}}
  \addto\captionsngerman{\renewcommand{\theoremname}{Theorem}}
  \providecommand{\claimname}{Behauptung}
  \providecommand{\corollaryname}{Korollar}
  \providecommand{\definitionname}{Definition}
  \providecommand{\examplename}{Beispiel}
  \providecommand{\lemmaname}{Lemma}
  \providecommand{\propositionname}{Satz}
  \providecommand{\questionname}{Frage}
  \providecommand{\remarkname}{Bemerkung}
\providecommand{\theoremname}{Theorem}

\begin{document}

\title{Algebraische Geometrie I}

\author{Prof. Dr. Venjakob}

\date{Vorlesung 17, 19 Oktober 2018}
\maketitle

\section*{Literatur}
\begin{itemize}
\item G�rtz, Wedhorn. \emph{Algebraic Geometry I}
\item Hartshorne. \emph{Algebraic Geometry}
\item Shafarevich. \emph{Basic Algebraic Geometry 1 \& 2}
\item Grothendieck. \emph{El�ments de g�ometrie alg�brique, EGA I-IV}
\end{itemize}

\paragraph{Kommutative Algebra}
\begin{itemize}
\item Br�ske, Ischebeck, Vogel. \emph{Kommutative Algebra}
\item Kunz. \emph{Einf�hrung in die kommutative Algebra und algebraische
Geometrie}
\end{itemize}
\tableofcontents{}

\newpage{}

\part{Pr�-Variet�ten}


\section{Einf�hrung}

\textbf{Algebraische Geometrie}\index{Algebraische Geometrie} kann
man verstehen, als das Studium von Systemen polynomialer Gleichungen
(in mehreren Variabelen). Damit ist die algebraische Geometrie eine
Verallgemeinerung der \textbf{linearen Algebra}, also statt $X$ auch
$X^{n}$, und auch der \textbf{Algebra}, durch Polynome in \emph{mehreren
}Variabelen.

\subsection{Frage}

Sei $k$ ein (algebraisch abgeschlossener) K�rper, und $f_{1},\ldots,f_{m}\in k[T_{1},\ldots,T_{n}]$
gegeben. Was sind die ``geometrischen Eigenschaften'' der Nullstellenmenge
\[
V(f_{1},\ldots,f_{n})\coloneqq\{(t_{1},\ldots,t_{n})\in k^{n}\mid f_{i}(t_{1},\ldots,t_{n})=0\ \forall i\}
\]


\subsection{Beispiel 1}

$f=T_{2}^{2}-T_{1}^{2}(T_{1}-1)\in k[T_{1},T_{2}]$. Die Nullstellenmenge
f�r $k=\mathbb{R}$ (\emph{aber: }tr�gerisch, da $\mathbb{R}$ nicht
algebraisch abgeschlossen!)

\begin{figure}
\caption{$T_{2}^{2}=T_{1}^{2}(T_{1}-1)=T_{1}^{3}-T_{1}^{2}$}
\end{figure}


\paragraph{Dimension 1. }

Glatte und singul�ren Punkten: $(0,0)$ singul�r. Alle anderen Punkte
verletzen eine eindeutig bestimmte Tangente.

\begin{figure}[h]
\caption{\textbf{Spitze} und \textbf{Doppelpunkt}}
\end{figure}

Vergleiche den \textbf{Satz �ber implizite Funktionen}. (Analysis,
Differentialgeometrie) 

$V(f)$ ist lokal diffeomorph zu $\mathbb{R}$ (= reelle Gerade) im
Punkt $(x_{1},x_{2})$ genau dann, wenn die Jacobi-Matrix 
\[
\left(\frac{\partial f}{\partial T_{1}},\frac{\partial f}{\partial T_{2}}\right)=\left(T_{1}(3T_{1}-2),\ 2T_{2}\right)
\]
 hat Rang 1 in $(x_{1},x_{2})$. Das ist �quivalent dazu, dass $(x_{1},x_{2})\neq(0,0)$.
Dies l�sst sich rein formal �ber beliebigen Grundk�rpern \textbf{algebraisch}
formulieren.

\subsection{Methoden.}

GAGA - G�ometrie alg�brique, g�ometrique analytique (Serre)\medskip{}

\begin{tabular}{|c|c|}
\hline 
Komplexe Geometrie ($\mathbb{C}$), Differentialgeometrie $(\mathbb{R})$ & Algebraische Geometrie\tabularnewline
\hline 
\hline 
Analytische Hilfsmittel & Kommutative Algebra\tabularnewline
\hline 
\end{tabular}



\section{Die Zariski-Topologie}
\label{sec:zariski-topologie}

\begin{defn}
\label{def:verschwindungsmenge}
Sei $M\subseteq k[T_{1},\ldots,T_{n}]=:k[\underline{T}]$ eine Teilmenge.
Mit
\[
V(M) :=\{(t_{1},\ldots,t_{n})\in k^n \mid f(t_{1},\ldots,t_{n})=0\ \boldsymbol{\forall f\in M}\}
\]

bezeichnen wir die gemeinsame \textbf{Nullstellen-(Verschwindungs-)Menge}\index{Nullstellen-Menge}
der Elemente aus $M$. (Manchmal auch $V(f_{i},i\in I)$ statt $V(\{f_{i},i\in I\})$.
\end{defn}

%% TODO:
%% das oben weg!!!
\paragraph{Notation}
Wir schreiben auch $V(f_i, i \in I)$ statt $V(\{f_i \mid i \in I\})$

\subsection{Eigenschaften}
\label{subsec:zariski-topologie-eigenschaften}
\begin{itemize}
\item $V(M)=V(\mathfrak{a})$, wenn $\mathfrak{a}=\langle M\rangle_{k[\underline{T}]}$ das
\emph{von} $M$ \emph{erzeugte Ideal} in $k[\underline{T}]$ bezeichnet.
\item Da $k[\underline{T}]$ noethersch (Hilbertscher Basissatz) ist, reichen
stets endlich viele $f_{1},\ldots,f_{n}\in M$:
\[
V(M)=V(f_{1},\ldots,f_{n})\qquad\text{falls }\mathfrak{a}=\langle f_{1},\ldots,f_{n}\rangle_{k[\underline{T}]}.
\]
\item $V(-)$ ist \textbf{inklusionsumkehrend}, $M'\subseteq M\implies V(M)\subseteq V(M')$.
\end{itemize}
\begin{prop}
\label{propdef:zariski-topologie}
Die Mengen $V(\mathfrak{a})$, $\mathfrak{a} \unlhd k[\underline{T}]$
ein Ideal, sind die \textbf{abgeschlossenen} Mengen einer Topologie
auf $k^{n}$, der sogenannten \textbf{Zariski-Topologie}\index{Zariski-Topologie}.
\begin{enumerate}
\item $\emptyset=V\left((1)\right)$, $k^{n}=V(0)$. 
\item $\bigcap_{i\in I}V(\mathfrak{a}_{i})=V\left(\sum_{i\in I}\mathfrak{a}_{i}\right)$
f�r beliebige Familien $(\mathfrak{a}_{i})_{i \in I}$ von Idealen.
\item $V(\mathfrak{a})\cup V(\mathfrak{a})=V(\mathfrak{ab})$ f�r $\mathfrak{a},\mathfrak{b}\unlhd k[\underline{T}]$
Ideale.
\end{enumerate}
\end{prop}
\begin{proof}
�bung / Algebra II. 

\-
\end{proof}




\section{Affine algebraische Mengen}
\label{sec:algebraische-mengen}
\begin{defn}
  \label{def:algebraische-mengen}
  \mbox{}
  \begin{itemize}
  \item $\mathbb{A}^{n}(k)$, der $\textbf{affine Raum der Dimension n}$ (�ber $k$),
    bezeichne $k^{n}$ mit der Zariski-Topologie.
  \item Abgeschlossene Teilmengen von $\mathbb{A}^{n}(k)$ hei�en affine abgeschlossene
    Mengen.
  \end{itemize}
\end{defn}
\begin{example}
  \label{bsp:algebraische-mengen-dim1}
  Da $k[T]$ ein Hauptidealring ist, sind die abgeschlossen Mengen in
  $\mathbb{A}^{1}(k)$: $\emptyset$, $\mathbb{A}^{1}$, Mengen der
  Form $V(f)$, $f\in k[T]\backslash\{k\}$ (endliche Teilmengen).%
  \begin{comment}
    F�r $f\in k$ ist $V(f)=\mathbb{A}^{1}$, denn die Einheiten im Polynomring
    $k[T]$ sind gegeben durch $k^{\times}$, und Ideale erzeugt von einer
    Einheit bilden den ganzen Ring. (siehe Algebra 1)
  \end{comment}
  {} Insbesondere sieht man, dass die Zariski-Topologie im Allgemeinen
  nicht Hausdorff ist. 
\end{example}
% 
\begin{example}
  \label{bsp:algebraische-mengen-dim2}
  $\mathbb{A}^{2}(k)$ hat zumindestens als abgeschlossene Mengen:
  \begin{itemize}
  \item $\emptyset$, $\mathbb{A}^{2}$;
  \item Einpunktige Mengen: $\{(x_{1},x_{2})\}=V(T_{1}-x_{1},T_{2}-x_{2})$;
  \item $V(f)$, $f\in k[T_{1},T_{2}]$ irreduzibel. 
  \end{itemize}
  Ferner alle endlichen Vereinigungen dieser Liste. (Dies sind in der
  Tat alle, denn sp�ter sehen wir: ``irreduzible'' abgeschlossene
  Mengen entsprechen den \emph{Primidealen}, und $k[T_{1},T_{2}]$ hat
  ``Krull-Dimension $2$''.)
\end{example}




\section{Der Hilbertsche Nullstellensatz}
\label{sec:nullstellensatz}
\begin{prop}
\label{prop:nullstellensatz}
Sei $K$ ein (nicht notwendigerweise algebraisch abgeschlossener) K�rper,
und $A$ eine endlich erzeugte $K$-Algebra. Dann ist $A$ Jacobson'sch,
d.h. f�r jedes Primideal $\mathfrak{p}\unlhd A$ gilt:
\[
\mathfrak{p}=\bigcap_{\mathfrak{m}\supseteq\mathfrak{p}}\mathfrak{m},\quad\mathfrak{m}\text{ maximales Ideal}
\]

Ist $\mathfrak{m}\unlhd A$ ein maximales Ideal, so ist die K�rpererweiterung
$K\subseteq A/\mathfrak{m}$ endlich.
\end{prop}
\begin{proof}
Algebra II / kommutative Algebra.
\end{proof}
\begin{cor}
\label{cor:nullstellensatz}
\mbox{}
\begin{enumerate}
\item Sei $A$ eine e.e. (endlich erzeugte) $k$-Algebra ($k$ sei algebraisch
abgeschlossen), $\mathfrak{m}\unlhd A$ ein maximales Ideal. Dann
ist $A/\mathfrak{m}=k$. 
\item Jedes maximale Ideal $\mathfrak{m}\unlhd k[\underline{T}]$ ist von der
Form $\mathfrak{m}=(T_{1}-x_{1},\ldots,T_{n}-x_{n})$ mit $x_{1},\ldots,x_{n}\in k$.
\item F�r ein Ideal  $\mathfrak{a}\unlhd k[\underline{T}]$ gilt:
\[
\rad(\mathfrak{a})=\sqrt{\mathfrak{a}}\overset{(i)}{=}\bigcap_{\mathfrak{a}\subseteq\mathfrak{p}\unlhd k[\underline{T}], \mathfrak{p} \text{prim}}\mathfrak{p}\overset{(ii)}{=}\bigcap_{\mathfrak{a}\subseteq\mathfrak{m}\unlhd k[\underline{T}], \mathfrak{m} \text{maximal}}\mathfrak{m}
\]
\end{enumerate}
\end{cor}
\begin{proof}
\mbox{}
\begin{enumerate}
\item $k\rightarrow A\rightarrow A/\mathfrak{m}$ ist Isomorphismus,  da
$k$ keine echte algebraische K�rpererweiterung besitzt.
\item Es ist
\begin{align*}
k[T_{1},\ldots,T_{n}] & \twoheadrightarrow k[\underline{T}]/\mathfrak{m}=k\\
T_{i} & \mapsto x_{i}
\end{align*}
surjektiv. Es folgt: $\mathfrak{m}=(T_{1}-x_{1},\ldots,T_{n}-x_{n})$, da letzteres
bereits maximal ist. ($\supseteq$ klar.)
\item (i) Algebra II. (ii) Theorem.
\end{enumerate}
\end{proof}




\section{Korrespondenz zwischen Radikalidealen und affinen algebraischen Mengen}

Sei $V(\mathfrak{A})\subseteq\mathbb{A}^{n}(k)$ affin algebraische
Menge, $\mathfrak{A}\subset k[\underline{T}]$.\textbf{ Es gilt:}
\[
V(\mathfrak{A})=V(\rad\mathfrak{A})
\]

mit $\rad\mathfrak{A}=\{f\in k[\underline{T}]\mid f^{n}\in\mathfrak{A}\text{ f�r }n>0\}$,
da
\[
f^{n}(x)=0\Leftrightarrow f(x)=0,
\]

d.h. verschiedene Ideale k�nnen dieselbe algebraische Menge beschreiben.
\begin{defn}
F�r eine Teilmenge $Z\subseteq\mathbb{A}^{n}(k)$ bezeichne
\[
I(Z)=\{f\in k[\underline{T}]\mid f(x)=0\ \forall x\in Z\}
\]

das Ideal aller auf $Z$ verschwindenden Polynomfunktionen.
\end{defn}
\begin{prop}
\mbox{}
\begin{enumerate}
\item Sei $\mathfrak{A}\subset k[\underline{T}]$ Ideal. Dann ist $I(V(\mathfrak{A}))=\rad(\mathfrak{A})$.
\item Sei $Z\subseteq\mathbb{A}^{n}(k)$ Teilmenge. Dann ist $V(I(Z))=\overline{Z}$,
der Abschluss von $Z$.
\end{enumerate}
\end{prop}
\begin{proof}
�bungsblatt 2.
\end{proof}
\medskip{}

$\mathfrak{A}$ hei�t \textbf{Radikalideal}\index{Radikalideal},
wenn $\mathfrak{A}=\rad(\mathfrak{A})$, oder �quivalent wenn $k[\underline{T}]/\mathfrak{A}$
\emph{reduziert} ist, d.h. keine nilpotente Elemente hat.
\begin{cor}
Wir erhalten eine 1-1 Korrespondenz
\begin{align*}
\{\text{abg. Mengen }\subseteq\mathbb{A}^{n}\} & \leftrightarrow\{\text{Radikalideale }\mathfrak{A}\subset k[\underline{T}]\}\\
Z & \mapsto I(Z)\\
V(\mathfrak{A}) & \mapsfrom\mathfrak{A}
\end{align*}

die sich zu einer 1-1 Korrespondenz
\begin{align*}
\left\{ \text{Punkte in }\mathbb{A}^{n}\right\}  & \leftrightarrow\left\{ \text{max. Ideale in }k[\underline{T}]\right\} \\
x=(x_{1},\ldots,x_{n}) & \mapsto\begin{array}{rl}
\mathfrak{m}_{x} & =I(\{x\})\\
 & =\ker(k[\underline{T}]\rightarrow k,\ T_{i}\mapsto x_{i})
\end{array}
\end{align*}

einschr�nkt.
\end{cor}



\include{Part1/AlgGeo1-Part1-6_Irreduzible-topologische-Raeume}


\section{Irreduzible affine algebraische Mengen}\label{sec:irreduzibilitaet-alg}
\begin{lem}\label{lem:charakterisierung-irreduzibel-alg}
Eine abgeschlossene Teilmenge $Z\subseteq\mathbb{A}^{n}(k)$ ist genau
dann irreduzibel, wenn $I(Z) \unlhd k[\underline{T}]$ ein Primideal ist. Insbesondere ist
$\mathbb{A}^{n}(k)$ irreduzibel.  
\end{lem}
\begin{proof}
$Z$ irreduzibel ist �quivalent zu 
\begin{align*}
 & (Z=\underbrace{V(\mathfrak{a})}_{\bigcap_{i} V(f_{i})}\cup\underbrace{V(\mathfrak{b})}_{\bigcap_{j} V(g_{j})}\quad\Rightarrow\quad V(\mathfrak{a})=Z\text{ oder }V(\mathfrak{b})=Z).\\
\Leftrightarrow\  & \forall f,g\in k[\underline{T}]:\ V(fg)=V(f)\cup V(g)\supseteq Z:\ V(f)\supseteq Z\text{ oder }V(g)\supseteq Z.\\
(*)\Leftrightarrow\  & \forall f,g\in k[\underline{T}]:\ fg\in I(V(fg))\subseteq I(Z):\ f\in I(Z)\text{ oder }g\in I(Z).\\
\Leftrightarrow\  & I(Z)\text{ ist Primideal.}
\end{align*}

({*}): $V(I(Z))=Z$, $I(V(\mathfrak{a}))=\rad(\mathfrak{a})$. 
\end{proof}
\begin{rem}\label{rem:korrespondenz-irreduzibel-prim}
Die Korrespondenz aus Korollar 11 schr�nkt sich ein zu
\[
\{\text{irred. abg. Teilmengen }\subseteq\mathbb{A}^{n}\}\overset{1:1}{\leftrightarrow}\{\text{Primideale in }k[\underline{T}]\}
\]
\end{rem}




\section{Quasikompakte und noethersche topologische R�ume}
\begin{defn}
Ein topologischer Raum $X$ hei�t \textbf{quasikompakt}\index{quasikompakt},
wenn jede offene �berdeckung von $X$ eine \emph{endliche} Teil�berdeckung
enth�lt. (,,quasi`` deutet an, dass $X$ in der Regel nicht Hausdorff'sch
ist!). Er hei�t \textbf{noethersch}\index{noethersch}, wenn jede
absteigende Kette
\[
X\supseteq Z_{1}\supseteq Z_{2}\supseteq\cdots
\]

abgeschlossener Teilmengen von $X$ station�r wird ($\Leftrightarrow$
jede aufsteigende Kette offener Teilmengen wird station�r).
\end{defn}
\begin{lem}
Sei $X$ ein noetherscher topologischer Raum. Dann gilt:
\begin{enumerate}
\item Jede abgeschlossene Teilmenge $Z$ von $X$ ist noethersch.
\item Jede offene Teilmenge $U$ von $X$ ist quasikompakt.
\item Jeder abgeschlossene Teilraum $Z$ von $X$ besitzt nur endlich viele
irreduzibele Komponenten.
\end{enumerate}
\end{lem}
\begin{proof}
\mbox{}
\begin{enumerate}
\item Nach Definition, da abgeschlossene Mengen von $Z$ auch solche von
$X$ sind.
\item $U=\bigcup_{i\in I}U_{i}$ offen; $\mathbb{A}$ nicht quasikompakt.
Dann ist $I_{1}\subset I_{2}\subset\cdots\subset I$ endliche Teilmenge
mit
\[
V_{1}\subsetneq V_{2}\subsetneq\cdots\neq U\quad\text{f�r }V_{j}=\bigcup_{i\in I}U_{i}.
\]
Widerspruch zu noethersch.
\item Es reicht zu zeigen: Jeder noethersche Raum ist Vereinigung endlich
vieler irreduzibeler Teilmengen. Da $X$ noethersch ist, folgt mit
dem \emph{Lemma von Zorn} dass jede nicht-leere Menge von algebraischen
Teilmengein in $X$ ein minimales Element besitzt. 
\[
\mathbb{A}:\quad\emptyset\neq\mathcal{M}:=\left\{ Z\subset X\text{ abg.}\mid Z\text{ ist \textbf{nicht} endl. Ver. irred. Mengen}\right\} 
\]
$\Rightarrow\exists$ minimales Element, sagen wir $Z$, in $\mathcal{M}$.

$\Rightarrow Z$ ist nicht irreduzibel.

$\Rightarrow Z=Z_{1}\cup Z_{2}$ mit $Z_{1},Z_{2}\subsetneq Z$ abgeschlossen.

$\Rightarrow$ ($Z$ minimal) $Z_{1},Z_{2}\notin\mathcal{M}$

$\Rightarrow Z\notin\mathcal{M}$. Widerspruch.

\end{enumerate}
\end{proof}
\begin{prop}
Jeder abgeschlossene Teilraum $X\subseteq\mathbb{A}^{n}(k)$ ist noethersch.
\end{prop}
\begin{proof}
Nach dem obigen Lemma ist nur zu zeigen, dass $\mathbb{A}^{n}(k)$
noethersch ist.

Absteigende Ketten abgeschlossener Teilmengen sind nach \emph{Korollar
11} in 1-1 Korrespondenz mit aufsteigenden Ketten von (Radikal-)Ideale
in $k[\underline{T}]$. Da $k[\underline{T}]$ nach dem Hilbertschen
Basissatz noethersch ist, werden letzere Ketten station�r.
\end{proof}
\begin{cor}[Prim�rzerlegung]
Sei $\mathfrak{A}=\rad(\mathfrak{A})\subseteq k[\underline{T}]$
ein Radikalideal. Dann gilt: $\mathfrak{A}$ ist Durchschnitt von
endlich vielen Primidealen, die sich jeweils nicht enthalten; diese
Darstellung ist eindeutig bis auf Reihenfolge.
\end{cor}
\begin{proof}
$V(\mathfrak{A})=\bigcup_{i=1}^{n}V(\mathfrak{b}_{i})$, $\mathfrak{b}_{i}$
Primideal.%
\begin{comment}
Station�re Kette folgt aus noethersch (Satz 21); mit Bemerkung 16
bzw. Lemma 17 folgt, dass die $\mathfrak{b}_{i}$ Primideale sind.
\end{comment}
{} \textcolor{blue}{Mit Satz 10 folgt:}
\[
\mathfrak{A}=\rad(\mathfrak{A})=I(V(\mathfrak{A}))=\bigcap_{i=1}^{n}\underbrace{I(V(\mathfrak{b}_{i}))}_{\mathfrak{b}_{i}'\text{ max. Primideale (L. 17)}}
\]
\end{proof}



\selectlanguage{english}%

\section{Morphismen von affinen algebraischen Mengen}
\begin{defn}
Seien $X\subseteq\mathbb{A}^{m}(k)$, $Y\subseteq\mathbb{A}^{n}(k)$
affine algebraische Mengen. Ein \textbf{Morphismus} $X\rightarrow Y$
affiner algebraischer Mengen ist eine Abbildung $f:X\rightarrow Y$
der zugrundeliegenden Mengen, sodass $f_{1},\ldots,f_{n}\in k[T_{1},\ldots,T_{m}]$
existieren, derart dass $\forall x\in X$ gilt:
\[
f(x)=(f_{1}(x),\ldots,f_{n}(x)).
\]
\emph{Bezeichne }daf�r $\hom(X,Y)$ Menge der Morphismen $X\rightarrow Y$. 
\end{defn}
\begin{rem}
$f:X\rightarrow Y$ l�sst sich immer fortsetzen zu einem Morphismus
\[
f:\mathbb{A}^{n}(k)\rightarrow\mathbb{A}^{m}(k),
\]

aber nicht eindeutig, es sei denn $X=\mathbb{A}^{m}(k)$.
\end{rem}

\paragraph{Komposition}

\[
\xymatrix@C=9pc{X\ar[r]_{f}^{f_{1},\ldots,f_{n}\in k[T_{1},\ldots,T_{m}]} & Y\ar[r]_{g}^{g_{1},\ldots,g_{r}\in k[T_{1}',\ldots,T_{m}']} & Z}
\]

mit $X\subseteq\mathbb{A}^{m}(k)$, $Y\subseteq\mathbb{A}^{n}(k)$,
$Z\subseteq\mathbb{A}^{r}(k)$. Es folgt:
\begin{align*}
g(f(x))=\, & (g_{1}(f_{1}(x),\ldots,f_{n}(x)),\ldots,g_{r}(f_{1}(x),\ldots,f_{n}(x))\\
:=\, & h_{1}(x),\ldots,h_{r}(x)
\end{align*}

d.h. $g\circ f$ ist durch Polynome $h_{i}\in k[T_{1},\ldots,T_{m}]$
gegeben, d.h. $g\circ f$ ist wieder ein Morphismus affiner algebrasischer
Mengen. Wir erhalten die \textbf{Kategorie affiner algebraischer Mengen}.
\begin{example}
\mbox{}
\begin{enumerate}
\item Sei die Abbildung
\begin{align*}
\mathbb{A}^{1}(k) & \rightarrow V(T_{2}-T_{1}^{2})\subseteq\mathbb{A}^{2}(k)\\
x & \mapsto(x,x^{2}).
\end{align*}
Diese Abbildung ist sogar ein \emph{Isomorphismus }affiner algebraischer
Mengen, da die Umkehrabbildung
\[
(x,y)\mapsto x
\]
ebenfalls ein Morphismus ist.
\item Sei char$(k)\neq2$. Die Abbildung
\begin{align*}
\mathbb{A}^{1}(k) & \rightarrow V(T_{2}^{2}-T_{1}^{2}(T_{1}+1))\\
x & \mapsto(x^{2}-1,x(x^{2}-1))
\end{align*}
ist ein Morphismus, aber \emph{nicht }bijektiv, da $1,-1$ beide auf
$(0,0)$ abgebildet werden.\selectlanguage{ngerman}%
\end{enumerate}
\end{example}




\section{Unzul�nglichkeiten des Begriffs der affinen algebraischen Mengen}
\label{sec:unzulaenglichkeiten-alg-mengen}
\begin{enumerate}
\item Offene Teilmengen affiner algebraischer Mengen tragen nicht in nat�rlicher
  Weise die Struktur einer affinen algebraischen Menge.
\item Insbesondere k�nnen wir affine algebraische Mengen nicht entlang offener
  Teilr�ume verkleben. (vgl. Mannigfaltigkeiten.)
\item Keine Unterscheidungsm�glichkeiten z.B. zwischen $\{(0,0)\}$, $V(T_{1})\cap V(T_{2})$
  und $V(T_{2})\cap V(T_{1}^{2}-T_{2})\subseteq\mathbb{A}^{2}(k)$,
  obwohl die ``geometrische Situation'' offensichtlich verschieden
  ist.
\end{enumerate}
Um die Punkte 1 und 2 zu verbessern, gehen wir im Folgenden zu ``R�umen mit Funktionen'' �ber, und verzichten darauf,
dass sich diese in einen affinen Raum $\mathbb{A}^{n}(k)$ einbetten
lassen.

Der Punkt 3 ist eine Motivation daf�r, sp�ter Schemata einzuf�hren.
(subtiler)




\paragraph*{Affine algebraische Mengen als R�ume von Funktionen}

\section{Der affine Koordinatenring}
\label{sec:koordinatenring}

Sei $X\subseteq\mathbb{A}^{n}(k)$ abgeschlossen. F�r den surjektiven
(Def. von Morphismen) $k$-Algebren-Homomorphismus
\begin{align*}
k[\underline{T}] & \xrightarrow{\varphi}\hom(X,\mathbb{A}^{1}(k))\\
f & \mapsto(x\mapsto f(x)),
\end{align*}

wobei die Morphismen in folgende Weise eine $k$-Algebra bilden:
\begin{align*}
(f+g)(x) & :=f(x)+g(x)\\
(fg)(x) & :=f(x)g(x)\\
(\alpha f)(x) & :=\alpha f(x)
\end{align*}

mit $f,g\in\hom(X,\mathbb{A}^{1}(k))$, $\alpha\in k$, gilt:
\[
\ker\varphi=I(X).
\]
\begin{defn}
\label{def:koordinatenring}
$\Gamma(X):=k[\underline{T}]/I(X)\cong_{k-\mathrm{Alg}}\hom(X,\mathbb{A}^{1}(k))$ hei�t der \textbf{affine
Koordinatenring }von $X$.

F�r $x=(x_{1},\ldots,x_{n})\in X$ gilt:
\begin{align*}
\mathfrak{m}_{x}:=\  & \ker(\Gamma(X)\twoheadrightarrow k,\,f\mapsto f(x))\\
=\  & \{f\in\Gamma(X)\mid f(x)=0\}\\
=\  & \pi((T_{1}-x_{1},\ldots,T_{n}-x_{n}))\\
=\  & \ker(\Gamma(\mathbb{A}^{n}(k))\twoheadrightarrow k)
\end{align*}

unter der Projektion $\pi:k[\underline{T}]=\Gamma(\mathbb{A}^{n}(k))\twoheadrightarrow\Gamma(X)$.
Es ist $\mathfrak{m}_{x}$ ein maximales Ideal von $\Gamma(X)$ mit
$\Gamma(X)/\mathfrak{m}_{x}\cong k$. F�r ein Ideal $\mathfrak{a}\unlhd\Gamma(X)$
sei
\[
V(\mathfrak{a}):=\{x\in X\mid f(x)=0\ \forall f\in\mathfrak{a}\}=V(\pi^{-1}(\mathfrak{a}))\cap X.
\]

Dies sind genau die abgeschlossenen Mengen von $X$ als Teilraum von
$\mathbb{A}^{n}(k)$ mit der induzierten Topologie, diese wird auch
\textbf{Zariski-Topologie} genannt. F�r $f\in\Gamma(X)$ setze:
\[
D_X(f) := D(f):=\{x\in X\mid f(x)\neq0\}=X\setminus V(f).
\]
\end{defn}
\begin{lem}
\label{lem:basis-zariski-topologie}
Die offenen Mengen $D(f)$, $f\in\Gamma(X)$, bilden eine Basis der
Topologie von $X$, d.h.
\[
\forall U\subseteq X\text{ offen }\exists f_{i}\in\Gamma(X),\,i\in I\quad\text{mit }U=\bigcup_{i\in I}D(f_{i})
\]
\end{lem}
\begin{proof}
$U=X\backslash V(\mathfrak{a})$ f�r ein $\mathfrak{a}\unlhd\Gamma(X)$,
$\mathfrak{a}=\langle f_{1},\ldots,f_{n}\rangle_{\Gamma(X)}$ . Wegen
\[
V(\mathfrak{a})=\bigcap_{i=1}^{n}V(f_{i})\quad\Rightarrow\quad U=\bigcup_{i=1}^{n}D(f_{i})
\]

Es reichen also sogar endlich viele $f_{i} \in \Gamma(X)$! 
\end{proof}
\begin{prop}
\label{prop:eigenschaften-koordinatenring}
Der Koordinatenring $\Gamma(X)$ einer affinen algebraischen Menge
$X$ ist eine endlich erzeugte $k$-Algebra, die reduziert ist (d.h.
keine nilpotenten Elemente $\neq0$ enth�lt). Ferner ist $X$ irreduzibel
genau dann, wenn $\Gamma(X)$ integer ist.
\end{prop}
\begin{proof}
$k[\underline{T}]\twoheadrightarrow\Gamma(X)$ impliziert, dass $\Gamma(X)$ als $k$-Algebra endlich
erzeugte ist. Es gilt:
\[
\Gamma(X)\text{ irreduzibel }\Leftrightarrow I(X)=\rad I(X).
\]

Denn mit Satz 10.ii) und Korollar 11 folgt:
\begin{align*}
X & =V(\mathfrak{a}):\,I(X)=\rad\mathfrak{a}\\
\Rightarrow\rad I(X) & =\rad\rad\mathfrak{a}=\rad\mathfrak{a}=I(X).
\end{align*}

Mit Lemma 17 folgt: $X$ irreduzibel

$\phantom{\quad}\Leftrightarrow I(X)$ prim

$\phantom{\quad}\Leftrightarrow\Gamma(X)=k[\underline{T}]/I(X)$ integer.
\end{proof}




\section{Funktorielle Eigenschaften von $\Gamma(X)$}
\label{sec:koordinatenring-funktiorialitaet}
\begin{prop}
  \label{prop:koordinatenringfunktor}
  Für einen Morphismus $X\xrightarrow{f}Y$ affiner algebraischer Mengen
  definiert 
  \begin{align*}
    \Gamma(f):\quad\Gamma(Y) & \rightarrow\Gamma(X)\\
    g & \mapsto g\circ f
  \end{align*}
  ein Homomorphismus von $k$-Algebren. Der so definierte \emph{kontravariante}
  Funktor
  \[
    \Gamma:\{\text{affine algebraische Mengen}\}\rightarrow\{\text{reduzierte endl. erz. }k\text{-Algebren}\}
  \]
  liefert eine Kategorienäquivalenz, welche durch Einschränkung eine Äquivalenz
  \[
    \Gamma:\{\text{irred. aff. alg. Mengem\}}\rightarrow\{\text{integre endl. erz. }k\text{-Algebren\}}
  \]
  induziert.
\end{prop}
\begin{proof}
  Sei $Y\xrightarrow{g}\mathbb{A}^{1}(k)\in\Gamma(Y)$ ein Morphismus. Es
  folgt:
  \[
    g\circ f:X\xrightarrow{f}Y\xrightarrow{g}\mathbb{A}^{1}(k)
  \] 
  ist Morphismus,  d.h. $g \circ f\in\Gamma(X)$. $\Gamma(f):\Gamma(Y)\rightarrow\Gamma(X)$
  ist ein $k$-Algebren-Homomorphismus mit $\Gamma(\text{id}_{X})=\text{id}_{\Gamma(X)}$. Da ferner gilt, dass $\Gamma(f_{1}\circ f_{2})=\Gamma(f_{2})\circ\Gamma(f_{1})$ ist $\Gamma$ ein kontravarianter Funktor.
  \begin{claim*}
    $\Gamma$ ist volltreu, d.h.
    \begin{align*}
      \Gamma:\hom(X,Y) & \rightarrow\hom_{k\text{-Alg}}(\Gamma(Y),\Gamma(X))\\
      f & \mapsto\Gamma(f)
    \end{align*}
    ist \emph{bijektiv} für alle affinen algebraischen Mengen $X,Y$.
  \end{claim*}
  \begin{proof}
    Wir konstruieren eine Umkehrabbildung wie folgt: Zu $\varphi:\Gamma(Y)\rightarrow\Gamma(X)$
    für $X\subseteq\mathbb{A}^{m}(k)$, $Y\subseteq\mathbb{A}^{n}(k)$ existiert ein Lift $\tilde\varphi$, s.d.
    \[
      \xymatrix{k[T_{1}',\ldots,T_{n}']\ar[r]^{\tilde{\varphi}}\ar@{->>}[d] & k[T_{1},\ldots,T_{m}]\ar@{->>}[d]\\
        \Gamma(Y)\ar[r]^{\varphi} & \Gamma(X)
      }
    \]
    kommutiert; $\tilde{\varphi}(T_{i}'):= f_i$ mit $f_i \in \pi^{-1}(\varphi(T_{i}')) \subseteq k[T_1,...,T_n]$, wobei $\pi : k[\underline{T}] \to \Gamma(X)$ die kanonische Projektion bezeichne. 
    Definiere:
    \begin{align*}
      f:X & \rightarrow Y\\
      x=(x_{1},\ldots,x_{n}) & \mapsto(\tilde{\varphi}(T_{1}')(x_{1},\ldots,x_{n}),\ldots,\tilde{\varphi}(T_{n}')(x_{1},\ldots,x_{n}))
    \end{align*}
  \end{proof}
  \begin{claim*}
    $\Gamma$ ist essentiell surjektiv, d.h. zu jeder reduzierten endlich
    erzeugten $k$-Algebra $A$ existiert eine affine algebraische Menge
    $X$ mit $A\cong\Gamma(X)$.
  \end{claim*}
  \begin{proof}
    Da nach Voraussetzung $A\cong k[T]/\mathfrak{a}$ für ein Radikalideal
    $\mathfrak{a}$, können wir etwa $X:=V(\mathfrak{a})\subseteq\mathbb{A}^{n}(k)$
    setzen. Der Rest folgt aus Satz 28.
  \end{proof}
\end{proof}
\begin{prop}
  \label{prop:funktiorialitaet-specm}
  Sei $f:X\rightarrow Y$ ein Morphismus affiner algebraischer Mengen und $\Gamma(f):\Gamma(Y)\rightarrow\Gamma(X)$
  der zugehörige Homomorphismus der Koordinatenringe. Dann gilt $\forall x\in X$:
  $\Gamma(f)^{-1}(\mathfrak{m}_{x})=\mathfrak{m}_{f(x)}$.
\end{prop}
\begin{proof}
  \[
    \Gamma(f)^{-1}(\mathfrak{m}_{x})=\{g\in\Gamma(Y)\mid g\circ f \in \mathfrak{m}_{x}\}=\{g\in\Gamma(Y)\mid g(f(x)) = 0 \} = \mathfrak{m}_{f(x)},
  \]
  da $\Gamma(f)(g) =g \circ f$.
\end{proof}


\selectlanguage{english}%

\section{R�ume mit Funktionen}

(Prototyp eines geometrischen Objektes, Spezialfall eines ``geringten
Raumes'' sp�ter.) Sei $k$ ein nicht notwendig algebraisch abgeschlossenen
K�rper.
\begin{defn}
\mbox{}
\begin{enumerate}
\item Ein \textbf{Raum mit Funktionen}\index{Raum mit Funktionen} besteht
aus den folgenden Daten:
\begin{itemize}
\item ein topologischer Raum $X$;
\item eine Familie von Unter-$K$-Algebren
\[
\mathcal{O}(U)\subseteq\text{Abb}(U,K),\quad\forall U\subseteq X\text{ offen }d.d
\]

\begin{enumerate}
\item Sind $U'\subset U\subset X$ offen und $f\in\mathcal{O}(U)$ so ist
$f|_{U'}\in\text{Abb}(U',K)$ in $\mathcal{O}(U')$.
\item (\textbf{Verklebungsaxiom}\index{Verklebungsaxiom}) Sind $U_{i}\subset X$
offen, $i\in I$, ist $U=\bigcup_{i\in I}U_{i}$, und sind $f_{i}\in\mathcal{O}(U_{i})$,
$i\in I$ gegeben mit
\[
f_{i}|_{U_{i}\cap U_{j}}=f_{j}|_{U_{i}\cap U_{j}}\quad\forall i,j\in I
\]
dann ist die eindeutige Abbildung
\[
f:U\rightarrow K\text{ mit }f|_{U_{i}}=f_{i}
\]
in $\mathcal{O}(U)$, bzw. $\exists_{1}f\in\mathcal{O}(U)$ mit $f|_{U_{i}}=f_{i}$.
\end{enumerate}
\end{itemize}
Bezeichne $\mathcal{O}$ oder $\mathcal{O}_{X}$ der oben genannten
Familie $(X,\mathcal{O}_{X})$, oder kurz bezeichne $X$ den Raum
mit Funktionen.
\item Ein \textbf{Morphismus}\index{Raum mit Funktionen!Morphismus} $(X,\mathcal{O}_{X})\rightarrow(Y,\mathcal{O}_{Y})$
von R�umen von Funktionen ist eine stetige Abbildung $g:X\rightarrow Y$,
so dass f�r alle $V\subseteq Y$ offen und $f\in\mathcal{O}_{Y}$
gilt:
\[
f\circ g|_{g^{-1}(V)}:g^{-1}(V)\rightarrow K
\]
liegt in $\mathcal{O}_{X}(g^{-1}(V))$.
\[
\xymatrix{X\ar[r]^{g} & Y\\
g^{1}(V)\ar[r]^{g|}\ar[d]_{f\circ g|_{f^{-1}(V)}}\ar@{^{(}->}[u] & V\ar[d]^{f}\ar@{^{(}->}[u]_{\text{offen}}\\
K\ar@{=}[r] & K
}
\]
\end{enumerate}
\end{defn}
Die R�ume von Funktionen �ber $K$ bilden eine Kategorie.
\begin{defn}[offene Unterr�ume von Funktionen]
F�r $(X,\mathcal{O}_{X})$ und $U\subset X$ offen bezeichne $(U,\mathcal{O}_{X|_{U}})$
den Raum mit Funktionen gegeben durch den topologischen Raum $U$
mit Funktionen $\mathcal{O}_{X|_{U}}(V):=\mathcal{O}_{X}(V)$ f�r
$V\underset{\text{offen}}{\subset}U\subset X$.
\end{defn}
\textbf{Ab jetzt} betrachten wir R�ume von Funktionen �ber $k$ algebraisch
abgeschlossen.\selectlanguage{ngerman}%




\section{Der Raum mit Funktionen zu einer affin algebraischen Menge}

\textbf{Ziel.} $X\subseteq\mathbb{A}^{n}(k)\mapsto(X,\mathcal{O}_{X})$
als irreduzibele affine algebraische Menge bzw. Zariski-Topologie.
D.h. wir m�ssen Mengen von Funktionen $\mathcal{O}_{X}(U)$ auf $U$,
$U\subset X$ offen, definieren. Diese werden als Teilmengen des Funktionenk�rpers
$K(X)$ definiert (dazu $X$ irreduzibel, sp�ter bei Schemata f�llt
diese Bedingung weg!)
\begin{defn}
$K(X):=\text{Quot}(\Gamma(X))$ hei�t \textbf{Funktionenk�rper} von
$X$. ($\Gamma(X)$ ist f�r $X$ irreduzibel nullteilerfrei.)

Elemente $\frac{f}{g}\in K(X)$, $f,g\in\Gamma(X)=\hom(X,\mathbb{A}^{1}(k))$,
$g\neq0$ lassen sich zumindest als Funktion auf der offenen Menge
$\mathcal{D}(g)\subset X$ auffassen, wenn auch nicht i.A. auf ganz
$X$.
\end{defn}
\begin{lem}
Gilt f�r $\frac{f_{1}}{g_{1}},\frac{f_{2}}{g_{2}}\in K(X)$, $f_{i},g_{i}\in\Gamma(X)$,
und einer offenen Teilmenge $\emptyset\neq U\subset\mathcal{D}(g_{1}g_{2})$
\[
\frac{f_{1}(x)}{g_{1}(x)}=\frac{f_{2}(x)}{g_{2}(x)}\qquad\forall x\in U,
\]

dann folgt $\frac{f_{1}}{g_{1}}=\frac{f_{2}}{g_{2}}$ in $K(X)$.
\end{lem}
\begin{proof}
Sei ohne Einschr�nkung der Allgemeinheit $g_{1}=g_{2}=g$. (Sonst
Erweitern!) 

$\Rightarrow(f_{1}-f_{2})(x)=0$ $\forall x\in U$.

$\Rightarrow\emptyset\neq U\subset V(f_{1}-f_{2})\subset X$ dicht,
d.h. $V(f_{1}-f_{2})=X$.

$\phantom{\Rightarrow\ }$$f_{1}-f_{2}\in IV(f_{1}-f_{2})=I(X)\equiv(0)$
in $\Gamma(X)$ 

$\Rightarrow f_{1}-f_{2}=0$.
\end{proof}
\begin{defn}
Sei $X$ eine irreduzibele affine algebraische Menge, $U\subset X$
offen. Sei $\Gamma(X)_{\mathfrak{m}_{x}}$ Lokalisierung von $\Gamma(X)$
bzgl. das maximale Ideal $\mathfrak{m}_{x}$ in $x\in X$.

\[
\mathcal{O}_{X}(U):=\bigcap_{x\in U}\Gamma(X)_{\mathfrak{m}_{x}}\subset K(X)
\]

d.h. f�r jedes $x\in U$ l�sst sich $f\in\mathcal{O}_{X}(U)$ schreiben
als $\frac{h}{g}$ mit $g(x)\neq0$.
\end{defn}
Wenn $f\in\Gamma(X)$ bezeichne $\Gamma(X)_{f}$ die Lokalisierung
von $\Gamma(X)$ bzgl. der multiplikativ abgeschlossenen Teilmenge
$\{1,f,f^{2},\ldots,f^{n}\ldots\}$. Dann l�sst sich
\[
\Gamma(X)_{\mathfrak{m}_{x}}=\bigcup_{f\in\Gamma(X)\backslash\mathfrak{m}_{x}}\Gamma(X)_{f}\subset K(X)
\]

schreiben. ``$\supset$'' klar, ``$\subset$'' $\frac{g}{f}$
mit $f(x)\neq0$ d.h. $f\notin\mathfrak{m}_{x}$ $\Rightarrow\frac{g}{f}\in\Gamma(X)_{f}$.


\paragraph{Es gilt:}
\begin{enumerate}
\item F�r $V\subset U\subset X$ offen kommutiert das folgende Diagramm:
\[
\xymatrix{\mathcal{O}_{X}(V)\ar@{^{(}->}[r] & \text{Abb}(V,k)\\
\mathcal{O}_{X}(U)\ar@{^{(}->}[r]\ar@{^{(}->}[u] & \text{Abb}(U,k)\ar[u]_{\text{Einschr�nkungsabb.}}
}
\]
mit $\mathcal{O}_{X}(U)\subset\mathcal{O}_{X}(V)$ nach Definition.
\item $\mathcal{O}_{X}(U)\rightarrow\text{Abb}(U,k)$, $f\mapsto(x\mapsto f(x):=\frac{g(x)}{f(x)}\in k)$
ist injektiv (Lemma 34) und wohldefiniert (k�rzen/Erweitern), wobei
$g,h\in\Gamma(X)$ mit $h\notin\mathfrak{m}_{x}$ mit $f=\frac{g}{h}$
nach Definition von $\mathcal{O}_{X}(U)$ existiert.
\item \textbf{Verklebungseigenschaft.} Sei $U=\bigcup_{i\in I}U_{i}$. Nach
Definition ist 
\begin{align*}
\mathcal{O}_{X}(U) & =\bigcap_{i}\mathcal{O}_{X}(U_{i})\subset K(X)\\
\ni f:U\rightarrow k & \quad\ni f_{i}:U_{i}\rightarrow k
\end{align*}
{[}Diagramm fehlt{]}. $\Rightarrow(X,\mathcal{O}_{X})$ ist Raum
mit Funktionen, \textbf{der zur irreduziblen affin algebraische Menge
geh�rige Raum von Funktionen.} 
\end{enumerate}
\begin{prop}[orig. 33]
F�r $(X,\mathcal{O}_{X})$ zu $X$ wie oben und $f\in\Gamma(X)$
gilt:
\[
\mathcal{O}_{X}(D(f))=\Gamma(X)_{f},
\]

insbesondere $\mathcal{O}_{X}(X)=\Gamma(X)$.
\end{prop}
\begin{proof}
$\Gamma(X)\subset\mathcal{D}(f)$ klar, da $f(x)\neq0$ $\forall x\in\mathcal{D}(f)$
bzw. $f\in P(X)\backslash\mathfrak{m}_{x}$. 

Sei nun $g$ in $\mathcal{O}_{X}(\mathcal{D}(f))$ gegeben, $(*)$
und $\mathfrak{A}:=\{h\in\Gamma(X)\mid hg\in\Gamma(X)\}\subset\Gamma(X)$
Ideal.

Dazu: $g\in\Gamma(X)_{g}$

$\Leftrightarrow g=\frac{k}{g^{n}}$ f�r ein $n$ und $k\in\Gamma(X)$

$\Leftrightarrow f^{n}\in\mathfrak{A}$ f�r ein $n$.

d.h. zu zeigen: $f\in\text{rad}(\mathfrak{A})=IV(\mathfrak{A})$ (Hilbertsche
Nullstellensatz)

$\Leftrightarrow f(x)=0$ $\forall x\in V(\mathfrak{A})$

Ist dazu $x\in X$ mit $f(x)\neq0$, wo $x\in\mathcal{D}(f)$, so
existiert nach Voraussetzung $(*)$ $f_{1},f_{2}\in\Gamma(X)$, $f_{2}\notin\mathfrak{m}_{x}$
mit $g=\frac{f_{1}}{f_{2}}$

$\Rightarrow f_{2}\in\mathfrak{A}$. Da $f_{2}(x)\neq0$:

$\Rightarrow x\notin V(\mathfrak{A})$.
\end{proof}
\begin{rem}[orig. 34]
\mbox{}
\begin{enumerate}
\item Im allgemeinen existieren f�r $f\in\mathcal{O}_{x}(U)$ \textbf{nicht}
$g,h\in\Gamma(X)$ mit $f=\frac{g}{h}$ und $h(x)\neq0$ $\forall x\in U$.
\item \textbf{Alternative Definition von $\mathcal{O}_{X}$, I.}
\[
\mathcal{O}_{X}(\mathcal{D}(f)):=\Gamma(X)_{f},\quad\forall f\in\Gamma(X).
\]
Da $\mathcal{D}(f)$ Basis der Topologie ist, kann es h�chstens einen
Raum mit Funktionen geben mit dieser Eigenschaft, es bleibt die Existenz
zu zeigen.
\item \textbf{Alternative Definition von $\mathcal{O}_{X}$, II.}

Direkt von einer integeren endlich erzeugten $k$-Algebra $A$ ausgehend
(die $X$ bis auf Isomorphie festlegt), aber ohne ``Koordinaten''
zu w�hlen.
\begin{align*}
X & :=\{\mathfrak{m}\subseteq A\mid\text{max. Ideale}\}
\end{align*}
Die \textbf{abgeschlossen Mengen} sind gegeben durch:

\[
V(\mathfrak{A}):=\{\mathfrak{m}\subseteq A\text{ max.}\mid\mathfrak{m}\supseteq\mathfrak{A}\},\quad\mathfrak{A}\subset A\text{ Ideal}.
\]

$\mathcal{O}_{X}(U):=\bigcap_{\mathfrak{m}\in U}A_{\mathfrak{m}}\subset\text{Quot}(A)$
f�r $U\subset X$ offen (vgl. sp�ter Schemata).
\end{enumerate}
\end{rem}




\section{Funktorialit�t der Konstruktion}
\begin{prop}[orig. 35]
Sei $f:X\rightarrow Y$ eine stetige Abbildung zwischen irreduzibler
affiner algebraischen Mengen. Es sind �quivalent:
\begin{enumerate}
\item $f$ ist ein Morphismus affiner algebraischen Mengen.
\item $\forall g\in\Gamma(Y)$ gilt $g\circ f\in\Gamma(X)$.
\item $f$ ist ein von R�umen von Funktionen, d.h. f�r alle $U\subseteq Y$
offen und alle $g\in\mathcal{O}_{Y}(U)$ gilt $g\circ f\in\mathcal{O}_{X}(f^{-1}(U))$.
\end{enumerate}
\end{prop}
\begin{proof}
\mbox{}
\begin{itemize}
\item $(i)\Leftrightarrow(ii)$ 

Satz 29.
\item $(iii)\Rightarrow(ii)$ 

$U:=Y$ + Satz 33.
\item $(ii)\Rightarrow(iii)$

Betrachte $\varphi:\Gamma(Y)\rightarrow\Gamma(X)$, $h\mapsto h\circ f$.
Aufgrund des Verklebungsaxioms reicht es, die Bedingung f�r $U$ von
der Form $\mathcal{D}(g)$ zu zeigen: Es gilt:
\[
f^{-1}(\mathcal{D}(g))=\{x\in X\mid\underbrace{g(f(x))}_{=\varphi(g)(x)}\neq0\}=\mathcal{D}(\varphi(g))
\]
Deswegen induziert $\varphi$:
\begin{align*}
H & \longmapsto H\circ f\\
\mathcal{O}_{Y}(\mathcal{D}(g)) & \longrightarrow\mathcal{O}_{X}(D(\varphi(g)))\\
 & \shortparallel\\
\Gamma(Y)_{g} & \longrightarrow\Gamma(X)_{\varphi(g)}\\
\frac{h}{g} & \longmapsto\frac{h\circ f}{(g\circ f)^{n}}
\end{align*}
mit $h\circ f\in\Gamma(X)$ nach Voraussetzung und $\varphi(g)=g\circ f\in\Gamma(X)$
nach Voraussetzung.

\end{itemize}
Insgesamt haben wir:
\end{proof}
\begin{thm}[orig. 36]
Die obige Konstruktion definiert einen volltreuen Funktor
\[
\text{\{irred. aff. abg. Mengen �ber }k\}\rightarrow\{\text{R�ume mit Funktionen �ber }k\}
\]
\end{thm}



%\selectlanguage{english}%

\part*{Pr�variet�ten}

\textbf{Ziel.} Klasse der affin-algebraischen Mengen, aufgefasst
als R�ume mit Funktionen durch Verkleben vergr��ern.

$(X,\mathcal{O}_{X})$ hei�t \textbf{zusammenh�ngend}, falls $X$
als topologischer Raum zusammenh�ngend ist.

\section{Definition von Pr�variet�ten}
\label{sec:def-praevarietaet}
\begin{defn}[orig. 37]
\label{def:affine-varietaet}
Eine \textbf{affine Variet�t}\index{affine Variet�t} ist ein Raum
mit Funktionen, der isomorph zu dem Raum mit Funktionen assoziiert zu einer irreduziblen affin-algebraischen Menge ist.
\end{defn}
%
\begin{defn}[orig. 38]
\label{def:praevarietaet}
Eine \textbf{Pr�variet�t} ist ein zusammenh�ngender Raum mit Funktionen
$(X,\mathcal{O}_{X})$, f�r den eine \emph{endliche }�berdeckung $X=\bigcup_{i=1}^{n}U_{i}$ durch offene Teilmengen $U_i \subseteq X$
existiert, d.d. $\forall i=1,\ldots,n$ $(U_{i},\mathcal{O}_{X|_{U_{i}}})$
eine affine Variet�t ist.  Insbesondere sind affine Variet�ten Pr�variet�ten!

Ein \textbf{Morphismus von Pr�variet�ten} ist ein Morphismus der entsprechenden R�ume mit Funktionen.\selectlanguage{ngerman}%
\end{defn}



Ein Morphismus von Pr�variet�ten ist ein Morphismus der entsprechenden
R�ume mit Funktionen. Insbesondere sind also affine Variet�ten Pr�variet�ten!
Sp�ter sehen wir: Variet�t = ,,separierte Pr�variet�t``. Affine
Variet�ten sind stets ,,separiert``, daf�r braucht man wieder von
,,affinen Pr�variet�ten`` zu reden. Ist $X$ eine affine Variet�t,
schreiben wir oft $\Gamma(X)$ f�r $\mathcal{O}_{x}(X)$ (vgl. Satz
33).

Unter einer \textbf{offenen affinen �berdeckung} einer Pr�variet�t
$X$ verstehen wir eine Famile von affinen Unterr�umen mit Funktionen
$U_{i}\subseteq X$, $i\in I$ die affine Variet�ten sind, und so
das $X=\bigcup U_{i}$.

\section{Vergleich mit differenzierbaren/komplexen Mannigfaltigkeiten}

\paragraph{Differential/Komplexe Geometrie}

Mannigfaltigkeiten werden via Kartenabbildungen mit differenzierbaren/holomorphen
�bergangsabbildungen definiert (hier problematisch, da offene Teile
affiner algebraischer Mengen i.A. keine solche Struktur wider besitzen.)
Jedoch:
\begin{align*}
\text{\{diff. Mfgkt.\}} & \longrightarrow\text{\{R�ume mit Fkt.}/\mathbb{R}\}\\
X & \longmapsto(X,\mathcal{O}_{X})\\
 & \phantom{\longmapsto}\mathcal{O}_{X}(U):=C^{\infty}(U,\mathbb{R}),\ U\subseteq X\text{ offen}
\end{align*}

ist ein volltreuer Funktor. Daher kann man differenzierbare Mannigfaltigkeiten
auch als diejenigen R�ume mit Funktionen �ber $\mathbb{R}$ definieren,
f�r die $X$ Hausdorff ist, und so dass eine offene �berdeckung durch
solche R�ume mit Funktionen �ber $\mathbb{R}$ existiert, die in obiger
Weise offene Teilmengen von $\mathbb{R}^{n}$ zugeordnet sind. (Analog
bei komplexen Mannigfaltigkeiten.)



\section{Topologische Eigenschaften von Pr�variet�ten}
\begin{lem}
F�r einen topologischen Raum $X$ und $U\subseteq X$ offen haben
wir eine Bijektion
\begin{align*}
\{Y\subseteq U\text{ irred. abg.}\} & \longleftrightarrow\{Z\subseteq X\text{ irred. abg. mit }Z\cap U\neq\emptyset\}\\
Y & \longmapsto\overline{Y}\text{ (Abschluss in }X)\\
Z\cap U & \longmapsfrom Z
\end{align*}
\end{lem}
\begin{proof}
Lemma 14: $Y\subseteq X$ irreduzibel $\Leftrightarrow\overline{Y}\subseteq X$
irreduzibel.

$Y\subseteq U$ abgeschlossen $\Leftrightarrow\exists A\subset X$
abgecshlossen: $Y=U\cap A$.

$\Rightarrow Y\subseteq\overline{Y}\subseteq A$ $\Rightarrow Y=U\cap\overline{Y}$

$Y$ irreduzibel in $U$ $\Rightarrow Y$ irreduzibel in $X$

$\Rightarrow$ (14) $\overline{Y}$ irreduzibel

$\Rightarrow Y\mapsto\overline{Y}\mapsto\overline{Y}\cap U=Y$. $\checkmark$

$\emptyset\neq\underbrace{Z\cap U}_{\text{irred. (S. 13v.)}}\underset{\text{offen}}{\subset}Z$
damit dicht da $Z$ irreduzibel (Satz 13.ii)

$\Rightarrow$ Abbildung $\leftarrow$ wohldefiniert 

$\Rightarrow\overline{Z\cap U}=Z$ 
\end{proof}
\begin{prop}
Sei $(X,\mathcal{O}_{X})$ eine Pr�variet�t.

$\Rightarrow X$ noethersch (insbesondere quasikompakt) und irreduzibel.
\end{prop}
\begin{proof}
Sei $X=\bigcup_{i=1}^{n}$ endliche eff. aff. �berdeckung und $X\supseteq Z_{1}\supseteq Z_{2}\supseteq\cdots$
eine absteigende Kette abgeschlossener Teilmengen.

$\Rightarrow U_{i}\cap Z_{1}\supseteq U_{i}\cap Z_{2}\supseteq\cdots$

$\Rightarrow$ abgeschlossene Teilmengen in $U_{i}$

$\Rightarrow\forall i$ $\exists n_{i}$: $U_{i}\cap Z_{n_{i}}=U_{i}\cap Z_{i+m}$.
 Setzen von $n:=\max n_{i}$ liefert:

$\forall i=1,\ldots,n$ $\forall m\geq n$: $U_{i}\cap Z_{m}=U_{i}\cap Z_{m+1}$

$\Rightarrow(Z_{i})$ wird station�r da $Z_{m}=\bigcup U_{i}\cap Z_{m}$.

$\Rightarrow X$ noethersch.

Zeige, $X$ ist irreduzibel:

Sei $X=X_{1}\cup\cdots\cup X_{n}$ die Zerlegung in irreduzibele Komponenten.

$\mathbb{A}$ $n\geq2$

$\Rightarrow\exists i_{0}\in\{2,\ldots,n\}$: $X_{1}\cap X_{i_{0}}\neq\emptyset$.
(Andernfalls gilt: $X=X_{1}\sqcup\underbrace{X\backslash X_{1}}_{=X_{2}\cup\cdots\cup X_{n}\text{ abg.}}$.
Widerspruch zu $X$ zusammenh�ngend.)

Sei ohne Einschr�nkung $i_{0}=2$. Sei $x\in X_{1}\cap X_{2}$, $x\in U\subset X$
offene affine �berdeckung (d.h. affine Variet�t).

$U$ irreduzibel $\Rightarrow\overline{U}$ (Abschluss in $X$) $\subseteq X_{j}$
f�r ein $j\in\{1,\ldots,n\}$

\textbf{Jedoch}: $x\in X_{i}\cap U\subseteq U$ irreduzibel ist $\underbrace{\overline{X_{i}\cap U}}_{\subset\overline{U}\subset X_{i}}=X_{i}$,
$i=1,2$

$\Rightarrow X_{1},X_{2}\subseteq X_{j}$. Widerspruch zu maximale
Komponente.
\end{proof}




\section{Offene Untervariet�ten}
\label{sec:offene-untervarietaeten}

Offene Teilmengen von affinen Variet�ten (und allgemeiner beliebigen
 Pr�variet�ten) sind wieder Pr�variet�ten. (aber i.A. nicht affin!)
\begin{lem}[orig. 41]
\label{lem:basisoffene-teilmengen-sind-affin}
Sei $X$ eine affine Variet�t, $f\in\mathcal{O}_{X}(X)$, $D(f)\subseteq X$. Die Lokalisierung
von $\Gamma(X)=\mathcal{O}_{X}(X)$ an $f$,
\[
\Gamma(X)_{f}=\Gamma(X)[T]/(Tf-1)
\]

ist eine integre, endlich erzeugte $k$-Algebra. $(Y,\mathcal{O}_{Y})$
bezeichne die zugeh�rige affine Variet�t. Dann gilt:
\[
(D(f),\mathcal{O}_{X}|_{D(f)})\cong(Y,\mathcal{O}_{Y})
\]

als R�ume mit Funktionen, d.h. $(D(f),\mathcal{O}_{X|_{D(f)}})$ ist
selbst affine Variet�t.
\end{lem}
\begin{proof}
$\mathcal{O}_{X}(D(f))=\mathcal{O}_{X}(X)_{f}$ muss affiner
Koordinatenring von $(\mathcal{D}(f),\mathcal{O}_{X|_{\mathcal{D}(f)}})$
sein, wenn letzterer Raum von Funktionen affin ist. $X\subseteq\mathbb{A}^{n}(k)$
korrespondiert zu dem Radikalideal:
\begin{align*}
\mathfrak{a} & :=I(X)\unlhd k[T_1, \ldots, T_n]\ \subseteq\ \mathfrak{a}':=(\mathfrak{a},fT_{n+1}-1)\subseteq k[T_{1},\ldots,T_{n+1}]
\end{align*}

mit Koordinatenringen:
\begin{align*}
\Gamma(X) & =k[T_{1},\ldots,T_{n}]/\mathfrak{a}\\
\Gamma(Y) & =\Gamma(X)_{f}=(k[T_{1},\ldots,T_{n}]/\mathfrak{a})[T_{n+1}]/(T_{n+1}f-1)\\
 & \cong k[T_{1},\ldots,T_{n+1}]/\mathfrak{a}'
\end{align*}

F�r $Y=V(\mathfrak{a}')\subseteq\mathbb{A}^{n+1}(k)$ induziert die
Abbildung
\[
\xymatrix{Y\subseteq\mathbb{A}^{n+1}(k)\ar@{-->}[d] & (x_{1},\ldots,x_{n+1})\ar@{|->}[d] & T_{i}\\
X\subseteq\mathbb{A}^{n}(k) & (x_{1},\ldots,x_{n}) & T_{i}\ar@{|->}[u]
}
\]

eine Bijektion $Y\xrightarrow{j} D_{X}(f)$  mit Umkehrabbildung
$(x_{0},\ldots,x_{n},\frac{1}{f(x_{0},\ldots,x_{n})})\mapsfrom(x_{0},\ldots,x_{n})$
\begin{claim*}
$j$ ist Isomorphismus von R�umen mit Funktionen:
\begin{enumerate}
\item $j$ ist \emph{stetig} (als Einschr�nkung einer stetigen Abbildung) $\checkmark$
\item $j$ ist \emph{offen}: F�r $\frac{g}{f^{n}}\in\Gamma(X)_{f} = \Gamma(Y)$ mit $g\in\Gamma(X)$ gilt
\begin{align*}
j\left(D_{Y}\left(\frac{g}{f^{n}}\right)\right) & =j\left(D_{Y}(gf)\right) & f\text{ Einheit}\\
 & =D_{X}(gf)\text{ offen}
\end{align*}

$\Rightarrow j$ Hom�morphismus.
\item $j$ induziert $\forall g\in\Gamma(X)$ Isomorphismen:
\begin{align*}
\mathcal{O}_{X}(D(fg)) & \longrightarrow\Gamma(Y)_{g}\\
s & \longmapsto s\circ j
\end{align*}
mit $\mathcal{O}_{X}(D(fg))=\Gamma(X)_{fg}=\Gamma(X)_{f})_{g}=\Gamma(Y)_{g}$.
Mit dem Verklebungsaxiom folgt: $j$ ist Morphismus von R�umen mit Funktionen.
\end{enumerate}
\end{claim*}
\end{proof}
\begin{prop}[orig. 42]
\label{prop:offener-teilraum-praevarietaet}
Sei $(X,\mathcal{O}_{X})$ Pr�variet�t, $\emptyset\neq U\subseteq X$
offen. Dann ist $(U,\mathcal{O}_{X}|_{U})$ eine Pr�variet�t und $U\hookrightarrow X$
ist Morphismus von Pr�variet�ten.
\end{prop}
\begin{proof}
$X$ ist irreduzibel, also folgt mit Satz \ref{prop:charakterisierung-irreduzibel}, dass $U$ zusammenh�ngend
ist. Nach Voraussetzung besitzt $X=\bigcup_{i} X_{i}$ eine affine, offene
�berdeckung. Es folgt:
\[
U=\bigcup_{i}(\underbrace{X_{i}\cap U}_{\text{offen in }X_{i}})=\bigcup_{i,j} D_{X_{i}}(f_{i,j})
\]

und $D_{X_{i}}(f_{i,j})$ ist eine affine Variet�t nach
Lemma \ref{lem:basisoffene-teilmengen-sind-affin}. Da $X$ noethersch
ist, folgt mit Lemma \ref{lem:eigenschaften-noethersch}, dass $U$ quasikompakt ist.

$\Rightarrow$ Es existiert eine endliche Teil�berdeckung, also ist $U$ Pr�variet�t. $\checkmark$

Die kanonische Inklusion $U \xhookrightarrow{i} X$ ist sicher stetig. F�r $f\in\mathcal{O}_{X}(V), V \subseteq X$ offen gilt mit dem Einschr�nkungsaxiom
\[
\mathcal{O}_{X}|_{U}(U\cap V)=\mathcal{O}_{X}(U\cap V)\ni f\circ i=f|_{U\cap V}
\]

Also ist $i$ Morphismus von Pr�variet�ten.
\end{proof}
Die offenen affinen Teilmengen einer Pr�variet�t $X$ ($\hat{=} U\subseteq X$
offen mit $(U,\mathcal{O}_{X}|_{U})$ affine Variet�t) bilden eine
Basis der Topologie von $X$, da $X$ durch offene affine Untervariet�ten
�berdeckt wird und letzere diese Eigenschaft nach Lemma \ref{lem:basisoffene-teilmengen-sind-affin} haben.



\section{Funktionenk�rper einer Pr�variet�t}
\label{sec:funktionenkorper-praevarietaet}
\begin{defn}[orig. 43]
\label{def:funktionenkoerper-praevarietaet}
F�r eine Pr�variet�t $X$ sind die rationalen Funktionenk�rper aller
nicht-leeren affin-offenen Teilmengen in nat�rlicher Weise zu einander
isomorph. Diesen K�rper $K(X)$ nennen wir den \textbf{rationalen Funktionenk�rper}von $X$.
\end{defn}
\begin{proof}
$\emptyset\neq U$, $V\subseteq X$ affine, offene Untervariet�ten. Da
$X$ irreduzibel ist, gilt nach \emph{Satz \ref{prop:charakterisierung-irreduzibel}}:
\[
\emptyset\neq U\cap V\subseteq U\text{ offen}.
\]

Nach Definition von $\mathcal{O}_{X}$ ist 
\[
\mathcal{O}_{X}(U)\subseteq\mathcal{O}_{X}(U\cap V)\subseteq K(U)=\text{Quot}(\mathcal{O}_{X}(U)).
\]

Das impliziert $\text{Quot}(\mathcal{O}_{X}(U\cap V))=K(U)$. Aus
Symmetriegr�nden ist aber damit auch bereits $K(V)=\text{Quot}(\mathcal{O}_{X}(U\cap V))$.
\end{proof}
\begin{rem}[orig. 44]
\label{rem:funktionenkoerper-nicht-funktoriell}
Bildung des des Funktionenk�rpers $K(\cdot)$ ist \textbf{nicht} funktoriell!
F�r $X\rightarrow Y$ Morphismus affiner Variet�ten ist die Abbildung
auf den Koordinatenringen $\Gamma(Y)\rightarrow\Gamma(X)$ i.A. \textbf{nicht}
injektiv, induziert also keine Abbildung $K(Y)\hookrightarrow K(X)$.

\emph{Jedoch}: Eine Isomorphie $X\xrightarrow{\sim}Y$ induziert $K(Y)\xrightarrow{\sim}K(X)$.
Allgemeiner sei $X\xrightarrow{\varphi} Y$ Morphismus mit $\text{im}(\varphi) \subseteq Y$
offen ($\Rightarrow$ dicht. Sp�ter: $X\xrightarrow{\varphi} Y$ \textbf{dominant},
gdw. $\text{im}(\varphi)\subseteq Y$ dicht) induziert in funktioreller Weise eine
Abbildung $K(Y)\hookrightarrow K(X)$.
\end{rem}
\begin{prop}[orig. 45]
\label{prop:charakterisierung-schnitte-praevarietaet}
Sei $X$ eine Pr�variet�t, $V\subseteq U\subseteq X$ offen. Dann gilt:

\begin{enumerate}
\item $\mathcal{O}_{X}(U)\subseteq K(X)$ ist $k$-Unteralgebra.

\item $\mathcal{O}_{X}(U)\rightarrow\mathcal{O}_{X}(V)$ ist Inklusion von Teilmengen des Funktionenk�rpers $K(X)$.

\item Insbesondere gilt f�r $U,V\subseteq X$ offen:
\[
\mathcal{O}_{X}(U\cup V)=\mathcal{O}_{X}(U)\cap\mathcal{O}_{X}(V).
\]
\end{enumerate}
\end{prop}
\begin{proof}
\mbox{}
\item[(ii)] Sei $\mathcal{O}_{X}(X)\ni f:X\rightarrow k$. Dann ist $f^{-1}(0)\subseteq X$
abgeschlossen, da f�r $W\subseteq X$ affin-offen beliebig gilt, dass
\[
f^{-1}(0)\cap W=V(f|_{W}).
\]
Dazu macht man sich klar: ,,abgeschlossen`` ist eine lokale Eigenschaft,
affin-offene $W$ bilden eine Basis der Topologie. 

$\Rightarrow\mathcal{O}_{X}(U)\hookrightarrow\mathcal{O}_{X}(V)$, $f\mapsto f|_{V}$
ist injektiv f�r $\emptyset\neq V\subseteq U\subseteq X$ offen.

$\Rightarrow V\subseteq f^{-1}(0)$ 

$\Rightarrow f^{-1}(0)=U$ 

$\Rightarrow f\equiv0$.
\item[(i)] $U\supseteq W$ affin-offene Untervariet�t.
\[
\begin{tikzcd}
 \mathcal{O}_{X}(W) \arrow[r, hook] & K(W) \text{ } k\text{-Algebren} \\
 \mathcal{O}_{X}(U) \arrow[u, hook] \arrow[ru, dashed, hook]
\end{tikzcd}
\]
\item[(iii)] Wir haben folgendes kommutatives Diagramm:

\[
\begin{tikzcd}
 {} & \mathcal{O}_{X}(U) \arrow[rd, hook] \\
 \mathcal{O}_{X}(U \cup V) \arrow[ur, hook] \arrow[rd, hook'] & {} & \mathcal{O}_{X}(U \cap V) \\
 {} & \mathcal{O}_{X}(V) \arrow[ru, hook']
\end{tikzcd}
\]
Nach dem Verklebungsaxiom ist die Sequenz
\[
\begin{tikzcd}
  0 \arrow[r] & \mathcal{O}_{X}(U\cup V) \arrow[r] & \mathcal{O}_{X}(U) \times \mathcal{O}_{X}(V) \arrow[r] & \mathcal{O}_{X}(U \cap V) \\
  {} & f \arrow[r, mapsto] & (f|_{U}, f|_{V}) & {} \\
  {} & {} & (g,h) \arrow[r, mapsto] & g|_{U \cap V} - h|_{U \cap V}
\end{tikzcd}
\]
exakt.


\end{proof}




\section{Abgeschlossene Unterpr�variet�ten}
\label{sec:abg-untervarietaeten}

Sei $X$ eine Pr�variet�t, $Z\subseteq X$ abgeschlossen und irreduzibel.

\textbf{Ziel.} $(Z,\mathcal{O}_{Z}')$ Raum von Funktionen erkl�ren.
Definiere dazu f�r $U \subseteq Z$ offen:

\[
\mathcal{O}_{Z}'(U):=\{f\in\text{Abb}(U,k)\mid\forall x\in U\ \exists x\in V\subseteq X\text{ offen},\ g\in\mathcal{O}_{X}(V) \text{ mit } f|_{U\cap V}=g|_{U\cap V}\}
\]

Damit ist $(Z,\mathcal{O}'_{Z})$ Raum von Funktionen (klar!) mit $\mathcal{O}'_{X}=\mathcal{O}_{X}$.
\begin{lem}[orig. 46]
\label{lem:abg-untervarietaeten-affine-varietaeten}
Seien $X\subseteq\mathbb{A}^{n}(k)$ eine irreduzible, affin-algebraische Menge
und $Z\subseteq X$ ein irreduzibler abgeschlossener Teilraum. Dann
ist $(Z,\mathcal{O}_{Z})=(Z,\mathcal{O}'_{Z})$.

Bezeichne ab jetzt stets $\mathcal{O}_{Z}$ f�r $\mathcal{O}_{Z'}$.
\end{lem}
\begin{proof}
$Z\subseteq X$ ist in beiden F�llen mit der Teilraumtopologie ausgestattet!
Ferner wissen wir, dass der Morphismus $Z\hookrightarrow X$ affin-algebraischer Mengen einen Morphismus $(Z,\mathcal{O}_{Z})\rightarrow(X,\mathcal{O}_{X})$
von Pr�variet�ten induziert. Nach Definition von $\mathcal{O}'$
folgt dann:
\[
\mathcal{O}'_{Z}(U)\subseteq\mathcal{O}_{Z}(U)\quad\text{f�r }U\subseteq Z\ \text{offen, denn:}
\] 
Ist $f \in \mathcal{O}_{Z}'(U)$ und $x \in U$ so existieren nach Definition eine offene Umgebung $x \in V_{x} \subseteq X$ und ein $g \in \mathcal{O}_{X}(V_{x})$ d.d. $f|_{U \cap V_{x}} = g|_{U \cap V_{x}}$. Damit gilt $g|_{Z \cap V_x} \in \mathcal{O}_{Z}(Z \cap V_{x})$. Mit dem Verklebungsaxiom erhalten wir also $f \in \mathcal{O}_{Z}(U)$.


Sei $f\in\mathcal{O}_{Z}(U)$ und $x\in U$ beliebig. Es folgt: $\exists h\in\Gamma(Z)$
mit $x\in D(h)\subseteq U$ und
\[
f|_{D(h)}=\frac{g}{h^{n}}\in\Gamma(Z)_{h}=\mathcal{O}_{Z}(D(h))
\]

f�r $n\geq0$ und $g\in\Gamma(Z)$ geeignet. Lifte $g,h\in\Gamma(Z)\twoheadleftarrow\Gamma(X)$
zu $\overline{g},\overline{h}\in\Gamma(X)$ und setze $V:=D(\overline{h})\subseteq X$.

$\Rightarrow x\in V$, $\frac{\overline{g}}{\overline{h}^{n}}\in\mathcal{O}_{X}(D(\overline{h}))$
und $f|_{U\cap V}=\frac{\overline{g}}{\overline{h}^{n}}|_{U\cap V}$.

$\Rightarrow f\in\mathcal{O}'_{Z}(U)$.
\end{proof}
\begin{cor}[orig. 47]
\label{cor:abg-untervarietaeten-sind-praevarietaeten}
Wenn $X$ eine Pr�variet�t ist, und $Z\subseteq X$ irreduzibel und abgeschlossen, dann ist $(Z,\mathcal{O}_{Z})$ ebenfalls eine Pr�variet�t.
\end{cor}
\begin{proof}
Es ist $X=\bigcup_{i}X_{i}$ f�r eine endliche affin-offene �berdeckung $(X_{i})_{i}$.
Damit ist 
\[
Z=\bigcup_{i}\left(Z\cap X_{i}\right) :=\bigcup_{i} Z_{i}
\]
 

mit $(Z_{i}, \mathcal{O}_{Z_{i}})$ affine Variet�t nach Lemma $\ref{lem:abg-untervarietaeten-affine-varietaeten}$.
\end{proof}




\section*{Beispiele (Projektiver Raum und projektive Variet�ten)}

\section{Homogene Polynome}
\begin{defn}[orig. 48]
Ein Polynom $f\in k[X_{0},\ldots,X_{n}]$ hei�t \textbf{homogen vom
Grad}\index{homogen} $d\in\mathbb{Z}_{\geq0}$, wenn $f$ die Summe
von Monomen von Grad $d$ ist. (Insbesondere ist f�r jedes $d$ das
Nullpolynom homogen von Grad $d$.)

\emph{Bezeichne} $k[X_{0},\ldots,X_{n}]_{d}$ der Untervektorraum
der Polynome vom Grad $d$.
\end{defn}
\begin{rem}[orig. 49]
Da \#$k$ unendlich ist, ist $f$ homogen vom Grad $d$. 

$\Leftrightarrow f(\lambda x_{0},\ldots,\lambda x_{n})=\lambda^{d}f(x_{0},\ldots,x_{n})$
$\forall x_{0},\ldots,x_{n}\in k$, $\lambda\in k^{\times}$. 

Es gilt: $k[X_{0},\ldots X_{n}]=\bigoplus_{d\geq0}k[X_{0},\ldots,X_{n}]_{d}$.
\end{rem}
\begin{lem}[orig. 50]
F�r $i\in\{0,\ldots,n\}$ und $d\geq0$ haben wir bijektive $k$-lineare
Abbildungen
\begin{align*}
k[X_{0},\ldots,X_{n}]_{d} & \longrightarrow\text{Polynome in }k[T_{0},\ldots,\hat{T}_{i},\ldots,T_{n}]\text{ v. Grad }\leq d\\
f & \overset{\Phi_{i}^{d}}{\longmapsto}f(T_{0},\ldots,\underbrace{1}_{i},\ldots,T_{n})\\
X_{i}^{d}g\left(\frac{X_{0}}{X_{i}},\ldots,\frac{\hat{X_{i}}}{X_{i}},\ldots,\frac{X_{n}}{X_{i}}\right) & \overset{\Psi_{i}^{d}}{\longmapsfrom}g
\end{align*}

\textbf{Dehomogenisierung }bzw. \textbf{Homogenisierung.}
\end{lem}
\begin{proof}
Es reicht, $\Psi_{i}^{d}\circ\Phi_{i}^{d}=\text{id}$, $\Phi_{i}^{d}\circ\Psi_{i}^{d}=\text{id}$
auf Monomen nachzurechnen, da alle Abbildungen $k$-linear sind. 
\end{proof}
Oft ist es n�tzlich, 
\[
k[T_{0},\ldots,\hat{T_{i}},\ldots,T_{n}]\text{ mit }
\]
 mit $k\left[\frac{X_{0}}{X_{i}},\ldots,\frac{\hat{X_{i}}}{X_{i}},\ldots,\frac{X_{n}}{X_{i}}\right]\underset{\text{Unterring}}{\subset}k(X_{0},\ldots,X_{n})$.




\section{Definition des projektiven Raumes}

Sei $X_{1}=X_{2}=\mathbb{A}^{1}$, $\tilde{U}_{1}\subseteq X_{1}=\tilde{U}_{2}\subseteq X_{2}=\mathbb{A}\backslash\{0\}$.
\begin{align*}
\tilde{U}_{1} & \overset{\sim}{\longrightarrow}\tilde{U}_{2}\\
x & \longmapsto\frac{1}{x}
\end{align*}

Verkleben von $X_{1}$ und $X_{2}$ entlang $\tilde{U}_{1}$ und $\tilde{U}_{2}$

\[
\mathbb{P}^{1}=\mathbb{A}^{1}\cup\{\infty\}=U_{1}\cup U_{2}.
\]

Allgemein: 
\[
\mathbb{P}^{n}=\bigcup_{i=1}^{n+1}U_{i}=\mathbb{A}^{n}\cup\mathbb{P}^{n-1}=\mathbb{A}^{n}\sqcup\mathbb{A}^{n-1}\sqcup\cdots\sqcup\mathbb{A}^{1}\sqcup\mathbb{A}^{0}
\]

\textbf{Idee}: $\mathbb{P}^{2}\supseteq\mathbb{A}^{2}$: Zwei verschiedene
Geraden in $\mathbb{P}^{2}$ schneiden sich genau in einem Punkt.\textbf{
Als Menge}:
\begin{align*}
\mathbb{P}^{n}(k): & =\{\text{Ursprungsgeraden in }k^{n+1}\}=\{1\text{-dim. }k\text{-UVR}\}\\
 & =(k^{n+1}\backslash\{0\})/k^{\times}
\end{align*}

Repr�sentanten dieser Klasse entsprechen: 
\[
\langle(x_{0},\ldots x_{n})\rangle_{k\text{-linear}}\longmapsfrom(x_{0}:\ldots:x_{n})
\]

\emph{�quivalenzrelation}: 
\[
(x_{0},\ldots,x_{n})\sim(x_{0}',\ldots,x_{n}')\Leftrightarrow\exists\lambda\in k^{\times}\ \text{mit}\ x_{i}=\lambda x_{i}'\ \forall i.
\]
 

\textbf{Bezeichne} Klassen $(x_{0}:\ldots:x_{n})$, $x_{i}$ \textbf{homogene}
Koordinaten auf $\mathbb{P}^{n}$
\[
U_{i}:=\{(x_{0}:\cdots:x_{n})\in\mathbb{P}^{n}\mid x_{i}\neq0\}\subseteq\mathbb{P}^{n}(k),\ 0\leq i\leq n
\]

ist wohldefiniert $\Leftrightarrow x_{i}=1$.
\[
\mathbb{P}^{n}(k)=\bigcup_{i=0}^{n}U_{i}
\]

Einen Isomorphismus 
\begin{align*}
U_{i} & \stackrel[\chi_{i}]{\cong}{\longrightarrow}\mathbb{A}^{n}(k)\\
(x_{0}:\ldots:x_{n}) & \longmapsto\left(\frac{x_{0}}{x_{i}},\ldots,\frac{\hat{x}_{i}}{x_{i}},\ldots,\frac{x_{n}}{x_{i}}\right)\\
(t_{0}:\cdots t_{i-1}:1:t_{i+1}:\cdots t_{n}) & \longmapsfrom(t_{0},\ldots,\hat{t}_{i},\ldots,t_{n})
\end{align*}

$U\subseteq\mathbb{P}^{n}$ ist genau dann offen, wenn $\kappa_{i}(U\cap U_{i})\subseteq\mathbb{A}^{n}$
offen ist. Beachte: der Durchschnitt 
\[
U_{i}\cap U_{j}=\mathcal{D}(T_{j})\subseteq U_{i}\text{ offen},\ i\neq j
\]

wenn auf $U_{i}\cong\mathbb{A}^{n}$ die Koordinaten $T_{0},\ldots,\hat{T}_{i},\ldots,T_{n}$
verwendet werden. Damit wird $\mathbb{P}^{n}(k)$ zu einem topologischen
Raum, der durch die $U_{i}$, $0\leq i\leq n$, offen �berdeckt wird.

\subsection{Regul�re Funktionen}

Sei $U\subseteq\mathbb{P}^{n}(k)$ eine beliebige offene Teilmenge.
Die regular�ren Funktionen auf $U$ sind
\[
\mathcal{O}_{\mathbb{P}^{n}}(U)=\{f\in\text{Abb}(U,k)\mid f|_{U\cap U_{i}}\in\mathcal{O}_{U_{i}}(U\cap U_{i})\}\qquad\forall i\in\{0,\ldots,n\}
\]

Dabei ist implizit verstanden, dass wir via $\kappa_{i}$ die $U_{i}$
als Raum mit Funktionen auffassen. Dabei erhalten wir insgesamt:
\[
\mathbb{P}^{n}(k)=(\mathbb{P}^{n}(k),\mathcal{O}_{\mathbb{P}^{n}})
\]

als Raum mit Funktionen.
\begin{prop}[orig 51]
F�r $U\subseteq\mathbb{P}^{n}$ offen gilt: $\mathcal{O}_{\mathbb{P}^{n}}(U)=\{f:U\rightarrow k\mid\forall x\in U$:
existiert $x\in V\subseteq U$ offen und $g,h\in k[X_{0},\ldots,X_{n}]$
homogen vom selben Grad, d.d. $\forall v\in V$: $h(v)\neq0$ und
$f(v)=\frac{g(v)}{h(v)}$.\} 
\end{prop}
Wohldefiniertheit: Sei $V=(x_{0}:\ldots:x_{n})$.
\[
f(\lambda x_{0},\ldots,\lambda x_{n})=\frac{g(\lambda x_{0},\ldots,\lambda x_{n})}{h(\lambda x_{0},\ldots,\lambda x_{n})}=\frac{\lambda^{d}g(x_{0},\ldots,x_{n})}{\lambda^{d}h(x_{0},\ldots,x_{n})}=f(x_{0},\ldots,x_{n})
\]

\begin{proof}
\mbox{}
\begin{itemize}
\item[,,$\subseteq$``] Sei $f\in\mathcal{O}_{\mathbb{P}^{n}}(U)$. Dann ist $f|_{U\cap U_{i}}\in\mathcal{O}_{U_{i}}(U\cap U_{i})$.
Es folgt:
\[
f=\frac{\tilde{g}}{\tilde{h}},\ \tilde{g},\tilde{h}\in k[T_{0},\ldots,\hat{T}_{i},\ldots,T_{n}]
\]
Definiere $d:=\max\{\deg(\tilde{g}),\deg(\tilde{h})\}$. Homogenisiere:
\[
g:=\psi_{i}^{d}(\tilde{g}),\ h:=\psi_{i}^{d}(\tilde{h})
\]
$\Rightarrow f=\frac{g}{h}$ lokal. 
\begin{align*}
f(x) & =\frac{\tilde{g}}{\tilde{h}}(\chi_{i}(x))\\
f((x_{0}:\cdots:x_{n})) & =\frac{\tilde{g}\left(\frac{x_{0}}{x_{i}},\ldots,\frac{\hat{x_{i}}}{x_{i}},\ldots,\frac{x_{n}}{x_{i}}\right)}{\tilde{h}\left(\frac{x_{0}}{x_{i}},\ldots,\frac{\hat{x_{i}}}{x_{i}},\ldots,\frac{x_{n}}{x_{i}}\right)}\\
 & =\frac{x_{i}^{d}\tilde{g}()}{x_{i}^{d}\tilde{h}()}\\
 & =\frac{\psi_{i}^{d}(\tilde{g})()}{\psi_{i}^{d}(\tilde{h})()}=\frac{g}{h}((x_{0}:\cdots:x_{n}))
\end{align*}
\item[,,$\supseteq$``] Sei $f$ in der rechten Menge: fixiere $i\in\{0,\ldots,n\}$ lokal
auf $U\cap U_{i}$ mit $f$ nach Voraussetzung in der Form $\frac{g}{h}$,
$g,h\in k[X_{0},\ldots,X_{n}]_{d}$, $d$ geignet. Definiere:
\[
\tilde{g}_{i}:=\frac{g}{X_{i}^{d}},\ \tilde{h}:=\frac{h}{X_{i}^{d}}\in k\left[\frac{X_{0}}{X_{i}},\ldots\frac{\hat{X_{i}}}{X_{i}},\ldots,\frac{X_{n}}{X_{i}}\right]
\]
$\Rightarrow f$ ist lokal der Form: $\frac{\tilde{g}}{\tilde{h}}$,
$\tilde{g},$$\tilde{h}\in k[T_{0},\ldots,\hat{T_{i}},\ldots T_{n}]$.

$\Rightarrow f|_{U\cap U_{i}}\in\mathcal{O}_{U_{i}}(U\cap U_{i})$.

\end{itemize}
\end{proof}
\begin{cor}[orig. 52]
F�r $i\in\{0,\ldots,n\}$ induziert
\[
U:\xrightarrow[\cong]{\chi_{i}}\mathbb{A}^{n}(k)
\]

einen Isomorphismus
\[
(U_{i},\mathcal{O}_{\mathbb{P}^{n}|_{U_{i}}})\xrightarrow{\cong}\mathbb{A}^{n}(k)
\]

von R�umen mit Funktionen. Insbesondere ist $\mathbb{P}^{n}(k)$ eine
Pr�variet�t.
\end{cor}
\begin{proof}
Zu zeigen: $\forall U\subset U_{i}$ offen gilt:
\[
\mathcal{O}_{\mathbb{P}^{n}(k)}(U)=\mathcal{O}_{U_{i}}(U)=\{f:U\rightarrow k\mid f\in\mathcal{O}_{U_{i}}(U)\}
\]

d.h. auf der rechten Seite muss die Bedingung nur f�r das fixierte
$i$ �berpr�ft werden. Dies folgt aus den Beweis des Satzes.
\end{proof}
Damit identifizieren die Funktionenk�rper 
\[
K(\mathbb{P}^{n}(k))=K(U_{i})=k\left(\frac{X_{0}}{X_{i}},\ldots,\frac{X_{n}}{X_{i}}\right)
\]

\begin{prop}[orig. 53]
$\mathcal{O}_{\mathbb{P}^{n}(k)}(\mathbb{P}^{n}(k))=k$. Insbesondere
ist $\mathbb{P}^{n}$ f�r $n\geq1$ \textbf{keine} affine Variet�t.
(Da der $k$-Algebra $k$ ja $\mathbb{A}^{0}(k)=\{\text{pt}\}$ als
affine Variet�t entspricht.)
\end{prop}
\begin{proof}
$k\subseteq\mathcal{O}_{\mathbb{P}^{n}(k)}(\mathbb{P}^{n}(k))$ konstante
Funktionen klar. Nach Satz 45 (iii) gilt:
\begin{align*}
\mathcal{O}_{\mathbb{P}^{n}}(\mathbb{P}^{n}) & =\bigcap_{i=0}^{n}\mathcal{O}_{\mathbb{P}^{n}}(U_{i})\subset K(\mathbb{P}^{n}(k))\\
 & =\bigcap_{i=0}^{n}k[t_{0},\ldots,\hat{t_{i}},\ldots,t_{n}]=k
\end{align*}
\end{proof}




\section{Projektive Variet�ten}
\label{sec:projektive-varietaeten}
\begin{defn}[orig. 54]
  \label{def:projektive-varietaeten}
  Abgeschlossene Unterpr�variet�ten eines projektiven Raumes $\mathbb{P}^{n}(k)$
  hei�en \textbf{projektive Variet�ten}.
\end{defn}
Vorsicht: f�r $x=(x_{0}:\ldots:x_{n})\in\mathbb{P}^{n}$, $f\in k[X_{0},\ldots,X_{n}]$
ist $f(x_{1},\ldots,x_{n})$ \emph{nicht} wohldefiniert, da von Repr�sentaten
abh�ngig, d.h. $f$ kann \emph{nicht}\textbf{ }als Funktion auf $\mathbb{P}^{n}$
aufgefasst werden. F�r \emph{homogene}\textbf{ }Polynome $f_{1},\ldots,f_{n}\in k[X_{0},\ldots X_{n}]$
(nicht notwendig vom selben Grad) k�nnen wir demnoch Verschwindungsmengen
definieren:
\[
  V_{+}(f_{1},\ldots,f_{n})=\{(x_{0}:\ldots:x_{n})\in\mathbb{P}^{n}\mid f_{j}(x_{0},\ldots,x_{n})=0\ \forall j\}
\]

Da $V_{+}(f_{1},\ldots,f_{n})\cap U_{i}=V(\Phi_{i}(f_{1}),\ldots,\Phi_{i}(f_{m}))$
ist $V_{+}(f_{1},\ldots,f_{m})$ abgeschlossen in $\mathbb{P}^{n}$.
Ist $V_{+}(f_{1},\ldots,f_{n})$ irreduzibel, so erhalten wir eine
projektive Variet�t. In der Tat entstehen alle projektiven Variet�ten
auf diese Weise, wie der folgende Satz zeigt:
\begin{prop}[orig. 55]
  \label{prop:charakterisierung-projektive-varietaeten}
  Sei $Z\subseteq\mathbb{P}^{n}(k)$ eine projektive Variet�t. Dann
  existieren homogene Polynome $f_{1},\ldots,f_{n}\in k[X_{0},\ldots,X_{n}]$,
  so dass
  \[
    Z=V_{+}(f_{1},\ldots,f_{n})
  \]

  gilt.
\end{prop}
\begin{proof}
  Betrachte: 
  \[
    \begin{array}{cc}
      \\
      \\
    \end{array}
  \]

  $f|_{f^{-1}(U_{i})}:f^{-1}(U_{i})\longrightarrow U_{i}$ ist Morphismus
  von Pr�variet�ten. Dann ist $f$ selber ein Morphismus von Pr�variet�ten.
  \begin{align*}
    \overline{Y}:= & Y\cup\{0\}\text{ Abschluss von }Y\text{ in }\mathbb{A}^{n+1}(k)\\
    \mathfrak{A}:= & I(\overline{Y})\subseteq k[X_{0},\ldots,X_{n}]
  \end{align*}

  Behauptung: $\mathfrak{A}$ wird von homogenen Polynomen erzeugt\emph{.
    Denn:} f�r $g\in\mathfrak{A}$, $g=\sum_{d}g_{d}$ Zerlegung in homogene
  Bestandteile vom Grad $d$. $\overline{Y}$ ist Vereinigung von Ursprungsgeraden
  im $k^{n+1}$, d.h. $\forall\lambda\in k^{\times}$ gilt:
  \[
    g(x_{0},\ldots,x_{n})=0\ \Leftrightarrow\ g(\lambda x_{0},\ldots,\lambda x_{n})=0
  \]

  Beweis durch Widerspruch. Nicht alle $g_{d}$ liegen in $\mathfrak{A}$.

  $\Rightarrow\exists(x_{0},\ldots,x_{n})\in\mathbb{A}^{n+1}(k)$, so
  dass $g(x_{0},\ldots,x_{n})=0$, aber $g_{d_{0}}(x_{0},\ldots,x_{n})\neq0$.

  $\Rightarrow0\,\not\equiv,\sum_{d}g_{d}(x_{0},\ldots,x_{n})T^{d}\in k[T]$

  $\Rightarrow(\exists\lambda\in k^{\times})$ $0\neq\sum_{d}g_{d}(x_{0},\ldots,x_{n})\lambda^{d}=\sum_{d}g_{d}(\lambda x_{0},\ldots,\lambda x_{n})=g(\lambda x_{0},\ldots,\lambda x_{n})=0$.
  Widerspruch.

  $\Rightarrow\mathfrak{A}=(f_{1},\ldots,f_{m})$, $f_{j}$ homogen.

  $\Rightarrow Z=V_{+}(f_{1},\ldots,f_{m})$. 

  \begin{align*}
    Z\ni(x_{0}:\cdots:x_{n}) & \Leftrightarrow(\lambda x_{0},\ldots,\lambda x_{n})\in\overline{Y}\ \forall\lambda\in k^{\times}\text{ und }\neq0\\
                             & \Leftrightarrow f_{i}(x_{0},\ldots,x_{n})=0\ \forall1\leq i\leq n,\ (x_{0},\ldots,x_{n})\in\mathbb{P}^{n}
  \end{align*}

  \rule[0.5ex]{1\columnwidth}{1pt}
\end{proof}
Zu Bemerkung 49 

Nach Satz 51 und Definition von $\mathcal{O}_{Z}'$ folgt: Ist $X$
eine projektive Variet�t und $U\subset X$ offen, so k�nnen wir 

$\mathcal{O}_{X}(U)=\{f:U\rightarrow k\mid\forall x\in U\ \exists x\in V\underset{\text{offen}}{\subset}U,\ g,h\in k[X_{0},\ldots,X_{n}]$
homogen vom gleichen Grad mit $h(v)\neq0,\ f(v)=\frac{g(v)}{h(v)},\ \forall v\in V\}$.
({*}) 

Insbesondere gilt:
\begin{prop}[orig. 56]
  \label{prop:charakterisierung-morphismen-proj-varietaeten}
  Seien $V\subseteq\mathbb{P}^{m}(k)$, $W\subset\mathbb{P}^{n}(k)$
  projektive Variet�ten und
  \[
    V\subseteq\mathbb{P}^{m}(k)\xrightarrow{\phi}W\subseteq\mathbb{P}^{n}(k)
  \]

  eine Abbildung. Dann ist $\phi$ eine Morphismus genau dann, wenn
  es zu jedem $x\in V$ eine offene Menge $x\in U_{x}\subset V$ und
  homogene Polynome $f_{0},\ldots,f_{n}\subseteq k[X_{0},\ldots,X_{m}]$
  vom selben Grad existiert mit
  \[
    \phi(y)=(f_{0}(y),\ldots,f_{n}(y))\quad\forall y\in U_{x}
  \]
\end{prop}
\begin{proof}
  \mbox{}
  \begin{itemize}
  \item ``$\Rightarrow$'', �bung.
  \item ``$\Leftarrow$''.
    \begin{enumerate}
    \item $\phi$ stetig: Sei $Z\subseteq W$ abgeschlossen. Ohne Einschr�nkung
      $Z=V_{+}(g)\cap W$ f�r ein homogenes Polynom $g$. Dann berechnet
      sich das Urbild
      \[
        \phi^{-1}(Z)=V_{+}(g\circ\phi)\cap V.
      \]
      Auf $U_{x}$, $x\in V$, ist $g\circ\phi$ als homogenes Polynom in
      $X_{0},\ldots,X_{n}$ gegeben. 

      $\Rightarrow V(g\circ\phi)\cap U_{x}=\phi^{-1}(Z)\cap U_{x}$ abgeschlossen
      in $U_{x}$ f�r alle $x$.

      $\Rightarrow\phi^{-1}(Z)\subseteq V$ abgeschlossen.
    \item Zu zeigen: $\forall W'\subseteq W$ offen, $g\in\mathcal{O}_{W}(W')$
      ist $g\circ\phi\in\mathcal{O}_{V}(\phi^{-1}(W'))$.

      $\Rightarrow$ ({*}) Es ex. eine offene Umgebung $W_{y}$ in $W'$
      mit $g=\frac{h}{q}$ auf $W_{y}$, $h,q$ homogen vom Grad $d$.

      $\Rightarrow\phi_{|U_{x}\cap\phi^{-1}(W_{y}):=\tilde{U}_{x}}$ ist
      auch von dieser Gestalt.

      $\Rightarrow$ ({*}) $\frac{h(f_{0},\ldots,f_{n})}{q(f_{0},\ldots,f_{n})}=g\circ\phi_{|\tilde{U}_{x}}\in\mathcal{O}_{V}(\tilde{U}_{x})$.
    \end{enumerate}
    $\Rightarrow$ (Verkleben) $g\circ\phi\in\mathcal{O}_{V}(\phi^{-1}(V))$.
  \end{itemize}
\end{proof}




\section{Koordinatenwechsel in $\mathbb{P}^{n}$}
\label{sec:koordinatenwechsel-projektiver-raum}

$A=(a_{ij})\in GL_{n+1}(k)$ eine invertierbare $k^{n+1}\rightarrow k^{n+1}$
lineare Abbildung, die Ursprungsgeraden in solche überführt, bzw.
die Äquivalenzrelation respektiert. Wir erhalten Abbildungen:
\begin{align*}
  \mathbb{P}^{n}(k) & \overset{\phi_{A}}{\longrightarrow}\mathbb{P}^{n}(k)\\
  (x_{0}:\ldots:x_{n}) & \longmapsto\left(\sum_{i=0}^{n}a_{0_{i}}x_{i}:\cdots:\sum_{i=0}^{n}a_{n_{i}}x_{i}\right),
\end{align*}

die nach Satz 56 ein Morphismus von Prävarietäten ist. Offensichtlich
gilt für $A,B\in GL_{n+1}(k)$:
\[
  \varphi_{A\cdot B}=\varphi_{A}\circ\varphi_{B}
\]

d.h. $\varphi_{A}$ ist insbesondere wieder ein Isomorphismus, \textbf{der
  durch $A$ bestimmte Koordinatenwechsel des $\mathbb{P}^{n}(k)$}.
\emph{Bezeichne} Aut$(\mathbb{P}^{n}(k))$ die Gruppe der Automorphismen
von $\mathbb{P}^{n}(k)$. Es folgt:
\[
  \varphi_{-}:GL_{n+1}(k)\rightarrow\text{Aut}(\mathbb{P}^{n}(k))
\]

ist ein Gruppenhomomorphismus mit 
\[
  Z:=\ker\varphi=\{\lambda E_{n+1},\ \lambda\in k^{\times}\}
\]

die Untergruppe der Skalarmatrizen. \emph{Später}:
\[
  PGL_{n+1}(k):=GL_{n+1}(k)/Z\twoheadrightarrow\text{Aut}(\mathbb{P}^{n}(k)),\quad Z\cong k^{\times}
\]

die \textbf{projektive lineare Gruppe}.
\begin{example*}
Sei $n=1$. Es ist
\begin{align*}
PGL_{2}(\mathbb{C}) & =\left\{ \begin{array}{rl}
\mathbb{P}^{1}(\mathbb{C}) & \rightarrow\mathbb{P}^{1}(\mathbb{C})\\
(z:w) & \mapsto(az+bw,cz+dw)
\end{array}\right\} \\
 & \leftrightarrow\text{Möbiustransformationen }z\mapsto\frac{az+b}{cz+d}
\end{align*}
\end{example*}




\section{Lineare Unterräume von $\mathbb{P}^{n}$}
\label{sec:lineare-unterraeume-von-pn}

Sei $\varphi:k^{n+1}\rightarrow k^{n+1}$ ein \emph{injektiver} Homomorphismus
von $k$-Vektorräumen. $\varphi$ induziert eine injektive Abbildung:
\[
  \imath:\mathbb{P}^{n}(k)\rightarrow\mathbb{P}^{n}(k)
\]

der ein Morphismus von Prävarietäten ist nach Satz 56. Das Bild von
$\imath$ ist eine abgeschlossene Untervarietät. Ist $A=(a_{ij})\in M_{l\times(n+1)}$
mit $\text{im}(\varphi)=\ker(k^{n+1}\xrightarrow{A}k)$ und
\[
  f_{i}:=\sum_{j=0}^{n}a_{ij}X_{j}\in k[X_{0},\ldots,X_{n}],
\]

so identifiziert $\imath$ $\mathbb{P}^{n}(k)$ mit $V_{+}(f_{1},\ldots,f_{l})$.
(Die Abbildung $\imath:\mathbb{P}^{n}(k)\rightarrow V_{+}(f_{1},\ldots,f_{l})$
ist ein Isomorphismus von Prävarietäten, mit Umkehrabbildung $\varphi^{-1}:\varphi(k^{n+1})\rightarrow k^{n+1}$
induziert.)
\begin{example*}
  $\mathbb{P}^{m}=V_{+}(X_{m+1},\ldots,X_{n})\subset\mathbb{P}^{n}$.
  Solche Unterräume heißen \textbf{lineare Unterräume} (der Dimension
  $m$).

  $m=0$: Punkte

  $m=1$: Geraden

  $m=2$: Ebenen

  $m=n-1$: Hyperebenen in $\mathbb{P}^{n}(k)$.
  \begin{itemize}
  \item Zu zwei Punkten $p\neq q\in\mathbb{P}^{n}(k)$ existiert genau eine
    gerade $\overline{pq}$ in $\mathbb{P}^{n}(k)$, die $p$ und $q$
    enthält, da zu zwei verschiedenen Ursprungsgeraden im $k^{n+1}$ genau
    eine Ebene (in $k^{n+1})$ existiert, die beide Geraden enthält.
  \end{itemize}
\end{example*}
\begin{itemize}
\item Je zwei verschiedene Geraden in $\mathbb{P}^{2}(k)$ schneiden sich
  in genau einem Punkt, da Geraden in $\mathbb{P}^{2}$ Ebenen in $k^{3}$
  entsprechen, und zwei Ebenen sich dort genau in einer Geraden, d.h.
  einem Punkt des $\mathbb{P}^{2}$, schneiden. Dimensionsformel (lineare
  Algebra):
  \[
    \dim E_{1}\cap E_{2}=-\underbrace{\dim E_{1}+E_{2}}_{3}+\underbrace{\dim E_{1}}_{2}-\underbrace{\dim E_{2}}_{2}=1
  \]
  \emph{Später}: Verallgemeinerung: Satz von Bézout für allgemeine Unterprävarietäten
  $V_{+}(f)$.
\end{itemize}



\section{Kegel}
\label{sec:Kegel}

Sei $H\subseteq\mathbb{P}^{n}(k)$ Hyperebene, $p\in\mathbb{P}^{n}(k)\backslash H$,
$X\subseteq H$ abgeschlossene Unterprävarietät.
\[
  \overline{X,p}:=\bigcup_{q\in X}\overline{qp}
\]

heißt \textbf{Kegel von $X$ über $p$}, es handelt sich um einen
abgeschlossenen Untervarietät von $\mathbb{P}^{n}(k)$. Ohne Einschränkung:$H=V_{+}(X_{n})$,
$p=(0:\cdots:1)$ (nach Koordinatenwechsel: $H\cong k^{n}\oplus p\cong k\}=k^{n+1}$.)
Für  
\begin{align*}
  X=V_{+}(f_{1},\ldots,f_{m})\subseteq\mathbb{P}^{n-1}(k)=H, & \quad f_{i}\in k[X_{0},\ldots,X_{n-1}]\\
  \Rightarrow X,p=V_{+}(\tilde{f}_{1},\ldots,\tilde{f}_{m})\subseteq\mathbb{P}^{n}(k), & \quad\tilde{f}_{i}\in k[X_{0},\ldots,X_{n}]
\end{align*}

Verallgemeinerung. Sei $\mathbb{P}^{n}(k)\cong\Lambda\subseteq\mathbb{P}^{n}(k)$
linearer Unterraum, $\psi\subseteq\mathbb{P}^{n}(k)$ komplementärer
linearer Unterraum, d.h. $\Lambda\cap\psi=\emptyset$ und $\mathbb{P}^{n}(k)$
ist der bekannte lineare Unterraum von $\mathbb{P}^{n}(k)$, der $\Lambda$
und $\psi$ enthält. $X\subseteq\psi$ abgeschlossene Unterprävarietät.

\textbf{Kegel von $X$ über $\Lambda$}: $\overline{X,\Lambda}=\bigcup_{q\in X}\overline{q,\Lambda}$,
wobei der von $q$ und $\Lambda$ aufgespannte lineare Unterraum $\overline{q,\Lambda}$
der kleinste Unterraum sei, der $q$ und $\Lambda$ enthält.


\newpage{}

\printindex{}
\end{document}
