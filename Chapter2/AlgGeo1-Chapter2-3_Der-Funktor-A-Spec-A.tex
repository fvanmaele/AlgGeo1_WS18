\section{Der Funktor $A\protect\mapsto\text{Spec}(A)$}

\textbf{Ziel:} Wir wollen einen kontravarianten Funktor
\begin{align*}
  \text{Ringe} & \longrightarrow\text{Top. Räume}\\
  A & \longmapsto\Spec A
\end{align*}
definieren. Sei $\varphi:A\longrightarrow B$ Ringhomomorphismus, $\mathfrak{q}$ Primideal. Es folgt:
$\varphi^{-1}(\mathfrak{q})\subset A$ Primideal ($A/\varphi^{-1}(\mathfrak{q})\hookrightarrow B/\mathfrak{q}$ integer)
d.h. wir erhalten eine Abbildung
\begin{align*}
  ^{a}\varphi=\Spec\varphi:\ \Spec B & \longrightarrow\Spec A\\
  \mathfrak{q} & \longmapsto\varphi^{-1}(\mathfrak{q})
\end{align*}

\begin{prop} \mbox{}
  \begin{enumerate}
  \item $(^{a}\varphi)^{-1}(V(M))=V(\varphi(M))$ für
    $M\subset\text{Spec }A$ Teilmenge, insbesondere
    $(^{a}\varphi)^{-1}(D(f))=D(\varphi(f))$, $f\in A$.
  \item
    $V(\varphi^{-1}(\mathfrak{b}))=\overline{^{a}\varphi(V(\mathfrak{b}))}$
    für $\mathfrak{b}\subset B$ Ideal.
  \end{enumerate}
\end{prop}

\begin{proof} \mbox{}
  \begin{enumerate}
  \item Für $\mathfrak{q}\in\text{Spec }B$ gilt:
    \begin{align}
      \mathfrak{q}\supset\varphi(M)
      & \Leftrightarrow\varphi^{-1}(\mathfrak{q})\supset M\\
      \mathfrak{q}\in V(\varphi(M)) & \\
      & \mathfrak{q}\in(^a\varphi)^{-1}(V(M))
    \end{align}
    Weiter:
    \begin{align}
      D(\varphi(f)) & =\Spec(B)\backslash V(\varphi(f)) \\
                    & =\Spec(B)\backslash (^a\varphi)^-1 (V(f)) \\
                    & = (^a\varphi)^-1 (D(f))
    \end{align}
     
  \item
    $\overline{^{a}\varphi(V(\mathfrak{b}))}=VI(^{a}\varphi(V\mathfrak{b}))$
    nach Satz 3. Nach Definition gilt:
    \begin{align*} I(^{a}\varphi(V(\mathfrak{b}))
      & =\bigcap_{\mathfrak{p}\in^{a}\varphi(V(\mathfrak{b}))}
        \mathfrak{p}=\bigcap_{\mathfrak{q}\in V(\mathfrak{b})}
        \varphi^{-1}(\mathfrak{q})\\ \text{komm. Alg. }
      & =\varphi^{-1}(\rad\mathfrak{b})\\
      & \overset{!}{=}\rad\varphi^{-1}(\mathfrak{b})
    \end{align*}
    Denn: Ohne Einschränkung $\mathfrak{b}=0$,
    $\varphi^{-1}(\mathfrak{b})=\ker\varphi$
    ($A/\varphi^{-1}(\mathfrak{b})\hookrightarrow B/\mathfrak{b})$. Es
    ist:
    \begin{align*}
      a\in\varphi^{-1}(\sqrt{0})
      & \Leftrightarrow\varphi(a)^{n}=\varphi(a^{n})=0
        \text{ für }n\text{geeignet}
    \end{align*} $V(\ )$
    liefert die Behauptung: $V(\rad\varphi^{-1}
    (\mathfrak{b}))=V(\varphi^{-1}(\mathfrak{b}))$ nach Satz 3.1.
  \end{enumerate}
\end{proof}
Insbesondere ist $^{a}\varphi:\Spec B\rightarrow\Spec A$
\emph{stetig}.  Wegen
\[
  ^{a}(\psi\circ\varphi)=^{a}\varphi\circ^{a}\psi
\]
für einen weiteren Ringhomomorphismus $\psi:B\rightarrow C$ ist
$A\mapsto\text{Spec }B$ der gesuchte kontravariante Funktor.

\begin{cor}
  $^{a}\varphi$ ist \textbf{dominant},
  d.h. $\text{Bild}(^{a}\varphi)\subset\text{Spec }A$ dicht
  $\Leftrightarrow$ Jedes Element in $\ker\varphi$ ist nilpotent:
  $\ker\varphi\subset\text{rad}(0)$.
\end{cor}

\begin{prop} \mbox{}
  \begin{enumerate}
  \item Ist $\varphi:A\rightarrow B$ ein surjektiver
    Ringhomomorphismus mit $\ker\varphi=:\mathfrak{a}$, dann ist
    $^{a}\varphi$ ein Homöomorphismus von $\Spec B$ auf
    $V(\mathfrak{a})\underset{\text{abg.}}{\subseteq}\Spec A$.
  \item Ist $S$ eine multiplikativ abgeschlossene Teilmenge von $A$,
    und $\varphi:A\longrightarrow S^{-1}A=:B$ die kanonische
    Lokalisierungsabbildung.  Dann ist $^{a}\varphi$ wieder ein
    Homöomorphismus, von $\Spec S^{-1}A$ auf $\{\mathfrak{p}\in\Spec A\mid
    S\cap\mathfrak{p}=\emptyset\}$.
  \end{enumerate}
\end{prop}

\begin{proof}
  $^{a}\varphi$ injektiv + Bild ist klar aus kommutative Algebra.
  \emph{Ferner}: Für $\mathfrak{q}\in\text{Spec }B$,
  $\mathfrak{b}\subseteq B$ Ideal gilt
  $\mathfrak{q}\supset\mathfrak{b}\Leftrightarrow
  \varphi^{-1}(\mathfrak{q})\supset\varphi^{-1}(\mathfrak{b})$, also
  \begin{align*}
    ^{a}\varphi(V(\mathfrak{b})) & =V(\varphi^{-1}(\mathfrak{b})),
  \end{align*}
  d.h. $^{a}\varphi$ ist abgeschlossen.
\end{proof}
