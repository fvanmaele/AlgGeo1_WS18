\section*{Exkursion über Garben}

Bisher: 
\begin{align}
X \text{affin alg. Menge} \longmapsto \Gamma(X) = \Hom(X, \mathbb{A}^{1})
\end{align}
Jetzt:
\begin{align}
\Spec A \longmapsfrom A
\end{align} 
d.h. $A$ soll den Funktionen auf $\Spec A$ entsprechen.
Für $x \in \Spec A$ definiert man $ev_{x} : A \to \kappa_A(x) := A_{\mathfrak{p}_x} / \mathfrak{p}_{x}A_{\mathfrak{p}_{x}} \cong \Quot(A/\mathfrak{p}_{x})$ durch $f \mapsto f(x) := ev_{x}(f) := f \mod \mathfrak{p}_{x}$.
Mit dieser Definition folgt insbesondere $D(f) = \{ x \in \Spec A \mid f(x) \neq 0\}$. Da $x \mapsto f(x)$ keine Funktion im engeren Sinne ist, können wir diese Konstruktion nicht als System von Funktionen auffassen.\\
Wichtige Aussagen: Restriktion $+$ Verklebung $\rightsquigarrow$ Garben.

\section{Prägarben und Garben}
\label{sec:garben}

\begin{defn}
\label{def:praegarbe}
Sei $X$ ein topologischer Raum. \\
$(i)$ Eine $\textbf{Prägarbe}\  \sheaf{F}$ auf X besteht aus den folgenden Daten:
\begin{itemize}
	\item eine Menge $\sheaf{F}(U)$ für jede offene Teilmenge $U \subseteq X$
	\item Eine $\textbf{Restriktionsabbildung} $ $res^{V}_{U} : \sheaf{F}(V) \to \sheaf{F}(U)$ für jedes Paar $U \subseteq V$ offen in $X$, so dass:
	\begin{itemize}
		\item $res^{U}_{U} = \id_{\sheaf{F}(U)}$
		\item $res^{W}_{U} = res^{V}_{U} \circ res^{W}_{V}$ für $U \subseteq V \subseteq W$ offen in $X$
	\end{itemize}
\end{itemize}
$(ii)$ Ein $\textbf{Morphismus von Prägarben}$ $\phi : \sheaf{F} \to \sheaf{G}$ ist eine Familie von Abbildungen $\{\phi_{U} : \sheaf{F}(U) \to \sheaf{G}(U) \mid U \subseteq X \text{ offen }\}$, so dass für alle Paare $U \subseteq V$ offen in $X$, das folgende Diagramm kommutiert:
\[
\xymatrix{\sheaf{F}(V)\ar^{\phi_V}@{->}[r]\ar^{res^{V}_{U}}@{->}[d]  & \sheaf{G}(V) \ar[d]^{res^{V}_{U}}\\
	\sheaf{F}(U)\ar^{\phi_{U}}@{->}[r] & \sheaf{G}(U) }
\]
\underline{Notation:} $U \subseteq V$, $s \in \mathcal{F}(V)$, dann: $s|_{U} := res^{V}_{U}(s)$.\\ Die Elemente in $\sheaf{F}(U)$ heißen $\textbf{Schnitte von } \sheaf{F} \textbf{ über }U$, $\Gamma(U, \sheaf{F}) := \sheaf{F}(U)$.
\end{defn}

Alternative Beschreibung:\\
$\ouv_{X}$: Kategorie offener Mengen von $X$ mit $\Hom(U, V) := \begin{cases} \emptyset, \text{ falls } U \not\subseteq V \\ \{ U \to V\}, \text{ falls } U \subseteq V \end{cases}$.
Eine $\textbf{Prägarbe auf }X$ ist ein kontravarianter Funktor $\mathcal{F} : \ouv_{X} \to \set$. Ersetzt man $\set$ durch eine Kategorie $\cat{C}$, so bekommt man Prägarben $\textbf{mit Werten in } \cat{C}$.
Ein Morphismus von Prägarben $\sheaf{F} \to \sheaf{G}$ ist eine natürliche Transformation $\sheaf{F} \implies \sheaf{G}$.\\
\\
Für eine Prägarbe $\sheaf{F}$ auf $X$, $U \subseteq X$ offen und $U = \bigcup_{i} U_{i}$ eine \underline{offene} Überdeckung von $U$, definiere:
\begin{align}
\rho : \sheaf{F}(U) \to \prod_{i}\sheaf{F}(U_{i}), s \mapsto (s|_{U_{i}})_{i}\\
b : \prod_{i} \sheaf{F}(U_{i}) \to \prod_{(i, j)} \sheaf{F}(U_{i} \cap U_{j}), (s_{i})_{i} \mapsto (s_{i}|_{U_{i} \cap U_{j}})_{(i,j)}\\
b' : \prod_{i} \sheaf{F}(U_{i}) \to \prod_{(i, j)} \sheaf{F}(U_{i} \cap U_{j}), (s_{i})_{i} \mapsto (s_{j}|_{U_{i} \cap U_{j}})_{(i,j)}
\end{align}

\begin{defn}
\label{def:garbe}
$(i)$ Eine Prägarbe $\sheaf{F}$ auf $X$ heißt $\textbf{Garbe}$, falls für alle offenen Teilmengen $U \subset X$ und alle offenen Überdeckungen $U = \bigcup_{i} U_{i}$ wie oben gilt:\\
\[
(Sh)\ \xymatrix{\sheaf{F}(U)\ar^{\rho}@{->}[r] & \prod_{i}\sheaf{F}(U_{i}) \ar^{b}@<+.5ex>[r] \ar_{b'}@<-.5ex>[r] & \prod_{(i, j)}\sheaf{F}(U_{i} \cap U_{j})}
\]d.h. $\rho$ ist injektiv und $\im\rho = \{ s \in \prod_{i} \sheaf{F}(U_{i}) \mid b(s) = b'(s)\}$, mit anderen Worten: $(\sheaf{F}(U), \rho)$ ist \textbf{Equalizer} von $b$ und $b'$.\\
Dabei ist $(Sh)$ äquivalent zu:
\begin{enumerate}
	\item[(Sh0)]   $\sheaf{F}(\emptyset)$ ist finales Objekt.
	\item[(Sh1)] Gilt für $s,s' \in \sheaf{F}(U)$  $s|_{U_{i}} = s'|_{U_{i}}$ für alle $i$, so folgt $s = s'$.
	\item[(Sh2)] Zu jeder Familie $(s_{i})_{i} \in \prod_{i}\sheaf{F}(U_{i})$ mit $s_{i}|_{U_{i} \cap U_{j}} = s_{j}|_{U_{i} \cap U_{j}}$ existiert ein $s \in \sheaf{F}(U)$ mit $s|_{U_{i}} = s_{i}$.
\end{enumerate}
$(ii)$ Ein \textbf{Morphismus von Garben} ist ein Morphismus der unterliegenden Prägarben.\\
\\
Wir erhalten die Kategorie $\psh_{X}(\set)$ der Mengenwertigen Prägarben auf $X$ und die \underline{volle} Unterkategorie $\sh_{X}(\set)$ der Mengenwertigen Garben auf $X$.
Analog erhalten wir Garben von abelschen Gruppen, Ringen, $R$-Moduln und $R$-Algebren.
\end{defn}

\begin{rem}
\label{rem:pathologien-garben}
\begin{enumerate}
	\item $\sheaf{F} \in \sh \implies \Gamma(\emptyset, \sheaf{F})$ ist einpunktig (wegen $(Sh)$ für die leere Überdeckung)
	\item $X = \{ pt \} \implies \sheaf{F}$ auf $X$ ist eindeutig durch $\sheaf{F}(X)$ bestimmt
\end{enumerate}
\end{rem}

%% TODO: Insert Abschnitt zu Limiten hier

\begin{example}
\label{bsp:beispiele-von-garben}	
\begin{enumerate}
	\item $\sheaf{F} \in \psh_{X}, U \subseteq X$ offen $\implies$ $\sheaf{F}|_{U} \in \psh_{U}$ mit $\Gamma(V, \sheaf{F}|_{U}) := \Gamma(V, \sheaf{F})$. Ist $\sheaf{F} \in \sh_{X}$, so ist $\sheaf{F}|_{U} \in \sh_{U}$. 
	\item Für $X,Y$ top. Räume definiert $\sheaf{F}$ gegeben durch $\Gamma(U, \sheaf{F}) := \mathcal{C}(U, Y) = \{ f : U \to Y \mid \ f \text{ stetig}\}$ und $res^{U}_{V} : f \mapsto f|_{V}$ eine Garbe.
	\item $k$ ein Körper, $(X,\sheaf{O}_X)$ ein Raum mit Funktionen$/_k$ $\implies$ $\sheaf{O}_{X}$ ist Garbe von $k$-Algebren auf $X$.
	\item Für einen top. Raum $X$ definiert $\sheaf{F}(U) := \{ f: U \to \mathbb R \mid f \text{ stetig und beschränkt}\}$ eine Prägarbe auf $X$, im Allgemeinen aber keine Garbe.
\end{enumerate}
\end{example}

Sei $\mathcal{B}$ eine Basis der Topologie von $X$ und $\sheaf{F} \in \sh_{X}$. Sei für $V \subseteq X$ offen $\mathcal{B}_{V} := \{ U \in \mathcal{B} \mid U \subseteq V\}$. Dann folgt wegen $(Sh)$:\\
\[
\sheaf{F}(V) \cong_{(\dagger)} \{ (s_{U})_{U} \in \prod_{U \in \mathcal{B}_{V}} \sheaf{F}(U) \mid \forall U' \subseteq U \in \mathcal{B}_{V}: s_{U}|_{U'} = s_{U'}\} \cong \varprojlim_{U \in \mathcal{B}_{V}} \sheaf{F}(U)
\]d.h. $\sheaf{F}$ ist bereits eindeutig durch die Schnitte auf einer Basis von $X$ bestimmt.\\
$(\dagger):$ einfache Folgerung aus $(Sh1)$.\\
\\
Eine \textbf{Prägarbe auf } $\mathcal{B}$ ist ein kontravarianter Funktor $\sheaf{F} : \mathcal{B} \to \set$. Jedes solche $\sheaf{F}$ induziert eine Prägarbe $\overline{\sheaf{F}}^{X}$ auf $X$ durch $\overline{\sheaf{F}}^{X}(V) := \varprojlim_{U \in \mathcal{B}_{V}} \sheaf{F}(U)$.
Für $U \in \mathcal{B}$ gilt dann $\overline{\sheaf{F}}^{X}(U) = \varprojlim_{U' \in \mathcal{B}_{U}}\sheaf{F}(U') = \sheaf{F}(U)$, da $U$ initial in $\mathcal{B}_{U}$.\\ Ein \textbf{Morphismus von Prägarben auf } $\mathcal{B}$ ist wieder ein Morphismus von Funktoren.


