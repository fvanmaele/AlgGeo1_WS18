\emph{Bisher}:

 Prävarietäten$_{/k}$ sind Verklebungen von $\emph{affinen Varietäten}_{/k}$ mit $k$ algebraisch abgeschlossen.

Dabei sind affine Varietäten$_{/k}$ äquivalent zu \emph{endlich erzeugten, integren
$k$-Algebren}, 

wobei die Punkte der Varietäten den maximalen Idealen der $k$-Algebren entsprechen.

\rule[0.5ex]{1\columnwidth}{1pt}

\textbf{Ziel}: 
Schemata sind Verklebungen von \emph{affinen Schemata}.

Dabei sollen affine Schemta äquivalent zu \emph{beliebigen} (kommutativen) Ringen sein. 

Die Punkte affiner Schemata werden den \emph{Primidealen} der zugehörigen Ringe entsprechen.

\rule[0.5ex]{1\columnwidth}{1pt}

\textbf{Methodik}: Wir wollen einen Funktor:
\begin{align*}
  A & \longmapsto (\underbrace{\Spec(A)}_{\text{top. Raum}}, \underbrace{\mathcal{O}_{\Spec(A)}}_{Garbe})
\end{align*}

$\mathcal{O}_{\Spec(A)}$ ist dabei ,,Garbe von Funktionen`` und verallgemeinert ,,Systeme von Funktionen``
für Raume mit Funktionen.

Wir erhalten insbesondere affine Schemata für $k$-Algebren über beliebigen Körpern $k$!

Grund für den Übergang zu Primidealen:

Für einen Ringhomomorphismus $\varphi:A\rightarrow B$, und ein maximales Ideal
$\mathfrak{m}\subseteq B$ ist $\mathfrak{m}^{c} := \varphi^{-1}(\mathfrak{m})$ im Allgemeinen \textbf{kein} maximales Ideal von $A$.
Wir erhalten in dieser Allgemeinheit also keinen Funktor auf den Maximalsprektren, wie bisher.

\section*{Das Ringspektrum als topologischer Raum}

\section{Definition von Spec(A)}
\label{sec:def-von-spec}

Sei $A$ stets ein kommutativer Ring. $\Spec(A)$ :=
$\{\mathfrak{p}\unlhd A \mid \mathfrak{p} \text{ prim}\}$.

Für $M\subseteq A$ definiert man
\begin{align*}
  V(M) := V_{A}(M) & :=\{\mathfrak{p}\in\Spec(A)\mid M \subseteq \mathfrak{p} \}=V(\langle M\rangle_{A})\\
  V(f) & :=V(\{f\})\text{\, für }f\in A
\end{align*}

\begin{lem}
\label{lem:zariski-top-auf-spektrum}
Es ist
  \begin{align*}
    \{\text{Ideale in }A\} & \longrightarrow\text{\{Teilmengen in }\Spec(A)\}\\
    \mathfrak{a} & \longmapsto V(\mathfrak{a})
  \end{align*}

  eine inklusionsumkehrende Abbildung. Weiter gilt:
  \begin{enumerate}
  \item $V(0)=\Spec(A)$, $V(1)=\emptyset$.
  \item $V\left(\bigcup_{i\in
        I}\mathfrak{a}_{i}\right)=V\left(\sum_{i\in
        I}\mathfrak{a}_{i}\right)=\bigcap_{i\in I}V(\mathfrak{a}_{i})$
  \item
    $V(\mathfrak{a}\cap\mathfrak{a}')=V(\mathfrak{a}\mathfrak{a}')=V(\mathfrak{a})\cup
    V(\mathfrak{a}')$
  \end{enumerate}
\end{lem}
\begin{proof} \mbox{}
  \begin{itemize}
  \item (1), (2) klar.
  \item
    (3). $\mathfrak{p}\supseteq\mathfrak{a}\cap\mathfrak{a}'\supseteq\mathfrak{a}\mathfrak{a}'$ $\Rightarrow\mathfrak{p}\supseteq\mathfrak{a}\mathfrak{a}'$.

    $\mathfrak{p}$ prim $\Rightarrow$ $\mathfrak{p}\supseteq\mathfrak{a}$ oder
    $\mathfrak{p}\supseteq\mathfrak{a}'$.

    $\Rightarrow\mathfrak{p}\supseteq\mathfrak{a}\cap\mathfrak{a}'$

  \end{itemize}
\end{proof}
\begin{defn}
\label{def:spec-als-top-raum}	
$\Spec(A)$ mit der Topologie, dessen abgeschlossene Mengen
  gerade die Mengen der Form $V(\mathfrak{a})$,
  $\mathfrak{a}\unlhd A$ ein Ideal sind, heißt das \textbf{(Prim)Spektrum von $A$}
  (mit der Zariski-Topologie).
  \begin{align*}
    x\in\Spec(A) & \leftrightarrow\mathfrak{p}_{x}\unlhd A\text{ prim}\\
    Y\subseteq\Spec(A), & \phantom{\leftrightarrow\
                              }I(Y):=\bigcap_{y \in Y}\mathfrak{p}_{y}
  \end{align*}

  $I(-)$ ist inklusionumkehrend, $I(\emptyset)=A$.
\end{defn}
\begin{prop}
\label{prop:nullstellensatz-primspektrum}
Seien $\mathfrak{a}\unlhd A$ ein Ideal und
  $Y\subseteq\Spec(A)$ eine Teilmenge. Dann gilt:
  \begin{enumerate}
  \item $\rad I(Y)=I(Y)$, $V(\mathfrak{a})=V(\rad\mathfrak{a})$
  \item $I(V(\mathfrak{a}))= \rad(\mathfrak{a})$,
    $V(I(Y))=\overline{Y}$ (Abschluss in $\Spec(A)$).
  \item Wir haben eine 1:1-Korrespondenz:
    \begin{align*}
      \{\mathfrak{a} \unlhd A\mid\mathfrak{a}=\rad\mathfrak{a}\}
      & \longrightarrow\{\text{abg. Teilmengen }Y\text{ in }\Spec(A)\}
    \end{align*}
  \end{enumerate}
\end{prop}
\begin{proof} \mbox{}
  \begin{enumerate}
  \item $V(\mathfrak{a})=V(\rad\mathfrak{a})$.
    \begin{itemize}
    \item ,,$\supseteq$``. Klar, da rad
      $\mathfrak{a}\supseteq\mathfrak{a}$.
    \item ,,$\subseteq$``. Aus $f^{r}\in\mathfrak{a}\subseteq\mathfrak{p}$
      folgt $f\in\mathfrak{p}$, da $\mathfrak{p}$ prim, also $\rad\mathfrak{a}\subseteq\mathfrak{p}.$
    \end{itemize}
  \item $\rad\mathfrak{a}=\bigcap_{x\in
      V(\mathfrak{a})}\mathfrak{p}_{x}=IV(\mathfrak{a})$.  Es ist:
    \begin{align*}
      V(\mathfrak{b})\supseteq Y & \Leftrightarrow \forall\mathfrak{p}\in Y:\
                                   \mathfrak{p}\supseteq\mathfrak{b}\\
                                 & \Leftrightarrow
                                   I(Y)\supseteq\mathfrak{b}.
    \end{align*} Damit ist $V(I(Y))$ die kleinste abgeschlossene
    Teilmenge, die $Y$ umfasst, d.h. $V(I(Y))=\overline{Y}$.
  \item Klar.
  \end{enumerate}
\end{proof}
