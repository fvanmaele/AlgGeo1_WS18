\section{Topologische Eigenschaften von Spec(A)}
\label{sec:topologische-eigenschaften-von-spec-A}

Definiere $D(f):=D_{A}(f):=\Spec(A)\setminus V(f)=\{x \in\Spec A \mid f\notin\mathfrak{p}_{x}\}$,
\begin{align*}
  \text{ev}_{x}:A & \longrightarrow A/\mathfrak{p}_{x}\subseteq \kappa_{x}(A) := \Quot(A/\mathfrak{p}_{x})\\
  f & \longmapsto f(x) := f(\mathfrak{p}_{x}) := f \mod \mathfrak{p}
\end{align*}

Für $x \in D(f)$ gilt dann $f(x) = \text{ev}_{x}(f) \neq 0$.

\textbf{Standard prinzipal offene Mengen}.
\begin{align*}
  D(0) & =\emptyset,\ D(1)=\Spec(A)=D(u),\ u\in A^{\times}\\    
  & D(f)\cap D(g) = D(fg)
\end{align*}

\begin{lem}
\label{lem:charakterisierung-ueberdeckungen-prinzipal}
Für $f_{i} \in A, i\in I$, $g\in A$ gilt:
  \begin{align*}
    D(g)\subseteq\bigcup_{i\in I}D(f_{i})
    & \Leftrightarrow g^{n}\in\mathfrak{a}=(f_{i},i\in I)\text{ für }n \in \mathbb{N} \text{ geeignet}\\
    & \Leftrightarrow g\in\rad(\mathfrak{a})
  \end{align*}
\end{lem}
\begin{proof} Es gilt:
  \begin{align*}
    D(g)\subseteq\bigcup_{i}D(f)
    & \Leftrightarrow V(g)\supseteq\bigcap_{i} V(f_{i})=V(\mathfrak{a})\\
    & \Leftrightarrow g\in\rad((g))\subseteq\rad(\mathfrak{a}) \text{ nach } \ref{prop:nullstellensatz-primspektrum}
  \end{align*}

  Für $g=1$, folgt:
  \[ \Spec(A)=\bigcup_{i\in I}D(f_{i})\Leftrightarrow\sum_{i\in
      I}Af_{i}=A
  \]
\end{proof}
\begin{prop}
\label{prop:prinzipal-offene-bilden-basis}
Die prinzipal offenen Mengen $D(f)$, $f\in A$, bilden
  eine Basis der Topologie von $\Spec(A)$, und sind
  quasikompakt. Insbesondere ist $\Spec(A)$ quasikompakt.
\end{prop}
\begin{proof} Nach Lemma \ref{lem:zariski-top-auf-spektrum}$.(ii)$ gilt:
  \[
    V(\mathfrak{a})=\bigcap_{f \in\mathfrak{a}}V(f)\Longrightarrow\Spec A\setminus
    V(\mathfrak{a})=\bigcup_{f\in\mathfrak{a}}D(f)\Rightarrow\text{Basis
      der Topologie}
  \]

  Sei $D(g)\subseteq\bigcup_{i}D(f_{i})$.

  \ref{lem:charakterisierung-ueberdeckungen-prinzipal} $\Rightarrow$ $g^{n}=\sum_{i\in I}a_{i}f_{i}$, $a_{i}\in A$
  fast alle 0.

  $\Rightarrow D(g)\subseteq\bigcup_{i\in J}D(f_{i})$ $\forall i\in J\subseteq I$ endlich

  $\Rightarrow D(g)$ quasikompakt.
\end{proof}

\begin{prop}
  $X\subseteq\text{Spec}(A)$ ist irreduzibel, genau dann, wenn
  $\varphi:=I(Y)\subset A$ prim ist. In diesem Fall ist
  $\{\mathfrak{p}\}\subset\overline{Y}$ dicht!
\end{prop}

\begin{proof} \mbox{}
  \begin{itemize}
  \item Sei $Y$ irreduzibel und $f,g\in A$ mit $fg\in\mathfrak{p}$.

    $\Rightarrow Y\subset\overline{Y}=VI(Y)\subseteq V(fg)=V(f)\cup V(g)$

    $\Rightarrow$ ($X$ irreduzibel) Ohne Einschränkung: $Y\subset V(f)$.

    $\Rightarrow
    f\in\bigcap_{f\in\mathfrak{q}}\mathfrak{q}=IV(f)\subset
    I(Y)=\mathfrak{p}$

    $\Rightarrow\mathfrak{p}$ Primideal.
  \item Sei umgekehrt $\mathfrak{p}=I(Y)$ ein Primideal.

    $\Rightarrow$ (Satz 3)
    $\overline{Y}=V(\mathfrak{p})=VI(\{\mathfrak{p}\})=\overline{\{\mathfrak{p}\}}$,
    d.h. $\overline{Y}$ ist der Abschluss der irred. Menge
    $\{\mathfrak{p}\}$ und daher selbst irreduzibel.

    $\Rightarrow$ (Lemma I.14): $Y$ ist auch irreduzibel, da dicht in
    $\overline{Y}$.
  \end{itemize}
\end{proof}
Warnung: im Allgemeinen ist $\mathfrak{p}$ nicht in $Y$!

\begin{cor}
  Die Abbildung
  \begin{align*}
    \text{Spec}(A) & \longrightarrow\text{\{abg. irred. Teilmengen von Spec }A\}\\
    \mathfrak{p} & \longmapsto V(\mathfrak{p})=\overline{\{\mathfrak{p}\}}
  \end{align*}
  ist eine Bijektion, unter der die maximalen Primideale von $A$ den
  irreduziblen Komponenten entsprechen.
\end{cor}

\begin{proof} Proposition 3 und 6.
\end{proof}

\begin{defn}
  Für ein topologischer Raum $X$ heißt $\eta\in X$ ein
  \textbf{generischer Punkt}, falls
  $\overline{\{\eta\}}=X$. Allgemeiner sagen wir für $x,x'\in X$, dass
  $x$ eine Verallgeimeinerung (eng. ,,generalization``) von $x'$ ist,
  bzw. $x'$ eine Spezialisierung von $x$, falls
  $x'\in\overline{\{x\}}$.
\end{defn}

\begin{rem}\mbox{}
  \begin{enumerate}
  \item $\eta\in X$ generisch $\Leftrightarrow\eta$ ist
    Verallgemeinerung von jedem Punkt von $X$.
  \item Existiert ein generischer Punkt in $X$, so ist $X$ als
    Abschluss einer irreduziblen Menge selbst irreduzibel.
  \item Für $X=\text{Spec}(A)$ gilt: $x'$ ist eine Spezialisierung von
    $x\Leftrightarrow\mathfrak{p}_{x}\subset\mathfrak{p}_{x'}$
    \begin{align*}
      \Leftrightarrow V(\mathfrak{p}_{x'}) & \subset V(\mathfrak{p}_{x})\\
      \shortparallel & \phantom{\subset\,}\shortparallel\\
      x'\in\overline{\{x')} & \in\overline{\{x\}}
    \end{align*}
    Ferner hat jede abgeschlossene irreduzible Teilmenge
    $Y\subset\text{Spec}(A)$ einen eindeutigen generischen Punkt (dies
    gilt nicht für beliebig irreduzible Teilmengen
    $Y\subset\text{Spec}(A)$).
  \end{enumerate}
\end{rem}
