\section{Der Funktor $A\protect\mapsto(\Spec A,\mathcal{O}_{\Spec A})$}
\begin{defn}[34]
  Ein lokal geringter Raum $(X,\mathcal{O}_{X})$ heißt \textbf{affines
    Schema}, falls ein Ring $A$ existiert d.d
  \[
    (X,\mathcal{O}_{X})\cong(\Spec A,\mathcal{O}_{\Spec A})
  \]

  Ein \textbf{Morphismus affiner Schemata} ist ein Morphismus lokal
  geringter Räume. Bezeichne $\aff$ die Kategorie der affinen Schemata.
  \begin{align*}
    \varphi:A & \longrightarrow B & \text{Ringhom.}\\
    f:X:=\Spec A & \longrightarrow Y:=\Spec A & \text{stetige Abb.}
  \end{align*}
\end{defn}

\textbf{Ziel: }Definiere $(f,f^{\flat}):X\rightarrow Y$ mit $f:=^{a}\varphi$
Morphismus von lokal geringter Räume und
\[
  f_{\Spec A}^{\flat}=\varphi:A=\mathcal{O}_{\Spec A}(\Spec A)\rightarrow f_{\ast}\mathcal{O}_{\Spec B}(\Spec B)=B
\]

Dazu: Für $s\in A$ gilt $f^{-1}(D(s))=D(\varphi(s))$ nach Proposition
2.10. Definiere 
\[
  f_{D(s)}^{\flat}:\mathcal{O}_{Y}(D(s))=A_{s}\rightarrow B_{\varphi(s)}=f_{\ast}\mathcal{O}_{X}(D(s))
\]

als die von $\varphi$ induzierte Abbildung. $f^{\flat}$ ist kompatibel
mit $\res_{D(t)}^{D(s)}$ für prinzipal offene Mengen $D(t)\subseteq D(s)$.
$B$ Basis $\Longrightarrow f^{\flat}:\mathcal{O}_{Y}\rightarrow f_{\ast}\mathcal{O}_{X}$
Homomorphismus von Ringgarben. Für $s=1$ erhalten wir $f_{\Spec A}^{\flat}=\varphi$!

Für $x\in X$ gilt:
\[
  \xymatrix{f^{\sharp}:\mathcal{O}_{Y,f(x)}=A_{\varphi^{-1}(\mathfrak{p}_{x})=\mathfrak{p}_{f(x)}}\ar[r] & B_{\mathfrak{p}_{x}}=\mathcal{O}_{X,x}\\
    A\ar[u]\ar[r]^{\varphi} & B\ar[u]
  }
  \qquad(*)
\]

ist der von $\varphi$ induzierte Homomophismus. $f_{x}^{\sharp}$
is lokal:
\[
  \varphi(\varphi^{-1}(\mathfrak{p}_{x}))\subseteq\mathfrak{p}_{x}
\]

\textbf{Bezeichne}: $^{a}\varphi$ für $\Spec(\varphi)=(f,f^{\flat})$,
$^{a}(\psi\circ\varphi)=^{a}\varphi\circ^{a}\psi$. Wir erhalten einen
kontravarianten Funktor
\[
  \Spec:\underline{Ring}\longrightarrow\aff.
\]

Für $f:(X,\mathcal{O}_{X})\rightarrow(Y,\mathcal{O}_{Y})$ Morphismus
von geringten Räumen erhalten wir einen Ringhomomorphismus
\[
  \Gamma(f):=f_{Y}^{\flat}:\Gamma(Y,\mathcal{O}_{Y})=\mathcal{O}_{Y}(Y)\rightarrow\Gamma(X,\mathcal{O}_{X})=(f_{\ast}\mathcal{O}_{X})(Y)=\mathcal{O}_{X}(X).
\]

So erhalten wir einen kontravarianten Funktor
\[
  \Gamma:\aff\longrightarrow\underline{Ring}.
\]

\begin{thm}[35]
  Die Funktoren $\Spec$ und $\Gamma$ definieren eine Anti-Äquivalenz
  zwischen der Kateogire der Ringe und der Kategorie der affinen Schemata.
\end{thm}

\begin{proof}
  $\Spec$ ist essentiell surjektiv per Definition. $\Gamma\circ\Spec$
  ist isomorph zu $\id_{\underline{Ring}}$ nach Konstruktion.Zu zeigen:
  \[
    \Hom_{\underline{Ring}}(A,B)\stackrel[\Gamma]{\Spec}{\rightleftharpoons}\Hom_{\aff}(\Spec B,\Spec A)
  \]

  sind zueinander invers. Es fehlt die Verkettung $\Spec\circ\Gamma=\id_{\aff}$.
  Sei $f\in\Hom_{\aff}(\Spec B,\Spec A)$, $\varphi:=\Gamma(f)$, $^{a}\varphi=f$.
  Für $\mathfrak{p}_{x}\in\Spec B=X$ ist $f_{x}^{\sharp}$ der eindeutig
  bestimmte Ringhomomorphismus, welcher das Diagramm
  \[
    \xymatrix{A\ar[r]^{f_{\Spec A}^{\flat}=\Gamma(f)=\varphi}\ar[d]_{\imath_{A}} & B\ar[d]^{\imath_{B}}\supset\imath_{B}^{-1}(\mathfrak{p}_{x}B_{\mathfrak{p}_{x}})=\mathfrak{p}_{x}\\
      A_{\mathfrak{p}_{f(x)}}\ar[r]_{f_{x}^{\#}\text{ lokal}} & B_{\mathfrak{p}_{x}}\underset{\max}{\supset}\mathfrak{p}_{x}B_{\mathfrak{p}_{x}}
    }
    \qquad(**)
  \]

  kommutieren lässt. Es gilt:
  \begin{align*}
    \mathfrak{p}_{f(x)}A_{\mathfrak{p}_{f(x)}} & \supseteq(f_{x}^{\sharp})^{-1}(\mathfrak{p}_{x}B\mathfrak{p}_{x})=\mathfrak{p}_{f(x)}A_{\mathfrak{p}_{f(x)}}\\
    \mathfrak{p}_{f(x)} & =\imath_{A}^{-1}(\mathfrak{p}_{f(x)}A\mathfrak{p}_{f(x)})\subset A
  \end{align*}

  $f_{x}^{\#}$ lokal $\Longrightarrow f_{x}^{\#}(\mathfrak{p}_{f(x)}A_{\mathfrak{p}_{f(x)}})\subset\mathfrak{p}_{x}B_{\mathfrak{p}_{x}}$
  $\Longrightarrow^{a}\varphi=f$ als stetige Abbildung. Wegen $(*)$
  lässt auch $(^{a}\varphi)_{x}^{\#}$ das Diagramm $(**)$ kommutieren.
  Proposition $\Longrightarrow(^{a}\varphi)^{\#}=f^{\#}$.
\end{proof}
