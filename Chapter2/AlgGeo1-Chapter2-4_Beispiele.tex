\section{Beispiele}
\label{sec:beispiele-spektren}
\begin{itemize}
\item $\Spec A=\emptyset\Leftrightarrow A=\{0\}$.
\item $A$ Körper oder Ring mit einem einzigem Primideal: $\Spec
  A=\{\mathfrak{p}\}$.
\item $A$ Artinsch: $\Spec A$ endlich und diskret (da maximale
  Primideale mit den minimalen Primidealen übereinstimmen)

  ($\Spec A=\Spec(A/\sqrt{0})$, $A/\sqrt{0}$ Produkt von Körpern.
  $\Spec(\prod_{i} A_{i})=\coprod_{i}\Spec(A_{i})$
\end{itemize}

\begin{example}
\label{bsp:spec-von-hauptidealring}	
  Sei $A$ Hauptidealring (z.B. $\mathbb{Z}$ oder $K[X]$). Falls
  $\mathfrak{p}$ ein maximales Ideal ist, dann ist
  $\mathfrak{p}=(\pi)$, $\pi$ Primelement in $A$.

  Alle Primideale sind maximal oder 0.

  Abg. Punkte von $\Spec A\leftrightarrow$ Primelemente modulo $A^{\times}$

  $\overline{\{\eta\}}=\Spec A$ für $\eta\in\Spec A$ mit $\mathfrak{p}_{\eta}=(0)$.

  Abgeschlossene Mengen $\Spec A\neq
  V(\mathfrak{a})\overset{0\neq\mathfrak{a}=(f)}{=}V(f)=\{(p_{1}),\ldots,(p_{n})\}$
  falls $f=p_{1}^{e_{1}}\cdots p_{n}^{e_{n}}$, $p_{i}$ paarweise
  verschieden, $e_{i}\geq1$, sind genau \emph{endliche Mengen abgeschlossener Punkte.}

  $g\neq0\neq f$:
  \begin{align*}
    V(f)\cap V(g) & =V(f,g)=V(d), & d=\text{ggT}(f,g)\\
    V(f)\cup V(g) & =V((f)\cap(g))=V(e), & e=\text{kgV}(f,g)
  \end{align*}

  Falls $A$ \emph{lokaler} Hauptidealring ist (also diskreter
  Bewertungsring, der kein Körper ist), dann:
  \[
    \Spec A=\{x,\eta\},\ \mathfrak{p}_{x}\text{ max. Ideal},\
    \mathfrak{p}_{\eta}=(0)
  \]

  $\{\eta\}$ einzige nicht-???  offene Menge.
\end{example}
 
\begin{example}
\label{bsp:zusammenhang-affine-varietaeten}	
  Sei $k$ algebraisch abgeschlossener Körper. Affine Varietäten
  $V\leftrightarrow$ endlich erzeugte $k$-Algebren $A$.

  $V=$\{max. Ideale in $A$\} $\subseteq\Spec(A)$

  Topologie auf $V$ ist die Unterraumtopologie von $\Spec(A)$.
\end{example}
 
\begin{example}
\label{bsp:spec-polynomring-ueber-hauptidealring}	
  Sei $R$ Hauptidealring, $A=R[T]$, $X=\Spec(A)$. $R$ faktoriell $\Rightarrow R[T]$ faktoriell, nach dem Satz von Gauß, mit Primidealen:
  \begin{enumerate}
  \item $p\in R$ prim
    \begin{proof}
      $p\in R$ prim $\Rightarrow R/pR$ Körper. Nach Proposition \ref{prop:spec-quotienten-lokalisierung} gilt:
      \[
        \overline{pR[T]}=V(pR[T])\cong\Spec\left(R/pR[T]\right)
      \]
      ein Hauptidealring mit unendlich vielen Elementen. Damit ist $pR[T]$
      \emph{nicht} maximal, sondern
      \[ V(pR[T])=\{pR[T], (f,p), f\in R[T]
          \text{ mit } \overline{f}\in R/p[T]
        \text{ irreduzibel}\}
      \]
    \end{proof}
  \item $f\in R[T]$ primitives Polynom, irreduzibel in $\Quot(R)[T]$
    \begin{proof}
      Sei $f$ primitives, irreduzibles Polynom.
      \begin{itemize}
      \item $l(f)\in R^{\times} \Rightarrow$ (Division mit Rest)
        $R\subseteq R[T]/pR[T]$ ist eine ganze Ringerweiterung
        und ein endl.-erz. freier $R$-Modul vom Rang $\deg(f)$.
        Angenommen, $fR[T]$ ist maximal. Dann ist $R[T]/fR[T]$ ein Körper,
        also $R$ ein Körper (da ganze Ringerweiterung). Widerspruch.
      \item Andernfalls kann $fR[T]$ ein maximales Ideal sein: $R$ habe nur
        endlich viele Primelemente.
        \[
          0\neq a := \prod_{p} p\in R,\ f:=aT-1
        \]
        Es folgt:
        \[
          R[T]/fR[T]\cong R[a^{-1}]=\Quot(R)
        \]
        also ist $fR[T]$ maximal.
      \end{itemize}
    \end{proof}
  \end{enumerate}
\end{example}

%%% Local Variables:
%%% mode: latex
%%% TeX-master: "../AlgGeo1"
%%% End:
