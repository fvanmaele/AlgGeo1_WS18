\section{Topologische Eigenschaften in Spec(A)}

Definiere $D(f):=D_{A}(f):=\text{Spec }A\backslash
V(f)=\{\mathfrak{p\in\text{Spec }}A\mid f\notin\mathfrak{p}\}$,
\begin{align*}
  \text{ev}_{x}:A & \longrightarrow A/\mathfrak{p}_{x}\subset\\
  f & \longmapsto f(x)=f(\mathfrak{p}_{x})\equiv f\mod\mathfrak{p}\not\equiv0
\end{align*}

\textbf{standard prinzipal offene Mengen}.
\begin{align*}
  D(0) & =\emptyset,\ D(1)=\text{Spec}(A)=D(u),\ u\in A^{\times}\\
  D(f)\cap D(g) & =D(fg)
\end{align*}

\begin{lem}[4] Für $f_{i,i\in I}$, $g\in A$ gilt:
  \begin{align*}
    D(g)\subset\bigcup_{i\in I}D(f_{i})
    & \Leftrightarrow g^{n}\in\mathfrak{a}=(f_{i},i\in I)\text{ für }n\text{ geignet}\\
    & \Leftrightarrow g\in\text{rad}(\mathfrak{a})
  \end{align*}
\end{lem}
\begin{proof} Es gilt:
  \begin{align*}
    D(g)\subset\bigcup_{i\in I}D(f)
    & \Leftrightarrow V(g)\supset\bigcap V(f_{i})=V(\mathfrak{a})\\
    \text{Prop 3.3}:\ & \Leftrightarrow g\in\text{rad}((g))\subset\text{rad}(\mathfrak{a})
  \end{align*}

  Da $g=1$, folgt:
  \[ \text{Spec}(A)=\bigcup_{i\in I}D(f_{i})\Leftrightarrow\sum_{i\in
      I}Af_{i}=A
  \]
\end{proof}
\begin{prop}[5] Die prinzipal offene Mengen $D(f)$, $f\in A$, bilden
  eine Basis der Topologie von $\text{Spec}(A)$, und sind
  quasi-kompakt. Insbesondere ist $\text{Spec}(A)$ quasi-kompakt.
\end{prop}
\begin{proof} Nach Lemma 1.(2) gilt:
  \[
    V(\mathfrak{a})=\bigcap_{\mathfrak{p}\in\mathfrak{a}}V(f)\Rightarrow\text{Spec}\backslash
    V(\mathfrak{a})=\bigcup_{f\in\mathfrak{a}}D(f)\Rightarrow\text{Basis
      der Top.}
  \]

  Sei $D(f)\subset\bigcup_{i\in I}D(g_{i})$.

  $\Rightarrow$ (Lemma 4) $f^{n}=\sum_{i\in I}a_{i}g_{i}$, $a_{i}\in A$
  fast alle 0.

  $\Rightarrow D(f)\subset\bigcup_{i\in J}D(g_{i})$ $\forall i\notin
  J\subset_{\text{endl}}I$

  $\Rightarrow D(f)$ quasikompakt.
\end{proof}
