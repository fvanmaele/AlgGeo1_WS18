Prävarietäten sind Verklebungen. $k$ algebraisch abgeschlossen:

Affine Varietäten $\leftrightarrow$ integere endlich erzeugte
$k$-Algebren.  Punkte $\hat{=}$ maximale Ideale.

\textbf{Ziel}: Schemata sind Verklebungen.

Affine Schemta $\leftrightarrow$ (kommutative) Ringe. Punkte $\hat{=}$
Primideale.

\textbf{Ziel}: Wir wollen einen Funktor:
\begin{align*}
  A & \longmapsto(\text{Spec}(A),\mathcal{O}_{\text{Spec}(A)})\\
  \text{Ring} & \longrightarrow\text{top. Raum}
\end{align*}

,,Garbe von Funktionen`` verallgemeinert ,,Systeme von Funktionen``
für Raume von Funktionen.

(Insbesondere $k$-Algebren über beliebige Körper $k$!)

(Sei $\varphi:A\rightarrow B$ Ringhomomorphismus,
$\mathfrak{m}\subseteq B$ maximales Ideal. Dann folgt i.A. nicht, dass
$\varphi^{-1}(\mathfrak{m})$ maximal ist. Wir haben also zu wenige
maximale Ideale.)

\section*{Das Ringspektrum als topologischer Raum}

\section{Definition von Spec(A)}

Sei $A$ stets ein kommutativer Ring. Spec(A) =
$\{\mathfrak{p}\subseteq A$ Primideal\}. Sei $M\subset A$.
\begin{align*}
  V(M) & =\{\mathfrak{p}\in\text{Spec}(A)\mid\mathfrak{p}\supset
         M\}=V\{\langle M\rangle\}\\
  V(f) & =V(\{f\})\text{\,für }f\in A
\end{align*}

\begin{lem}[1] Es ist
  \begin{align*}
    \{\text{Ideale in }A\} & \longrightarrow\text{\{Teilmengen in Spec}(A)\}\\
    \mathfrak{A} & \longmapsto V(\mathfrak{A})
  \end{align*}

  ist eine inklusionsumkehrende Abbildung. Es gilt:
  \begin{enumerate}
  \item $V(0)=\text{Spec}(A)$, $V(1)=\emptyset$.
  \item $V\left(\bigcup_{i\in
        I}\mathfrak{a}_{i}\right)=V\left(\sum_{i\in
        I}\mathfrak{a}_{i}\right)=\bigcap_{i\in I}V(\mathfrak{a}_{i})$
  \item
    $V(\mathfrak{a}\cap\mathfrak{a}')=V(\mathfrak{a}\mathfrak{a}')=V(\mathfrak{a})\cup
    V(\mathfrak{a}')$
  \end{enumerate}
\end{lem}
\begin{proof} \mbox{}
  \begin{itemize}
  \item (1), (2) klar.
  \item
    (3). $\mathfrak{p}\supset\mathfrak{a}\cap\mathfrak{a}'\supset\mathfrak{a}\mathfrak{a}'$.

    $\Rightarrow\mathfrak{p}\supset\mathfrak{a}\mathfrak{a}'$.

    $\Rightarrow$ (Primideal) $\mathfrak{p}\supset\mathfrak{a}$ oder
    $\mathfrak{p}\supset\mathfrak{a}'$.

    $\Rightarrow\mathfrak{p}\supset\mathfrak{a}\cap\mathfrak{a}'$

  \end{itemize}
\end{proof}
\begin{defn} Spec(A) mit der Topologie, dessen abgeschlossene Mengen
  gerade die Mengen der Form $V(\mathfrak{a})$,
  $\mathfrak{a}\subset A$ ein Ideal sind, heißt (Prim)Spektrum von $A$
  (mit der Zariski-Topologie).
  \begin{align*}
    x\in\text{Spec}(A) & \leftrightarrow\mathfrak{p}_{x}\subset A\text{ Primideal}\\
    Y\subset\text{Spec}(A), & \phantom{\leftrightarrow\
                              }I(Y):=\bigcap_{\mathfrak{p}\in Y}\mathfrak{p}
  \end{align*}

  $I(-)$ ist inklusionserhaltend, $I(\emptyset)=A$.
\end{defn}
\begin{prop} $\mathfrak{a}\subset A$ Ideal,
  $Y\subset\text{Spec}(A)$. Dann gilt:
  \begin{enumerate}
  \item $\text{rad }I(Y)=I(Y)$, $V(\mathfrak{a})=V(\text{rad
    }\mathfrak{a})$
  \item $I(V(\mathfrak{a}))=\text{rad}(\mathfrak{a})$,
    $V(I(Y))=\overline{Y}$ (Abschluss in $\text{Spec}\,A$).
  \item Wir haben eine 1:1-Korrespondenz:
    \begin{align*} \{\mathfrak{a}\subset A\mid\mathfrak{a}=\text{rad
      }\mathfrak{a}\} & \longrightarrow\{\text{abg. Teilmengen }Y\text{ in
                        Spec }A\}
    \end{align*}
  \end{enumerate}
\end{prop}
\begin{proof} \mbox{}
  \begin{enumerate}
  \item $V(\mathfrak{a})=V(\text{rad }\mathfrak{a})$.
    \begin{itemize}
    \item ,,$\supseteq$``. Klar, da rad
      $\mathfrak{a}\supseteq\mathfrak{a}$.
    \item ,,$\subseteq$``. Aus $f^{r}\in\mathfrak{a}\subseteq\mathfrak{a}$
      folgt $f\in\mathfrak{p}$, da $\mathfrak{p}$ Primideal. Damit:
      $\text{rad }\mathfrak{a}\subset\mathfrak{p}.$
    \end{itemize}
  \item $\text{rad }\mathfrak{a}=\bigcap_{\mathfrak{p}\in
      V(\mathfrak{a})}\mathfrak{p}=IV(\mathfrak{a})$.  Es ist:
    \begin{align*}
      V(\mathfrak{b})\supseteq Y & \Leftrightarrow(\forall\mathfrak{p}\in Y:\
                                   \mathfrak{p}\supset\mathfrak{b})\\
                                 & \Leftrightarrow
                                   I(Y)\supseteq\mathfrak{b}.
    \end{align*} Damit ist $V(I(Y))$ die kleinste abgeschlossene
    Teilmenge, die $Y$ umfasst, d.h. $V(I(Y))=\overline{Y}$.
  \item
  \end{enumerate}
\end{proof}
