
\section{Quasikompakte und noethersche topologische R�ume}

\subsection{Definition 18}

Ein topologischer Raum $X$ hei�t \textbf{quasikompakt}\index{quasikompakt},
wenn jede offene �berdeckung von $X$ eine \emph{endliche} Teil�berdeckung
enth�lt. (,,quasi`` deutet an, dass $X$ in der Regel nicht Hausdorff'sch
ist!). Er hei�t \textbf{noethersch}\index{noethersch}, wenn jede
absteigende Kette
\[
X\supseteq Z_{1}\supseteq Z_{2}\supseteq\cdots
\]

abgeschlossener Teilmengen von $X$ station�r wird ($\Leftrightarrow$
jede aufsteigende Kette offener Teilmengen wird station�r).

\subsection{Lemma 19}

Sei $X$ ein noetherscher topologischer Raum. Dann gilt:
\begin{enumerate}
\item Jede abgeschlossene Teilmenge $Z$ von $X$ ist noethersch.
\item Jede offene Teilmenge $U$ von $X$ ist quasikompakt.
\item Jeder abgeschlossene Teilraum $Z$ von $X$ besitzt nur endlich viele
irreduzibele Komponenten.
\end{enumerate}

\subsubsection{Beweis (Lemma 19)}
\begin{enumerate}
\item Nach Definition, da abgeschlossene Mengen von $Z$ auch solche von
$X$ sind.
\item $U=\bigcup_{i\in I}U_{i}$ offen ; $\mathbb{A}$ nicht quasikompakt.
Dann ist $I_{1}\subset I_{2}\subset\cdots\subset I$ endliche Teilmenge
mit
\[
V_{1}\subsetneq V_{2}\subsetneq\cdots\neq U\quad\text{f�r }V_{j}=\bigcup_{i\in I}U_{i}.
\]
Widerspruch zu noethersch.
\item Es reicht zu zeigen: Jeder noethersche Raum ist Vereinigung endlich
vieler irreduzibeler Teilmengen.

$X$ noethersch $\overset{\text{Zorn}}{\Rightarrow}$ Jede nicht-leere
Menge von algebraischen Teilmengen in $X$ besitzt ein minimales Element.
\[
\mathbb{A}:\quad\emptyset\neq\mathcal{M}\coloneqq\left\{ Z\subset X\text{ abg.}\mid Z\text{ ist \textbf{nicht} endl. Ver. irred. Mengen}\right\} 
\]
$\Rightarrow\exists$ minimales Element, sagen wir $Z$, in $\mathcal{M}$.

$\Rightarrow Z$ ist nicht irreduzibel.

$\Rightarrow Z=Z_{1}\cup Z_{2}$ mit $Z_{1},Z_{2}\subsetneq Z$ abgeschlossen.

$\overset{Z\text{ minimal}}{\Rightarrow}Z_{1},Z_{2}\notin\mathcal{M}$
$\Rightarrow Z\notin\mathcal{M}$. Widerspruch.
\end{enumerate}

\subsection{Satz 20}

Jeder abgeschlossene Teilraum $X\subseteq\mathbb{A}^{n}(k)$ ist noethersch.

\subsubsection{Beweis (Satz 20)}

Nach dem obigen Lemma ist nur zu zeigen, dass $\mathbb{A}^{n}(k)$
noethersch ist.

Absteigende Ketten abgeschlossener Teilmengen $\stackrel[\text{Kor. 11}]{I()}{\leftrightarrow}$
aufsteigende Ketten von (Radikal-)Ideale in $k[\underline{T}]$. Da
$k[\underline{T}]$ nach dem Hilbertschen Basissatz noethersch ist,
werden letzere Ketten station�r.

\subsection{Korollar 21 (Prim�rzerlegung)}

Sei $\mathfrak{A}=\rad(\mathfrak{A})\subseteq k[\underline{T}]$ ein
Radikalideal. Dann gilt: $\mathfrak{A}$ ist Durchschnitt von endlich
vielen Primidealen, die sich jeweils nicht enthalten; diese Darstellung
ist eindeutig bis auf Reihenfolge.

\subsubsection{Beweis (Korollar 21)}

$V(\mathfrak{A})=\bigcup_{i=1}^{n}V(\mathfrak{b}_{i})$, $\mathfrak{b}_{i}$
Primideal, h
\[
\mathfrak{A}=\rad(\mathfrak{A})=I(V(\mathfrak{A}))=\bigcap_{i=1}^{n}\underbrace{I(V(\mathfrak{b}_{i}))}_{\mathfrak{b}_{i}'\text{ max. Primideale (L. 17)}}
\]

