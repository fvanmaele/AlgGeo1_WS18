\section{Yoneda-Lemma}

\textbf{Ziel:} $h_{X}$ beschreibt $X$ eindeutig.

Erinnerung: $\mathcal{F},\mathcal{G}:\mathcal{A}\rightarrow\mathcal{B}$
Funktoren, natürliche Transformation $f\in\Hom(\mathcal{F},\mathcal{G})$:
\[
  f=\{f(X):\mathcal{F}(X):\rightarrow\mathcal{G}(X))_{X\in\mathcal{A}}.
\]

Wir erhalten Kategorien: $\func(\mathcal{A},\mathcal{B})$, \textbf{hier:
}$\mathcal{C}=\schs0$, $\hat{C}=\func(\mathcal{C}^{\op},\set)$.
Wir erhalten einen Funktor
\begin{align*}
  \mathcal{C} & \longrightarrow\hat{\mathcal{C}},\\
  X & \longmapsto h_{X},\\
  f & \longmapsto\text{Pullback }f^{\ast}.
\end{align*}

\begin{prop}[5]
  Sei $X\in\mathcal{C}$, $\mathcal{F}\in\hat{\mathcal{C}}$. Dann
  ist die Abbildung
  \begin{align*}
    \Hom_{\hat{C}}(h_{X},\mathcal{F}) & \longrightarrow\mathcal{F}(X)\\
    \alpha & \longmapsto\alpha(X)(\id_{X})\in\Hom(h_{X}(X),\mathcal{F}(X))
  \end{align*}

  bijektiv und funktoriell.
\end{prop}

Insbesondere ist der obige Funktor $\mathcal{C}\rightarrow\hat{\mathcal{C}}$
volltreu (wähle $\mathcal{F}=h_{Y}$!)
\begin{proof}
  Umkehrabbildung:
  \begin{align*}
    \mathcal{F}(X) & \longrightarrow\Hom_{\hat{\mathcal{C}}}(h_{X},\mathcal{F})\\
    \xi & \longmapsto\alpha_{\xi}=(\alpha_{\xi}(Y))_{Y\in\schs0}
  \end{align*}

  mit
  \begin{align*}
    \alpha_{\xi}(Y):\Hom(Y,X)=h_{X}(Y) & \longrightarrow\mathcal{F}(Y)\\
    f & \longmapsto\mathcal{F}(f)(\xi)\in\Hom(\mathcal{F}(X),\mathcal{F}(Y))
  \end{align*}
\end{proof}
