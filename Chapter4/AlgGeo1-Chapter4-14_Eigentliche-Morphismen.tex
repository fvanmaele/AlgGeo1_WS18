\section{Eigentliche Morphismen}

(eng. ,,proper``)\medskip{}

Ist $f:X\rightarrow Y$ stetige Abbildung zwischen topologischen Räume,
dann heißt $f$ \textbf{eigentlich}, wenn die Urbilder kompakter Mengen
kompat sind.

Sei $X$ Hausdorff, $Y$ lokal kompakt. Dann ist $f$ eigentlich $\Longleftrightarrow f$
universell abgeschlossen. (Bourbaki, Topologie générale, $I$.10 nr.
3, Prop. 7)
\begin{defn}[39]
  Ein Morphismus $f:X\rightarrow Y$ von Schemata heißt \textbf{eigentlich},
  wenn:
  \begin{enumerate}
  \item $f$ von endlichem Typ.
  \item $f$ separiert.
  \item $f$ universell abgeschlossen.
  \end{enumerate}
  Ein $Y$-Schema heißt \textbf{eigentlich}, wenn der Strukturmorphismus
  eigentlich ist. ,,eigentlich`` ist lokal auf dem Ziel.\medskip{}
\end{defn}

\noindent\shadowbox{\begin{minipage}[t]{1\columnwidth - 2\fboxsep - 2\fboxrule - \shadowsize}%
    $f:X\rightarrow Y$ heißt von \textbf{endlichem Typ}, wenn $\forall U\subset Y$
    offen eine Überdeckung $f^{-1}(U)=\bigcup_{i=1}^{n}V_{i}$ existiert,
    sodass $\forall i:\mathcal{O}_{X}(V_{i})$ ist endlich erzeugte $\mathcal{O}_{Y}(U)$-Algebra.%
  \end{minipage}}
\begin{defn}[40]
  Ein Morphismus $f:X\rightarrow Y$ heißt \textbf{projektiv}, wenn
  er sich faktorisieren lässt als
  \[
    \xymatrix{X\ar[rr]^{f}\ar[rd]_{\text{abg. Imm.}}^{g} &  & Y\\
      & \mathbb{P}_{Y}^{n}\ar[ur]_{\text{kan. Morph.}}
    }
  \]

  für ein $n\in\mathbb{N}$.
\end{defn}

\begin{prop}[41]
  Sei $\mathbb{P}$ eine der folgenden Eigenschaften:
  \begin{itemize}
  \item[I.] von endlichem Typ;
  \item[II.] eigentlich;
  \item[III.] projektiv.
  \item[IV.] (separiert)
  \end{itemize}
  Dann gilt:
  \begin{enumerate}
  \item Abgeschlossene Immersionen erfüllen $\mathbb{P}$.
  \item $\mathbb{P}$ ist stabil unter Komposition.
  \item $\mathbb{P}$ ist stabil unter Basiswechsel.
  \item Falls $f:X\rightarrow Z$, $g:Y\rightarrow Z$ $\mathbb{P}$ erfüllen,
    dann auch $X\times_{Z}Y\rightarrow Z$.
  \item Erfüllt $X\xrightarrow{f}Y\xrightarrow{g}Z$ $\mathbb{P}$, so auch
    $f$, falls:
    \begin{itemize}
    \item $f$ quasi-kompakt (d.h. $Y$ hat offene affine Überdeckung, deren
      Urbilder quasi-kompakt sind).
    \item $g$ separiert, im Falle II, III.
    \item (stetig im Fall IV)
    \end{itemize}
  \end{enumerate}
\end{prop}

Vergleiche: Lin, ,,Algebraic Geometry and Arithmetic curves``, Prop
24, 3.16, 3.32, (3.9).
\begin{proof}[Beweis (Skizze)]
\end{proof}
\begin{prop}[42]
  Sei $f:X\rightarrow Y$ surjektiver Morphismus von $S$-Schemata,
  und $Y$ separiert von endlichem Typ über $S$, sowie $X$ eigentlich
  über $S$. Dann ist $Y$ eigentlich über $S$.
\end{prop}

\begin{proof}
  $f$ surjektiv $\Longrightarrow\forall T\rightarrow S$ ist $f_{(T)}:X_{T}\rightarrow Y_{T}$
  surjektiv.

  $\Longrightarrow\xymatrix{A\underset{\text{abg.}}{\subset}Y_{T}\ar[r] & T\\
    & \varphi^{-1}(A)\underset{\text{abg.}}{\subset}X_{T}\ar@{->>}[ul]_{\varphi}\ar[u]_{X\text{ eig.}\Rightarrow\text{abg.}}
  }
  $

  $\Longrightarrow Y_{T}\rightarrow T$ abgeschlossen.

  $\Longrightarrow Y\rightarrow S$ universal abgeschlossen.
\end{proof}
\begin{prop}[43]
  Sei $X$ eigentliches Schema über $S=\Spec A$. Dann ist $\mathcal{O}_{X}(X)$
  ganz über $A$. Ist $X=\Spec B$ affin, so ist $B$ endlich über $A$.
\end{prop}

\begin{proof}
  Lin, 3.17/3.18.
\end{proof}
\begin{cor}
  Sei $X$ reduzierte eigentliche Varietät über $k$. Dann ist $\mathcal{O}_{X}(X)$
  endlich-dimensionaler $k$-Vektorraum.
\end{cor}

\begin{thm}[45]
  Sei $\mathcal{O}_{K}$ Bewertungsring, $K=\Quot(\mathcal{O}_{X})$,
  $X/\mathcal{O}_{K}$ eigentlich. Dann ist $X_{\mathcal{O}_{K}}(\mathcal{O}_{K})\rightarrow X_{K}(K)$
  bijektiv.
\end{thm}

\textbf{Bewertungskriterium }(vgl. Hartshorne, Lin 3.26)
\[
  \xymatrix{\Spec K\ar[r]\ar[d] & X\ar[d]^{f}\\
    \Spec\mathcal{O}_{K}\ar@{-->}[ur]|-{\exists!}\ar[r] & Y
  }
\]

$f$ eigentlich $\Longleftrightarrow$ universelle Eigenschaft oben
erfüllt. (Theorem auf $X\times_{Y}\Spec\mathcal{O}_{K}\rightarrow\Spec\mathcal{O}_{K}$).
\begin{thm}[46, Lin III 3.30]
  Jeder projektive Morphismus ist eigentlich.
\end{thm}

Zum Beispiel: abelsche Varietäten, etwa elliptische Kurven. Theorem
46 $\Longrightarrow E(\mathbb{Q}_{p})=E(\mathbb{Z}_{p})$.
