\section{Diagonal für Schemata}
\begin{prop}[27]
  Für affine $S$-Schemata
  \begin{align*}
    X & =\Spec(B)\overset{u}{\longrightarrow}S=\Spec(R)\\
    Y & =\Spec(A)\overset{v}{\longrightarrow}S
  \end{align*}

  und $S$-Morphismus $f=\Spec\varphi:X\rightarrow Y$ zu einem $R$-Algebra
  Morphismus $\varphi:A\rightarrow B$, entsprechen $\Delta_{X/S}$
  und $\Gamma_{f}$ den folgenden surjektiven Ringhomomorphismen:
  \[
    \begin{array}{rl}
      \Delta_{B/R}:B\otimes_{R}B & \longrightarrow B\\
      b\otimes b' & \longmapsto bb'
    \end{array},\qquad
    \begin{array}{rl}
      \Gamma_{\varphi}:A\otimes_{R}B & \longrightarrow B\\
      a\otimes b & \longmapsto\varphi(a)b
    \end{array}.
  \]
  
  Insbesondere sind $\Delta_{X/S}$, $\Gamma_{j}$ abgeschlossene Immersionen.
\end{prop}

Im allgemeinen sind $\Delta_{X/S}$, $\Gamma_{f}$ Immersionen (nicht
notwendig abgeschlossen!): Seien $Z,Z'\subset X$ Unterschemata. $\Longrightarrow Z\times_{S}Z'\subset X\times_{S}X$
Unterschemata (Immersionen und stabil unter Basiswechsel und Komposition),
und
\[
  Z\cap Z'=\Delta_{X/S}^{-1}(Z\times_{S}Z')\qquad(*)
\]

\begin{prop}[28]
  Seien $X,Y\in\schs$, $f,g:X\rightarrow Y$ $S$-Morphismen. Dann
  sind $\Delta_{X/S}$, $\Gamma_{f}$, $\ker(f,g)\rightarrow X$ Immersionen.
\end{prop}

\begin{proof}
  Es reicht zu zeigen: $\Delta_{X/S}$ ist eine Immersion (und $(2)$
  in Proposition 26, da ,,Immersion`` stabil ist unter Basiswechsel)
  lokal bzgl. Ziel. Sei also ohne Einschränkung $S$ affin. Falls $X=\bigcup_{i\in I}U_{i}$
  offene Überdeckung, dann ist $\Delta_{X/S}(X)=\bigcup_{i\in I}U_{i}\times_{S}U_{i}$
  offene Überdeckung. 
  \[
    \xymatrix@C=4pc{X\ar[r]_(0.3){\text{abg. Imm.}} & \bigcup_{i\in I}(U_{i}\times_{S}U_{i})\ar@{^{(}->}_(0.6){\text{off. Imm.}}[r] & X\times_{S}X}
  \]
  
  $(*)\Longrightarrow$ ohne Einschränkung, $X$ affin. Wende nun Proposition
  27 an.
\end{proof}
Das Unterschema
\begin{itemize}
\item $X\cong\Delta_{X/S}(X)\subset X\times_{S}X$ heißt die \textbf{Diagonale
    von $X\times_{S}X$}.
\item $\Gamma_{f}(X)\subset X\times_{S}Y$ heißt der \textbf{Graph von $f$}.
\end{itemize}

\begin{rem}[29]
  \mbox{}
  \begin{enumerate}
  \item Ein Unterschema $T\subset X\times_{S}Y$ ist der Graph eines $S$-Morphismus
    $f:X\rightarrow Y$ genau dann, wenn $p|_{T}:T\rightarrow X\times_{S}Y\underset{p}{\rightarrow}X$
    ein Isomorphismus ist, \emph{denn} $f=q\circ(p|_{T})^{-1}$.
  \item Im Allgemeinen ist die mengentheoretische Inklusion
    \[
      \Delta_{X/S}(X)\subset\{z\in X\times_{S}X\mid f(z)=g(z)\}
    \]
    \emph{keine} Gleichheit!
  \end{enumerate}
\end{rem}
