
\section{Die Zariski-Topologie}

\subsection{Definition 2}

Sei $M\subset k[T_{1},\ldots,T_{n}]\eqqcolon k[\underline{T}]$ eine
Teilmenge. Mit
\[
V(M)=\{(t_{1},\ldots,t_{n})\in k\mid f(t_{1},\ldots,t_{n})=0\ \boldsymbol{\forall f\in M}\}
\]

bezeichnen wir die gemeinsame \textbf{Nullstellen-(Verschwindungs-)Menge}\index{Nullstellen-Menge}
der Elemente aus $M$. (Manchmal auch $V(f_{i},i\in I)$ statt $V(\{f_{i},i\in I\})$.

\subsection{Eigenschaften}
\begin{itemize}
\item $V(M)=V(\mathfrak{A})$, wenn $\mathfrak{A}=\langle M\rangle$ das
\emph{von} $M$ \emph{erzeugte Ideal} in $k[I]$ bezeichnet.
\item Da $k[\underline{T}]$ noethersch (Hilbertscher Basissatz) ist, reichen
stets endlich viele $f_{1},\ldots,f_{n}\in M$:
\[
V(M)=V(f_{1},\ldots,f_{n})\qquad\text{falls }\mathfrak{A}=\langle f_{1},\ldots,f_{n}\rangle.
\]
\item $V(-)$ ist \textbf{inklusionsumkehrend}, $M'\subset M\Rightarrow V(M)\subseteq V(M')$.
\end{itemize}

\subsection{Satz 3}

Die Mengen $V(\mathfrak{A})$, $\mathfrak{A}\subset k[\underline{T}]$
ein Ideal, sind die \textbf{abgeschlossenen} Mengen einer Topologie
auf $k^{n}$, der sogenannten \textbf{Zariski-Topologie}\index{Zariski-Topologie}.
\begin{enumerate}
\item $\emptyset=V\left((1)\right)$, $k^{n}=V(0)$. 
\item $\bigcap_{i\in I}V(\mathfrak{A}_{i})=V\left(\sum_{i\in I}\mathfrak{A}_{i}\right)$
f�r beliebige Familien $(\mathfrak{A}_{i})$ von Idealen.
\item $V(\mathfrak{A})\cup V(\mathfrak{B})=V(\mathfrak{AB})$ f�r $\mathfrak{A},\mathfrak{B}\subset k[\underline{T}]$
Ideale.
\end{enumerate}

\subsubsection{Beweis (Satz 3)}

�bung / Algebra II. \-
