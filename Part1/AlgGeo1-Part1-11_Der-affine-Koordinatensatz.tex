
\paragraph*{Affine algebraische Mengen als R�ume von Funktionen}

\section{Der affine Koordinatensatz}

Sei $X\subseteq\mathbb{A}^{n}(k)$ abgeschlossen. F�r den surjektiven
(Def. von Morphismen) $k$-Algebren-Homomorphismus
\begin{align*}
k[I] & \xrightarrow{\varphi}\hom(X,\mathbb{A}^{1}(k))\\
f & \mapsto(x\mapsto f(x)),
\end{align*}

wobei die Morphismen in folgende Weise eine $k$-Algebra bilden:
\begin{align*}
(f+g)(x) & :=f(x)+g(x)\\
(fg)(x) & :=f(x)g(x)\\
(\alpha f)(x) & :=\alpha f(x)
\end{align*}

mit $f,g\in\hom(X,\mathbb{A}^{1}(k))$, $\alpha\in k$. Es gilt:
\[
\ker\varphi=I(X)
\]

\begin{defn}
$\Gamma(X):=k[I]/I(X)\cong\hom(X,\mathbb{A}^{1}(k))$ hei�t der \textbf{affine
Koordinatenring }von $X$.

F�r $x=(x_{1},\ldots,x_{n})\in X$ gilt:
\begin{align*}
\mathfrak{m}_{x}:=\  & \ker(\Gamma(X)\twoheadrightarrow k,\,f\mapsto f(x))\\
=\  & \{f\in\Gamma(X)\mid f(x)=0\}\\
=\  & \text{Bild von }(T_{1}-x_{1},\ldots,T_{n}-x_{n})\\
=\  & \ker(\Gamma(\mathbb{A}^{n}(k))\twoheadrightarrow k)
\end{align*}

unter der Projektion $\pi:k[\underline{T}]=\Gamma(\mathbb{A}^{n}(k))\twoheadrightarrow\Gamma(X)$.
Es ist $\mathfrak{m}_{x}$ ein maximales Ideal von $\Gamma(X)$ mit
$\Gamma(X)/\mathfrak{m}_{x}=k$. F�r ein Ideal $\mathfrak{A}\subset\Gamma(X)$
setze
\[
V(\mathfrak{A})=\{x\in X\mid f(x)=0\ \forall f\in\mathfrak{A}\}=V(\pi^{-1}(\mathfrak{A}))\cap X.
\]

Dies sind genau die abgeschlossenen Mengen von $X$ als Teilraum in
$\mathbb{A}^{n}(k)$ mit der induzierten Topologie, diese wird auch
\textbf{Zariski-Topologie} genannt. F�r $f\in\Gamma(X)$ setze:
\[
D(f):=\{x\in X\mid f(x)\neq0\}=X\backslash V(f).
\]
\end{defn}
\begin{lem}
Die offenen Mengen $D(f)$, $f\in\Gamma(X)$, bilden eine Basis der
Topologie, d.h.
\[
\forall U\subset X\text{ offen }\exists f_{i}\in\Gamma(X),\,i\in I,\quad\text{mit }U=\bigcup_{i\in I}D(f_{i})
\]
\end{lem}
\begin{proof}
$U=X\backslash V(\mathfrak{A})$ f�r ein $\mathfrak{A}\subset\Gamma(X)$,
$\mathfrak{A}=\langle f_{1},\ldots,f_{n}\rangle$ . Wegen
\[
V(\mathfrak{A})=\bigcap_{i=1}^{n}V(f_{i})\quad\Rightarrow\quad U=\bigcup_{i=1}^{n}D(f_{i})
\]

Es reichen also sogar endlich viele $f_{i}$! 
\end{proof}
\begin{prop}
Der Koordinatenring $\Gamma(X)$ einer affinen algebraischen Menge
$X$ ist eine endlich erzeugte $k$-Algebra, die reduziert ist (d.h.
keine nilpotenten Elemente $\neq0$ enth�lt). Ferner ist $X$ irreduzibel
genau dann, wenn $\Gamma(X)$ integer ist.
\end{prop}
\begin{proof}
$k[\underline{T}]\twoheadrightarrow\Gamma(X)$ impliziert ``endlich
erzeugte $k$-Algebra''. Es ist:
\[
\Gamma(X)\text{ irreduzibel }\Leftrightarrow I(X)=\rad I(X).
\]

Denn mit Satz 10.ii) und Korollar 11 folgt:
\begin{align*}
X & =V(\mathfrak{A}):\,I(X)=\rad\mathfrak{A}\\
\Rightarrow\rad I(X) & =\rad\rad\mathfrak{A}=\rad\mathfrak{A}=I(X).
\end{align*}

Mit Lemma 17 folgt: $X$ irreduzibel

$\phantom{\quad}\Leftrightarrow I(X)$ Primideal.

$\phantom{\quad}\Leftrightarrow\Gamma(X)=k[T]/T(X)$.
\end{proof}

