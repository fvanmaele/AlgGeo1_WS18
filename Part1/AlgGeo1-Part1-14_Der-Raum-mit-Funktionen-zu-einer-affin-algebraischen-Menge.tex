
\section{Der Raum mit Funktionen zu einer affin-algebraischen Menge}
\label{sec:alg-mengen-raeume-mit-fkt}

\textbf{Ziel.} Wir wollen jeder irreduziblen affin algebraischen Menge $X\subseteq\mathbb{A}^{n}(k)$ einen Raum mit Funktionen $(X,\mathcal{O}_{X})$ zuordnen.
D.h. wir müssen Mengen von Funktionen $\mathcal{O}_{X}(U) \leq \text{Abb}(U,k)$,
$U\subseteq X$ offen, definieren. Diese werden als Teilmengen des Funktionenkörpers
$K(X)$ definiert (dazu $X$ irreduzibel, später bei Schemata fällt
diese Bedingung weg!)
\begin{defn}
  \label{def:funktionenkoerper}
  Für eine irreduzible, affin-algebraische Menge $X$ heißt $K(X):=\text{Quot}(\Gamma(X))$ \textbf{Funktionenkörper} von $X$.

  Elemente $\frac{f}{g}\in K(X)$, $f,g\in\Gamma(X)=\hom(X,\mathbb{A}^{1}(k))$,
  $g\neq0$ lassen sich zumindest als Funktion auf der offenen Menge
  $D(g)\subseteq X$ auffassen, wenn auch i.A. nicht auf ganz
  $X$.
\end{defn}
\begin{lem}
  \label{lem:gleichheit-im-funktionenkoerper}
  Gilt für $\frac{f_{1}}{g_{1}},\frac{f_{2}}{g_{2}}\in K(X)$, $f_{i},g_{i}\in\Gamma(X)$,
  und einer offenen Teilmenge $\emptyset\neq U\subseteq D(g_{1}g_{2})$
  \[
    \frac{f_{1}(x)}{g_{1}(x)}=\frac{f_{2}(x)}{g_{2}(x)}\qquad\forall x\in U,
  \]

  dann folgt $\frac{f_{1}}{g_{1}}=\frac{f_{2}}{g_{2}}$ in $K(X)$.
\end{lem}
\begin{proof}
  Sei ohne Einschränkung der Allgemeinheit $g_{1}=g_{2}=g$. (Sonst
  Erweitern!) 

  $\Rightarrow(f_{1}-f_{2})(x)=0$ $\forall x\in U$.

  $\Rightarrow\emptyset\neq U\subseteq V(f_{1}-f_{2})\subseteq X$ dicht,
  d.h. $V(f_{1}-f_{2})=X$.

  $\phantom{\Rightarrow\ }$$f_{1}-f_{2}\in I()V(f_{1}-f_{2}))=I(X)\equiv(0)$
  in $\Gamma(X)$ 

  $\Rightarrow f_{1}-f_{2}=0$.
\end{proof}
\begin{defn}
  \label{def:alg-menge-als-raum-mit-fkt}
  Sei $X$ eine irreduzible affin-algebraische Menge, $U\subseteq X$
  offen. Für $x \in X$ bezeichne $\Gamma(X)_{\mathfrak{m}_{x}}$ die Lokalisierung von $\Gamma(X)$ an der multiplikativ abgeschlossenen Menge $S := \Gamma(X) \setminus \mathfrak{m}_{x}$.
  \[
    \mathcal{O}_{X}(U):=\bigcap_{x\in U}\Gamma(X)_{\mathfrak{m}_{x}}\subseteq K(X)
  \]

  d.h. für jedes $x\in U$ lässt sich $f\in\mathcal{O}_{X}(U)$ schreiben
  als $\frac{h}{g} \in K(X)$ mit $g(x)\neq0$.
\end{defn}
Für $f\in\Gamma(X)$ bezeichne $\Gamma(X)_{f}$ die Lokalisierung
von $\Gamma(X)$ an der multiplikativ abgeschlossenen Menge
$\{1,f,f^{2},\ldots,f^{n}\ldots\}$. Dann lässt sich
\[
  \Gamma(X)_{\mathfrak{m}_{x}}=\bigcup_{f\in\Gamma(X)\backslash\mathfrak{m}_{x}}\Gamma(X)_{f}\subseteq K(X)
\]

schreiben. ``$\supseteq$'': klar, ``$\subseteq$'': $\frac{g}{f}$
mit $f(x)\neq0$ d.h. $f\notin\mathfrak{m}_{x}$ $\Rightarrow\frac{g}{f}\in\Gamma(X)_{f}$.


\paragraph{Es gilt:}
\begin{enumerate}
\item Für $V\subseteq U\subseteq X$ offen kommutiert das folgende Diagramm:
  \[
    \xymatrix{\mathcal{O}_{X}(V)\ar@{^{(}->}[r] & \text{Abb}(V,k)\\
      \mathcal{O}_{X}(U)\ar@{^{(}->}[r]\ar@{^{(}->}[u] & \text{Abb}(U,k)\ar[u]_{\text{Einschränkungsabb.}}
    }
  \]
  mit $\mathcal{O}_{X}(U) \hookrightarrow \mathcal{O}_{X}(V), f \mapsto f|_V$ nach Definition.
\item $\mathcal{O}_{X}(U)\rightarrow\text{Abb}(U,k)$, $f\mapsto(x\mapsto f(x):=\frac{g(x)}{f(x)}\in k)$
  ist injektiv (Lemma 34) und wohldefiniert (kürzen/erweitern), wobei
  $g,h\in\Gamma(X)$ mit $h\notin\mathfrak{m}_{x}$ mit $f=\frac{g}{h}$
  nach Definition von $\mathcal{O}_{X}(U)$ existiert.
\item \textbf{Verklebungseigenschaft.} Sei $U=\bigcup_{i\in I}U_{i}$. Nach
  Definition ist 
  \begin{align*}
    \mathcal{O}_{X}(U) & =\bigcap_{i}\mathcal{O}_{X}(U_{i})\subseteq K(X)\\
    \ni f:U\rightarrow k & \quad\ni f_{i}:U_{i}\rightarrow k
  \end{align*}
  {[}Diagramm fehlt{]}. $(X,\mathcal{O}_{X})$ ist Raum
  mit Funktionen, \textbf{der zur irreduziblen affin algebraische Menge
    assoziierte Raum von Funktionen}. 
\end{enumerate}
\begin{prop}[orig. 33]
  \label{prop:fkt-auf-basis}
  Für $(X,\mathcal{O}_{X})$ zu $X$ wie oben und $f\in\Gamma(X)$
  gilt:
  \[
    \mathcal{O}_{X}(D(f))=\Gamma(X)_{f},
  \]

  insbesondere $\mathcal{O}_{X}(X)=\Gamma(X)$.
\end{prop}
\begin{proof}
  $\Gamma(X)_f\subseteq \mathcal{O}_{X}(D(f))$ klar, da $f(x)\neq0$ $\forall x\in D(f)$
  bzw. $f\in \Gamma(X)\setminus\mathfrak{m}_{x}$. 

  Sei nun $g$ in $\mathcal{O}_{X}(D(f))$ gegeben, $(*)$
  und $\mathfrak{a}:=\{h\in\Gamma(X)\mid hg\in\Gamma(X)\}\unlhd\Gamma(X)$.

  Dann gilt: $g\in\Gamma(X)_{f}$

  $\Leftrightarrow g=\frac{k}{f^{n}}$ für ein $n$ und $k\in\Gamma(X)$

  $\Leftrightarrow f^{n}\in\mathfrak{a}$ für ein $n$.

  d.h. zu zeigen: $f\in\text{rad}(\mathfrak{a})=I(V(\mathfrak{a}))$ (Hilbertscher
  Nullstellensatz)

  $\Leftrightarrow f(x)=0$ $\forall x\in V(\mathfrak{a})$

  Ist dazu $x\in X$ mit $f(x)\neq0$, also $x\in D(f)$, so
  existieren wegen $g \in \mathcal{O}_{X}(D(f))$ \\ Funktionen $f_{1},f_{2}\in\Gamma(X)$, $f_{2}\notin\mathfrak{m}_{x}$
  mit $g=\frac{f_{1}}{f_{2}}$, also gilt $f_{2}\in\mathfrak{a}$. 

  Da $f_{2}(x)\neq0$ folgt weiter $x\notin V(\mathfrak{a})$.
\end{proof}
\begin{rem}[orig. 34]
  \label{rem:globale-darstellung-von-fkt}
  \mbox{}
  \begin{enumerate}
  \item Im Allgemeinen existieren für $f\in\mathcal{O}_{x}(U)$ \textbf{nicht notwendigerweise}
    $g,h\in\Gamma(X)$ mit $f=\frac{g}{h}$ und $h(x)\neq0$ $\forall x\in U$.
  \item \textbf{Alternative Definition von $\mathcal{O}_{X}$, I.}
    \[
      \mathcal{O}_{X}(D(f)):=\Gamma(X)_{f},\quad\forall f\in\Gamma(X).
    \]
    Da $(D(f))_{f \in \Gamma(X)}$ Basis der Topologie bildet, kann es höchstens einen
    Raum mit Funktionen mit dieser Eigenschaft geben, es bleibt die Existenz
    zu zeigen.
  \item \textbf{Alternative Definition von $\mathcal{O}_{X}$, II.}

    Direkt von einer integeren endlich erzeugten $k$-Algebra $A$ ausgehend
    (die $X$ bis auf Isomorphie festlegt), aber ohne ``Koordinaten''
    zu wählen.
    \begin{align*}
      X & :=\{\mathfrak{m}\unlhd A\mid\ \mathfrak{m} \text{ ist max. Ideal}\}
    \end{align*}
    Die \textbf{abgeschlossenen Mengen} sind gegeben durch:
    \[
      V(\mathfrak{a}):=\{\mathfrak{m} \in X \mid\mathfrak{m}\supseteq\mathfrak{a}\},\quad\mathfrak{a}\unlhd A\text{ Ideal}.
    \]

    $\mathcal{O}_{X}(U):=\bigcap_{\mathfrak{m}\in U}A_{\mathfrak{m}}\subseteq\text{Quot}(A)$
    für $U\subseteq X$ offen (vgl. später Schemata).
  \end{enumerate}
\end{rem}

