
\section{Affine algebraische Mengen}
\label{sec:algebraische-mengen}
\begin{defn}
\label{def:algebraische-mengen}
\mbox{}
\begin{itemize}
\item $\mathbb{A}^{n}(k)$, der $\textbf{affine Raum der Dimension n}$ (�ber $k$),
bezeichne $k^{n}$ mit der Zariski-Topologie.
\item Abgeschlossene Teilmengen von $\mathbb{A}^{n}(k)$ hei�en affine abgeschlossene
Mengen.
\end{itemize}
\end{defn}
\begin{example}
\label{bsp:algebraische-mengen-dim1}
Da $k[T]$ ein Hauptidealring ist, sind die abgeschlossen Mengen in
$\mathbb{A}^{1}(k)$: $\emptyset$, $\mathbb{A}^{1}$, Mengen der
Form $V(f)$, $f\in k[T]\backslash\{k\}$ (endliche Teilmengen).%
\begin{comment}
F�r $f\in k$ ist $V(f)=\mathbb{A}^{1}$, denn die Einheiten im Polynomring
$k[T]$ sind gegeben durch $k^{\times}$, und Ideale erzeugt von einer
Einheit bilden den ganzen Ring. (siehe Algebra 1)
\end{comment}
{} Insbesondere sieht man, dass die Zariski-Topologie im Allgemeinen
nicht Hausdorff ist. 
\end{example}
%
\begin{example}
\label{bsp:algebraische-mengen-dim2}
$\mathbb{A}^{2}(k)$ hat zumindestens als abgeschlossene Mengen:
\begin{itemize}
\item $\emptyset$, $\mathbb{A}^{2}$;
\item Einpunktige Mengen: $\{(x_{1},x_{2})\}=V(T_{1}-x_{1},T_{2}-x_{2})$;
\item $V(f)$, $f\in k[T_{1},T_{2}]$ irreduzibel. 
\end{itemize}
Ferner alle endlichen Vereinigungen dieser Liste. (Dies sind in der
Tat alle, denn sp�ter sehen wir: ``irreduzible'' abgeschlossene
Mengen entsprechen den \emph{Primidealen}, und $k[T_{1},T_{2}]$ hat
``Krull-Dimension $2$''.)
\end{example}

