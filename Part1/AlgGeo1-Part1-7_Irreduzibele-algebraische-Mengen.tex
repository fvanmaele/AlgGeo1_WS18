
\section{Irreduzibele affine algebraische Mengen}

\subsection{Lemma 16}

Eine abgeschlossene Teilmenge $Z\subseteq\mathbb{A}^{n}(k)$ ist genau
dann irreduzibel, wenn $I(Z)$ ein Primideal ist. Insbesondere ist
$\mathbb{A}^{n}$ irreduzibel.  

\subsubsection{Beweis (Lemma 16)}

$Z$ irreduzibel $\Leftrightarrow(Z=\underbrace{V(\mathfrak{A})}_{\bigcap V(f_{i})}\cup\underbrace{V(\mathfrak{b})}_{\bigcap V(g_{j})}\Rightarrow V(\mathfrak{A})=Z$
oder $V(\mathfrak{b})=Z$) 

$\Leftrightarrow\forall f,g\in k[\underline{T}]$ ist $V(fg)=V(f)\cup V(g)\supseteq Z$
gilt $V(f)\supset Z$ oder $V(g)\supseteq Z$.

$\stackrel[I(V(\mathfrak{A}))=\rad(\mathfrak{A})]{V(I(Z))=Z}{\Leftrightarrow}\forall f,g\in k[\underline{T}]$
ist $fg\in I(V(fg)\subseteq I(Z)$ gilt $f\in I(Z)$ oder $g\in I(Z)$.

$\Leftrightarrow I(Z)$ ist Primideal.

\subsection{Bemerkung 17}

Die Korrespondenz aus Korollar 11 schr�nkt sich ein zu
\[
\{\text{irred. abg. Teilmengen }\subseteq\mathbb{A}^{n}\}\overset{1:1}{\leftrightarrow}\{\text{Primideale in }k[\underline{T}]\}
\]

