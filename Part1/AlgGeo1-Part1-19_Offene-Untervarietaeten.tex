
\section{Offene Untervariet�ten}

,,Offene Teilmengen von affinen Variet�ten (abgeschlossene beliebige
 Pr�variet�ten) sind wieder Pr�variet�ten`` (aber i.A. nicht affin!)
\begin{lem}[orig. 41]
Sei $X$ affine Variet�t, $f\in\mathcal{O}_{X}(X)$, $\mathcal{D}(f)\subseteq X$.
Die Lokalisierung von $\Gamma(X)=\mathcal{O}_{X}(X)$ in $f$,
\[
\Gamma(X)_{f}=\Gamma(X)[T]/(Tf-1)
\]

ist eine integere endlich erzeugte $k$-Algebra. Dabei $(Y,\mathcal{O}_{Y})$
bezeichnet die zugeh�rige Variet�t. Es folgt:
\[
(D(f),\mathcal{O}_{X|_{D(f)}})\cong(Y,\mathcal{O}_{Y})
\]

als R�ume mit Funktionen, d.h. $(D(f),\mathcal{O}_{X|_{D(f)}})$ ist
affine Variet�t.
\end{lem}
\begin{proof}
$\mathcal{O}_{X}(\mathcal{D}(f))=\mathcal{O}_{X}(X)_{f}$ muss affiner
Koordinatenring von $(\mathcal{D}(f),\mathcal{O}_{X|_{\mathcal{D}(f)}})$
sein, wenn letzterer Raum von Funktionen affin ist. $X\subseteq\mathbb{A}^{n}(k)$
korrespondiert zu dem Radikalideal:
\begin{align*}
\mathfrak{A} & =I(X)\subseteq k[T_{1},\ldots,T_{n}]\ \subset\ \mathfrak{A}'=(\mathfrak{A},fT_{n+1}-1)\subseteq k[T_{1},\ldots,T_{n+1}]
\end{align*}

mit Koordinatenringen:
\begin{align*}
\Gamma(X) & =k[T_{1},\ldots,T_{n}]/\mathfrak{A}\\
\Gamma(Y) & =\Gamma(X)_{f}=(k[T_{1},\ldots,T_{n}]/\mathfrak{A})[T_{n+1}]/(T_{n+1}f-1)\\
 & =k[T_{1},\ldots,T_{n+1}]/\mathfrak{A}'
\end{align*}

F�r $Y=V(\mathfrak{A}')\subseteq\mathbb{A}^{n+1}(k)$ induziert die
Abbildung
\[
\xymatrix{Y\subseteq\mathbb{A}^{n+1}(k)\ar@{-->}[d] & (x_{1},\ldots,x_{n+1})\ar@{|->}[d] & T_{i}\\
X\subseteq\mathbb{A}^{n}(k) & (x_{1},\ldots,x_{n}) & T_{i}\ar@{|->}[u]
}
\]

eine Bijketion $Y\xrightarrow{j}\mathcal{D}_{X}(f)$  mit Umkehrabbildung
\begin{claim*}
$j$ ist Isomorphismus von R�umen mit Funktionen:
\begin{enumerate}
\item $j$ ist \emph{stetig} (als Einschr�nkung stetiger Funktionen) $\checkmark$
\item $j$ ist \emph{offen}: $g\in\Gamma(X)$, $\Gamma(Y)=\Gamma(X)_{f}$,
$\frac{g}{f^{n}}\in\Gamma(X)_{f}$,
\begin{align*}
j\left(D_{Y}\left(\frac{g}{f^{n}}\right)\right) & =j\left(\mathcal{D}_{Y}(gf)\right) & f\text{ Einheit}\\
 & =\mathcal{D}_{X}(gf)\text{ offen}
\end{align*}

$\Rightarrow j$ Hom�morphismus.
\item $j$ induziert $\forall g\in\Gamma(X)$ Isomorphismen:
\begin{align*}
\mathcal{O}_{X}(\mathcal{D}(fg)) & \longrightarrow\Gamma(Y)_{g}\\
s & \longmapsto s\circ j
\end{align*}
\end{enumerate}
\end{claim*}
\end{proof}

