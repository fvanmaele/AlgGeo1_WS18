
\section{Der Hilbertsche Nullstellensatz}
\label{sec:nullstellensatz}
\begin{prop}
\label{prop:nullstellensatz}
Sei $K$ ein (nicht notwendigerweise algebraisch abgeschlossener) K�rper,
und $A$ eine endlich erzeugte $K$-Algebra. Dann ist $A$ Jacobson'sch,
d.h. f�r jedes Primideal $\mathfrak{p}\unlhd A$ gilt:
\[
\mathfrak{p}=\bigcap_{\mathfrak{m}\supseteq\mathfrak{p}}\mathfrak{m},\quad\mathfrak{m}\text{ maximales Ideal}
\]

Ist $\mathfrak{m}\unlhd A$ ein maximales Ideal, so ist die K�rpererweiterung
$K\subseteq A/\mathfrak{m}$ endlich.
\end{prop}
\begin{proof}
Algebra II / kommutative Algebra.
\end{proof}
\begin{cor}
\label{cor:nullstellensatz}
\mbox{}
\begin{enumerate}
\item Sei $A$ eine e.e. (endlich erzeugte) $k$-Algebra ($k$ sei algebraisch
abgeschlossen), $\mathfrak{m}\unlhd A$ ein maximales Ideal. Dann
ist $A/\mathfrak{m}=k$. 
\item Jedes maximale Ideal $\mathfrak{m}\unlhd k[\underline{T}]$ ist von der
Form $\mathfrak{m}=(T_{1}-x_{1},\ldots,T_{n}-x_{n})$ mit $x_{1},\ldots,x_{n}\in k$.
\item F�r ein Ideal  $\mathfrak{a}\unlhd k[\underline{T}]$ gilt:
\[
\rad(\mathfrak{a})=\sqrt{\mathfrak{a}}\overset{(i)}{=}\bigcap_{\mathfrak{a}\subseteq\mathfrak{p}\unlhd k[\underline{T}], \mathfrak{p} \text{prim}}\mathfrak{p}\overset{(ii)}{=}\bigcap_{\mathfrak{a}\subseteq\mathfrak{m}\unlhd k[\underline{T}], \mathfrak{m} \text{maximal}}\mathfrak{m}
\]
\end{enumerate}
\end{cor}
\begin{proof}
\mbox{}
\begin{enumerate}
\item $k\rightarrow A\rightarrow A/\mathfrak{m}$ ist Isomorphismus,  da
$k$ keine echte algebraische K�rpererweiterung besitzt.
\item Es ist
\begin{align*}
k[T_{1},\ldots,T_{n}] & \twoheadrightarrow k[\underline{T}]/\mathfrak{m}=k\\
T_{i} & \mapsto x_{i}
\end{align*}
surjektiv. Es folgt: $\mathfrak{m}=(T_{1}-x_{1},\ldots,T_{n}-x_{n})$, da letzteres
bereits maximal ist. ($\supseteq$ klar.)
\item (i) Algebra II. (ii) Theorem.
\end{enumerate}
\end{proof}

