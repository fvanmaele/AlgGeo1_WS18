
\section{Der Hilbertsche Nullstellensatz}

\subsection{Satz 7}

Sei $K$ ein (nicht notwendig algebraisch abgeschlossener) K�rper,
und $A$ eine endlich erzeugte $K$-Algebra. Dann ist $A$ Jacobson'sch,
d.h. f�r jedes Primideal $\mathfrak{p}\subset A$ gilt:
\[
\mathfrak{p}=\bigcap_{\mathfrak{m}\supseteq\mathfrak{p}}\mathfrak{m},\quad\mathfrak{m}\text{ max. Ideale}
\]

Ist $\mathfrak{m}\subset A$ ein maximales Ideal, so ist die K�rpererweiterung
$K\subset\faktor{A}{\mathfrak{m}}$ endlich.

\subsubsection{Beweis (Satz 7)}

Algebra II / kommutative Algebra.

\subsection{Korollar 8}
\begin{enumerate}
\item Sei $A$ eine e.e. (endlich erzeugte) $k$-Algebra ($k$ sei algebraisch
abgeschlossen), $\mathfrak{m}\subseteq A$ ein maximales Ideal. Dann
ist $\faktor{A}{\mathfrak{m}}=k$. 
\item Jedes maximale Ideal $\mathfrak{m}\subset k[\underline{T}]$ hat die
Form $\mathfrak{m}=(T_{1}-x_{1},\ldots,T_{n}-x_{n})$ mit $x_{1},\ldots,x_{n}\in k$.
\item F�r ein $k$-Ideal  $\mathfrak{A}\subset k[\underline{T}]$ gilt:
\[
\rad(\mathfrak{A)=\sqrt{\mathfrak{A}}}\overset{(i)}{=}\bigcap_{\mathfrak{A}\subseteq\mathfrak{p}\subseteq k[\underline{T}]}\mathfrak{p}=\bigcap_{\mathfrak{A}\subseteq\mathfrak{m}\text{ max.}\subseteq k[\underline{T}]}\mathfrak{m}
\]
\end{enumerate}

\subsubsection{Beweis (Korollar 8)}
\begin{enumerate}
\item $k\rightarrow A\rightarrow\faktor{A}{\mathfrak{m}}$ ist Isomorphismus,
 da $k$ keine echte algebraische K�rpererweiterung besteht.
\item Es ist
\begin{align*}
k[T_{1},\ldots,T_{n}] & \longrightarrow\faktor{T}{\mathfrak{m}}=k\\
T_{i} & \longmapsto x_{i}
\end{align*}
Es folgt: $\mathfrak{m}=(T_{1}-x_{1},\ldots,T_{n}-x_{n})$, da letztes
bereits maximal. ($\supseteq$ klar.)
\item (i) Algebra II. (ii) Theorem.
\end{enumerate}

