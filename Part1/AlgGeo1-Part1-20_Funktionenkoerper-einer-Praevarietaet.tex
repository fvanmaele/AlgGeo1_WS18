
\section{Funktionenk�rper einer Pr�variet�t}
\label{sec:funktionenkorper-praevarietaet}
\begin{defn}[orig. 43]
\label{def:funktionenkoerper-praevarietaet}
F�r eine Pr�variet�t $X$ sind die rationalen Funktionenk�rper aller
nicht-leeren affin-offenen Teilmengen in nat�rlicher Weise zu einander
isomorph. Diesen K�rper $K(X)$ nennen wir den \textbf{rationalen Funktionenk�rper}von $X$.
\end{defn}
\begin{proof}
$\emptyset\neq U$, $V\subseteq X$ affine, offene Untervariet�ten. Da
$X$ irreduzibel ist, gilt nach \emph{Satz \ref{prop:charakterisierung-irreduzibel}}:
\[
\emptyset\neq U\cap V\subseteq U\text{ offen}.
\]

Nach Definition von $\mathcal{O}_{X}$ ist 
\[
\mathcal{O}_{X}(U)\subseteq\mathcal{O}_{X}(U\cap V)\subseteq K(U)=\text{Quot}(\mathcal{O}_{X}(U)).
\]

Das impliziert $\text{Quot}(\mathcal{O}_{X}(U\cap V))=K(U)$. Aus
Symmetriegr�nden ist aber damit auch bereits $K(V)=\text{Quot}(\mathcal{O}_{X}(U\cap V))$.
\end{proof}
\begin{rem}[orig. 44]
\label{rem:funktionenkoerper-nicht-funktoriell}
Bildung des des Funktionenk�rpers $K(\cdot)$ ist \textbf{nicht} funktoriell!
F�r $X\rightarrow Y$ Morphismus affiner Variet�ten ist die Abbildung
auf den Koordinatenringen $\Gamma(Y)\rightarrow\Gamma(X)$ i.A. \textbf{nicht}
injektiv, induziert also keine Abbildung $K(Y)\hookrightarrow K(X)$.

\emph{Jedoch}: Eine Isomorphie $X\xrightarrow{\sim}Y$ induziert $K(Y)\xrightarrow{\sim}K(X)$.
Allgemeiner sei $X\xrightarrow{\varphi} Y$ Morphismus mit $\text{im}(\varphi) \subseteq Y$
offen ($\Rightarrow$ dicht. Sp�ter: $X\xrightarrow{\varphi} Y$ \textbf{dominant},
gdw. $\text{im}(\varphi)\subseteq Y$ dicht) induziert in funktioreller Weise eine
Abbildung $K(Y)\hookrightarrow K(X)$.
\end{rem}
\begin{prop}[orig. 45]
\label{prop:charakterisierung-schnitte-praevarietaet}
Sei $X$ eine Pr�variet�t, $V\subseteq U\subseteq X$ offen. Dann gilt:

\begin{enumerate}
\item $\mathcal{O}_{X}(U)\subseteq K(X)$ ist $k$-Unteralgebra.

\item $\mathcal{O}_{X}(U)\rightarrow\mathcal{O}_{X}(V)$ ist Inklusion von Teilmengen des Funktionenk�rpers $K(X)$.

\item Insbesondere gilt f�r $U,V\subseteq X$ offen:
\[
\mathcal{O}_{X}(U\cup V)=\mathcal{O}_{X}(U)\cap\mathcal{O}_{X}(V).
\]
\end{enumerate}
\end{prop}
\begin{proof}
\mbox{}
\item[(ii)] Sei $\mathcal{O}_{X}(X)\ni f:X\rightarrow k$. Dann ist $f^{-1}(0)\subseteq X$
abgeschlossen, da f�r $W\subseteq X$ affin-offen beliebig gilt, dass
\[
f^{-1}(0)\cap W=V(f|_{W}).
\]
Dazu macht man sich klar: ,,abgeschlossen`` ist eine lokale Eigenschaft,
affin-offene $W$ bilden eine Basis der Topologie. 

$\Rightarrow\mathcal{O}_{X}(U)\hookrightarrow\mathcal{O}_{X}(V)$, $f\mapsto f|_{V}$
ist injektiv f�r $\emptyset\neq V\subseteq U\subseteq X$ offen.

$\Rightarrow V\subseteq f^{-1}(0)$ 

$\Rightarrow f^{-1}(0)=U$ 

$\Rightarrow f\equiv0$.
\item[(i)] $U\supseteq W$ affin-offene Untervariet�t.
\[
\begin{tikzcd}
 \mathcal{O}_{X}(W) \arrow[r, hook] & K(W) \text{ } k\text{-Algebren} \\
 \mathcal{O}_{X}(U) \arrow[u, hook] \arrow[ru, dashed, hook]
\end{tikzcd}
\]
\item[(iii)] Wir haben folgendes kommutatives Diagramm:

\[
\begin{tikzcd}
 {} & \mathcal{O}_{X}(U) \arrow[rd, hook] \\
 \mathcal{O}_{X}(U \cup V) \arrow[ur, hook] \arrow[rd, hook'] & {} & \mathcal{O}_{X}(U \cap V) \\
 {} & \mathcal{O}_{X}(V) \arrow[ru, hook']
\end{tikzcd}
\]
Nach dem Verklebungsaxiom ist die Sequenz
\[
\begin{tikzcd}
  0 \arrow[r] & \mathcal{O}_{X}(U\cup V) \arrow[r] & \mathcal{O}_{X}(U) \times \mathcal{O}_{X}(V) \arrow[r] & \mathcal{O}_{X}(U \cap V) \\
  {} & f \arrow[r, mapsto] & (f|_{U}, f|_{V}) & {} \\
  {} & {} & (g,h) \arrow[r, mapsto] & g|_{U \cap V} - h|_{U \cap V}
\end{tikzcd}
\]
exakt.


\end{proof}

