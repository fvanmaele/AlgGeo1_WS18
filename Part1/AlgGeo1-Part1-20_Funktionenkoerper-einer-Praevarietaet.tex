
\section{Funktionenk�rper einer Pr�variet�t}
\begin{defn}[orig. 43]
F�r eine Pr�variet�t $X$ sind die rationalen Funktionenk�rper aller
nicht-leeren affinen offenen Teilmengen in nat�rlicher Weise zu einander
isomorph. Dieser K�rper nennen wir den \textbf{rationalen Funktionenk�rper
}von $X$: $K(X)$.
\end{defn}
\begin{proof}
$\emptyset\neq U$, $V\subset X$ affine offene Untervariet�t. Da
$X$ irreduzibel ist, gilt nach \emph{Satz 40}:
\[
\emptyset\neq U\cap V\subset U\text{ offen}.
\]

Nach Definition von $\mathcal{O}_{X}$ ist 
\[
\mathcal{O}_{X}(U)\subseteq\mathcal{O}_{X}(U\cap V)\subset K(U)=\text{Quot}(\mathcal{O}_{X}(U)).
\]

Das impliziert $\text{Quot}(\mathcal{O}_{X}(U\cap V))=K(U)$. Aus
Symmetriegr�nden ist aber $K(V)=\text{Quot}(\mathcal{O}_{X}(U\cap V))$.
\end{proof}
\begin{rem}[orig. 44]
Das Bild $K(\ )$ des Funktionenk�rpers ist \textbf{nicht} funktoriell!
F�r $X\xrightarrow{f}Y$ Morphismus affiner Variet�ten ist die Abbildung
auf den Koordinatenringen $\Gamma(Y)\rightarrow\Gamma(X)$ i.A. \textbf{nicht}
injektiv, also gibt es kein $K(Y)\hookrightarrow K(X)$.

\emph{Jedoch}: Eine Isomorphie $X\xrightarrow{\sim}Y$ induziert $K(Y)\xrightarrow{\sim}K(X)$.
Allgemeiner sei $X\rightarrow Y$ Morphismus mit Bild $\subset Y$
offen ($\Rightarrow$ dicht. Sp�ter $X\rightarrow Y$ \textbf{dominant},
d.h. Bild $\subset Y$ dicht.) induziert in funktioreller Weise eine
Abbildung $K(Y)\hookrightarrow K(X)$.
\end{rem}
\begin{prop}[orig. 45]
Sei $X$ eine Pr�variet�t, $V\subseteq U\subseteq X$ offen. Es folgt:

$\mathcal{O}_{X}(U)\subset K(X)$ $k$-Unteralgebra.

$\mathcal{O}(U)\rightarrow\mathcal{O}(V)$ ist Inklusion von Teilmengen
des Funktionenk�rpers $K(X)$.

Insbesondere gilt f�r $U,V\subset X$ offen:
\[
\mathcal{O}_{X}(U\cup V)=\mathcal{O}_{X}(U)\cap\mathcal{O}_{X}(V).
\]
\end{prop}
\begin{proof}
\mbox{}
\begin{enumerate}
\item[2.] Sei $\mathcal{O}(X)\ni f:X\rightarrow k$. Dann ist $f^{-1}(0)\subseteq X$
abgeschlossen, da f�r $W\subseteq X$ offen affin beliebig gilt:
\[
f^{-1}(0)\cap W=V(f_{|_{W}}).
\]
Dazu macht man sich klar: ,,abgeschlossen`` ist eine lokale Eigenschaft,
und die $W$ bilden eine Basis der Topologie. 

$\Rightarrow\mathcal{O}(U)\hookrightarrow\mathcal{O}(V)$, $f\mapsto\sigma$
injektiv f�r $\emptyset\neq V\subseteq U\subset X$ offen.

$\Rightarrow V\subset f^{-1}(0)$ 

$\Rightarrow f^{-1}(0)=U$ 

$\Rightarrow f\equiv0$.
\item[1.] (i) $U\supset W$ offen affine Variet�t. $\Rightarrow$
\[
\xymatrix{\mathcal{O}(W)\ar[r]\subset & K(W)\ k\text{-Variet�t}\\
\mathcal{O}(U)\ar@{^{(}->}[u]
}
\]
\item Verklebungsaxiom: 
\end{enumerate}
\end{proof}

