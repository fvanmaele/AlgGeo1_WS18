
\section{Irreduzibele topologische R�ume}

Die folgenden topologische Begriffe sind nur interessant, da $\mathbb{A}^{n}(k)$
($n>0$) kein Hausdorff'scher Raum ist.

\subsection{Definition 11}

Ein topologischer Raum $X$ hei�t \textbf{irreduzibel}\index{irreduzibel},
wenn $X\neq\emptyset$ und $X$ sich \emph{nicht} als Vereinigung
zweier echter abgeschlossenen Teilmengen darstellen l�sst, d.h
\[
X=A_{1}\cup A_{2},\ A_{i}\ \text{abg.}\quad\Rightarrow\quad A_{1}=X\text{ oder }A_{2}=X.
\]

$Z\subset X$ hei�t irreduzibel, falls $Z$ mit der induzierten Topologie
irreduzibel ist.

\subsection{Satz 12}

F�r einen topologischen Raum $X$ sind �quivalent:
\begin{enumerate}
\item $X$ ist irreduzibel.
\item Je zwei nichtleere offenen Teilmengen von $X$ haben nicht-leeren
Durchschnitt.
\item Jede nichtleere offene Teilmenge $U\subset X$ ist dicht in $X$.
\item Jede nichtleere offene Teilmenge $U\subset X$ ist zusammenh�ngend.
\item Jede nichtleere offene Teilmenge $U\subset X$ ist irreduzibel.
\end{enumerate}

\subsubsection{Beweis (Satz 12)}

$(i)\Leftrightarrow(ii)$ Komplementsmengen.

$(ii)\Leftrightarrow(iii)$ da: $(U\subset X$ dicht $\Leftrightarrow U\cap O\neq\emptyset$
f�r jedes offene $\emptyset\neq O\subset X$)

$(iii)\Rightarrow(iv)$ klar.

$(iv)\Rightarrow(iii)$. Sei $\emptyset\neq U$ offen und zusammenh�ngend.

$\Rightarrow U=U_{1}\sqcup U_{2}$

$\emptyset\neq U_{i}\underset{\text{offen}}{\subset}U\underset{\text{offen}}{\subset}X$

$\Rightarrow U_{1}\cap U_{2}=\emptyset$, Widerspruch zu (iii).

$(v)\Rightarrow(i)$ klar $(U=X)$.

$(iii)\Rightarrow(v)$. Sei $\emptyset\neq U\underset{\text{offen}}{\subset}X$.
Ist $\emptyset\neq V\underset{\text{offen}}{\subset}U\Rightarrow V\underset{\text{offen}}{\subseteq}X$

$\Rightarrow V$ dicht in $X$, irreduzibel  in $U$

$\overset{(iii)\Rightarrow(i)}{\Rightarrow}U$ irreduzibel.

\subsection{Lemma 13}

Eine Teilmenge $Y$ ist genau dann irreduzibel, wenn ihr Abschluss
$\overline{Y}$ dies ist.

\subsubsection{Beweis (Lemma 13)}

$Y$ irreduzibel $\Leftrightarrow\forall U,V\subset X$ offen mit
$U\cap Y\neq\emptyset\neq V\cap Y$ gilt $Y\cap(U\cap V)\neq\emptyset$

$\Leftrightarrow\overline{Y}$ irreduzibel.

\subsection{Definition 14}

Eine maximale irreduzibele Teilmenge eines topologischen Raumes $X$
hei�t \textbf{irreduzibele Komponente}\index{irreduzibele Komponente}
von $X$.

\subsection{Bemerkung 15}
\begin{enumerate}
\item Jede irreduzibele Komponente ist abgeschlossen nach Lemma 14.
\item $X$ ist Vereinigung seiner irreduzibelen Komponenten, \emph{denn}: 

die Menge der irreduzibelen Teilmengen von $X$ ist \textbf{induktiv
geordnet}: f�r jede aufsteigende Kette irreduzibeler Teilmengen ist
die Vereinigung wieder irreduzibel. (Satz 13 (ii)). Mit dem \textbf{Lemma
von Zorn} folgt: Jede irreduzibele Teilmenge ist in einer irreduzibelen
Komponente enthalten. Damit ist jeder Punkt in einer irreduzibelen
Komponente enthalten.
\end{enumerate}

