
\section{Die Zariski-Topologie}
\label{sec:zariski-topologie}

\begin{defn}
\label{def:verschwindungsmenge}
Sei $M\subseteq k[T_{1},\ldots,T_{n}]=:k[\underline{T}]$ eine Teilmenge.
Mit
\[
V(M) :=\{(t_{1},\ldots,t_{n})\in k^n \mid f(t_{1},\ldots,t_{n})=0\ \boldsymbol{\forall f\in M}\}
\]

bezeichnen wir die gemeinsame \textbf{Nullstellen-(Verschwindungs-)Menge}\index{Nullstellen-Menge}
der Elemente aus $M$. (Manchmal auch $V(f_{i},i\in I)$ statt $V(\{f_{i},i\in I\})$.
\end{defn}

%% TODO:
%% das oben weg!!!
\paragraph{Notation}
Wir schreiben auch $V(f_i, i \in I)$ statt $V(\{f_i \mid i \in I\})$

\subsection{Eigenschaften}
\label{subsec:zariski-topologie-eigenschaften}
\begin{itemize}
\item $V(M)=V(\mathfrak{a})$, wenn $\mathfrak{a}=\langle M\rangle_{k[\underline{T}]}$ das
\emph{von} $M$ \emph{erzeugte Ideal} in $k[\underline{T}]$ bezeichnet.
\item Da $k[\underline{T}]$ noethersch (Hilbertscher Basissatz) ist, reichen
stets endlich viele $f_{1},\ldots,f_{n}\in M$:
\[
V(M)=V(f_{1},\ldots,f_{n})\qquad\text{falls }\mathfrak{a}=\langle f_{1},\ldots,f_{n}\rangle_{k[\underline{T}]}.
\]
\item $V(-)$ ist \textbf{inklusionsumkehrend}, $M'\subseteq M\implies V(M)\subseteq V(M')$.
\end{itemize}
\begin{prop}
\label{propdef:zariski-topologie}
Die Mengen $V(\mathfrak{a})$, $\mathfrak{a} \unlhd k[\underline{T}]$
ein Ideal, sind die \textbf{abgeschlossenen} Mengen einer Topologie
auf $k^{n}$, der sogenannten \textbf{Zariski-Topologie}\index{Zariski-Topologie}.
\begin{enumerate}
\item $\emptyset=V\left((1)\right)$, $k^{n}=V(0)$. 
\item $\bigcap_{i\in I}V(\mathfrak{a}_{i})=V\left(\sum_{i\in I}\mathfrak{a}_{i}\right)$
f�r beliebige Familien $(\mathfrak{a}_{i})_{i \in I}$ von Idealen.
\item $V(\mathfrak{a})\cup V(\mathfrak{a})=V(\mathfrak{ab})$ f�r $\mathfrak{a},\mathfrak{b}\unlhd k[\underline{T}]$
Ideale.
\end{enumerate}
\end{prop}
\begin{proof}
�bung / Algebra II. 

\-
\end{proof}

