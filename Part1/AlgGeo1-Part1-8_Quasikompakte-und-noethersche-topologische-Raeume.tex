
\section{Quasikompakte und noethersche topologische R�ume}
\label{sec:quasikompakt-noethersch}
\begin{defn}
\label{def:quasikompakt/noethersch}
Ein topologischer Raum $X$ hei�t \textbf{quasikompakt}\index{quasikompakt},
falls jede offene �berdeckung von $X$ eine \emph{endliche} Teil�berdeckung
enth�lt. (,,quasi`` deutet an, dass $X$ in der Regel nicht Hausdorff'sch
ist!). Er hei�t \textbf{noethersch}\index{noethersch}, wenn jede
absteigende Kette
\[
X\supseteq Z_{1}\supseteq Z_{2}\supseteq\cdots
\]

abgeschlossener Teilmengen von $X$ station�r wird ($\Leftrightarrow$
jede aufsteigende Kette offener Teilmengen wird station�r).
\end{defn}
\begin{lem}
\label{lem:eigenschaften-noethersch}
Sei $X$ ein noetherscher topologischer Raum. Dann gilt:
\begin{enumerate}
\item Jede abgeschlossene Teilmenge $Z \subseteq X$ ist noethersch.
\item Jede offene Teilmenge $U \subseteq X$ ist quasikompakt.
\item Jeder abgeschlossene Teilraum $Z \subseteq X$ besitzt nur endlich viele
irreduzible Komponenten.
\end{enumerate}
\end{lem}
\begin{proof}
\mbox{}
\begin{enumerate}
\item Nach Definition, da abgeschlossene Mengen von $Z$ auch solche von
$X$ sind.
\item $U=\bigcup_{i\in I}U_{i}$ offen; Angenommen $U$ w�re nicht quasikompakt.
Dann gibt es eine Folge $I_{1}\subseteq I_{2}\subseteq\cdots\subseteq I$ von Teilmengen
mit
\[
V_{1}\subsetneq V_{2}\subsetneq\cdots\neq U\quad\text{f�r }V_{j}=\bigcup_{i\in I_{j}}U_{i}.
\]
Widerspruch zu noethersch.
\item Es reicht zu zeigen: Jeder noethersche Raum ist Vereinigung endlich
vieler irreduzibler Teilmengen. Da $X$ noethersch ist, folgt mit
dem \emph{Lemma von Zorn} dass jede nichtleere Menge von algebraischen
Teilmengen in $X$ ein minimales Element besitzt. 
\[
\text{Angenommen:} \mathcal{M}:=\left\{ Z\subseteq X\text{ abg.}\mid Z\text{ ist \textbf{nicht} endl. Vereinigung irred. Mengen}\right\} \text{ w�re nichtleer.}
\]
$\Rightarrow\exists$ minimales Element, sagen wir $Z$, in $\mathcal{M}$.

$\Rightarrow Z$ ist nicht irreduzibel.

$\Rightarrow Z=Z_{1}\cup Z_{2}$ mit $Z_{1},Z_{2}\subsetneq Z$ abgeschlossen.

$\Rightarrow$ ($Z$ minimal) $Z_{1},Z_{2}\notin\mathcal{M}$

$\Rightarrow Z\notin\mathcal{M}$. Widerspruch.

\end{enumerate}
\end{proof}
\begin{prop}
\label{prop:algebraische-mengen-noethersch}
Jeder abgeschlossene Teilraum $X\subseteq\mathbb{A}^{n}(k)$ ist noethersch.
\end{prop}
\begin{proof}
Nach dem obigen Lemma ist nur zu zeigen, dass $\mathbb{A}^{n}(k)$
noethersch ist.

Absteigende Ketten abgeschlossener Teilmengen sind nach \emph{Korollar
11} in 1-1 Korrespondenz mit aufsteigenden Ketten von (Radikal-)Idealen
in $k[\underline{T}]$. Da $k[\underline{T}]$ nach dem Hilbertschen
Basissatz noethersch ist, werden letzere Ketten station�r.
\end{proof}
\begin{cor}[Prim�rzerlegung]
\label{cor:primaerzerlegung}
Sei $\mathfrak{a}=\rad(\mathfrak{a})\unlhd k[\underline{T}]$
ein Radikalideal. Dann gilt: $\mathfrak{a}$ ist Durchschnitt von
endlich vielen Primidealen, die sich jeweils paarweise nicht enthalten; diese
Darstellung ist eindeutig bis auf Reihenfolge.
\end{cor}
\begin{proof}
$V(\mathfrak{a})=\bigcup_{i=1}^{n}V(\mathfrak{b}_{i})$, $\mathfrak{b}_{i}$
Primideal.%
\begin{comment}
Station�re Kette folgt aus noethersch (Satz 21); mit Bemerkung 16
bzw. Lemma 17 folgt, dass die $\mathfrak{b}_{i}$ Primideale sind.
\end{comment}
{} \textcolor{blue}{Mit Satz 10 folgt:}
\[
\mathfrak{a}=\rad(\mathfrak{a})=I(V(\mathfrak{a}))=\bigcap_{i=1}^{n}\underbrace{I(V(\mathfrak{b}_{i}))}_{\mathfrak{b}_{i}\text{ minimale Primideale (\ref{lem:charakterisierung-irreduzibel-alg})}}
\]
\end{proof}

