\section{Quadriken}
\label{sec:quadriken}

Sei char$(k)\neq2$ in diesem Abschnitt.
\begin{defn}[orig. 57]
  \label{def:quadrik}
  Eine abgeschlossene Unterprävarietät $Q\subseteq\mathbb{P}^{n}(k)$
  von der Form $V_{+}(q)$, $q\in k[X_{0},\ldots,X_{n}]_{2}\backslash\{0\}$
  heißt \textbf{Quadrik}.
  \[
    Q=V_{+}(q)
  \]

  Zur quadratischen Form $q$ gehört eine Bilinearform $\beta$ auf
  $k^{n+1}$, 
  \[
    \beta(v,w):=\frac{1}{2}(q(v+w)-q(v)-q(w)),\quad v,w\in k^{n+1}
  \]

Es gibt eine Basis von $k^{n+1}$, sodass die Strukturmatrix $B$
von $\beta$ die Gestalt
\[
  B=\begin{pmatrix}
    \begin{array}{ccc}
      1\\
      & \ddots\\
      &  & 1
    \end{array} & 0\\
    0 &
    \begin{array}{ccc}
      0\\
      & \ddots\\
      &  & 0
    \end{array}
  \end{pmatrix}
\]

hat, d.h. Koordinatenwechsel zur Basiswechselmatrix liefert einen
Isomorphismus
\[
  Q\xrightarrow{\cong}V_{+}(X_{0}^{2}+\cdots+X_{r-1}^{2}),\quad r=\text{rg }B
\]
\end{defn}
\begin{lem}[orig. 58]
  \label{lem:platzhalter}
\end{lem}
\begin{prop}[orig. 59]
  \label{prop:quadrik-in-normalform}
  Ist $r\neq s$, so sind $V_{+}(T_{0}^{2}+\cdots+T_{r-1}^{2})$ und
  $V_{+}(T_{0}^{2}+\cdots+T_{s-1}^{2})$ nicht isomorph.
\end{prop}
\begin{proof}
  (später: Es gibt keinen Koordinatenwechsel von $\mathbb{P}^{n}(k)$,
  der beide Mengen identifiziert, damit auch kein Automorphismus in
  $\mathbb{P}^{n}(k)$.)
\end{proof}

\begin{defn}
  \label{def:dim-und-rang-einer-quadrik}
  Eine Quadrik $Q\subseteq\mathbb{P}^{n}(k)$ mit $Q\cong V_{+}(T_{0}^{2}+\cdots+T_{r-1}^{2})$,
  $r\geq1$, habe die \textbf{Dimension $n-1$} und den \textbf{Rang $r$}. (nach Satz eindeutig!)
\end{defn}

