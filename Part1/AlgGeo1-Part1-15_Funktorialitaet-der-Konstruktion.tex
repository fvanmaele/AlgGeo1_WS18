
\section{Funktorialit�t der Konstruktion}
\label{sec:funktorialit�t-affine-varietaet}
\begin{prop}[orig. 35]\label{prop:charakterisierung-morphismen-alg-mengen}
Sei $f:X\rightarrow Y$ eine stetige Abbildung zwischen irreduziblen
affin-algebraischen Mengen. Es sind �quivalent:
\begin{enumerate}
\item $f$ ist ein Morphismus affin-algebraischer Mengen.
\item $\forall g\in\Gamma(Y)$ gilt $g\circ f\in\Gamma(X)$.
\item $f$ ist ein Morphismus von R�umen von Funktionen, d.h. f�r alle $U\subseteq Y$
offen und alle $g\in\mathcal{O}_{Y}(U)$ gilt $g\circ f\in\mathcal{O}_{X}(f^{-1}(U))$.
\end{enumerate}
\end{prop}
\begin{proof}
\mbox{}
\begin{itemize}
\item $(i)\Leftrightarrow(ii)$ 

Folgt aus Satz $\ref{prop:koordinatenringfunktor}$.
\item $(iii)\Rightarrow(ii)$ 

$U:=Y$ und Satz $\ref{prop:fkt-auf-basis}$.
\item $(ii)\Rightarrow(iii)$

Betrachte $\Gamma(f):\Gamma(Y)\rightarrow\Gamma(X)$, $h\mapsto h\circ f$.
Aufgrund des Verklebungsaxioms reicht es, die Bedingung f�r $U$ von
der Form $D(g)$ zu zeigen; hier gilt:
\[
f^{-1}(D(g))=\{x\in X\mid\underbrace{g(f(x))}_{=\Gamma(f)(g)(x)}\neq0\}=D(g \circ f)
\]
Deswegen induziert $\Gamma(f)$:
\begin{align*}
h & \longmapsto h\circ f\\
\mathcal{O}_{Y}(D(g)) & \longrightarrow\mathcal{O}_{X}(D(g\circ f))\\
 & \shortparallel\\
\Gamma(Y)_{g} & \longrightarrow\Gamma(X)_{g \circ f}\\
\frac{h}{g^{n}} & \longmapsto\frac{h\circ f}{(g\circ f)^{n}}
\end{align*}
mit $h\circ f, g\circ f\in\Gamma(X)$ nach Voraussetzung.

\end{itemize}
\end{proof}
Insgesamt erhalten wir:
\begin{thm}[orig. 36]
\label{thm:aequivalenz-alg-mengen-aff-varietaeten}
Die obige Konstruktion definiert einen volltreuen Funktor
\[
\text{\{irreduzible aff. alg. Mengen �ber }k\}\rightarrow\{\text{R�ume mit Funktionen �ber }k\}.
\]
\end{thm}

