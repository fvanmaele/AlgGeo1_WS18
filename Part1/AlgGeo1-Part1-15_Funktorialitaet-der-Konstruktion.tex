\selectlanguage{english}%

\section{Funktorialit�t der Konstruktion}

\subsection*{Satz 38 (orig. 35)}

Sei $f:X\rightarrow Y$ eine stetige Abbildung zwischen irreduzibler
affiner algebraischen Mengen. Es sind �quivalent:
\begin{enumerate}
\item $f$ ist ein Morphismus affiner algebraischen Mengen.
\item $\forall g\in\Gamma(Y)$ gilt $g\circ f\in\Gamma(X)$.
\item $f$ ist ein von R�umen von Funktionen, d.h. f�r alle $U\subseteq Y$
offen und alle $g\in\mathcal{O}_{Y}(U)$ gilt $g\circ f\in\mathcal{O}_{X}(f^{-1}(U))$.
\end{enumerate}

\subsection*{Beweis (Satz 38)}

$(i)\Leftrightarrow(ii)$ Satz 29.

$(iii)\Rightarrow(ii)$ $U:=Y$ + Satz 33.

$(ii)\Rightarrow(iii)$ Betrachte $\varphi:\Gamma(Y)\rightarrow\Gamma(X)$,
$h\mapsto h\circ f$. Aufgrund des Verklebungsaxioms reicht es, die
Bedingung f�r $U$ von der Form $\mathcal{D}(g)$ zu zeigen: Es gilt
\[
f^{-1}(\mathcal{D}(g))=\{x\in X\mid\underbrace{g(f(x))}_{=\varphi(g)(x)}\neq0\}=\mathcal{D}(\varphi(g))
\]
Deswegen induziert $\varphi$
\begin{align*}
H & \longmapsto H\circ f\\
\mathcal{O}_{Y}(\mathcal{D}(g)) & \phantom{\longrightarrow}\mathcal{O}_{X}(D(\varphi(g)))\\
=\Gamma(Y)_{g} & \longrightarrow=\Gamma(X)_{\varphi(g)}\\
\frac{h}{g} & \longmapsto\frac{h\circ f}{(g\circ f)^{n}}
\end{align*}
mit $h\circ f\in\Gamma(X)$ nach Voraussetzung und $\varphi(g)=g\circ f\in\Gamma(X)$
nach Voraussetzung.

Insgesamt haben wir:

\subsection*{Theorem 39 (orig. 36)}

Die obige Konstruktion definiert einen volltreuen Funktor
\[
\text{\{irred. aff. abg. Mengen �ber }k\}\rightarrow\{\text{R�ume mit Funktionen �ber }k\}
\]
\selectlanguage{ngerman}%

