
\section{Koordinatenwechsel in $\mathbb{P}^{n}$}
\label{sec:koordinatenwechsel-projektiver-raum}

$A=(a_{ij})\in GL_{n+1}(k)$ eine invertierbare $k^{n+1}\rightarrow k^{n+1}$
lineare Abbildung, die Ursprungsgeraden in solche �berf�hrt, bzw.
die �quivalenzrelation respektiert. Wir erhalten Abbildungen:
\begin{align*}
  \mathbb{P}^{n}(k) & \overset{\phi_{A}}{\longrightarrow}\mathbb{P}^{n}(k)\\
  (x_{0}:\ldots:x_{n}) & \longmapsto\left(\sum_{i=0}^{n}a_{0_{i}}x_{i}:\cdots:\sum_{i=0}^{n}a_{n_{i}}x_{i}\right),
\end{align*}

die nach Satz 56 ein Morphismus von Pr�variet�ten ist. Offensichtlich
gilt f�r $A,B\in GL_{n+1}(k)$:
\[
  \varphi_{A\cdot B}=\varphi_{A}\circ\varphi_{B}
\]

d.h. $\varphi_{A}$ ist insbesondere wieder ein Isomorphismus, \textbf{der
  durch $A$ bestimmte Koordinatenwechsel des $\mathbb{P}^{n}(k)$}.
\emph{Bezeichne} Aut$(\mathbb{P}^{n}(k))$ die Gruppe der Automorphismen
von $\mathbb{P}^{n}(k)$. Es folgt:
\[
  \varphi_{-}:GL_{n+1}(k)\rightarrow\text{Aut}(\mathbb{P}^{n}(k))
\]

ist ein Gruppenhomomorphismus mit 
\[
  Z:=\ker\varphi=\{\lambda E_{n+1},\ \lambda\in k^{\times}\}
\]

die Untergruppe der Skalarmatrizen. \emph{Sp�ter}:
\[
  PGL_{n+1}(k):=GL_{n+1}(k)/Z\twoheadrightarrow\text{Aut}(\mathbb{P}^{n}(k)),\quad Z\cong k^{\times}
\]

die \textbf{projektive lineare Gruppe}.
\begin{example*}
Sei $n=1$. Es ist
\begin{align*}
PGL_{2}(\mathbb{C}) & =\left\{ \begin{array}{rl}
\mathbb{P}^{1}(\mathbb{C}) & \rightarrow\mathbb{P}^{1}(\mathbb{C})\\
(z:w) & \mapsto(az+bw,cz+dw)
\end{array}\right\} \\
 & \leftrightarrow\text{M�biustransformationen }z\mapsto\frac{az+b}{cz+d}
\end{align*}
\end{example*}

