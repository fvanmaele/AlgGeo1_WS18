
\section{Topologische Eigenschaften von Pr�variet�ten}\label{sec:topologische-eigenschaften-von-praevarietaeten}
\begin{lem}\label{lem:bijektion-irred-teilraeume}
F�r einen topologischen Raum $X$ und $U\subseteq X$ offen haben
wir eine Bijektion
\begin{align*}
\{Y\subseteq U\text{ irred. abg.}\} & \longleftrightarrow\{Z\subseteq X\text{ irred. abg. mit }Z\cap U\neq\emptyset\}\\
Y & \longmapsto\overline{Y}\text{ (Abschluss in }X)\\
Z\cap U & \longmapsfrom Z
\end{align*}
\end{lem}
\begin{proof}
Lemma \ref{lem:irreduzibel-abschluss}: $Y\subseteq X$ irreduzibel $\Leftrightarrow\overline{Y}\subseteq X$
irreduzibel.

$Y\subseteq U$ abgeschlossen $\Leftrightarrow\exists A\subseteq X$
abgeschlossen: $Y=U\cap A$.

$\Rightarrow Y\subseteq\overline{Y}\subseteq A$ $\Rightarrow Y=U\cap\overline{Y}$

$Y$ irreduzibel in $U$ $\Rightarrow Y$ irreduzibel in $X$

$\Rightarrow$ $\overline{Y}$ irreduzibel nach \ref{lem:irreduzibel-abschluss}

$\Rightarrow Y\mapsto\overline{Y}\mapsto\overline{Y}\cap U=Y$. $\checkmark$

$\emptyset\neq Z\cap U \subseteq Z$
damit dicht da $Z$ irreduzibel (Satz \ref{prop:charakterisierung-irreduzibel} ii. und v.)

Also ist die Abbildung $\leftarrow$ wohldefiniert.

$\Rightarrow\overline{Z\cap U}=Z$ 
\end{proof}
\begin{prop}\label{prop:praevarietaeten-noethersch-irreduzibel}
Sei $(X,\mathcal{O}_{X})$ eine Pr�variet�t.

Dann ist $X$ noethersch (insbesondere quasikompakt) und irreduzibel.
\end{prop}
\begin{proof}
Sei $X=\bigcup_{i=1}^{n}$ endliche offene aff. �berdeckung und $X\supseteq Z_{1}\supseteq Z_{2}\supseteq\cdots$
eine absteigende Kette abgeschlossener Teilmengen.

$\Rightarrow U_{i}\cap Z_{1}\supseteq U_{i}\cap Z_{2}\supseteq\cdots$
, ist eine absteigende Kette abgeschlossener Teilmengen von $U_{i}$

$\Rightarrow\forall i$ $\exists n_{i} \in \mathbb{N}$: $U_{i}\cap Z_{n_{i}}=U_{i}\cap Z_{i+m}$ f�r alle $m \in \mathbb{N}$.
 Setzen wir $n:=\max n_{i}$, so folgt:

$\forall i=1,\ldots,n$ $\forall m\geq n$: $U_{i}\cap Z_{m}=U_{i}\cap Z_{m+1}$

$\Rightarrow(Z_{i})_{i}$ wird station�r da $Z_{m}=\bigcup_{i} U_{i}\cap Z_{m}$.

$X$ ist demnach noethersch.

$X$ ist weiter irreduzibel:

Sei $X=X_{1}\cup\cdots\cup X_{n}$ die Zerlegung in irreduzible Komponenten.

Angenommen es w�re $n\geq2$.

$\Rightarrow\exists i_{0}\in\{2,\ldots,n\}$: $X_{1}\cap X_{i_{0}}\neq\emptyset$.
(Andernfalls gilt: $X=X_{1}\sqcup\underbrace{X\backslash X_{1}}_{=X_{2}\cup\cdots\cup X_{n}\text{ abg.}}$, im Widerspruch dazu, dass $X$ zusammenh�ngend ist.)

Sei ohne Einschr�nkung $i_{0}=2$. Sei $x\in X_{1}\cap X_{2}$, $x\in U\subseteq X$ offen, affin (d.h. affine Variet�t).

$U$ irreduzibel $\Rightarrow\overline{U}$ (Abschluss in $X$) $\subseteq X_{j}$
f�r ein $j\in\{1,\ldots,n\}$

\textbf{Jedoch}: Da $x\in X_{i}\cap U\subseteq U$ irreduzibel ist, ist $\underbrace{\overline{X_{i}\cap U}}_{\subseteq\overline{U}\subseteq X_{i}}=X_{i}$,
$i=1,2$

$\Rightarrow X_{1},X_{2}\subseteq X_{j}$. Widerspruch zu maximale
Komponente.
\end{proof}

