Ein Morphismus von Pr�variet�ten ist ein Morphismus der entsprechenden
R�ume mit Funktionen. Insbesondere sind also affine Variet�ten Pr�variet�ten!

Sp�ter: Variet�t = ,,separierte Pr�variet�t``

Affine Variet�ten sind stets ,,separiert``, daf�r braucht man wieder
von ,,affinen Pr�variet�ten`` zu reden.

Ist $X$ eine affine Variet�t, schreiben wir oft $\Gamma(X)$ f�r
$\mathcal{O}_{x}(X)$ (vgl. Satz 33).

Unter einer \textbf{offenen affinen �berdeckung} einer Pr�variet�t
$X$ verstehen wir eine Famile von affinen Unterr�umen mit Funktionen
$U_{i}\subseteq X$, $i\in I$ die affine Variet�ten sind, und so
das $X=\bigcup U_{i}$.

\section{Vergleich mit differenzierbaren/komplexen Mannigfaltigkeiten}

\paragraph{Differential/Komplexe Geometrie}

Mannigfaltigkeiten werden via Kartenabbildungen mit differenzierbaren/holomorphen
�bergangsabbildungen definiert (hier problematisch, da offene Teile
affiner algebraischer Mengen i.A. keine solche Struktur wider besitzen.)

Jedoch:
\begin{align*}
\text{\{diff. Mfgkt.\}} & \longrightarrow\text{\{R�ume mit Fkt.}/\mathbb{R}\}\\
X & \longmapsto(X,\mathcal{O}_{X})\\
 & \phantom{\longmapsto}\mathcal{O}_{X}(U):=C^{\infty}(U,\mathbb{R}),\ U\subseteq X\text{ offen}
\end{align*}

ist ein volltreuer Funktor.

Daher kann man differenzierbare Mannigfaltigkeiten auch als diejenigen
R�ume mit Funktionen �ber $\mathbb{R}$ definieren, f�r die $X$ Hausdorff
ist, und so dass eine offene �berdeckung durch solche R�ume mit Funktionen
�ber $\mathbb{R}$ existiert, die in obiger Weise offene Teilmengen
von $\mathbb{R}^{n}$ zugeordnet sind. (Analog bei komplexen Mannigfaltigkeiten.)
