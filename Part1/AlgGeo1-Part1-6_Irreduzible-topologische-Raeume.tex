
\section{Irreduzible topologische Räume}
\label{sec:irreduzibilitaet-top}

Die folgenden topologischen Begriffe sind nur interessant, da $\mathbb{A}^{n}(k)$
($n>0$) kein Hausdorff'scher Raum ist.
\begin{defn}
  \label{def:irreduzibel}
  Ein topologischer Raum $X$ heißt \textbf{irreduzibel}\index{irreduzibel},
  falls $X\neq\emptyset$ und $X$ sich \emph{nicht} als Vereinigung
  zweier echter abgeschlossener Teilmengen darstellen lässt, d.h
  \[
    X=A_{1}\cup A_{2},\ A_{i}\ \text{abg.}\quad\implies\quad A_{1}=X\text{ oder }A_{2}=X.
  \]

  $Z\subseteq X$ heißt irreduzibel, falls $Z$ mit der induzierten Topologie
  irreduzibel ist.
\end{defn}
\begin{prop}
  \label{prop:charakterisierung-irreduzibel}
  Für einen topologischen Raum $X \neq \emptyset$ sind äquivalent:
  \begin{enumerate}
  \item $X$ ist irreduzibel.
  \item Je zwei nichtleere offene Teilmengen von $X$ haben nicht-leeren
    Durchschnitt.
  \item Jede nichtleere offene Teilmenge $U\subseteq X$ ist dicht in $X$.
  \item Jede nichtleere offene Teilmenge $U\subseteq X$ ist zusammenhängend.
  \item Jede nichtleere offene Teilmenge $U\subseteq X$ ist irreduzibel.
  \end{enumerate}
\end{prop}
\begin{proof}
  \mbox{}
  \begin{itemize}
  \item $(i)\Leftrightarrow(ii)$

    Komplementärmengen.
  \item $(ii)\Leftrightarrow(iii)$ 

    Es ist: $U\subseteq X$ dicht $\Leftrightarrow U\cap O\neq\emptyset$
    für jedes offene $\emptyset\neq O\subseteq X$.
  \item $(iii)\Rightarrow(iv)$

    Klar. 
  \item $(iv)\Rightarrow(iii)$

    Sei $\emptyset\neq U$ offen und zusammenhängend. Es folgt:
    \[
      U=U_{1}\sqcup U_{2},\qquad\emptyset\neq U_{i}\underset{\text{offen}}{\subseteq}U\underset{\text{offen}}{\subseteq}X
    \]
    Damit ist $U_{1}\cap U_{2}=\emptyset$, ein Widerspruch zu (iii).
  \item $(v)\Rightarrow(i)$ 

    Klar. $(U=X)$
  \item $(iii)\Rightarrow(v)$

    Sei $\emptyset\neq U\underset{\text{offen}}{\subseteq}X$. Ist $\emptyset\neq V\underset{\text{offen}}{\subseteq}U$,
    so ist $V\underset{\text{offen}}{\subseteq}X$. Es folgt: $V$ ist
    dicht in $X$ und irreduzibel in $U$. Mit $(iii)\Rightarrow(i)$
    folgt, dass $U$ irreduzibel ist. 

  \end{itemize}
\end{proof}
\begin{lem}
  \label{lem:irreduzibel-abschluss}
  Eine Teilmenge $Y$ ist genau dann irreduzibel, wenn ihr Abschluss $\overline{Y}$ dies ist.
\end{lem}
\begin{proof}
  $Y$ irreduzibel

  $\Leftrightarrow\forall U,V\subseteq X$ offen mit $U\cap Y\neq\emptyset\neq V\cap Y$,
  gilt $Y\cap(U\cap V)\neq\emptyset$.

  $\Leftrightarrow\overline{Y}$ irreduzibel 
\end{proof}
\begin{defn}
  \label{def:irreduzible-komponente}
  Eine maximale irreduzible Teilmenge eines topologischen Raumes $X$
  heißt \textbf{irreduzible Komponente}\index{irreduzible Komponente}
  von $X$.
\end{defn}
\begin{rem}
  \label{rem:irreduzibel}
  \mbox{}
  \begin{enumerate}
  \item Jede irreduzible Komponente ist abgeschlossen nach Lemma 14.
  \item $X$ ist Vereinigung seiner irreduziblen Komponenten, \emph{denn}: 

    die Menge der irreduziblen Teilmengen von $X$ ist \textbf{induktiv
      geordnet}: für jede aufsteigende Kette irreduzibler Teilmengen ist
    die Vereinigung wieder irreduzibel (Satz 13 (ii)). Mit dem \textbf{Lemma
      von Zorn} folgt: Jede irreduzible Teilmenge ist in einer irreduziblen
    Komponente enthalten. Damit ist jeder Punkt in einer irreduziblen
    Komponente enthalten.
  \end{enumerate}
\end{rem}

