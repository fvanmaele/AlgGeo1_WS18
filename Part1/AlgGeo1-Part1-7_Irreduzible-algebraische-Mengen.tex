
\section{Irreduzible affine algebraische Mengen}
\label{sec:irreduzibilitaet-alg}
\begin{lem}
\label{lem:charakterisierung-irreduzibel-alg}
Eine abgeschlossene Teilmenge $Z\subseteq\mathbb{A}^{n}(k)$ ist genau
dann irreduzibel, wenn $I(Z) \unlhd k[\underline{T}]$ ein Primideal ist. Insbesondere ist
$\mathbb{A}^{n}(k)$ irreduzibel.  
\end{lem}
\begin{proof}
$Z$ irreduzibel ist �quivalent zu 
\begin{align*}
 & (Z=\underbrace{V(\mathfrak{a})}_{\bigcap_{i} V(f_{i})}\cup\underbrace{V(\mathfrak{b})}_{\bigcap_{j} V(g_{j})}\quad\Rightarrow\quad V(\mathfrak{a})=Z\text{ oder }V(\mathfrak{b})=Z).\\
\Leftrightarrow\  & \forall f,g\in k[\underline{T}]:\ V(fg)=V(f)\cup V(g)\supseteq Z:\ V(f)\supseteq Z\text{ oder }V(g)\supseteq Z.\\
(*)\Leftrightarrow\  & \forall f,g\in k[\underline{T}]:\ fg\in I(V(fg))\subseteq I(Z):\ f\in I(Z)\text{ oder }g\in I(Z).\\
\Leftrightarrow\  & I(Z)\text{ ist Primideal.}
\end{align*}

({*}): $V(I(Z))=Z$, $I(V(\mathfrak{a}))=\rad(\mathfrak{a})$. 
\end{proof}
\begin{rem}
\label{rem:korrespondenz-irreduzibel-prim}
Die Korrespondenz aus Korollar 11 schr�nkt sich ein zu
\[
\{\text{irred. abg. Teilmengen }\subseteq\mathbb{A}^{n}\}\overset{1:1}{\leftrightarrow}\{\text{Primideale in }k[\underline{T}]\}
\]
\end{rem}

