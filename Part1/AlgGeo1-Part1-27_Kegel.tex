%\selectlanguage{english}%

\section{Kegel}
\label{sec:Kegel}

Sei $H\subseteq\mathbb{P}^{n}(k)$ Hyperebene, $p\in\mathbb{P}^{n}(k)\backslash H$,
$X\subseteq H$ abgeschlossene Unterpr�variet�t.
\[
\overline{X,p}:=\bigcup_{q\in X}\overline{qp}
\]

hei�t \textbf{Kegel von $X$ �ber $p$}, es handelt sich um einen
abgeschlossenen Untervariet�t von $\mathbb{P}^{n}(k)$. Ohne Einschr�nkung:$H=V_{+}(X_{n})$,
$p=(0:\cdots:1)$ (nach Koordinatenwechsel: $H\cong k^{n}\oplus p\cong k\}=k^{n+1}$.)
F�r  
\begin{align*}
X=V_{+}(f_{1},\ldots,f_{m})\subseteq\mathbb{P}^{n-1}(k)=H, & \quad f_{i}\in k[X_{0},\ldots,X_{n-1}]\\
\Rightarrow X,p=V_{+}(\tilde{f}_{1},\ldots,\tilde{f}_{m})\subseteq\mathbb{P}^{n}(k), & \quad\tilde{f}_{i}\in k[X_{0},\ldots,X_{n}]
\end{align*}

Verallgemeinerung. Sei $\mathbb{P}^{n}(k)\cong\Lambda\subseteq\mathbb{P}^{n}(k)$
linearer Unterraum, $\psi\subseteq\mathbb{P}^{n}(k)$ komplement�rer
linearer Unterraum, d.h. $\Lambda\cap\psi=\emptyset$ und $\mathbb{P}^{n}(k)$
ist der bekannte lineare Unterraum von $\mathbb{P}^{n}(k)$, der $\Lambda$
und $\psi$ enth�lt. $X\subseteq\psi$ abgeschlossene Unterpr�variet�t.

\textbf{Kegel von $X$ �ber $\Lambda$}: $\overline{X,\Lambda}=\bigcup_{q\in X}\overline{q,\Lambda}$,
wobei der von $q$ und $\Lambda$ aufgespannte lineare Unterraum $\overline{q,\Lambda}$
der kleinste Unterraum sei, der $q$ und $\Lambda$ enth�lt.

\section{Quadriken}
\label{sec:quadriken}

Sei char$(k)\neq2$ in diesem Abschnitt.
\begin{defn}[orig. 57]
\label{def:quadrik}
Eine abgeschlossene Unterpr�variet�t $Q\subseteq\mathbb{P}^{n}(k)$
von der Form $V_{+}(q)$, $q\in k[X_{0},\ldots,X_{n}]_{2}\backslash\{0\}$
hei�t \textbf{Quadrik}.
\[
Q=V_{+}(q)
\]

Zur quadratischen Form $q$ geh�rt eine Bilinearform $\beta$ auf
$k^{n+1}$, 
\[
\beta(v,w):=\frac{1}{2}(q(v+w)-q(v)-q(w)),\quad v,w\in k^{n+1}
\]

Es gibt eine Basis von $k^{n+1}$, sodass die Strukturmatrix $B$
von $\beta$ die Gestalt
\[
B=\begin{pmatrix}\begin{array}{ccc}
1\\
 & \ddots\\
 &  & 1
\end{array} & 0\\
0 & \begin{array}{ccc}
0\\
 & \ddots\\
 &  & 0
\end{array}
\end{pmatrix}
\]

hat, d.h. Koordinatenwechsel zur Basiswechselmatrix liefert einen
Isomorphismus
\[
Q\xrightarrow{\cong}V_{+}(X_{0}^{2}+\cdots+X_{r-1}^{2}),\quad r=\text{rg }B
\]
\end{defn}
\begin{lem}[orig. 58]
\label{lem:platzhalter}
\end{lem}
\begin{prop}[orig. 59]
\label{prop:quadrik-in-normalform}
Ist $r\neq s$, so sind $V_{+}(T_{0}^{2}+\cdots+T_{r-1}^{2})$ und
$V_{+}(T_{0}^{2}+\cdots+T_{s-1}^{2})$ nicht isomorph.
\end{prop}
\begin{proof}
(sp�ter: Es gibt keinen Koordinatenwechsel von $\mathbb{P}^{n}(k)$,
der beide Mengen identifiziert, damit auch kein Automorphismus in
$\mathbb{P}^{n}(k)$.)
\end{proof}

\begin{defn}
\label{def:dim-und-rang-einer-quadrik}
Eine Quadrik $Q\subseteq\mathbb{P}^{n}(k)$ mit $Q\cong V_{+}(T_{0}^{2}+\cdots+T_{r-1}^{2})$,
$r\geq1$, habe die \textbf{Dimension $n-1$} und den \textbf{Rang $r$}. (nach Satz eindeutig!)\selectlanguage{ngerman}%
\end{defn}

