\section{Kegel}
\label{sec:Kegel}

Sei $H\subseteq\mathbb{P}^{n}(k)$ Hyperebene, $p\in\mathbb{P}^{n}(k)\backslash H$,
$X\subseteq H$ abgeschlossene Unterprävarietät.
\[
  \overline{X,p}:=\bigcup_{q\in X}\overline{qp}
\]

heißt \textbf{Kegel von $X$ über $p$}, es handelt sich um einen
abgeschlossenen Untervarietät von $\mathbb{P}^{n}(k)$. Ohne Einschränkung:$H=V_{+}(X_{n})$,
$p=(0:\cdots:1)$ (nach Koordinatenwechsel: $H\cong k^{n}\oplus p\cong k\}=k^{n+1}$.)
Für  
\begin{align*}
  X=V_{+}(f_{1},\ldots,f_{m})\subseteq\mathbb{P}^{n-1}(k)=H, & \quad f_{i}\in k[X_{0},\ldots,X_{n-1}]\\
  \Rightarrow X,p=V_{+}(\tilde{f}_{1},\ldots,\tilde{f}_{m})\subseteq\mathbb{P}^{n}(k), & \quad\tilde{f}_{i}\in k[X_{0},\ldots,X_{n}]
\end{align*}

Verallgemeinerung. Sei $\mathbb{P}^{n}(k)\cong\Lambda\subseteq\mathbb{P}^{n}(k)$
linearer Unterraum, $\psi\subseteq\mathbb{P}^{n}(k)$ komplementärer
linearer Unterraum, d.h. $\Lambda\cap\psi=\emptyset$ und $\mathbb{P}^{n}(k)$
ist der bekannte lineare Unterraum von $\mathbb{P}^{n}(k)$, der $\Lambda$
und $\psi$ enthält. $X\subseteq\psi$ abgeschlossene Unterprävarietät.

\textbf{Kegel von $X$ über $\Lambda$}: $\overline{X,\Lambda}=\bigcup_{q\in X}\overline{q,\Lambda}$,
wobei der von $q$ und $\Lambda$ aufgespannte lineare Unterraum $\overline{q,\Lambda}$
der kleinste Unterraum sei, der $q$ und $\Lambda$ enthält.
