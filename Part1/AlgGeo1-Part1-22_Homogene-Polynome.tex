
\section*{Beispiele (Projektiver Raum und projektive Variet�ten)}

\section{Homogene Polynome}
\label{sec:homogene-polynome}
\begin{defn}[orig. 48]
\label{def:homogen}
Ein Polynom $f\in k[X_{0},\ldots,X_{n}]$ hei�t \textbf{homogen vom
Grad}\index{homogen} $d\in\mathbb{Z}_{\geq0}$, falls $f$ die Summe
von Monomen von Grad $d$ ist. (Insbesondere ist f�r jedes $d$ das
Nullpolynom homogen von Grad $d$.)

\emph{Es bezeichne} $k[X_{0},\ldots,X_{n}]_{d}$ den $k$-Untervektorraum
der Polynome \textbf{homogen vom Grad $d$}, $k[X_{0}, \ldots, X_{n}]_{\leq n}$ den $k$-Untervektorraum \textbf{aller Polynome vom Grad $\leq n$}.
\end{defn}
\begin{rem}[orig. 49]
\label{rem:charakterisierung-homogen}
Da \#$k$ unendlich ist, ist $f$ homogen vom Grad $d$
$\Leftrightarrow f(\lambda x_{0},\ldots,\lambda x_{n})=\lambda^{d}f(x_{0},\ldots,x_{n})$
$\forall x_{0},\ldots,x_{n}\in k$, $\lambda\in k^{\times}$. 

Es gilt: $k[X_{0},\ldots X_{n}]=\bigoplus_{d\geq0}k[X_{0},\ldots,X_{n}]_{d}$.
\end{rem}
\begin{lem}[orig. 50]
\label{lem:homogenisierung-dehomogenisierung}
F�r $i\in\{0,\ldots,n\}$ und $d\geq0$ haben wir bijektive $k$-lineare
Abbildungen
\begin{align*}
k[X_{0},\ldots,X_{n}]_{d} & \longrightarrow k[T_{0},\ldots,\hat{T}_{i},\ldots,T_{n}]_{\leq d}\\
f & \overset{\Phi_{i}^{d}}{\longmapsto}f(T_{0},\ldots,\underbrace{1}_{i},\ldots,T_{n})\\
X_{i}^{d}g\left(\frac{X_{0}}{X_{i}},\ldots,\hat{\frac{X_{i}}{X_{i}}},\ldots,\frac{X_{n}}{X_{i}}\right) & \overset{\Psi_{i}^{d}}{\longmapsfrom}g
\end{align*}

\textbf{Dehomogenisierung }bzw. \textbf{Homogenisierung.}
\end{lem}
\begin{proof}
Es reicht, $\Psi_{i}^{d}\circ\Phi_{i}^{d}=\text{id}$, $\Phi_{i}^{d}\circ\Psi_{i}^{d}=\text{id}$
auf Monomen nachzurechnen, da alle Abbildungen $k$-linear sind. 
\end{proof}
Oft ist es n�tzlich, 
$k[T_{0},\ldots,\hat{T_{i}},\ldots,T_{n}]$ mit $k\left[\frac{X_{0}}{X_{i}},\ldots,\hat{\frac{X_{i}}{X_{i}}},\ldots,\frac{X_{n}}{X_{i}}\right] \hookrightarrow k(X_{0},\ldots,X_{n})$ zu identifizieren.

