
\section*{Beispiele (Projektiver Raum und projektive Variet�ten)}

\section{Homogene Polynome}
\begin{defn}[orig. 48]
Ein Polynom $f\in k[X_{0},\ldots,X_{n}]$ hei�t \textbf{homogen vom
Grad}\index{homogen} $d\in\mathbb{Z}_{\geq0}$, wenn $f$ die Summe
von Monomen von Grad $d$ ist. (Insbesondere ist f�r jedes $d$ das
Nullpolynom homogen von Grad $d$.)

\emph{Bezeichne} $k[X_{0},\ldots,X_{n}]_{d}$ der Untervektorraum
der Polynome vom Grad $d$.
\end{defn}
\begin{rem}[orig. 49]
Da \#$k$ unendlich ist, ist $f$ homogen vom Grad $d$. 

$\Leftrightarrow f(\lambda x_{0},\ldots,\lambda x_{n})=\lambda^{d}f(x_{0},\ldots,x_{n})$
$\forall x_{0},\ldots,x_{n}\in k$, $\lambda\in k^{\times}$. 

Es gilt: $k[X_{0},\ldots X_{n}]=\bigoplus_{d\geq0}k[X_{0},\ldots,X_{n}]_{d}$.
\end{rem}
\begin{lem}[orig. 50]
F�r $i\in\{0,\ldots,n\}$ und $d\geq0$ haben wir bijektive $k$-lineare
Abbildungen
\begin{align*}
k[X_{0},\ldots,X_{n}]_{d} & \longrightarrow\text{Polynome in }k[T_{0},\ldots,\hat{T}_{i},\ldots,T_{n}]\text{ v. Grad }\leq d\\
f & \overset{\Phi_{i}^{d}}{\longmapsto}f(T_{0},\ldots,\underbrace{1}_{i},\ldots,T_{n})\\
g\left(\frac{x_{0}}{x_{i}},\ldots,\frac{\hat{x_{i}}}{x_{i}},\ldots,\frac{x_{n}}{x_{i}}\right) & \overset{\Psi_{i}^{d}}{\longmapsfrom}g
\end{align*}

\textbf{Dehomogenisierung }bzw. \textbf{Homogenisierung.}
\end{lem}
\begin{proof}
Es reicht, $\Psi_{i}^{d}\circ\Phi_{i}^{d}=\text{id}$, $\Phi_{i}^{d}\circ\Psi_{i}^{d}=\text{id}$
auf Monomen nachzurechnen, da alle Abbildungen $k$-linear sind. 
\end{proof}
Oft ist es n�tzlich, $k[T_{0},\ldots,\hat{T_{i}},\ldots,T_{n}]$ mit
$k\left[\frac{x_{0}}{x_{i}},\ldots,\frac{\hat{x_{i}}}{x_{i}},\ldots,\frac{x_{n}}{x_{i}}\right]\underset{\text{Unterring}}{\subset}k(X_{0},\ldots,X_{n})$.


\subsection{Definition des projektiven Raumes}

Sei $X_{1}=X_{2}=\mathbb{A}^{1}$, $\tilde{U}_{1}\subseteq X_{1}=\tilde{U}_{2}\subseteq X_{2}=\mathbb{A}\backslash\{0\}$.
\begin{align*}
\tilde{U}_{1} & \overset{\sim}{\longrightarrow}\tilde{U}_{2}\\
x & \longmapsto\frac{1}{x}
\end{align*}

Verkleben von $X_{1}$ und $X_{2}$ entlang $\tilde{U}_{1}$ und $\tilde{U}_{2}$

$\mathbb{P}^{1}=\mathbb{A}^{1}\cup\{\infty\}=U_{1}\cup U_{2}$. 

Allgemein: 
\[
\mathbb{P}^{n}=\bigcup_{i=1}^{n+1}U_{i}=\mathbb{A}^{n}\cup\mathbb{P}^{n-1}=\mathbb{A}^{n}\sqcup\mathbb{A}^{n-1}\sqcup\cdots\sqcup\mathbb{A}^{1}\sqcup\mathbb{A}^{0}
\]

Idee: $\mathbb{P}^{2}\supseteq\mathbb{A}^{2}$: Zwei verschiedene
Geraden in $\mathbb{P}^{2}$ schneiden sich genau in einem Punkt.

\textbf{Als Menge}:
\begin{align*}
\mathbb{P}^{n}(k): & =\{\text{Ursprungsgeraden in }k^{n+1}\}=\{1\text{-dim. }k\text{-UVR}\}\\
 & =(k^{n+1}\backslash\{0\})/k^{\times}
\end{align*}

Repr�sentanten dieser Klasse entsprechen: 
\[
\langle(x_{0},\ldots x_{n})\rangle_{k\text{-linear}}\longmapsfrom(x_{0}:\ldots:x_{n})
\]

�quivalenzrelation: $(x_{0},\ldots,x_{n})\sim(x_{0}',\ldots,x_{n}')\Leftrightarrow\exists\lambda\in k^{\times}$
mit $x_{i}=\lambda x_{i}'$ $\forall i$

\textbf{Bezeichne} Klassen $(x_{0}:\ldots:x_{n})$, $x_{i}$ \textbf{homogene}
Koordinaten auf $\mathbb{P}^{n}$
\[
U_{i}:=\{(x_{0}:\cdots:x_{n})\in\mathbb{P}^{n}\mid x_{i}\neq0\}\subseteq\mathbb{P}^{n}(k),\ 0\leq i\leq n
\]

ist wohldefiniert $\Leftrightarrow x_{i}=1$.
\[
\mathbb{P}^{n}(k)=\bigcup_{i=0}^{n}U_{i}
\]

Einen Isomorphismus 
\begin{align*}
U_{i} & \stackrel[\chi_{i}]{\cong}{\longrightarrow}\mathbb{A}^{n}(k)\\
(x_{0}:\ldots:x_{n}) & \longmapsto\left(\frac{x_{0}}{x_{i}},\ldots,\frac{\hat{x}_{i}}{x_{i}},\ldots,\frac{x_{n}}{x_{i}}\right)\\
(t_{0}:\cdots t_{i-1}:1:t_{i+1}:\cdots t_{n}) & \longmapsfrom(t_{0},\ldots,\hat{t}_{i},\ldots,t_{n})
\end{align*}
 
