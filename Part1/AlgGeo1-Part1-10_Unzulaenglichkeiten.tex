
\section{Unzulänglichkeiten des Begriffs der affinen algebraischen Mengen}
\label{sec:unzulaenglichkeiten-alg-mengen}
\begin{enumerate}
\item Offene Teilmengen affiner algebraischer Mengen tragen nicht in natürlicher
  Weise die Struktur einer affinen algebraischen Menge.
\item Insbesondere können wir affine algebraische Mengen nicht entlang offener
  Teilräume verkleben. (vgl. Mannigfaltigkeiten.)
\item Keine Unterscheidungsmöglichkeiten z.B. zwischen $\{(0,0)\}$,
  $V(T_{1})\cap V(T_{2})$ und
  $V(T_{2})\cap V(T_{1}^{2}-T_{2})\subseteq\mathbb{A}^{2}(k)$, obwohl
  die ``geometrische Situation'' offensichtlich verschieden ist.
\end{enumerate}
Um die Punkte 1 und 2 zu verbessern, gehen wir im Folgenden zu ``Räumen mit Funktionen'' über, und verzichten darauf,
dass sich diese in einen affinen Raum $\mathbb{A}^{n}(k)$ einbetten
lassen.

Der Punkt 3 ist eine Motivation dafür, später Schemata einzuführen.
(subtiler)

