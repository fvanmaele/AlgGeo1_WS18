\selectlanguage{english}%

\section{Unzul�nglichkeiten des Begriffs der affinen algebraischen Mengen}\label{sec:unzulaenglichkeiten-alg-mengen}
\begin{enumerate}
\item Offene Teilmengen affiner algebraischer Mengen tragen nicht in nat�rlicher
Weise die Struktur einer affinen algebraischen Menge.
\item Insbesondere k�nnen wir affine algebraische Mengen nicht entlang offener
Teilr�ume verkleben. (vgl. Mannigfaltigkeiten.)
\item Keine Unterscheidungsm�glichkeiten z.B. zwischen $\{(0,0)\}$, $V(T_{1})\cap V(T_{2})$
und $V(T_{2})\cap V(T_{1}^{2}-T_{2})\subseteq\mathbb{A}^{2}(k)$,
obwohl die ``geometrische Situation'' offensichtlich verschieden
ist.
\end{enumerate}
Um die Punkte 1 und 2 zu verbessern, gehen wir im Folgenden zu ``R�umen mit Funktionen'' �ber, und verzichten darauf,
dass sich diese in einen affinen Raum $\mathbb{A}^{n}(k)$ einbetten
lassen.

Der Punkt 3 ist eine Motivation daf�r, sp�ter Schemata einzuf�hren.
(subtiler)\selectlanguage{ngerman}%

