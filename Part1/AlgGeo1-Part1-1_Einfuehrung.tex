
\section{Einf�hrung}

\textbf{Algebraische Geometrie}\index{Algebraische Geometrie} kann
man verstehen, als das Studium von Systemen polynomialer Gleichungen
(in mehreren Variabelen). Damit ist die algebraische Geometrie eine
Verallgemeinerung der \textbf{linearen Algebra}, also statt $X$ auch
$X^{n}$, und auch der \textbf{Algebra}, durch Polynome in \emph{mehreren
}Variabelen.
\begin{question*}
Sei $k$ ein (algebraisch abgeschlossener) K�rper, und $f_{1},\ldots,f_{m}\in k[T_{1},\ldots,T_{n}]$
gegeben. Was sind die ``geometrischen Eigenschaften'' der Nullstellenmenge
\[
V(f_{1},\ldots,f_{n}):=\{(t_{1},\ldots,t_{n})\in k^{n}\mid f_{i}(t_{1},\ldots,t_{n})=0\ \forall i\}
\]
\end{question*}
\begin{example}
Sei $f=T_{2}^{2}-T_{1}^{2}(T_{1}-1)\in k[T_{1},T_{2}]$. Die Nullstellenmenge
f�r $k=\mathbb{R}$ (\emph{aber: }tr�gerisch, da $\mathbb{R}$ nicht
algebraisch abgeschlossen!) ist gegeben durch:
\end{example}
\begin{figure}
\caption{$T_{2}^{2}=T_{1}^{2}(T_{1}-1)=T_{1}^{3}-T_{1}^{2}$}
\end{figure}


\paragraph{Dimension 1. }

Glatte und singul�ren Punkten: $(0,0)$ singul�r. Alle anderen Punkte
verletzen eine eindeutig bestimmte Tangente.

\begin{figure}[h]
\caption{\textbf{Spitze} und \textbf{Doppelpunkt}}
\end{figure}

Vergleiche den \textbf{Satz �ber implizite Funktionen}. (Analysis,
Differentialgeometrie) 

$V(f)$ ist lokal diffeomorph zu $\mathbb{R}$ (= reelle Gerade) im
Punkt $(x_{1},x_{2})$ genau dann, wenn die Jacobi-Matrix 
\[
\left(\frac{\partial f}{\partial T_{1}},\frac{\partial f}{\partial T_{2}}\right)=\left(T_{1}(3T_{1}-2),\ 2T_{2}\right)
\]
 hat Rang 1 in $(x_{1},x_{2})$. Das ist �quivalent dazu, dass $(x_{1},x_{2})\neq(0,0)$.
Dies l�sst sich rein formal �ber beliebigen Grundk�rpern \textbf{algebraisch}
formulieren.

\paragraph{Methoden.}

GAGA - G�ometrie alg�brique, g�ometrique analytique (Serre)\medskip{}

\begin{tabular}{|c|c|}
\hline 
Komplexe Geometrie ($\mathbb{C}$), Differentialgeometrie $(\mathbb{R})$ & Algebraische Geometrie\tabularnewline
\hline 
\hline 
Analytische Hilfsmittel & Kommutative Algebra\tabularnewline
\hline 
\end{tabular}
