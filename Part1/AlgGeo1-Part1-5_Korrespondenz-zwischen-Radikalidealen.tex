
\section{Korrespondenz zwischen Radikalidealen und affinen algebraischen Mengen}

Sei $V(\mathfrak{A})\subseteq\mathbb{A}^{n}(k)$ affin algebraische
Menge, $\mathfrak{A}\subset k[\underline{T}]$.

\textbf{Es gilt:} $V(\mathfrak{A})=V(\text{rad\,}\mathfrak{A})$ mit
 $\rad\mathfrak{A}=\{f\in k[\underline{T}]\mid f^{n}\in\mathfrak{A}$
f�r $n>0\}$, da
\[
f^{n}(x)=0\Leftrightarrow f(x)=0,
\]

d.h. verschiedene Ideale k�nnen dieselbe algebraische Menge beschreiben.

\subsection{Definition 9}

F�r eine Teilmenge $Z\subseteq\mathbb{A}^{n}(k)$ bezeichne
\[
I(Z)=\{f\in k[\underline{T}]\mid f(x)=0\ \forall x\in Z\}
\]

das Ideal aller auf $Z$ verschwindenden Polynomfunktionen.

\subsection{Satz 10}
\begin{enumerate}
\item $\mathfrak{A}\subset k[\underline{T}]$ Ideal $\Rightarrow I(V(\mathfrak{A}))=\rad(\mathfrak{A})$.
\item $Z\subseteq\mathbb{A}^{n}(k)$ Teilmenge $\Rightarrow V(I(Z))=\overline{Z}$,
der Abschluss von $Z$.
\end{enumerate}

\subsubsection{Beweis (Satz 10)}

Algebra II, �bungsaufgabe. 

\medskip{}

$\mathfrak{A}$ hei�t \textbf{Radikalideal}\index{Radikalideal},
wenn $\mathfrak{A}=\rad(\mathfrak{A})$, oder �quivalent wenn $\faktor{k[\underline{T}]}{\mathfrak{A}}$
\emph{reduziert} ist, d.h. keine nilpotente Elemente hat.

\subsection{Korollar 11}

Wir erhalten eine 1-1 Korrespondenz
\begin{align*}
\{\text{abg. Mengen }\subseteq\mathbb{A}^{n}\} & \leftrightarrow\{\text{Radikalideale }\mathfrak{A}\subset k[\underline{T}]\}\\
Z & \mapsto I(Z)\\
V(\mathfrak{A}) & \mapsfrom\mathfrak{A}
\end{align*}

die sich zu einer 1-1 Korrespondenz
\begin{align*}
\left\{ \text{Punkte in }\mathbb{A}^{n}\right\}  & \leftrightarrow\left\{ \text{max. Ideale in }k[\underline{T}]\right\} \\
x=(x_{1},\ldots,x_{n}) & \mapsto\begin{array}{rl}
\mathfrak{m}_{x} & =I(\{x\})\\
 & =\ker(k[\underline{T}]\rightarrow k,\ T_{i}\mapsto x_{i})
\end{array}
\end{align*}

einschr�nkt.
