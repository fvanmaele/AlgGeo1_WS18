
\section{Korrespondenz zwischen Radikalidealen und affinen algebraischen Mengen}
\label{sec:radikalideale-und-algebraische-mengen}

Sei $V(\mathfrak{a})\subseteq\mathbb{A}^{n}(k)$ affin algebraische
Menge, $\mathfrak{a}\unlhd k[\underline{T}]$ ein Ideal.\textbf{ Es gilt:}
\[
V(\mathfrak{a})=V(\rad\mathfrak{a})
\]

mit $\rad\mathfrak{a}=\{f\in k[\underline{T}]\mid f^{n}\in\mathfrak{a}\text{ f�r ein }n>0\}$,
da
\[
f^{n}(x)=0\Leftrightarrow f(x)=0,
\]

d.h. verschiedene Ideale k�nnen dieselbe algebraische Menge beschreiben.
\begin{defn}
\label{def:verschwindungsideal}
F�r eine Teilmenge $Z\subseteq\mathbb{A}^{n}(k)$ bezeichne
\[
I(Z):=\{f\in k[\underline{T}]\mid f(x)=0\ \forall x\in Z\}
\]

das \textbf{Verschwindungsideal von Z}, das Ideal aller auf $Z$ verschwindenden Polynomfunktionen.
\end{defn}
\begin{prop}
\label{prop:verschwindungsmenge-verschwindungsideal}
\mbox{}
\begin{enumerate}
\item Sei $\mathfrak{a}\unlhd k[\underline{T}]$ Ideal. Dann ist $I(V(\mathfrak{a}))=\rad(\mathfrak{a})$.
\item Sei $Z\subseteq\mathbb{A}^{n}(k)$ Teilmenge. Dann ist $V(I(Z))=\overline{Z}$,
der Abschluss von $Z$ in $\mathbb{A}^{n}(k)$.
\end{enumerate}
\end{prop}
\begin{proof}
�bungsblatt 2.
\end{proof}
\medskip{}

$\mathfrak{a}$ hei�t \textbf{Radikalideal}\index{Radikalideal},
falls $\mathfrak{a}=\rad(\mathfrak{a})$, oder �quivalent falls $k[\underline{T}]/\mathfrak{a}$
\emph{reduziert} ist, d.h. keine nilpotente Elemente ungleich $0$ hat.
\begin{cor}\label{korrespondenz-radikalideal-abgeschlossene-mengen}
Wir erhalten eine 1-1 Korrespondenz
\begin{align*}
\{\text{abg. Mengen }\subseteq\mathbb{A}^{n}\} & \leftrightarrow\{\text{Radikalideale }\mathfrak{a}\unlhd k[\underline{T}]\}\\
Z & \mapsto I(Z)\\
V(\mathfrak{a}) & \mapsfrom\mathfrak{a}
\end{align*}

die sich zu einer 1-1 Korrespondenz
\begin{align*}
\left\{ \text{Punkte in }\mathbb{A}^{n}\right\}  & \leftrightarrow\left\{ \text{max. Ideale in }k[\underline{T}]\right\} \\
x=(x_{1},\ldots,x_{n}) & \mapsto\begin{array}{rl}
\mathfrak{m}_{x} & =I(\{x\})\\
 & =\ker(k[\underline{T}]\rightarrow k,\ T_{i}\mapsto x_{i})
\end{array}
\end{align*}

einschr�nkt.
\end{cor}

