
\section{Definition des projektiven Raumes}
\label{sec:def-projektiver-raum}

Seien $X_{1}=X_{2}=\mathbb{A}^{1}$, $\tilde{U}_{1}\subseteq X_{1}, \tilde{U}_{2}\subseteq X_{2}$ mit $\tilde{U}_{1} = \tilde{U}_{2} = \mathbb{A}^{1}\setminus\{0\}$.
\begin{align*}
  \tilde{U}_{1} & \overset{\sim}{\longrightarrow}\tilde{U}_{2}\\
  x & \longmapsto\frac{1}{x}
\end{align*}

Verkleben von $X_{1}$ und $X_{2}$ entlang $\tilde{U}_{1} \overset{\sim}\longrightarrow \tilde{U}_{2}$ liefert die \textbf{projektive Gerade}

\[
  \mathbb{P}^{1}=\mathbb{A}^{1}\cup\{\infty\}=U_{1}\cup U_{2}.
\]

Allgemein: 
\[
  \mathbb{P}^{n}=\bigcup_{i=1}^{n+1}U_{i}=\mathbb{A}^{n}\cup\mathbb{P}^{n-1}=\mathbb{A}^{n}\sqcup\mathbb{A}^{n-1}\sqcup\cdots\sqcup\mathbb{A}^{1}\sqcup\mathbb{A}^{0}
\]

\textbf{Idee}: $\mathbb{P}^{2}\supseteq\mathbb{A}^{2}$: Zwei verschiedene
Geraden in $\mathbb{P}^{2}$ schneiden sich genau in einem Punkt.

\textbf{Als Menge}:
\begin{align*}
  \mathbb{P}^{n}(k): & =\{\text{Ursprungsgeraden in }k^{n+1}\}=\{1\text{-dim. }k\text{-Unterr�ume}\}\\
                     & =(k^{n+1}\backslash\{0\})/k^{\times}
\end{align*}

Man schreibt meist kurz $(x_{0}:\ldots:x_{n})$ f�r den Repr�sentanten der Klasse von $\langle(x_{0},\ldots x_{n})\rangle_{k}$ und nennt $(x_{0}:\ldots:x_{n})$ \textbf{homogene Koordinaten} auf $\mathbb{P}^{n}$.


\emph{�quivalenzrelation}: 
\[
  (x_{0},\ldots,x_{n})\sim(x_{0}',\ldots,x_{n}')\Leftrightarrow\exists\lambda\in k^{\times}\ \text{mit}\ x_{i}=\lambda x_{i}'\ \forall i.
\]
Die Mengen
\[
  U_{i}:=\{(x_{0}:\ldots:x_{n})\in\mathbb{P}^{n}\mid x_{i}\neq0\}\subseteq\mathbb{P}^{n}(k),\ 0\leq i\leq n
\]

sind wohldefiniert und �berdecken $\mathbb{P}^{n}(k)$:
\[
  \mathbb{P}^{n}(k)=\bigcup_{i=0}^{n}U_{i}
\]

Weiter hat man eine Bijektion
\begin{align*}
  U_{i} & \stackrel[1:1]{\kappa_{i}}{\longrightarrow}\mathbb{A}^{n}(k)\\
  (x_{0}:\ldots:x_{n}) & \longmapsto\left(\frac{x_{0}}{x_{i}},\ldots,\frac{\hat{x}_{i}}{x_{i}},\ldots,\frac{x_{n}}{x_{i}}\right)\\
  (t_{0}:\cdots t_{i-1}:1:t_{i+1}:\cdots t_{n}) & \longmapsfrom(t_{0},\ldots,\hat{t}_{i},\ldots,t_{n})
\end{align*}

�ber die $\kappa_{i}$ definiert man nun eine Topologie auf $\mathbb{P}^{n}(k)$ durch:

$U\subseteq\mathbb{P}^{n}(k)$ ist genau dann offen, wenn $\kappa_{i}(U\cap U_{i})\subseteq\mathbb{A}^{n}(k)$
offen ist f�r alle $i$. 

Es gilt: 
\[
  U_{i}\cap U_{j}= D(T_{j})\subseteq U_{i}\text{ offen},\ i\neq j
\]

wenn auf $U_{i}\cong\mathbb{A}^{n}$ die Koordinaten $T_{0},\ldots,\hat{T}_{i},\ldots,T_{n}$
verwendet werden. Damit wird $\mathbb{P}^{n}(k)$ zu einem topologischen
Raum, der durch die $U_{i}$, $0\leq i\leq n$, offen �berdeckt wird.

\subsection{Regul�re Funktionen}
\label{subsec:regulaere-fkt-auf-projektivem-raum}

Sei $U\subseteq\mathbb{P}^{n}(k)$ eine beliebige offene Teilmenge.
Die regular�ren Funktionen auf $U$ sind definiert als
\[
  \mathcal{O}_{\mathbb{P}^{n}}(U) :=\{f\in\text{Abb}(U,k)\mid f|_{U\cap U_{i}}\in\mathcal{O}_{U_{i}}(U\cap U_{i})\}\qquad\forall i\in\{0,\ldots,n\}
\]

wobei wir die $U_{i}$ via $\kappa_{i}$ implizit als Raum mit Funktionen auffassen. Insgesamt erhalten wir:
\[
  \mathbb{P}^{n}(k)=(\mathbb{P}^{n}(k),\mathcal{O}_{\mathbb{P}^{n}})
\]

als Raum mit Funktionen.
\begin{prop}[orig 51]
  \label{prop:charakterisierung-reg-fkt-projektiver-raum}
  F�r $U\subseteq\mathbb{P}^{n}$ offen gilt: $\mathcal{O}_{\mathbb{P}^{n}}(U)=\{f:U\rightarrow k\mid\forall x\in U$:
  $\exists x\in V\subseteq U$ offen, $d\geq 0$ und $g,h\in k[X_{0},\ldots,X_{n}]_{d}$
  homogen vom selben Grad $d$, d.d. $\forall v\in V$: $h(v)\neq0$ und
  $f(v)=\frac{g(v)}{h(v)}\}$ 
\end{prop}
Wohldefiniertheit: Sei $v=(x_{0}:\ldots:x_{n})$.
\[
  f(\lambda x_{0},\ldots,\lambda x_{n})=\frac{g(\lambda x_{0},\ldots,\lambda x_{n})}{h(\lambda x_{0},\ldots,\lambda x_{n})}=\frac{\lambda^{d}g(x_{0},\ldots,x_{n})}{\lambda^{d}h(x_{0},\ldots,x_{n})}=f(x_{0},\ldots,x_{n})
\]

\begin{proof}
  \mbox{}
  \begin{itemize}
  \item[,,$\subseteq$``:] Sei $f\in\mathcal{O}_{\mathbb{P}^{n}}(U)$. Dann ist $f|_{U\cap U_{i}}\in\mathcal{O}_{U_{i}}(U\cap U_{i})$.
    Es folgt:
    \[
      f=\frac{\tilde{g}}{\tilde{h}},\ \tilde{g},\tilde{h}\in k[T_{0},\ldots,\hat{T}_{i},\ldots,T_{n}]
    \]
    Definiere $d:=\max\{\deg(\tilde{g}),\deg(\tilde{h})\}$. Homogenisiere:
    \[
      g:=\psi_{i}^{d}(\tilde{g}),\ h:=\psi_{i}^{d}(\tilde{h})
    \]
    $\Rightarrow f=\frac{g}{h}$ lokal. 
    \begin{align*}
      f(x) & =\frac{\tilde{g}}{\tilde{h}}(\kappa_{i}(x))\\
      f((x_{0}:\cdots:x_{n})) & =\frac{\tilde{g}\left(\frac{x_{0}}{x_{i}},\ldots,\frac{\hat{x_{i}}}{x_{i}},\ldots,\frac{x_{n}}{x_{i}}\right)}{\tilde{h}\left(\frac{x_{0}}{x_{i}},\ldots,\frac{\hat{x_{i}}}{x_{i}},\ldots,\frac{x_{n}}{x_{i}}\right)}\\
           & =\frac{x_{i}^{d}\tilde{g}(\ldots)}{x_{i}^{d}\tilde{h}(\ldots)}\\
           & =\frac{\psi_{i}^{d}(\tilde{g})(\ldots)}{\psi_{i}^{d}(\tilde{h})(\ldots)}=\frac{g}{h}(x_{0}:\ldots:x_{n})
    \end{align*}
  \item[,,$\supseteq$``:] Sei $f$ in der rechten Menge, fixiere $i\in\{0,\ldots,n\}$. Nach Voraussetzung ist $f$ lokal
    auf $U\cap U_{i}$ von der Form $f = \frac{g}{h}$, $g,h\in k[X_{0},\ldots,X_{n}]_{d}$, $d\geq 0$ geeignet. Definiere:
    \[
      \tilde{g}_{i}:=\frac{g}{X_{i}^{d}},\ \tilde{h}:=\frac{h}{X_{i}^{d}}\in k\left[\frac{X_{0}}{X_{i}},\ldots\hat{\frac{X_{i}}{X_{i}}},\ldots,\frac{X_{n}}{X_{i}}\right]
    \]
    $\Rightarrow f$ ist lokal von der Form: $\frac{\tilde{g}}{\tilde{h}}$,
    $\tilde{g},$$\tilde{h}\in k[T_{0},\ldots,\hat{T_{i}},\ldots T_{n}]$.

    $\Rightarrow f|_{U\cap U_{i}}\in\mathcal{O}_{U_{i}}(U\cap U_{i})$, also $f \in \mathcal{O}_{\mathbb{P}^{n}}(U)$.

  \end{itemize}
\end{proof}
\begin{cor}[orig. 52]
  \label{cor:affine-ueberdeckung-des-projektiven-raumes}
  F�r $i\in\{0,\ldots,n\}$ induziert
  \[
    U\xrightarrow[\cong]{\kappa_{i}}\mathbb{A}^{n}(k)
  \]

  einen Isomorphismus
  \[
    (U_{i},\mathcal{O}_{\mathbb{P}^{n}|_{U_{i}}})\xrightarrow{\cong}\mathbb{A}^{n}(k)
  \]

  von R�umen mit Funktionen. Insbesondere ist $\mathbb{P}^{n}(k)$ eine
  Pr�variet�t.
\end{cor}
\begin{proof}
  Zu zeigen: $\forall U\subseteq U_{i}$ offen gilt
  \[
    \mathcal{O}_{\mathbb{P}^{n}(k)}(U)=\mathcal{O}_{U_{i}}(U)=\{f:U\rightarrow k\mid f\in\mathcal{O}_{U_{i}}(U)\}
  \]

  d.h. auf der rechten Seite muss die Bedingung nur f�r das fixierte
  $i$ �berpr�ft werden. Dies folgt aus dem Beweis von Satz \ref{prop:charakterisierung-reg-fkt-projektiver-raum}.
\end{proof}
Damit identifizieren sich die Funktionenk�rper 
\[
  K(\mathbb{P}^{n}(k))=K(U_{i})=k\left(\frac{X_{0}}{X_{i}},\ldots,\frac{X_{n}}{X_{i}}\right)
\]

\begin{prop}[orig. 53]
  \label{prop:globale-schnitte-des-proj-raums}
  $\mathcal{O}_{\mathbb{P}^{n}(k)}(\mathbb{P}^{n}(k))=k$. Insbesondere
  ist $\mathbb{P}^{n}$ f�r $n\geq1$ \textbf{keine} affine Variet�t.
  (Da der $k$-Algebra $A = k$ ja $\mathbb{A}^{0}(k)=\{\text{pt}\}$ als
  affine Variet�t entspricht.)
\end{prop}
\begin{proof}
  $k\subseteq\mathcal{O}_{\mathbb{P}^{n}(k)}(\mathbb{P}^{n}(k))$ klar, da konstante Funktionen. Nach Satz \ref{prop:charakterisierung-schnitte-praevarietaet} $(iii)$ gilt:
  \begin{align*}
    \mathcal{O}_{\mathbb{P}^{n}}(\mathbb{P}^{n}) & =\bigcap_{i=0}^{n}\mathcal{O}_{\mathbb{P}^{n}}(U_{i})\subseteq K(\mathbb{P}^{n}(k))\\
                                                 & =\bigcap_{i=0}^{n}k[t_{0},\ldots,\hat{t_{i}},\ldots,t_{n}]=k
  \end{align*}
\end{proof}

