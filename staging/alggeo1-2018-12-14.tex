%% LyX 2.3.2 created this file.  For more info, see http://www.lyx.org/.
%% Do not edit unless you really know what you are doing.
\documentclass[ngerman]{article}
\usepackage[T1]{fontenc}
\usepackage[latin9]{inputenc}
\usepackage{enumitem}
\usepackage{amsmath}
\usepackage{amsthm}
\usepackage{amssymb}

\makeatletter
%%%%%%%%%%%%%%%%%%%%%%%%%%%%%% Textclass specific LaTeX commands.
\newlength{\lyxlabelwidth}      % auxiliary length 
\theoremstyle{plain}
\newtheorem{thm}{\protect\theoremname}
\theoremstyle{plain}
\newtheorem{prop}[thm]{\protect\propositionname}
\newlist{casenv}{enumerate}{4}
\setlist[casenv]{leftmargin=*,align=left,widest={iiii}}
\setlist[casenv,1]{label={{\itshape\ \casename} \arabic*.},ref=\arabic*}
\setlist[casenv,2]{label={{\itshape\ \casename} \roman*.},ref=\roman*}
\setlist[casenv,3]{label={{\itshape\ \casename\ \alph*.}},ref=\alph*}
\setlist[casenv,4]{label={{\itshape\ \casename} \arabic*.},ref=\arabic*}
\theoremstyle{plain}
\newtheorem{lem}[thm]{\protect\lemmaname}
\theoremstyle{remark}
\newtheorem*{rem*}{\protect\remarkname}
\theoremstyle{plain}
\newtheorem{cor}[thm]{\protect\corollaryname}
\theoremstyle{definition}
\newtheorem{defn}[thm]{\protect\definitionname}

%%%%%%%%%%%%%%%%%%%%%%%%%%%%%% User specified LaTeX commands.
\usepackage{faktor}
\usepackage{mathtools}
\usepackage{bbold}
\usepackage{tikz}
\usetikzlibrary{arrows}
\DeclareMathOperator{\im}{im}
\DeclareMathOperator{\rel}{rel}
\DeclareMathOperator{\id}{id}
\DeclareMathOperator{\proj}{proj}
\DeclareMathOperator{\diag}{diag}
\DeclareMathOperator{\rot}{rot}
\DeclareMathOperator{\link}{link}
\DeclareMathOperator{\CAT}{CAT}
\DeclareMathOperator{\lab}{lab}
\DeclareMathOperator{\Cay}{Cay}
\DeclareMathOperator{\Pot}{Pot}
\DeclareMathOperator{\catk}{CAT(\kappa)}
\DeclareMathOperator{\cat}{CAT}
\DeclareMathOperator{\Aut}{Aut}
\DeclareMathOperator{\Isom}{Isom}
\DeclareMathOperator{\Pc}{Pc}
\DeclareMathOperator{\diam}{diam}
\DeclareMathOperator{\Spec}{Spec}
\DeclareMathOperator{\Quot}{Quot}

\makeatother

\usepackage{babel}
\providecommand{\casename}{Fall}
\providecommand{\corollaryname}{Korollar}
\providecommand{\definitionname}{Definition}
\providecommand{\lemmaname}{Lemma}
\providecommand{\propositionname}{Satz}
\providecommand{\remarkname}{Bemerkung}
\providecommand{\theoremname}{Theorem}

\begin{document}
\begin{prop}[15]
F�r $X\subset\Spec A$ affin gilt:
\[
X\text{ noethersch }\Leftrightarrow A\text{ noethersch}
\]
\end{prop}

\begin{proof}
\mbox{}
\begin{casenv}
\item[``$\Leftarrow$''] $X$ �berdeckt sich selbst mit $\Gamma(X,\mathcal{O}_{X})=A$ noethersch.
\item[``$\Rightarrow$''] Sei $I\subset A$ beliebiges Ideal. Zu zeigen: $I$ ist endlich erzeugt.
Nach Voraussetzung ist
\[
X=\bigcup_{i=1}^{n}\Spec A_{i},\quad A_{i}\text{ noethersch}.
\]
Ohne Einschr�nking: $A_{i}=A_{f_{i}}$ und noethersch. Daraus folgt:
$J_{i}=IA_{f_{i}}=I_{f_{i}}$ sind endlich erzeugt, Behauptung folgt
aus Lemma 16.
\end{casenv}
\end{proof}
\begin{lem}[16]
$\Spec(A)=\cup_{i\in I}D(f_{i})$, $\#I<\infty$, $M$ $A$-Modul.
Dann:
\[
M\text{ e.e. �ber }A\Leftrightarrow M_{f_{i}}\text{ e.e. }A_{f_{i}}\text{-Modul }\forall i\in I.
\]
\end{lem}

\begin{proof}
\mbox{}
\begin{casenv}
\item[``$\Rightarrow$'' ] Endlich erzeugt hei�t $A^{n}\twoheadrightarrow M$, Lokalisierung
exakt also $A_{f_{i}}^{n}\twoheadrightarrow M_{f_{i}}$ exakt.
\item[``$\Leftarrow$''] $M_{f_{i}}$ werden von $\frac{m_{ij}}{f_{i}^{n_{ij}}}$, $j=1,\ldots,r_{i}$,
$m_{ij}\in M$, $n_{ij}\in\mathbb{N}_{0}$ als $A_{f_{i}}$-Modul
erzeugt.

$\Rightarrow N:=\langle m_{ij}\rangle_{A}\subset M$ ist endlich erzeugt
und $N_{f_{i}}=M_{f_{i}}$.

$\Rightarrow(M/N)_{\mathfrak{p}}=(M_{f_{i}}/N_{f_{i}})_{\mathfrak{p}}=0$
f�r alle Primideale $\mathfrak{p}\in\Spec A$.

$\Rightarrow$ (Lokal-Global-Prinzip aus der kommutativen Algebra)
$N=M$.

\end{casenv}
\end{proof}
\begin{rem*}
$X$ noethersches Schema. Dann ist $X$ als topologischer Raum noethersch.
\end{rem*}
\begin{proof}
F�r $X$ affin klar, sonst $X=\cup_{i=1}^{r}X_{i}$, $X_{i}=\Spec(A_{i})$
noethersch. Sei
\[
X\supseteq Z_{1}\supseteq\cdots\supseteq Z_{n}\supseteq\cdots
\]

absteigende Kette abgeschlossener Teilmengen. $(Z_{j}\cap X_{i})_{j}$
absteigende Kette abgeschlossener Teilmengen in $X_{i}$.

$\Longrightarrow$ (endliche �berdeckung) $\exists N$ d.d. $Z_{j}\cap X_{i}=Z_{N}\cap X_{i}$
f�r alle $j\geq N$.

$\Longrightarrow Z_{j}=Z_{N}$.
\end{proof}
\begin{cor}[17]
Sei $X$ (lokal) noethersches Schema, $U\subset X$ offenes Unterschema.
Dann ist $U$ ein (lokal) noethersches Schema.
\end{cor}

\begin{proof}
Lokal noethersch $X=\bigcup U_{i}$, $U_{i}\cap U=\bigcup D(f_{i})$.
Sei $X$ noethersch. Dann ist der topologische Raum $X$ noethersch.
Nach Lemma $I.20$ ist dann jede offene Teilmenge quasi-kompakt.
\end{proof}

\section{Generische Punkte }
\begin{prop}[18]
Die Abbildung
\begin{align*}
X & \longrightarrow\{Z\subset X\mid\text{abg., irred.}\}\\
x & \longmapsto\overline{\{x\}}
\end{align*}

ist eine Bijektion, d.h. jede irreduzible abgeschlossene Teilmenge
enth�lt genau einen generischen Punkt.
\end{prop}

\begin{proof}
Gilt f�r affine Schemata nach Korollar II.7. Sei $Z\subset X$ irreduzibel,
abgeschlossen sowie $U\subset X$ affin offen mit $Z\cap U\neq\emptyset$.

$\Longrightarrow\overline{Z\cap U}^{X}=Z$, da $Z$ irreduzibel.

$\Longrightarrow Z\cap U$ irreduzibel mit generischen Punkt $x$,
$\overline{\{x\}}^{Z\cap U}=Z\cap U$.

$\Longrightarrow\overline{\{x\}}^{X}=Z$.

Umgekehrt: Sei $z\in Z$ generischer Punkt.

$\Longrightarrow[U\subset X$ offen mit $U\cap Z\neq\emptyset$ $\Rightarrow z\in U]$
d.h. Eindeutigkeit im affinen Fall impliziert allgemeiner Fall.
\end{proof}
``Generische Punkte reduzieren gewisse Aussagen auf das Studium von
\emph{einem} Punkt''.
\begin{prop}[19]
Sei $f:X\rightarrow Y$ offener Morphismus von Schemata. Sei $Y=\overline{\{\eta\}}^{Y}$
irreduzibel. Dann:
\[
f^{-1}(\eta)\text{ irreduzibel}\Leftrightarrow X\text{ irreduzibel}
\]
\end{prop}

\begin{proof}
$f$ offen $\Rightarrow\overline{\{f^{-1}(x)\}}=f^{-1}(\overline{\{\eta\}})=f^{-1}(Y)=X$.
Mit Lemma I.14: $Z$ irreduzibel $\Leftrightarrow\overline{Z}$ irreduzibel.
\end{proof}
Topologische R�ume von Schemata sind fast nie Hausdorffsch, aber:
\begin{prop}[20]
Sei $X$ Schema. Dann ist der unterliegende topologische Raum ein
$T_{0}$-Raum, d.h.
\[
\forall x\neq y\in X\ \exists U\subset X\text{ offen, mit \textbf{entweder }}x\in U\text{ oder }y\in U.
\]
\end{prop}

\begin{proof}
Ohne Einschr�nkung: $X$ affin, $x=\mathfrak{p}_{x}$, $y=\mathfrak{p}_{y}\in\Spec(\Gamma(X,\mathcal{O}_{X}))$.
Falls $\mathfrak{p}_{x}\subsetneq\mathfrak{p}_{y}$ w�hle
\[
\mathfrak{p}_{x}\in U=X\backslash\underbrace{V(\mathfrak{p}_{y})}_{\ni\mathfrak{p}_{y}}
\]

andernfalls $\exists f\in\mathfrak{p}_{x}\backslash\mathfrak{p}_{y}$,
d.h. $U=D(f)$ enth�lt $y$ aber nicht $x$.
\end{proof}
Sp�ter: Separiertheit von Schemata als ``Hausdorffsch''-Ersatz.

\section{Reduzierte und ganze Schemata }
\begin{defn}[21]
Ein Schema $X$ hei�t 
\begin{enumerate}
\item \textbf{reduziert}, falls alle $\mathcal{O}_{X,x}$, $x\in X$, reduzierte
Ringe sind.
\item \textbf{ganz}, falls $X$ reduziert und irreduzibel ist.
\end{enumerate}
\end{defn}

\begin{prop}[22]
\mbox{}
\begin{enumerate}
\item $X$ schema ist reduziert (ganz) $\Leftrightarrow\Gamma(U,\mathcal{O}_{X})$
reduziert (integer) f�r alle $U\subseteq X$ offen.
\item Sei $X$ ganz. Dann ist der Halm $\mathcal{O}_{X,x}$ integer $\forall x\in X$.
(Die Umkehrung ist im Allgemeinen falsch!)
\end{enumerate}
\end{prop}

\begin{proof}
\mbox{}
\begin{enumerate}
\item \textbf{reduzibel,} ``$\Rightarrow$''. $f\in\Gamma(U,\mathcal{O}_{X})$
mit $f^{n}=0$. Angenommen, $f\neq0$. Dann gibt es ein $x\in U$
mit $f_{x}\neq0$ in $\mathcal{O}_{X,x}$, $f_{x}^{n}=0$. Widerspruch

\textbf{reduzibel,} ``$\Leftarrow$''. Sei $\overline{f}\in\mathcal{O}_{X,x}$
nilpotent. Dann gibt es ein $x\in U\subset X$ offen und $f\in\Gamma(U,\mathcal{O}_{X})$
mit $f_{x}=\overline{f}$. Ohne Einschr�nkng: $f$ nilpotent (mit
$U$ verkleinern sodass $f^{n}|_{U}=0$). Nach Voraussetzung ist dann
$f=0$, also $\overline{f}=0$.

\textbf{ganz,} ``$\Rightarrow$''. Sei $X$ ganz. Dann ist $U\subset X$
offen ganz nach den Definitionen. Daher reicht es zu zeigen, dass
$\Gamma(X,\mathcal{O}_{X})$ integer ist. Seien $f,g\in\mathcal{O}_{X}(X)$
mit $fg=0$. Dann ist $X=V(f)\cup V(g)$. $X$ ist irreduzibel, also
etwa $X=V(f)$. \emph{Behauptung}: $f=0$.

Da Verschwinden aufgrund des Garbenaxioms eine lokale Frage ist, setze
ohne Einschr�nking $X=\Spec A$ affin. Es folgt: $f\in\bigcap_{\Spec A}\mathfrak{p}=\sqrt{(0)}=0$.

\textbf{ganz, }``$\Leftarrow$''. $\Gamma(U,\mathcal{O}_{X})$ integer,
also reduziert. Nach (1, reduziert) ist $X$ reduziert. Angenommen
es gibt $\emptyset\neq U_{1},U_{2}\subset X$ offen mit $\emptyset=U_{1}\cap U_{2}$.
Nach den Garbenaxiomen enth�lt dann $\Gamma(U_{1}\cup U_{2},\mathcal{O}_{X})=\Gamma(U_{1},\mathcal{O}_{X})\times\Gamma(U_{2},\mathcal{O}_{X}$)
Nullteiler $(1,0)\cdot(0,1)=0$. Widerspruch.
\item Folgt aus 1, da $A$ integer, $0\notin S$. Es folgt: $A_{S}$ integer
($\subseteq\Quot(A)$).
\end{enumerate}
\end{proof}
\begin{rem*}
$X=\Spec A$ ganz $\Leftrightarrow A$ integer, $\eta\in X$ generischer
Punkt $\Leftrightarrow(0)\subset A$. Es ist $\mathcal{O}_{X,\eta}=A_{(0)}=\Quot(A)$,
d.h. f�r jedes ganze Schema $X$ gilt: $\mathcal{O}_{X,\eta}$ ist
K�rper (mit generischer Punkt $\eta$).
\end{rem*}
\begin{defn}[23]
$X$ ganz, $\eta\in X$ generischer Punkt. Dann hei�t $K(X):=\mathcal{O}_{X,\eta}$
der \textbf{Funktionenk�rper} von $X$.
\end{defn}


\end{document}
