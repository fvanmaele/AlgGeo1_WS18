%% LyX 2.3.2 created this file.  For more info, see http://www.lyx.org/.
%% Do not edit unless you really know what you are doing.
\documentclass[oneside,ngerman]{book}
\usepackage[T1]{fontenc}
\usepackage[utf8]{inputenc}
\setcounter{secnumdepth}{3}
\setcounter{tocdepth}{3}
\usepackage{amsmath}
\usepackage{amsthm}
\usepackage{amssymb}
\usepackage{wasysym}
\usepackage[all]{xy}

\makeatletter
%%%%%%%%%%%%%%%%%%%%%%%%%%%%%% Textclass specific LaTeX commands.
\theoremstyle{plain}
\ifx\thechapter\undefined
	\newtheorem{thm}{\protect\theoremname}
\else
	\newtheorem{thm}{\protect\theoremname}[chapter]
\fi
\theoremstyle{remark}
\newtheorem*{rem*}{\protect\remarkname}
\theoremstyle{definition}
\newtheorem{defn}[thm]{\protect\definitionname}

%%%%%%%%%%%%%%%%%%%%%%%%%%%%%% User specified LaTeX commands.

%% User-specified commands
\DeclareMathOperator{\rad}{rad}
\DeclareMathOperator{\Spec}{Spec}
\DeclareMathOperator{\maxspec}{MaxSpec}
\DeclareMathOperator{\Quot}{Quot}
\DeclareMathOperator{\im}{\mathrm{im}}
\DeclareMathOperator{\Hom}{\mathrm{Hom}}
\DeclareMathOperator{\Mor}{\mathrm{Mor}}
\DeclareMathOperator{\id}{\mathrm{id}}
\DeclareMathOperator{\res}{res}
\DeclareMathOperator{\Abb}{Abb}
\DeclareMathOperator{\nil}{nil}
\DeclareMathOperator{\supp}{supp}

%%sets
\DeclareMathOperator{\CC}{\mathbb{C}}
\DeclareMathOperator{\RR}{\mathbb{R}}
\DeclareMathOperator{\QQ}{\mathbb{Q}}
\DeclareMathOperator{\ZZ}{\mathbb{Z}}
\DeclareMathOperator{\NN}{\mathbb{N}}

%% categories
\DeclareMathOperator{\ouv}{\mathcal{O}uv}
\DeclareMathOperator{\set}{\underline{Set}}
\DeclareMathOperator{\ab}{\underline{Ab}}
\DeclareMathOperator{\cattop}{\underline{Top}}
\DeclareMathOperator{\cring}{\underline{CRing}}
\DeclareMathOperator{\ring}{\underline{Ring}}
\DeclareMathOperator{\psh}{\underline{\mathcal{PS}h}}
\DeclareMathOperator{\sh}{\underline{\mathcal{S}h}}
\DeclareMathOperator{\aff}{\underline{Aff}}
\DeclareMathOperator{\schs}{\underline{Sch/S}}
\DeclareMathOperator{\pres}{\underline{Prevar/S}}
\DeclareMathOperator{\prek}{\underline{Prevar/k}}
\DeclareMathOperator{\affk}{\underline{AffVar/k}}
\DeclareMathOperator{\topcp}{\underline{TopCP}}
\DeclareMathOperator{\Top}{\underline{Top}}

\makeatother

\usepackage{babel}
\providecommand{\definitionname}{Definition}
\providecommand{\remarkname}{Bemerkung}
\providecommand{\theoremname}{Theorem}

\begin{document}

\section{Prävarietäten als Schemata}

Wir wollen einen Funktor von der Kategorie der Prävarietäten in die
Kategorie der Schemata sodass, wenn wir eine geweissen Unterkategorie
von $\sh$ betrachten, eine Äquivalenz von Kategorien entsteht.\medskip{}

\textbf{Erinnerung: }$k=\overline{k}$: $\mathbb{A}_{k}^{2}=\Spec(k[X,Y])$
besteht aus
\begin{itemize}
\item Punte des $\mathbb{A}^{2}(k)$ $\leadsto$ maximale Ideale, 0-dimensionale
Teilmengen.
\item Irreduzible Kurve $f(x,y)=0$ $\leadsto$ Primideale, 1-dimensionale
Teilmengen.
\item Generischer Punkt 0 $\leadsto$ 2-dimensionale Teilmengen.
\end{itemize}
Wie können wir die zusätzlichen Punkte für den Funktor
\[
\pres\longrightarrow\schs
\]

präzisieren? Sei $X$ ein topologischer Raum, in dem alle Punkte abgeschlossen
sind. Betrachte 
\[
t(X)=\{Z\subset X\mid Z\text{ irreduzibel abgeschlossen}\},
\]

versehen mit der Topologie: I$t(X)\supset t(Z)$, $Z\subseteq_{\text{abg.}}X$
bilden die abgeschlossenen Mengen. Überprüfe: $Z_{1},Z_{2},Z_{i}\subset X$
abgeschlossen $\Longrightarrow t(\cap_{i}Z_{i})=\cap_{i}t(Z_{i})$,
$t(Z_{1}\cup Z_{2})=t(Z_{i})\cup t(Z_{2})$. Ist $f:X\rightarrow Y$
stetig, so auch
\begin{align*}
t(f):t(X) & \longrightarrow t(Y)\\
Z & \longmapsto\overline{f(Z)}
\end{align*}

denn:
\begin{enumerate}
\item $\overline{f(Z)}$ irreduzibel: Sei $\overline{f(Z)}=A_{1}\cup A_{2}$,
$A_{i}\neq\emptyset$. Dann existiert $z_{1},z_{2}\in Z$ mit $f(z_{i})\in A_{i}$,
denn sonst gilt $f(z)\subseteq A_{1}$. $Z\subseteq f^{-1}(f(Z))$
abgeschlossen. $\Longrightarrow Z=(f^{-1}(A_{1})\cap Z)\cup(f^{-1}(A_{2}\cap Z))$,
Widerspruch.
\item Sei $t(Y')\subseteq t(Y)$ abgeschlossen. $t(f)^{-1}(t(Y))=\{Z\in t(X)$,
$\overline{f(Z)}\in t(Y')\}$ denn:
\begin{itemize}
\item[,,$\subseteq$``] $\overline{f(Z)}\subset Y'$ $\Longrightarrow Z\in f^{-1}(\overline{f(Z))}\subset f^{-1}(Y)=t(f^{-1}(Y))$
\item[,,$\supseteq$``] $z\in f^{-1}(Y)$ abgeschlossen $\Longrightarrow f(Z)\in\overline{f(Z)}\subset\overline{Y'}\subset Y'$.
\end{itemize}
\end{enumerate}
Wir erhalten einen Funktor
\[
t:\topcp\longrightarrow\Top
\]

Die irreduziblen Mengen von $f(X)$ sind gerade die $t(X)$, $Z\subseteq X$
irreduzibel. $Z\in t(Z)$ ist der eindeutige generische Punkt. Sei
\begin{align*}
\alpha_{X}:X & \longrightarrow t(X)\\
x & \longmapsto\{x\}\text{ irred. abg.}
\end{align*}

So ist die Abbildung
\begin{align*}
\text{\{abg. Tm. von }f(X)\} & \longrightarrow\text{\{abg. Tm. von }X\}\\
A=t(Z) & \longmapsto\alpha_{X}^{-1}(A)=\{x\in X:\{x\}\in t(Z)\}=Z
\end{align*}

eine Bijektion. $\Longrightarrow\alpha_{X}$ ist Homömorphismus von
$X$ auf die abgeschlossenen Punkte $|t(X)|$ von $t(X)$ {[}irred.
abg. Teilmengen $Z$ von $X$, die in $t(X)$ abgeschlossen sind.{]}

Es ist $\{Z\}=t(Z')$ für ein $Z'\subset X$ abgeschlossen. $\Longrightarrow$
Nur ein Punkt $x\in X$ in $Z$, sonst $\{x\}\subsetneq Z\subset Z'$
beide in $t(Z')$.

Es ist $|t(X)|\subset t(Y)$ eine sehr dichte Menge (Bijektion oben).
\begin{thm}[31]
Der Funktor $X\mapsto(t(X),(\alpha_{X})_{\ast}\mathcal{O}_{X})$
induziert eine Äquivalenz von Kategorien:
\begin{align*}
t:\{\prek\} & \overset{1:1}{\longleftrightarrow}\text{\{integere }k\text{-Schemata v. endl. Typ\}}\\
\{\affk\} & \overset{1:1}{\longleftrightarrow}\text{\{affine }k\text{-Schemata v. endl. Typ}\}
\end{align*}
\end{thm}

\begin{proof}
Ist $X$ eine affine Varietät über $k$ mit $\Gamma(X)=A$, so ist
$X=\maxspec(A)$. $\Longrightarrow t(X)=\Spec A$ (vgl Kapitel I),
$\mathcal{O}_{X}(D(f))=A_{f}$, $f\in A$. $\Longrightarrow$ Behauptung
im affinen Fall.

Ist $f:X\rightarrow Y$ Morphismus von Prävarietäten, so erhalten
wir 
\begin{align*}
t(f):t(X) & \longrightarrow t(Y),\\
(\alpha_{Y})_{\ast}\mathcal{O}_{Y} & \longrightarrow t(f)_{\ast}((\alpha_{X})_{\ast}\mathcal{O}_{X})
\end{align*}

Morphismus lokal geringter Räume, da ein Morphismus von Garben auf
$X$ und $Y$ durch Komposition von Abbildungen gegeben ist!

Quasi-inverser Funktor $(X,\mathcal{O}_{X})\mapsto(X(k),\mathcal{O}_{X(k)}=\alpha^{-1}\mathcal{O}_{X})$
geringter Raum. \textbf{(1)} $\alpha^{-1}(U)=U\cap(X)\overset{1:1}{\longleftrightarrow}U$
offene Teilmenge. \textbf{Behauptung}: Bild $(X(k),\mathcal{O}_{X(k)})$
ist Raum mit Funktionen: Sei $V\subseteq U\subseteq X$ offen. \textbf{(2)}
Das Diagramm
\[
\xymatrix{\mathcal{O}_{X(k)}(U\cap X(k))\ar[d]_{\res}\ar[r] & \Abb(U\cap X(\xi),\xi)\ar[d]^{\res}\\
\mathcal{O}_{X(k)}(V\cap X(k))\ar@{^{(}->}[r] & \Abb(V\cap X(k),k)
}
\]

kommutiert. Dazu $f\in\mathcal{O}_{X(k)}(U\cap X(k))\overset{(1)}{=}\mathcal{O}_{X}(U)$,
wir assoziieren es der Abbildung 
\[
U\cap X(k)\longrightarrow k,\quad x\mapsto f(x):=\pi_{x}(f),
\]

mit
\[
\xymatrix{\pi_{x}:\mathcal{O}_{X}(U)\ar[r]\ar[d]^{\res} & \mathcal{O}_{X,x}\ar[r] & \kappa(x)=k\\
\mathcal{O}_{X}(V)\ar[ur]
}
\]

$\Longrightarrow(2)$. \textbf{(3)} $f,g$ mit derselben Funktion
\[
f\equiv g:U'\rightarrow k\overset{!}{\Longrightarrow}f=g
\]

Garbenaxiom $\Longrightarrow$ kann lokal überprüft werden: $U=\Spec A$
und $\pi_{x}(f)=\pi_{x}(g)$ für alle $x\in\maxspec$ 
\[
\Longrightarrow f-g\in\bigcap_{\mathfrak{m}\in\maxspec(A)}\mathfrak{m}=\nil(A)=0,
\]

da $A$ lokal reduzierte $k$-Algebra. Da sich $X$ durch endlich
viele affine Schemata der Form $\Spec A$, $A$ integer endlich erzeugte
$k$-Algebra, überdecken lässt, ist der Raum mit Funktion $X(k)$
eine Prävarietät. Die Konstruktion ist funktoriell, da jede Menge
von Schemata von endlichem Typ über $K$ abgeschlossene Punkte auf
abgeschlossene Punkt schickt nach Proposition 28.

Um zu zeigen, dass beide Funktoren Quasi-Inverse zueinander sind,
benutze den affinen (Varietät/Schemata) Fall, so wie die Garbenaxiome.
\end{proof}
\begin{rem*}[32]
\mbox{}Es gilt:

\begin{align*}
\kappa(x) & =\kappa(X(k))\\
\mathbb{A}_{k}^{n} & \longleftrightarrow\mathbb{A}(k)\\
\mathbb{P}_{k}^{n} & \longleftrightarrow\mathbb{P}^{n}(k)
\end{align*}
\end{rem*}

\section*{Unterschemata und Immersion (Einbettungen)}

\section{Offene und abgeschlossene Einbettung}
\begin{defn}[33]
Ein Morphismus $j:Y\rightarrow X$ von Schemata heißt \textbf{offene
Einbettung}, falls die unterliegende stetige Abbildung ein Homöomorphismus
von $Y$ auf eine \emph{offene} Menge $U\subset X$ ist, sowie der
Garbenhomomorphismus $\mathcal{O}_{X}\rightarrow j_{\ast}\mathcal{O}_{Y}$
einen Isomorphismus $\mathcal{O}_{X|U}\cong j_{\ast}\mathcal{O}_{Y}$
von Garben über $U$ induziert.
\end{defn}

,,$j$ induziert Isomorphismus zu $Y$ und offenen Unterschemata
$U$``
\begin{defn}[34]
Sei $(X,\mathcal{O}_{X})$ ein geringter Raum. Eine Untergarbe $\mathcal{I}\subset\mathcal{O}_{X}$
heißt \textbf{Idealgarbe}, falls $\Gamma(U,\mathcal{I})\unlhd\Gamma(U,\mathcal{O}_{X})$
Ideal ist für alle $U\subseteq X$ offen. Es bezeichne $\mathcal{O}_{X}/\mathcal{I}$
die Quotientengarbe assoziiert von der Prägarbe $U\mapsto\mathcal{O}_{X}(U)/\mathcal{I}(U)$.
Dies ist eine Prägarbe mit $\mathcal{O}_{X}\rightarrow\mathcal{O}_{X}/\mathcal{I}$
surjektiv, denn auf Halme:
\[
\underset{\underset{x\in U}{\longrightarrow}}{\lim}(\mathcal{O}_{X}(U)\twoheadrightarrow\mathcal{O}_{X}(U)/\mathcal{I}(U))=\mathcal{O}_{X,x}\twoheadrightarrow(\mathcal{O}_{X}/\mathcal{I})_{x}.
\]
\end{defn}

\begin{defn}[35]
Sei $X$ ein Schemata.
\begin{enumerate}
\item Ein \textbf{abgeschlossenes Unterschemata von $X$ }ist gegeben durch
eine abgeschlossene Menge $Z\subseteq X$ ($i:Z\rightarrow X$ Inklusion),
sowie eine Garbe $\mathcal{O}_{Z}$ auf $Z$, sodass $(Z,\mathcal{O}_{Z})$
ein Schemata und $i_{\ast}\mathcal{O}_{Z}\cong\mathcal{O}_{X}/I$
für eine Idealgarbe $I\subset\mathcal{O}_{X}$.
\item Ein Morphismus $i:Z\rightarrow X$ von Schemata heißt \textbf{abgeschlossene
Einbettung}, falls die unterliegende stetige Abbildung einen Homöomorphismus
zwischen $Z$ und eine abgeschlossene Teilmenge von $X$ ist, und
der Garbenhomomorphismus $i^{\flat}:\mathcal{O}_{X}\rightarrow i_{\ast}\mathcal{O}_{X}$
surjektiv ist.
\end{enumerate}
Ist $Z\subseteq X$ ein abgeschlossenes Unterschemata wie in (1),
so ist $(i,i^{\flat})$ eine abgeschlossene Einbettung. Umgekehrt
bestimmt jede abgeschlossene Einbettung einen Isomorphismus von seiner
Quelle auf ein eindeutiges abgeschlossenes Unterschemata seines Ziels.

\textbf{Warnung:} Nicht für jede Idealgarbe $\mathcal{I}$ ist
\[
(Z=\supp\mathcal{O}_{X}/\mathcal{I},\mathcal{O}_{X}/\mathcal{I})
\]

ein Schema. Später: gilt gdw. $\mathcal{I}$ quasi-kompakt ist.
\end{defn}

\begin{thm}[36]
Sei $X=\Spec A$. Dann ist die Abbildung
\begin{align*}
\text{\{Ideale }A\} & \overset{1:1}{\longleftrightarrow}\{\text{abg. Unterschemata von }X\}\\
\mathfrak{a} & \longmapsto V(\mathfrak{a})\cong\Spec(A/\mathfrak{a})
\end{align*}

eine Bijektion. Insbesondere ist jedes abgeschlossene Unterschemata
eines affinen Schematas affin.
\end{thm}


\end{document}
