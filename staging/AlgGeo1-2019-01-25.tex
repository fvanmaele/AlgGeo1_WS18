%% LyX 2.3.2 created this file.  For more info, see http://www.lyx.org/.
%% Do not edit unless you really know what you are doing.
\documentclass[oneside,ngerman]{book}
\usepackage[T1]{fontenc}
\usepackage[utf8]{inputenc}
\setcounter{secnumdepth}{3}
\setcounter{tocdepth}{3}
\usepackage{amsmath}
\usepackage{amsthm}
\usepackage{amssymb}
\usepackage[all]{xy}

\makeatletter
%%%%%%%%%%%%%%%%%%%%%%%%%%%%%% Textclass specific LaTeX commands.
\theoremstyle{plain}
\ifx\thechapter\undefined
	\newtheorem{thm}{\protect\theoremname}
\else
	\newtheorem{thm}{\protect\theoremname}[chapter]
\fi
\theoremstyle{definition}
\newtheorem{example}[thm]{\protect\examplename}
\theoremstyle{plain}
\newtheorem{prop}[thm]{\protect\propositionname}
\theoremstyle{remark}
\newtheorem{rem}[thm]{\protect\remarkname}
\theoremstyle{definition}
\newtheorem{defn}[thm]{\protect\definitionname}

%%%%%%%%%%%%%%%%%%%%%%%%%%%%%% User specified LaTeX commands.
%% User-specified commands
\DeclareMathOperator{\rad}{rad}
\DeclareMathOperator{\Spec}{Spec}
\DeclareMathOperator{\maxspec}{MaxSpec}
\DeclareMathOperator{\Quot}{Quot}
\DeclareMathOperator{\im}{\mathrm{im}}
\DeclareMathOperator{\Hom}{\mathrm{Hom}}
\DeclareMathOperator{\Mor}{\mathrm{Mor}}
\DeclareMathOperator{\id}{\mathrm{id}}
\DeclareMathOperator{\res}{res}
\DeclareMathOperator{\Abb}{Abb}
\DeclareMathOperator{\supp}{supp}
\DeclareMathOperator{\red}{red}
\DeclareMathOperator{\op}{op}
\DeclareMathOperator{\obj}{Obj}
\DeclareMathOperator{\nil}{nil}

%% sets
\DeclareMathOperator{\CC}{\mathbb{C}}
\DeclareMathOperator{\RR}{\mathbb{R}}
\DeclareMathOperator{\QQ}{\mathbb{Q}}
\DeclareMathOperator{\ZZ}{\mathbb{Z}}
\DeclareMathOperator{\NN}{\mathbb{N}}

%% categories
\DeclareMathOperator{\ouv}{\mathcal{O}uv}
\DeclareMathOperator{\set}{\underline{Set}}
\DeclareMathOperator{\ab}{\underline{Ab}}
\DeclareMathOperator{\cattop}{\underline{Top}}
\DeclareMathOperator{\cring}{\underline{CRing}}
\DeclareMathOperator{\ring}{\underline{Ring}}
\DeclareMathOperator{\psh}{\underline{\mathcal{PS}h}}
\DeclareMathOperator{\sh}{\underline{\mathcal{S}h}}
\DeclareMathOperator{\aff}{\underline{Aff}}
\DeclareMathOperator{\sch}{\underline{Sch}}
\DeclareMathOperator{\schk}{\underline{Sch/k}}
\DeclareMathOperator{\schs}{\underline{Sch/S}}
\DeclareMathOperator{\schsn}{\underline{Sch/S_{0}}}
\DeclareMathOperator{\schz}{\underline{Sch/\mathbb{Z}}}
\DeclareMathOperator{\pres}{\underline{Prevar/S}}
\DeclareMathOperator{\prek}{\underline{Prevar/k}}
\DeclareMathOperator{\affk}{\underline{AffVar/k}}
\DeclareMathOperator{\topcp}{\underline{TopCP}}
\DeclareMathOperator{\Top}{\underline{Top}}
\DeclareMathOperator{\grp}{\underline{Grp}}
\DeclareMathOperator{\func}{\underline{Func}}

\makeatother

\usepackage{babel}
\providecommand{\definitionname}{Definition}
\providecommand{\examplename}{Beispiel}
\providecommand{\propositionname}{Satz}
\providecommand{\remarkname}{Bemerkung}
\providecommand{\theoremname}{Theorem}

\begin{document}
\begin{example}[25]
In der Kategorie $\mathcal{C}=\set$ gilt für:
\[
\xymatrix{X\ar[r]^{f,g}\ar[rd]_{u} & Y\ar[d]^{v}\\
 & S
}
\]
\begin{itemize}
\item[] $\begin{array}{rl}
\Delta_{u}:X & \longrightarrow X\times X=\{(x,x')\in X\times X\mid u(x)=u(x')\}\\
x & \longmapsto(x,x)
\end{array}$
\item[] $\begin{array}{rl}
\Gamma_{j}:X & \longrightarrow X\times_{S}Y=\{(x,y)\in X\times Y\mid u(x)=v(y)\}\\
x & \longmapsto(x,f(x))
\end{array}$
\item[] $\ker(f,g)=\{x\in X\mid f(x)=g(x)\}$
\end{itemize}
Da $p\circ\Gamma_{j}=\id_{X}$, sind $\Gamma_{j}$ und $\Delta_{X/S}=\Gamma_{\id_{X}}$
Monomorphismen.
\end{example}

\begin{prop}[26]
Seien $f,g:X\rightarrow Y$ $S$-Morphismen.
\begin{enumerate}
\item $\Delta_{X/S}=\Gamma_{\id_{X}}$,

$\Gamma_{f}=\ker(\xymatrix{X\times_{S}Y\ar@<1ex>[r]^{q}\ar@<-1ex>[r]_{f\circ p} & Y}
)\longrightarrow X\times_{S}Y$ kanonisch.
\item Alle Rechtecke des folgenden kommutativen Diagramms sind kartesisch:
\[
\xymatrix{\ker(f,g)\ar[r]^{\text{canon.}}\ar[d] & X\ar[r]^{f}\ar[d]_{\Gamma_{f}} & Y\ar[d]^{\Delta_{Y/S}}\\
X\ar[r]_{\Gamma_{g}} & X\times_{S}Y\ar[r]_{f\times\id_{S}} & Y\times_{S}Y
}
\]
\item Sei $s:S\rightarrow X$ ein Schnitt von $f$ ($f\circ s=\id_{S}$).
Dann ist das folgende Diagramm kartesisch:
\[
\xymatrix{S\ar[r]^{s}\ar[d]_{s} & X\ar[d]^{\Gamma_{s\circ f}}\\
X\ar[r]_{\Delta_{X/S}} & \quad X\times_{S}X
}
\]
\end{enumerate}
\end{prop}

\begin{proof}
Nach dem Yoneda-Lemma reicht es, denn Fall $\mathcal{C}=\set$ zu
verifizieren. Dies ist elementar aufgrund der Beschreibung in Beispiel
25.
\end{proof}
Insbesondere existiert $\ker(f,g)$ stets!

\section{Diagonal für Schemata}
\begin{prop}[27]
Für affine $S$-Schemata
\begin{align*}
X & =\Spec(B)\overset{u}{\longrightarrow}S=\Spec(R)\\
Y & =\Spec(A)\overset{v}{\longrightarrow}S
\end{align*}

und $S$-Morphismus $f=\Spec\varphi:X\rightarrow Y$ zu einem $R$-Algebra
Morphismus $\varphi:A\rightarrow B$, entsprechen $\Delta_{X/S}$
und $\Gamma_{f}$ den folgenden surjektiven Ringhomomorphismen:
\[
\begin{array}{rl}
\Delta_{B/R}:B\otimes_{R}B & \longrightarrow B\\
b\otimes b' & \longmapsto bb'
\end{array},\qquad\begin{array}{rl}
\Gamma_{\varphi}:A\otimes_{R}B & \longrightarrow B\\
a\otimes b & \longmapsto\varphi(a)b
\end{array}.
\]

Insbesondere sind $\Delta_{X/S}$, $\Gamma_{j}$ abgeschlossene Immersionen.
\end{prop}

Im allgemeinen sind $\Delta_{X/S}$, $\Gamma_{f}$ Immersionen (nicht
notwendig abgeschlossen!): Seien $Z,Z'\subset X$ Unterschemata. $\Longrightarrow Z\times_{S}Z'\subset X\times_{S}X$
Unterschemata (Immersionen und stabil unter Basiswechsel und Komposition),
und
\[
Z\cap Z'=\Delta_{X/S}^{-1}(Z\times_{S}Z')\qquad(*)
\]

\begin{prop}
Seien $X,Y\in\schs$, $f,g:X\rightarrow Y$ $S$-Morphismen. Dann
sind $\Delta_{X/S}$, $\Gamma_{f}$, $\ker(f,g)\rightarrow X$ Immersionen.
\end{prop}

\begin{proof}
Es reicht zu zeigen: $\Delta_{X/S}$ ist eine Immersion (und $(2)$
in Proposition 26, da ,,Immersion`` stabil ist unter Basiswechsel)
lokal bzgl. Ziel. Sei also ohne Einschränkung $S$ affin. Falls $X=\bigcup_{i\in I}U_{i}$
offene Überdeckung, dann ist $\Delta_{X/S}(X)=\bigcup_{i\in I}U_{i}\times_{S}U_{i}$
offene Überdeckung. 
\[
\xymatrix{X\ar[r]_{\text{abg. Imm.}} & \bigcup_{i\in I}(U_{i}\times_{S}U_{i})\ar@{^{(}->}_{\text{off. Imm.}}[r] & X\times_{S}X}
\]

$(*)\Longrightarrow$ ohne Einschränkung, $X$ affin. Wende nun Proposition
27 an.
\end{proof}
Das Unterschema
\begin{itemize}
\item $X\cong\Delta_{X/S}(X)\subset X\times_{S}X$ heißt die \textbf{Diagonale
von $X\times_{S}X$}.
\item $\Gamma_{f}(X)\subset X\times_{S}Y$ heißt der \textbf{Graph von $f$}.
\end{itemize}
\begin{rem}[29]
\mbox{}
\begin{enumerate}
\item Ein Unterschema $T\subset X\times_{S}Y$ ist der Graph eines $S$-Morphismus
$f:X\rightarrow Y$ genau dann, wenn $p|_{T}:T\rightarrow X\times_{S}Y\underset{p}{\rightarrow}X$
ein Isomorphismus ist, \emph{denn }$f=q\circ(p|_{T})^{-1}$.
\item Im Allgemeinen ist die mengentheoretische Inklusion
\[
\Delta_{X/S}(X)\subset\{z\in X\times_{S}X\mid f(z)=g(z)\}
\]
\emph{keine} Gleichheit!
\end{enumerate}
\end{rem}


\section{Separierte Morphismen}

Erinnerung: Für einen topologischen Raum $X$ sind äquivalent:
\begin{enumerate}
\item $X$ ist Hausdorff.
\item $\Delta\subset X\times X$ ist abgeschlossen bzgl. der Produkttopolgie.
\item Für jedes Paar von stetigen Abbildungen $f,g:Y\rightarrow X$ ist
$\ker(f,g)\subset X$ abgeschlossen.
\item Für jeder stetige Morphismus $f:Y\rightarrow X$ ist $\Gamma_{f}\subset X\times Y$
abgeschlossen.
\end{enumerate}
Für ein Schema $X$ ist der unterliegende topologische Raum selten
Hausdorff, aber $(2)-(4)$ geben sinnvolle Konzepte für Schemata (im
Allgemeinen ist die Produkttopologie ungleich der Faserprodukttopologie).
\begin{defn}[30]
Ein Morphismus $v:Y\rightarrow S$ von Schemata heißt separiert,
falls folgende äquivalente Bedingungen erfüllt sind.
\begin{enumerate}
\item $\Delta_{Y/S}$ ist eine \emph{abgeschlossene} Immersion.
\item Für jedees Paar $f,g:X\rightarrow Y$ ist $\ker(f,g)\subset X$ ein
abgeschlossenes Unterschema.
\item Für jeden $S$-Morphismus $f:X\rightarrow Y$ ist $\Gamma_{f}$ eine
abgeschlossene Immersion.
\end{enumerate}
Dann heißt auch $Y$ ist \textbf{separiert über $S$}. Ein Schema
$Y$ heißt \textbf{separiert}, falls es separiert über $\mathbb{Z}$
ist.
\end{defn}

\begin{proof}
Die Äquivalenz folgt nach Proposition 26, und dass ,,abgeschlossene
Immersion`` stabil unter Basiswechsel ist.
\end{proof}
Nach Proposition 27 ist jeder Morphismus zwischen affinen Schemata
separiert. Insbesondere ist jedes affine Schemata separiert.
\begin{prop}[31]
Seien $X,Y\in\schs$, $Y$ separiert über $S$, $U\subset X$ offenes
dichtes Unterschema, $f,g:X\rightarrow Y$ $S$-Morphismus mit $f|_{U}=g|_{U}$.
Dann ist $f|_{X_{\red}}=g|_{X_{\red}}$.
\end{prop}

\begin{proof}
Nach Voraussetzung ist $U\subseteq\ker(f,g)$. Da $Y$ separiert ist
über $S$, ist $\ker(f,g)\subset X$ abgeschlossenes Unterschema.
Da $U$ dicht ist in $X$, ist der unterliegende topologische Raum
von $\ker(f,g)$ gleich $X$. $\Longrightarrow X_{\red}\subseteq\ker(f,g)$
als Schema.
\end{proof}
\begin{example}[32]
Sei
\[
\,\xymatrix{\ar@{-}[r] & :\ar@{-}[r] & \,}
\]

affine Gerade mit Doppelpunkt (siehe Beispiel 11, III.5) ist \emph{nicht}
separiert: $V\subset U$ offen, nicht abgeschlossen.
\[
j,j':U\longrightarrow U\cup_{V}U\quad\Rightarrow\quad\ker(j,j')=V\subset U\subset X\text{\,nicht abg.!}
\]
\end{example}

\begin{rem}[33]
$\mathbb{P}$ sei eine Eigenschaft von Morphismen, sodass gilt:
\begin{itemize}
\item stabil unter Komposition und Basiswechsel;
\item jede (abgeschlossene) Immersion erfült $\mathbb{P}$.
\end{itemize}
Für jedes kommutative Diagramm:
\[
\xymatrix{X\ar[r]^{f}\ar[rd]_{u} & Y\ar[d]^{v}\\
 & S
}
\]

mit $u$ erfüllt $\mathbb{P}$ (und $v$ seperariert) $\Longrightarrow f$
erfüllt $\mathbb{P}$, da:
\[
f:X\xrightarrow[\text{(abg.) Imm. erfüllt }\mathbb{P}]{\Gamma_{f}}X\times_{S}Y\xrightarrow[\text{Basisw. erfüllt }\mathbb{P}]{q}Y
\]

erfüllt $\mathbb{P}$, wegen stabil unter Komposition.
\end{rem}

\begin{prop}
\mbox{}
\begin{enumerate}
\item Jeder Monomorphismsu von Schemata (insbesondere jede Immersion) ist
separiert.
\item Die Eigenschaft ,,separiert`` ist stabil unter Komposition, stabil
unter Basiswechsel, und lokal bzgl. Ziel.
\item Ist die Komposition $X\rightarrow Y\rightarrow Z$ zweier Morphismen
separiert, so auch $X\rightarrow Y$.
\item $f:X\rightarrow Y$ ist seperariert genau dann, wenn $f_{\red}:X_{\red}\rightarrow Y_{\red}$
separiert ist.
\end{enumerate}
\end{prop}

\begin{proof}
\mbox{}
\begin{enumerate}
\item Wenn $f$ Monomorphismus ist (d.h. injektiv auf $T$-wertigen Punkten
für alle Schemata $T$), dann ist $\Delta_{f}$ Isomorphismus (d.h.
bijektiv auf allen $T$-wertigen Punkte für alle $T$). Insbesondere
ist $\Delta_{f}$ eine abgeschlossene Immersion.
\item Seien $f:X\rightarrow Y$, $g:Y\rightarrow Z$ separierte Schemata-Morphismen,
$p,q:X\times_{Y}X\rightarrow X$ die zwei Projektionen. Das folgende
Diagramm ist kommutativ, und das rechte Viereck ist kartesisch (überprüfe
in $\set$):
\[
\xymatrix{X\ar[r]^{\Delta_{f}}\ar[rd]_{\Delta_{g\circ f}} & X\times_{Y}X\ar[r]^{f\circ p=f\circ q}\ar[d]|-{(p,q)_{Z}} & Y\ar[d]^{\Delta_{g}}\\
 & X\times_{Z}X\ar[r]_{f\times f} & Y\times_{Z}Y.
}
\]
Da $\Delta_{g}$ abgeschlossene Immersion, ist $(p,q)_{Z}$ abgeschlossene
Immersion $\Longrightarrow$ die Komposition $\Delta_{g\circ f}$
ist abgeschlossene Immersion $\Longrightarrow g\circ f$ ist separiert.

$\Delta_{f}$ abgeschlossene Immersion $\Longrightarrow\Delta_{f_{(S')}}$
ist abgeschlossene Immersion. Dies zeigt das ,,separiert`` abgeschlossen
ist unter Basiswechsel. Weiter ist ,,separiert`` lokal bzgl. Ziel,
da dies gilt für ,,abgeschlossene Immersion``.
\item Folgt aus (1), (2) nach Bemerkung 33. ($u=``\circ"$, $v=Y\rightarrow Z$,
$f:X\rightarrow Y$)
\item Sei $f:X\rightarrow S$ Morphismus, $i:X_{\red}\rightarrow X$ kanonische
Immersion. Dann ist $i$ surjektive Immersion, also ein universeller
Homöomorphismus. Identifizieren von $X_{\red}\times_{S_{\red}}X_{\red}$
mit $X_{\red}\times_{S}X_{\red}$ liefert $\Delta_{f}\circ i=(i\times_{S}i)\circ\Delta_{f_{\red}}$.
$\Longrightarrow\Delta_{f}$ ist abgeschlossene Immersion genau dann
wenn $\Delta_{f_{\red}}$ abgeschlossene Immersion.
\end{enumerate}
\end{proof}

\end{document}
