%% LyX 2.3.2 created this file.  For more info, see http://www.lyx.org/.
%% Do not edit unless you really know what you are doing.
\documentclass[oneside,ngerman]{book}
\usepackage[T1]{fontenc}
\usepackage[utf8]{inputenc}
\setcounter{secnumdepth}{3}
\setcounter{tocdepth}{3}
\usepackage{amsmath}
\usepackage{amsthm}
\usepackage{amssymb}
\usepackage[all]{xy}

\makeatletter
%%%%%%%%%%%%%%%%%%%%%%%%%%%%%% Textclass specific LaTeX commands.
\theoremstyle{plain}
\ifx\thechapter\undefined
	\newtheorem{thm}{\protect\theoremname}
\else
	\newtheorem{thm}{\protect\theoremname}[chapter]
\fi
\theoremstyle{definition}
\newtheorem{defn}[thm]{\protect\definitionname}
\theoremstyle{definition}
\newtheorem{example}[thm]{\protect\examplename}

%%%%%%%%%%%%%%%%%%%%%%%%%%%%%% User specified LaTeX commands.
%% User-specified commands
\DeclareMathOperator{\rad}{rad}
\DeclareMathOperator{\Spec}{Spec}
\DeclareMathOperator{\maxspec}{MaxSpec}
\DeclareMathOperator{\Quot}{Quot}
\DeclareMathOperator{\im}{\mathrm{im}}
\DeclareMathOperator{\Hom}{\mathrm{Hom}}
\DeclareMathOperator{\Mor}{\mathrm{Mor}}
\DeclareMathOperator{\id}{\mathrm{id}}
\DeclareMathOperator{\res}{res}
\DeclareMathOperator{\Abb}{Abb}
\DeclareMathOperator{\supp}{supp}
\DeclareMathOperator{\red}{red}
\DeclareMathOperator{\op}{op}
\DeclareMathOperator{\obj}{Obj}
\DeclareMathOperator{\nil}{nil}

%% sets
\DeclareMathOperator{\CC}{\mathbb{C}}
\DeclareMathOperator{\RR}{\mathbb{R}}
\DeclareMathOperator{\QQ}{\mathbb{Q}}
\DeclareMathOperator{\ZZ}{\mathbb{Z}}
\DeclareMathOperator{\NN}{\mathbb{N}}

%% categories
\DeclareMathOperator{\ouv}{\mathcal{O}uv}
\DeclareMathOperator{\set}{\underline{Set}}
\DeclareMathOperator{\ab}{\underline{Ab}}
\DeclareMathOperator{\cattop}{\underline{Top}}
\DeclareMathOperator{\cring}{\underline{CRing}}
\DeclareMathOperator{\ring}{\underline{Ring}}
\DeclareMathOperator{\psh}{\underline{\mathcal{PS}h}}
\DeclareMathOperator{\sh}{\underline{\mathcal{S}h}}
\DeclareMathOperator{\aff}{\underline{Aff}}
\DeclareMathOperator{\sch}{\underline{Sch}}
\DeclareMathOperator{\schk}{\underline{Sch/k}}
\DeclareMathOperator{\schs}{\underline{Sch/S}}
\DeclareMathOperator{\schsn}{\underline{Sch/S_{0}}}
\DeclareMathOperator{\schz}{\underline{Sch/\mathbb{Z}}}
\DeclareMathOperator{\pres}{\underline{Prevar/S}}
\DeclareMathOperator{\prek}{\underline{Prevar/k}}
\DeclareMathOperator{\affk}{\underline{AffVar/k}}
\DeclareMathOperator{\topcp}{\underline{TopCP}}
\DeclareMathOperator{\Top}{\underline{Top}}
\DeclareMathOperator{\grp}{\underline{Grp}}
\DeclareMathOperator{\func}{\underline{Func}}

\makeatother

\usepackage{babel}
\providecommand{\definitionname}{Definition}
\providecommand{\examplename}{Beispiel}
\providecommand{\theoremname}{Theorem}

\begin{document}
\begin{thm}[45]
Sei $\mathcal{O}_{K}$ Bewertungsring, $K=\Quot(\mathcal{O}_{X})$,
$X/\mathcal{O}_{K}$ eigentlich. Dann ist $X_{\mathcal{O}_{K}}(\mathcal{O}_{K})\rightarrow X_{K}(K)$
bijektiv.
\end{thm}

\textbf{Bewertungskriterium }(vgl. Hartshorne, Lin 3.26)
\[
\xymatrix{\Spec K\ar[r]\ar[d] & X\ar[d]^{f}\\
\Spec\mathcal{O}_{K}\ar@{-->}[ur]|-{\exists!}\ar[r] & Y
}
\]

$f$ eigentlich $\Longleftrightarrow$ universelle Eigenschaft oben
erfüllt. (Theorem auf $X\times_{Y}\Spec\mathcal{O}_{K}\rightarrow\Spec\mathcal{O}_{K}$).
\begin{thm}[46, Lin III 3.30]
Jeder projektive Morphismus ist eigentlich.
\end{thm}

Zum Beispiel: abelsche Varietäten, etwa elliptische Kurven. Theorem
46 $\Longrightarrow E(\mathbb{Q}_{p})=E(\mathbb{Z}_{p})$.

\chapter{Dimensionen}

\section{Allgemeine Schemata}
\begin{defn}[1]
Für einen topologischen Raum $X$ ist die (Krull-)Dimension das Supremum
der Länge aller Ketten
\[
Z_{0}\subsetneq Z_{1}\subsetneq\cdots\subsetneq Z_{n}\subseteq X
\]

irreduzibler abgeschlossener Teilmengen $Z_{i}$. $X$ sei \textbf{von
Dimension $n$}, falls alle irreduzible Komponenten von $X$ die Dimension
$n$ haben ($\dim\emptyset=-\infty$, sonst $\dim X\in\mathbb{N}\cup\{+\infty\}$).

Die Dimension eines Schemas ist per Definition die Dimension des unterliegenden
topologischen Raums, also $\dim X=\dim X_{\red}$.
\end{defn}

\begin{example}[2]
\mbox{}
\begin{enumerate}
\item 
\end{enumerate}
\end{example}


\end{document}
