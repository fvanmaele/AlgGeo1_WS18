%% LyX 2.3.2 created this file.  For more info, see http://www.lyx.org/.
%% Do not edit unless you really know what you are doing.
\documentclass[oneside,ngerman]{book}
\usepackage[T1]{fontenc}
\usepackage[utf8]{inputenc}
\setcounter{secnumdepth}{3}
\setcounter{tocdepth}{3}
\usepackage{amsmath}
\usepackage{amsthm}
\usepackage{amssymb}
\usepackage[all]{xy}

\makeatletter
%%%%%%%%%%%%%%%%%%%%%%%%%%%%%% Textclass specific LaTeX commands.
\theoremstyle{plain}
\ifx\thechapter\undefined
	\newtheorem{thm}{\protect\theoremname}
\else
	\newtheorem{thm}{\protect\theoremname}[chapter]
\fi
\theoremstyle{plain}
\newtheorem{prop}[thm]{\protect\propositionname}
\theoremstyle{plain}
\newtheorem{cor}[thm]{\protect\corollaryname}
\theoremstyle{plain}
\newtheorem{lem}[thm]{\protect\lemmaname}
\theoremstyle{definition}
\newtheorem{example}[thm]{\protect\examplename}
\theoremstyle{definition}
\newtheorem{defn}[thm]{\protect\definitionname}

%%%%%%%%%%%%%%%%%%%%%%%%%%%%%% User specified LaTeX commands.
%% User-specified commands
\DeclareMathOperator{\rad}{rad}
\DeclareMathOperator{\Spec}{Spec}
\DeclareMathOperator{\maxspec}{MaxSpec}
\DeclareMathOperator{\Quot}{Quot}
\DeclareMathOperator{\im}{\mathrm{im}}
\DeclareMathOperator{\Hom}{\mathrm{Hom}}
\DeclareMathOperator{\Mor}{\mathrm{Mor}}
\DeclareMathOperator{\id}{\mathrm{id}}
\DeclareMathOperator{\res}{res}
\DeclareMathOperator{\Abb}{Abb}
\DeclareMathOperator{\nil}{nil}
\DeclareMathOperator{\supp}{supp}
\DeclareMathOperator{\red}{red}
\DeclareMathOperator{\op}{op}
\DeclareMathOperator{\obj}{Obj}

%%sets
\DeclareMathOperator{\CC}{\mathbb{C}}
\DeclareMathOperator{\RR}{\mathbb{R}}
\DeclareMathOperator{\QQ}{\mathbb{Q}}
\DeclareMathOperator{\ZZ}{\mathbb{Z}}
\DeclareMathOperator{\NN}{\mathbb{N}}

%% categories
\DeclareMathOperator{\ouv}{\mathcal{O}uv}
\DeclareMathOperator{\set}{\underline{Set}}
\DeclareMathOperator{\ab}{\underline{Ab}}
\DeclareMathOperator{\cattop}{\underline{Top}}
\DeclareMathOperator{\cring}{\underline{CRing}}
\DeclareMathOperator{\ring}{\underline{Ring}}
\DeclareMathOperator{\psh}{\underline{\mathcal{PS}h}}
\DeclareMathOperator{\sh}{\underline{\mathcal{S}h}}
\DeclareMathOperator{\aff}{\underline{Aff}}
\DeclareMathOperator{\sch}{\underline{Sch}}
\DeclareMathOperator{\schs}{\underline{Sch/S}}
\DeclareMathOperator{\schs0}{\underline{Sch/S_{0}}}
\DeclareMathOperator{\pres}{\underline{Prevar/S}}
\DeclareMathOperator{\prek}{\underline{Prevar/k}}
\DeclareMathOperator{\affk}{\underline{AffVar/k}}
\DeclareMathOperator{\topcp}{\underline{TopCP}}
\DeclareMathOperator{\Top}{\underline{Top}}
\DeclareMathOperator{\grp}{\underline{Grp}}
\DeclareMathOperator{\func}{\underline{Func}}

\makeatother

\usepackage{babel}
\providecommand{\corollaryname}{Korollar}
\providecommand{\definitionname}{Definition}
\providecommand{\examplename}{Beispiel}
\providecommand{\lemmaname}{Lemma}
\providecommand{\propositionname}{Satz}
\providecommand{\theoremname}{Theorem}

\begin{document}
\begin{prop}[44]
\mbox{}
\begin{enumerate}
\item Der geringte Raum $X_{\red}=(X,\mathcal{O}_{X}/\mathcal{N})$ ist
ein Schema, also ein abgeschlossenes Unterschema von $X$ mit demselben
topologischen Raum wie $X$.
\item Falls $X'\subset X$ ein weiteres solches Unterschema ist, dann gibt
es eine abgeschlossene Einbettung $f:X_{\red}\rightarrow X'$, sodass
das Diagramm
\[
\xymatrix{X_{\red}\ar@{^{(}->}[r]\ar[d] & X\\
X'\ar@{^{(}->}[ur]
}
\]
kommutiert.
\item $X_{\red}$ ist reduziert und heißt das \textbf{unterliegend reduzierte
Unterschema von $X$.}
\item Falls $X=\Spec A$ affin, gilt $X_{\red}=\Spec(A/\nil(A))$.
\end{enumerate}
\end{prop}

\begin{proof}
Ohne Einschränkung sei $X=\Spec A$ $\Longrightarrow U\mapsto\nil(\Gamma(U,\mathcal{O}_{X}))$
ist bereits eine Garbe, da 
\[
\nil(\mathcal{O}_{X}(D(f))=\nil(A_{f})=\nil(A)A_{f}
\]

für alle $f\in A$. $\Longrightarrow X_{\red}=\Spec(A/\nil A)$ offensichtlich
reduziert.\textbf{ Universelle Eigenschaft:} zu zeigen $\mathcal{O}_{X}\rightarrow\mathcal{O}_{X}/\mathcal{N}$
faktorisiert:
\[
\xymatrix{\mathcal{O}_{X}\ar[r]\ar[rd] & \mathcal{O}_{X}/\mathcal{N}\\
 & \mathcal{O}_{X'}\ar[u]
}
,
\]

d.h. $\ker(\mathcal{O}_{X}\rightarrow\mathcal{O}_{X'})\subset\mathcal{N}$.
Es reicht zu zeigen:
\[
\ker(\mathcal{O}_{X}(U)\rightarrow\mathcal{O}_{X'}(U))\subset\Gamma(U,\mathcal{N})
\]

für alle $U$ offen affin. Ohne Einschränkung $X=\Spec A$, $X'$
abgeschlossenes Unterschema $\Longrightarrow$ affin: $X'=\Spec B$.

Zu zeigen: $\ker(A\rightarrow B)\subset\nil A$. Da nach Voraussetzung
$\Spec B\rightarrow\Spec A$ bijektiv ist, folgt:
\[
\ker(A\rightarrow B)\subset\bigcap_{\mathfrak{g}\in\Spec A}\mathfrak{g}=\nil A
\]
\end{proof}
\begin{cor}[45]
$(X_{\red},i_{X}:X_{red}\rightarrow X)$ ist durch die universelle
Eigenschaft eindeutig bis auf eindeutige Isomorphie bestimmt.
\end{cor}

\begin{lem}[46]
Jede Einbettung $i:Z\rightarrow X$ ist ein Monomorphismus in $\sch$.
\end{lem}

\begin{proof}
\mbox{}
\begin{itemize}
\item Stetige Abbildung $Z\hookrightarrow X$ klar.
\item Die Garbenabbildung $i_{Y}^{\#}$ ist surjektiv.
\end{itemize}
\end{proof}
%
\begin{proof}[Beweis von Korollar 45]
Sei $X'_{\red}$ ein weiteres Schema mit universeller Eigenschaft
\[
\exists f:X_{\red}\rightarrow X'_{\red},\quad g:X'_{\red}\rightarrow X_{\red}
\]

so dass
\[
\xymatrix{X_{\red}\ar[r]^{f}\ar[rd]_{i_{X}} & X'_{\red}\ar[r]^{g}\ar[d]^{i_{X'}} & X_{\red}\ar[ld]^{i_{X}}\\
 & X
}
,\quad i_{X}\circ(g\circ f)=i_{X}\circ\id_{X_{\red}}
\]

$i_{X}$ Monomorphismus $\Longrightarrow g\circ f=\id_{X_{\red}}=f\circ g$.
Auf $(X_{\red},i_{X})=\{\id\}$ $\Longrightarrow$ Eindeutig.
\end{proof}
$(\cdot)_{\red}$ ist ein Funktor, wie die folgende Proposition zeigt:
\begin{prop}[47]
Sei $f:X\rightarrow Y$ ein Schemata-Morphismus. Dann gibt es:
\[
\xymatrix{X_{\red}\ar[r]^{f_{\red}} & Y_{\red}\\
X_{\red}\ar@{^{(}->}[r]^{i_{X}}\ar[d]_{f_{\red}} & X\ar[d]^{f}\\
Y_{\red}\ar@{^{(}->}[r]_{i_{Y}} & Y
}
,\quad\text{d.d.}
\]

Für weitere Morphismen $g:Y\rightarrow Z$ gilt
\[
(g\circ f)_{\red}=g_{\red}\circ f_{\red}.
\]
\end{prop}

\begin{proof}
$i_{Y}$ Monomorphismus $\Longrightarrow f_{\red}$ eindeutig. \textbf{Existenz:
}Nach Verklebungs-Lemma $\Longrightarrow$ ohne Einschränkung $X=\Spec A$,
$Y=\Spec B$, $f\hat{=}\ \varphi:B\rightarrow A$.

$\Longrightarrow\varphi(\nil(B))\subset\nil(A)$

$\Longrightarrow\varphi_{\red}:B/\nil(B)\rightarrow A/\nil(A)$

$\Longrightarrow f_{\red}:\Spec(A/\nil(A))\rightarrow\Spec(B/\nil(B))$.
\end{proof}
\begin{prop}[48]
Sei $X$ Schemata, $Z\subset X$ lokal abgeschlossene Teilmenge.
Dann existiert ein eindeutig bestimmtes reduziertes Unterschema mit
topologischem Raum $Z$.
\end{prop}

\begin{proof}
Eindeutigkeit: Korollar 45. Existenz: Verklebungslemma $\Longrightarrow$
ohne Einschränkung $X=\Spec A$ affin und $Z\subset X$ abgeschlossen
(sonst Überdeckung von $X$ zu $Z\subset_{\text{abg.}}U\subset_{\text{abg.}}X$)
$\Longrightarrow\exists\mathfrak{a}\subset A$ so dass $Z=V(\mathfrak{A})$
$\Longrightarrow Z'=\Spec(A/\mathfrak{a})$ ist abgeschlossenes Unterschema
von $X$ mit topologischem Raum $Z$. Satz 44 $\Longrightarrow\exists Z'_{\red}\subset Z'\subset X$.
\end{proof}
Damit besitzt für ein lokal abgeschlossene Teilmenge die geordnete
Menge (bzgl. Inklusion) von Unterschema, denen topologischen Raum
$Z$ umfassen, ein eindeutiges minimales Element $Z_{\red}$, das
\textbf{reduzierte Unterschema} mit unterliegendem Raum $Z$.

\chapter{Faserprodukte}

\section{Der ,,Punkte-Funktor``}

Kontravarianter Funktor, $\forall X\in\sch$,
\begin{align*}
h_{X}:(\sch)^{\op} & \longrightarrow\set\\
S & \longmapsto h_{X}(S):=\Hom_{\sch}(S,X)\\
(f:T\rightarrow S) & \longmapsto(\Hom(S,X)\overset{f^{\ast}}{\rightarrow}\Hom(T,X),\ g\mapsto g\circ f)
\end{align*}

$f^{\ast}=h_{X}(S)$ heißen $S$-wertige Punkte von $X$. \textbf{Notation:
}$X(S)$, $X(R)$, falls $S=\Spec R$.

\textbf{Relative Version:} $X\in\schs0$, $S_{0}$ fixes Schemata.
\begin{align*}
\schs0 & \longrightarrow\set\\
S & \longmapsto\Hom_{S_{0}}(S,X)
\end{align*}

\textbf{Notation: $X_{S_{0}}(S)$, $X_{R_{0}}(S)$, $X_{S_{0}}(R)$,
$X_{R_{0}}(R)$}
\begin{example}
Sei $k$ algebraisch abgeschlossen, $X/k$ von endlichem Typ, $x\in X_{k}(k)$.
Dann ist
\[
\im(\Spec k\overset{x}{\longrightarrow}X)\in X
\]
abgeschlossener Punkt. $x\mapsto\im(x)$ liefert Bijektion, $X_{k}(k)\rightarrow|X|$
Menge der abgeschlossenen Punkte.
\end{example}

\begin{example}
Sei $X=\mathbb{A}^{n}=\Spec(\mathbb{Z}[T_{1},\ldots T_{n}])$. Dann:
\begin{align*}
\mathbb{A}^{n}(S) & =\Hom_{\sch}(S,\mathbb{A}^{n})=\Hom_{\ring}(\mathbb{Z}[T_{1},\ldots,T_{n}],\mathcal{O}_{S}(S))\\
 & =\Gamma(S,\mathcal{O}_{S})^{n}
\end{align*}
\end{example}

\begin{example}
Sei $X=\Spec(R[T_{1},\ldots,T_{n})/(f_{1},\ldots,f_{m}))$, $S$ ein
$R$-Schema. Dann:
\begin{align*}
X_{R}(S) & =\Hom_{R\text{-Alg}}(R[I]/(f),\mathcal{O}_{S}(S))\\
 & =\{s\in\mathcal{O}_{S}(S)^{n}\mid f_{1}(s)=\cdots=f_{m}(s)=0\}
\end{align*}
\end{example}

\begin{example}
Sei $X=\Spec\mathbb{Z}[T,T^{-1}]$. Dann:
\[
X(S)=\Hom(\mathbb{Z}[T,T^{-1}],\mathcal{O}_{S}(S))=\Gamma(S,\mathcal{O}_{S})^{\times}.
\]
Hier sogar $h_{X}:\sch\rightarrow\grp$. $X$ ist eine abelsche Gruppe.
\end{example}


\section{Yoneda-Lemma}

\textbf{Ziel:} $h_{X}$ beschreibt $X$ eindeutig.

Erinnerung: $\mathcal{F},\mathcal{G}:\mathcal{A}\rightarrow\mathcal{B}$
Funktoren, natürliche Transformation $f\in\Hom(\mathcal{F},\mathcal{G})$:
\[
f=\{f(X):\mathcal{F}(X):\rightarrow\mathcal{G}(X))_{X\in\mathcal{A}}.
\]

Wir erhalten Kategorien: $\func(\mathcal{A},\mathcal{B})$, \textbf{hier:
}$\mathcal{C}=\schs0$, $\hat{C}=\func(\mathcal{C}^{\op},\set)$.
Wir erhalten einen Funktor
\begin{align*}
\mathcal{C} & \longrightarrow\hat{\mathcal{C}},\\
X & \longmapsto h_{X},\\
f & \longmapsto\text{Pullback }f^{\ast}.
\end{align*}

\begin{prop}[5]
Sei $X\in\mathcal{C}$, $\mathcal{F}\in\hat{\mathcal{C}}$. Dann
ist die Abbildung
\begin{align*}
\Hom_{\hat{C}}(h_{X},\mathcal{F}) & \longrightarrow\mathcal{F}(X)\\
\alpha & \longmapsto\alpha(X)(\id_{X})\in\Hom(h_{X}(X),\mathcal{F}(X))
\end{align*}

bijektiv und funktoriell.
\end{prop}

Insbesondere ist der obige Funktor $\mathcal{C}\rightarrow\hat{\mathcal{C}}$
volltreu (wähle $\mathcal{F}=h_{Y}$!)
\begin{proof}
Umkehrabbildung:
\begin{align*}
\mathcal{F}(X) & \longrightarrow\Hom_{\hat{\mathcal{C}}}(h_{X},\mathcal{F})\\
\xi & \longmapsto\alpha_{\xi}=(\alpha_{\xi}(Y))_{Y\in\schs0}
\end{align*}

mit
\begin{align*}
\alpha_{\xi}(Y):\Hom(Y,X)=h_{X}(Y) & \longrightarrow\mathcal{F}(Y)\\
f & \longmapsto\mathcal{F}(f)(\xi)\in\Hom(\mathcal{F}(X),\mathcal{F}(Y))
\end{align*}
\end{proof}

\section{Faserprodukte in beliebigen Kategorien}

Sei $\mathcal{C}$ eine Kategorie, $S\in\obj(\mathcal{C})$, $f:X\rightarrow S$,
$g:Y\rightarrow S$ Morphismen.
\begin{defn}[6]
Ein Tupel $(Z,p,q)$ mit $Z\in\obj(\mathcal{C})$ und Morphismen
$p:Z\rightarrow X$, $q:Z\rightarrow Y$, $f\circ p=g\circ q$, hei{\small{}ßt
}\textbf{\small{}Faserprodukt}{\small{} von $X$ und $Y$ über $S$
(bzw. von $f,g$), falls für jedes $T\in\obj(\mathcal{C})$ und Paare
$(u:T\rightarrow X,v:T\rightarrow Y)$ von Morphismen mit $f\circ u=g\circ v$
genau ein Morphismus $w:T\rightarrow Z$ existiert mit $p\circ w=u$,
$g\circ w=v$.}{\small\par}
\end{defn}

\textbf{Notation: }$X\times_{S}Y$ oder $X\times_{f,S,g}Y:=Z$, $(u,v)_{S}:=w$.
\[
\xymatrix{T\ar@/^{1pc}/[rrd]^{u}\ar@/_{1pc}/[ddr]_{v}\ar[dr]|-{\exists_{1}}\\
 & X\times_{S}Y\ar[r]^{p}\ar[d]^{q} & X\ar[d]^{f}\\
 & Y\ar[r]_{g} & S
}
\]

Ist $S$ ein finales Objekt in $\mathcal{C}$, so ist $X\times_{S}Y=X\times Y$
das kategorielle Produkt.
\begin{example}[7]
\mbox{}
\begin{enumerate}
\item $\mathcal{C}=\set$. $X\times_{S}Y=\{(x,y)\in X\times Y\mid f(x)=g(x)(\}$.
\item $\mathcal{C}=\Top$, $f:X\rightarrow S$, $g:Y\rightarrow S$ stetige
Abbildungen. Versehe $\{(x,y)\mid f(x)=g(y)\}\subset X\times Y$ mit
der Topologie induziert von der Produkttopologie auf $X\times Y$.
Dies ist ein Faserprodukt in $\Top$.
\end{enumerate}
\end{example}


\end{document}
