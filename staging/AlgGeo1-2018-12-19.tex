%% LyX 2.3.2 created this file.  For more info, see http://www.lyx.org/.
%% Do not edit unless you really know what you are doing.
\documentclass[oneside,ngerman]{book}
\usepackage[T1]{fontenc}
\usepackage[utf8]{inputenc}
\setcounter{secnumdepth}{3}
\setcounter{tocdepth}{3}
\usepackage{verbatim}
\usepackage{amsmath}
\usepackage{amsthm}
\usepackage{amssymb}

\makeatletter
%%%%%%%%%%%%%%%%%%%%%%%%%%%%%% Textclass specific LaTeX commands.
\theoremstyle{plain}
\ifx\thechapter\undefined
	\newtheorem{thm}{\protect\theoremname}
\else
	\newtheorem{thm}{\protect\theoremname}[chapter]
\fi
\theoremstyle{plain}
\newtheorem{prop}[thm]{\protect\propositionname}
\theoremstyle{remark}
\newtheorem*{rem*}{\protect\remarkname}
\theoremstyle{definition}
\newtheorem{defn}[thm]{\protect\definitionname}
\theoremstyle{plain}
\newtheorem{lem}[thm]{\protect\lemmaname}

%%%%%%%%%%%%%%%%%%%%%%%%%%%%%% User specified LaTeX commands.
%% User-specified commands
\DeclareMathOperator{\rad}{rad}
\DeclareMathOperator{\Spec}{Spec}
\DeclareMathOperator{\Specm}{Specm}
\DeclareMathOperator{\maxspec}{MaxSpec}
\DeclareMathOperator{\Quot}{Quot}
\DeclareMathOperator{\im}{\mathrm{im}}
\DeclareMathOperator{\Hom}{\mathrm{Hom}}
\DeclareMathOperator{\Mor}{\mathrm{Mor}}
\DeclareMathOperator{\id}{\mathrm{id}}
\DeclareMathOperator{\res}{res}
\DeclareMathOperator{\Abb}{Abb}
\DeclareMathOperator{\supp}{supp}
\DeclareMathOperator{\red}{red}
\DeclareMathOperator{\op}{op}
\DeclareMathOperator{\obj}{Obj}
\DeclareMathOperator{\nil}{nil}

%% sets
\DeclareMathOperator{\CC}{\mathbb{C}}
\DeclareMathOperator{\RR}{\mathbb{R}}
\DeclareMathOperator{\QQ}{\mathbb{Q}}
\DeclareMathOperator{\ZZ}{\mathbb{Z}}
\DeclareMathOperator{\NN}{\mathbb{N}}

%% categories
\DeclareMathOperator{\ouv}{\mathcal{O}uv}
\DeclareMathOperator{\set}{\underline{Set}}
\DeclareMathOperator{\ab}{\underline{Ab}}
\DeclareMathOperator{\cattop}{\underline{Top}}
\DeclareMathOperator{\cring}{\underline{CRing}}
\DeclareMathOperator{\ring}{\underline{Ring}}
\DeclareMathOperator{\psh}{\underline{\mathcal{PS}h}}
\DeclareMathOperator{\sh}{\underline{\mathcal{S}h}}
\DeclareMathOperator{\aff}{\underline{Aff}}
\DeclareMathOperator{\sch}{\underline{Sch}}
\DeclareMathOperator{\schk}{\underline{Sch/k}}
\DeclareMathOperator{\schs}{\underline{Sch/S}}
\DeclareMathOperator{\schsn}{\underline{Sch/S_{0}}}
\DeclareMathOperator{\schz}{\underline{Sch/\mathbb{Z}}}
\DeclareMathOperator{\pres}{\underline{Prevar/S}}
\DeclareMathOperator{\prek}{\underline{Prevar/k}}
\DeclareMathOperator{\affk}{\underline{AffVar/k}}
\DeclareMathOperator{\topcp}{\underline{TopCP}}
\DeclareMathOperator{\Top}{\underline{Top}}
\DeclareMathOperator{\grp}{\underline{Grp}}
\DeclareMathOperator{\func}{\underline{Func}}

\makeatother

\usepackage{babel}
\providecommand{\definitionname}{Definition}
\providecommand{\lemmaname}{Lemma}
\providecommand{\propositionname}{Satz}
\providecommand{\remarkname}{Bemerkung}
\providecommand{\theoremname}{Theorem}

\begin{document}
\begin{prop}[24]
Sei $X$ noethersches irreduzibles Schema, $\eta\in X$ generischer
Punkt. Dann sind äquivalent:
\begin{enumerate}
\item $\mathcal{O}_{X,\eta}$ ist reduziert.
\item $\exists\emptyset\neq U\subset X$ reduziertes offenes Unterschema.
\end{enumerate}
Für $(ii)$ sagt man auch: $\mathcal{O}_{X,x}$ ist \textbf{generisch}
reduziert.
\end{prop}

\begin{proof}
\mbox{}
\begin{itemize}
\item[$``\Rightarrow"$] Sei ohne Einschränkung $X=\Spec A$ affin, $A$ nach Voraussetzung
noethersch, $\eta\leftrightarrow\mathfrak{p}$ eindeutiges minimales
Primideal ($A$ irreduzibel). $\Longrightarrow\mathfrak{p}=(f_{1},\ldots,f_{n})_{A}$
endlich erzeugt. $\Longrightarrow\frac{f_{i}}{1}\in\nil(A_{\mathfrak{p}})=(0)\subset A_{\mathfrak{p}}=\mathcal{O}_{X,\eta}$,
da $\mathcal{O}_{X,\eta}$ reduziert. $\exists g\in A\backslash\mathfrak{p}$
$\Longrightarrow$ d.d. $\frac{f_{1}}{1}=0\sim A_{g}$. $\Longrightarrow0=\nil(A)_{g}=\nil(A_{g})$,
d.h. $A_{g}$ ist reduziert, d.h. $U:=D(g)$.
\item[$``\Leftarrow"$] $\emptyset\neq U\subset X$ reduziert offen. $\Longrightarrow\eta\in U$
s.d. $\mathcal{O}_{X,x}=\mathcal{O}_{X,\eta}$ reduziert.
\end{itemize}
\end{proof}
\begin{rem*}
Analog zeigt man: $X$ noethersches Schema und $\mathcal{O}_{X,x}$
reduzibel für ein $x\in X$ $\Longrightarrow\exists x\in U\subset X$
offen, d.d. $U$ reduziert ist.
\end{rem*}

\section*{Prävarietäten als Schema}

\textbf{Ziel: }``$X$ affine Varietät $\mapsto\Spec\Gamma(X,\mathcal{O}_{X})$``
(kein generischer Punkt, $\neq$ Schema). Verklebe zu: ``$X$ Prävarietät
über $k$ $\mapsto k$-Schema``. Welches Bild hat dieser Funktor?

\section{Schemata von endlichem Typ über $k$}

Sei $X$ affine Varietät über $k$, $k$ algebraisch Abgeschlossen.
Dann ist $A=\Gamma(X,\mathcal{O}_{X})$ eine endlich erzeugte $k$-Algebra.
\begin{defn}[25]
Sei $k$ Körper, $X\rightarrow\Spec(k)$ $k$-Schema. $X$ heißt:
\begin{itemize}
\item \textbf{lokal von endlichem Typ}, (l.v.e.T./$k$), falls eine affine
offene Überdeckung $X=\bigcup_{i\in I}U_{i}$ existiert, $U_{i}=\Spec A_{i}$,
mit $A_{i}$ endlich erzeugte $k$-Algebra für alle $i$.
\item \textbf{von endlichem Typ} (v.e.T$/k$), falls $X$ lokal von endlichem
Typ und quasi-kompakt ist.
\end{itemize}
\end{defn}

\begin{rem*}
Jedes $k$-Schema welches (lokal) von endlichem Typ ist, ist (lokal)
noethersch. (Da jede endlich erzeugte $k$-Algebra noethersch ist.)
\end{rem*}
\begin{prop}[26]
Sei $X$ $l.v.e.T/k$, $U\subset X$ offen affin. Dann ist $B:=\Gamma(U,\mathcal{O}_{X})$
eine endlich erzeugte $k$-Algebra.
\end{prop}

\begin{proof}
$U=\bigcup_{i=1}^{n}D(f_{i})$, $f_{i}\in B$ geeignet nach Lemma
3.3. $B$ ist endlich erzeugte $k$-Algebra $\Longrightarrow B_{f_{i}}=B\left[\frac{1}{f_{i}}\right]$
ist endlich erzeugte $k$-Algebra. Mit dem folgenden Lemma für $A=k$
folgt die Behauptung.
\end{proof}
\begin{lem}[28]
Sei $A$ ein Ring und $B$ eine $A$-Algebra, $\mathcal{L}:A\rightarrow B$
Ringhomomrphismus, $f_{1},\ldots,f_{n}\in B$ mit $(f_{1},\ldots,f_{n})=(1)$,
und so dass $B_{f_{i}}$ eine endlich erzeugte $A$-Algebra ist $\forall i$.
Dann ist $B$ eine endlich erzeugte $A$-Algebra.
\end{lem}

\begin{proof}
(Vergleiche Lemma 16.) Nach Voraussetzung gibt es ein $g_{i}\in B$
mit $\sum_{i}g_{i}f_{i}=1$. Da $B_{f_{i}}$ endlich erzeugt $\forall i$,
gibt es $b_{ij}$, $j\in J$ endlich, welche $B_{f_{i}}$ als $A$-Algebra
erzeugen. Setze $b_{ij}=c_{ij}/f_{i}^{m}$ mit $c_{ij}\in B$ für
$m\geq0$ geeignet (unabhäng von $i,j$).

Sei $C:=A$-Unteralgebra von $B$, erzeugt von $g_{i},f_{i},c_{ij}$,
d.h. endlich erzeugt über $A$. \emph{Behauptung}: $C=B$.

Sei $b\in B$. $\Longrightarrow$ $\exists N\gg0$ mit $f_{i}^{N}b\in C$
für alle $i$. Da $\sum_{i}g_{i}f_{i}=1$, ist $(f_{1},\ldots,f_{n})_{C}=(1)$.
Lemma 2.4 $\Longrightarrow$ $(f_{i}^{N},\ldots,f_{n}^{N})_{C}=(1)$.
$\Longrightarrow$ $\exists u_{1},\ldots,u_{n}\in C$ sodass $\sum_{i}u_{i}f_{i}^{N}=1$.
$\Longrightarrow b=\sum_{i}u_{i}\underbrace{f_{i}^{N}b}_{\in C}\in C$.
\end{proof}
\begin{prop}[28]
Sei $k$ algebraisch abgeschlossen, $X$ $k$-Schema l.v.e.T.$/k$.
Dann besteht die Menge der abgeschlossenen Punkte $|X|$ genau aus
den Punkten mit $\kappa(x)=k$, d.h. nach Proposition 7 gilt
\[
|X|=X(k)=\Hom_{k}(\Spec k,X).
\]
\end{prop}

\begin{proof}
Hilbert'scher Nullstellensatz $\Longrightarrow(x\in X$ abgeschlossen
$\Rightarrow\kappa(x)=k$). Daher reicht es zu zeigen: ($x\in X$
\emph{nicht} abgeschlossen, d.h. $\mathfrak{p}_{x}$ maximal $\Rightarrow\kappa(x)\neq k$).
Dazu: $\exists x\in U=\Spec(A)\subset X$ offen, mit $x$ \emph{nicht
abgeschlossen} in $U$. $\Longleftrightarrow\mathfrak{p}=\mathfrak{p}_{x}\in\Spec(A)$
\emph{nicht} maximal, d.h. $A/\mathfrak{p}_{x}$ ist \emph{kein} Körper.
$\Longrightarrow k\rightarrow(A/\mathfrak{p}_{x})\hookrightarrow\Quot(A/\mathfrak{p}_{x})=\kappa(x)$
ist \emph{echte} Inklusion.

Behauptung: $\kappa(x)$ ist nicht algebraisch abgeschlossen, d.h.
nicht abstrakt isomorph zu $k$. Denn: \emph{Noether-Normalisierung}:

$A/\mathfrak{p}$ ist endlich über $k[X_{1},\ldots,X_{n}]$. Nach
Lemma I.9 folgt $n>0$ (mit $A/\mathfrak{p}$ über $k$ ganz $\Longrightarrow A/\mathfrak{p}$
Körper). $\Longrightarrow\kappa(x)$ ist endliche Erweiterung von
$k(X_{1},\ldots,X_{n})$, $n>0$. $\Longrightarrow k$ nicht algebraisch
abgeschlossen ($[k(X_{1},\ldots,X_{n})(\sqrt[n]{X_{1}}):k(X)]\rightarrow\infty$,
$n\rightarrow\infty$)
\end{proof}
\begin{rem*}
Im Allgemeinen $\exists x\in U\subset X$ offen mit $\{x\}\subset U$
abgeschlossen, aber $\{x\}\subset X$ \emph{nicht} abgeschlossen ($X=\Spec\mathcal{O}$,
$\mathcal{O}$ DVR, $U=\{x\}$, $x=\eta$ generischer Punkt.) Für
$X$ lokal von endlichem Typ über $k$ (nicht notwendig algebraisch
abgeschlossen) kann nach der Proposition \emph{nicht} geschehen.
\end{rem*}

\section{Sehr dichte Teilmengen}
\begin{defn}[29]
Sei $X$ topologischer Raum. Eine Teilmenge $Y\subset X$ heißt \textbf{sehr
dicht}, falls die folgenden äquivalenten Bedingungen gelten:
\begin{enumerate}
\item $U\mapsto U\cap Y$ definiert eine Bijektion:
\[
\{\text{offenen Teilmengen in }X\}\leftrightarrow\{\text{offene Teilmengen in }Y\}.
\]
\item $F\mapsto F\cap Y$ definiert eine Bijektion:
\[
\{\text{abgeschlossene Teilmengen in }X\}\leftrightarrow\{\text{abgeschlossene Teilmengen in }Y\}.
\]
\item Für alle $F\subseteq X$ abgeschlossen gilt: $F=\overline{F\cap Y}$.
\item Jede lokal abgeschlossene Teilmenge $Z\neq\emptyset$ von $X$ enthält
einen Punkt aus $Y$.
\end{enumerate}
\end{defn}

\begin{proof}
Die Äquivalenz von $(i)$, $(ii)$ und $(iii)$ ist klar.
\begin{itemize}
\item[$(iii)\Rightarrow(iv)$] Für abgeschlossene Teilmengen $F'\subsetneq F$ von $X$ setze $Z:=F\backslash F'$.
Angenommen $(F\cap Y)\backslash(F'\cap Y)=Z\cap Y=\emptyset$. $\Longrightarrow F\cap Y=F'\cap Y$.
$(iii)\Longrightarrow F=F'$, Widerspruch.
\item[$(iv)\Rightarrow(ii)$] Sei $F,F'\subset X$ abgeschlossen mit $F\cap Y=F'\cap Y$. $\Longleftrightarrow((F\cup F')\backslash(F\cap F'))\cap Y=\emptyset$.
$\Longrightarrow(F\cup F')\backslash(F\cap F')=\emptyset$. $\Longrightarrow F=F'$.
\end{itemize}
\end{proof}
\begin{prop}[30]
Sei $X$ l.v.e.T über $k$ algebraisch abgeschlossen. Dann ist $|X|$
sehr dicht in $X$.
\end{prop}

\begin{proof}
Zeige: Bedingung $(iv)$. Sei $\emptyset\neq A\subset X$ lokal abgeschlossen.
Ohne Einschränkung: 
\[
A\subset_{\text{abg. }}U=\Spec A\subset_{\text{off.}}X.
\]

Nach Voraussetzung ist $A$ endlich-erzeugte $k$-Algebra. $\emptyset\neq A=V(\mathfrak{a})$
mit $\mathfrak{a}\subset\mathfrak{m}\subset A$ für ein maximales
Ideal $\mathfrak{m}$. $\Longrightarrow V(\mathfrak{a})$ enthält
abgeschlossenen Punkt $x\in\mathfrak{m}$. Proposition 28 $\Longrightarrow x$
ist abgeschlossen in $X$, da $\kappa(x)=k$%
\begin{comment}
unlesbar
\end{comment}
. $\Longrightarrow A\cap|X|\neq\emptyset$.
\end{proof}

\end{document}
