%% LyX 2.3.2 created this file.  For more info, see http://www.lyx.org/.
%% Do not edit unless you really know what you are doing.
\documentclass[oneside,ngerman]{book}
\usepackage[T1]{fontenc}
\usepackage[utf8]{inputenc}
\setcounter{secnumdepth}{3}
\setcounter{tocdepth}{3}
\usepackage{amsmath}
\usepackage{amsthm}
\usepackage[all]{xy}

\makeatletter
%%%%%%%%%%%%%%%%%%%%%%%%%%%%%% Textclass specific LaTeX commands.
\theoremstyle{plain}
\ifx\thechapter\undefined
	\newtheorem{thm}{\protect\theoremname}
\else
	\newtheorem{thm}{\protect\theoremname}[chapter]
\fi
\theoremstyle{plain}
\newtheorem{prop}[thm]{\protect\propositionname}

%%%%%%%%%%%%%%%%%%%%%%%%%%%%%% User specified LaTeX commands.
%% User-specified commands
\DeclareMathOperator{\rad}{rad}
\DeclareMathOperator{\Spec}{Spec}
\DeclareMathOperator{\maxspec}{MaxSpec}
\DeclareMathOperator{\Quot}{Quot}
\DeclareMathOperator{\im}{\mathrm{im}}
\DeclareMathOperator{\Hom}{\mathrm{Hom}}
\DeclareMathOperator{\Mor}{\mathrm{Mor}}
\DeclareMathOperator{\id}{\mathrm{id}}
\DeclareMathOperator{\res}{res}
\DeclareMathOperator{\Abb}{Abb}
\DeclareMathOperator{\nil}{nil}
\DeclareMathOperator{\supp}{supp}
\DeclareMathOperator{\red}{red}
\DeclareMathOperator{\op}{op}
\DeclareMathOperator{\obj}{Obj}

%%sets
\DeclareMathOperator{\CC}{\mathbb{C}}
\DeclareMathOperator{\RR}{\mathbb{R}}
\DeclareMathOperator{\QQ}{\mathbb{Q}}
\DeclareMathOperator{\ZZ}{\mathbb{Z}}
\DeclareMathOperator{\NN}{\mathbb{N}}

%% categories
\DeclareMathOperator{\ouv}{\mathcal{O}uv}
\DeclareMathOperator{\set}{\underline{Set}}
\DeclareMathOperator{\ab}{\underline{Ab}}
\DeclareMathOperator{\cattop}{\underline{Top}}
\DeclareMathOperator{\cring}{\underline{CRing}}
\DeclareMathOperator{\ring}{\underline{Ring}}
\DeclareMathOperator{\psh}{\underline{\mathcal{PS}h}}
\DeclareMathOperator{\sh}{\underline{\mathcal{S}h}}
\DeclareMathOperator{\aff}{\underline{Aff}}
\DeclareMathOperator{\sch}{\underline{Sch}}
\DeclareMathOperator{\schs}{\underline{Sch/S}}
\DeclareMathOperator{\schsn}{\underline{Sch/S_{0}}}
\DeclareMathOperator{\pres}{\underline{Prevar/S}}
\DeclareMathOperator{\prek}{\underline{Prevar/k}}
\DeclareMathOperator{\affk}{\underline{AffVar/k}}
\DeclareMathOperator{\topcp}{\underline{TopCP}}
\DeclareMathOperator{\Top}{\underline{Top}}
\DeclareMathOperator{\grp}{\underline{Grp}}
\DeclareMathOperator{\func}{\underline{Func}}

\makeatother

\usepackage{babel}
\providecommand{\propositionname}{Satz}
\providecommand{\theoremname}{Theorem}

\begin{document}
Sei $X,X'\in\schs$, $f:X'\rightarrow X$ Morphismus in $\schs$,
$g:=f\times_{S}\id_{Y}$.
\[
\xymatrix{Z'=X'\times_{S}Y\ar[r]^{g}\ar[d]_{p'} & Z=X\times_{S}Y\ar[r]^{q}\ar[d]_{p'} & Y\ar[d]\\
X'\ar[r]^{f} & X\ar[r] & S
}
\]

kommutiert. Da $q\circ g=q'$ Projektion auf $Y$ ist, ist das große
und damit auch beide Diagramme kartesisch. (Proposition 10)
\begin{prop}[14]
$f$ induziert einen Homömorphismus von $X'$ auf $f(X')$ und:
\begin{enumerate}
\item $f_{x'}^{\#}:\mathcal{O}_{X,f(x')}\rightarrow\mathcal{O}_{X',x'}$
sei surjektiv $\forall x'\in X'$ und es existiert eine offene affine
Umgebung $U'$ von $f(x')$, sodass $f^{-1}(U')$ quasi-kompakt ist,
oder
\item $f_{x'}^{\#}$ ist bijektiv $\forall x'\in X'$.
\end{enumerate}
Dann gilt:
\begin{enumerate}
\item $g$ ist ein Homöomorphismus von $Z'$ auf $g(Z')=p^{-1}(f(X'))$.
\item $\forall z'\in Z'$ haben wir das induzierte Diagramm für lokale Ringe:
\[
\xymatrix{\mathcal{O}_{Z',z'}\ar[d] & \mathcal{O}_{Z,g(z')}\ar[l]^{g_{z'}^{\#}}\ar[d]^{p_{g(z)}^{\#}}\\
\mathcal{O}_{X',p'(z')} & \mathcal{O}_{X,p(g(z'))}\ar[l]^{f_{p'(z')}}
}
\]
\end{enumerate}
\begin{itemize}
\item $g_{z'}^{\#}$ ist surjektiv;
\item $\ker(g_{z'}^{\#})$ ist von $p_{g(z')}^{\#}(\ker f_{p'(z')}^{\#})$
erzeugt.
\end{itemize}
\end{prop}


\end{document}
