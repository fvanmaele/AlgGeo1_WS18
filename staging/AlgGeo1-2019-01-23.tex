%% LyX 2.3.2 created this file.  For more info, see http://www.lyx.org/.
%% Do not edit unless you really know what you are doing.
\documentclass[oneside,ngerman]{book}
\usepackage[T1]{fontenc}
\usepackage[utf8]{inputenc}
\setcounter{secnumdepth}{3}
\setcounter{tocdepth}{3}
\usepackage{amsmath}
\usepackage{amsthm}
\usepackage{amssymb}
\usepackage{stmaryrd}
\usepackage[all]{xy}

\makeatletter
%%%%%%%%%%%%%%%%%%%%%%%%%%%%%% Textclass specific LaTeX commands.
\theoremstyle{plain}
\ifx\thechapter\undefined
	\newtheorem{thm}{\protect\theoremname}
\else
	\newtheorem{thm}{\protect\theoremname}[chapter]
\fi
\theoremstyle{definition}
\newtheorem{defn}[thm]{\protect\definitionname}
\theoremstyle{definition}
\newtheorem{example}[thm]{\protect\examplename}
\theoremstyle{plain}
\newtheorem{lem}[thm]{\protect\lemmaname}
\theoremstyle{remark}
\newtheorem{notation}[thm]{\protect\notationname}
\theoremstyle{definition}
\newtheorem*{defn*}{\protect\definitionname}
\theoremstyle{plain}
\newtheorem{prop}[thm]{\protect\propositionname}
\theoremstyle{plain}
\newtheorem{cor}[thm]{\protect\corollaryname}
\theoremstyle{remark}
\newtheorem*{rem*}{\protect\remarkname}
\theoremstyle{definition}
\newtheorem*{example*}{\protect\examplename}

%%%%%%%%%%%%%%%%%%%%%%%%%%%%%% User specified LaTeX commands.
%% User-specified commands
\DeclareMathOperator{\rad}{rad}
\DeclareMathOperator{\Spec}{Spec}
\DeclareMathOperator{\maxspec}{MaxSpec}
\DeclareMathOperator{\Quot}{Quot}
\DeclareMathOperator{\im}{\mathrm{im}}
\DeclareMathOperator{\Hom}{\mathrm{Hom}}
\DeclareMathOperator{\Mor}{\mathrm{Mor}}
\DeclareMathOperator{\id}{\mathrm{id}}
\DeclareMathOperator{\res}{res}
\DeclareMathOperator{\Abb}{Abb}
\DeclareMathOperator{\supp}{supp}
\DeclareMathOperator{\red}{red}
\DeclareMathOperator{\op}{op}
\DeclareMathOperator{\obj}{Obj}
\DeclareMathOperator{\nil}{nil}

%% sets
\DeclareMathOperator{\CC}{\mathbb{C}}
\DeclareMathOperator{\RR}{\mathbb{R}}
\DeclareMathOperator{\QQ}{\mathbb{Q}}
\DeclareMathOperator{\ZZ}{\mathbb{Z}}
\DeclareMathOperator{\NN}{\mathbb{N}}

%% categories
\DeclareMathOperator{\ouv}{\mathcal{O}uv}
\DeclareMathOperator{\set}{\underline{Set}}
\DeclareMathOperator{\ab}{\underline{Ab}}
\DeclareMathOperator{\cattop}{\underline{Top}}
\DeclareMathOperator{\cring}{\underline{CRing}}
\DeclareMathOperator{\ring}{\underline{Ring}}
\DeclareMathOperator{\psh}{\underline{\mathcal{PS}h}}
\DeclareMathOperator{\sh}{\underline{\mathcal{S}h}}
\DeclareMathOperator{\aff}{\underline{Aff}}
\DeclareMathOperator{\sch}{\underline{Sch}}
\DeclareMathOperator{\schk}{\underline{Sch/k}}
\DeclareMathOperator{\schs}{\underline{Sch/S}}
\DeclareMathOperator{\schsn}{\underline{Sch/S_{0}}}
\DeclareMathOperator{\schz}{\underline{Sch/\mathbb{Z}}}
\DeclareMathOperator{\pres}{\underline{Prevar/S}}
\DeclareMathOperator{\prek}{\underline{Prevar/k}}
\DeclareMathOperator{\affk}{\underline{AffVar/k}}
\DeclareMathOperator{\topcp}{\underline{TopCP}}
\DeclareMathOperator{\Top}{\underline{Top}}
\DeclareMathOperator{\grp}{\underline{Grp}}
\DeclareMathOperator{\func}{\underline{Func}}

\makeatother

\usepackage{babel}
\providecommand{\corollaryname}{Korollar}
\providecommand{\definitionname}{Definition}
\providecommand{\examplename}{Beispiel}
\providecommand{\lemmaname}{Lemma}
\providecommand{\notationname}{Notation}
\providecommand{\propositionname}{Satz}
\providecommand{\remarkname}{Bemerkung}
\providecommand{\theoremname}{Theorem}

\begin{document}

\section{Fasern von Morphismen}

\textbf{Ziel:} $\xymatrix{X\ar[d]^{f}\supset f^{-1}(s)\\
S\ni s
}
$ als Schema.
\begin{defn}[18]
Für den kanonischen Morphismus $\Spec(\kappa(s))$ nennen wir
\[
X_{s}:=X\otimes_{S}\kappa(s)
\]

die \textbf{Faser von $f$ in $s$}, ein $\kappa(s)$-Schema. Proposition
14 $\Longrightarrow$
\[
\xymatrix{X_{s}\ar[r]\ar[d] & S\times_{S}X\ar[r]^{q=\id_{X}}\ \ \ar[d]_{f} & X\ar[d]\\
\Spec\kappa(s)\ar[r]_{\text{canon.}} & S\ar[r]_{\id_{S}} & S
}
\qquad\text{kommutativ,}
\]

besagt $X_{s}\cong f^{-1}(s)$ (=homoömorph zu Bild(canon)), d.h.
wir können $f^{-1}(s)$ als Schema auffassen.

\textbf{Denkweise}: $\xymatrix{X\ar[d]_{f}\\
S
}
\cong$ Familie von $\kappa(s)$-Schemata $X_{s}$, parametrisiert durch
Punkte von $S$.
\end{defn}

\begin{example}[19]
Sei $k$ algebraisch abgeschlossen.
\[
X(k):=\{(u,t,s)\in\mathbb{A}^{3}(k)\mid ut=s\}
\]

Da $UT-s\in k[U,T,S]$ irreduzibel ist, ist $X(k)$ eine affine Varietät.
$\leftrightarrow$ $X=\Spec(k[U,T,S]/(UT-S)$ ist ganzes $k$-Schema.
Sei $S=\mathbb{A}^{1}$,
\begin{align*}
X & \longrightarrow S\\
(u,t,s) & \longmapsto s
\end{align*}

Projektion. $s\in\mathbb{A}^{1}(k)$, $X_{s}=\Spec A_{s}$, 
\begin{align*}
A_{s} & =k[U,T,S]/(UT-S)\otimes_{k[S]}k[S]/(S-s)\\
 & \cong k[U,T]/(UT-S).
\end{align*}

$UT-s\in k[U,T]$ irreduzibel für $s\neq0$, reduzibel für $s=0$.
$\Longrightarrow X\rightarrow S$ definiert Familie $X_{s}$ von $k$-Schemata,
sodass $X_{0}$ reduzibel, $X_{s}$ irreduzibel für $s\neq0$.
\end{example}

\begin{lem}[20]
Sei das Diagramm
\[
\xymatrix{ & X\times_{S}Y\ar[ld]_{p}\ar[rd]^{q}\\
x\in X\ar[rd]_{f} &  & Y\ni y\ar[ld]^{g}\\
\Spec\kappa(x)\ar[u]^{\xi} & S & \Spec\kappa(y)\ar[u]_{\psi}
}
\]

Dann gilt:
\begin{enumerate}
\item Es gibt ein $z\in X\times_{S}Y$ mit $p(z)=x$, $q(z)=y$, genau dann
wenn $f(x)=g(y).$
\item Es gelte $(1)$, setze $s:=f(x)=g(y)$. Dann ist
\begin{align*}
\zeta:=\xi\times_{S}\psi:Z:=\Spec(\kappa(x)\otimes_{\kappa(s)}\kappa(y)) & \longrightarrow X\times_{S}Y
\end{align*}
ein Homöomorphismus von $Z$ auf Teilraum $\zeta(Z)=p^{-1}(x)\cap q^{-1}(y)$.
\end{enumerate}
\end{lem}

\begin{proof}
Setze $Z:=p^{-1}(x)\times_{(X\times_{S}Y)}q^{-1}(y)$, und betrachte:
\[
\xymatrix{ &  & Z\ar[rd]^{g'}\ar[ld]_{f'}\\
 & p^{-1}(x)\ar[rd]\ar[ld] &  & q^{-1}(y)\ar[ld]\ar[rd]\\
\Spec\kappa(s)\ar[rd] &  & X\times_{S}Y\ar[ld]\ar[rd] &  & \Spec\kappa(y)\ar[ld]\\
 & X\ar[rd] &  & Y\ar[ld]\\
 &  & S
}
\]

Wende Proposition 14 zweifach an $\Longrightarrow$
\[
g'(Z)=i^{-1}(p^{-1}(x))=q^{-1}(y)\cap p^{-1}(x).
\]
\end{proof}

\section{Eigenschaften von Schemata-Morphismen}
\begin{notation}[21]
Sei $\mathbb{P}$ Eigenschaft von Morphismen von Schemata.
\begin{enumerate}
\item $\mathbb{P}$ heißt \textbf{lokal im Ziel}, falls für alle Morphismen
$f$ und alle offenen Überdeckungen $S=\bigcup_{j\in J}S_{j}$ gilt:
\[
f:X\rightarrow S\text{ erfüllt }\mathbb{P}\Longleftrightarrow f|_{f^{-1}(S_{j})}:f^{-1}(S_{j}):S_{j}\text{ erfüllt }\mathbb{P}\ \forall j\in J
\]
\item $\mathbb{P}$ heißt \textbf{lokal in der Quelle}, falls für alle $f:X\rightarrow Y$
und alle offenen Überdeckungen $X=\bigcup_{i\in I}U_{i}$ gilt:
\[
f\text{ erfüllt }\mathbb{P}\Longleftrightarrow f|_{U_{i}}:U_{i}\rightarrow Y\text{ erfüllt }\mathbb{P}\ \forall i\in I
\]
\end{enumerate}
\end{notation}

\begin{defn*}
Ein Morphismus $f:X\rightarrow Y$ von Schemata heißt \textbf{(treu)flach},
falls $\forall x\in X$ die Abbildung
\[
\mathcal{O}_{Y,f(x)}\longrightarrow\mathcal{O}_{X,x}
\]

(treu)flach ist.
\end{defn*}
\begin{prop}[22]
Die folgende Eigenschaft von Schemata-Morphismen sind:
\begin{enumerate}
\item stabil unter Komposition: ,,injektiv``, ,,surjektiv``, ,,bijektiv``,
,,homöomorph``, ,,flach``, ,,treuflach``, ,,offen``, ,,abgeschlossen``,
,,offene Immersion``, ,,abgeschlossene Immersion``, ,,Immersion``;
\item stabil unter Basiswechsel: ,,surjektiv``, ,,offene Immersion``,
,,abgeschlossene Immersion``, ,,Immersion``, ,,flach``, ,,treuflach``;
\item lokal bzgl. Ziel: ,,surjektiv``, ,,bijektiv``, ,,homöomorph``,
,,offen``, ,,abgeschlossen``, ,,offene Immersion``, ,,abgeschlossene
Immersion``, ,,Immersion``, ,,flach``, ,,treuflach``;
\item lokal bzgl. Quelle: ,,offen``, ,,flach``.
\end{enumerate}
\end{prop}

\begin{proof}
Die Fälle $(1)$, $(3)$, $(4)$ sind klar. Der Fall $(2)$ für offene/abgeschlossene
Immersionen wird in Abschnitt 9 behandelt. Lemma 20 $\Longrightarrow$
,,surjektiv`` stabil unter Basiswechsel.
\[
\xymatrix{X\times_{S}Y\ar[r]^{q} & Y & y\ar@{|->}[d]\\
X\ar@{->>}[r]^{f} & S & s\\
\exists x\ar@/_{1pc}/@{|->}[rru]
}
\]

Sei $X\rightarrow S$ (treu)flach und $S'\rightarrow S$ beliebig.
Sei in Fall $(3)$, $(4)$ ohne Einschränkung $X=\Spec A$, $S=\Spec R$,
$S'=\Spec R'$, $A$ (treu)flache $R$-Algebra. $\Longrightarrow A\otimes_{R}R'$
(treu)flache $R'$-Algebra, d.h. $f_{(S')}$ ist treuflach.
\[
(A\otimes_{R}R')\otimes_{R'}M\cong A\otimes_{R}M
\]
\end{proof}
\begin{cor}[23]
Die folgende Eigenschaften sind stabil unter Komposition, stabil
unter Basiswechsel und lokal bzgl. Ziel:

,,universal injektiv``, ,,universal bijektiv``, ,,universal homöomorph``,
,,universell offen``, ,,universell abgeschlossen``.
\end{cor}


\section{Urbilder und Schema-theoretische Schnitte von Unterschemata}

Sei $f:X\rightarrow Y$ ein Morphismus von Schemata und $i:Z\rightarrow Y$
eine Immersion.
\[
\xymatrix{Z\times_{Y}X\ar[r]\ar[d]_{i_{(X)}} & Z\ar[d]^{i}\\
X\ar[r]_{f} & Y
}
\]

Proposition 14 $\Longrightarrow i_{(X)}$ ist surjektiv auf Halmen,
Homöomorphismus von $Z\times_{Y}X$ auf lokal abgeschlossene Teilmenge
$f^{-1}(Z)$ (genau $f^{-1}(i(Z))$), d.h. $i_{(X)}$ ist Immersion.
Fasse $Z\times_{Y}X$ als Unterschema von $X$ auf, das \textbf{Urbild
von $Z$ unter $f$}.
\begin{rem*}
\mbox{}
\begin{enumerate}
\item Ist $Z\subset Y$ offenes Unterschema, so auch $f^{-1}(Z)\subset X$.
\item Ist $Z=V(\mathfrak{p})$ abgeschlossenes Unterschema, $\mathfrak{p}\subset\mathcal{O}_{Y}$
Idealgarbe, so auch 
\begin{align*}
f^{-1}(Z) & =V(f^{\ast^{-1}}(\mathfrak{p})\mathcal{O}_{X})\\
 & =\text{Bild}(f^{\ast^{-1}}(\mathfrak{p})\rightarrow f^{\ast^{-1}}\mathcal{O}_{Y}\rightarrow\mathcal{O}_{X}).
\end{align*}
\end{enumerate}
\textbf{Spezialfall}: Durchschnitt von 2 Unterschemata $i:Y\rightarrow X$,
$j:Z\rightarrow X$:
\[
Y\cap Z:=Y\times_{X}Z=i^{-1}(Z)=j^{-1}(Y)
\]

heißt \textbf{(Schema-theoretischer) Durchschnitt von $Y$ und $Z$
in $X$}.
\end{rem*}

\subsubsection*{Universelle Eigenschaft (aus univ. Eig. Faserprodukt)}

Ein Morphismus $h:T\rightarrow X$ faktorisiert durch $Y\cap Z$ genau
dann wenn $h$ faktorisiert durch $Y$ und $Z$. Sind $Y=V(\mathfrak{p})$,
$Z=V(\mathfrak{q})$ abgeschlossene Unterschemata, so folgt:
\begin{align*}
V(\mathfrak{p})\cap V(\mathfrak{q}) & =V(\mathfrak{p}+\mathfrak{q})\\
A/\mathfrak{p}\otimes_{A}A/\mathfrak{q} & \cong A/\mathfrak{q+}\mathfrak{p}
\end{align*}

\begin{example*}
$f_{1},\ldots,f_{r},g_{1},\ldots,g_{s}\in R[X_{0},\ldots,X_{n}]$
homogene Polynome. Dann ist:
\[
V_{+}(f_{1},\ldots,f_{r})\cap V_{+}(g_{1},\ldots,g_{s})=V_{+}(f_{1},\ldots,f_{r},g_{1},\ldots,g_{s})\subseteq\mathbb{P}_{R}^{n}
\]
\end{example*}

\section{Affine und projektive Räume über beliebige Basen}
\begin{itemize}
\item[] $\mathbb{A}^{n}:=\mathbb{A}_{\mathbb{Z}}^{n}$, $S$ beliebiges Schema.
\item[] $\mathbb{A}_{S}^{n}:=\mathbb{A}^{n}\times_{\mathbb{Z}}S$ \textbf{affiner
Raum der relativen Dimension $n$ über $S$}.
\item[] $\mathbb{P}_{S}^{n}:=\mathbb{P}^{n}\times_{\mathbb{Z}}S$ \textbf{projektiver
Raum der relativen Dimension $n$ über $S$}.
\item[] $S=\Spec R$ affin:
\begin{align*}
\mathbb{A}_{S}^{n} & =\Spec(\mathbb{Z}[T_{1},\ldots,T_{n}]\otimes_{\mathbb{Z}}R)=\Spec(R[T_{1},\ldots,T_{n}])\\
 & =\mathbb{A}_{R}^{n}\text{ wie zuvor!}\\
\mathbb{P}_{S}^{n} & =\mathbb{P}_{R}^{n}\text{ analog.}
\end{align*}
\item[] $\mathbb{A}_{n}^{S}\times_{S}S'=\mathbb{A}^{n}\times_{\mathbb{Z}}S\times_{S}S'=\mathbb{A}_{S'}^{n}$
\item[] $\mathbb{P}_{S}^{n}\times_{S}S'=\mathbb{P}_{S'}^{n}$ für einen beliebigen
Basiswechsel $S'\rightarrow S$.
\end{itemize}
Sei $X$ ein beliebiges Schema.
\begin{align*}
\Gamma(X,\mathcal{O}_{X}) & =\Hom_{\ring}(\mathbb{Z}[T],\Gamma(X,\mathcal{O}_{X}))\\
 & =\Hom_{\schz}(X,\mathbb{A}_{\mathbb{Z}}^{1})\\
\varphi(T) & \mapsfrom\varphi
\end{align*}

Sei $X$ ein $S$-Schema.
\[
\Gamma(X,\mathcal{O}_{X})=\Hom_{\schs}(X,\mathbb{A}_{S}^{1})
\]


\section{Diagonal, Graph und Kern in beliebigen Kategorien}

Sei $\mathcal{C}$ Kategorie mit Faserprodukten, $S\in\mathcal{C}$,
$X,T\in\mathcal{C}/S$, $X_{S}(T)$ Menge der $S$-Morphismen.
\begin{defn}[24]
Der Morphismus
\begin{enumerate}
\item $\Delta_{X/S}:=\Delta_{u}:=(\id_{X},\id_{X}):X\rightarrow X\times_{S}X$,
$u:X\rightarrow S$, heißt \textbf{Diagonale }(diagonaler Morphismus)
\textbf{von $X$ über $S$}.
\item Sei $f:X\rightarrow Y\in$ Morph/$S$. Der Morphismus
\[
\Gamma_{j}:=(\id_{X},f)_{S}:X\longrightarrow X\times_{S}Y
\]
heißt der \textbf{Graph(morphismus) von $f$}.
\item Seien $f,g:X\rightarrow Y\in$ Morph/$S$. Ein $S$-Monomorphismus
$i:K\rightarrow X$ heißt \textbf{(Differenzen)kern} von $f$ und
$g$, falls für alle $T\in\mathcal{C}/S$ die Abbildung $i(T):K_{S}(T)\rightarrow X_{S}(T)$
injektiv ist mit
\[
\text{Bild}(i(T))=\{x\in X_{S}(T)\mid f(T)(x)=g(T)(x)\}.
\]
Beizchne $K(f,g)_{S}$ oder $\ker(f,g)$, $i$ ,,kanonisch``. Mit
anderen Worten, $\ker(f,g)$ separiert den Funktor
\begin{align*}
\mathcal{C}/S & \longrightarrow\sch\\
T & \longmapsto\{x\in X_{S}(T)\mid f(T)(x)=g(T)(x)\}
\end{align*}
\end{enumerate}
\end{defn}


\end{document}
