%% LyX 2.3.2 created this file.  For more info, see http://www.lyx.org/.
%% Do not edit unless you really know what you are doing.
\documentclass[oneside,ngerman]{book}
\usepackage[T1]{fontenc}
\usepackage[utf8]{inputenc}
\setcounter{secnumdepth}{3}
\setcounter{tocdepth}{3}
\usepackage{amsmath}
\usepackage{amsthm}
\usepackage{amssymb}
\usepackage{stackrel}
\usepackage[all]{xy}

\makeatletter
%%%%%%%%%%%%%%%%%%%%%%%%%%%%%% Textclass specific LaTeX commands.
\theoremstyle{plain}
\ifx\thechapter\undefined
	\newtheorem{thm}{\protect\theoremname}
\else
	\newtheorem{thm}{\protect\theoremname}[chapter]
\fi
\theoremstyle{definition}
\newtheorem{defn}[thm]{\protect\definitionname}
\theoremstyle{definition}
\newtheorem{example}[thm]{\protect\examplename}

%%%%%%%%%%%%%%%%%%%%%%%%%%%%%% User specified LaTeX commands.
\usepackage{mathtools}

%% User-specified commands
\DeclareMathOperator{\rad}{rad}
\DeclareMathOperator{\Spec}{Spec}
\DeclareMathOperator{\Quot}{Quot}
\DeclareMathOperator{\im}{\mathrm{im}}
\DeclareMathOperator{\Hom}{\mathrm{Hom}}
\DeclareMathOperator{\Mor}{\mathrm{Mor}}
\DeclareMathOperator{\id}{\mathrm{id}}
\DeclareMathOperator{\res}{res}

%%sets
\DeclareMathOperator{\CC}{\mathbb{C}}
\DeclareMathOperator{\RR}{\mathbb{R}}
\DeclareMathOperator{\QQ}{\mathbb{Q}}
\DeclareMathOperator{\ZZ}{\mathbb{Z}}
\DeclareMathOperator{\NN}{\mathbb{N}}

%% categories
\DeclareMathOperator{\ouv}{\mathcal{O}uv}
\DeclareMathOperator{\set}{\underline{Set}}
\DeclareMathOperator{\ab}{\underline{Ab}}
\DeclareMathOperator{\cattop}{\underline{Top}}
\DeclareMathOperator{\cring}{\underline{CRing}}
\DeclareMathOperator{\psh}{\underline{\mathcal{PS}h}}
\DeclareMathOperator{\sh}{\underline{\mathcal{S}h}}
\DeclareMathOperator{\aff}{\underline{Aff}}

\makeatother

\usepackage{babel}
\providecommand{\definitionname}{Definition}
\providecommand{\examplename}{Beispiel}
\providecommand{\theoremname}{Theorem}

\begin{document}

\section{Der Funktor $A\protect\mapsto(\Spec A,\mathcal{O}_{\Spec A})$}
\begin{defn}[34]
Ein lokal geringter Raum $(X,\mathcal{O}_{X})$ heißt \textbf{affines
Schema}, falls ein Ring $A$ existiert d.d
\[
(X,\mathcal{O}_{X})\cong(\Spec A,\mathcal{O}_{\Spec A})
\]

Ein \textbf{Morphismus affiner Schemata} ist ein Morphismus lokal
geringter Räume. Bezeichne $\aff$ die Kategorie der affinen Schemata.
\begin{align*}
\varphi:A & \longrightarrow B & \text{Ringhom.}\\
f:X:=\Spec A & \longrightarrow Y:=\Spec A & \text{stetige Abb.}
\end{align*}
\end{defn}

\textbf{Ziel: }Definiere $(f,f^{\flat}):X\rightarrow Y$ mit $f:=^{a}\varphi$
Morphismus von lokal geringter Räume und
\[
f_{\Spec A}^{\flat}=\varphi:A=\mathcal{O}_{\Spec A}(\Spec A)\rightarrow f_{\ast}\mathcal{O}_{\Spec B}(\Spec B)=B
\]

Dazu: Für $s\in A$ gilt $f^{-1}(D(s))=D(\varphi(s))$ nach Proposition
2.10. Definiere 
\[
f_{D(s)}^{\flat}:\mathcal{O}_{Y}(D(s))=A_{s}\rightarrow B_{\varphi(s)}=f_{\ast}\mathcal{O}_{X}(D(s))
\]

als die von $\varphi$ induzierte Abbildung. $f^{\flat}$ ist kompatibel
mit $\res_{D(t)}^{D(s)}$ für prinzipal offene Mengen $D(t)\subseteq D(s)$.
$B$ Basis $\Longrightarrow f^{\flat}:\mathcal{O}_{Y}\rightarrow f_{\ast}\mathcal{O}_{X}$
Homomorphismus von Ringgarben. Für $s=1$ erhalten wir $f_{\Spec A}^{\flat}=\varphi$!

Für $x\in X$ gilt:
\[
\xymatrix{f^{\sharp}:\mathcal{O}_{Y,f(x)}=A_{\varphi^{-1}(\mathfrak{p}_{x})=\mathfrak{p}_{f(x)}}\ar[r] & B_{\mathfrak{p}_{x}}=\mathcal{O}_{X,x}\\
A\ar[u]\ar[r]^{\varphi} & B\ar[u]
}
\qquad(*)
\]

ist der von $\varphi$ induzierte Homomophismus. $f_{x}^{\sharp}$
is lokal:
\[
\varphi(\varphi^{-1}(\mathfrak{p}_{x}))\subseteq\mathfrak{p}_{x}
\]

\textbf{Bezeichne}: $^{a}\varphi$ für $\Spec(\varphi)=(f,f^{\flat})$,
$^{a}(\psi\circ\varphi)=^{a}\varphi\circ^{a}\psi$. Wir erhalten einen
kontravarianten Funktor
\[
\Spec:\underline{Ring}\longrightarrow\aff.
\]

Für $f:(X,\mathcal{O}_{X})\rightarrow(Y,\mathcal{O}_{Y})$ Morphismus
von geringten Räumen erhalten wir einen Ringhomomorphismus
\[
\Gamma(f):=f_{Y}^{\flat}:\Gamma(Y,\mathcal{O}_{Y})=\mathcal{O}_{Y}(Y)\rightarrow\Gamma(X,\mathcal{O}_{X})=(f_{\ast}\mathcal{O}_{X})(Y)=\mathcal{O}_{X}(X).
\]

So erhalten wir einen kontravarianten Funktor
\[
\Gamma:\aff\longrightarrow\underline{Ring}.
\]

\begin{thm}[35]
Die Funktoren $\Spec$ und $\Gamma$ definieren eine Anti-Äquivalenz
zwischen der Kateogire der Ringe und der Kategorie der affinen Schemata.
\end{thm}

\begin{proof}
$\Spec$ ist essentiell surjektiv per Definition. $\Gamma\circ\Spec$
ist isomorph zu $\id_{\underline{Ring}}$ nach Konstruktion.Zu zeigen:
\[
\Hom_{\underline{Ring}}(A,B)\stackrel[\Gamma]{\Spec}{\rightleftharpoons}\Hom_{\aff}(\Spec B,\Spec A)
\]

sind zueinander invers. Es fehlt die Verkettung $\Spec\circ\Gamma=\id_{\aff}$.
Sei $f\in\Hom_{\aff}(\Spec B,\Spec A)$, $\varphi:=\Gamma(f)$, $^{a}\varphi=f$.
Für $\mathfrak{p}_{x}\in\Spec B=X$ ist $f_{x}^{\sharp}$ der eindeutig
bestimmte Ringhomomorphismus, welcher das Diagramm
\[
\xymatrix{A\ar[r]^{f_{\Spec A}^{\flat}=\Gamma(f)=\varphi}\ar[d]_{\imath_{A}} & B\ar[d]^{\imath_{B}}\supset\imath_{B}^{-1}(\mathfrak{p}_{x}B_{\mathfrak{p}_{x}})=\mathfrak{p}_{x}\\
A_{\mathfrak{p}_{f(x)}}\ar[r]_{f_{x}^{\#}\text{ lokal}} & B_{\mathfrak{p}_{x}}\underset{\max}{\supset}\mathfrak{p}_{x}B_{\mathfrak{p}_{x}}
}
\qquad(**)
\]

kommutieren lässt. Es gilt:
\begin{align*}
\mathfrak{p}_{f(x)}A_{\mathfrak{p}_{f(x)}} & \supseteq(f_{x}^{\sharp})^{-1}(\mathfrak{p}_{x}B\mathfrak{p}_{x})=\mathfrak{p}_{f(x)}A_{\mathfrak{p}_{f(x)}}\\
\mathfrak{p}_{f(x)} & =\imath_{A}^{-1}(\mathfrak{p}_{f(x)}A\mathfrak{p}_{f(x)})\subset A
\end{align*}

$f_{x}^{\#}$ lokal $\Longrightarrow f_{x}^{\#}(\mathfrak{p}_{f(x)}A_{\mathfrak{p}_{f(x)}})\subset\mathfrak{p}_{x}B_{\mathfrak{p}_{x}}$
$\Longrightarrow^{a}\varphi=f$ als stetige Abbildung. Wegen $(*)$
lässt auch $(^{a}\varphi)_{x}^{\#}$ das Diagramm $(**)$ kommutieren.
Proposition $\Longrightarrow(^{a}\varphi)^{\#}=f^{\#}$.
\end{proof}

\section{Beispiele}
\begin{example}[36, Integritätsbereiche]
Sei $A$ integer, $K=\Quot(A)$. Sei $X=\Spec A$, $\eta=(0)$. Dann
ist$\overline{\{\eta\}}=\Spec X$, d.h. jede nicht-leere offene Menge
$U\subset X$ enthält $\eta$. Es folgt: $\mathcal{O}_{X,y}=A_{(0)}=K$.
Für alle $f\in A$ gilt nach Definition 
\[
\mathcal{O}_{X}(D(f))=A_{f}\subset U.
\]

Sei $U\subset X$ beliebig offen. Es folgt:
\[
\mathcal{O}_{X}(U)=\underset{\underset{D(f)\subset U}{\longleftarrow}}{\lim}\mathcal{O}_{X}(D(f)=\bigcap_{\underset{D(f)\subset U}{f\in A}}A_{f}\subseteq K.
\]

Wie im Beweis von Satz 1.37 ist $a_{F}=\bigcap_{\mathfrak{p}\in D(f)}A_{\mathfrak{p}},$
also $\mathcal{O}_{X}(U)=\bigcap_{x\in U}\mathcal{O}_{X,x}$.
\end{example}

\begin{example}[37, Prinzipal offene Unterschemata affiner Schemata]
Sei $X=\Spec A$, $f\in A$. Sei $j:\Spec A_{f}\rightarrow\Spec A$
induziert von $A\rightarrow A_{f}$. $\Longrightarrow j:\Spec A_{j}\rightarrow D(f)$
ist Homoömorphismus (Proposition 2.12). Für alle $x\in D(f)$ ist
$j_{x}^{\#}$ der kanonische Isomorphismus $A_{\mathfrak{p}_{x}}\overset{\cong}{\rightarrow}(A_{f})_{\mathfrak{p}_{x}}$.
$\Longrightarrow(j,j^{\#})$ induziert einen Isomorphismus $\Spec A_{f}\cong(D(f),\mathcal{O}_{X|D(f)})$.
\end{example}

\begin{example}[38, Abgeschlossene Unterschemata affiner Schemata]
Sei $X=\Spec A$ und $\mathfrak{a}$ ein Ideal von $A$. Sei $\imath:\Spec A/\mathfrak{a}\rightarrow\Spec A$
der von $A\rightarrow A/\mathfrak{a}$ induzierte Morphismus affiner
Schemata. Nach Proposition 2.12 induziert $\imath$ einen Homöomorphismus
$\Spec A/\mathfrak{a}\overset{\cong}{\rightarrow}V(\mathfrak{a})\subseteq\Spec A$.
Sei $\overline{\mathfrak{p}_{x}}$ das Bild von $\mathfrak{p}_{x}$
in $A/\mathfrak{a}$. Für alle $x\in V(\mathfrak{a})$ ist der Morphismus
$i_{x}^{\flat}$ der kanonische Homomorphismus $A_{\mathfrak{p}_{x}}\rightarrow(A/\mathfrak{a})_{\overline{\mathfrak{p_{x}}}}$.
($=0$, falls $x\in V(\mathfrak{a})$, also $\mathfrak{a}\notin f_{x}$.)
Schreibe kurz $V(\mathfrak{a})$ für den lokal geringten Raum 
\begin{align*}
\left(V(\mathfrak{a}),\imath_{x}(\mathcal{O}_{\Spec A/\mathfrak{a}})|_{V(\mathfrak{a})}\right) & \stackrel[\imath]{\cong}{\longleftarrow}\Spec(A/\mathfrak{a})
\end{align*}

Da $x\in V(\mathfrak{a})$, ist $\imath_{x}(\mathcal{O}_{\Spec A/\mathfrak{a}})|_{V(\mathfrak{a})}\overset{\cong}{\longrightarrow}i_{x}\mathcal{O}_{\Spec A/\mathfrak{a}}$.
\end{example}

\begin{example}[39]
Sei $B$ ein Ring und $\mathfrak{b}\subset B$ Ideal. $V(\mathfrak{b}^{n})=V(\mathfrak{b})\subset\Spec B$
als abgeschlossene Teilmenge hängt \emph{nicht }von $n\geq1$ ab,
aber $\Spec(B/\mathfrak{b}^{n})=V(\mathfrak{b}^{n})$ als offenes
Schemata sehr wohl!

Etwa: $B=k[T]$, $b=(T)$ mit $k$ ein algebraisch abgeschlossener
Körper. Für die abgeschlossenen Punkte von $\mathbb{A}_{k}^{1}=\Spec k[T]$
gilt:
\begin{align*}
\Spec(k[T]) & \longleftrightarrow k\\
\mathfrak{b} & \longleftrightarrow0
\end{align*}

Sei $A=k[T]/(T^{n})$, $X=\Spec A=\{x\}$. Es gilt: 
\begin{align*}
\mathcal{O}_{X}(X) & =\mathcal{O}_{X,x}=A\\
\kappa(x) & =k\\
n>1:\ 0\neq\mathfrak{m}_{x} & =T\mod T^{n}
\end{align*}

Betrachte $X\subset\mathbb{A}_{k}^{1}$ als abgeschlossenes ,,Unterschemata``
welches ,,konzentriert in einem Punkt`` ist. $B=k[T]$ ist $k$-Algebra
von Funktionen auf $\mathbb{A}^{1}(k)$. (vgl. Beispiel 2.14.) Die
Einschränkung einer solchen Funktion $f\in k[T]$ auf $X$ ist gegeben
durch $k[T]\rightarrow k[T]/(T^{n})$. Wir unterscheiden:
\begin{itemize}
\item[$n=1$.] $k[T]/(T^{n})=k$, $f\mapsto f(0)$.
\item[$n>1$.] $k[T]/(T^{n})\neq k$, $f\mapsto$ (,,Taylor-Entwicklung`` von
$A$ um 0 der Länge $n-1$). $\{x\}\subset\mathbb{A}_{k}^{1}$ hat
,,infinitesimale Ausdehnung der Länge $n-1$ in $\mathbb{A}_{k}^{1}$``
\end{itemize}
Sei nun $\mathbb{A}_{k}^{2}:=\Spec(k[T,U])$ betrachtet als $\{(u,t)\mid u,f\in k\}=k^{2}$.
Sei $\mathfrak{a}_{1}=(U)$, $\mathfrak{a}_{2}=(U-T^{n})$. Diese
definieren:
\[
X_{1}=\{u,t)\in\mathbb{A}^{2}(k)\mid u=0\},\quad X_{2}=\{(u,t)\in\mathbb{A}^{2}(k)\mid u=t^{n}\}.
\]

Es ist $X_{1}\cap X_{2}=\{(0,0)\}$ als Menge. Aber für $n>1$ treffen
sich beide Mengen \emph{nicht} transversal! Als affine Schemata wird
später der Schnitt als $\Spec k[T,U]/(\mathfrak{a}_{1}+\mathfrak{a}_{2})$
definiert, also eine präzisere Beschreibung als Durchschnitt.
\end{example}


\end{document}
