%% LyX 2.3.2 created this file.  For more info, see http://www.lyx.org/.
%% Do not edit unless you really know what you are doing.
\documentclass[oneside,ngerman]{book}
\usepackage[T1]{fontenc}
\usepackage[latin9]{inputenc}
\setcounter{secnumdepth}{3}
\setcounter{tocdepth}{3}
\usepackage{enumitem}
\usepackage{amsmath}
\usepackage{amsthm}
\usepackage{amssymb}
\usepackage[all]{xy}

\makeatletter
%%%%%%%%%%%%%%%%%%%%%%%%%%%%%% Textclass specific LaTeX commands.
\newlength{\lyxlabelwidth}      % auxiliary length 
\theoremstyle{plain}
\newtheorem{thm}{\protect\theoremname}
\theoremstyle{definition}
\newtheorem{example}[thm]{\protect\examplename}
\theoremstyle{definition}
\newtheorem*{xca*}{\protect\exercisename}
\theoremstyle{definition}
\newtheorem{defn}[thm]{\protect\definitionname}

%%%%%%%%%%%%%%%%%%%%%%%%%%%%%% User specified LaTeX commands.
%% User-specified commands
\DeclareMathOperator{\rad}{rad}
\DeclareMathOperator{\Spec}{Spec}
\DeclareMathOperator{\Quot}{Quot}
\DeclareMathOperator{\res}{res}
\DeclareMathOperator{\im}{\mathrm{im}}
\DeclareMathOperator{\Hom}{\mathrm{Hom}}
\DeclareMathOperator{\Mor}{\mathrm{Mor}}
\DeclareMathOperator{\id}{\mathrm{id}}

%%sets
\DeclareMathOperator{\CC}{\mathbb{C}}
\DeclareMathOperator{\RR}{\mathbb{R}}
\DeclareMathOperator{\QQ}{\mathbb{Q}}
\DeclareMathOperator{\ZZ}{\mathbb{Z}}
\DeclareMathOperator{\NN}{\mathbb{N}}

%% categories
\DeclareMathOperator{\ouv}{\mathcal{O}uv}
\DeclareMathOperator{\set}{\underline{Set}}
\DeclareMathOperator{\ab}{\underline{Ab}}
\DeclareMathOperator{\cattop}{\underline{Top}}
\DeclareMathOperator{\cring}{\underline{CRing}}
\DeclareMathOperator{\ring}{\underline{Ring}}
\DeclareMathOperator{\psh}{\underline{\mathcal{PS}h}}
\DeclareMathOperator{\sh}{\underline{\mathcal{S}h}}
\DeclareMathOperator{\sch}{\underline{Sch}}
\DeclareMathOperator{\schs}{\underline{Sch/S}}

\makeatother

\usepackage{babel}
\providecommand{\definitionname}{Definition}
\providecommand{\examplename}{Beispiel}
\providecommand{\exercisename}{Aufgabe}
\providecommand{\theoremname}{Theorem}

\begin{document}
\begin{proof}
Als topologischer Raum: $\coprod_{i\in I}U_{i}/\sim$ mit $x_{i}\in U_{i}\sim x_{j}\in U_{j}:\Leftrightarrow x_{i}\in U_{ij}$,
$x_{j}\in U_{ji}$ und $x_{j}=\varphi_{ji}(x_{i})$. Nach Eigenschaft
$(b)$ ist $\sim$ eine �quivalenzrelation. Dann sind $\psi_{i}:U_{i}\rightarrow X$
injektiv. Ferner haben wir $\forall i,j\in I$ die Eigenschaft $\psi_{i}(U_{i})=\psi_{i}(U_{i})\cap\psi_{j}(U_{j})$.

$X$ hat also als topologischer Raum die Quotiententopologie, d.h.
die feinste Topologie sodass alle Abbildungen $\psi_{i}$ stetig sind.
$U\subset X$ offen genau dann, wenn $\psi_{i}^{-1}(U)\subset U_{i}$
dort offen sind $\forall i\in I$. Insbesondere sind $\psi_{i}(U_{i})$
und $\psi_{i}(U_{j})=\psi_{i}(U_{i})\cap\psi_{j}(U_{j})$ offen in
$X$.

Als lokal geringter Raum: ``Verkleben der Strukturgarben auf $U_{i}$''.
$\mathcal{O}_{X}$ ist eindeutig auf einer Basis $B$ der Topologie
definiert. Ohne Einschr�nkung reicht es hier, die Schnitte nur auf
$U\subset X$ offen mit $U\subset\psi_{i}(U_{i})$ f�r ein $i\in I$.
In dem Fall:
\[
\mathcal{O}_{X}(U)=\mathcal{O}_{U_{i}}(\psi_{i}^{-1}(U))
\]

F�r $U\subset\psi_{i}(U_{i})\cap\psi_{j}(U_{j})$
\[
\xymatrix{U_{ij}\ar@{^{(}->}[r]\ar[d]_{\cong} & U_{i}\\
U_{ji}\ar@{^{(}->}[r] & U_{i} & X\supset U
}
\]

Dann gilt:

\[
\mathcal{O}_{U_{i}}(\psi_{i}^{-1}(U))=\mathcal{O}_{U_{ij}}(\psi_{i}^{-1}(U))\cong\mathcal{O}_{U_{ji}}(\psi_{j}^{-1}(U))=\mathcal{O}_{U_{j}}(\psi_{j}^{-1}(U))
\]

Es folgt: $\mathcal{O}_{X}(U)$ unabh�ngig von Wahlen von $i$! Wir
halten damit $\mathcal{O}_{X}$ Ringgarbe auf $X$, sodass $(X,\mathcal{O}_{X})$
lokal geringter Raum da $(U_{i},\mathcal{O}_{U_{i}})$ lokal geringter
Raum $\forall i\in I$, $U_{i}\xrightarrow[\psi_{i}]{\cong}(\psi_{i}(U_{i}),\mathcal{O}_{X}|_{\psi_{i}(U_{i})})$
als lokal geringter Raum. Damit ist $X$ ein Schema und $X=\cup U_{i}$.

Spezialfall: $U_{ij}=\emptyset$ f�r alle $i\neq j\in I$, $\coprod U_{i}$
``disjunkte Vereinigung''.
\end{proof}
\begin{example}[10]
$X_{1},\ldots,X_{n}$ affine Schemata, $X_{i}=\Spec A_{i}$. Dann
ist
\[
\coprod X_{i}\cong\Spec\left(\prod_{i=1}^{n}A_{i}\right)\text{ offen}.
\]

(nicht f�r unendlich viele affin!)
\end{example}

\begin{example}[11]
$I=\{1,2\}$, $U_{12}\subset U_{1}\xrightarrow{\varphi}U_{21}\subset U_{2}$.
\[
X\underset{\text{offen}}{\subset}V=U_{1}\cup_{\varphi}U_{2},
\]

$\Gamma(V,\mathcal{O}_{X})=\{(s_{1},s_{2})\in\Gamma(V\cap U_{1},\mathcal{O}_{U_{1}})\times\Gamma(V\cap U_{2},\mathcal{O}_{U_{2}})\}$,
$\varphi^{\flat}(S_{2}|_{U_{21}\cap V})=S_{1}|_{U_{12}\cap V}$.
\end{example}

\textbf{Affine Gerade mit Doppelpunkt}: $k$ K�rper.
\begin{align*}
U_{1} & =U_{2}=\mathbb{A}_{k}^{1}=\Spec(k[T])\cong x\text{ abg.}\\
U_{12} & :=U_{1}\backslash\{x\}\\
U_{21} & :=U_{2}\backslash\{x\},\ \varphi=\id
\end{align*}

$X=U_{1}\cup_{\varphi}U_{2}$.  
\begin{xca*}
$X$ ist \textbf{nicht} affin!
\end{xca*}

\section{Der projektive Raum als Schema }

Sei $R$ Ring. $\mathbb{P}_{R}^{n}$ Verklebung von $(n+1)$-Kopien
des 
\begin{align*}
\mathbb{A}_{R}^{n} & =\Spec(R[T_{1},\ldots,T_{n}])\\
\shortparallel\\
\Spec\left(R\left[\frac{X_{0}}{X_{i}},\ldots,\frac{\hat{X}_{i}}{X_{i}},\ldots,\frac{X_{n}}{X_{i}}\right]\right)=U_{i}, & i=0,\ldots,n
\end{align*}

Verklebungs Datum:
\begin{align*}
B:= & R[X_{0},\ldots,X_{n},X_{0}^{-1},\ldots,X_{n}^{-1}]\\
U_{ij} & :=D\left(\frac{X_{j}}{X_{i}}\right)\subset U_{i},\ \text{OE}\ i\neq j\leq n,\\
U_{ii} & =U_{i},\ \varphi_{ii}=\id
\end{align*}

$\varphi_{ji}:U_{ij}\rightarrow U_{ji}$ definiert durch
\[
\xymatrix{R\left[\frac{X_{0}}{X_{i}},\ldots,\frac{\hat{X}_{i}}{X_{i}},\ldots,\frac{X_{n}}{X_{i}}\right]_{\frac{x_{j}}{x_{i}}}\ar@{=}[r]^{''\id''}\ar@{^{(}->}[dr] & R\left[\frac{x_{0}}{x_{j}},\ldots,\frac{\hat{x}_{j}}{x_{j}},\ldots,\frac{x_{n}}{x_{j}}\right]_{\frac{x_{i}}{x_{j}}}\ar@{^{(}->}[d]\\
 & B
}
\]

Mit ``$\id$'' folgt: Kozykelbedingung automatisch, $U_{i}\rightarrow\Spec(R)$
verkleben von $\mathbb{P}_{R}^{n}\rightarrow\Spec(R)$ mit $\mathbb{P}_{R}^{n}:=\coprod U_{i}/\sim$
Schema (�ber $R$). ``Der projektive Raum relativer Dimension $n$
�ber $R$''.
\begin{xca*}
$R\xrightarrow{\sim}\Gamma(\mathbb{P}_{R}^{n},\mathcal{O}_{\mathbb{P}_{R}^{n}})$
(Strukturgarbe) d.h. f�r $n>0$ ist $\mathbb{P}_{R}^{n}$ nicht affin
(mit $\mathbb{P}_{R}^{n}=\Spec(R)$).
\end{xca*}

\section{Nullstellenmenge im projektiven Raum }

Sei $I\subset R[X_{0},\ldots,X_{n}]$ homogenes Ideal, d.h. erzeugt
von homogenen Elementen.

\[
\Leftrightarrow I=\bigoplus_{d}I\cap R[X_{1},\ldots,X_{n}]_{d})
\]

Ziel: $V_{+}(I)\rightarrow\mathbb{P}_{R}^{n}$ Morphismus von Schemata
\begin{align*}
R[X_{0},\ldots,X_{n}] & \longrightarrow R\left[\frac{X_{0}}{X_{i}},\ldots,\frac{\hat{X_{i}}}{X_{i}},\ldots,\frac{X_{n}}{X_{i}}\right]=\Gamma(U_{i},\mathcal{O}_{U_{i}})\\
I & \longmapsto\Phi_{i}(I)\text{ das vom Bild von }I\text{ erzeugte Ideal}
\end{align*}

Verklebe $V_{i}:=\Spec(\Gamma(U_{i},\mathcal{O}_{U_{i}})/\Phi_{i})\subseteq U_{i}$
 entlang
\[
V_{ij}=D_{V_{i}}\left(\frac{X_{j}}{X_{i}}\right)\xrightarrow{\cong}V_{ji}
\]

Beachte: $f\in I$, $\deg(f)=d$, $X_{i}^{d}\Phi_{i}(f)=X_{j}^{d}\Phi_{j}(f)$,
d.h. $\Phi_{i}(f)$ und $\Phi_{j}(f)$ unterscheiden sich in einer
Einheit auf $D\left(\frac{X_{i}}{X_{j}}\right)$.

$\Longrightarrow\Phi_{i}(I)=\Phi_{j}(I)$ in $\Gamma(U_{ij},\mathcal{O}_{U_{i}})$.
$\Longrightarrow V_{ij}=V_{ji}$ und Kozykelbedingungen �bertr�gt
sich von $U_{ij}\subset U_{i}$.

$\Longrightarrow$ Verkleben liefert Schema $V_{+}(I)$ + 

D.h. jedes solche $I$ definiert ein Schema $V_{+}(I)\rightarrow\mathbb{P}_{R}^{n}$.

\chapter*{Grundlegende Eigenschaften von Schemata und Morphismen}

\section{Topologische Eigenschaften }
\begin{defn}[12]
Ein Schema $X$ hei�t \textbf{zusammenh�ngend}, \textbf{quasi-kompakt}
bzw. \textbf{irreduzibel}, falls der unterliegende topologische Raum
diese Eigenschaft besitzt.
\end{defn}

\begin{itemize}
\item Nach Proposition II.5 ist jedes affine Schema quasi-kompakt.
\item $\coprod_{i=0}\Spec(R)$ ist \emph{nicht }quasi-kompakt.
\end{itemize}
\begin{defn}[13]
$f:X\rightarrow Y$ hei�t \textbf{injektiv} (surjektiv, bijektiv),
falls die zugrendlegende stetige Abbildung diese Eigenschaft hat.
Ebenso f�r ``offen'', ``abgeschlossen'', ``Hom�morphismus''.
\end{defn}

Warnung: Hom�morphismen von Schemata sind im Allgemeinen \emph{keine
}Isomorphismen!

\section{Noethersche Schemata }
\begin{defn}[14]
Ein Schema hei�t \textbf{lokal noethersch}, falls eine affine offene
�berdeckung $X=\bigcup U_{i}$ existiert, d.d. alle $\Gamma(U_{i},\mathcal{O}_{X})$
(affine Koordinatenringe) \textbf{noethersch} sind. $X$ hei�t \textbf{noethersch},
falls zus�tlich quasi-kompakt.
\end{defn}

Faktum: Lokalisierung noetherscher Ringe bleiben noethersch.

$\Rightarrow a)$ Jedes lokal noethersche Schema besitzt eine Basis
der Topologie aus noetherschen affin offenen Unterschemata.

b) $X$ lokal noethersch. Dann ist $\mathcal{O}_{X,x}$ noethersch
$\forall x\in X$.

F�r offene Schemata gilt ferner: lokal noethersch $\Rightarrow$ noethersch.
\end{document}
