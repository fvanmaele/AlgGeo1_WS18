%% LyX 2.3.2 created this file.  For more info, see http://www.lyx.org/.
%% Do not edit unless you really know what you are doing.
\documentclass[oneside,ngerman]{book}
\usepackage[T1]{fontenc}
\usepackage[utf8]{inputenc}
\setcounter{secnumdepth}{3}
\setcounter{tocdepth}{3}
\usepackage{fancybox}
\usepackage{calc}
\usepackage{amsmath}
\usepackage{amsthm}
\usepackage{amssymb}
\usepackage[all]{xy}

\makeatletter
%%%%%%%%%%%%%%%%%%%%%%%%%%%%%% Textclass specific LaTeX commands.
\theoremstyle{plain}
\ifx\thechapter\undefined
	\newtheorem{thm}{\protect\theoremname}
\else
	\newtheorem{thm}{\protect\theoremname}[chapter]
\fi
\theoremstyle{definition}
\newtheorem{example}[thm]{\protect\examplename}
\theoremstyle{plain}
\newtheorem{prop}[thm]{\protect\propositionname}
\theoremstyle{definition}
\newtheorem{defn}[thm]{\protect\definitionname}
\theoremstyle{plain}
\newtheorem{cor}[thm]{\protect\corollaryname}

%%%%%%%%%%%%%%%%%%%%%%%%%%%%%% User specified LaTeX commands.
%% User-specified commands
\DeclareMathOperator{\rad}{rad}
\DeclareMathOperator{\Spec}{Spec}
\DeclareMathOperator{\maxspec}{MaxSpec}
\DeclareMathOperator{\Quot}{Quot}
\DeclareMathOperator{\im}{\mathrm{im}}
\DeclareMathOperator{\Hom}{\mathrm{Hom}}
\DeclareMathOperator{\Mor}{\mathrm{Mor}}
\DeclareMathOperator{\id}{\mathrm{id}}
\DeclareMathOperator{\res}{res}
\DeclareMathOperator{\Abb}{Abb}
\DeclareMathOperator{\supp}{supp}
\DeclareMathOperator{\red}{red}
\DeclareMathOperator{\op}{op}
\DeclareMathOperator{\obj}{Obj}
\DeclareMathOperator{\nil}{nil}

%% sets
\DeclareMathOperator{\CC}{\mathbb{C}}
\DeclareMathOperator{\RR}{\mathbb{R}}
\DeclareMathOperator{\QQ}{\mathbb{Q}}
\DeclareMathOperator{\ZZ}{\mathbb{Z}}
\DeclareMathOperator{\NN}{\mathbb{N}}

%% categories
\DeclareMathOperator{\ouv}{\mathcal{O}uv}
\DeclareMathOperator{\set}{\underline{Set}}
\DeclareMathOperator{\ab}{\underline{Ab}}
\DeclareMathOperator{\cattop}{\underline{Top}}
\DeclareMathOperator{\cring}{\underline{CRing}}
\DeclareMathOperator{\ring}{\underline{Ring}}
\DeclareMathOperator{\psh}{\underline{\mathcal{PS}h}}
\DeclareMathOperator{\sh}{\underline{\mathcal{S}h}}
\DeclareMathOperator{\aff}{\underline{Aff}}
\DeclareMathOperator{\sch}{\underline{Sch}}
\DeclareMathOperator{\schk}{\underline{Sch/k}}
\DeclareMathOperator{\schs}{\underline{Sch/S}}
\DeclareMathOperator{\schsn}{\underline{Sch/S_{0}}}
\DeclareMathOperator{\schz}{\underline{Sch/\mathbb{Z}}}
\DeclareMathOperator{\pres}{\underline{Prevar/S}}
\DeclareMathOperator{\prek}{\underline{Prevar/k}}
\DeclareMathOperator{\affk}{\underline{AffVar/k}}
\DeclareMathOperator{\topcp}{\underline{TopCP}}
\DeclareMathOperator{\Top}{\underline{Top}}
\DeclareMathOperator{\grp}{\underline{Grp}}
\DeclareMathOperator{\func}{\underline{Func}}

\makeatother

\usepackage{babel}
\providecommand{\corollaryname}{Korollar}
\providecommand{\definitionname}{Definition}
\providecommand{\examplename}{Beispiel}
\providecommand{\propositionname}{Satz}
\providecommand{\theoremname}{Theorem}

\begin{document}
\begin{example}[35]
Sei $S$ beliebiges Schema, $n\in\mathbb{N}$. Dann ist $\mathbb{A}_{S}^{n}$
separiert über $S$, ebenso jedes Unterschema, denn $\mathbb{A}_{S}^{n}=\mathbb{A}_{\mathbb{Z}}^{n}\times_{\Spec\mathbb{Z}}S$
und ,,separiert`` ist stabil unter Basiswechsel nach Proposition
34.
\end{example}

\begin{prop}[36]
Sei $S=\Spec R$ affin und $X$ ein $S$-Schema. Dann sind äquivalent:
\begin{enumerate}
\item $X$ ist separiert.
\item Für je zwei offene affine $U,V\subseteq X$ ist $U\cap V$ affin,
und
\begin{align*}
\rho_{U,V}:\mathcal{O}_{X}(U)\otimes_{R}\mathcal{O}_{X}(V) & \longrightarrow\mathcal{O}_{X}(U\cap V),\\
s\otimes t & \longmapsto s|_{U\cap V}\cdot t|_{V\cap U}.
\end{align*}
\item Es gibt eine offene affine Überdeckung $X=\bigcup_{i\in I}U_{i}$,
sodass $\forall i,j\in I$: $\rho_{U,V}$ ist surjektiv.
\end{enumerate}
\end{prop}

\begin{proof}
Für alle offene affine $U,V\subseteq X$ gilt:
\[
U\cap V=\Delta_{X/S}^{-1}(U\times_{S}V).
\]

,,abgeschlossene Immersion`` ist lokal auf dem Ziel, daher: $\Delta_{X/S}$
ist abgeschlossene Immersion

$\Longleftrightarrow$ für alle $U,V\subseteq X$ offen affin ist
\[
U\cap V\xrightarrow{\Delta_{X/S}|_{U\cap V}}U\times_{S}V
\]

abgeschlossene Immersion.

$\Longleftrightarrow$ für jede offene affine Überdeckung $X=\bigcup_{i\in I}U_{i}$
und alle $i,j\in I$ ist
\[
U_{i}\cap U_{j}\longrightarrow U_{i}\times_{S}U_{j}
\]

abgeschlossene Immersion. Sind $U=\Spec A$, $V=\Spec B$ affin, so
ist auch
\[
U\times_{S}V=\Spec(A\otimes_{R}B)
\]

affin. Daher:
\[
U\cap V\longrightarrow U\times_{S}V
\]

abgeschlossene Immersion. $\Longleftrightarrow\rho_{U,V}$ surjektiv.
\end{proof}
\begin{example}[37]
Für jedes Schema $S$ und $n\in\mathbb{N}$ ist $\mathbb{P}_{S}^{n}$
separiert über $S$, denn ,,separiert`` ist lokal auf dem Ziel (Proposition
34), daher ohne Einschränkung $S=\Spec R$ affin. Sei $\mathbb{P}_{R}^{n}=\bigcup_{i=0}^{n}U_{i}$
mit $U_{i}=\Spec R\left[\frac{X_{0}}{X_{i}},\ldots,\frac{\widehat{X_{i}}}{X_{i}},\ldots,\frac{X_{n}}{X_{i}}\right]$
und
\begin{align*}
\rho_{U_{i},U_{j}}:R & \left[\frac{X_{0}}{X_{i}},\ldots,\frac{\widehat{X_{i}}}{X_{i}},\ldots,\frac{X_{n}}{X_{i}}\right]\otimes_{R}R\left[\frac{X_{0}}{X_{j}},\ldots,\frac{\widehat{X_{j}}}{X_{j}},\ldots,\frac{X_{n}}{X_{j}}\right]\\
 & \longrightarrow R\left[\frac{X_{0}}{X_{i}},\ldots,\frac{\widehat{X_{i}}}{X_{i}},\ldots,\frac{X_{n}}{X_{i}}\right]\left[\frac{X_{i}}{X_{j}}\right]
\end{align*}

ist surjektiv.
\end{example}

\begin{example}[38]
Sei $k$ algebraisch abgeschlossener Körper, $X$ Prävarietät über
$k$, d.h. ganzes Schema von endlichem Typ über $k$. $X$ heißt \textbf{Varietät},
wenn $X$ separabel ist. Affine Prävarietäten sind also automatisch
Varietäten. $\mathbb{P}^{n}(k)$ ist Varietät (Beispiel 37) $\Longrightarrow$
Jede quasi-projektive Prävarietät ist eine Varietät!
\end{example}


\section{Eigentliche Morphismen}

(eng. ,,proper``)\medskip{}

Ist $f:X\rightarrow Y$ stetige Abbildung zwischen topologischen Räume,
dann heißt $f$ \textbf{eigentlich}, wenn die Urbilder kompakter Mengen
kompat sind.

Sei $X$ Hausdorff, $Y$ lokal kompakt. Dann ist $f$ eigentlich $\Longleftrightarrow f$
universell abgeschlossen. (Bourbaki, Topologie générale, $I$.10 nr.
3, Prop. 7)
\begin{defn}[39]
Ein Morphismus $f:X\rightarrow Y$ von Schemata heißt \textbf{eigentlich},
wenn:
\begin{enumerate}
\item $f$ von endlichem Typ.
\item $f$ separiert.
\item $f$ universell abgeschlossen.
\end{enumerate}
Ein $Y$-Schema heißt \textbf{eigentlich}, wenn der Strukturmorphismus
eigentlich ist. ,,eigentlich`` ist lokal auf dem Ziel.\medskip{}
\end{defn}

\noindent\shadowbox{\begin{minipage}[t]{1\columnwidth - 2\fboxsep - 2\fboxrule - \shadowsize}%
$f:X\rightarrow Y$ heißt von \textbf{endlichem Typ}, wenn $\forall U\subset Y$
offen eine Überdeckung $f^{-1}(U)=\bigcup_{i=1}^{n}V_{i}$ existiert,
sodass $\forall i:\mathcal{O}_{X}(V_{i})$ ist endlich erzeugte $\mathcal{O}_{Y}(U)$-Algebra.%
\end{minipage}}
\begin{defn}[40]
Ein Morphismus $f:X\rightarrow Y$ heißt \textbf{projektiv}, wenn
er sich faktorisieren lässt als
\[
\xymatrix{X\ar[rr]^{f}\ar[rd]_{\text{abg. Imm.}}^{g} &  & Y\\
 & \mathbb{P}_{Y}^{n}\ar[ur]_{\text{kan. Morph.}}
}
\]

für ein $n\in\mathbb{N}$.
\end{defn}

\begin{prop}[41]
Sei $\mathbb{P}$ eine der folgenden Eigenschaften:
\begin{itemize}
\item[I.] von endlichem Typ;
\item[II.] eigentlich;
\item[III.] projektiv.
\item[IV.] (separiert)
\end{itemize}
Dann gilt:
\begin{enumerate}
\item Abgeschlossene Immersionen erfüllen $\mathbb{P}$.
\item $\mathbb{P}$ ist stabil unter Komposition.
\item $\mathbb{P}$ ist stabil unter Basiswechsel.
\item Falls $f:X\rightarrow Z$, $g:Y\rightarrow Z$ $\mathbb{P}$ erfüllen,
dann auch $X\times_{Z}Y\rightarrow Z$.
\item Erfüllt $X\xrightarrow{f}Y\xrightarrow{g}Z$ $\mathbb{P}$, so auch
$f$, falls:
\begin{itemize}
\item $f$ quasi-kompakt (d.h. $Y$ hat offene affine Überdeckung, deren
Urbilder quasi-kompakt sind).
\item $g$ separiert, im Falle II, III.
\item (stetig im Fall IV)
\end{itemize}
\end{enumerate}
\end{prop}

Vergleiche: Lin, ,,Algebraic Geometry and Arithmetic curves``, Prop
24, 3.16, 3.32, (3.9).
\begin{proof}[Beweis (Skizze)]
\end{proof}
\begin{prop}[42]
Sei $f:X\rightarrow Y$ surjektiver Morphismus von $S$-Schemata,
und $Y$ separiert von endlichem Typ über $S$, sowie $X$ eigentlich
über $S$. Dann ist $Y$ eigentlich über $S$.
\end{prop}

\begin{proof}
$f$ surjektiv $\Longrightarrow\forall T\rightarrow S$ ist $f_{(T)}:X_{T}\rightarrow Y_{T}$
surjektiv.

$\Longrightarrow\xymatrix{A\underset{\text{abg.}}{\subset}Y_{T}\ar[r] & T\\
 & \varphi^{-1}(A)\underset{\text{abg.}}{\subset}X_{T}\ar@{->>}[ul]_{\varphi}\ar[u]_{X\text{ eig.}\Rightarrow\text{abg.}}
}
$

$\Longrightarrow Y_{T}\rightarrow T$ abgeschlossen.

$\Longrightarrow Y\rightarrow S$ universal abgeschlossen.
\end{proof}
\begin{prop}[43]
Sei $X$ eigentliches Schema über $S=\Spec A$. Dann ist $\mathcal{O}_{X}(X)$
ganz über $A$. Ist $X=\Spec B$ affin, so ist $B$ endlich über $A$.
\end{prop}

\begin{proof}
Lin, 3.17/3.18.
\end{proof}
\begin{cor}
Sei $X$ reduzierte eigentliche Varietät über $k$. Dann ist $\mathcal{O}_{X}(X)$
endlich-dimensionaler $k$-Vektorraum.
\end{cor}


\end{document}
