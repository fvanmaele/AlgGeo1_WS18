%% LyX 2.3.2 created this file.  For more info, see http://www.lyx.org/.
%% Do not edit unless you really know what you are doing.
\documentclass[oneside,ngerman]{book}
\usepackage[T1]{fontenc}
\usepackage[utf8]{inputenc}
\setcounter{secnumdepth}{3}
\setcounter{tocdepth}{3}
\usepackage{verbatim}
\usepackage{amsmath}
\usepackage{amsthm}
\usepackage{amssymb}
\usepackage[all]{xy}

\makeatletter
%%%%%%%%%%%%%%%%%%%%%%%%%%%%%% Textclass specific LaTeX commands.
\theoremstyle{plain}
\ifx\thechapter\undefined
	\newtheorem{thm}{\protect\theoremname}
\else
	\newtheorem{thm}{\protect\theoremname}[chapter]
\fi
\theoremstyle{plain}
\newtheorem{prop}[thm]{\protect\propositionname}
\theoremstyle{plain}
\newtheorem{cor}[thm]{\protect\corollaryname}

%%%%%%%%%%%%%%%%%%%%%%%%%%%%%% User specified LaTeX commands.
%% User-specified commands
\DeclareMathOperator{\rad}{rad}
\DeclareMathOperator{\Spec}{Spec}
\DeclareMathOperator{\Specm}{Specm}
\DeclareMathOperator{\maxspec}{MaxSpec}
\DeclareMathOperator{\Quot}{Quot}
\DeclareMathOperator{\im}{\mathrm{im}}
\DeclareMathOperator{\Hom}{\mathrm{Hom}}
\DeclareMathOperator{\Mor}{\mathrm{Mor}}
\DeclareMathOperator{\id}{\mathrm{id}}
\DeclareMathOperator{\res}{res}
\DeclareMathOperator{\Abb}{Abb}
\DeclareMathOperator{\supp}{supp}
\DeclareMathOperator{\red}{red}
\DeclareMathOperator{\op}{op}
\DeclareMathOperator{\obj}{Obj}
\DeclareMathOperator{\nil}{nil}

%% sets
\DeclareMathOperator{\CC}{\mathbb{C}}
\DeclareMathOperator{\RR}{\mathbb{R}}
\DeclareMathOperator{\QQ}{\mathbb{Q}}
\DeclareMathOperator{\ZZ}{\mathbb{Z}}
\DeclareMathOperator{\NN}{\mathbb{N}}

%% categories
\DeclareMathOperator{\ouv}{\mathcal{O}uv}
\DeclareMathOperator{\set}{\underline{Set}}
\DeclareMathOperator{\ab}{\underline{Ab}}
\DeclareMathOperator{\cattop}{\underline{Top}}
\DeclareMathOperator{\cring}{\underline{CRing}}
\DeclareMathOperator{\ring}{\underline{Ring}}
\DeclareMathOperator{\psh}{\underline{\mathcal{PS}h}}
\DeclareMathOperator{\sh}{\underline{\mathcal{S}h}}
\DeclareMathOperator{\aff}{\underline{Aff}}
\DeclareMathOperator{\sch}{\underline{Sch}}
\DeclareMathOperator{\schk}{\underline{Sch/k}}
\DeclareMathOperator{\schs}{\underline{Sch/S}}
\DeclareMathOperator{\schsn}{\underline{Sch/S_{0}}}
\DeclareMathOperator{\schz}{\underline{Sch/\mathbb{Z}}}
\DeclareMathOperator{\pres}{\underline{Prevar/S}}
\DeclareMathOperator{\prek}{\underline{Prevar/k}}
\DeclareMathOperator{\affk}{\underline{AffVar/k}}
\DeclareMathOperator{\topcp}{\underline{TopCP}}
\DeclareMathOperator{\Top}{\underline{Top}}
\DeclareMathOperator{\grp}{\underline{Grp}}
\DeclareMathOperator{\func}{\underline{Func}}

\makeatother

\usepackage{babel}
\providecommand{\corollaryname}{Korollar}
\providecommand{\propositionname}{Satz}
\providecommand{\theoremname}{Theorem}

\begin{document}
\textbf{Ab jetzt: }Alle Faserprodukte mögen in $\mathcal{C}$ existieren.

Sei Morphismus $h:T\rightarrow S$ in $\mathcal{C}$, ein \textbf{$S$-Objekt}.
(kurz $T$, \textbf{$h$ Strukturmorphismus} von $T$.)\medskip{}

Für $S$-Objekte $h:T\rightarrow S$ und $f:X\rightarrow S$ schreibe
$\Hom_{S}(T,X):=X_{S}(T)$ für Morphismen $w:T\rightarrow X$ mit
$f\circ w=h$, die \textbf{$S$-Morphismen}. Nenne $X_{S}(T)$ die
\textbf{Menge der $T$-wertigen Punkte von $X$ (über $S$)}. Dies
definiert eine Kategorie $\mathcal{C}/S$ mit finalem Objekt $\id_{S}$.\medskip{}

\textbf{Faserprodukt }$X\times_{S}Y$ ist Produkt von $S$-Objekten
$f$ und $g$ in $\mathcal{C}/S$.
\begin{align*}
(X\times_{S}Y)(T) & =X_{S}(T)\times Y_{S}(T)\\
\text{(UAE)\ }\Hom_{S}(T,X\times_{S}Y) & \overset{\sim}{\longrightarrow}\Hom_{S}(T,X)\times\Hom_{S}(T,Y)\\
w & \longmapsto(p\circ w,q\circ w)
\end{align*}

für alle $h:T\rightarrow S$.

\subsection*{Funktorialität}

Seien $X,Y,X',Y'\in\mathcal{C}/S$, $u:X\rightarrow X'$, $v:Y\rightarrow Y'$
$S$-Morphismus. $\Longrightarrow\exists_{1}$ Morphismus $u\times_{S}v$
(oder nur $u\times v$): $X\times_{S}Y\rightarrow X'\times_{S}Y'$.
d.d.
\[
\xymatrix{X\times_{S}Y\ar[rd]|-{u\times v}\ar[r]^{p}\ar[d]_{q} & X\ar[rd]^{u}\\
Y\ar[rd]_{v} & X'\times Y'\ar[r]\ar[d] & X'\ar[d]\\
 & Y'\ar[r] & S
}
\qquad\text{kommutiert.}
\]

Setze $u\times v:=(u\circ p,v\circ q)_{S}$ (Universelle Eigenschaft
von $X'\times_{S}Y'$!)\medskip{}

\textbf{Yoneda-Lemma:}
\[
f:X\rightarrow Y\in\Mor(\mathcal{C}/S)\leftrightarrow(f_{S}(T):X_{S}(T)\rightarrow Y_{S}(T))_{T\in\mathcal{C}/S}\text{ funktoriell in }T
\]

\begin{prop}[8, Eigenschaften des Faserprodukts]
Sei $X,Y,Z\in\mathcal{C}/S$. Dann gibt es \textbf{kanonische Isomorphismen}
(funktoriell in $X,Y,Z$),
\begin{enumerate}
\item $X\times_{S}S\xrightarrow{\sim}X$
\item $X\times_{S}Y\xrightarrow{\sim}Y\times_{S}X$
\item $(X\times_{S}Y)\times_{S}Z\xrightarrow{\sim}X\times_{S}(Y\times_{S}Z)$
\end{enumerate}
auf $T$-wertigen Punkten, für alle $h:T\rightarrow S$ $S$-Objekt,
gegeben durch:
\begin{align*}
X_{S}(T)\times S_{S}(T) & \xrightarrow{\sim}X_{S}(T), & (x,h)\mapsto x,\\
X_{S}(T)\times Y_{S}(T) & \xrightarrow{\sim}Y_{S}(T)\times X_{S}(T), & (x,y)\mapsto(y,x),\\
(X_{S}(T)\times Y_{S}(T))\times Z_{S}(T) & \xrightarrow{\sim}X_{S}(T)\times(Y_{S}(T)\times Z_{S}(T)), & ((x,y),z)\mapsto(x,(y,z)).
\end{align*}
\end{prop}

Sei $Z\in\mathcal{C}/S$. Ein kommutatives Diagramm in $\mathcal{C}$:
\[
\xymatrix{Z\ar[r]^{u}\ar[d]_{v}\ar@{}[dr]|{\square} & X\ar[d]^{f}\\
Y\ar[r]_{g} & S
}
\]

heißt \textbf{kartesisch} falls $(u,v)_{S}:Z\rightarrow X\times_{S}Y$
ein Isomorphismus ist (und dabei automatisch in $\mathcal{C}/S$).
\[
\xymatrix{Z\ar[d]_{u}\ar@{-->}[rd]|-{(u,v)_{S}}\ar[r]^{v} & Y\\
X & X\times_{S}Y\ar[l]^{p}\ar[u]_{q}
}
\]

Nach dem Yoneda-Lemma ist dies äquivalent dazu, dass das Diagramm:
\[
\xymatrix{\Hom_{\mathcal{C}}(T,Z)\ar[r]^{u(T)}\ar[d]_{v(T)} & \Hom_{\mathcal{C}}(T,X)\ar[d]^{f(T)}\\
\Hom_{\mathcal{C}}(T,Y)\ar[r]_{g(T)} & \Hom_{\mathcal{C}}(T,S)
}
\qquad(*)
\]

kartesisch ist für alle $T\in\mathcal{C}$ (Beispiel 4.7). Dies ist
äquivalent zu: %
\begin{comment}
Pfeile klappen hier nicht besonders gut, evt. Isomorphismus $\Hom_{\mathcal{C}}(T,Z)\rightarrow\cdots$
benennen.
\end{comment}
\[
\xymatrix{\Hom_{\mathcal{C}}(T,Z)\overset{!}{\cong}\ar[r]\ar[d]_{s}^{(u,v)_{S}(T)} & \Hom_{\mathcal{C}}(T,X)\times_{\Hom_{\mathcal{C}}(T,S)}\Hom_{\mathcal{C}}(T,Y)\\
\Hom_{\mathcal{C}}(T,X\times_{S}Y)\ar[ur]_{f\mapsto(p\circ h,q\circ h)}
}
\]

mit $\Hom_{\mathcal{C}}(T,X)\times_{\Hom_{\mathcal{C}}(T,S)}\Hom_{\mathcal{C}}(T,Y):=\{(h_{1},h_{2})\mid f\circ f_{1}=g\circ h_{2}\}$.
\begin{comment}
Hier ist noch ein weiteres Diagramm, mit sich kreuzenden Pfeilen.
\end{comment}

\begin{prop}[10]
Sei das Diagramm
\[
\xymatrix{X''\ar[r]^{g'}\ar[d] & X'\ar[r]^{g}\ar[d]\ar@{}[dr]|{\square} & X\ar[d]\\
S''\ar[r]_{f'} & S'\ar[r]_{f} & S
}
\]

kommutativ, mit rechts ein kartesisches Diagramm. Dann gilt:
\[
\xymatrix{X''\ar[r]\ar[d]\ar@{}[dr]|{\square} & X'\ar[d]\\
S''\ar[r] & S'
}
\quad\Longleftrightarrow\quad\xymatrix{X''\ar[r]\ar[d]\ar@{}[dr]|{\square} & X\ar[d]\\
S''\ar[r] & S
}
\]
\end{prop}

\begin{proof}
Zeige in der Kategorie $\mathcal{C}=\set$, und wende das Yoneda-Lemma,
$(*)$ an.
\end{proof}

\section{Faserprodukte von Schemata}
\begin{prop}[11]
\end{prop}

\begin{thm}[12]
\end{thm}

\begin{cor}[13]
\end{cor}


\end{document}
